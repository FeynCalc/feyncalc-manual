% arara: pdflatex
% arara: pdflatex
% arara: pdflatex
% arara: latexmk: { clean: partial }

% !TeX program = pdflatex
% Options for packages loaded elsewhere

\documentclass[11pt,a4paper,
parskip=half, % half a line vertical space between paragraphs
%headsepline,               % Kopfzeile mit horizontaler Liniel
%headings=big
]{scrreprt}
\usepackage[T1]{fontenc}
\usepackage[utf8]{inputenc}

\usepackage[a4paper, includefoot, left=1.5cm, right=1.5cm, top=2cm, bottom=2 cm]{geometry}

\author{}
\date{}

%\PassOptionsToPackage{unicode}{hyperref}
\PassOptionsToPackage{hyphens}{url}
%\usepackage{unicode-math}

\usepackage{amsmath,amssymb}

%https://tex.stackexchange.com/questions/60746/breqn-does-not-automatically-break-lines
\newcommand{\breakingcomma}{%
	\begingroup\lccode`~=`,
	\lowercase{\endgroup\expandafter\def\expandafter~\expandafter{~\penalty0 }}}


\usepackage{breqn}
\usepackage{fleqn}
\usepackage{lmodern}
%\usepackage{newtxtext}
\usepackage{PTSerif}
\usepackage{textcomp} % provide euro and other symbols

\usepackage{upquote} % Use upquote if available, for straight quotes in verbatim environments

\usepackage[]{microtype}
\UseMicrotypeSet[protrusion]{basicmath} % disable protrusion for tt fonts

\usepackage{fancyvrb}
\usepackage{fvextra}
\usepackage{xcolor}
\usepackage{subfiles}
\usepackage{xurl} % add URL line breaks if available

%\usepackage[htt]{hyphenat}



\usepackage{hyperref}

\hypersetup{
	pdftitle = {FeynCalc Manual},
	pdfauthor = {Vladyslav Shtabovenko, Rolf Mertig and Frederik Orellana},
	colorlinks,
	filecolor=black,
	linkcolor=teal,
	urlcolor=teal
}

%\hypersetup{
%	hidelinks,
%	pdfcreator={LaTeX via pandoc}}
\urlstyle{same} % disable monospaced font for URLs
%\usepackage{color}

\newcommand{\VerbBar}{|}
\newcommand{\VERB}{\Verb[commandchars=\\\{\}]}

\DefineVerbatimEnvironment
{Highlighting}{Verbatim}
{
	frame=single,
	breaklines=true, 
	breakanywhere=true, breaksymbol=, 
	breakanywheresymbolpre=,
	framerule=0.5mm,
	fontfamily=tt,
%	framesep=3mm,
%	rulecolor=\color{red},
%	fillcolor=\color{yellow},
	commandchars=\\\{\}}

\newenvironment{Shaded}{}{}
\newcommand{\AlertTok}[1]{\textcolor[rgb]{1.00,0.00,0.00}{\textbf{#1}}}
\newcommand{\AnnotationTok}[1]{\textcolor[rgb]{0.38,0.63,0.69}{\textbf{\textit{#1}}}}
\newcommand{\AttributeTok}[1]{\textcolor[rgb]{0.49,0.56,0.16}{#1}}
\newcommand{\BaseNTok}[1]{\textcolor[rgb]{0.25,0.63,0.44}{#1}}
\newcommand{\BuiltInTok}[1]{#1}
\newcommand{\CharTok}[1]{\textcolor[rgb]{0.25,0.44,0.63}{#1}}
\newcommand{\CommentTok}[1]{\textcolor[rgb]{0.38,0.63,0.69}{\textit{#1}}}
\newcommand{\CommentVarTok}[1]{\textcolor[rgb]{0.38,0.63,0.69}{\textbf{\textit{#1}}}}
\newcommand{\ConstantTok}[1]{\textcolor[rgb]{0.53,0.00,0.00}{#1}}
\newcommand{\ControlFlowTok}[1]{\textcolor[rgb]{0.00,0.44,0.13}{\textbf{#1}}}
\newcommand{\DataTypeTok}[1]{\textcolor[rgb]{0.56,0.13,0.00}{#1}}
\newcommand{\DecValTok}[1]{\textcolor[rgb]{0.25,0.63,0.44}{#1}}
\newcommand{\DocumentationTok}[1]{\textcolor[rgb]{0.73,0.13,0.13}{\textit{#1}}}
\newcommand{\ErrorTok}[1]{\textcolor[rgb]{1.00,0.00,0.00}{\textbf{#1}}}
\newcommand{\ExtensionTok}[1]{#1}
\newcommand{\FloatTok}[1]{\textcolor[rgb]{0.25,0.63,0.44}{#1}}
\newcommand{\FunctionTok}[1]{\textcolor[rgb]{0.02,0.16,0.49}{#1}}
\newcommand{\ImportTok}[1]{#1}
\newcommand{\InformationTok}[1]{\textcolor[rgb]{0.38,0.63,0.69}{\textbf{\textit{#1}}}}
\newcommand{\KeywordTok}[1]{\textcolor[rgb]{0.00,0.44,0.13}{\textbf{#1}}}
\newcommand{\NormalTok}[1]{#1}
\newcommand{\OperatorTok}[1]{\textcolor[rgb]{0.40,0.40,0.40}{#1}}
\newcommand{\OtherTok}[1]{\textcolor[rgb]{0.00,0.44,0.13}{#1}}
\newcommand{\PreprocessorTok}[1]{\textcolor[rgb]{0.74,0.48,0.00}{#1}}
\newcommand{\RegionMarkerTok}[1]{#1}
\newcommand{\SpecialCharTok}[1]{\textcolor[rgb]{0.25,0.44,0.63}{#1}}
\newcommand{\SpecialStringTok}[1]{\textcolor[rgb]{0.73,0.40,0.53}{#1}}
\newcommand{\StringTok}[1]{\textcolor[rgb]{0.25,0.44,0.63}{#1}}
\newcommand{\VariableTok}[1]{\textcolor[rgb]{0.10,0.09,0.49}{#1}}
\newcommand{\VerbatimStringTok}[1]{\textcolor[rgb]{0.25,0.44,0.63}{#1}}
\newcommand{\WarningTok}[1]{\textcolor[rgb]{0.38,0.63,0.69}{\textbf{\textit{#1}}}}

\setlength{\emergencystretch}{3em} % prevent overfull lines
\providecommand{\tightlist}{%
	\setlength{\itemsep}{0pt}\setlength{\parskip}{0pt}}
%\setcounter{secnumdepth}{-\maxdimen} % remove section numbering


% allow page breaks 
\allowdisplaybreaks


%%%%%%%%%%%%%%%%%%% TOC CUSTOMIZATION %%%%%%%%%%%%%%%%%%%%%%%%%%%%%%%
\addtocontents{toc}{\protect\hypertarget{toc}{}} % Add a link target to the TOC itself
\renewcommand\TOCLineLeaderFill[1][]{\hfill} % get rid of the dots between the entry name and the page number
\addtokomafont{partentry}{\Huge} % Part Überschriften
\addtokomafont{disposition}{\sffamily\Large} % Chapter Überschriften
% Let us have more space between subsection numbers and the titles, for things like 3.1.17. XXX ; 
\DeclareTOCStyleEntry[numwidth=2.5em]{tocline}{part}
\DeclareTOCStyleEntry[numwidth=2.5em]{tocline}{chapter}
\DeclareTOCStyleEntry[numwidth=3.5em]{tocline}{section}
\DeclareTOCStyleEntry[numwidth=4.5em]{tocline}{subsection}
\DeclareTOCStyleEntry[numwidth=6.5em]{tocline}{subsubsection}

\setcounter{tocdepth}{1}

% enable numbering of subsubsections
%\setcounter{secnumdepth}{1}
%%%%%%%%%%%%%%%%%%%%%%%%%%%%%%%%%%%%%%%%%%%%%%%%%%%%%%%%%%%%%%%%%%%%%


\begin{document}
	
\title{FeynCalc Manual}
\author{Vladyslav Shtabovenko, Rolf Mertig and Frederik Orellana}
\maketitle

% Here comes the TOC!
\tableofcontents
	

\chapter{Useful information}

\subfile{pages/Install.tex}
\subfile{pages/Load.tex}
\subfile{pages/FrequentlyAskedQuestions.tex}
\subfile{pages/FeynArts.tex}
\subfile{pages/FeynArtsSigns.tex}
\subfile{pages/Parallelization.tex}
\subfile{pages/UpperLowerIndices.tex}
\subfile{pages/CustomModels.tex}
\subfile{pages/Gamma5.tex}
\subfile{pages/TensorReduction.tex}
\subfile{pages/DiracAlgebraFormulas.tex}
\subfile{pages/Renormalization.tex}
\subfile{pages/MasterIntegrals.tex}
\subfile{pages/Development.tex}

\chapter{Tutorials}

\subfile{pages/InternalExternal.tex}
\subfile{pages/Contractions.tex}
\subfile{pages/Dimensions.tex}
\subfile{pages/Kinematics.tex}
\subfile{pages/Expansions.tex}
\subfile{pages/Indices.tex}
\subfile{pages/DiracAlgebra.tex}
\subfile{pages/ColorAlgebra.tex}
\subfile{pages/Loops.tex}
\subfile{pages/Nonrelativistic.tex}
\subfile{pages/LightCone.tex}

\chapter{Basic objects}

\subfile{pages/Abbreviation.tex}
\subfile{pages/AntiQuarkField.tex}
\subfile{pages/QuarkField.tex}
\subfile{pages/QuarkFieldPsi.tex}
\subfile{pages/QuarkFieldPsiDagger.tex}
\subfile{pages/QuarkFieldChi.tex}
\subfile{pages/QuarkFieldChiDagger.tex}
\subfile{pages/CA.tex}
\subfile{pages/CF.tex}
\subfile{pages/CGA.tex}
\subfile{pages/CGS.tex}
\subfile{pages/CGAD.tex}
\subfile{pages/CGSD.tex}
\subfile{pages/CGAE.tex}
\subfile{pages/CGSE.tex}
\subfile{pages/CSI.tex}
\subfile{pages/CSID.tex}
\subfile{pages/CSIE.tex}
\subfile{pages/CSIS.tex}
\subfile{pages/CSISD.tex}
\subfile{pages/CSISE.tex}
\subfile{pages/CSP.tex}
\subfile{pages/CSPD.tex}
\subfile{pages/CSPE.tex}
\subfile{pages/CV.tex}
\subfile{pages/CVD.tex}
\subfile{pages/CVE.tex}
\subfile{pages/CartesianIndex.tex}
\subfile{pages/CartesianMomentum.tex}
\subfile{pages/CartesianPair.tex}
\subfile{pages/DiracBasis.tex}
\subfile{pages/DeltaFunction.tex}
\subfile{pages/DeltaFunctionPrime.tex}
\subfile{pages/DeltaFunctionDoublePrime.tex}
\subfile{pages/DiracGamma.tex}
\subfile{pages/GA.tex}
\subfile{pages/GA5.tex}
\subfile{pages/GS.tex}
\subfile{pages/GAD.tex}
\subfile{pages/GSD.tex}
\subfile{pages/GAE.tex}
\subfile{pages/GSE.tex}
\subfile{pages/GALP.tex}
\subfile{pages/GALN.tex}
\subfile{pages/GALR.tex}
\subfile{pages/GALPD.tex}
\subfile{pages/GALND.tex}
\subfile{pages/GALRD.tex}
\subfile{pages/GSLP.tex}
\subfile{pages/GSLN.tex}
\subfile{pages/GSLR.tex}
\subfile{pages/GSLPD.tex}
\subfile{pages/GSLND.tex}
\subfile{pages/GSLRD.tex}
\subfile{pages/DiracIndexDelta.tex}
\subfile{pages/DIDelta.tex}
\subfile{pages/DiracSigma.tex}
\subfile{pages/DOT.tex}
\subfile{pages/Eps.tex}
\subfile{pages/LC.tex}
\subfile{pages/LCD.tex}
\subfile{pages/CLC.tex}
\subfile{pages/CLCD.tex}
\subfile{pages/EpsilonUV.tex}
\subfile{pages/EpsilonIR.tex}
\subfile{pages/Epsilon.tex}
\subfile{pages/ExplicitLorentzIndex.tex}
\subfile{pages/LorentzIndex.tex}
\subfile{pages/ExplicitDiracIndex.tex}
\subfile{pages/DiracIndex.tex}
\subfile{pages/ExplicitPauliIndex.tex}
\subfile{pages/PauliIndex.tex}
\subfile{pages/ExplicitSUNIndex.tex}
\subfile{pages/SUNIndex.tex}
\subfile{pages/ExplicitSUNFIndex.tex}
\subfile{pages/SUNFIndex.tex}
\subfile{pages/FAD.tex}
\subfile{pages/SFAD.tex}
\subfile{pages/CFAD.tex}
\subfile{pages/GFAD.tex}
\subfile{pages/FeynAmpDenominator.tex}
\subfile{pages/FCGV.tex}
\subfile{pages/FCIteratedIntegral.tex}
\subfile{pages/FCHPL.tex}
\subfile{pages/FCGPL.tex}
\subfile{pages/FCPartialFractionForm.tex}
\subfile{pages/FCTensor.tex}
\subfile{pages/FCVariable.tex}
\subfile{pages/FreeIndex.tex}
\subfile{pages/GrassmannParity.tex}
\subfile{pages/ImplicitDiracIndex.tex}
\subfile{pages/ImplicitPauliIndex.tex}
\subfile{pages/ImplicitSUNFIndex.tex}
\subfile{pages/NegativeInteger.tex}
\subfile{pages/NonCommutative.tex}
\subfile{pages/PositiveInteger.tex}
\subfile{pages/PositiveNumber.tex}
\subfile{pages/FUNCTION.tex}
\subfile{pages/DCHN.tex}
\subfile{pages/DiracChain.tex}
\subfile{pages/FeynAmp.tex}
\subfile{pages/FeynAmpList.tex}
\subfile{pages/FCPartialD.tex}
\subfile{pages/LeftNablaD.tex}
\subfile{pages/LeftRightNablaD.tex}
\subfile{pages/LeftRightNablaD2.tex}
\subfile{pages/LeftPartialD.tex}
\subfile{pages/LeftRightPartialD.tex}
\subfile{pages/LeftRightPartialD2.tex}
\subfile{pages/RightNablaD.tex}
\subfile{pages/RightPartialD.tex}
\subfile{pages/FCTopology.tex}
\subfile{pages/FV.tex}
\subfile{pages/FVD.tex}
\subfile{pages/FVE.tex}
\subfile{pages/FVLP.tex}
\subfile{pages/FVLN.tex}
\subfile{pages/FVLR.tex}
\subfile{pages/FVLPD.tex}
\subfile{pages/FVLND.tex}
\subfile{pages/FVLRD.tex}
\subfile{pages/GaugeField.tex}
\subfile{pages/GaugeXi.tex}
\subfile{pages/GluonField.tex}
\subfile{pages/IFPD.tex}
\subfile{pages/KD.tex}
\subfile{pages/KDD.tex}
\subfile{pages/KDE.tex}
\subfile{pages/Li2.tex}
\subfile{pages/Li3.tex}
\subfile{pages/Li4.tex}
\subfile{pages/LightConePerpendicularComponent.tex}
\subfile{pages/Momentum.tex}
\subfile{pages/MT.tex}
\subfile{pages/MTD.tex}
\subfile{pages/MTE.tex}
\subfile{pages/MTLP.tex}
\subfile{pages/MTLN.tex}
\subfile{pages/MTLR.tex}
\subfile{pages/MTLPD.tex}
\subfile{pages/MTLND.tex}
\subfile{pages/MTLRD.tex}
\subfile{pages/Nf.tex}
\subfile{pages/Pair.tex}
\subfile{pages/PCHN.tex}
\subfile{pages/PauliChain.tex}
\subfile{pages/PauliEta.tex}
\subfile{pages/PauliXi.tex}
\subfile{pages/PauliIndexDelta.tex}
\subfile{pages/PIDelta.tex}
\subfile{pages/PauliSigma.tex}
\subfile{pages/Polarization.tex}
\subfile{pages/PolarizationVector.tex}
\subfile{pages/PlusDistribution.tex}
\subfile{pages/PropagatorDenominator.tex}
\subfile{pages/PD.tex}
\subfile{pages/StandardPropagatorDenominator.tex}
\subfile{pages/CartesianPropagatorDenominator.tex}
\subfile{pages/GenericPropagatorDenominator.tex}
\subfile{pages/QuantumField.tex}
\subfile{pages/ScaleMu.tex}
\subfile{pages/SD.tex}
\subfile{pages/SUNDelta.tex}
\subfile{pages/SDF.tex}
\subfile{pages/SUNFDelta.tex}
\subfile{pages/SI.tex}
\subfile{pages/SID.tex}
\subfile{pages/SIE.tex}
\subfile{pages/SIS.tex}
\subfile{pages/SISD.tex}
\subfile{pages/SISE.tex}
\subfile{pages/SmallDelta.tex}
\subfile{pages/SmallEpsilon.tex}
\subfile{pages/SmallVariable.tex}
\subfile{pages/Spinor.tex}
\subfile{pages/SpinorU.tex}
\subfile{pages/SpinorUBar.tex}
\subfile{pages/SpinorV.tex}
\subfile{pages/SpinorVBar.tex}
\subfile{pages/SpinorUD.tex}
\subfile{pages/SpinorUBarD.tex}
\subfile{pages/SpinorVD.tex}
\subfile{pages/SpinorVBarD.tex}
\subfile{pages/SP.tex}
\subfile{pages/SPD.tex}
\subfile{pages/SPE.tex}
\subfile{pages/SPLP.tex}
\subfile{pages/SPLN.tex}
\subfile{pages/SPLR.tex}
\subfile{pages/SPLPD.tex}
\subfile{pages/SPLND.tex}
\subfile{pages/SPLRD.tex}
\subfile{pages/StandardMatrixElement.tex}
\subfile{pages/SUND.tex}
\subfile{pages/SUNF.tex}
\subfile{pages/SUNN.tex}
\subfile{pages/SUNT.tex}
\subfile{pages/SUNTF.tex}
\subfile{pages/TC.tex}
\subfile{pages/TGA.tex}
\subfile{pages/TemporalMomentum.tex}
\subfile{pages/TemporalPair.tex}
\subfile{pages/Tf.tex}
\subfile{pages/Zeta2.tex}
\subfile{pages/Zeta4.tex}
\subfile{pages/Zeta6.tex}
\subfile{pages/Zeta8.tex}
\subfile{pages/Zeta10.tex}

\chapter{Basic functions}

\subfile{pages/Apart1.tex}
\subfile{pages/Apart3.tex}
\subfile{pages/Cases2.tex}
\subfile{pages/Coefficient2.tex}
\subfile{pages/Combine.tex}
\subfile{pages/Complement1.tex}
\subfile{pages/Collect2.tex}
\subfile{pages/DataType.tex}
\subfile{pages/Expand2.tex}
\subfile{pages/ExpandAll2.tex}
\subfile{pages/Explicit.tex}
\subfile{pages/Factor1.tex}
\subfile{pages/Factor2.tex}
\subfile{pages/Factor3.tex}
\subfile{pages/FactorList2.tex}
\subfile{pages/FC.tex}
\subfile{pages/FCAbbreviate.tex}
\subfile{pages/FCAntiSymmetrize.tex}
\subfile{pages/FCDeclareHeader.tex}
\subfile{pages/FCPrint.tex}
\subfile{pages/FCReloadAddOns.tex}
\subfile{pages/FCReloadFunctionFromFile.tex}
\subfile{pages/FCClearCache.tex}
\subfile{pages/FCMemoryAvailable.tex}
\subfile{pages/FCShowCache.tex}
\subfile{pages/FCUseCache.tex}
\subfile{pages/FCCheckSyntax.tex}
\subfile{pages/FCCheckVersion.tex}
\subfile{pages/FCCompareResults.tex}
\subfile{pages/FCCompareNumbers.tex}
\subfile{pages/FCDisableTraditionalFormOutput.tex}
\subfile{pages/FCEnableTraditionalFormOutput.tex}
\subfile{pages/FCFactorOut.tex}
\subfile{pages/FCFilePatch.tex}
\subfile{pages/FCGetNotebookDirectory.tex}
\subfile{pages/FCHighlight.tex}
\subfile{pages/FCE.tex}
\subfile{pages/FeynCalcExternal.tex}
\subfile{pages/FCF.tex}
\subfile{pages/FeynCalcForm.tex}
\subfile{pages/FCI.tex}
\subfile{pages/FeynCalcInternal.tex}
\subfile{pages/FCMakeIndex.tex}
\subfile{pages/FCMakeSymbols.tex}
\subfile{pages/FCMatchSolve.tex}
\subfile{pages/FCPatternFreeQ.tex}
\subfile{pages/FCProgressBar.tex}
\subfile{pages/FCReplaceAll.tex}
\subfile{pages/FCReorderList.tex}
\subfile{pages/FCReplaceRepeated.tex}
\subfile{pages/FCShowReferenceCard.tex}
\subfile{pages/FCSplit.tex}
\subfile{pages/FCProductSplit.tex}
\subfile{pages/PartitHead.tex}
\subfile{pages/SelectFree.tex}
\subfile{pages/SelectNotFree.tex}
\subfile{pages/SelectFree2.tex}
\subfile{pages/SelectNotFree2.tex}
\subfile{pages/SelectSplit.tex}
\subfile{pages/FCSymmetrize.tex}
\subfile{pages/FeynCalcHowToCite.tex}
\subfile{pages/FI.tex}
\subfile{pages/FreeQ2.tex}
\subfile{pages/FRH.tex}
\subfile{pages/ILimit.tex}
\subfile{pages/MLimit.tex}
\subfile{pages/Isolate.tex}
\subfile{pages/KK.tex}
\subfile{pages/Map2.tex}
\subfile{pages/MemSet.tex}
\subfile{pages/NTerms.tex}
\subfile{pages/NumericalFactor.tex}
\subfile{pages/NumericQ1.tex}
\subfile{pages/Power2.tex}
\subfile{pages/PowerFactor.tex}
\subfile{pages/PowerSimplify.tex}
\subfile{pages/XYT.tex}
\subfile{pages/Series2.tex}
\subfile{pages/Series3.tex}
\subfile{pages/SetStandardMatrixElements.tex}
\subfile{pages/Solve2.tex}
\subfile{pages/Solve3.tex}
\subfile{pages/SumP.tex}
\subfile{pages/SumS.tex}
\subfile{pages/SumT.tex}
\subfile{pages/TimedIntegrate.tex}
\subfile{pages/ToFCPartialFractionForm.tex}
\subfile{pages/FromFCPartialFractionForm.tex}
\subfile{pages/TBox.tex}
\subfile{pages/TypesettingExplicitLorentzIndex.tex}
\subfile{pages/DollarTypesettingDim4.tex}
\subfile{pages/DollarTypesettingDimD.tex}
\subfile{pages/DollarTypesettingDimE.tex}
\subfile{pages/Variables2.tex}

\chapter{Lorentz and Cartesian tensors}

\subfile{pages/Amputate.tex}
\subfile{pages/CartesianToLorentz.tex}
\subfile{pages/CartesianPairContract.tex}
\subfile{pages/CartesianScalarProduct.tex}
\subfile{pages/ChangeDimension.tex}
\subfile{pages/CompleteSquare.tex}
\subfile{pages/Contract.tex}
\subfile{pages/DeclareFCTensor.tex}
\subfile{pages/DummyIndexFreeQ.tex}
\subfile{pages/EpsContract.tex}
\subfile{pages/EpsContractFreeQ.tex}
\subfile{pages/EpsEvaluate.tex}
\subfile{pages/ExpandScalarProduct.tex}
\subfile{pages/FCCanonicalizeDummyIndices.tex}
\subfile{pages/FCClearScalarProducts.tex}
\subfile{pages/FCGetDimensions.tex}
\subfile{pages/FCGetDummyIndices.tex}
\subfile{pages/FCGetFreeIndices.tex}
\subfile{pages/FCGetScalarProducts.tex}
\subfile{pages/FCPermuteMomentaRules.tex}
\subfile{pages/FCRenameDummyIndices.tex}
\subfile{pages/FCReplaceMomenta.tex}
\subfile{pages/FCReplaceD.tex}
\subfile{pages/FCRerouteMomenta.tex}
\subfile{pages/FCSchoutenBruteForce.tex}
\subfile{pages/FCSetScalarProducts.tex}
\subfile{pages/ScalarProduct.tex}
\subfile{pages/FCSetMetricSignature.tex}
\subfile{pages/FCGetMetricSignature.tex}
\subfile{pages/FourDivergence.tex}
\subfile{pages/FourLaplacian.tex}
\subfile{pages/FreeIndexFreeQ.tex}
\subfile{pages/LorentzToCartesian.tex}
\subfile{pages/MomentumCombine.tex}
\subfile{pages/MomentumExpand.tex}
\subfile{pages/PairContract.tex}
\subfile{pages/PairContract2.tex}
\subfile{pages/PairContract3.tex}
\subfile{pages/Schouten.tex}
\subfile{pages/SetMandelstam.tex}
\subfile{pages/SetTemporalComponent.tex}
\subfile{pages/TensorFunction.tex}
\subfile{pages/ThreeDivergence.tex}
\subfile{pages/TrickMandelstam.tex}
\subfile{pages/ToLightConeComponents.tex}
\subfile{pages/Uncontract.tex}
\subfile{pages/UnDeclareFCTensor.tex}

\chapter{Dirac algebra}

\subfile{pages/Anti5.tex}
\subfile{pages/Chisholm.tex}
\subfile{pages/DiracChainJoin.tex}
\subfile{pages/FCFADiracChainJoin.tex}
\subfile{pages/DiracChainCombine.tex}
\subfile{pages/DiracChainExpand.tex}
\subfile{pages/DiracChainFactor.tex}
\subfile{pages/DiracEquation.tex}
\subfile{pages/DiracGammaCombine.tex}
\subfile{pages/DiracGammaExpand.tex}
\subfile{pages/DiracOrder.tex}
\subfile{pages/DiracReduce.tex}
\subfile{pages/DiracSigmaExpand.tex}
\subfile{pages/DiracSigmaExplicit.tex}
\subfile{pages/DiracSimplify.tex}
\subfile{pages/DiracSubstitute5.tex}
\subfile{pages/DiracSubstitute67.tex}
\subfile{pages/DiracTrace.tex}
\subfile{pages/DiracTrick.tex}
\subfile{pages/EpsChisholm.tex}
\subfile{pages/FCCCT.tex}
\subfile{pages/FCChargeConjugateTransposed.tex}
\subfile{pages/FCDiracIsolate.tex}
\subfile{pages/FCGetDiracGammaScheme.tex}
\subfile{pages/FCSetDiracGammaScheme.tex}
\subfile{pages/GordonSimplify.tex}
\subfile{pages/SirlinSimplify.tex}
\subfile{pages/SpinorChainEvaluate.tex}
\subfile{pages/SpinorChainChiralSplit.tex}
\subfile{pages/SpinorChainTranspose.tex}
\subfile{pages/SpinorChainTrick.tex}
\subfile{pages/ToDiracGamma67.tex}
\subfile{pages/ToDiracSigma.tex}
\subfile{pages/ToLarin.tex}

\chapter{Pauli algebra}

\subfile{pages/FCGetPauliSigmaScheme.tex}
\subfile{pages/FCSetPauliSigmaScheme.tex}
\subfile{pages/FCPauliIsolate.tex}
\subfile{pages/PauliChainJoin.tex}
\subfile{pages/PauliChainCombine.tex}
\subfile{pages/PauliChainExpand.tex}
\subfile{pages/PauliChainFactor.tex}
\subfile{pages/PauliOrder.tex}
\subfile{pages/PauliSigmaCombine.tex}
\subfile{pages/PauliSigmaExpand.tex}
\subfile{pages/PauliSimplify.tex}
\subfile{pages/PauliTrace.tex}
\subfile{pages/PauliTrick.tex}

\chapter{Algebra of noncommutative objects}

\subfile{pages/AntiCommutator.tex}
\subfile{pages/Calc.tex}
\subfile{pages/Trick.tex}
\subfile{pages/Commutator.tex}
\subfile{pages/CommutatorExplicit.tex}
\subfile{pages/CommutatorOrder.tex}
\subfile{pages/DeclareNonCommutative.tex}
\subfile{pages/DotExpand.tex}
\subfile{pages/DotSimplify.tex}
\subfile{pages/FCMatrixIsolate.tex}
\subfile{pages/FCMatrixProduct.tex}
\subfile{pages/FCTraceExpand.tex}
\subfile{pages/FCTraceFactor.tex}
\subfile{pages/NonCommFreeQ.tex}
\subfile{pages/NonCommQ.tex}
\subfile{pages/NonCommHeadQ.tex}
\subfile{pages/TR.tex}
\subfile{pages/UnDeclareAllAntiCommutators.tex}
\subfile{pages/UnDeclareAllCommutators.tex}
\subfile{pages/UnDeclareAntiCommutator.tex}
\subfile{pages/UnDeclareCommutator.tex}
\subfile{pages/UnDeclareNonCommutative.tex}

\chapter{$SU(N)$ algebra}

\subfile{pages/FCColorIsolate.tex}
\subfile{pages/CalcColorFactor.tex}
\subfile{pages/SUNDeltaContract.tex}
\subfile{pages/SUNFDeltaContract.tex}
\subfile{pages/SUNFierz.tex}
\subfile{pages/SUNSimplify.tex}
\subfile{pages/SUNTrace.tex}

\chapter{Loop integrals}

\subfile{pages/A0.tex}
\subfile{pages/A00.tex}
\subfile{pages/Apart2.tex}
\subfile{pages/ApartFF.tex}
\subfile{pages/FCLoopCreatePartialFractioningRules.tex}
\subfile{pages/B0.tex}
\subfile{pages/B00.tex}
\subfile{pages/B1.tex}
\subfile{pages/B11.tex}
\subfile{pages/C0.tex}
\subfile{pages/CTdec.tex}
\subfile{pages/Tdec.tex}
\subfile{pages/D0.tex}
\subfile{pages/DB0.tex}
\subfile{pages/DB1.tex}
\subfile{pages/FCApart.tex}
\subfile{pages/FCClausen.tex}
\subfile{pages/FCGramMatrix.tex}
\subfile{pages/FCGramDeterminant.tex}
\subfile{pages/FCDiffEqChangeVariables.tex}
\subfile{pages/FCDiffEqSolve.tex}
\subfile{pages/FCHideEpsilon.tex}
\subfile{pages/FCShowEpsilon.tex}
\subfile{pages/FCIntegral.tex}
\subfile{pages/FCIteratedIntegralEvaluate.tex}
\subfile{pages/FCIteratedIntegralSimplify.tex}
\subfile{pages/FCFeynmanFindDivergences.tex}
\subfile{pages/FCFeynmanRegularizeDivergence.tex}
\subfile{pages/FCFeynmanParameterJoin.tex}
\subfile{pages/FCFeynmanParametrize.tex}
\subfile{pages/FCFeynmanPrepare.tex}
\subfile{pages/FCFeynmanProjectiveQ.tex}
\subfile{pages/FCFeynmanProjectivize.tex}
\subfile{pages/FCMellinJoin.tex}
\subfile{pages/FCLoopAddEdgeTags.tex}
\subfile{pages/FCLoopGraphPlot.tex}
\subfile{pages/FCLoopIntegralToGraph.tex}
\subfile{pages/FCLoopPropagatorsToLineMomenta.tex}
\subfile{pages/FCLoopAddMissingHigherOrdersWarning.tex}
\subfile{pages/FCLoopApplyTopologyMappings.tex}
\subfile{pages/FCLoopCreateRuleGLIToGLI.tex}
\subfile{pages/FCLoopFindMomentumShifts.tex}
\subfile{pages/FCLoopFindIntegralMappings.tex}
\subfile{pages/FCLoopFindSubtopologies.tex}
\subfile{pages/FCLoopFindTopologies.tex}
\subfile{pages/FCLoopFindTopologyMappings.tex}
\subfile{pages/FCLoopPakOrder.tex}
\subfile{pages/FCLoopToPakForm.tex}
\subfile{pages/FCGraphCuttableQ.tex}
\subfile{pages/FCGraphFindPath.tex}
\subfile{pages/FCLoopAugmentTopology.tex}
\subfile{pages/FCLoopBasisCreateScalarProducts.tex}
\subfile{pages/FCLoopBasisFindCompletion.tex}
\subfile{pages/FCLoopBasisGetSize.tex}
\subfile{pages/FCLoopBasisIncompleteQ.tex}
\subfile{pages/FCLoopBasisOverdeterminedQ.tex}
\subfile{pages/FCLoopBasisSplit.tex}
\subfile{pages/FCLoopBasisExtract.tex}
\subfile{pages/FCLoopCanonicalize.tex}
\subfile{pages/FCLoopCreateRulesToGLI.tex}
\subfile{pages/FCLoopEikonalPropagatorFreeQ.tex}
\subfile{pages/FCLoopExtract.tex}
\subfile{pages/FCLoopFindSectors.tex}
\subfile{pages/FCLoopFindTensorBasis.tex}
\subfile{pages/FCLoopFromGLI.tex}
\subfile{pages/FCLoopGetEtaSigns.tex}
\subfile{pages/FCLoopGetKinematicInvariants.tex}
\subfile{pages/FCLoopGLIDifferentiate.tex}
\subfile{pages/FCLoopGLILowerDimension.tex}
\subfile{pages/FCLoopGLIRaiseDimension.tex}
\subfile{pages/FCLoopAddScalingParameter.tex}
\subfile{pages/FCLoopGLIExpand.tex}
\subfile{pages/FCLoopIBPReducableQ.tex}
\subfile{pages/FCLoopIntegralToPropagators.tex}
\subfile{pages/FCLoopIsolate.tex}
\subfile{pages/FCLoopMixedIntegralQ.tex}
\subfile{pages/FCLoopMixedToCartesianAndTemporal.tex}
\subfile{pages/FCLoopNonIntegerPropagatorPowersFreeQ.tex}
\subfile{pages/FCLoopPakScalelessQ.tex}
\subfile{pages/FCLoopScalelessQ.tex}
\subfile{pages/FCLoopPropagatorPowersCombine.tex}
\subfile{pages/FCLoopPropagatorPowersExpand.tex}
\subfile{pages/FCLoopPropagatorsToTopology.tex}
\subfile{pages/FCLoopRemovePropagator.tex}
\subfile{pages/FCLoopReplaceQuadraticEikonalPropagators.tex}
\subfile{pages/FCLoopSamePropagatorHeadsQ.tex}
\subfile{pages/FCLoopRemoveNegativePropagatorPowers.tex}
\subfile{pages/FCLoopSelectTopology.tex}
\subfile{pages/FCLoopSwitchEtaSign.tex}
\subfile{pages/FCLoopSingularityStructure.tex}
\subfile{pages/FCLoopSolutionList.tex}
\subfile{pages/FCLoopSplit.tex}
\subfile{pages/FCLoopTopologyNameToSymbol.tex}
\subfile{pages/FCLoopTensorReduce.tex}
\subfile{pages/FCLoopValidTopologyQ.tex}
\subfile{pages/FCMultiLoopTID.tex}
\subfile{pages/FeynAmpDenominatorCombine.tex}
\subfile{pages/FeynAmpDenominatorExplicit.tex}
\subfile{pages/FeynAmpDenominatorSimplify.tex}
\subfile{pages/FDS.tex}
\subfile{pages/FeynAmpDenominatorSplit.tex}
\subfile{pages/FromGFAD.tex}
\subfile{pages/GammaExpand.tex}
\subfile{pages/GenPaVe.tex}
\subfile{pages/PaVe.tex}
\subfile{pages/GLI.tex}
\subfile{pages/GLIMultiply.tex}
\subfile{pages/Hill.tex}
\subfile{pages/HypergeometricAC.tex}
\subfile{pages/HypergeometricIR.tex}
\subfile{pages/HypergeometricSE.tex}
\subfile{pages/HypExplicit.tex}
\subfile{pages/HypInt.tex}
\subfile{pages/IntegrateByParts.tex}
\subfile{pages/PartialIntegrate.tex}
\subfile{pages/NPointTo4Point.tex}
\subfile{pages/OneLoopSimplify.tex}
\subfile{pages/ToHypergeometric.tex}
\subfile{pages/PaVeToABCD.tex}
\subfile{pages/PaVeOrder.tex}
\subfile{pages/PaVeLimitTo4.tex}
\subfile{pages/PaVeReduce.tex}
\subfile{pages/PaVeUVPart.tex}
\subfile{pages/SimplifyDeltaFunction.tex}
\subfile{pages/Sn.tex}
\subfile{pages/TarcerToFC.tex}
\subfile{pages/TFIOrder.tex}
\subfile{pages/TID.tex}
\subfile{pages/TIDL.tex}
\subfile{pages/ToDistribution.tex}
\subfile{pages/ToFI.tex}
\subfile{pages/ToTFI.tex}
\subfile{pages/ToGFAD.tex}
\subfile{pages/ToPaVe.tex}
\subfile{pages/ToPaVe2.tex}
\subfile{pages/ToSFAD.tex}
\subfile{pages/TrickIntegrate.tex}

\chapter{Export and import}

\subfile{pages/FCGVToSymbol.tex}
\subfile{pages/FCLoopGLIToSymbol.tex}
\subfile{pages/FCToTeXReorder.tex}
\subfile{pages/FCToTeXPreviewTermOrder.tex}
\subfile{pages/FeynCalc2FORM.tex}
\subfile{pages/FeynCalcToLaTeX.tex}
\subfile{pages/FORM2FeynCalc.tex}
\subfile{pages/StringChomp.tex}
\subfile{pages/SMPToSymbol.tex}
\subfile{pages/Write2.tex}

\chapter{Feynman rules and amplitudes}

\subfile{pages/BackgroundGluonVertex.tex}
\subfile{pages/ComplexConjugate.tex}
\subfile{pages/CovariantD.tex}
\subfile{pages/CovariantFieldDerivative.tex}
\subfile{pages/CDr.tex}
\subfile{pages/DoPolarizationSums.tex}
\subfile{pages/PolarizationSum.tex}
\subfile{pages/ExpandPartialD.tex}
\subfile{pages/ExplicitPartialD.tex}
\subfile{pages/FAPatch.tex}
\subfile{pages/FCAttachTypesettingRule.tex}
\subfile{pages/FCRemoveTypesettingRules.tex}
\subfile{pages/FCFAConvert.tex}
\subfile{pages/FCPrepareFAAmp.tex}
\subfile{pages/FCTP.tex}
\subfile{pages/FCTripleProduct.tex}
\subfile{pages/FermionSpinSum.tex}
\subfile{pages/FeynRule.tex}
\subfile{pages/FieldDerivative.tex}
\subfile{pages/FDr.tex}
\subfile{pages/FieldStrength.tex}
\subfile{pages/FunctionalD.tex}
\subfile{pages/GhostPropagator.tex}
\subfile{pages/GHP.tex}
\subfile{pages/GluonGhostVertex.tex}
\subfile{pages/GGV.tex}
\subfile{pages/GluonPropagator.tex}
\subfile{pages/GP.tex}
\subfile{pages/GluonSelfEnergy.tex}
\subfile{pages/GluonVertex.tex}
\subfile{pages/GV.tex}
\subfile{pages/QuarkGluonVertex.tex}
\subfile{pages/QGV.tex}
\subfile{pages/QuarkPropagator.tex}
\subfile{pages/QP.tex}
\subfile{pages/ScalarGluonVertex.tex}
\subfile{pages/ShiftPartialD.tex}
\subfile{pages/SquareAmplitude.tex}
\subfile{pages/SMP.tex}
\subfile{pages/SMVertex.tex}
\subfile{pages/ToStandardMatrixElement.tex}
\subfile{pages/QCDFeynmanRuleConvention.tex}

\chapter{Tables}

\subfile{pages/Amplitude.tex}
\subfile{pages/AnomalousDimension.tex}
\subfile{pages/CheckDB.tex}
\subfile{pages/Convolute.tex}
\subfile{pages/ConvoluteTable.tex}
\subfile{pages/CounterTerm.tex}
\subfile{pages/Gamma1.tex}
\subfile{pages/Gamma2.tex}
\subfile{pages/Gamma3.tex}
\subfile{pages/GammaEpsilon.tex}
\subfile{pages/Integrate2.tex}
\subfile{pages/Integrate3.tex}
\subfile{pages/Integrate5.tex}
\subfile{pages/InverseMellin.tex}
\subfile{pages/Kummer.tex}
\subfile{pages/Lagrangian.tex}
\subfile{pages/Nielsen.tex}
\subfile{pages/SimplifyPolyLog.tex}
\subfile{pages/SPL.tex}
\subfile{pages/SplittingFunction.tex}

\chapter{Options}

\subfile{pages/DollarDisableMemSet.tex}
\subfile{pages/DollarFAPatch.tex}
\subfile{pages/DollarFCCheckContext.tex}
\subfile{pages/DollarFCCloudTraditionalForm.tex}
\subfile{pages/DollarFCTraditionalFormOutput.tex}
\subfile{pages/DollarFeynArtsDirectory.tex}
\subfile{pages/DollarFeynCalcDevelopmentVersion.tex}
\subfile{pages/DollarFeynCalcDirectory.tex}
\subfile{pages/DollarFeynCalcLastCommitDateHash.tex}
\subfile{pages/DollarFeynCalcStartupMessages.tex}
\subfile{pages/DollarLoadAddOns.tex}
\subfile{pages/DollarMultiplications.tex}
\subfile{pages/DollarRenameFeynCalcObjects.tex}
\subfile{pages/DollarParallelizeFeynCalc.tex}
\subfile{pages/DollarContainers.tex}
\subfile{pages/DollarDistributiveFunctions.tex}
\subfile{pages/DollarFCAdvice.tex}
\subfile{pages/DollarFCMemoryAvailable.tex}
\subfile{pages/DollarFCShowIEta.tex}
\subfile{pages/DollarFortranContinuationCharacter.tex}
\subfile{pages/DollarKeepLogDivergentScalelessIntegrals.tex}
\subfile{pages/DollarLeviCivitaSign.tex}
\subfile{pages/DollarLimitTo4.tex}
\subfile{pages/DollarLimitTo4IRUnsafe.tex}
\subfile{pages/DollarVeryVerbose.tex}
\subfile{pages/DollarAbbreviations.tex}
\subfile{pages/DollarAL.tex}
\subfile{pages/DollarFCTensorList.tex}
\subfile{pages/DollarFeynCalcVersion.tex}
\subfile{pages/DollarMU.tex}
\subfile{pages/DollarNonComm.tex}
\subfile{pages/DollarScalarProducts.tex}
\subfile{pages/A0ToB0.tex}
\subfile{pages/AuxiliaryMomenta.tex}
\subfile{pages/AugmentedTopologyMarker.tex}
\subfile{pages/B0Real.tex}
\subfile{pages/B0Unique.tex}
\subfile{pages/Bracket.tex}
\subfile{pages/BReduce.tex}
\subfile{pages/CartesianIndexNames.tex}
\subfile{pages/ClearHeads.tex}
\subfile{pages/Collecting.tex}
\subfile{pages/CombineGraphs.tex}
\subfile{pages/CounterT.tex}
\subfile{pages/CouplingConstant.tex}
\subfile{pages/CustomIndexNames.tex}
\subfile{pages/D0Convention.tex}
\subfile{pages/DetectLoopTopologies.tex}
\subfile{pages/Dimension.tex}
\subfile{pages/DiracIndexNames.tex}
\subfile{pages/DiracSpinorNormalization.tex}
\subfile{pages/DiracTraceEvaluate.tex}
\subfile{pages/Divideout.tex}
\subfile{pages/DotPower.tex}
\subfile{pages/DotSimplifyRelations.tex}
\subfile{pages/DropIndexSum.tex}
\subfile{pages/DropScaleless.tex}
\subfile{pages/DropSumOver.tex}
\subfile{pages/DummyIndex.tex}
\subfile{pages/EpsDiscard.tex}
\subfile{pages/EpsExpand.tex}
\subfile{pages/EpsilonOrder.tex}
\subfile{pages/EtaSign.tex}
\subfile{pages/ExceptHeads.tex}
\subfile{pages/ExcludeMasses.tex}
\subfile{pages/Expanding.tex}
\subfile{pages/ExtraFactor.tex}
\subfile{pages/ExtraPropagators.tex}
\subfile{pages/ExtraVariables.tex}
\subfile{pages/FactorFull.tex}
\subfile{pages/Factoring.tex}
\subfile{pages/FactoringDenominator.tex}
\subfile{pages/Factorout.tex}
\subfile{pages/FAModelsDirectory.tex}
\subfile{pages/FCDoControl.tex}
\subfile{pages/FCParallelize.tex}
\subfile{pages/FCJoinDOTs.tex}
\subfile{pages/FCVerbose.tex}
\subfile{pages/FeynmanIntegralPrefactor.tex}
\subfile{pages/FinalFunction.tex}
\subfile{pages/FinalSubstitutions.tex}
\subfile{pages/ForceSave.tex}
\subfile{pages/FORMAbbreviations.tex}
\subfile{pages/FORMEpilog.tex}
\subfile{pages/FORMIdStatements.tex}
\subfile{pages/FORMProlog.tex}
\subfile{pages/FortranFormatDoublePrecision.tex}
\subfile{pages/FunctionLimits.tex}
\subfile{pages/Gauge.tex}
\subfile{pages/IncomingMomenta.tex}
\subfile{pages/IndexPosition.tex}
\subfile{pages/InitialFunction.tex}
\subfile{pages/InitialSubstitutions.tex}
\subfile{pages/InsideDiracTrace.tex}
\subfile{pages/InsidePauliTrace.tex}
\subfile{pages/IntegralTable.tex}
\subfile{pages/IntermediateSubstitutions.tex}
\subfile{pages/IsolateFast.tex}
\subfile{pages/IsolateNames.tex}
\subfile{pages/IsolatePlus.tex}
\subfile{pages/IsolatePrint.tex}
\subfile{pages/IsolateSplit.tex}
\subfile{pages/IsolateTimes.tex}
\subfile{pages/LarinMVV.tex}
\subfile{pages/LightPak.tex}
\subfile{pages/DollarFCDefaultLightconeVectorN.tex}
\subfile{pages/DollarFCDefaultLightconeVectorNB.tex}
\subfile{pages/Loop.tex}
\subfile{pages/LoopMomenta.tex}
\subfile{pages/LorentzIndexNames.tex}
\subfile{pages/Mandelstam.tex}
\subfile{pages/MultiLoop.tex}
\subfile{pages/NoSave.tex}
\subfile{pages/NumberOfPolarizations.tex}
\subfile{pages/NotMomentum.tex}
\subfile{pages/OtherLoopMomenta.tex}
\subfile{pages/OutgoingMomenta.tex}
\subfile{pages/PairCollect.tex}
\subfile{pages/PartialDRelations.tex}
\subfile{pages/PatchModelsOnly.tex}
\subfile{pages/PauliIndexNames.tex}
\subfile{pages/PauliReduce.tex}
\subfile{pages/PauliTraceEvaluate.tex}
\subfile{pages/PaVeAutoOrder.tex}
\subfile{pages/PaVeAutoReduce.tex}
\subfile{pages/PaVeIntegralHeads.tex}
\subfile{pages/PaVeOrderList.tex}
\subfile{pages/PostFortranFile.tex}
\subfile{pages/Prefactor.tex}
\subfile{pages/PreferredIntegrals.tex}
\subfile{pages/PreferredTopologies.tex}
\subfile{pages/PreFortranFile.tex}
\subfile{pages/PreservePropagatorStructures.tex}
\subfile{pages/QuarkMass.tex}
\subfile{pages/ReduceGamma.tex}
\subfile{pages/ReduceToScalars.tex}
\subfile{pages/Rename.tex}
\subfile{pages/SameSideExternalEdges.tex}
\subfile{pages/SchoutenAllowNegativeGain.tex}
\subfile{pages/SchoutenAllowZeroGain.tex}
\subfile{pages/SelectGraphs.tex}
\subfile{pages/SetDimensions.tex}
\subfile{pages/SmallVariables.tex}
\subfile{pages/SplitSymbolicPowers.tex}
\subfile{pages/SubLoop.tex}
\subfile{pages/SubtopologyMarker.tex}
\subfile{pages/SUNFJacobi.tex}
\subfile{pages/SUNIndexNames.tex}
\subfile{pages/SUNFIndexNames.tex}
\subfile{pages/SUNTraceEvaluate.tex}
\subfile{pages/SUNNToCACF.tex}
\subfile{pages/TensorReductionBasisChange.tex}
\subfile{pages/TraceDimension.tex}
\subfile{pages/TraceOfOne.tex}
\subfile{pages/Transversality.tex}
\subfile{pages/TransversePolarizationVectors.tex}
\subfile{pages/UndoChiralSplittings.tex}
\subfile{pages/UsePaVeBasis.tex}
\subfile{pages/UseTIDL.tex}
\subfile{pages/UseWriteString.tex}
\subfile{pages/VirtualBoson.tex}
\subfile{pages/West.tex}
\subfile{pages/WriteOut.tex}
\subfile{pages/WriteOutPaVe.tex}
\subfile{pages/WriteStringOutput.tex}
\subfile{pages/ZeroMomentumInsertion.tex}

\chapter{Misc}

\subfile{pages/CalculateCounterTerm.tex}
\subfile{pages/GO.tex}
\subfile{pages/Integratedx.tex}
\subfile{pages/DollarMIntegrate.tex}
\subfile{pages/DollarOPEWard.tex}
\subfile{pages/OPE.tex}
\subfile{pages/OPE1Loop.tex}
\subfile{pages/OPE2TID.tex}
\subfile{pages/OPEDelta.tex}
\subfile{pages/OPEi.tex}
\subfile{pages/OPEInt.tex}
\subfile{pages/OPEIntegrate.tex}
\subfile{pages/OPEIntegrate2.tex}
\subfile{pages/OPEIntegrateDelta.tex}
\subfile{pages/OPEj.tex}
\subfile{pages/OPEk.tex}
\subfile{pages/OPEl.tex}
\subfile{pages/OPEm.tex}
\subfile{pages/OPEn.tex}
\subfile{pages/OPEo.tex}
\subfile{pages/OPESum.tex}
\subfile{pages/OPESumExplicit.tex}
\subfile{pages/OPESumSimplify.tex}
\subfile{pages/QO.tex}
\subfile{pages/SO.tex}
\subfile{pages/SOD.tex}
\subfile{pages/SymbolicSum2.tex}
\subfile{pages/SymbolicSum3.tex}
\subfile{pages/Twist2AlienOperator.tex}
\subfile{pages/Twist2CounterOperator.tex}
\subfile{pages/Twist2GluonOperator.tex}
\subfile{pages/Twist2QuarkOperator.tex}
\subfile{pages/Twist3QuarkOperator.tex}
\subfile{pages/Twist4GluonOperator.tex}

\chapter{Deprecated or legacy functions}

\subfile{pages/AlphaStrong.tex}
\subfile{pages/AlphaFS.tex}
\subfile{pages/DollarBreitMaison.tex}
\subfile{pages/DollarLarin.tex}
\subfile{pages/ChiralityProjector.tex}
\subfile{pages/ClearScalarProducts.tex}
\subfile{pages/DiracMatrix.tex}
\subfile{pages/DiracSlash.tex}
\subfile{pages/DiracSpinor.tex}
\subfile{pages/FourVector.tex}
\subfile{pages/Gstrong.tex}
\subfile{pages/IFPDOn.tex}
\subfile{pages/IFPDOff.tex}
\subfile{pages/LeviCivita.tex}
\subfile{pages/DollarLoadFeynArts.tex}
\subfile{pages/DollarLoadPhi.tex}
\subfile{pages/DollarLoadTARCER.tex}
\subfile{pages/MetricTensor.tex}
\subfile{pages/OneLoop.tex}
\subfile{pages/OneLoopSum.tex}
\subfile{pages/PartialFourVector.tex}
\subfile{pages/PropagatorDenominatorExplicit.tex}
\subfile{pages/ScalarProductCancel.tex}
\subfile{pages/SPC.tex}
\subfile{pages/ScalarProductExpand.tex}
%\subfile{pages/DollarTypesettingDim4.tex}
%\subfile{pages/DiracSimplify.tex}
%\subfile{pages/DiracEquation.tex}

\end{document}