% !TeX program = pdflatex
% !TeX root = A0.tex

\documentclass[../FeynCalcManual.tex]{subfiles}
\begin{document}
\hypertarget{a0}{%
\section{A0}\label{a0}}

\texttt{A0[\allowbreak{}m^2]} is the Passarino-Veltman one-point
integral \(A_0.\).

\subsection{See also}

\hyperlink{toc}{Overview}, \hyperlink{b0}{B0}, \hyperlink{c0}{C0},
\hyperlink{d0}{D0}, \hyperlink{pave}{PaVe}.

\subsection{Examples}

By default \(A_0\) is not expressed in terms of \(B_0\).

\begin{Shaded}
\begin{Highlighting}[]
\NormalTok{A0}\OperatorTok{[}\FunctionTok{m}\SpecialCharTok{\^{}}\DecValTok{2}\OperatorTok{]}
\end{Highlighting}
\end{Shaded}

\begin{dmath*}\breakingcomma
\text{A}_0\left(m^2\right)
\end{dmath*}

\begin{Shaded}
\begin{Highlighting}[]
\FunctionTok{SetOptions}\OperatorTok{[}\NormalTok{A0}\OperatorTok{,}\NormalTok{ A0ToB0 }\OtherTok{{-}\textgreater{}} \ConstantTok{True}\OperatorTok{]}\NormalTok{; }
 
\NormalTok{A0}\OperatorTok{[}\FunctionTok{m}\SpecialCharTok{\^{}}\DecValTok{2}\OperatorTok{]}
\end{Highlighting}
\end{Shaded}

\begin{dmath*}\breakingcomma
-\frac{2 m^2 \;\text{B}_0\left(0,m^2,m^2\right)}{2-D}
\end{dmath*}

\begin{Shaded}
\begin{Highlighting}[]
\FunctionTok{SetOptions}\OperatorTok{[}\NormalTok{A0}\OperatorTok{,}\NormalTok{ A0ToB0 }\OtherTok{{-}\textgreater{}} \ConstantTok{False}\OperatorTok{]}\NormalTok{;}
\end{Highlighting}
\end{Shaded}

According to the rules of dimensional regularization \(A_0(0)\) is set
to 0.

\begin{Shaded}
\begin{Highlighting}[]
\NormalTok{A0}\OperatorTok{[}\DecValTok{0}\OperatorTok{]}
\end{Highlighting}
\end{Shaded}

\begin{dmath*}\breakingcomma
0
\end{dmath*}

\begin{Shaded}
\begin{Highlighting}[]
\NormalTok{A0}\OperatorTok{[}\NormalTok{SmallVariable}\OperatorTok{[}\FunctionTok{M}\SpecialCharTok{\^{}}\DecValTok{2}\OperatorTok{]]}
\end{Highlighting}
\end{Shaded}

\begin{dmath*}\breakingcomma
0
\end{dmath*}
\end{document}
