% !TeX program = pdflatex
% !TeX root = A00.tex

\documentclass[../FeynCalcManual.tex]{subfiles}
\begin{document}
\hypertarget{a00}{
\section{A00}\label{a00}\index{A00}}

\texttt{A00[\allowbreak{}m^2]} is the Passarino-Veltman coefficient
function \(A_{00}\), i.e.~the coefficient function multiplying
\(g^{\mu \nu}\). The argument is a scalar and has mass dimension 2.

\subsection{See also}

\hyperlink{toc}{Overview}, \hyperlink{a0}{A0}, \hyperlink{b0}{B0},
\hyperlink{c0}{C0}, \hyperlink{d0}{D0}, \hyperlink{pave}{PaVe}.

\subsection{Examples}

\(A_{00}\) get automatically reduced to \(A_0\)

\begin{Shaded}
\begin{Highlighting}[]
\NormalTok{A00}\OperatorTok{[}\FunctionTok{m}\SpecialCharTok{\^{}}\DecValTok{2}\OperatorTok{]}
\end{Highlighting}
\end{Shaded}

\begin{dmath*}\breakingcomma
\frac{m^2 \;\text{A}_0\left(m^2\right)}{D}
\end{dmath*}

According to the rules of dimensional regularization \(A_{00}(0)\) is
set to 0.

\begin{Shaded}
\begin{Highlighting}[]
\NormalTok{A00}\OperatorTok{[}\DecValTok{0}\OperatorTok{]}
\end{Highlighting}
\end{Shaded}

\begin{dmath*}\breakingcomma
0
\end{dmath*}
\end{document}
