% !TeX program = pdflatex
% !TeX root = Abbreviation.tex

\documentclass[../FeynCalcManual.tex]{subfiles}
\begin{document}
\hypertarget{abbreviation}{%
\section{Abbreviation}\label{abbreviation}}

\texttt{Abbreviation} is a function used by \texttt{OneLoop} and
\texttt{PaVeReduce} for generating smaller files when saving results to
the hard disk. The convention is that a definition like
\texttt{GP = GluonPropagator} should be accompanied by the definition
\texttt{Abbreviation[\allowbreak{}GluonPropagator] = HoldForm[\allowbreak{}GP]}.

\subsection{See also}

\hyperlink{toc}{Overview}, \hyperlink{abbreviations}{\$Abbreviations},
\hyperlink{oneloop}{OneLoop}, \hyperlink{pavereduce}{PaVeReduce},
\hyperlink{writeout}{WriteOut}, \hyperlink{writeoutpave}{WriteOutPaVe},
\hyperlink{gluonpropagator}{GluonPropagator},
\hyperlink{gluonvertex}{GluonVertex},
\hyperlink{quarkpropagator}{QuarkPropagator}.

\subsection{Examples}

\begin{Shaded}
\begin{Highlighting}[]
\NormalTok{GP}\OperatorTok{[}\FunctionTok{p}\OperatorTok{,} \OperatorTok{\{}\SpecialCharTok{\textbackslash{}}\OperatorTok{[}\NormalTok{Mu}\OperatorTok{],} \FunctionTok{a}\OperatorTok{\},} \OperatorTok{\{}\SpecialCharTok{\textbackslash{}}\OperatorTok{[}\NormalTok{Nu}\OperatorTok{],} \FunctionTok{b}\OperatorTok{\}]}
\end{Highlighting}
\end{Shaded}

\begin{dmath*}\breakingcomma
\Pi _{ab}^{\mu \nu }(p)
\end{dmath*}
\end{document}
