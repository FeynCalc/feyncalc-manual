% !TeX program = pdflatex
% !TeX root = Amplitude.tex

\documentclass[../FeynCalcManual.tex]{subfiles}
\begin{document}
\hypertarget{amplitude}{
\section{Amplitude}\label{amplitude}\index{Amplitude}}

\texttt{Amplitude} is a database of Feynman amplitudes.
\texttt{Amplitude[\allowbreak{}"name"]} returns the amplitude
corresponding to the string \texttt{"name"}. A list of all defined names
is obtained with \texttt{Amplitude[\allowbreak{}]}. New amplitudes can
be added to the file \texttt{"Amplitude.m"}. It is strongly recommended
to use names that reflect the process.

The option \texttt{Gauge -> 1} means `t Hooft Feynman gauge;

\texttt{Polarization -> 0} gives unpolarized OPE-type amplitudes,
\texttt{Polarization -> 1} the polarized ones.

\subsection{See also}

\hyperlink{toc}{Overview}, \hyperlink{feynamp}{FeynAmp}.

\subsection{Examples}

\begin{Shaded}
\begin{Highlighting}[]
\NormalTok{Amplitude}\OperatorTok{[]} \SpecialCharTok{//} \FunctionTok{Length}
\end{Highlighting}
\end{Shaded}

\begin{dmath*}\breakingcomma
98
\end{dmath*}

This is the amplitude of a gluon self-energy diagram:

\begin{Shaded}
\begin{Highlighting}[]
\NormalTok{Amplitude}\OperatorTok{[}\StringTok{"se1g1"}\OperatorTok{]} 
 
\NormalTok{Explicit}\OperatorTok{[}\SpecialCharTok{\%}\OperatorTok{]}
\end{Highlighting}
\end{Shaded}

\begin{dmath*}\breakingcomma
\text{SUNDeltaContract}\left(f^{\text{FCGV}(\text{a})\text{FCGV}(\text{c})\text{FCGV}(\text{e})} f^{\text{FCGV}(\text{b})\text{FCGV}(\text{d})\text{FCGV}(\text{f})} \Pi _{\text{FCGV}(\text{e})\text{FCGV}(\text{f})}^{\text{FCGV}(\beta )\text{FCGV}(\sigma )}(\text{FCGV}(\text{q})) V^{\text{FCGV}(\mu )\text{FCGV}(\alpha )\text{FCGV}(\beta )}(\text{FCGV}(\text{p})\text{, }\;\text{FCGV}(\text{q})-\text{FCGV}(\text{p})\text{, }-\text{FCGV}(\text{q})) V^{\text{FCGV}(\nu )\text{FCGV}(\rho )\text{FCGV}(\sigma )}(-\text{FCGV}(\text{p})\text{, }\;\text{FCGV}(\text{p})-\text{FCGV}(\text{q})\text{, }\;\text{FCGV}(\text{q})) \Pi _{\text{FCGV}(\text{c})\text{FCGV}(\text{d})}^{\text{FCGV}(\alpha )\text{FCGV}(\rho )}(\text{FCGV}(\text{p})-\text{FCGV}(\text{q}))\right)
\end{dmath*}

\begin{dmath*}\breakingcomma
-\frac{1}{\text{FCGV}(\text{q})^2 (\text{FCGV}(\text{p})-\text{FCGV}(\text{q}))^2}g_s^2 g^{\text{FCGV}(\alpha )\text{FCGV}(\rho )} g^{\text{FCGV}(\beta )\text{FCGV}(\sigma )} f^{\text{FCGV}(\text{a})\text{FCGV}(\text{d})\text{FCGV}(\text{f})} f^{\text{FCGV}(\text{b})\text{FCGV}(\text{d})\text{FCGV}(\text{f})} \left(g^{\text{FCGV}(\beta )\text{FCGV}(\mu )} \left(-\text{FCGV}(\text{p})^{\text{FCGV}(\alpha )}-\text{FCGV}(\text{q})^{\text{FCGV}(\alpha )}\right)+g^{\text{FCGV}(\alpha )\text{FCGV}(\mu )} \left(2 \;\text{FCGV}(\text{p})^{\text{FCGV}(\beta )}-\text{FCGV}(\text{q})^{\text{FCGV}(\beta )}\right)+g^{\text{FCGV}(\alpha )\text{FCGV}(\beta )} \left(2 \;\text{FCGV}(\text{q})^{\text{FCGV}(\mu )}-\text{FCGV}(\text{p})^{\text{FCGV}(\mu )}\right)\right) \left(g^{\text{FCGV}(\rho )\text{FCGV}(\sigma )} \left(\text{FCGV}(\text{p})^{\text{FCGV}(\nu )}-2 \;\text{FCGV}(\text{q})^{\text{FCGV}(\nu )}\right)+g^{\text{FCGV}(\nu )\text{FCGV}(\sigma )} \left(\text{FCGV}(\text{p})^{\text{FCGV}(\rho )}+\text{FCGV}(\text{q})^{\text{FCGV}(\rho )}\right)+g^{\text{FCGV}(\nu )\text{FCGV}(\rho )} \left(\text{FCGV}(\text{q})^{\text{FCGV}(\sigma )}-2 \;\text{FCGV}(\text{p})^{\text{FCGV}(\sigma )}\right)\right)
\end{dmath*}
\end{document}
