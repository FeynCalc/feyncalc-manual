% !TeX program = pdflatex
% !TeX root = Anti5.tex

\documentclass[../FeynCalcManual.tex]{subfiles}
\begin{document}
\hypertarget{anti5}{
\section{Anti5}\label{anti5}\index{Anti5}}

\texttt{Anti5[\allowbreak{}exp]} anticommutes all \(\gamma^5\) in exp to
the right. \texttt{Anti5[\allowbreak{}exp,\ \allowbreak{}n]}
anticommutes all \(\gamma^5\) \(n\)-times to the right.
\texttt{Anti5[\allowbreak{}exp,\ \allowbreak{}-n]} anticommutes all
\(\gamma^5\) \(n\)-times to the left.

\subsection{See also}

\hyperlink{toc}{Overview}, \hyperlink{diracorder}{DiracOrder},
\hyperlink{diracsimplify}{DiracSimplify},
\hyperlink{diractrick}{DiracTrick}.

\subsection{Examples}

\begin{Shaded}
\begin{Highlighting}[]
\NormalTok{GA}\OperatorTok{[}\DecValTok{5}\OperatorTok{,} \SpecialCharTok{\textbackslash{}}\OperatorTok{[}\NormalTok{Mu}\OperatorTok{]]} 
 
\NormalTok{Anti5}\OperatorTok{[}\SpecialCharTok{\%}\OperatorTok{]} 
 
\NormalTok{Anti5}\OperatorTok{[}\SpecialCharTok{\%}\OperatorTok{,} \SpecialCharTok{{-}}\DecValTok{1}\OperatorTok{]}
\end{Highlighting}
\end{Shaded}

\begin{dmath*}\breakingcomma
\bar{\gamma }^5.\bar{\gamma }^{\mu }
\end{dmath*}

\begin{dmath*}\breakingcomma
-\bar{\gamma }^{\mu }.\bar{\gamma }^5
\end{dmath*}

\begin{dmath*}\breakingcomma
\bar{\gamma }^5.\bar{\gamma }^{\mu }
\end{dmath*}

\begin{Shaded}
\begin{Highlighting}[]
\NormalTok{GA}\OperatorTok{[}\DecValTok{5}\OperatorTok{,} \SpecialCharTok{\textbackslash{}}\OperatorTok{[}\NormalTok{Alpha}\OperatorTok{],} \SpecialCharTok{\textbackslash{}}\OperatorTok{[}\FunctionTok{Beta}\OperatorTok{],} \SpecialCharTok{\textbackslash{}}\OperatorTok{[}\FunctionTok{Gamma}\OperatorTok{],} \SpecialCharTok{\textbackslash{}}\OperatorTok{[}\NormalTok{Delta}\OperatorTok{]]} 
 
\NormalTok{Anti5}\OperatorTok{[}\SpecialCharTok{\%}\OperatorTok{,} \DecValTok{2}\OperatorTok{]} 
 
\NormalTok{Anti5}\OperatorTok{[}\SpecialCharTok{\%\%}\OperatorTok{,} \FunctionTok{Infinity}\OperatorTok{]} 
 
\NormalTok{Anti5}\OperatorTok{[}\SpecialCharTok{\%}\OperatorTok{,} \SpecialCharTok{{-}}\FunctionTok{Infinity}\OperatorTok{]}
\end{Highlighting}
\end{Shaded}

\begin{dmath*}\breakingcomma
\bar{\gamma }^5.\bar{\gamma }^{\alpha }.\bar{\gamma }^{\beta }.\bar{\gamma }^{\gamma }.\bar{\gamma }^{\delta }
\end{dmath*}

\begin{dmath*}\breakingcomma
\bar{\gamma }^{\alpha }.\bar{\gamma }^{\beta }.\bar{\gamma }^5.\bar{\gamma }^{\gamma }.\bar{\gamma }^{\delta }
\end{dmath*}

\begin{dmath*}\breakingcomma
\bar{\gamma }^{\alpha }.\bar{\gamma }^{\beta }.\bar{\gamma }^{\gamma }.\bar{\gamma }^{\delta }.\bar{\gamma }^5
\end{dmath*}

\begin{dmath*}\breakingcomma
\bar{\gamma }^5.\bar{\gamma }^{\alpha }.\bar{\gamma }^{\beta }.\bar{\gamma }^{\gamma }.\bar{\gamma }^{\delta }
\end{dmath*}

In the naive \(\gamma^5\)-scheme \(D\)-dimensional \(\gamma\)-matrices
anticommute with \(\gamma^5\).

\begin{Shaded}
\begin{Highlighting}[]
\NormalTok{GA5 . GAD}\OperatorTok{[}\SpecialCharTok{\textbackslash{}}\OperatorTok{[}\NormalTok{Mu}\OperatorTok{]]} 
 
\NormalTok{Anti5}\OperatorTok{[}\SpecialCharTok{\%}\OperatorTok{]}
\end{Highlighting}
\end{Shaded}

\begin{dmath*}\breakingcomma
\bar{\gamma }^5.\gamma ^{\mu }
\end{dmath*}

\begin{dmath*}\breakingcomma
-\gamma ^{\mu }.\bar{\gamma }^5
\end{dmath*}

\texttt{Anti5} also works in the t'Hooft-Veltman-Breitenlohner-Maison
scheme

\begin{Shaded}
\begin{Highlighting}[]
\NormalTok{FCSetDiracGammaScheme}\OperatorTok{[}\StringTok{"BMHV"}\OperatorTok{]}\NormalTok{; }
 
\NormalTok{Anti5}\OperatorTok{[}\NormalTok{GA5 . GAD}\OperatorTok{[}\SpecialCharTok{\textbackslash{}}\OperatorTok{[}\NormalTok{Mu}\OperatorTok{]]]}
\end{Highlighting}
\end{Shaded}

\begin{dmath*}\breakingcomma
2 \hat{\gamma }^{\mu }.\bar{\gamma }^5-\gamma ^{\mu }.\bar{\gamma }^5
\end{dmath*}

\begin{Shaded}
\begin{Highlighting}[]
\NormalTok{FCSetDiracGammaScheme}\OperatorTok{[}\StringTok{"NDR"}\OperatorTok{]}\NormalTok{;}
\end{Highlighting}
\end{Shaded}

\end{document}
