% !TeX program = pdflatex
% !TeX root = AntiCommutator.tex

\documentclass[../FeynCalcManual.tex]{subfiles}
\begin{document}
\hypertarget{anticommutator}{
\section{AntiCommutator}\label{anticommutator}\index{AntiCommutator}}

\texttt{AntiCommutator[\allowbreak{}x,\ \allowbreak{}y] = c} defines the
anti-commutator of the non commuting objects \texttt{x} and \texttt{y}.

\subsection{See also}

\hyperlink{toc}{Overview}, \hyperlink{commutator}{Commutator},
\hyperlink{commutatorexplicit}{CommutatorExplicit},
\hyperlink{declarenoncommutative}{DeclareNonCommutative},
\hyperlink{dotsimplify}{DotSimplify}.

\subsection{Examples}

This declares \texttt{a} and \texttt{b} as noncommutative variables.

\begin{Shaded}
\begin{Highlighting}[]
\NormalTok{DeclareNonCommutative}\OperatorTok{[}\FunctionTok{a}\OperatorTok{,} \FunctionTok{b}\OperatorTok{]} 
 
\NormalTok{AntiCommutator}\OperatorTok{[}\FunctionTok{a}\OperatorTok{,} \FunctionTok{b}\OperatorTok{]} 
 
\NormalTok{CommutatorExplicit}\OperatorTok{[}\SpecialCharTok{\%}\OperatorTok{]}
\end{Highlighting}
\end{Shaded}

\begin{dmath*}\breakingcomma
\{a,\medspace b\}
\end{dmath*}

\begin{dmath*}\breakingcomma
a.b+b.a
\end{dmath*}

\begin{Shaded}
\begin{Highlighting}[]
\NormalTok{CommutatorExplicit}\OperatorTok{[}\NormalTok{AntiCommutator}\OperatorTok{[}\FunctionTok{a} \SpecialCharTok{+} \FunctionTok{b}\OperatorTok{,} \FunctionTok{a} \SpecialCharTok{{-}} \DecValTok{2} \FunctionTok{b} \OperatorTok{]]}
\end{Highlighting}
\end{Shaded}

\begin{dmath*}\breakingcomma
(a-2 b).(a+b)+(a+b).(a-2 b)
\end{dmath*}

\begin{Shaded}
\begin{Highlighting}[]
\NormalTok{DotSimplify}\OperatorTok{[}\NormalTok{AntiCommutator}\OperatorTok{[}\FunctionTok{a} \SpecialCharTok{+} \FunctionTok{b}\OperatorTok{,} \FunctionTok{a} \SpecialCharTok{{-}} \DecValTok{2} \FunctionTok{b} \OperatorTok{]]}
\end{Highlighting}
\end{Shaded}

\begin{dmath*}\breakingcomma
-a.b-b.a+2 a.a-4 b.b
\end{dmath*}

\begin{Shaded}
\begin{Highlighting}[]
\NormalTok{DeclareNonCommutative}\OperatorTok{[}\FunctionTok{c}\OperatorTok{,} \FunctionTok{d}\OperatorTok{,}\NormalTok{ ct}\OperatorTok{,} \FunctionTok{dt}\OperatorTok{]}
\end{Highlighting}
\end{Shaded}

Defining \texttt{\{\allowbreak{}c,\ \allowbreak{}d\} = z} results in
replacements of \texttt{c.d} by \texttt{z-d.c.}

\begin{Shaded}
\begin{Highlighting}[]
\NormalTok{AntiCommutator}\OperatorTok{[}\FunctionTok{c}\OperatorTok{,} \FunctionTok{d}\OperatorTok{]} \ExtensionTok{=} \FunctionTok{z} 
 
\NormalTok{DotSimplify}\OperatorTok{[} \FunctionTok{d}\NormalTok{ . }\FunctionTok{c}\NormalTok{ . }\FunctionTok{d} \OperatorTok{]}
\end{Highlighting}
\end{Shaded}

\begin{dmath*}\breakingcomma
z
\end{dmath*}

\begin{dmath*}\breakingcomma
d z-d.d.c
\end{dmath*}

\begin{Shaded}
\begin{Highlighting}[]
\NormalTok{AntiCommutator}\OperatorTok{[}\FunctionTok{dt}\OperatorTok{,}\NormalTok{ ct}\OperatorTok{]} \ExtensionTok{=}\NormalTok{ zt}
\end{Highlighting}
\end{Shaded}

\begin{dmath*}\breakingcomma
\text{zt}
\end{dmath*}

\begin{Shaded}
\begin{Highlighting}[]
\NormalTok{DotSimplify}\OperatorTok{[}\FunctionTok{dt}\NormalTok{ . ct . }\FunctionTok{dt}\OperatorTok{]}
\end{Highlighting}
\end{Shaded}

\begin{dmath*}\breakingcomma
\text{dt} \;\text{zt}-\text{ct}.\text{dt}.\text{dt}
\end{dmath*}

\begin{Shaded}
\begin{Highlighting}[]
\NormalTok{UnDeclareNonCommutative}\OperatorTok{[}\FunctionTok{a}\OperatorTok{,} \FunctionTok{b}\OperatorTok{,} \FunctionTok{c}\OperatorTok{,} \FunctionTok{d}\OperatorTok{,}\NormalTok{ ct}\OperatorTok{,} \FunctionTok{dt}\OperatorTok{]} 
 
\NormalTok{UnDeclareAllAntiCommutators}\OperatorTok{[]}
\end{Highlighting}
\end{Shaded}

\end{document}
