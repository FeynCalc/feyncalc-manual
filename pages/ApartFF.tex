% !TeX program = pdflatex
% !TeX root = ApartFF.tex

\documentclass[../FeynCalcManual.tex]{subfiles}
\begin{document}
\hypertarget{apartff}{
\section{ApartFF}\label{apartff}\index{ApartFF}}

\texttt{ApartFF[\allowbreak{}amp,\ \allowbreak{}\{\allowbreak{}q1,\ \allowbreak{}q2,\ \allowbreak{}...\}]}
partial fractions loop integrals by decomposing them into simpler
integrals that contain only linearly independent propagators. It uses
\texttt{FCApart} as a backend and is equally suitable for 1-loop and
multi-loop integrals.

\texttt{FCApart} implements an algorithm based on
\href{https://arxiv.org/abs/1204.2314}{arXiv:1204.2314} by F. Feng that
seems to employ a variety Leinartas's algorithm
(cf.~\href{https://arxiv.org/abs/1206.4740}{arXiv:1206.4740}). Unlike
Feng's \href{https://github.com/F-Feng/APart}{\$Apart} that is
applicable to general multivariate polynomials, \texttt{FCApart} is
tailored to work only with FeynCalc's \texttt{FeynAmpDenominator},
\texttt{Pair} and \texttt{CartesianPair} symbols, i.e.~it is less
general in this respect.

\texttt{ApartFF[\allowbreak{}amp * extraPiece1,\ \allowbreak{}extraPiece2,\ \allowbreak{}\{\allowbreak{}q1,\ \allowbreak{}q2,\ \allowbreak{}...\}]}
is a special working mode of \texttt{ApartFF}, where the final result of
partial fractioning \texttt{amp*extraPiece1} is multiplied by
\texttt{extraPiece2}. It is understood, that
\texttt{extraPiece1*extraPiece2} should be unity, e. g. when
\texttt{extraPiece1} is an \texttt{FAD}, while extraPiece is an
\texttt{SPD} inverse to it. This mode should be useful for nonstandard
integrals where the desired partial fraction decomposition can be
performed only after multiplying \texttt{amp} with \texttt{extraPiece1}.

\subsection{See also}

\hyperlink{toc}{Overview}, \hyperlink{fcapart}{FCApart},
\hyperlink{feynampdenominatorsimplify}{FeynAmpDenominatorSimplify},
\hyperlink{fcloopfindtensorbasis}{FCLoopFindTensorBasis}.

\subsection{Examples}

\begin{Shaded}
\begin{Highlighting}[]
\NormalTok{FCClearScalarProducts}\OperatorTok{[]}
\end{Highlighting}
\end{Shaded}

\begin{Shaded}
\begin{Highlighting}[]
\NormalTok{SPD}\OperatorTok{[}\FunctionTok{q}\OperatorTok{,} \FunctionTok{q}\OperatorTok{]}\NormalTok{ FAD}\OperatorTok{[\{}\FunctionTok{q}\OperatorTok{,} \FunctionTok{m}\OperatorTok{\}]} 
 
\NormalTok{ApartFF}\OperatorTok{[}\SpecialCharTok{\%}\OperatorTok{,} \OperatorTok{\{}\FunctionTok{q}\OperatorTok{\}]}
\end{Highlighting}
\end{Shaded}

\begin{dmath*}\breakingcomma
\frac{q^2}{q^2-m^2}
\end{dmath*}

\begin{dmath*}\breakingcomma
\frac{m^2}{q^2-m^2}
\end{dmath*}

\begin{Shaded}
\begin{Highlighting}[]
\NormalTok{SPD}\OperatorTok{[}\FunctionTok{q}\OperatorTok{,} \FunctionTok{p}\OperatorTok{]}\NormalTok{ SPD}\OperatorTok{[}\FunctionTok{q}\OperatorTok{,} \FunctionTok{r}\OperatorTok{]}\NormalTok{ FAD}\OperatorTok{[\{}\FunctionTok{q}\OperatorTok{\},} \OperatorTok{\{}\FunctionTok{q} \SpecialCharTok{{-}} \FunctionTok{p}\OperatorTok{\},} \OperatorTok{\{}\FunctionTok{q} \SpecialCharTok{{-}} \FunctionTok{r}\OperatorTok{\}]} 
 
\NormalTok{ApartFF}\OperatorTok{[}\SpecialCharTok{\%}\OperatorTok{,} \OperatorTok{\{}\FunctionTok{q}\OperatorTok{\}]}
\end{Highlighting}
\end{Shaded}

\begin{dmath*}\breakingcomma
\frac{(p\cdot q) (q\cdot r)}{q^2.(q-p)^2.(q-r)^2}
\end{dmath*}

\begin{dmath*}\breakingcomma
\frac{p^2 r^2}{4 q^2.(q-p)^2.(q-r)^2}+\frac{p^2+2 (q\cdot r)+2 r^2}{4 q^2.(-p+q+r)^2}+-\frac{p^2}{4 q^2.(q-p)^2}-\frac{r^2}{4 q^2.(q-r)^2}
\end{dmath*}

\begin{Shaded}
\begin{Highlighting}[]
\NormalTok{FAD}\OperatorTok{[\{}\FunctionTok{q}\OperatorTok{\},} \OperatorTok{\{}\FunctionTok{q} \SpecialCharTok{{-}} \FunctionTok{p}\OperatorTok{\},} \OperatorTok{\{}\FunctionTok{q} \SpecialCharTok{+} \FunctionTok{p}\OperatorTok{\}]} 
 
\NormalTok{ApartFF}\OperatorTok{[}\SpecialCharTok{\%}\OperatorTok{,} \OperatorTok{\{}\FunctionTok{q}\OperatorTok{\}]}
\end{Highlighting}
\end{Shaded}

\begin{dmath*}\breakingcomma
\frac{1}{q^2.(q-p)^2.(p+q)^2}
\end{dmath*}

\begin{dmath*}\breakingcomma
\frac{1}{p^2 q^2.(q-p)^2}-\frac{1}{p^2 q^2.(q-2 p)^2}
\end{dmath*}

\begin{Shaded}
\begin{Highlighting}[]
\NormalTok{SPD}\OperatorTok{[}\FunctionTok{p}\OperatorTok{,}\NormalTok{ q1}\OperatorTok{]}\NormalTok{ SPD}\OperatorTok{[}\FunctionTok{p}\OperatorTok{,}\NormalTok{ q2}\OperatorTok{]}\SpecialCharTok{\^{}}\DecValTok{2}\NormalTok{ FAD}\OperatorTok{[\{}\NormalTok{q1}\OperatorTok{,} \FunctionTok{m}\OperatorTok{\},} \OperatorTok{\{}\NormalTok{q2}\OperatorTok{,} \FunctionTok{m}\OperatorTok{\},}\NormalTok{ q1 }\SpecialCharTok{{-}} \FunctionTok{p}\OperatorTok{,}\NormalTok{ q2 }\SpecialCharTok{{-}} \FunctionTok{p}\OperatorTok{,}\NormalTok{ q1 }\SpecialCharTok{{-}}\NormalTok{ q2}\OperatorTok{]} 
 
\NormalTok{ApartFF}\OperatorTok{[}\SpecialCharTok{\%}\OperatorTok{,} \OperatorTok{\{}\NormalTok{q1}\OperatorTok{,}\NormalTok{ q2}\OperatorTok{\}]}
\end{Highlighting}
\end{Shaded}

\begin{dmath*}\breakingcomma
\frac{(p\cdot \;\text{q1}) (p\cdot \;\text{q2})^2}{\left(\text{q1}^2-m^2\right).\left(\text{q2}^2-m^2\right).(\text{q1}-p)^2.(\text{q2}-p)^2.(\text{q1}-\text{q2})^2}
\end{dmath*}

\begin{dmath*}\breakingcomma
\frac{\left(m^2+p^2\right)^3}{8 \left(\text{q1}^2-m^2\right).\left(\text{q2}^2-m^2\right).(\text{q1}-p)^2.(\text{q1}-\text{q2})^2.(\text{q2}-p)^2}-\frac{\left(m^2+p^2\right)^2}{4 \left(\text{q1}^2-m^2\right).\left(\text{q2}^2-m^2\right).(\text{q1}-p)^2.(\text{q1}-\text{q2})^2}+\frac{\left(m^2+p^2\right) \left(m^2+2 p^2\right)}{4 \;\text{q1}^2.\text{q2}^2.\left((\text{q1}-p)^2-m^2\right).(\text{q1}-\text{q2})^2}-\frac{\left(m^2+p^2\right) (p\cdot \;\text{q1})}{4 \;\text{q1}^2.\text{q2}^2.(\text{q1}-\text{q2})^2.\left((\text{q2}-p)^2-m^2\right)}-\frac{\left(m^2+p^2\right) (p\cdot \;\text{q1})}{4 \left(\text{q1}^2-m^2\right).\left(\text{q2}^2-m^2\right).(\text{q1}-\text{q2})^2.(\text{q2}-p)^2}-\frac{p\cdot \;\text{q1}}{4 \left(\text{q2}^2-m^2\right).(\text{q1}-p)^2.(\text{q1}-\text{q2})^2}-\frac{m^2+p\cdot \;\text{q1}+p^2}{4 \left(\text{q1}^2-m^2\right).(\text{q1}-\text{q2})^2.(\text{q2}-p)^2}+\frac{m^2+2 (p\cdot \;\text{q1})+p^2}{8 \left(\text{q1}^2-m^2\right).\left(\text{q2}^2-m^2\right).(\text{q1}-\text{q2})^2}
\end{dmath*}

\begin{Shaded}
\begin{Highlighting}[]
\NormalTok{SPD}\OperatorTok{[}\FunctionTok{q}\OperatorTok{,} \FunctionTok{p}\OperatorTok{]}\NormalTok{ FAD}\OperatorTok{[\{}\FunctionTok{q}\OperatorTok{,} \FunctionTok{m}\OperatorTok{\},} \OperatorTok{\{}\FunctionTok{q} \SpecialCharTok{{-}} \FunctionTok{p}\OperatorTok{,} \DecValTok{0}\OperatorTok{\}]} 
 
\NormalTok{ApartFF}\OperatorTok{[}\SpecialCharTok{\%}\OperatorTok{,} \OperatorTok{\{}\FunctionTok{q}\OperatorTok{\}]}
\end{Highlighting}
\end{Shaded}

\begin{dmath*}\breakingcomma
\frac{p\cdot q}{\left(q^2-m^2\right).(q-p)^2}
\end{dmath*}

\begin{dmath*}\breakingcomma
\frac{m^2+p^2}{2 q^2.\left((q-p)^2-m^2\right)}-\frac{1}{2 \left(q^2-m^2\right)}
\end{dmath*}

If the propagators should not be altered via momentum shifts
(e.g.~because they belong to a previously identified topology), use the
option \texttt{FDS->False}

\begin{Shaded}
\begin{Highlighting}[]
\NormalTok{int }\ExtensionTok{=}\NormalTok{ SPD}\OperatorTok{[}\NormalTok{q2}\OperatorTok{,} \FunctionTok{p}\OperatorTok{]}\NormalTok{ SPD}\OperatorTok{[}\NormalTok{q1}\OperatorTok{,} \FunctionTok{p}\OperatorTok{]}\NormalTok{ FAD}\OperatorTok{[\{}\NormalTok{q1}\OperatorTok{,} \FunctionTok{m}\OperatorTok{\},} \OperatorTok{\{}\NormalTok{q2}\OperatorTok{,} \FunctionTok{m}\OperatorTok{\},}\NormalTok{ q1 }\SpecialCharTok{{-}} \FunctionTok{p}\OperatorTok{,}\NormalTok{ q2 }\SpecialCharTok{{-}} \FunctionTok{p}\OperatorTok{,}\NormalTok{ q2 }\SpecialCharTok{{-}}\NormalTok{ q1}\OperatorTok{]}
\end{Highlighting}
\end{Shaded}

\begin{dmath*}\breakingcomma
\frac{(p\cdot \;\text{q1}) (p\cdot \;\text{q2})}{\left(\text{q1}^2-m^2\right).\left(\text{q2}^2-m^2\right).(\text{q1}-p)^2.(\text{q2}-p)^2.(\text{q2}-\text{q1})^2}
\end{dmath*}

\begin{Shaded}
\begin{Highlighting}[]
\NormalTok{ApartFF}\OperatorTok{[}\NormalTok{int}\OperatorTok{,} \OperatorTok{\{}\NormalTok{q1}\OperatorTok{,}\NormalTok{ q2}\OperatorTok{\}]}
\end{Highlighting}
\end{Shaded}

\begin{dmath*}\breakingcomma
\frac{\left(m^2+p^2\right)^2}{4 \left(\text{q1}^2-m^2\right).\left(\text{q2}^2-m^2\right).(\text{q1}-p)^2.(\text{q1}-\text{q2})^2.(\text{q2}-p)^2}+\frac{m^2+p^2}{2 \;\text{q1}^2.\text{q2}^2.\left((\text{q1}-p)^2-m^2\right).(\text{q1}-\text{q2})^2}-\frac{m^2+p^2}{2 \left(\text{q1}^2-m^2\right).\left(\text{q2}^2-m^2\right).(\text{q1}-p)^2.(\text{q1}-\text{q2})^2}-\frac{1}{2 \left(\text{q1}^2-m^2\right).(\text{q1}-\text{q2})^2.(\text{q2}-p)^2}+\frac{1}{4 \left(\text{q1}^2-m^2\right).\left(\text{q2}^2-m^2\right).(\text{q1}-\text{q2})^2}
\end{dmath*}

\begin{Shaded}
\begin{Highlighting}[]
\NormalTok{ApartFF}\OperatorTok{[}\NormalTok{int}\OperatorTok{,} \OperatorTok{\{}\NormalTok{q1}\OperatorTok{,}\NormalTok{ q2}\OperatorTok{\},}\NormalTok{ FDS }\OtherTok{{-}\textgreater{}} \ConstantTok{False}\OperatorTok{]}
\end{Highlighting}
\end{Shaded}

\begin{dmath*}\breakingcomma
\frac{\left(m^2+p^2\right)^2}{4 \left(\text{q1}^2-m^2\right).\left(\text{q2}^2-m^2\right).(\text{q1}-p)^2.(\text{q2}-p)^2.(\text{q2}-\text{q1})^2}+\frac{m^2+p^2}{4 \left(\text{q1}^2-m^2\right).(\text{q1}-p)^2.(\text{q2}-p)^2.(\text{q2}-\text{q1})^2}-\frac{m^2+p^2}{4 \left(\text{q1}^2-m^2\right).\left(\text{q2}^2-m^2\right).(\text{q1}-p)^2.(\text{q2}-\text{q1})^2}-\frac{m^2+p^2}{4 \left(\text{q1}^2-m^2\right).\left(\text{q2}^2-m^2\right).(\text{q2}-p)^2.(\text{q2}-\text{q1})^2}+\frac{m^2+p^2}{4 \left(\text{q2}^2-m^2\right).(\text{q1}-p)^2.(\text{q2}-p)^2.(\text{q2}-\text{q1})^2}-\frac{1}{4 \left(\text{q1}^2-m^2\right).(\text{q2}-p)^2.(\text{q2}-\text{q1})^2}-\frac{1}{4 \left(\text{q2}^2-m^2\right).(\text{q1}-p)^2.(\text{q2}-\text{q1})^2}+\frac{1}{4 \left(\text{q1}^2-m^2\right).\left(\text{q2}^2-m^2\right).(\text{q2}-\text{q1})^2}+\frac{1}{4 (\text{q1}-p)^2.(\text{q2}-p)^2.(\text{q2}-\text{q1})^2}
\end{dmath*}

If the partial fractioning should be performed only w. r. t. the
denominators but not numerators, use the option
\texttt{Numerator->False}

\begin{Shaded}
\begin{Highlighting}[]
\NormalTok{int }\ExtensionTok{=}\NormalTok{ FAD}\OperatorTok{[}\FunctionTok{k}\OperatorTok{,} \FunctionTok{p} \SpecialCharTok{{-}} \FunctionTok{k}\OperatorTok{,} \OperatorTok{\{}\FunctionTok{k}\OperatorTok{,} \FunctionTok{m}\OperatorTok{\}]}\NormalTok{ SPD}\OperatorTok{[}\FunctionTok{p}\OperatorTok{,} \FunctionTok{k}\OperatorTok{]}
\end{Highlighting}
\end{Shaded}

\begin{dmath*}\breakingcomma
\frac{k\cdot p}{k^2.(p-k)^2.\left(k^2-m^2\right)}
\end{dmath*}

\begin{Shaded}
\begin{Highlighting}[]
\NormalTok{ApartFF}\OperatorTok{[}\NormalTok{int}\OperatorTok{,} \OperatorTok{\{}\FunctionTok{k}\OperatorTok{\}]}
\end{Highlighting}
\end{Shaded}

\begin{dmath*}\breakingcomma
\frac{m^2+p^2}{2 m^2 k^2.\left((k-p)^2-m^2\right)}+-\frac{1}{2 m^2 \left(k^2-m^2\right)}-\frac{p^2}{2 m^2 k^2.(k-p)^2}
\end{dmath*}

\begin{Shaded}
\begin{Highlighting}[]
\NormalTok{ApartFF}\OperatorTok{[}\NormalTok{int}\OperatorTok{,} \OperatorTok{\{}\FunctionTok{k}\OperatorTok{\},} \FunctionTok{Numerator} \OtherTok{{-}\textgreater{}} \ConstantTok{False}\OperatorTok{]}
\end{Highlighting}
\end{Shaded}

\begin{dmath*}\breakingcomma
\frac{k\cdot p}{m^2 k^2.\left((k-p)^2-m^2\right)}-\frac{k\cdot p}{m^2 k^2.(k-p)^2}
\end{dmath*}

Using the option \texttt{FeynAmpDenominator ->False} we can specify that
integrals without numerators should not be partial fractioned

\begin{Shaded}
\begin{Highlighting}[]
\NormalTok{int }\ExtensionTok{=}\NormalTok{ FAD}\OperatorTok{[}\FunctionTok{k}\OperatorTok{,} \FunctionTok{p} \SpecialCharTok{{-}} \FunctionTok{k}\OperatorTok{,} \OperatorTok{\{}\FunctionTok{k}\OperatorTok{,} \FunctionTok{m}\OperatorTok{\}]}\NormalTok{ (SPD}\OperatorTok{[}\FunctionTok{q}\OperatorTok{]} \SpecialCharTok{+}\NormalTok{ SPD}\OperatorTok{[}\FunctionTok{p}\OperatorTok{,} \FunctionTok{k}\OperatorTok{]}\NormalTok{)}
\end{Highlighting}
\end{Shaded}

\begin{dmath*}\breakingcomma
\frac{k\cdot p+q^2}{k^2.(p-k)^2.\left(k^2-m^2\right)}
\end{dmath*}

\begin{Shaded}
\begin{Highlighting}[]
\NormalTok{ApartFF}\OperatorTok{[}\NormalTok{int}\OperatorTok{,} \OperatorTok{\{}\FunctionTok{k}\OperatorTok{\}]}
\end{Highlighting}
\end{Shaded}

\begin{dmath*}\breakingcomma
\frac{m^2+p^2+2 q^2}{2 m^2 k^2.\left((k-p)^2-m^2\right)}+-\frac{1}{2 m^2 \left(k^2-m^2\right)}-\frac{p^2+2 q^2}{2 m^2 k^2.(k-p)^2}
\end{dmath*}

\begin{Shaded}
\begin{Highlighting}[]
\NormalTok{ApartFF}\OperatorTok{[}\NormalTok{int}\OperatorTok{,} \OperatorTok{\{}\FunctionTok{k}\OperatorTok{\},}\NormalTok{ FeynAmpDenominator }\OtherTok{{-}\textgreater{}} \ConstantTok{False}\OperatorTok{]}
\end{Highlighting}
\end{Shaded}

\begin{dmath*}\breakingcomma
\frac{q^2}{k^2.\left(k^2-m^2\right).(p-k)^2}+\frac{m^2+p^2}{2 m^2 k^2.\left((k-p)^2-m^2\right)}+-\frac{1}{2 m^2 \left(k^2-m^2\right)}-\frac{p^2}{2 m^2 k^2.(k-p)^2}
\end{dmath*}

The \texttt{extraPiece}-trick is useful for cases where a direct partial
fractioning is not possible

\begin{Shaded}
\begin{Highlighting}[]
\NormalTok{int }\ExtensionTok{=}\NormalTok{ (SFAD}\OperatorTok{[\{\{}\DecValTok{0}\OperatorTok{,} \FunctionTok{k}\NormalTok{ . }\FunctionTok{l}\OperatorTok{\}\},} \FunctionTok{p} \SpecialCharTok{{-}} \FunctionTok{k}\OperatorTok{]}\NormalTok{ SPD}\OperatorTok{[}\FunctionTok{k}\OperatorTok{,} \FunctionTok{p}\OperatorTok{]}\NormalTok{)}
\end{Highlighting}
\end{Shaded}

\begin{dmath*}\breakingcomma
\frac{k\cdot p}{(k\cdot l+i \eta ).((p-k)^2+i \eta )}
\end{dmath*}

Here \texttt{ApartFF} cannot do anything

\begin{Shaded}
\begin{Highlighting}[]
\NormalTok{ApartFF}\OperatorTok{[}\NormalTok{int}\OperatorTok{,} \OperatorTok{\{}\FunctionTok{k}\OperatorTok{\}]}
\end{Highlighting}
\end{Shaded}

\begin{dmath*}\breakingcomma
\frac{k\cdot p}{(k\cdot l+i \eta ).((p-k)^2+i \eta )}
\end{dmath*}

Multiplying the integral with unity
\texttt{FAD[\allowbreak{}k]*SPD[\allowbreak{}k]} we can cast into a more
desirable form

\begin{Shaded}
\begin{Highlighting}[]
\NormalTok{ApartFF}\OperatorTok{[}\NormalTok{int FAD}\OperatorTok{[}\FunctionTok{k}\OperatorTok{],}\NormalTok{ SPD}\OperatorTok{[}\FunctionTok{k}\OperatorTok{],} \OperatorTok{\{}\FunctionTok{k}\OperatorTok{\}]} \SpecialCharTok{//}\NormalTok{ ApartFF}\OperatorTok{[}\NormalTok{\#}\OperatorTok{,} \OperatorTok{\{}\FunctionTok{k}\OperatorTok{\}]}\NormalTok{ \&}
\end{Highlighting}
\end{Shaded}

\begin{dmath*}\breakingcomma
\frac{k^2}{2 (k\cdot l+i \eta ).((p-k)^2+i \eta )}+\frac{p^2}{2 (k\cdot l+i \eta ).((k-p)^2+i \eta )}
\end{dmath*}

Here we need a second call to \texttt{ApartFF} since the first execution
doesn't drop scaleless integrals or perform any shifts in the
denominators.

Other examples of doing partial fraction decomposition for eikonal
integrals (e.g.~in SCET)

\begin{Shaded}
\begin{Highlighting}[]
\NormalTok{int1 }\ExtensionTok{=}\NormalTok{ SFAD}\OperatorTok{[\{\{}\DecValTok{0}\OperatorTok{,}\NormalTok{ nb . k2}\OperatorTok{\}\},} \OperatorTok{\{\{}\DecValTok{0}\OperatorTok{,}\NormalTok{ nb . (k1 }\SpecialCharTok{+}\NormalTok{ k2)}\OperatorTok{\}\}]}
\end{Highlighting}
\end{Shaded}

\begin{dmath*}\breakingcomma
\frac{1}{(\text{k2}\cdot \;\text{nb}+i \eta ).((\text{k1}+\text{k2})\cdot \;\text{nb}+i \eta )}
\end{dmath*}

\begin{Shaded}
\begin{Highlighting}[]
\NormalTok{ApartFF}\OperatorTok{[}\NormalTok{int1}\OperatorTok{,} \OperatorTok{\{}\NormalTok{k1}\OperatorTok{,}\NormalTok{ k2}\OperatorTok{\}]}
\end{Highlighting}
\end{Shaded}

\begin{dmath*}\breakingcomma
\frac{1}{(\text{k2}\cdot \;\text{nb}+i \eta ).(\text{k1}\cdot \;\text{nb}+\text{k2}\cdot \;\text{nb}+i \eta )}
\end{dmath*}

\begin{Shaded}
\begin{Highlighting}[]
\NormalTok{ApartFF}\OperatorTok{[}\NormalTok{ApartFF}\OperatorTok{[}\NormalTok{SFAD}\OperatorTok{[\{\{}\DecValTok{0}\OperatorTok{,}\NormalTok{ nb . k1}\OperatorTok{\}\}]}\NormalTok{ int1}\OperatorTok{,}\NormalTok{ SPD}\OperatorTok{[}\NormalTok{nb}\OperatorTok{,}\NormalTok{ k1}\OperatorTok{],} \OperatorTok{\{}\NormalTok{k1}\OperatorTok{,}\NormalTok{ k2}\OperatorTok{\}],} \OperatorTok{\{}\NormalTok{k1}\OperatorTok{,}\NormalTok{ k2}\OperatorTok{\}]}
\end{Highlighting}
\end{Shaded}

\begin{dmath*}\breakingcomma
\frac{1}{(\text{k1}\cdot \;\text{nb}+i \eta ).(\text{k2}\cdot \;\text{nb}+i \eta )}-\frac{1}{(\text{k1}\cdot \;\text{nb}+i \eta ).(\text{k1}\cdot \;\text{nb}+\text{k2}\cdot \;\text{nb}+i \eta )}
\end{dmath*}

\begin{Shaded}
\begin{Highlighting}[]
\NormalTok{int2 }\ExtensionTok{=}\NormalTok{ SPD}\OperatorTok{[}\NormalTok{nb}\OperatorTok{,}\NormalTok{ k2}\OperatorTok{]}\NormalTok{ SFAD}\OperatorTok{[\{\{}\DecValTok{0}\OperatorTok{,}\NormalTok{ nb . (k1 }\SpecialCharTok{+}\NormalTok{ k2)}\OperatorTok{\}\}]}
\end{Highlighting}
\end{Shaded}

\begin{dmath*}\breakingcomma
\frac{\text{k2}\cdot \;\text{nb}}{((\text{k1}+\text{k2})\cdot \;\text{nb}+i \eta )}
\end{dmath*}

\begin{Shaded}
\begin{Highlighting}[]
\NormalTok{ApartFF}\OperatorTok{[}\NormalTok{int2}\OperatorTok{,} \OperatorTok{\{}\NormalTok{k1}\OperatorTok{,}\NormalTok{ k2}\OperatorTok{\}]}
\end{Highlighting}
\end{Shaded}

\begin{dmath*}\breakingcomma
\frac{\text{k2}\cdot \;\text{nb}}{(\text{k1}\cdot \;\text{nb}+\text{k2}\cdot \;\text{nb}+i \eta )}
\end{dmath*}

\begin{Shaded}
\begin{Highlighting}[]
\NormalTok{ApartFF}\OperatorTok{[}\NormalTok{ApartFF}\OperatorTok{[}\NormalTok{SFAD}\OperatorTok{[\{\{}\DecValTok{0}\OperatorTok{,}\NormalTok{ nb . k1}\OperatorTok{\}\}]}\NormalTok{ int2}\OperatorTok{,}\NormalTok{ SPD}\OperatorTok{[}\NormalTok{nb}\OperatorTok{,}\NormalTok{ k1}\OperatorTok{],} \OperatorTok{\{}\NormalTok{k1}\OperatorTok{,}\NormalTok{ k2}\OperatorTok{\}],} \OperatorTok{\{}\NormalTok{k1}\OperatorTok{,}\NormalTok{ k2}\OperatorTok{\}]}
\end{Highlighting}
\end{Shaded}

\begin{dmath*}\breakingcomma
-\frac{\text{k1}\cdot \;\text{nb}}{(\text{k1}\cdot \;\text{nb}+\text{k2}\cdot \;\text{nb}+i \eta )}
\end{dmath*}

If we are working with a subset of propagators from a full integral, one
should better turn off loop momentum shifts and the dropping of
scaleless integrals

\begin{Shaded}
\begin{Highlighting}[]
\NormalTok{ApartFF}\OperatorTok{[}\NormalTok{ApartFF}\OperatorTok{[}\NormalTok{SFAD}\OperatorTok{[\{\{}\DecValTok{0}\OperatorTok{,}\NormalTok{ nb . k1}\OperatorTok{\}\}]}\NormalTok{ int2}\OperatorTok{,}\NormalTok{ SPD}\OperatorTok{[}\NormalTok{nb}\OperatorTok{,}\NormalTok{ k1}\OperatorTok{],} \OperatorTok{\{}\NormalTok{k1}\OperatorTok{,}\NormalTok{ k2}\OperatorTok{\}],} \OperatorTok{\{}\NormalTok{k1}\OperatorTok{,}\NormalTok{ k2}\OperatorTok{\},}\NormalTok{ FDS }\OtherTok{{-}\textgreater{}} \ConstantTok{False}\OperatorTok{,} 
\NormalTok{  DropScaleless }\OtherTok{{-}\textgreater{}} \ConstantTok{False}\OperatorTok{]}
\end{Highlighting}
\end{Shaded}

\begin{dmath*}\breakingcomma
1-\frac{\text{k1}\cdot \;\text{nb}}{(\text{k1}\cdot \;\text{nb}+\text{k2}\cdot \;\text{nb}+i \eta )}
\end{dmath*}

When we need to deal with linearly dependent external momenta, there
might be some relations between scalar products involving those momenta
and loop momenta. Since the routine cannot determine those relations
automatically, we need to supply them by hand via the option
\texttt{FinalSubstitutions}

\begin{Shaded}
\begin{Highlighting}[]
\NormalTok{FCClearScalarProducts}\OperatorTok{[]}\NormalTok{;}
\NormalTok{int3 }\ExtensionTok{=}\NormalTok{ SPD}\OperatorTok{[}\NormalTok{p1}\OperatorTok{,} \FunctionTok{q}\OperatorTok{]}\NormalTok{ FAD}\OperatorTok{[\{}\FunctionTok{q}\OperatorTok{,} \FunctionTok{m}\OperatorTok{\},} \OperatorTok{\{}\FunctionTok{q} \SpecialCharTok{{-}}\NormalTok{ p1 }\SpecialCharTok{{-}}\NormalTok{ p2}\OperatorTok{\}]}
\end{Highlighting}
\end{Shaded}

\begin{dmath*}\breakingcomma
\frac{\text{p1}\cdot q}{\left(q^2-m^2\right).(-\text{p1}-\text{p2}+q)^2}
\end{dmath*}

Supplying the kinematics alone doesn't work

\begin{Shaded}
\begin{Highlighting}[]
\NormalTok{kinRules }\ExtensionTok{=} \OperatorTok{\{}\NormalTok{SPD}\OperatorTok{[}\NormalTok{p1}\OperatorTok{]} \OtherTok{{-}\textgreater{}} \FunctionTok{M}\SpecialCharTok{\^{}}\DecValTok{2}\OperatorTok{,}\NormalTok{ SPD}\OperatorTok{[}\NormalTok{p2}\OperatorTok{]} \OtherTok{{-}\textgreater{}} \FunctionTok{M}\SpecialCharTok{\^{}}\DecValTok{2}\OperatorTok{,}\NormalTok{ SPD}\OperatorTok{[}\NormalTok{p1}\OperatorTok{,}\NormalTok{ p2}\OperatorTok{]} \OtherTok{{-}\textgreater{}} \FunctionTok{M}\SpecialCharTok{\^{}}\DecValTok{2}\OperatorTok{\}}
\end{Highlighting}
\end{Shaded}

\begin{dmath*}\breakingcomma
\left\{\text{p1}^2\to M^2,\text{p2}^2\to M^2,\text{p1}\cdot \;\text{p2}\to M^2\right\}
\end{dmath*}

\begin{Shaded}
\begin{Highlighting}[]
\NormalTok{ApartFF}\OperatorTok{[}\NormalTok{SPD}\OperatorTok{[}\NormalTok{p1}\OperatorTok{,} \FunctionTok{q}\OperatorTok{]}\NormalTok{ FAD}\OperatorTok{[\{}\FunctionTok{q}\OperatorTok{,} \FunctionTok{m}\OperatorTok{\},} \OperatorTok{\{}\FunctionTok{q} \SpecialCharTok{{-}}\NormalTok{ p1 }\SpecialCharTok{{-}}\NormalTok{ p2}\OperatorTok{\}],} \OperatorTok{\{}\FunctionTok{q}\OperatorTok{\},}\NormalTok{ FinalSubstitutions }\OtherTok{{-}\textgreater{}}\NormalTok{ kinRules}\OperatorTok{]}
\end{Highlighting}
\end{Shaded}

\begin{dmath*}\breakingcomma
\frac{\text{p1}\cdot q}{\left(q^2-m^2\right).(-\text{p1}-\text{p2}+q)^2}
\end{dmath*}

Using \texttt{FCLoopFindTensorBasis} we see that \texttt{p2} is
proportional to \texttt{p1}. Hence, we have an equality between
\texttt{p1.q} and \texttt{p2.q}

\begin{Shaded}
\begin{Highlighting}[]
\NormalTok{FCLoopFindTensorBasis}\OperatorTok{[\{}\NormalTok{p1}\OperatorTok{,}\NormalTok{ p2}\OperatorTok{\},}\NormalTok{ kinRules}\OperatorTok{,} \FunctionTok{n}\OperatorTok{]}
\end{Highlighting}
\end{Shaded}

\begin{dmath*}\breakingcomma
\left(
\begin{array}{c}
 \;\text{p1} \\
 \;\text{p2} \\
 \;\text{p2}\to \;\text{p1} \;\text{FCGV}(\text{Prefactor})(1) \\
\end{array}
\right)
\end{dmath*}

Supplying this information to \texttt{ApartFF} we can finally achieve
the desired simplification

\begin{Shaded}
\begin{Highlighting}[]
\NormalTok{ApartFF}\OperatorTok{[}\NormalTok{SPD}\OperatorTok{[}\NormalTok{p1}\OperatorTok{,} \FunctionTok{q}\OperatorTok{]}\NormalTok{ FAD}\OperatorTok{[\{}\FunctionTok{q}\OperatorTok{,} \FunctionTok{m}\OperatorTok{\},} \OperatorTok{\{}\FunctionTok{q} \SpecialCharTok{{-}}\NormalTok{ p1 }\SpecialCharTok{{-}}\NormalTok{ p2}\OperatorTok{\}],} \OperatorTok{\{}\FunctionTok{q}\OperatorTok{\},}\NormalTok{ FinalSubstitutions }\OtherTok{{-}\textgreater{}} \FunctionTok{Join}\OperatorTok{[}\NormalTok{kinRules}\OperatorTok{,} 
    \OperatorTok{\{}\NormalTok{SPD}\OperatorTok{[}\FunctionTok{q}\OperatorTok{,}\NormalTok{ p2}\OperatorTok{]} \OtherTok{{-}\textgreater{}}\NormalTok{ SPD}\OperatorTok{[}\FunctionTok{q}\OperatorTok{,}\NormalTok{ p1}\OperatorTok{]\}]]}
\end{Highlighting}
\end{Shaded}

\begin{dmath*}\breakingcomma
\frac{m^2+4 M^2}{4 \left(q^2-m^2\right).(-\text{p1}-\text{p2}+q)^2}-\frac{1}{4 \left(q^2-m^2\right)}
\end{dmath*}
\end{document}
