% !TeX program = pdflatex
% !TeX root = B0.tex

\documentclass[../FeynCalcManual.tex]{subfiles}
\begin{document}
\hypertarget{b0}{
\section{B0}\label{b0}\index{B0}}

\texttt{B0[\allowbreak{}pp,\ \allowbreak{}ma^2,\ \allowbreak{}mb^2]} is
the Passarino-Veltman two-point integral \(B_0\). All arguments are
scalars and have dimension mass squared. If the option \texttt{BReduce}
is set to \texttt{True}, certain \texttt{B0}'s are reduced to
\texttt{A0}'s. Setting the option \texttt{B0Unique} to \texttt{True}
simplifies \texttt{B0[\allowbreak{}a,\ \allowbreak{}0,\ \allowbreak{}a]}
and \texttt{B0[\allowbreak{}0,\ \allowbreak{}0,\ \allowbreak{}a]}.

\subsection{See also}

\hyperlink{toc}{Overview}, \hyperlink{b1}{B1}, \hyperlink{b00}{B00},
\hyperlink{b11}{B11}, \hyperlink{pave}{PaVe}.

\subsection{Examples}

\begin{Shaded}
\begin{Highlighting}[]
\NormalTok{B0}\OperatorTok{[}\NormalTok{SP}\OperatorTok{[}\FunctionTok{p}\OperatorTok{,} \FunctionTok{p}\OperatorTok{],} \FunctionTok{m}\SpecialCharTok{\^{}}\DecValTok{2}\OperatorTok{,} \FunctionTok{m}\SpecialCharTok{\^{}}\DecValTok{2}\OperatorTok{]}
\end{Highlighting}
\end{Shaded}

\begin{dmath*}\breakingcomma
\text{B}_0\left(\overline{p}^2,m^2,m^2\right)
\end{dmath*}

\begin{Shaded}
\begin{Highlighting}[]
\NormalTok{B0}\OperatorTok{[}\DecValTok{0}\OperatorTok{,} \DecValTok{0}\OperatorTok{,} \FunctionTok{m}\SpecialCharTok{\^{}}\DecValTok{2}\OperatorTok{,}\NormalTok{ B0Unique }\OtherTok{{-}\textgreater{}} \ConstantTok{True}\OperatorTok{,}\NormalTok{ B0Real }\OtherTok{{-}\textgreater{}} \ConstantTok{True}\OperatorTok{]}
\end{Highlighting}
\end{Shaded}

\begin{dmath*}\breakingcomma
\text{B}_0\left(0,m^2,m^2\right)+1
\end{dmath*}

\begin{Shaded}
\begin{Highlighting}[]
\NormalTok{B0}\OperatorTok{[}\FunctionTok{m}\SpecialCharTok{\^{}}\DecValTok{2}\OperatorTok{,} \DecValTok{0}\OperatorTok{,} \FunctionTok{m}\SpecialCharTok{\^{}}\DecValTok{2}\OperatorTok{,}\NormalTok{ B0Unique }\OtherTok{{-}\textgreater{}} \ConstantTok{True}\OperatorTok{,}\NormalTok{ B0Real }\OtherTok{{-}\textgreater{}} \ConstantTok{True}\OperatorTok{]}
\end{Highlighting}
\end{Shaded}

\begin{dmath*}\breakingcomma
\text{B}_0\left(0,m^2,m^2\right)+2
\end{dmath*}

\begin{Shaded}
\begin{Highlighting}[]
\NormalTok{B0}\OperatorTok{[}\DecValTok{0}\OperatorTok{,} \FunctionTok{m}\SpecialCharTok{\^{}}\DecValTok{2}\OperatorTok{,} \FunctionTok{m}\SpecialCharTok{\^{}}\DecValTok{2}\OperatorTok{]}
\end{Highlighting}
\end{Shaded}

\begin{dmath*}\breakingcomma
\text{B}_0\left(0,m^2,m^2\right)
\end{dmath*}
\end{document}
