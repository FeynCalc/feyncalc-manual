% !TeX program = pdflatex
% !TeX root = B00.tex

\documentclass[../FeynCalcManual.tex]{subfiles}
\begin{document}
\hypertarget{b00}{%
\section{B00}\label{b00}}

\texttt{B00[\allowbreak{}pp,\ \allowbreak{}ma^2,\ \allowbreak{}mb^2]} is
the Passarino-Veltman \(B_{00}\)-function, i.e., the coefficient
function of the metric tensor. All arguments are scalars and have
dimension mass squared.

\subsection{See also}

\hyperlink{toc}{Overview}, \hyperlink{b0}{B0}, \hyperlink{b1}{B1},
\hyperlink{pave}{PaVe}.

\subsection{Examples}

\begin{Shaded}
\begin{Highlighting}[]
\NormalTok{B00}\OperatorTok{[}\NormalTok{SPD}\OperatorTok{[}\FunctionTok{p}\OperatorTok{],} \FunctionTok{m}\SpecialCharTok{\^{}}\DecValTok{2}\OperatorTok{,} \FunctionTok{M}\SpecialCharTok{\^{}}\DecValTok{2}\OperatorTok{]}
\end{Highlighting}
\end{Shaded}

\begin{dmath*}\breakingcomma
\frac{\left(m^2-2 m M+M^2-p^2\right) \left(m^2+2 m M+M^2-p^2\right) \;\text{B}_0\left(p^2,m^2,M^2\right)}{4 (1-D) p^2}+\frac{\text{A}_0\left(M^2\right) \left(m^2-M^2-p^2\right)}{4 (1-D) p^2}-\frac{\text{A}_0\left(m^2\right) \left(m^2-M^2+p^2\right)}{4 (1-D) p^2}
\end{dmath*}

\begin{Shaded}
\begin{Highlighting}[]
\NormalTok{B00}\OperatorTok{[}\NormalTok{SPD}\OperatorTok{[}\FunctionTok{p}\OperatorTok{],} \FunctionTok{m}\SpecialCharTok{\^{}}\DecValTok{2}\OperatorTok{,} \FunctionTok{m}\SpecialCharTok{\^{}}\DecValTok{2}\OperatorTok{]}
\end{Highlighting}
\end{Shaded}

\begin{dmath*}\breakingcomma
-\frac{\left(4 m^2-p^2\right) \;\text{B}_0\left(p^2,m^2,m^2\right)}{4 (1-D)}-\frac{\text{A}_0\left(m^2\right)}{2 (1-D)}
\end{dmath*}

\begin{Shaded}
\begin{Highlighting}[]
\NormalTok{B00}\OperatorTok{[}\NormalTok{SPD}\OperatorTok{[}\FunctionTok{p}\OperatorTok{],} \FunctionTok{m}\SpecialCharTok{\^{}}\DecValTok{2}\OperatorTok{,} \FunctionTok{M}\SpecialCharTok{\^{}}\DecValTok{2}\OperatorTok{,}\NormalTok{ BReduce }\OtherTok{{-}\textgreater{}} \ConstantTok{False}\OperatorTok{]}
\end{Highlighting}
\end{Shaded}

\begin{dmath*}\breakingcomma
\text{B}_{00}\left(p^2,m^2,M^2\right)
\end{dmath*}

\begin{Shaded}
\begin{Highlighting}[]
\NormalTok{B00}\OperatorTok{[}\DecValTok{0}\OperatorTok{,} \FunctionTok{m}\SpecialCharTok{\^{}}\DecValTok{2}\OperatorTok{,} \FunctionTok{m}\SpecialCharTok{\^{}}\DecValTok{2}\OperatorTok{]}
\end{Highlighting}
\end{Shaded}

\begin{dmath*}\breakingcomma
-\frac{m^2 \;\text{B}_0\left(0,m^2,m^2\right)}{1-D}-\frac{\text{A}_0\left(m^2\right)}{2 (1-D)}
\end{dmath*}

\begin{Shaded}
\begin{Highlighting}[]
\NormalTok{B00}\OperatorTok{[}\NormalTok{SmallVariable}\OperatorTok{[}\FunctionTok{M}\SpecialCharTok{\^{}}\DecValTok{2}\OperatorTok{],} \FunctionTok{m}\SpecialCharTok{\^{}}\DecValTok{2}\OperatorTok{,} \FunctionTok{m}\SpecialCharTok{\^{}}\DecValTok{2}\OperatorTok{]}
\end{Highlighting}
\end{Shaded}

\begin{dmath*}\breakingcomma
-\frac{m^2 \;\text{B}_0\left(M^2,m^2,m^2\right)}{1-D}-\frac{\text{A}_0\left(m^2\right)}{2 (1-D)}
\end{dmath*}
\end{document}
