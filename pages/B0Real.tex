% !TeX program = pdflatex
% !TeX root = B0Real.tex

\documentclass[../FeynCalcManual.tex]{subfiles}
\begin{document}
\hypertarget{b0real}{
\section{B0Real}\label{b0real}\index{B0Real}}

\texttt{B0Real} is an option of \texttt{B0} (default \texttt{False}). If
set to \texttt{True}, \texttt{B0} is assumed to be real and the relation
\texttt{B0[\allowbreak{}a,\ \allowbreak{}0,\ \allowbreak{}a] = 2 + B0[\allowbreak{}0,\ \allowbreak{}a,\ \allowbreak{}a]}
is applied.

\subsection{See also}

\hyperlink{toc}{Overview}, \hyperlink{b0}{B0}.

\subsection{Examples}

By default the arguments are not assumed real.

\begin{Shaded}
\begin{Highlighting}[]
\NormalTok{B0}\OperatorTok{[}\FunctionTok{s}\OperatorTok{,} \DecValTok{0}\OperatorTok{,} \FunctionTok{s}\OperatorTok{]}
\end{Highlighting}
\end{Shaded}

\begin{dmath*}\breakingcomma
\text{B}_0(s,0,s)
\end{dmath*}

With \texttt{B0Real->True}, transformation is done.

\begin{Shaded}
\begin{Highlighting}[]
\NormalTok{B0}\OperatorTok{[}\FunctionTok{s}\OperatorTok{,} \DecValTok{0}\OperatorTok{,} \FunctionTok{s}\OperatorTok{,}\NormalTok{ B0Real }\OtherTok{{-}\textgreater{}} \ConstantTok{True}\OperatorTok{,}\NormalTok{ B0Unique }\OtherTok{{-}\textgreater{}} \ConstantTok{True}\OperatorTok{]}
\end{Highlighting}
\end{Shaded}

\begin{dmath*}\breakingcomma
\text{B}_0(0,s,s)+2
\end{dmath*}
\end{document}
