% !TeX program = pdflatex
% !TeX root = B0Unique.tex

\documentclass[../FeynCalcManual.tex]{subfiles}
\begin{document}
\hypertarget{b0unique}{
\section{B0Unique}\label{b0unique}\index{B0Unique}}

\texttt{B0Unique} is an option of \texttt{B0}. If set to \texttt{True},
\texttt{B0[\allowbreak{}0,\ \allowbreak{}0,\ \allowbreak{}m2]} is
replaced with
\texttt{(B0[\allowbreak{}0,\ \allowbreak{}m2,\ \allowbreak{}m2]+1)} and
\texttt{B0[\allowbreak{}m2,\ \allowbreak{}0,\ \allowbreak{}m2]}
simplifies to
\texttt{(B0[\allowbreak{}0,\ \allowbreak{}m2,\ \allowbreak{}m2]+2)}.

\subsection{See also}

\hyperlink{toc}{Overview}, \hyperlink{b0}{B0}.

\subsection{Examples}

By default no transformation is done.

\begin{Shaded}
\begin{Highlighting}[]
\NormalTok{B0}\OperatorTok{[}\DecValTok{0}\OperatorTok{,} \DecValTok{0}\OperatorTok{,} \FunctionTok{s}\OperatorTok{]}
\end{Highlighting}
\end{Shaded}

\begin{dmath*}\breakingcomma
\text{B}_0(0,0,s)
\end{dmath*}

With \texttt{B0Real->True} following transformation is applied

\begin{Shaded}
\begin{Highlighting}[]
\NormalTok{B0}\OperatorTok{[}\DecValTok{0}\OperatorTok{,} \DecValTok{0}\OperatorTok{,} \FunctionTok{s}\OperatorTok{,}\NormalTok{ B0Unique }\OtherTok{{-}\textgreater{}} \ConstantTok{True}\OperatorTok{]}
\end{Highlighting}
\end{Shaded}

\begin{dmath*}\breakingcomma
\text{B}_0(0,s,s)+1
\end{dmath*}
\end{document}
