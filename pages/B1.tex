% !TeX program = pdflatex
% !TeX root = B1.tex

\documentclass[../FeynCalcManual.tex]{subfiles}
\begin{document}
\hypertarget{b1}{%
\section{B1}\label{b1}}

\texttt{B1[\allowbreak{}pp,\ \allowbreak{}ma^2,\ \allowbreak{}mb^2]} the
Passarino-Veltman \(B_1\)-function. All arguments are scalars and have
dimension mass squared.

\subsection{See also}

\hyperlink{toc}{Overview}, \hyperlink{b0}{B0}, \hyperlink{b00}{B00},
\hyperlink{b11}{B11}, \hyperlink{pave}{PaVe},
\hyperlink{pavereduce}{PaVeReduce}.

\subsection{Examples}

\begin{Shaded}
\begin{Highlighting}[]
\NormalTok{B1}\OperatorTok{[}\NormalTok{SPD}\OperatorTok{[}\FunctionTok{p}\OperatorTok{],} \FunctionTok{m}\SpecialCharTok{\^{}}\DecValTok{2}\OperatorTok{,} \FunctionTok{M}\SpecialCharTok{\^{}}\DecValTok{2}\OperatorTok{]}
\end{Highlighting}
\end{Shaded}

\begin{dmath*}\breakingcomma
-\frac{\left(m^2-M^2+p^2\right) \;\text{B}_0\left(p^2,m^2,M^2\right)}{2 p^2}+\frac{\text{A}_0\left(m^2\right)}{2 p^2}-\frac{\text{A}_0\left(M^2\right)}{2 p^2}
\end{dmath*}

\begin{Shaded}
\begin{Highlighting}[]
\NormalTok{B1}\OperatorTok{[}\NormalTok{SPD}\OperatorTok{[}\FunctionTok{p}\OperatorTok{],} \FunctionTok{m}\SpecialCharTok{\^{}}\DecValTok{2}\OperatorTok{,} \FunctionTok{M}\SpecialCharTok{\^{}}\DecValTok{2}\OperatorTok{,}\NormalTok{ BReduce }\OtherTok{{-}\textgreater{}} \ConstantTok{False}\OperatorTok{]}
\end{Highlighting}
\end{Shaded}

\begin{dmath*}\breakingcomma
\text{B}_1\left(p^2,m^2,M^2\right)
\end{dmath*}

\begin{Shaded}
\begin{Highlighting}[]
\NormalTok{B1}\OperatorTok{[}\NormalTok{SP}\OperatorTok{[}\FunctionTok{p}\OperatorTok{],} \FunctionTok{m}\SpecialCharTok{\^{}}\DecValTok{2}\OperatorTok{,} \FunctionTok{m}\SpecialCharTok{\^{}}\DecValTok{2}\OperatorTok{]}
\end{Highlighting}
\end{Shaded}

\begin{dmath*}\breakingcomma
-\frac{1}{2} \;\text{B}_0\left(\overline{p}^2,m^2,m^2\right)
\end{dmath*}

\begin{Shaded}
\begin{Highlighting}[]
\NormalTok{B1}\OperatorTok{[}\NormalTok{SPD}\OperatorTok{[}\FunctionTok{p}\OperatorTok{],} \FunctionTok{m}\SpecialCharTok{\^{}}\DecValTok{2}\OperatorTok{,} \FunctionTok{m}\SpecialCharTok{\^{}}\DecValTok{2}\OperatorTok{,}\NormalTok{ BReduce }\OtherTok{{-}\textgreater{}} \ConstantTok{False}\OperatorTok{]}
\end{Highlighting}
\end{Shaded}

\begin{dmath*}\breakingcomma
\text{B}_1\left(p^2,m^2,m^2\right)
\end{dmath*}

\begin{Shaded}
\begin{Highlighting}[]
\NormalTok{B1}\OperatorTok{[}\FunctionTok{m}\SpecialCharTok{\^{}}\DecValTok{2}\OperatorTok{,} \FunctionTok{m}\SpecialCharTok{\^{}}\DecValTok{2}\OperatorTok{,} \DecValTok{0}\OperatorTok{]}
\end{Highlighting}
\end{Shaded}

\begin{dmath*}\breakingcomma
\frac{\text{A}_0\left(m^2\right)}{2 m^2}-\text{B}_0\left(m^2,0,m^2\right)
\end{dmath*}

\begin{Shaded}
\begin{Highlighting}[]
\NormalTok{B1}\OperatorTok{[}\FunctionTok{m}\SpecialCharTok{\^{}}\DecValTok{2}\OperatorTok{,} \FunctionTok{m}\SpecialCharTok{\^{}}\DecValTok{2}\OperatorTok{,} \DecValTok{0}\OperatorTok{,}\NormalTok{ BReduce }\OtherTok{{-}\textgreater{}} \ConstantTok{False}\OperatorTok{]}
\end{Highlighting}
\end{Shaded}

\begin{dmath*}\breakingcomma
\text{B}_1\left(m^2,m^2,0\right)
\end{dmath*}

\begin{Shaded}
\begin{Highlighting}[]
\NormalTok{B1}\OperatorTok{[}\DecValTok{0}\OperatorTok{,} \DecValTok{0}\OperatorTok{,} \FunctionTok{m}\SpecialCharTok{\^{}}\DecValTok{2}\OperatorTok{]}
\end{Highlighting}
\end{Shaded}

\begin{dmath*}\breakingcomma
\text{B}_1\left(0,0,m^2\right)
\end{dmath*}

\begin{Shaded}
\begin{Highlighting}[]
\NormalTok{B1}\OperatorTok{[}\NormalTok{pp}\OperatorTok{,}\NormalTok{ SmallVariable}\OperatorTok{[}\NormalTok{SMP}\OperatorTok{[}\StringTok{"m\_e"}\OperatorTok{]}\SpecialCharTok{\^{}}\DecValTok{2}\OperatorTok{],} \FunctionTok{Subsuperscript}\OperatorTok{[}\FunctionTok{m}\OperatorTok{,} \DecValTok{2}\OperatorTok{,} \DecValTok{2}\OperatorTok{]]}
\end{Highlighting}
\end{Shaded}

\begin{dmath*}\breakingcomma
-\frac{(\text{pp}-m_2^2) \;\text{B}_0\left(\text{pp},m_e^2,m_2^2\right)}{2 \;\text{pp}}-\frac{\text{A}_0(m_2^2)}{2 \;\text{pp}}
\end{dmath*}
\end{document}
