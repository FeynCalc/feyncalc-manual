% !TeX program = pdflatex
% !TeX root = BReduce.tex

\documentclass[../FeynCalcManual.tex]{subfiles}
\begin{document}
\hypertarget{breduce}{%
\section{BReduce}\label{breduce}}

\texttt{BReduce} is an option for \texttt{B0}, \texttt{B00},
\texttt{B1}, \texttt{B11} determining whether reductions to \texttt{A0}
and \texttt{B0} will be done.

\subsection{See also}

\hyperlink{toc}{Overview}, \hyperlink{a0}{A0}, \hyperlink{b0}{B0},
\hyperlink{b00}{B00}, \hyperlink{b1}{B1}, \hyperlink{b11}{B11}.

\subsection{Examples}

By default \(B_0\) is not expressed in terms of \(A_0\).

\begin{Shaded}
\begin{Highlighting}[]
\NormalTok{B0}\OperatorTok{[}\DecValTok{0}\OperatorTok{,} \FunctionTok{s}\OperatorTok{,} \FunctionTok{s}\OperatorTok{]}
\end{Highlighting}
\end{Shaded}

\begin{dmath*}\breakingcomma
\text{B}_0(0,s,s)
\end{dmath*}

With \texttt{BReduce -> True}, transformation is done.

\begin{Shaded}
\begin{Highlighting}[]
\NormalTok{B0}\OperatorTok{[}\DecValTok{0}\OperatorTok{,} \FunctionTok{s}\OperatorTok{,} \FunctionTok{s}\OperatorTok{,}\NormalTok{ BReduce }\OtherTok{{-}\textgreater{}} \ConstantTok{True}\OperatorTok{]}
\end{Highlighting}
\end{Shaded}

\begin{dmath*}\breakingcomma
-\frac{(2-D) \;\text{A}_0(s)}{2 s}
\end{dmath*}
\end{document}
