% !TeX program = pdflatex
% !TeX root = C0.tex

\documentclass[../FeynCalcManual.tex]{subfiles}
\begin{document}
\hypertarget{c0}{%
\section{C0}\label{c0}}

\texttt{C0[\allowbreak{}p10,\ \allowbreak{}p12,\ \allowbreak{}p20,\ \allowbreak{}m1^2,\ \allowbreak{}m2^2,\ \allowbreak{}m3^2]}
is the scalar Passarino-Veltman \(C_0\) function. The convention for the
arguments is that if the denominator of the integrand has the form
\(([q^2-m1^2] [(q+p1)^2-m2^2] [(q+p2)^2-m3^2])\), the first three
arguments of C0 are the scalar products \(p10 = p1^2\),
\(p12 = (p1-p2).(p1-p2)\), \(p20 = p2^2\).

\subsection{See also}

\hyperlink{toc}{Overview}, \hyperlink{b0}{B0}, \hyperlink{d0}{D0},
\hyperlink{pave}{PaVe}, \hyperlink{paveorder}{PaVeOrder}.

\subsection{Examples}

\begin{Shaded}
\begin{Highlighting}[]
\NormalTok{C0}\OperatorTok{[}\FunctionTok{a}\OperatorTok{,} \FunctionTok{b}\OperatorTok{,} \FunctionTok{c}\OperatorTok{,}\NormalTok{ m12}\OperatorTok{,}\NormalTok{ m22}\OperatorTok{,}\NormalTok{ m32}\OperatorTok{]}
\end{Highlighting}
\end{Shaded}

\begin{dmath*}\breakingcomma
\text{C}_0(a,b,c,\text{m12},\text{m22},\text{m32})
\end{dmath*}

\begin{Shaded}
\begin{Highlighting}[]
\NormalTok{C0}\OperatorTok{[}\FunctionTok{b}\OperatorTok{,} \FunctionTok{a}\OperatorTok{,} \FunctionTok{c}\OperatorTok{,}\NormalTok{ m32}\OperatorTok{,}\NormalTok{ m22}\OperatorTok{,}\NormalTok{ m12}\OperatorTok{]} \SpecialCharTok{//}\NormalTok{ PaVeOrder}
\end{Highlighting}
\end{Shaded}

\begin{dmath*}\breakingcomma
\text{C}_0(a,b,c,\text{m12},\text{m22},\text{m32})
\end{dmath*}

\begin{Shaded}
\begin{Highlighting}[]
\NormalTok{PaVeOrder}\OperatorTok{[}\NormalTok{C0}\OperatorTok{[}\FunctionTok{b}\OperatorTok{,} \FunctionTok{a}\OperatorTok{,} \FunctionTok{c}\OperatorTok{,}\NormalTok{ m32}\OperatorTok{,}\NormalTok{ m22}\OperatorTok{,}\NormalTok{ m12}\OperatorTok{],}\NormalTok{ PaVeOrderList }\OtherTok{{-}\textgreater{}} \OperatorTok{\{}\FunctionTok{c}\OperatorTok{,} \FunctionTok{a}\OperatorTok{\}]}
\end{Highlighting}
\end{Shaded}

\begin{dmath*}\breakingcomma
\text{C}_0(c,a,b,\text{m32},\text{m12},\text{m22})
\end{dmath*}
\end{document}
