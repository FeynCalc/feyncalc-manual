% !TeX program = pdflatex
% !TeX root = CGA.tex

\documentclass[../FeynCalcManual.tex]{subfiles}
\begin{document}
\hypertarget{cga}{
\section{CGA}\label{cga}\index{CGA}}

\texttt{CGA[\allowbreak{}i]} can be used as input for \(\gamma^i\) in 4
dimensions, where \texttt{i} is a Cartesian index, and is transformed
into \texttt{DiracGamma[\allowbreak{}CartesianIndex[\allowbreak{}i]]} by
\texttt{FeynCalcInternal}.

\subsection{See also}

\hyperlink{toc}{Overview}, \hyperlink{ga}{GA},
\hyperlink{diracgamma}{DiracGamma}.

\subsection{Examples}

\begin{Shaded}
\begin{Highlighting}[]
\NormalTok{CGA}\OperatorTok{[}\FunctionTok{i}\OperatorTok{]}
\end{Highlighting}
\end{Shaded}

\begin{dmath*}\breakingcomma
\overline{\gamma }^i
\end{dmath*}

\begin{Shaded}
\begin{Highlighting}[]
\NormalTok{CGA}\OperatorTok{[}\FunctionTok{i}\OperatorTok{,} \FunctionTok{j}\OperatorTok{]} \SpecialCharTok{{-}}\NormalTok{ CGA}\OperatorTok{[}\FunctionTok{j}\OperatorTok{,} \FunctionTok{i}\OperatorTok{]}
\end{Highlighting}
\end{Shaded}

\begin{dmath*}\breakingcomma
\overline{\gamma }^i.\overline{\gamma }^j-\overline{\gamma }^j.\overline{\gamma }^i
\end{dmath*}

\begin{Shaded}
\begin{Highlighting}[]
\FunctionTok{StandardForm}\OperatorTok{[}\NormalTok{FCI}\OperatorTok{[}\NormalTok{CGA}\OperatorTok{[}\FunctionTok{i}\OperatorTok{]]]}

\CommentTok{(*DiracGamma[CartesianIndex[i]]*)}
\end{Highlighting}
\end{Shaded}

\begin{Shaded}
\begin{Highlighting}[]
\NormalTok{CGA}\OperatorTok{[}\FunctionTok{i}\OperatorTok{,} \FunctionTok{j}\OperatorTok{,} \FunctionTok{k}\OperatorTok{,} \FunctionTok{l}\OperatorTok{]}
\end{Highlighting}
\end{Shaded}

\begin{dmath*}\breakingcomma
\overline{\gamma }^i.\overline{\gamma }^j.\overline{\gamma }^k.\overline{\gamma }^l
\end{dmath*}

\begin{Shaded}
\begin{Highlighting}[]
\FunctionTok{StandardForm}\OperatorTok{[}\NormalTok{CGA}\OperatorTok{[}\FunctionTok{i}\OperatorTok{,} \FunctionTok{j}\OperatorTok{,} \FunctionTok{k}\OperatorTok{,} \FunctionTok{l}\OperatorTok{]]}

\CommentTok{(*CGA[i] . CGA[j] . CGA[k] . CGA[l]*)}
\end{Highlighting}
\end{Shaded}

\begin{Shaded}
\begin{Highlighting}[]
\NormalTok{DiracSimplify}\OperatorTok{[}\NormalTok{DiracTrace}\OperatorTok{[}\NormalTok{CGA}\OperatorTok{[}\FunctionTok{i}\OperatorTok{,} \FunctionTok{j}\OperatorTok{,} \FunctionTok{k}\OperatorTok{,} \FunctionTok{l}\OperatorTok{]]]}
\end{Highlighting}
\end{Shaded}

\begin{dmath*}\breakingcomma
4 \bar{\delta }^{il} \bar{\delta }^{jk}-4 \bar{\delta }^{ik} \bar{\delta }^{jl}+4 \bar{\delta }^{ij} \bar{\delta }^{kl}
\end{dmath*}

\begin{Shaded}
\begin{Highlighting}[]
\NormalTok{CGA}\OperatorTok{[}\FunctionTok{i}\OperatorTok{]}\NormalTok{ . (CGS}\OperatorTok{[}\FunctionTok{p}\OperatorTok{]} \SpecialCharTok{+} \FunctionTok{m}\NormalTok{) . CGA}\OperatorTok{[}\FunctionTok{j}\OperatorTok{]}
\end{Highlighting}
\end{Shaded}

\begin{dmath*}\breakingcomma
\overline{\gamma }^i.\left(\overline{\gamma }\cdot \overline{p}+m\right).\overline{\gamma }^j
\end{dmath*}
\end{document}
