% !TeX program = pdflatex
% !TeX root = CGAE.tex

\documentclass[../FeynCalcManual.tex]{subfiles}
\begin{document}
\hypertarget{cgae}{%
\section{CGAE}\label{cgae}}

\texttt{CGAE[\allowbreak{}i]} can be used as input for \(\gamma ^i\) in
\(D-4\) dimensions, where \texttt{i} is a Cartesian index, and is
transformed into
\texttt{DiracGamma[\allowbreak{}CartesianIndex[\allowbreak{}i,\ \allowbreak{}D-4],\ \allowbreak{}D-4]}
by \texttt{FeynCalcInternal}.

\subsection{See also}

\hyperlink{toc}{Overview}, \hyperlink{gae}{GAE},
\hyperlink{diracgamma}{DiracGamma}.

\subsection{Examples}

\begin{Shaded}
\begin{Highlighting}[]
\NormalTok{CGAE}\OperatorTok{[}\FunctionTok{i}\OperatorTok{]}
\end{Highlighting}
\end{Shaded}

\begin{dmath*}\breakingcomma
\hat{\gamma }^i
\end{dmath*}

\begin{Shaded}
\begin{Highlighting}[]
\NormalTok{CGAE}\OperatorTok{[}\FunctionTok{i}\OperatorTok{,} \FunctionTok{j}\OperatorTok{]} \SpecialCharTok{{-}}\NormalTok{ CGAE}\OperatorTok{[}\FunctionTok{j}\OperatorTok{,} \FunctionTok{i}\OperatorTok{]}
\end{Highlighting}
\end{Shaded}

\begin{dmath*}\breakingcomma
\hat{\gamma }^i.\hat{\gamma }^j-\hat{\gamma }^j.\hat{\gamma }^i
\end{dmath*}

\begin{Shaded}
\begin{Highlighting}[]
\FunctionTok{StandardForm}\OperatorTok{[}\NormalTok{FCI}\OperatorTok{[}\NormalTok{CGAE}\OperatorTok{[}\FunctionTok{i}\OperatorTok{]]]}

\CommentTok{(*DiracGamma[CartesianIndex[i, {-}4 + D], {-}4 + D]*)}
\end{Highlighting}
\end{Shaded}

\begin{Shaded}
\begin{Highlighting}[]
\NormalTok{CGAE}\OperatorTok{[}\FunctionTok{i}\OperatorTok{,} \FunctionTok{j}\OperatorTok{,} \FunctionTok{k}\OperatorTok{,} \FunctionTok{l}\OperatorTok{]}
\end{Highlighting}
\end{Shaded}

\begin{dmath*}\breakingcomma
\hat{\gamma }^i.\hat{\gamma }^j.\hat{\gamma }^k.\hat{\gamma }^l
\end{dmath*}

\begin{Shaded}
\begin{Highlighting}[]
\FunctionTok{StandardForm}\OperatorTok{[}\NormalTok{CGAE}\OperatorTok{[}\FunctionTok{i}\OperatorTok{,} \FunctionTok{j}\OperatorTok{,} \FunctionTok{k}\OperatorTok{,} \FunctionTok{l}\OperatorTok{]]}

\CommentTok{(*CGAE[i] . CGAE[j] . CGAE[k] . CGAE[l]*)}
\end{Highlighting}
\end{Shaded}

\end{document}
