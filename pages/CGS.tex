% !TeX program = pdflatex
% !TeX root = CGS.tex

\documentclass[../FeynCalcManual.tex]{subfiles}
\begin{document}
\hypertarget{cgs}{
\section{CGS}\label{cgs}\index{CGS}}

\texttt{CGS[\allowbreak{}p]} is transformed into
\texttt{DiracGamma[\allowbreak{}CartesianMomentum[\allowbreak{}p]]} by
\texttt{FeynCalcInternal}.

\texttt{CGS[\allowbreak{}p,\ \allowbreak{}q,\ \allowbreak{}...]} is
equivalent to \texttt{CGS[\allowbreak{}p].CGS[\allowbreak{}q]}.

\subsection{See also}

\hyperlink{toc}{Overview}, \hyperlink{gs}{GS},
\hyperlink{diracgamma}{DiracGamma}.

\subsection{Examples}

\begin{Shaded}
\begin{Highlighting}[]
\NormalTok{CGS}\OperatorTok{[}\FunctionTok{p}\OperatorTok{]}
\end{Highlighting}
\end{Shaded}

\begin{dmath*}\breakingcomma
\overline{\gamma }\cdot \overline{p}
\end{dmath*}

\begin{Shaded}
\begin{Highlighting}[]
\NormalTok{CGS}\OperatorTok{[}\FunctionTok{p}\OperatorTok{]} \SpecialCharTok{//}\NormalTok{ FCI }\SpecialCharTok{//} \FunctionTok{StandardForm}

\CommentTok{(*DiracGamma[CartesianMomentum[p]]*)}
\end{Highlighting}
\end{Shaded}

\begin{Shaded}
\begin{Highlighting}[]
\NormalTok{CGS}\OperatorTok{[}\FunctionTok{p}\OperatorTok{,} \FunctionTok{q}\OperatorTok{,} \FunctionTok{r}\OperatorTok{,} \FunctionTok{s}\OperatorTok{]}
\end{Highlighting}
\end{Shaded}

\begin{dmath*}\breakingcomma
\left(\overline{\gamma }\cdot \overline{p}\right).\left(\overline{\gamma }\cdot \overline{q}\right).\left(\overline{\gamma }\cdot \overline{r}\right).\left(\overline{\gamma }\cdot \overline{s}\right)
\end{dmath*}

\begin{Shaded}
\begin{Highlighting}[]
\NormalTok{CGS}\OperatorTok{[}\FunctionTok{p}\OperatorTok{,} \FunctionTok{q}\OperatorTok{,} \FunctionTok{r}\OperatorTok{,} \FunctionTok{s}\OperatorTok{]} \SpecialCharTok{//} \FunctionTok{StandardForm}

\CommentTok{(*CGS[p] . CGS[q] . CGS[r] . CGS[s]*)}
\end{Highlighting}
\end{Shaded}

\begin{Shaded}
\begin{Highlighting}[]
\NormalTok{CGS}\OperatorTok{[}\FunctionTok{q}\OperatorTok{]}\NormalTok{ . (CGS}\OperatorTok{[}\FunctionTok{p}\OperatorTok{]} \SpecialCharTok{+} \FunctionTok{m}\NormalTok{) . CGS}\OperatorTok{[}\FunctionTok{q}\OperatorTok{]}
\end{Highlighting}
\end{Shaded}

\begin{dmath*}\breakingcomma
\left(\overline{\gamma }\cdot \overline{q}\right).\left(\overline{\gamma }\cdot \overline{p}+m\right).\left(\overline{\gamma }\cdot \overline{q}\right)
\end{dmath*}
\end{document}
