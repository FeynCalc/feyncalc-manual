% !TeX program = pdflatex
% !TeX root = CGSE.tex

\documentclass[../FeynCalcManual.tex]{subfiles}
\begin{document}
\hypertarget{cgse}{%
\section{CGSE}\label{cgse}}

\texttt{CGSE[\allowbreak{}p]} is transformed into
\texttt{DiracGamma[\allowbreak{}CartesianMomentum[\allowbreak{}p,\ \allowbreak{}D-4],\ \allowbreak{}D-4]}
by FeynCalcInternal.

\texttt{CGSE[\allowbreak{}p,\ \allowbreak{}q,\ \allowbreak{}...]} is
equivalent to \texttt{CGSE[\allowbreak{}p].CGSE[\allowbreak{}q]}.

\subsection{See also}

\hyperlink{toc}{Overview}, \hyperlink{gse}{GSE},
\hyperlink{diracgamma}{DiracGamma}.

\subsection{Examples}

\begin{Shaded}
\begin{Highlighting}[]
\NormalTok{CGSE}\OperatorTok{[}\FunctionTok{p}\OperatorTok{]}
\end{Highlighting}
\end{Shaded}

\begin{dmath*}\breakingcomma
\hat{\gamma }\cdot \hat{p}
\end{dmath*}

\begin{Shaded}
\begin{Highlighting}[]
\NormalTok{CGSE}\OperatorTok{[}\FunctionTok{p}\OperatorTok{]} \SpecialCharTok{//}\NormalTok{ FCI }\SpecialCharTok{//} \FunctionTok{StandardForm}

\CommentTok{(*DiracGamma[CartesianMomentum[p, {-}4 + D], {-}4 + D]*)}
\end{Highlighting}
\end{Shaded}

\begin{Shaded}
\begin{Highlighting}[]
\NormalTok{CGSE}\OperatorTok{[}\FunctionTok{p}\OperatorTok{,} \FunctionTok{q}\OperatorTok{,} \FunctionTok{r}\OperatorTok{,} \FunctionTok{s}\OperatorTok{]}
\end{Highlighting}
\end{Shaded}

\begin{dmath*}\breakingcomma
\left(\hat{\gamma }\cdot \hat{p}\right).\left(\hat{\gamma }\cdot \hat{q}\right).\left(\hat{\gamma }\cdot \hat{r}\right).\left(\hat{\gamma }\cdot \hat{s}\right)
\end{dmath*}

\begin{Shaded}
\begin{Highlighting}[]
\NormalTok{CGSE}\OperatorTok{[}\FunctionTok{p}\OperatorTok{,} \FunctionTok{q}\OperatorTok{,} \FunctionTok{r}\OperatorTok{,} \FunctionTok{s}\OperatorTok{]} \SpecialCharTok{//} \FunctionTok{StandardForm}

\CommentTok{(*CGSE[p] . CGSE[q] . CGSE[r] . CGSE[s]*)}
\end{Highlighting}
\end{Shaded}

\begin{Shaded}
\begin{Highlighting}[]
\NormalTok{CGSE}\OperatorTok{[}\FunctionTok{q}\OperatorTok{]}\NormalTok{ . (CGSE}\OperatorTok{[}\FunctionTok{p}\OperatorTok{]} \SpecialCharTok{+} \FunctionTok{m}\NormalTok{) . CGSE}\OperatorTok{[}\FunctionTok{q}\OperatorTok{]}
\end{Highlighting}
\end{Shaded}

\begin{dmath*}\breakingcomma
\left(\hat{\gamma }\cdot \hat{q}\right).\left(m+\hat{\gamma }\cdot \hat{p}\right).\left(\hat{\gamma }\cdot \hat{q}\right)
\end{dmath*}
\end{document}
