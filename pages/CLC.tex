% !TeX program = pdflatex
% !TeX root = CLC.tex

\documentclass[../FeynCalcManual.tex]{subfiles}
\begin{document}
\hypertarget{clc}{
\section{CLC}\label{clc}\index{CLC}}

\texttt{CLC[\allowbreak{}m,\ \allowbreak{}n,\ \allowbreak{}r]} evaluates
to
\texttt{Eps[\allowbreak{}CartesianIndex[\allowbreak{}m],\ \allowbreak{}CartesianIndex[\allowbreak{}n],\ \allowbreak{}CartesianIndex[\allowbreak{}r]]}
applying \texttt{FeynCalcInternal}.

\texttt{CLC[\allowbreak{}m,\ \allowbreak{}...][\allowbreak{}p,\ \allowbreak{}...]}
evaluates to
\texttt{Eps[\allowbreak{}CartesianIndex[\allowbreak{}m],\ \allowbreak{}...,\ \allowbreak{}CartesianMomentum[\allowbreak{}p],\ \allowbreak{}...]}
applying \texttt{FeynCalcInternal}.

When some indices of a Levi-Civita-tensor are contracted with 3-vectors,
FeynCalc suppresses explicit dummy indices by putting those vectors into
the corresponding index slots. For example,
\(\varepsilon^{p_1 p_2 p_3}\) (accessible via
\texttt{CLC[\allowbreak{}][\allowbreak{}p1,\ \allowbreak{}p2,\ \allowbreak{}p3]})
correspond to \(\varepsilon^{i j k} p_1^i p_2^j p_3^k\).

\subsection{See also}

\hyperlink{toc}{Overview}, \hyperlink{lc}{LC}, \hyperlink{eps}{Eps}.

\subsection{Examples}

\begin{Shaded}
\begin{Highlighting}[]
\NormalTok{CLC}\OperatorTok{[}\FunctionTok{i}\OperatorTok{,} \FunctionTok{j}\OperatorTok{,} \FunctionTok{k}\OperatorTok{]}
\end{Highlighting}
\end{Shaded}

\begin{dmath*}\breakingcomma
\bar{\epsilon }^{ijk}
\end{dmath*}

\begin{Shaded}
\begin{Highlighting}[]
\NormalTok{CLC}\OperatorTok{[}\FunctionTok{i}\OperatorTok{,} \FunctionTok{j}\OperatorTok{,} \FunctionTok{k}\OperatorTok{]} \SpecialCharTok{//}\NormalTok{ FCI }\SpecialCharTok{//} \FunctionTok{StandardForm}

\CommentTok{(*Eps[CartesianIndex[i], CartesianIndex[j], CartesianIndex[k]]*)}
\end{Highlighting}
\end{Shaded}

\begin{Shaded}
\begin{Highlighting}[]
\NormalTok{CLC}\OperatorTok{[}\FunctionTok{i}\OperatorTok{][}\FunctionTok{p}\OperatorTok{,} \FunctionTok{q}\OperatorTok{]}
\end{Highlighting}
\end{Shaded}

\begin{dmath*}\breakingcomma
\bar{\epsilon }^{i\overline{p}\overline{q}}
\end{dmath*}

\begin{Shaded}
\begin{Highlighting}[]
\NormalTok{CLC}\OperatorTok{[}\FunctionTok{i}\OperatorTok{][}\FunctionTok{p}\OperatorTok{,} \FunctionTok{q}\OperatorTok{]} \SpecialCharTok{//}\NormalTok{ FCI }\SpecialCharTok{//} \FunctionTok{StandardForm}

\CommentTok{(*Eps[CartesianIndex[i], CartesianMomentum[p], CartesianMomentum[q]]*)}
\end{Highlighting}
\end{Shaded}

\begin{Shaded}
\begin{Highlighting}[]
\NormalTok{Contract}\OperatorTok{[}\NormalTok{CLC}\OperatorTok{[}\FunctionTok{i}\OperatorTok{,} \FunctionTok{j}\OperatorTok{,} \FunctionTok{k}\OperatorTok{]}\NormalTok{ CLC}\OperatorTok{[}\FunctionTok{i}\OperatorTok{,} \FunctionTok{l}\OperatorTok{,} \FunctionTok{m}\OperatorTok{]]}
\end{Highlighting}
\end{Shaded}

\begin{dmath*}\breakingcomma
\bar{\delta }^{jl} \bar{\delta }^{km}-\bar{\delta }^{jm} \bar{\delta }^{kl}
\end{dmath*}

\begin{Shaded}
\begin{Highlighting}[]
\NormalTok{CLC}\OperatorTok{[}\FunctionTok{i}\OperatorTok{,} \FunctionTok{j}\OperatorTok{,} \FunctionTok{k}\OperatorTok{]}\NormalTok{ CV}\OperatorTok{[}\FunctionTok{Subscript}\OperatorTok{[}\FunctionTok{p}\OperatorTok{,} \DecValTok{1}\OperatorTok{],} \FunctionTok{i}\OperatorTok{]}\NormalTok{ CV}\OperatorTok{[}\FunctionTok{Subscript}\OperatorTok{[}\FunctionTok{p}\OperatorTok{,} \DecValTok{2}\OperatorTok{],} \FunctionTok{j}\OperatorTok{]}\NormalTok{ CV}\OperatorTok{[}\FunctionTok{Subscript}\OperatorTok{[}\FunctionTok{p}\OperatorTok{,} \DecValTok{3}\OperatorTok{],} \FunctionTok{k}\OperatorTok{]} 
 
\NormalTok{Contract}\OperatorTok{[}\SpecialCharTok{\%}\OperatorTok{]} 
  
 
\end{Highlighting}
\end{Shaded}

\begin{dmath*}\breakingcomma
\overline{p}_1{}^i \overline{p}_2{}^j \overline{p}_3{}^k \bar{\epsilon }^{ijk}
\end{dmath*}

\begin{dmath*}\breakingcomma
\bar{\epsilon }^{\overline{p}_1\overline{p}_2\overline{p}_3}
\end{dmath*}
\end{document}
