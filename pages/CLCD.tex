% !TeX program = pdflatex
% !TeX root = CLCD.tex

\documentclass[../FeynCalcManual.tex]{subfiles}
\begin{document}
\hypertarget{clcd}{
\section{CLCD}\label{clcd}\index{CLCD}}

\texttt{CLCD[\allowbreak{}m,\ \allowbreak{}n,\ \allowbreak{}r]}
evaluates to
\texttt{Eps[\allowbreak{}CartesianIndex[\allowbreak{}m,\ \allowbreak{}D-1],\ \allowbreak{}CartesianIndex[\allowbreak{}n,\ \allowbreak{}D-1],\ \allowbreak{}CartesianIndex[\allowbreak{}r,\ \allowbreak{}D-1]]}
applying \texttt{FeynCalcInternal}.

\texttt{CLC[\allowbreak{}m,\ \allowbreak{}...][\allowbreak{}p,\ \allowbreak{}...]}
evaluates to
\texttt{Eps[\allowbreak{}CartesianIndex[\allowbreak{}m,\ \allowbreak{}D-1],\ \allowbreak{}...,\ \allowbreak{}CartesianMomentum[\allowbreak{}p,\ \allowbreak{}D-1],\ \allowbreak{}...]}
applying \texttt{FeynCalcInternal}.

When some indices of a Levi-Civita-tensor are contracted with 3-vectors,
FeynCalc suppresses explicit dummy indices by putting those vectors into
the corresponding index slots. For example,
\(\varepsilon^{p_1 p_2 p_3}\) (accessible via
\texttt{CLCD[\allowbreak{}][\allowbreak{}p1,\ \allowbreak{}p2,\ \allowbreak{}p3]})
correspond to \(\varepsilon^{i j k} p_1^i p_2^j p_3^k\).

\subsection{See also}

\hyperlink{toc}{Overview}, \hyperlink{lcd}{LCD}, \hyperlink{eps}{Eps}.

\subsection{Examples}

\begin{Shaded}
\begin{Highlighting}[]
\NormalTok{CLCD}\OperatorTok{[}\FunctionTok{i}\OperatorTok{,} \FunctionTok{j}\OperatorTok{,} \FunctionTok{k}\OperatorTok{]}
\end{Highlighting}
\end{Shaded}

\begin{dmath*}\breakingcomma
\overset{\text{}}{\epsilon }^{ijk}
\end{dmath*}

\begin{Shaded}
\begin{Highlighting}[]
\NormalTok{CLCD}\OperatorTok{[}\FunctionTok{i}\OperatorTok{,} \FunctionTok{j}\OperatorTok{,} \FunctionTok{k}\OperatorTok{]} \SpecialCharTok{//}\NormalTok{ FCI }\SpecialCharTok{//} \FunctionTok{StandardForm}

\CommentTok{(*Eps[CartesianIndex[i, {-}1 + D], CartesianIndex[j, {-}1 + D], CartesianIndex[k, {-}1 + D]]*)}
\end{Highlighting}
\end{Shaded}

\begin{Shaded}
\begin{Highlighting}[]
\NormalTok{CLCD}\OperatorTok{[}\FunctionTok{i}\OperatorTok{,} \FunctionTok{j}\OperatorTok{][}\FunctionTok{p}\OperatorTok{]}
\end{Highlighting}
\end{Shaded}

\begin{dmath*}\breakingcomma
\overset{\text{}}{\epsilon }^{ijp}
\end{dmath*}

\begin{Shaded}
\begin{Highlighting}[]
\NormalTok{CLCD}\OperatorTok{[}\FunctionTok{i}\OperatorTok{,} \FunctionTok{j}\OperatorTok{][}\FunctionTok{p}\OperatorTok{]} \SpecialCharTok{//}\NormalTok{ FCI }\SpecialCharTok{//} \FunctionTok{StandardForm}

\CommentTok{(*Eps[CartesianIndex[i, {-}1 + D], CartesianIndex[j, {-}1 + D], CartesianMomentum[p, {-}1 + D]]*)}
\end{Highlighting}
\end{Shaded}

\begin{Shaded}
\begin{Highlighting}[]
\NormalTok{CLCD}\OperatorTok{[}\FunctionTok{i}\OperatorTok{,} \FunctionTok{j}\OperatorTok{][}\FunctionTok{p}\OperatorTok{]}\NormalTok{ CLCD}\OperatorTok{[}\FunctionTok{i}\OperatorTok{,} \FunctionTok{j}\OperatorTok{][}\FunctionTok{q}\OperatorTok{]} \SpecialCharTok{//}\NormalTok{ Contract }\SpecialCharTok{//}\NormalTok{ Factor2}
\end{Highlighting}
\end{Shaded}

\begin{dmath*}\breakingcomma
(2-D) (3-D) (p\cdot q)
\end{dmath*}

\begin{Shaded}
\begin{Highlighting}[]
\NormalTok{CLCD}\OperatorTok{[}\FunctionTok{i}\OperatorTok{,} \FunctionTok{j}\OperatorTok{,} \FunctionTok{k}\OperatorTok{]}\NormalTok{ CVD}\OperatorTok{[}\FunctionTok{Subscript}\OperatorTok{[}\FunctionTok{p}\OperatorTok{,} \DecValTok{1}\OperatorTok{],} \FunctionTok{i}\OperatorTok{]}\NormalTok{ CVD}\OperatorTok{[}\FunctionTok{Subscript}\OperatorTok{[}\FunctionTok{p}\OperatorTok{,} \DecValTok{2}\OperatorTok{],} \FunctionTok{j}\OperatorTok{]}\NormalTok{ CVD}\OperatorTok{[}\FunctionTok{Subscript}\OperatorTok{[}\FunctionTok{p}\OperatorTok{,} \DecValTok{3}\OperatorTok{],} \FunctionTok{k}\OperatorTok{]} 
 
\NormalTok{Contract}\OperatorTok{[}\SpecialCharTok{\%}\OperatorTok{]} 
  
 
\end{Highlighting}
\end{Shaded}

\begin{dmath*}\breakingcomma
p_1{}^i p_2{}^j p_3{}^k \overset{\text{}}{\epsilon }^{ijk}
\end{dmath*}

\begin{dmath*}\breakingcomma
\overset{\text{}}{\epsilon }^{p_1p_2p_3}
\end{dmath*}
\end{document}
