% !TeX program = pdflatex
% !TeX root = CSI.tex

\documentclass[../FeynCalcManual.tex]{subfiles}
\begin{document}
\hypertarget{csi}{
\section{CSI}\label{csi}\index{CSI}}

\texttt{CSI[\allowbreak{}i]} can be used as input for 3-dimensional
\(\sigma ^i\) with 3-dimensional Cartesian index \texttt{i} and is
transformed into
\texttt{PauliSigma[\allowbreak{}CartesianIndex[\allowbreak{}i]]} by
\texttt{FeynCalcInternal}.

\subsection{See also}

\hyperlink{toc}{Overview}, \hyperlink{paulisigma}{PauliSigma}.

\subsection{Examples}

\begin{Shaded}
\begin{Highlighting}[]
\NormalTok{CSI}\OperatorTok{[}\FunctionTok{i}\OperatorTok{]}
\end{Highlighting}
\end{Shaded}

\begin{dmath*}\breakingcomma
\overline{\sigma }^i
\end{dmath*}

\begin{Shaded}
\begin{Highlighting}[]
\NormalTok{CSI}\OperatorTok{[}\FunctionTok{i}\OperatorTok{,} \FunctionTok{j}\OperatorTok{]} \SpecialCharTok{{-}}\NormalTok{ CSI}\OperatorTok{[}\FunctionTok{j}\OperatorTok{,} \FunctionTok{i}\OperatorTok{]}
\end{Highlighting}
\end{Shaded}

\begin{dmath*}\breakingcomma
\overline{\sigma }^i.\overline{\sigma }^j-\overline{\sigma }^j.\overline{\sigma }^i
\end{dmath*}

\begin{Shaded}
\begin{Highlighting}[]
\FunctionTok{StandardForm}\OperatorTok{[}\NormalTok{FCI}\OperatorTok{[}\NormalTok{CSI}\OperatorTok{[}\FunctionTok{i}\OperatorTok{]]]}

\CommentTok{(*PauliSigma[CartesianIndex[i]]*)}
\end{Highlighting}
\end{Shaded}

\begin{Shaded}
\begin{Highlighting}[]
\NormalTok{CSI}\OperatorTok{[}\FunctionTok{i}\OperatorTok{,} \FunctionTok{j}\OperatorTok{,} \FunctionTok{k}\OperatorTok{,} \FunctionTok{l}\OperatorTok{]}
\end{Highlighting}
\end{Shaded}

\begin{dmath*}\breakingcomma
\overline{\sigma }^i.\overline{\sigma }^j.\overline{\sigma }^k.\overline{\sigma }^l
\end{dmath*}

\begin{Shaded}
\begin{Highlighting}[]
\FunctionTok{StandardForm}\OperatorTok{[}\NormalTok{CSI}\OperatorTok{[}\FunctionTok{i}\OperatorTok{,} \FunctionTok{j}\OperatorTok{,} \FunctionTok{k}\OperatorTok{,} \FunctionTok{l}\OperatorTok{]]}

\CommentTok{(*CSI[i] . CSI[j] . CSI[k] . CSI[l]*)}
\end{Highlighting}
\end{Shaded}

\end{document}
