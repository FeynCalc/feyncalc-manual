% !TeX program = pdflatex
% !TeX root = CSID.tex

\documentclass[../FeynCalcManual.tex]{subfiles}
\begin{document}
\hypertarget{csid}{%
\section{CSID}\label{csid}}

\texttt{CSID[\allowbreak{}i]} can be used as input for
\(D-1\)-dimensional \(\sigma^i\) with \(D-1\)-dimensional Cartesian
index \texttt{i} and is transformed into
\texttt{PauliSigma[\allowbreak{}CartesianIndex[\allowbreak{}i,\ \allowbreak{}D-1],\ \allowbreak{}D-1]}
by \texttt{FeynCalcInternal}.

\subsection{See also}

\hyperlink{toc}{Overview}, \hyperlink{paulisigma}{PauliSigma}.

\subsection{Examples}

\begin{Shaded}
\begin{Highlighting}[]
\NormalTok{CSID}\OperatorTok{[}\FunctionTok{i}\OperatorTok{]}
\end{Highlighting}
\end{Shaded}

\begin{dmath*}\breakingcomma
\sigma ^i
\end{dmath*}

\begin{Shaded}
\begin{Highlighting}[]
\NormalTok{CSID}\OperatorTok{[}\FunctionTok{i}\OperatorTok{,} \FunctionTok{j}\OperatorTok{]} \SpecialCharTok{{-}}\NormalTok{ CSID}\OperatorTok{[}\FunctionTok{j}\OperatorTok{,} \FunctionTok{i}\OperatorTok{]}
\end{Highlighting}
\end{Shaded}

\begin{dmath*}\breakingcomma
\sigma ^i.\sigma ^j-\sigma ^j.\sigma ^i
\end{dmath*}

\begin{Shaded}
\begin{Highlighting}[]
\FunctionTok{StandardForm}\OperatorTok{[}\NormalTok{FCI}\OperatorTok{[}\NormalTok{CSID}\OperatorTok{[}\FunctionTok{i}\OperatorTok{]]]}

\CommentTok{(*PauliSigma[CartesianIndex[i, {-}1 + D], {-}1 + D]*)}
\end{Highlighting}
\end{Shaded}

\begin{Shaded}
\begin{Highlighting}[]
\NormalTok{CSID}\OperatorTok{[}\FunctionTok{i}\OperatorTok{,} \FunctionTok{j}\OperatorTok{,} \FunctionTok{k}\OperatorTok{,} \FunctionTok{l}\OperatorTok{]}
\end{Highlighting}
\end{Shaded}

\begin{dmath*}\breakingcomma
\sigma ^i.\sigma ^j.\sigma ^k.\sigma ^l
\end{dmath*}

\begin{Shaded}
\begin{Highlighting}[]
\FunctionTok{StandardForm}\OperatorTok{[}\NormalTok{CSID}\OperatorTok{[}\FunctionTok{i}\OperatorTok{,} \FunctionTok{j}\OperatorTok{,} \FunctionTok{k}\OperatorTok{,} \FunctionTok{l}\OperatorTok{]]}

\CommentTok{(*CSID[i] . CSID[j] . CSID[k] . CSID[l]*)}
\end{Highlighting}
\end{Shaded}

\end{document}
