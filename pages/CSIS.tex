% !TeX program = pdflatex
% !TeX root = CSIS.tex

\documentclass[../FeynCalcManual.tex]{subfiles}
\begin{document}
\hypertarget{csis}{%
\section{CSIS}\label{csis}}

CSIS{[}p{]}can be used as input for 3-dimensional \(\sigma ^i p^i\) with
3-dimensional Cartesian vector p and is transformed into
PauliSigma{[}CartesianMomentum{[}p{]}{]} by FeynCalcInternal.

\subsection{See also}

\hyperlink{toc}{Overview}, \hyperlink{paulisigma}{PauliSigma}.

\subsection{Examples}

\begin{Shaded}
\begin{Highlighting}[]
\NormalTok{CSIS}\OperatorTok{[}\FunctionTok{p}\OperatorTok{]}
\end{Highlighting}
\end{Shaded}

\begin{dmath*}\breakingcomma
\overline{\sigma }\cdot \overline{p}
\end{dmath*}

\begin{Shaded}
\begin{Highlighting}[]
\NormalTok{CSIS}\OperatorTok{[}\FunctionTok{p}\OperatorTok{]} \SpecialCharTok{//}\NormalTok{ FCI }\SpecialCharTok{//} \FunctionTok{StandardForm}

\CommentTok{(*PauliSigma[CartesianMomentum[p]]*)}
\end{Highlighting}
\end{Shaded}

\begin{Shaded}
\begin{Highlighting}[]
\NormalTok{CSIS}\OperatorTok{[}\FunctionTok{p}\OperatorTok{,} \FunctionTok{q}\OperatorTok{,} \FunctionTok{r}\OperatorTok{,} \FunctionTok{s}\OperatorTok{]}
\end{Highlighting}
\end{Shaded}

\begin{dmath*}\breakingcomma
\left(\overline{\sigma }\cdot \overline{p}\right).\left(\overline{\sigma }\cdot \overline{q}\right).\left(\overline{\sigma }\cdot \overline{r}\right).\left(\overline{\sigma }\cdot \overline{s}\right)
\end{dmath*}

\begin{Shaded}
\begin{Highlighting}[]
\NormalTok{CSIS}\OperatorTok{[}\FunctionTok{p}\OperatorTok{,} \FunctionTok{q}\OperatorTok{,} \FunctionTok{r}\OperatorTok{,} \FunctionTok{s}\OperatorTok{]} \SpecialCharTok{//} \FunctionTok{StandardForm}

\CommentTok{(*CSIS[p] . CSIS[q] . CSIS[r] . CSIS[s]*)}
\end{Highlighting}
\end{Shaded}

\end{document}
