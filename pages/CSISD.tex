% !TeX program = pdflatex
% !TeX root = CSISD.tex

\documentclass[../FeynCalcManual.tex]{subfiles}
\begin{document}
\hypertarget{csisd}{%
\section{CSISD}\label{csisd}}

CSISD{[}p{]} can be used as input for D-1-dimensional \(\sigma ^i p^i\)
with D-1-dimensional Cartesian vector p and is transformed into
PauliSigma{[}CartesianMomentum{[}p,D-1{]},D-1{]} by FeynCalcInternal.

\subsection{See also}

\hyperlink{toc}{Overview}, \hyperlink{paulisigma}{PauliSigma}.

\subsection{Examples}

\begin{Shaded}
\begin{Highlighting}[]
\NormalTok{CSISD}\OperatorTok{[}\FunctionTok{p}\OperatorTok{]}
\end{Highlighting}
\end{Shaded}

\begin{dmath*}\breakingcomma
\sigma \cdot p
\end{dmath*}

\begin{Shaded}
\begin{Highlighting}[]
\NormalTok{CSISD}\OperatorTok{[}\FunctionTok{p}\OperatorTok{]} \SpecialCharTok{//}\NormalTok{ FCI }\SpecialCharTok{//} \FunctionTok{StandardForm}

\CommentTok{(*PauliSigma[CartesianMomentum[p, {-}1 + D], {-}1 + D]*)}
\end{Highlighting}
\end{Shaded}

\begin{Shaded}
\begin{Highlighting}[]
\NormalTok{CSISD}\OperatorTok{[}\FunctionTok{p}\OperatorTok{,} \FunctionTok{q}\OperatorTok{,} \FunctionTok{r}\OperatorTok{,} \FunctionTok{s}\OperatorTok{]}
\end{Highlighting}
\end{Shaded}

\begin{dmath*}\breakingcomma
(\sigma \cdot p).(\sigma \cdot q).(\sigma \cdot r).(\sigma \cdot s)
\end{dmath*}

\begin{Shaded}
\begin{Highlighting}[]
\NormalTok{CSISD}\OperatorTok{[}\FunctionTok{p}\OperatorTok{,} \FunctionTok{q}\OperatorTok{,} \FunctionTok{r}\OperatorTok{,} \FunctionTok{s}\OperatorTok{]} \SpecialCharTok{//} \FunctionTok{StandardForm}

\CommentTok{(*CSISD[p] . CSISD[q] . CSISD[r] . CSISD[s]*)}
\end{Highlighting}
\end{Shaded}

\end{document}
