% !TeX program = pdflatex
% !TeX root = CSISE.tex

\documentclass[../FeynCalcManual.tex]{subfiles}
\begin{document}
\hypertarget{csise}{%
\section{CSISE}\label{csise}}

CSISE{[}p{]} can be used as input for D-4-dimensional \(\sigma^i p^i\)
with \(D-4\)-dimensional Cartesian vector p and is transformed into
\texttt{PauliSigma[\allowbreak{}CartesianMomentum[\allowbreak{}p,\ \allowbreak{}D-4],\ \allowbreak{}D-4]}
by FeynCalcInternal.

\subsection{See also}

\hyperlink{toc}{Overview}, \hyperlink{paulisigma}{PauliSigma}.

\subsection{Examples}

\begin{Shaded}
\begin{Highlighting}[]
\NormalTok{CSISE}\OperatorTok{[}\FunctionTok{p}\OperatorTok{]}
\end{Highlighting}
\end{Shaded}

\begin{dmath*}\breakingcomma
\hat{\sigma }\cdot \hat{p}
\end{dmath*}

\begin{Shaded}
\begin{Highlighting}[]
\NormalTok{CSISE}\OperatorTok{[}\FunctionTok{p}\OperatorTok{]} \SpecialCharTok{//}\NormalTok{ FCI }\SpecialCharTok{//} \FunctionTok{StandardForm}

\CommentTok{(*PauliSigma[CartesianMomentum[p, {-}4 + D], {-}4 + D]*)}
\end{Highlighting}
\end{Shaded}

\begin{Shaded}
\begin{Highlighting}[]
\NormalTok{CSISE}\OperatorTok{[}\FunctionTok{p}\OperatorTok{,} \FunctionTok{q}\OperatorTok{,} \FunctionTok{r}\OperatorTok{,} \FunctionTok{s}\OperatorTok{]}
\end{Highlighting}
\end{Shaded}

\begin{dmath*}\breakingcomma
\left(\hat{\sigma }\cdot \hat{p}\right).\left(\hat{\sigma }\cdot \hat{q}\right).\left(\hat{\sigma }\cdot \hat{r}\right).\left(\hat{\sigma }\cdot \hat{s}\right)
\end{dmath*}

\begin{Shaded}
\begin{Highlighting}[]
\NormalTok{CSISE}\OperatorTok{[}\FunctionTok{p}\OperatorTok{,} \FunctionTok{q}\OperatorTok{,} \FunctionTok{r}\OperatorTok{,} \FunctionTok{s}\OperatorTok{]} \SpecialCharTok{//} \FunctionTok{StandardForm}

\CommentTok{(*CSISE[p] . CSISE[q] . CSISE[r] . CSISE[s]*)}
\end{Highlighting}
\end{Shaded}

\end{document}
