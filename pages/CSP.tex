% !TeX program = pdflatex
% !TeX root = CSP.tex

\documentclass[../FeynCalcManual.tex]{subfiles}
\begin{document}
\hypertarget{csp}{
\section{CSP}\label{csp}\index{CSP}}

\texttt{CSP[\allowbreak{}p,\ \allowbreak{}q]} is the 3-dimensional
scalar product of \texttt{p} with \texttt{q} and is transformed into
\texttt{CartesianPair[\allowbreak{}CartesianMomentum[\allowbreak{}p],\ \allowbreak{}CartesianMomentum[\allowbreak{}q]]}
by \texttt{FeynCalcInternal}.

\texttt{CSP[\allowbreak{}p]} is the same as
\texttt{CSP[\allowbreak{}p,\ \allowbreak{}p]} (\(=p^2\)).

\subsection{See also}

\hyperlink{toc}{Overview}, \hyperlink{sp}{SP},
\hyperlink{scalarproduct}{ScalarProduct},
\hyperlink{cartesianscalarproduct}{CartesianScalarProduct}.

\subsection{Examples}

\begin{Shaded}
\begin{Highlighting}[]
\NormalTok{CSP}\OperatorTok{[}\FunctionTok{p}\OperatorTok{,} \FunctionTok{q}\OperatorTok{]} \SpecialCharTok{+}\NormalTok{ CSP}\OperatorTok{[}\FunctionTok{q}\OperatorTok{]}
\end{Highlighting}
\end{Shaded}

\begin{dmath*}\breakingcomma
\overline{p}\cdot \overline{q}+\overline{q}^2
\end{dmath*}

\begin{Shaded}
\begin{Highlighting}[]
\NormalTok{CSP}\OperatorTok{[}\FunctionTok{p} \SpecialCharTok{{-}} \FunctionTok{q}\OperatorTok{,} \FunctionTok{q} \SpecialCharTok{+} \DecValTok{2} \FunctionTok{p}\OperatorTok{]}
\end{Highlighting}
\end{Shaded}

\begin{dmath*}\breakingcomma
(\overline{p}-\overline{q})\cdot (2 \overline{p}+\overline{q})
\end{dmath*}

\begin{Shaded}
\begin{Highlighting}[]
\NormalTok{Calc}\OperatorTok{[}\NormalTok{ CSP}\OperatorTok{[}\FunctionTok{p} \SpecialCharTok{{-}} \FunctionTok{q}\OperatorTok{,} \FunctionTok{q} \SpecialCharTok{+} \DecValTok{2} \FunctionTok{p}\OperatorTok{]} \OperatorTok{]}
\end{Highlighting}
\end{Shaded}

\begin{dmath*}\breakingcomma
-\overline{p}\cdot \overline{q}+2 \overline{p}^2-\overline{q}^2
\end{dmath*}

\begin{Shaded}
\begin{Highlighting}[]
\NormalTok{ExpandScalarProduct}\OperatorTok{[}\NormalTok{CSP}\OperatorTok{[}\FunctionTok{p} \SpecialCharTok{{-}} \FunctionTok{q}\OperatorTok{]]}
\end{Highlighting}
\end{Shaded}

\begin{dmath*}\breakingcomma
-2 \left(\overline{p}\cdot \overline{q}\right)+\overline{p}^2+\overline{q}^2
\end{dmath*}

\begin{Shaded}
\begin{Highlighting}[]
\NormalTok{CSP}\OperatorTok{[}\FunctionTok{a}\OperatorTok{,} \FunctionTok{b}\OperatorTok{]} \SpecialCharTok{//} \FunctionTok{StandardForm}

\CommentTok{(*CSP[a, b]*)}
\end{Highlighting}
\end{Shaded}

\begin{Shaded}
\begin{Highlighting}[]
\NormalTok{CSP}\OperatorTok{[}\FunctionTok{a}\OperatorTok{,} \FunctionTok{b}\OperatorTok{]} \SpecialCharTok{//}\NormalTok{ FCI }\SpecialCharTok{//} \FunctionTok{StandardForm}

\CommentTok{(*CartesianPair[CartesianMomentum[a], CartesianMomentum[b]]*)}
\end{Highlighting}
\end{Shaded}

\begin{Shaded}
\begin{Highlighting}[]
\NormalTok{CSP}\OperatorTok{[}\FunctionTok{a}\OperatorTok{,} \FunctionTok{b}\OperatorTok{]} \SpecialCharTok{//}\NormalTok{ FCI }\SpecialCharTok{//}\NormalTok{ FCE }\SpecialCharTok{//} \FunctionTok{StandardForm}

\CommentTok{(*CSP[a, b]*)}
\end{Highlighting}
\end{Shaded}

\end{document}
