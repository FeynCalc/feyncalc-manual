% !TeX program = pdflatex
% !TeX root = CSPD.tex

\documentclass[../FeynCalcManual.tex]{subfiles}
\begin{document}
\hypertarget{cspd}{
\section{CSPD}\label{cspd}\index{CSPD}}

\texttt{CSPD[\allowbreak{}p,\ \allowbreak{}q]} is the
\(D-1\)-dimensional scalar product of \texttt{p} with \texttt{q} and is
transformed into
\texttt{CartesianPair[\allowbreak{}CartesianMomentum[\allowbreak{}p,\ \allowbreak{}D-1],\ \allowbreak{}CartesianMomentum[\allowbreak{}q,\ \allowbreak{}D-1]]}
by \texttt{FeynCalcInternal}.

\texttt{CSPD[\allowbreak{}p]} is the same as
\texttt{CSPD[\allowbreak{}p,\ \allowbreak{}p]} (\(=p^2\)).

\subsection{See also}

\hyperlink{toc}{Overview}, \hyperlink{spd}{SPD},
\hyperlink{scalarproduct}{ScalarProduct},
\hyperlink{cartesianscalarproduct}{CartesianScalarProduct}.

\subsection{Examples}

\begin{Shaded}
\begin{Highlighting}[]
\NormalTok{CSPD}\OperatorTok{[}\FunctionTok{p}\OperatorTok{,} \FunctionTok{q}\OperatorTok{]} \SpecialCharTok{+}\NormalTok{ CSPD}\OperatorTok{[}\FunctionTok{q}\OperatorTok{]}
\end{Highlighting}
\end{Shaded}

\begin{dmath*}\breakingcomma
p\cdot q+q^2
\end{dmath*}

\begin{Shaded}
\begin{Highlighting}[]
\NormalTok{CSPD}\OperatorTok{[}\FunctionTok{p} \SpecialCharTok{{-}} \FunctionTok{q}\OperatorTok{,} \FunctionTok{q} \SpecialCharTok{+} \DecValTok{2} \FunctionTok{p}\OperatorTok{]}
\end{Highlighting}
\end{Shaded}

\begin{dmath*}\breakingcomma
(p-q)\cdot (2 p+q)
\end{dmath*}

\begin{Shaded}
\begin{Highlighting}[]
\NormalTok{Calc}\OperatorTok{[}\NormalTok{ CSPD}\OperatorTok{[}\FunctionTok{p} \SpecialCharTok{{-}} \FunctionTok{q}\OperatorTok{,} \FunctionTok{q} \SpecialCharTok{+} \DecValTok{2} \FunctionTok{p}\OperatorTok{]} \OperatorTok{]}
\end{Highlighting}
\end{Shaded}

\begin{dmath*}\breakingcomma
-p\cdot q+2 p^2-q^2
\end{dmath*}

\begin{Shaded}
\begin{Highlighting}[]
\NormalTok{ExpandScalarProduct}\OperatorTok{[}\NormalTok{CSPD}\OperatorTok{[}\FunctionTok{p} \SpecialCharTok{{-}} \FunctionTok{q}\OperatorTok{]]}
\end{Highlighting}
\end{Shaded}

\begin{dmath*}\breakingcomma
-2 (p\cdot q)+p^2+q^2
\end{dmath*}

\begin{Shaded}
\begin{Highlighting}[]
\NormalTok{CSPD}\OperatorTok{[}\FunctionTok{a}\OperatorTok{,} \FunctionTok{b}\OperatorTok{]} \SpecialCharTok{//} \FunctionTok{StandardForm}

\CommentTok{(*CSPD[a, b]*)}
\end{Highlighting}
\end{Shaded}

\begin{Shaded}
\begin{Highlighting}[]
\NormalTok{CSPD}\OperatorTok{[}\FunctionTok{a}\OperatorTok{,} \FunctionTok{b}\OperatorTok{]} \SpecialCharTok{//}\NormalTok{ FCI }\SpecialCharTok{//} \FunctionTok{StandardForm}

\CommentTok{(*CartesianPair[CartesianMomentum[a, {-}1 + D], CartesianMomentum[b, {-}1 + D]]*)}
\end{Highlighting}
\end{Shaded}

\begin{Shaded}
\begin{Highlighting}[]
\NormalTok{CSPD}\OperatorTok{[}\FunctionTok{a}\OperatorTok{,} \FunctionTok{b}\OperatorTok{]} \SpecialCharTok{//}\NormalTok{ FCI }\SpecialCharTok{//}\NormalTok{ FCE }\SpecialCharTok{//} \FunctionTok{StandardForm}

\CommentTok{(*CSPD[a, b]*)}
\end{Highlighting}
\end{Shaded}

\begin{Shaded}
\begin{Highlighting}[]
\NormalTok{FCE}\OperatorTok{[}\NormalTok{ChangeDimension}\OperatorTok{[}\NormalTok{CSP}\OperatorTok{[}\FunctionTok{p}\OperatorTok{,} \FunctionTok{q}\OperatorTok{],} \FunctionTok{D}\OperatorTok{]]} \SpecialCharTok{//} \FunctionTok{StandardForm}

\CommentTok{(*CSPD[p, q]*)}
\end{Highlighting}
\end{Shaded}

\end{document}
