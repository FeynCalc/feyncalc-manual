% !TeX program = pdflatex
% !TeX root = CSPE.tex

\documentclass[../FeynCalcManual.tex]{subfiles}
\begin{document}
\hypertarget{cspe}{
\section{CSPE}\label{cspe}\index{CSPE}}

\texttt{CSPE[\allowbreak{}p,\ \allowbreak{}q]} is the
\(D-4\)-dimensional scalar product of \texttt{p} with \texttt{q} and is
transformed into
\texttt{CartesianPair[\allowbreak{}CartesianMomentum[\allowbreak{}p,\ \allowbreak{}D-4],\ \allowbreak{}CartesianMomentum[\allowbreak{}q,\ \allowbreak{}D-4]]}
by \texttt{FeynCalcInternal}.

\texttt{CSPE[\allowbreak{}p]} is the same as
\texttt{CSPE[\allowbreak{}p,\ \allowbreak{}p]} ( \(=p^2\)).

\subsection{See also}

\hyperlink{toc}{Overview}, \hyperlink{spe}{SPE},
\hyperlink{scalarproduct}{ScalarProduct},
\hyperlink{cartesianscalarproduct}{CartesianScalarProduct}.

\subsection{Examples}

\begin{Shaded}
\begin{Highlighting}[]
\NormalTok{CSPE}\OperatorTok{[}\FunctionTok{p}\OperatorTok{,} \FunctionTok{q}\OperatorTok{]} \SpecialCharTok{+}\NormalTok{ CSPE}\OperatorTok{[}\FunctionTok{q}\OperatorTok{]}
\end{Highlighting}
\end{Shaded}

\begin{dmath*}\breakingcomma
\hat{p}\cdot \hat{q}+\hat{q}^2
\end{dmath*}

\begin{Shaded}
\begin{Highlighting}[]
\NormalTok{CSPE}\OperatorTok{[}\FunctionTok{p} \SpecialCharTok{{-}} \FunctionTok{q}\OperatorTok{,} \FunctionTok{q} \SpecialCharTok{+} \DecValTok{2} \FunctionTok{p}\OperatorTok{]}
\end{Highlighting}
\end{Shaded}

\begin{dmath*}\breakingcomma
(\hat{p}-\hat{q})\cdot (2 \hat{p}+\hat{q})
\end{dmath*}

\begin{Shaded}
\begin{Highlighting}[]
\NormalTok{Calc}\OperatorTok{[}\NormalTok{ CSPE}\OperatorTok{[}\FunctionTok{p} \SpecialCharTok{{-}} \FunctionTok{q}\OperatorTok{,} \FunctionTok{q} \SpecialCharTok{+} \DecValTok{2} \FunctionTok{p}\OperatorTok{]} \OperatorTok{]}
\end{Highlighting}
\end{Shaded}

\begin{dmath*}\breakingcomma
-\hat{p}\cdot \hat{q}+2 \hat{p}^2-\hat{q}^2
\end{dmath*}

\begin{Shaded}
\begin{Highlighting}[]
\NormalTok{ExpandScalarProduct}\OperatorTok{[}\NormalTok{CSPE}\OperatorTok{[}\FunctionTok{p} \SpecialCharTok{{-}} \FunctionTok{q}\OperatorTok{]]}
\end{Highlighting}
\end{Shaded}

\begin{dmath*}\breakingcomma
-2 \left(\hat{p}\cdot \hat{q}\right)+\hat{p}^2+\hat{q}^2
\end{dmath*}

\begin{Shaded}
\begin{Highlighting}[]
\NormalTok{CSPE}\OperatorTok{[}\FunctionTok{a}\OperatorTok{,} \FunctionTok{b}\OperatorTok{]} \SpecialCharTok{//} \FunctionTok{StandardForm}

\CommentTok{(*CSPE[a, b]*)}
\end{Highlighting}
\end{Shaded}

\begin{Shaded}
\begin{Highlighting}[]
\NormalTok{CSPE}\OperatorTok{[}\FunctionTok{a}\OperatorTok{,} \FunctionTok{b}\OperatorTok{]} \SpecialCharTok{//}\NormalTok{ FCI }\SpecialCharTok{//} \FunctionTok{StandardForm}

\CommentTok{(*CartesianPair[CartesianMomentum[a, {-}4 + D], CartesianMomentum[b, {-}4 + D]]*)}
\end{Highlighting}
\end{Shaded}

\begin{Shaded}
\begin{Highlighting}[]
\NormalTok{CSPE}\OperatorTok{[}\FunctionTok{a}\OperatorTok{,} \FunctionTok{b}\OperatorTok{]} \SpecialCharTok{//}\NormalTok{ FCI }\SpecialCharTok{//}\NormalTok{ FCE }\SpecialCharTok{//} \FunctionTok{StandardForm}

\CommentTok{(*CSPE[a, b]*)}
\end{Highlighting}
\end{Shaded}

\begin{Shaded}
\begin{Highlighting}[]
\NormalTok{FCE}\OperatorTok{[}\NormalTok{ChangeDimension}\OperatorTok{[}\NormalTok{CSP}\OperatorTok{[}\FunctionTok{p}\OperatorTok{,} \FunctionTok{q}\OperatorTok{],} \FunctionTok{D} \SpecialCharTok{{-}} \DecValTok{4}\OperatorTok{]]} \SpecialCharTok{//} \FunctionTok{StandardForm}

\CommentTok{(*CSPE[p, q]*)}
\end{Highlighting}
\end{Shaded}

\end{document}
