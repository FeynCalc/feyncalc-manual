% !TeX program = pdflatex
% !TeX root = CTdec.tex

\documentclass[../FeynCalcManual.tex]{subfiles}
\begin{document}
\hypertarget{ctdec}{
\section{CTdec}\label{ctdec}\index{CTdec}}

\texttt{CTdec[\allowbreak{}\{\allowbreak{}\{\allowbreak{}qi,\ \allowbreak{}a\},\ \allowbreak{}\{\allowbreak{}qj,\ \allowbreak{}b\},\ \allowbreak{}...\},\ \allowbreak{}\{\allowbreak{}p1,\ \allowbreak{}p2,\ \allowbreak{}...\}]}
or
\texttt{CTdec[\allowbreak{}exp,\ \allowbreak{}\{\allowbreak{}\{\allowbreak{}qi,\ \allowbreak{}a\},\ \allowbreak{}\{\allowbreak{}qj,\ \allowbreak{}b\},\ \allowbreak{}...\},\ \allowbreak{}\{\allowbreak{}p1,\ \allowbreak{}p2,\ \allowbreak{}...\}]}
calculates the tensorial decomposition formulas for Cartesian integrals.
The more common ones are saved in TIDL.

\subsection{See also}

\hyperlink{toc}{Overview}, \hyperlink{tdec}{Tdec},
\hyperlink{tidl}{TIDL}, \hyperlink{tid}{TID}.

\subsection{Examples}

Check that
\(\int d^{D-1} q \, f(p,q) q^i =  \frac{p^i}{p^2} \int d^{D-1} q \, f(p,q) p \cdot q\)

\begin{Shaded}
\begin{Highlighting}[]
\NormalTok{CTdec}\OperatorTok{[\{\{}\FunctionTok{q}\OperatorTok{,} \FunctionTok{i}\OperatorTok{\}\},} \OperatorTok{\{}\FunctionTok{p}\OperatorTok{\}]}
\end{Highlighting}
\end{Shaded}

\begin{dmath*}\breakingcomma
\left\{\left\{\text{X1}\to p\cdot q,\text{X2}\to p^2\right\},\frac{\text{X1} p^i}{\text{X2}}\right\}
\end{dmath*}

\begin{Shaded}
\begin{Highlighting}[]
\SpecialCharTok{\%}\OperatorTok{[[}\DecValTok{2}\OperatorTok{]]} \OtherTok{/.} \SpecialCharTok{\%}\OperatorTok{[[}\DecValTok{1}\OperatorTok{]]}
\end{Highlighting}
\end{Shaded}

\begin{dmath*}\breakingcomma
\frac{p^i (p\cdot q)}{p^2}
\end{dmath*}

\begin{Shaded}
\begin{Highlighting}[]
\NormalTok{CTdec}\OperatorTok{[\{\{}\FunctionTok{q}\OperatorTok{,} \FunctionTok{i}\OperatorTok{\}\},} \OperatorTok{\{}\FunctionTok{p}\OperatorTok{\},} \FunctionTok{List} \OtherTok{{-}\textgreater{}} \ConstantTok{False}\OperatorTok{]}
\end{Highlighting}
\end{Shaded}

\begin{dmath*}\breakingcomma
\frac{p^i (p\cdot q)}{p^2}
\end{dmath*}

This calculates integral transformation for any
\(\int d^{D-1} q_1 d^{D-1} q_2 d^{D-1} q_3 f (p, q_1, q_2, q_3) q_1^i q_2^j q_3^k\).

\begin{Shaded}
\begin{Highlighting}[]
\NormalTok{CTdec}\OperatorTok{[\{\{}\FunctionTok{Subscript}\OperatorTok{[}\FunctionTok{q}\OperatorTok{,} \DecValTok{1}\OperatorTok{],} \FunctionTok{i}\OperatorTok{\},} \OperatorTok{\{}\FunctionTok{Subscript}\OperatorTok{[}\FunctionTok{q}\OperatorTok{,} \DecValTok{2}\OperatorTok{],} \FunctionTok{j}\OperatorTok{\},} \OperatorTok{\{}\FunctionTok{Subscript}\OperatorTok{[}\FunctionTok{q}\OperatorTok{,} \DecValTok{3}\OperatorTok{],} \FunctionTok{k}\OperatorTok{\}\},} \OperatorTok{\{}\FunctionTok{p}\OperatorTok{\},} \FunctionTok{List} \OtherTok{{-}\textgreater{}} \ConstantTok{False}\OperatorTok{]}
\end{Highlighting}
\end{Shaded}

\begin{dmath*}\breakingcomma
\frac{p^k \delta ^{ij} \left(p\cdot q_3\right) \left(\left(p\cdot q_1\right) \left(p\cdot q_2\right)-p^2 \left(q_1\cdot q_2\right)\right)}{(2-D) p^4}+\frac{p^j \delta ^{ik} \left(p\cdot q_2\right) \left(\left(p\cdot q_1\right) \left(p\cdot q_3\right)-p^2 \left(q_1\cdot q_3\right)\right)}{(2-D) p^4}+\frac{p^i \delta ^{jk} \left(p\cdot q_1\right) \left(\left(p\cdot q_2\right) \left(p\cdot q_3\right)-p^2 \left(q_2\cdot q_3\right)\right)}{(2-D) p^4}-\frac{p^i p^j p^k \left((D-1) \left(p\cdot q_1\right) \left(p\cdot q_2\right) \left(p\cdot q_3\right)+2 \left(p\cdot q_1\right) \left(p\cdot q_2\right) \left(p\cdot q_3\right)-p^2 \left(q_1\cdot q_2\right) \left(p\cdot q_3\right)-p^2 \left(q_1\cdot q_3\right) \left(p\cdot q_2\right)-p^2 \left(q_2\cdot q_3\right) \left(p\cdot q_1\right)\right)}{(2-D) p^6}
\end{dmath*}

\begin{Shaded}
\begin{Highlighting}[]
\NormalTok{Contract}\OperatorTok{[}\SpecialCharTok{\%}\NormalTok{ CVD}\OperatorTok{[}\FunctionTok{p}\OperatorTok{,} \FunctionTok{i}\OperatorTok{]}\NormalTok{ CVD}\OperatorTok{[}\FunctionTok{p}\OperatorTok{,} \FunctionTok{j}\OperatorTok{]}\NormalTok{ CVD}\OperatorTok{[}\FunctionTok{p}\OperatorTok{,} \FunctionTok{k}\OperatorTok{]]} \SpecialCharTok{//} \FunctionTok{Factor}
\end{Highlighting}
\end{Shaded}

\begin{dmath*}\breakingcomma
\left(p\cdot q_1\right) \left(p\cdot q_2\right) \left(p\cdot q_3\right)
\end{dmath*}
\end{document}
