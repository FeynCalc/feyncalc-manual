% !TeX program = pdflatex
% !TeX root = CV.tex

\documentclass[../FeynCalcManual.tex]{subfiles}
\begin{document}
\hypertarget{cv}{%
\section{CV}\label{cv}}

\texttt{CV[\allowbreak{}p,\ \allowbreak{}i]} is a 3-dimensional
Cartesian vector and is transformed into
\texttt{CartesianPair[\allowbreak{}CartesianMomentum[\allowbreak{}p],\ \allowbreak{}CartesianIndex[\allowbreak{}i]]}
by \texttt{FeynCalcInternal}.

\subsection{See also}

\hyperlink{toc}{Overview}, \hyperlink{fv}{FV}, \hyperlink{pair}{Pair},
\hyperlink{cartesianpair}{CartesianPair}.

\subsection{Examples}

\begin{Shaded}
\begin{Highlighting}[]
\NormalTok{CV}\OperatorTok{[}\FunctionTok{p}\OperatorTok{,} \FunctionTok{i}\OperatorTok{]}
\end{Highlighting}
\end{Shaded}

\begin{dmath*}\breakingcomma
\overline{p}^i
\end{dmath*}

\begin{Shaded}
\begin{Highlighting}[]
\NormalTok{CV}\OperatorTok{[}\FunctionTok{p} \SpecialCharTok{{-}} \FunctionTok{q}\OperatorTok{,} \FunctionTok{i}\OperatorTok{]}
\end{Highlighting}
\end{Shaded}

\begin{dmath*}\breakingcomma
\left(\overline{p}-\overline{q}\right)^i
\end{dmath*}

\begin{Shaded}
\begin{Highlighting}[]
\NormalTok{FCI}\OperatorTok{[}\NormalTok{CV}\OperatorTok{[}\FunctionTok{p}\OperatorTok{,} \FunctionTok{i}\OperatorTok{]]} \SpecialCharTok{//} \FunctionTok{StandardForm}

\CommentTok{(*CartesianPair[CartesianIndex[i], CartesianMomentum[p]]*)}
\end{Highlighting}
\end{Shaded}

\texttt{ExpandScalarProduct} is used to expand momenta in \texttt{CV}

\begin{Shaded}
\begin{Highlighting}[]
\NormalTok{ExpandScalarProduct}\OperatorTok{[}\NormalTok{CV}\OperatorTok{[}\FunctionTok{p} \SpecialCharTok{{-}} \FunctionTok{q}\OperatorTok{,} \FunctionTok{i}\OperatorTok{]]}
\end{Highlighting}
\end{Shaded}

\begin{dmath*}\breakingcomma
\overline{p}^i-\overline{q}^i
\end{dmath*}
\end{document}
