% !TeX program = pdflatex
% !TeX root = CVD.tex

\documentclass[../FeynCalcManual.tex]{subfiles}
\begin{document}
\hypertarget{cvd}{
\section{CVD}\label{cvd}\index{CVD}}

\texttt{CVD[\allowbreak{}p,\ \allowbreak{}i]} is a \(D-1\)-dimensional
Cartesian vector and is transformed into
\texttt{CartesianPair[\allowbreak{}CartesianMomentum[\allowbreak{}p,\ \allowbreak{}D],\ \allowbreak{}CartesianIndex[\allowbreak{}i,\ \allowbreak{}D]]}
by \texttt{FeynCalcInternal}.

\subsection{See also}

\hyperlink{toc}{Overview}, \hyperlink{fvd}{FVD}, \hyperlink{pair}{Pair},
\hyperlink{cartesianpair}{CartesianPair}.

\subsection{Examples}

\begin{Shaded}
\begin{Highlighting}[]
\NormalTok{CVD}\OperatorTok{[}\FunctionTok{p}\OperatorTok{,} \FunctionTok{i}\OperatorTok{]}
\end{Highlighting}
\end{Shaded}

\begin{dmath*}\breakingcomma
p^i
\end{dmath*}

\begin{Shaded}
\begin{Highlighting}[]
\NormalTok{CVD}\OperatorTok{[}\FunctionTok{p} \SpecialCharTok{{-}} \FunctionTok{q}\OperatorTok{,} \FunctionTok{i}\OperatorTok{]}
\end{Highlighting}
\end{Shaded}

\begin{dmath*}\breakingcomma
(p-q)^i
\end{dmath*}

\begin{Shaded}
\begin{Highlighting}[]
\NormalTok{FCI}\OperatorTok{[}\NormalTok{CVD}\OperatorTok{[}\FunctionTok{p}\OperatorTok{,} \FunctionTok{i}\OperatorTok{]]} \SpecialCharTok{//} \FunctionTok{StandardForm}

\CommentTok{(*CartesianPair[CartesianIndex[i, {-}1 + D], CartesianMomentum[p, {-}1 + D]]*)}
\end{Highlighting}
\end{Shaded}

\texttt{ExpandScalarProduct} is used to expand momenta in \texttt{CVD}

\begin{Shaded}
\begin{Highlighting}[]
\NormalTok{ExpandScalarProduct}\OperatorTok{[}\NormalTok{CVD}\OperatorTok{[}\FunctionTok{p} \SpecialCharTok{{-}} \FunctionTok{q}\OperatorTok{,} \FunctionTok{i}\OperatorTok{]]}
\end{Highlighting}
\end{Shaded}

\begin{dmath*}\breakingcomma
p^i-q^i
\end{dmath*}
\end{document}
