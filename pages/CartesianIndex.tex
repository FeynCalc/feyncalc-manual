% !TeX program = pdflatex
% !TeX root = CartesianIndex.tex

\documentclass[../FeynCalcManual.tex]{subfiles}
\begin{document}
\hypertarget{cartesianindex}{%
\section{CartesianIndex}\label{cartesianindex}}

\texttt{CartesianIndex} is the head of Cartesian indices. The internal
representation of a \(3\)-dimensional \texttt{i} is
\texttt{CartesianIndex[\allowbreak{}i]}.

For other than three dimensions:
\texttt{CartesianIndex[\allowbreak{}i,\ \allowbreak{}Dimension]}.

\texttt{CartesianIndex[\allowbreak{}i,\ \allowbreak{}3]} simplifies to
\texttt{CartesianIndex[\allowbreak{}i]}. The first argument cannot be an
integer.

\subsection{See also}

\hyperlink{toc}{Overview}, \hyperlink{lorentzindex}{LorentzIndex},
\hyperlink{explicitlorentzindex}{ExplicitLorentzIndex}.

\subsection{Examples}

This denotes a 3-dimensional Cartesian index.

\begin{Shaded}
\begin{Highlighting}[]
\NormalTok{CartesianIndex}\OperatorTok{[}\FunctionTok{i}\OperatorTok{]}
\end{Highlighting}
\end{Shaded}

\begin{dmath*}\breakingcomma
i
\end{dmath*}

An optional second argument can be given for a dimension different from
3.

\begin{Shaded}
\begin{Highlighting}[]
\NormalTok{CartesianIndex}\OperatorTok{[}\FunctionTok{i}\OperatorTok{,} \FunctionTok{D} \SpecialCharTok{{-}} \DecValTok{1}\OperatorTok{]}
\end{Highlighting}
\end{Shaded}

\begin{dmath*}\breakingcomma
i
\end{dmath*}

\begin{Shaded}
\begin{Highlighting}[]
\NormalTok{CartesianIndex}\OperatorTok{[}\FunctionTok{i}\OperatorTok{,} \FunctionTok{D} \SpecialCharTok{{-}} \DecValTok{4}\OperatorTok{]}
\end{Highlighting}
\end{Shaded}

\begin{dmath*}\breakingcomma
i
\end{dmath*}
\end{document}
