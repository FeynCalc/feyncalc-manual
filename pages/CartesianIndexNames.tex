% !TeX program = pdflatex
% !TeX root = CartesianIndexNames.tex

\documentclass[../FeynCalcManual.tex]{subfiles}
\begin{document}
\hypertarget{cartesianindexnames}{%
\section{CartesianIndexNames}\label{cartesianindexnames}}

\texttt{CartesianIndexNames} is an option for \texttt{FCFAConvert},
\texttt{FCCanonicalizeDummyIndices} and other functions. It renames the
generic dummy Cartesian indices to the indices in the supplied list.

\subsection{See also}

\hyperlink{toc}{Overview}, \hyperlink{fcfaconvert}{FCFAConvert},
\hyperlink{fccanonicalizedummyindices}{FCCanonicalizeDummyIndices},
\hyperlink{cartesianindexnames}{CartesianIndexNames}.

\subsection{Examples}

\begin{Shaded}
\begin{Highlighting}[]
\NormalTok{CLC}\OperatorTok{[}\NormalTok{i1}\OperatorTok{,}\NormalTok{ i2}\OperatorTok{,}\NormalTok{ i3}\OperatorTok{]}\NormalTok{ CGA}\OperatorTok{[}\NormalTok{i1}\OperatorTok{,}\NormalTok{ i2}\OperatorTok{,}\NormalTok{ i3}\OperatorTok{]} 
 
\NormalTok{FCCanonicalizeDummyIndices}\OperatorTok{[}\SpecialCharTok{\%}\OperatorTok{]}
\end{Highlighting}
\end{Shaded}

\begin{dmath*}\breakingcomma
\overline{\gamma }^{\text{i1}}.\overline{\gamma }^{\text{i2}}.\overline{\gamma }^{\text{i3}} \bar{\epsilon }^{\text{i1}\;\text{i2}\;\text{i3}}
\end{dmath*}

\begin{dmath*}\breakingcomma
\overline{\gamma }^{\text{FCGV}(\text{ci201})}.\overline{\gamma }^{\text{FCGV}(\text{ci202})}.\overline{\gamma }^{\text{FCGV}(\text{ci203})} \bar{\epsilon }^{\text{FCGV}(\text{ci201})\text{FCGV}(\text{ci202})\text{FCGV}(\text{ci203})}
\end{dmath*}

\begin{Shaded}
\begin{Highlighting}[]
\NormalTok{CLC}\OperatorTok{[}\NormalTok{i1}\OperatorTok{,}\NormalTok{ i2}\OperatorTok{,}\NormalTok{ i3}\OperatorTok{]}\NormalTok{ CGA}\OperatorTok{[}\NormalTok{i1}\OperatorTok{,}\NormalTok{ i2}\OperatorTok{,}\NormalTok{ i3}\OperatorTok{]} 
 
\NormalTok{FCCanonicalizeDummyIndices}\OperatorTok{[}\SpecialCharTok{\%}\OperatorTok{,}\NormalTok{ CartesianIndexNames }\OtherTok{{-}\textgreater{}} \OperatorTok{\{}\FunctionTok{i}\OperatorTok{,} \FunctionTok{j}\OperatorTok{,} \FunctionTok{k}\OperatorTok{\}]}
\end{Highlighting}
\end{Shaded}

\begin{dmath*}\breakingcomma
\overline{\gamma }^{\text{i1}}.\overline{\gamma }^{\text{i2}}.\overline{\gamma }^{\text{i3}} \bar{\epsilon }^{\text{i1}\;\text{i2}\;\text{i3}}
\end{dmath*}

\begin{dmath*}\breakingcomma
\overline{\gamma }^i.\overline{\gamma }^j.\overline{\gamma }^k \bar{\epsilon }^{ijk}
\end{dmath*}
\end{document}
