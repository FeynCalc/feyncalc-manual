% !TeX program = pdflatex
% !TeX root = CartesianMomentum.tex

\documentclass[../FeynCalcManual.tex]{subfiles}
\begin{document}
\hypertarget{cartesianmomentum}{%
\section{CartesianMomentum}\label{cartesianmomentum}}

\texttt{CartesianMomentum[\allowbreak{}p]} is the head of a 3-momentum
\texttt{p}. The internal representation of a \(3\)-dimensional
\texttt{p} is \texttt{CartesianMomentum[\allowbreak{}p]}. For other than
three dimensions:
\texttt{CartesianMomentum[\allowbreak{}p,\ \allowbreak{}Dimension]}.
\texttt{CartesianMomentum[\allowbreak{}p,\ \allowbreak{}3]} simplifies
to \texttt{CartesianMomentum[\allowbreak{}p]}.

\subsection{See also}

\hyperlink{toc}{Overview}, \hyperlink{momentum}{Momentum},
\hyperlink{temporalmomentum}{TemporalMomentum}.

\subsection{Examples}

This is a 3-dimensional momentum

\begin{Shaded}
\begin{Highlighting}[]
\NormalTok{CartesianMomentum}\OperatorTok{[}\FunctionTok{p}\OperatorTok{]}
\end{Highlighting}
\end{Shaded}

\begin{dmath*}\breakingcomma
\overline{p}
\end{dmath*}

As an optional second argument the dimension must be specified if it is
different from 3

\begin{Shaded}
\begin{Highlighting}[]
\NormalTok{CartesianMomentum}\OperatorTok{[}\FunctionTok{p}\OperatorTok{,} \FunctionTok{D} \SpecialCharTok{{-}} \DecValTok{1}\OperatorTok{]}
\end{Highlighting}
\end{Shaded}

\begin{dmath*}\breakingcomma
p
\end{dmath*}

The dimension index is suppressed in the output.

\begin{Shaded}
\begin{Highlighting}[]
\NormalTok{CartesianMomentum}\OperatorTok{[}\FunctionTok{p}\OperatorTok{,} \FunctionTok{d} \SpecialCharTok{{-}} \DecValTok{1}\OperatorTok{]}
\end{Highlighting}
\end{Shaded}

\begin{dmath*}\breakingcomma
p
\end{dmath*}

\begin{Shaded}
\begin{Highlighting}[]
\NormalTok{a1 }\ExtensionTok{=}\NormalTok{ CartesianMomentum}\OperatorTok{[}\SpecialCharTok{{-}}\FunctionTok{q}\OperatorTok{]}
\end{Highlighting}
\end{Shaded}

\begin{dmath*}\breakingcomma
-\overline{q}
\end{dmath*}

\begin{Shaded}
\begin{Highlighting}[]
\NormalTok{a1 }\SpecialCharTok{//} \FunctionTok{StandardForm}

\CommentTok{(*{-}CartesianMomentum[q]*)}
\end{Highlighting}
\end{Shaded}

\begin{Shaded}
\begin{Highlighting}[]
\NormalTok{a2 }\ExtensionTok{=}\NormalTok{ CartesianMomentum}\OperatorTok{[}\FunctionTok{p} \SpecialCharTok{{-}} \FunctionTok{q}\OperatorTok{]} \SpecialCharTok{+}\NormalTok{ CartesianMomentum}\OperatorTok{[}\DecValTok{2} \FunctionTok{q}\OperatorTok{]}
\end{Highlighting}
\end{Shaded}

\begin{dmath*}\breakingcomma
\left(\overline{p}-\overline{q}\right)+2 \overline{q}
\end{dmath*}

\begin{Shaded}
\begin{Highlighting}[]
\NormalTok{a2 }\SpecialCharTok{//} \FunctionTok{StandardForm}

\CommentTok{(*CartesianMomentum[p {-} q] + 2 CartesianMomentum[q]*)}
\end{Highlighting}
\end{Shaded}

\begin{Shaded}
\begin{Highlighting}[]
\NormalTok{a2 }\SpecialCharTok{//}\NormalTok{ MomentumExpand }\SpecialCharTok{//} \FunctionTok{StandardForm}

\CommentTok{(*CartesianMomentum[p] + CartesianMomentum[q]*)}
\end{Highlighting}
\end{Shaded}

\begin{Shaded}
\begin{Highlighting}[]
\NormalTok{a2 }\SpecialCharTok{//}\NormalTok{ MomentumCombine }\SpecialCharTok{//} \FunctionTok{StandardForm}

\CommentTok{(*CartesianMomentum[p + q]*)}
\end{Highlighting}
\end{Shaded}

Notice that when changing the dimension, one must specify its value as
if the the 3-vector were the spatial component of the corresponding
4-vector

\begin{Shaded}
\begin{Highlighting}[]
\NormalTok{ChangeDimension}\OperatorTok{[}\NormalTok{CartesianMomentum}\OperatorTok{[}\FunctionTok{p}\OperatorTok{],} \FunctionTok{d} \SpecialCharTok{{-}} \DecValTok{1}\OperatorTok{]} \SpecialCharTok{//} \FunctionTok{StandardForm}

\CommentTok{(*CartesianMomentum[p, {-}2 + d]*)}
\end{Highlighting}
\end{Shaded}

\begin{Shaded}
\begin{Highlighting}[]
\NormalTok{ChangeDimension}\OperatorTok{[}\NormalTok{CartesianMomentum}\OperatorTok{[}\FunctionTok{p}\OperatorTok{],} \FunctionTok{d}\OperatorTok{]} \SpecialCharTok{//} \FunctionTok{StandardForm}

\CommentTok{(*CartesianMomentum[p, {-}1 + d]*)}
\end{Highlighting}
\end{Shaded}

\begin{Shaded}
\begin{Highlighting}[]
\FunctionTok{Clear}\OperatorTok{[}\NormalTok{a1}\OperatorTok{,}\NormalTok{ a2}\OperatorTok{]}
\end{Highlighting}
\end{Shaded}

\end{document}
