% !TeX program = pdflatex
% !TeX root = CartesianPair.tex

\documentclass[../FeynCalcManual.tex]{subfiles}
\begin{document}
\hypertarget{cartesianpair}{
\section{CartesianPair}\label{cartesianpair}\index{CartesianPair}}

\texttt{CartesianPair[\allowbreak{}a,\ \allowbreak{}b]} is a special
pairing used in the internal representation. \texttt{a} and \texttt{b}
may have heads \texttt{CartesianIndex} or \texttt{CartesianMomentum}. If
both \texttt{a} and \texttt{b} have head \texttt{CartesianIndex}, the
Kronecker delta is understood. If \texttt{a} and \texttt{b} have head
\texttt{CartesianMomentum}, a Cartesian scalar product is meant. If one
of \texttt{a} and \texttt{b} has head \texttt{CartesianIndex} and the
other \texttt{CartesianMomentum}, a Cartesian vector \(p^i\) is
understood.

\subsection{See also}

\hyperlink{toc}{Overview}, \hyperlink{pair}{Pair},
\hyperlink{temporalpair}{TemporalPair}.

\subsection{Examples}

This represents a three-dimensional Kronecker delta

\begin{Shaded}
\begin{Highlighting}[]
\NormalTok{CartesianPair}\OperatorTok{[}\NormalTok{CartesianIndex}\OperatorTok{[}\FunctionTok{i}\OperatorTok{],}\NormalTok{ CartesianIndex}\OperatorTok{[}\FunctionTok{j}\OperatorTok{]]}
\end{Highlighting}
\end{Shaded}

\begin{dmath*}\breakingcomma
\bar{\delta }^{ij}
\end{dmath*}

This is a \(D-1\)-dimensional Kronecker delta

\begin{Shaded}
\begin{Highlighting}[]
\NormalTok{CartesianPair}\OperatorTok{[}\NormalTok{CartesianIndex}\OperatorTok{[}\FunctionTok{i}\OperatorTok{,} \FunctionTok{D} \SpecialCharTok{{-}} \DecValTok{1}\OperatorTok{],}\NormalTok{ CartesianIndex}\OperatorTok{[}\FunctionTok{j}\OperatorTok{,} \FunctionTok{D} \SpecialCharTok{{-}} \DecValTok{1}\OperatorTok{]]}
\end{Highlighting}
\end{Shaded}

\begin{dmath*}\breakingcomma
\delta ^{ij}
\end{dmath*}

If the Cartesian indices live in different dimensions, this gets
resolved according to the t'Hoft-Veltman-Breitenlohner-Maison
prescription

\begin{Shaded}
\begin{Highlighting}[]
\NormalTok{CartesianPair}\OperatorTok{[}\NormalTok{CartesianIndex}\OperatorTok{[}\FunctionTok{i}\OperatorTok{,} \FunctionTok{D} \SpecialCharTok{{-}} \DecValTok{1}\OperatorTok{],}\NormalTok{ CartesianIndex}\OperatorTok{[}\FunctionTok{j}\OperatorTok{]]}
\end{Highlighting}
\end{Shaded}

\begin{dmath*}\breakingcomma
\bar{\delta }^{ij}
\end{dmath*}

\begin{Shaded}
\begin{Highlighting}[]
\NormalTok{CartesianPair}\OperatorTok{[}\NormalTok{CartesianIndex}\OperatorTok{[}\FunctionTok{i}\OperatorTok{,} \FunctionTok{D} \SpecialCharTok{{-}} \DecValTok{1}\OperatorTok{],}\NormalTok{ CartesianIndex}\OperatorTok{[}\FunctionTok{j}\OperatorTok{,} \FunctionTok{D} \SpecialCharTok{{-}} \DecValTok{4}\OperatorTok{]]}
\end{Highlighting}
\end{Shaded}

\begin{dmath*}\breakingcomma
\hat{\delta }^{ij}
\end{dmath*}

\begin{Shaded}
\begin{Highlighting}[]
\NormalTok{CartesianPair}\OperatorTok{[}\NormalTok{CartesianIndex}\OperatorTok{[}\FunctionTok{i}\OperatorTok{],}\NormalTok{ CartesianIndex}\OperatorTok{[}\FunctionTok{j}\OperatorTok{,} \FunctionTok{D} \SpecialCharTok{{-}} \DecValTok{4}\OperatorTok{]]}
\end{Highlighting}
\end{Shaded}

\begin{dmath*}\breakingcomma
0
\end{dmath*}

A \(3\)-dimensional Cartesian vector

\begin{Shaded}
\begin{Highlighting}[]
\NormalTok{CartesianPair}\OperatorTok{[}\NormalTok{CartesianIndex}\OperatorTok{[}\FunctionTok{i}\OperatorTok{],}\NormalTok{ CartesianMomentum}\OperatorTok{[}\FunctionTok{p}\OperatorTok{]]}
\end{Highlighting}
\end{Shaded}

\begin{dmath*}\breakingcomma
\overline{p}^i
\end{dmath*}

A \(D-1\)-dimensional Cartesian vector

\begin{Shaded}
\begin{Highlighting}[]
\NormalTok{CartesianPair}\OperatorTok{[}\NormalTok{CartesianIndex}\OperatorTok{[}\FunctionTok{i}\OperatorTok{,} \FunctionTok{D} \SpecialCharTok{{-}} \DecValTok{1}\OperatorTok{],}\NormalTok{ CartesianMomentum}\OperatorTok{[}\FunctionTok{p}\OperatorTok{,} \FunctionTok{D} \SpecialCharTok{{-}} \DecValTok{1}\OperatorTok{]]}
\end{Highlighting}
\end{Shaded}

\begin{dmath*}\breakingcomma
p^i
\end{dmath*}

\(3\)-dimensional scalar products of Cartesian vectors

\begin{Shaded}
\begin{Highlighting}[]
\NormalTok{CartesianPair}\OperatorTok{[}\NormalTok{CartesianMomentum}\OperatorTok{[}\FunctionTok{q}\OperatorTok{],}\NormalTok{ CartesianMomentum}\OperatorTok{[}\FunctionTok{p}\OperatorTok{]]}
\end{Highlighting}
\end{Shaded}

\begin{dmath*}\breakingcomma
\overline{p}\cdot \overline{q}
\end{dmath*}

\begin{Shaded}
\begin{Highlighting}[]
\NormalTok{CartesianPair}\OperatorTok{[}\NormalTok{CartesianMomentum}\OperatorTok{[}\FunctionTok{p}\OperatorTok{],}\NormalTok{ CartesianMomentum}\OperatorTok{[}\FunctionTok{p}\OperatorTok{]]}
\end{Highlighting}
\end{Shaded}

\begin{dmath*}\breakingcomma
\overline{p}^2
\end{dmath*}

\begin{Shaded}
\begin{Highlighting}[]
\NormalTok{CartesianPair}\OperatorTok{[}\NormalTok{CartesianMomentum}\OperatorTok{[}\FunctionTok{p} \SpecialCharTok{{-}} \FunctionTok{q}\OperatorTok{],}\NormalTok{ CartesianMomentum}\OperatorTok{[}\FunctionTok{p}\OperatorTok{]]}
\end{Highlighting}
\end{Shaded}

\begin{dmath*}\breakingcomma
\overline{p}\cdot (\overline{p}-\overline{q})
\end{dmath*}

\begin{Shaded}
\begin{Highlighting}[]
\NormalTok{CartesianPair}\OperatorTok{[}\NormalTok{CartesianMomentum}\OperatorTok{[}\FunctionTok{p}\OperatorTok{],}\NormalTok{ CartesianMomentum}\OperatorTok{[}\FunctionTok{p}\OperatorTok{]]}\SpecialCharTok{\^{}}\DecValTok{2}
\end{Highlighting}
\end{Shaded}

\begin{dmath*}\breakingcomma
\overline{p}^4
\end{dmath*}

\begin{Shaded}
\begin{Highlighting}[]
\NormalTok{CartesianPair}\OperatorTok{[}\NormalTok{CartesianMomentum}\OperatorTok{[}\FunctionTok{p}\OperatorTok{],}\NormalTok{ CartesianMomentum}\OperatorTok{[}\FunctionTok{p}\OperatorTok{]]}\SpecialCharTok{\^{}}\DecValTok{3}
\end{Highlighting}
\end{Shaded}

\begin{dmath*}\breakingcomma
\overline{p}^6
\end{dmath*}

\begin{Shaded}
\begin{Highlighting}[]
\NormalTok{ExpandScalarProduct}\OperatorTok{[}\NormalTok{CartesianPair}\OperatorTok{[}\NormalTok{CartesianMomentum}\OperatorTok{[}\FunctionTok{p} \SpecialCharTok{{-}} \FunctionTok{q}\OperatorTok{],}\NormalTok{ CartesianMomentum}\OperatorTok{[}\FunctionTok{p}\OperatorTok{]]]}
\end{Highlighting}
\end{Shaded}

\begin{dmath*}\breakingcomma
\overline{p}^2-\overline{p}\cdot \overline{q}
\end{dmath*}

\begin{Shaded}
\begin{Highlighting}[]
\NormalTok{CartesianPair}\OperatorTok{[}\NormalTok{CartesianMomentum}\OperatorTok{[}\SpecialCharTok{{-}}\FunctionTok{q}\OperatorTok{],}\NormalTok{ CartesianMomentum}\OperatorTok{[}\FunctionTok{p}\OperatorTok{]]} \SpecialCharTok{+} 
\NormalTok{  CartesianPair}\OperatorTok{[}\NormalTok{CartesianMomentum}\OperatorTok{[}\FunctionTok{q}\OperatorTok{],}\NormalTok{ CartesianMomentum}\OperatorTok{[}\FunctionTok{p}\OperatorTok{]]}
\end{Highlighting}
\end{Shaded}

\begin{dmath*}\breakingcomma
0
\end{dmath*}
\end{document}
