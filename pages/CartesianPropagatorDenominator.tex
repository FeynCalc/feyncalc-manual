% !TeX program = pdflatex
% !TeX root = CartesianPropagatorDenominator.tex

\documentclass[../FeynCalcManual.tex]{subfiles}
\begin{document}
\hypertarget{cartesianpropagatordenominator}{%
\section{CartesianPropagatorDenominator}\label{cartesianpropagatordenominator}}

\texttt{CartesianPropagatorDenominator[\allowbreak{}propSq  + ...,\ \allowbreak{}propEik + ...,\ \allowbreak{}m^2,\ \allowbreak{}\{\allowbreak{}n,\ \allowbreak{}s\}]}
encodes a generic Cartesian propagator denominator of the form
\(\frac{1}{[(q1+...)^2 + q1.p1 + ... + m^2 + s*I \eta]^n}\)

\texttt{propSq} should be of the form
\texttt{CartesianMomentum[\allowbreak{}q1,\ \allowbreak{}D - 1]}, while
\texttt{propEik} should look like
\texttt{CartesianPair[\allowbreak{}CartesianMomentum[\allowbreak{}q1,\ \allowbreak{}D - 1],\ \allowbreak{}CartesianMomentum[\allowbreak{}p1,\ \allowbreak{}D - 1]}.

\texttt{CartesianPropagatorDenominator} is an internal object. To enter
such propagators in FeynCalc you should use \texttt{CFAD}.

\subsection{See also}

\hyperlink{toc}{Overview}, \hyperlink{cfad}{CFAD},
\hyperlink{feynampdenominator}{FeynAmpDenominator}.

\subsection{Examples}

Standard \(3\)-dimensional Cartesian propagator

\begin{Shaded}
\begin{Highlighting}[]
\NormalTok{FeynAmpDenominator}\OperatorTok{[}\NormalTok{CartesianPropagatorDenominator}\OperatorTok{[}\NormalTok{CartesianMomentum}\OperatorTok{[}\FunctionTok{p}\OperatorTok{,} \FunctionTok{D} \SpecialCharTok{{-}} \DecValTok{1}\OperatorTok{],} \DecValTok{0}\OperatorTok{,} \FunctionTok{m}\SpecialCharTok{\^{}}\DecValTok{2}\OperatorTok{,} \OperatorTok{\{}\DecValTok{1}\OperatorTok{,} \SpecialCharTok{{-}}\DecValTok{1}\OperatorTok{\}]]}
\end{Highlighting}
\end{Shaded}

\begin{dmath*}\breakingcomma
\frac{1}{(p^2+m^2-i \eta )}
\end{dmath*}

Here we switch the sign of the mass term

\begin{Shaded}
\begin{Highlighting}[]
\NormalTok{FeynAmpDenominator}\OperatorTok{[}\NormalTok{CartesianPropagatorDenominator}\OperatorTok{[}\NormalTok{CartesianMomentum}\OperatorTok{[}\FunctionTok{p}\OperatorTok{,} \FunctionTok{D} \SpecialCharTok{{-}} \DecValTok{1}\OperatorTok{],} \DecValTok{0}\OperatorTok{,} \SpecialCharTok{{-}}\FunctionTok{m}\SpecialCharTok{\^{}}\DecValTok{2}\OperatorTok{,} \OperatorTok{\{}\DecValTok{1}\OperatorTok{,} \SpecialCharTok{{-}}\DecValTok{1}\OperatorTok{\}]]}
\end{Highlighting}
\end{Shaded}

\begin{dmath*}\breakingcomma
\frac{1}{(p^2-m^2-i \eta )}
\end{dmath*}

And here also the sign of \(i \eta\)

\begin{Shaded}
\begin{Highlighting}[]
\NormalTok{FeynAmpDenominator}\OperatorTok{[}\NormalTok{CartesianPropagatorDenominator}\OperatorTok{[}\NormalTok{CartesianMomentum}\OperatorTok{[}\FunctionTok{p}\OperatorTok{,} \FunctionTok{D} \SpecialCharTok{{-}} \DecValTok{1}\OperatorTok{],} \DecValTok{0}\OperatorTok{,} \SpecialCharTok{{-}}\FunctionTok{m}\SpecialCharTok{\^{}}\DecValTok{2}\OperatorTok{,} \OperatorTok{\{}\DecValTok{1}\OperatorTok{,} \SpecialCharTok{+}\DecValTok{1}\OperatorTok{\}]]}
\end{Highlighting}
\end{Shaded}

\begin{dmath*}\breakingcomma
\frac{1}{(p^2-m^2+i \eta )}
\end{dmath*}

Eikonal Cartesian propagator with a residual mass term

\begin{Shaded}
\begin{Highlighting}[]
\NormalTok{FeynAmpDenominator}\OperatorTok{[}\NormalTok{CartesianPropagatorDenominator}\OperatorTok{[}\DecValTok{0}\OperatorTok{,} 
\NormalTok{   CartesianPair}\OperatorTok{[}\NormalTok{CartesianMomentum}\OperatorTok{[}\FunctionTok{p}\OperatorTok{,} \FunctionTok{D} \SpecialCharTok{{-}} \DecValTok{1}\OperatorTok{],}\NormalTok{ CartesianMomentum}\OperatorTok{[}\FunctionTok{q}\OperatorTok{,} \FunctionTok{D} \SpecialCharTok{{-}} \DecValTok{1}\OperatorTok{]],} \FunctionTok{m}\SpecialCharTok{\^{}}\DecValTok{2}\OperatorTok{,} \OperatorTok{\{}\DecValTok{1}\OperatorTok{,} \SpecialCharTok{{-}}\DecValTok{1}\OperatorTok{\}]]}
\end{Highlighting}
\end{Shaded}

\begin{dmath*}\breakingcomma
\frac{1}{(p\cdot q+m^2-i \eta )}
\end{dmath*}
\end{document}
