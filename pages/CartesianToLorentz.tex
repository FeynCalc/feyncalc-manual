% !TeX program = pdflatex
% !TeX root = CartesianToLorentz.tex

\documentclass[../FeynCalcManual.tex]{subfiles}
\begin{document}
\hypertarget{cartesiantolorentz}{
\section{CartesianToLorentz}\label{cartesiantolorentz}\index{CartesianToLorentz}}

\texttt{CartesianToLorentz[\allowbreak{}exp]} rewrites Cartesian tensors
in form of Lorentz tensors (when possible). Using options one can
specify which types of tensors should be converted.

\subsection{See also}

\hyperlink{toc}{Overview},
\hyperlink{lorentztocartesian}{LorentzToCartesian}.

\subsection{Examples}

\begin{Shaded}
\begin{Highlighting}[]
\NormalTok{CGS}\OperatorTok{[}\FunctionTok{p}\OperatorTok{]} 
 
\SpecialCharTok{\%} \SpecialCharTok{//}\NormalTok{ CartesianToLorentz}
\end{Highlighting}
\end{Shaded}

\begin{dmath*}\breakingcomma
\overline{\gamma }\cdot \overline{p}
\end{dmath*}

\begin{dmath*}\breakingcomma
p^0 \bar{\gamma }^0-\bar{\gamma }\cdot \overline{p}
\end{dmath*}

\begin{Shaded}
\begin{Highlighting}[]
\NormalTok{CSP}\OperatorTok{[}\FunctionTok{p}\OperatorTok{,} \FunctionTok{q}\OperatorTok{]} 
 
\SpecialCharTok{\%} \SpecialCharTok{//}\NormalTok{ CartesianToLorentz}
\end{Highlighting}
\end{Shaded}

\begin{dmath*}\breakingcomma
\overline{p}\cdot \overline{q}
\end{dmath*}

\begin{dmath*}\breakingcomma
p^0 q^0-\overline{p}\cdot \overline{q}
\end{dmath*}
\end{document}
