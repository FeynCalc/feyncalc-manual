% !TeX program = pdflatex
% !TeX root = Cases2.tex

\documentclass[../FeynCalcManual.tex]{subfiles}
\begin{document}
\hypertarget{cases2}{
\section{Cases2}\label{cases2}\index{Cases2}}

\texttt{Cases2[\allowbreak{}expr,\ \allowbreak{}f]} returns a list of
all objects in \texttt{expr} with head \texttt{f}.

\texttt{Cases2[\allowbreak{}expr,\ \allowbreak{}f]} is equivalent to
\texttt{Cases2[\allowbreak{}\{\allowbreak{}expr\},\ \allowbreak{}f[\allowbreak{}___],\ \allowbreak{}Infinity]//Union}.

\texttt{Cases2[\allowbreak{}expr,\ \allowbreak{}f,\ \allowbreak{}g,\ \allowbreak{}...]}
or
\texttt{Cases2[\allowbreak{}expr,\ \allowbreak{}\{\allowbreak{}f,\ \allowbreak{}g,\ \allowbreak{}...\}]}
is equivalent to
\texttt{Cases[\allowbreak{}\{\allowbreak{}expr\},\ \allowbreak{}f[\allowbreak{}___] | g[\allowbreak{}___] ...]}.

\subsection{See also}

\hyperlink{toc}{Overview}, \hyperlink{variables2}{Variables2}.

\subsection{Examples}

\begin{Shaded}
\begin{Highlighting}[]
\NormalTok{Cases2}\OperatorTok{[}\FunctionTok{f}\OperatorTok{[}\FunctionTok{a}\OperatorTok{]} \SpecialCharTok{+} \FunctionTok{f}\OperatorTok{[}\FunctionTok{b}\OperatorTok{]}\SpecialCharTok{\^{}}\DecValTok{2} \SpecialCharTok{+} \FunctionTok{f}\OperatorTok{[}\FunctionTok{c}\OperatorTok{,} \FunctionTok{d}\OperatorTok{],} \FunctionTok{f}\OperatorTok{]}
\end{Highlighting}
\end{Shaded}

\begin{dmath*}\breakingcomma
\{f(a),f(b),f(c,d)\}
\end{dmath*}

\begin{Shaded}
\begin{Highlighting}[]
\NormalTok{Cases2}\OperatorTok{[}\FunctionTok{Sin}\OperatorTok{[}\FunctionTok{x}\OperatorTok{]} \FunctionTok{Sin}\OperatorTok{[}\FunctionTok{y} \SpecialCharTok{{-}} \FunctionTok{z}\OperatorTok{]} \SpecialCharTok{+} \FunctionTok{g}\OperatorTok{[}\FunctionTok{y}\OperatorTok{],} \FunctionTok{Sin}\OperatorTok{,} \FunctionTok{g}\OperatorTok{]}
\end{Highlighting}
\end{Shaded}

\begin{dmath*}\breakingcomma
\{g(y),\sin (x),\sin (y-z)\}
\end{dmath*}

\begin{Shaded}
\begin{Highlighting}[]
\NormalTok{Cases2}\OperatorTok{[}\FunctionTok{Sin}\OperatorTok{[}\FunctionTok{x}\OperatorTok{]} \FunctionTok{Sin}\OperatorTok{[}\FunctionTok{y} \SpecialCharTok{{-}} \FunctionTok{z}\OperatorTok{]} \SpecialCharTok{+} \FunctionTok{g}\OperatorTok{[}\FunctionTok{x}\OperatorTok{]} \SpecialCharTok{+} \FunctionTok{g}\OperatorTok{[}\FunctionTok{a}\OperatorTok{,} \FunctionTok{b}\OperatorTok{,} \FunctionTok{c}\OperatorTok{],} \OperatorTok{\{}\FunctionTok{Sin}\OperatorTok{,} \FunctionTok{g}\OperatorTok{\}]}
\end{Highlighting}
\end{Shaded}

\begin{dmath*}\breakingcomma
\{g(x),g(a,b,c),\sin (x),\sin (y-z)\}
\end{dmath*}

\begin{Shaded}
\begin{Highlighting}[]
\NormalTok{Cases2}\OperatorTok{[}\NormalTok{GS}\OperatorTok{[}\FunctionTok{p}\OperatorTok{]}\NormalTok{ . GS}\OperatorTok{[}\FunctionTok{q}\OperatorTok{]} \SpecialCharTok{+}\NormalTok{ SP}\OperatorTok{[}\FunctionTok{p}\OperatorTok{,} \FunctionTok{p}\OperatorTok{],} \FunctionTok{Dot}\OperatorTok{]}
\end{Highlighting}
\end{Shaded}

\begin{dmath*}\breakingcomma
\left\{\left(\bar{\gamma }\cdot \overline{p}\right).\left(\bar{\gamma }\cdot \overline{q}\right)\right\}
\end{dmath*}
\end{document}
