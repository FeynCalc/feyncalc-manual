% !TeX program = pdflatex
% !TeX root = ChangeDimension.tex

\documentclass[../FeynCalcManual.tex]{subfiles}
\begin{document}
\hypertarget{changedimension}{
\section{ChangeDimension}\label{changedimension}\index{ChangeDimension}}

\texttt{ChangeDimension[\allowbreak{}exp,\ \allowbreak{}dim]} changes
all \texttt{LorentzIndex} and \texttt{Momentum} symbols in \texttt{exp}
to dimension \texttt{dim} (and also Levi-Civita-tensors, Dirac slashes
and Dirac matrices).

Notice that the dimension of \texttt{CartesianIndex} and
\texttt{CartesianMomentum} objects will be changed to \texttt{dim-1},
not \texttt{dim}.

\subsection{See also}

\hyperlink{toc}{Overview}, \hyperlink{lorentzindex}{LorentzIndex},
\hyperlink{momentum}{Momentum}, \hyperlink{diracgamma}{DiracGamma},
\hyperlink{eps}{Eps}.

\subsection{Examples}

Remember that \texttt{LorentzIndex[\allowbreak{}mu,\ \allowbreak{}4]} is
simplified to \texttt{LorentzIndex[\allowbreak{}mu]} and
\texttt{Momentum[\allowbreak{}p,\ \allowbreak{}4]} to
\texttt{Momentum[\allowbreak{}p]}. Thus the following objects are
defined in four dimensions.

\begin{Shaded}
\begin{Highlighting}[]
\OperatorTok{\{}\NormalTok{LorentzIndex}\OperatorTok{[}\SpecialCharTok{\textbackslash{}}\OperatorTok{[}\NormalTok{Mu}\OperatorTok{]],}\NormalTok{ Momentum}\OperatorTok{[}\FunctionTok{p}\OperatorTok{]\}} 
 
\NormalTok{ex }\ExtensionTok{=}\NormalTok{ ChangeDimension}\OperatorTok{[}\SpecialCharTok{\%}\OperatorTok{,} \FunctionTok{D}\OperatorTok{]}
\end{Highlighting}
\end{Shaded}

\begin{dmath*}\breakingcomma
\left\{\mu ,\overline{p}\right\}
\end{dmath*}

\begin{dmath*}\breakingcomma
\{\mu ,p\}
\end{dmath*}

\begin{Shaded}
\begin{Highlighting}[]
\NormalTok{ex }\SpecialCharTok{//} \FunctionTok{StandardForm}

\CommentTok{(*\{LorentzIndex[\textbackslash{}[Mu], D], Momentum[p, D]\}*)}
\end{Highlighting}
\end{Shaded}

This changes all non-4-dimensional objects to 4-dimensional ones

\begin{Shaded}
\begin{Highlighting}[]
\NormalTok{ChangeDimension}\OperatorTok{[}\SpecialCharTok{\%\%}\OperatorTok{,} \DecValTok{4}\OperatorTok{]} \SpecialCharTok{//} \FunctionTok{StandardForm}

\CommentTok{(*\{LorentzIndex[\textbackslash{}[Mu]], Momentum[p]\}*)}
\end{Highlighting}
\end{Shaded}

Consider the following list of 4- and D-dimensional objects

\begin{Shaded}
\begin{Highlighting}[]
\OperatorTok{\{}\NormalTok{GA}\OperatorTok{[}\SpecialCharTok{\textbackslash{}}\OperatorTok{[}\NormalTok{Mu}\OperatorTok{],} \SpecialCharTok{\textbackslash{}}\OperatorTok{[}\NormalTok{Nu}\OperatorTok{]]}\NormalTok{ MT}\OperatorTok{[}\SpecialCharTok{\textbackslash{}}\OperatorTok{[}\NormalTok{Mu}\OperatorTok{],} \SpecialCharTok{\textbackslash{}}\OperatorTok{[}\NormalTok{Nu}\OperatorTok{]],}\NormalTok{ GAD}\OperatorTok{[}\SpecialCharTok{\textbackslash{}}\OperatorTok{[}\NormalTok{Mu}\OperatorTok{],} \SpecialCharTok{\textbackslash{}}\OperatorTok{[}\NormalTok{Nu}\OperatorTok{]]}\NormalTok{ MTD}\OperatorTok{[}\SpecialCharTok{\textbackslash{}}\OperatorTok{[}\NormalTok{Mu}\OperatorTok{],} \SpecialCharTok{\textbackslash{}}\OperatorTok{[}\NormalTok{Nu}\OperatorTok{]]} \FunctionTok{f}\OperatorTok{[}\FunctionTok{D}\OperatorTok{]\}} 
 
\NormalTok{DiracTrick }\SpecialCharTok{/}\NormalTok{@ Contract }\SpecialCharTok{/}\NormalTok{@ }\SpecialCharTok{\%} 
 
\NormalTok{DiracTrick }\SpecialCharTok{/}\NormalTok{@ Contract }\SpecialCharTok{/}\NormalTok{@ ChangeDimension}\OperatorTok{[}\SpecialCharTok{\%\%}\OperatorTok{,} \FunctionTok{n}\OperatorTok{]}
\end{Highlighting}
\end{Shaded}

\begin{dmath*}\breakingcomma
\left\{\bar{\gamma }^{\mu }.\bar{\gamma }^{\nu } \bar{g}^{\mu \nu },f(D) \gamma ^{\mu }.\gamma ^{\nu } g^{\mu \nu }\right\}
\end{dmath*}

\begin{dmath*}\breakingcomma
\{4,D f(D)\}
\end{dmath*}

\begin{dmath*}\breakingcomma
\{n,n f(D)\}
\end{dmath*}

Any explicit occurrence of \(D\) (like in \(f(D)\)) is not replaced by
\texttt{ChangeDimension}.

\begin{Shaded}
\begin{Highlighting}[]
\NormalTok{LC}\OperatorTok{[}\SpecialCharTok{\textbackslash{}}\OperatorTok{[}\NormalTok{Mu}\OperatorTok{],} \SpecialCharTok{\textbackslash{}}\OperatorTok{[}\NormalTok{Nu}\OperatorTok{],} \SpecialCharTok{\textbackslash{}}\OperatorTok{[}\NormalTok{Rho}\OperatorTok{],} \SpecialCharTok{\textbackslash{}}\OperatorTok{[}\NormalTok{Sigma}\OperatorTok{]]} 
 
\NormalTok{ChangeDimension}\OperatorTok{[}\SpecialCharTok{\%}\OperatorTok{,} \FunctionTok{D}\OperatorTok{]} 
 
\NormalTok{Factor2}\OperatorTok{[}\NormalTok{Contract}\OperatorTok{[}\SpecialCharTok{\%\^{}}\DecValTok{2}\OperatorTok{]]}
\end{Highlighting}
\end{Shaded}

\begin{dmath*}\breakingcomma
\bar{\epsilon }^{\mu \nu \rho \sigma }
\end{dmath*}

\begin{dmath*}\breakingcomma
\overset{\text{}}{\epsilon }^{\mu \nu \rho \sigma }
\end{dmath*}

\begin{dmath*}\breakingcomma
(1-D) (2-D) (3-D) D
\end{dmath*}

\begin{Shaded}
\begin{Highlighting}[]
\NormalTok{Contract}\OperatorTok{[}\NormalTok{LC}\OperatorTok{[}\SpecialCharTok{\textbackslash{}}\OperatorTok{[}\NormalTok{Mu}\OperatorTok{],} \SpecialCharTok{\textbackslash{}}\OperatorTok{[}\NormalTok{Nu}\OperatorTok{],} \SpecialCharTok{\textbackslash{}}\OperatorTok{[}\NormalTok{Rho}\OperatorTok{],} \SpecialCharTok{\textbackslash{}}\OperatorTok{[}\NormalTok{Sigma}\OperatorTok{]]}\SpecialCharTok{\^{}}\DecValTok{2}\OperatorTok{]}
\end{Highlighting}
\end{Shaded}

\begin{dmath*}\breakingcomma
-24
\end{dmath*}
\end{document}
