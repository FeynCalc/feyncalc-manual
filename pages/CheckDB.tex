% !TeX program = pdflatex
% !TeX root = CheckDB.tex

\documentclass[../FeynCalcManual.tex]{subfiles}
\begin{document}
\hypertarget{checkdb}{%
\section{CheckDB}\label{checkdb}}

\texttt{CheckDB[\allowbreak{}exp,\ \allowbreak{}fil]} saves (with
\texttt{Put}) or retrieves (with \texttt{Get}) \texttt{exp} from a file
\texttt{fil}. It checks if the setting of the option \texttt{Directory}
is a valid directory name and if \texttt{fil} is a valid file name and
does exist. If it does, \texttt{Get[\allowbreak{}fil]} is executed. If
\texttt{fil} does not exist, \texttt{exp} gets evaluated and saved to
\texttt{fil}.

Saving and evaluating can be further controlled with the options
\texttt{ForceSave} and \texttt{NoSave}. If the option \texttt{Check} is
set to \texttt{False} the return value is what is evaluated (see above).
If \texttt{Check} is set to \texttt{True} the return value is
\texttt{True} or \texttt{False} depending on whether the evaluation of
\texttt{exp} agrees with what is loaded from \texttt{fil} or
\texttt{fil} does not exist.

Default value of Check : \texttt{False}. If \texttt{fil} ends with
\texttt{".Gen"} or \texttt{".Mod"}, the setting of \texttt{Directory} is
ignored and \texttt{fil} is saved in the \texttt{"CouplingVectors"}
subdirectory of \texttt{"Phi"}. If \texttt{fil} ends with
\texttt{".Fac"}, the setting of Directory is ignored and \texttt{fil} is
saved in the \texttt{"Factors"} subdirectory of \texttt{"Phi"}.

If \texttt{fil} is a file name with full path, the setting of
\texttt{Directory} is also ignored.

\subsection{See also}

\hyperlink{toc}{Overview}

\subsection{Examples}

The first time the Table function is evaluated and the result saved into
the test.s file.

\begin{Shaded}
\begin{Highlighting}[]
\NormalTok{CheckDB}\OperatorTok{[}\FunctionTok{Table}\OperatorTok{[}\FunctionTok{WriteString}\OperatorTok{[}\StringTok{"stdout"}\OperatorTok{,} \StringTok{"test "}\OperatorTok{]}\NormalTok{; }\FunctionTok{i}\OperatorTok{,} \OperatorTok{\{}\FunctionTok{i}\OperatorTok{,} \DecValTok{2}\OperatorTok{\}],} \StringTok{"test.s"}\OperatorTok{]}
\end{Highlighting}
\end{Shaded}

test test

\begin{dmath*}\breakingcomma
\{1,2\}
\end{dmath*}

Executing the same a second time will just load the result from test.s
and not evaluate the Table function.

\begin{Shaded}
\begin{Highlighting}[]
\NormalTok{CheckDB}\OperatorTok{[}\FunctionTok{Table}\OperatorTok{[}\FunctionTok{WriteString}\OperatorTok{[}\StringTok{"stdout"}\OperatorTok{,} \StringTok{"test "}\OperatorTok{]}\NormalTok{; }\FunctionTok{i}\OperatorTok{,} \OperatorTok{\{}\FunctionTok{i}\OperatorTok{,} \DecValTok{2}\OperatorTok{\}],} \StringTok{"test.s"}\OperatorTok{]}
\end{Highlighting}
\end{Shaded}

\begin{dmath*}\breakingcomma
\{1,2\}
\end{dmath*}

This shows the actual saved value of test.s.

\begin{Shaded}
\begin{Highlighting}[]
\FunctionTok{Import}\OperatorTok{[}\NormalTok{ToFileName}\OperatorTok{[}\FunctionTok{Directory} \OtherTok{/.} \FunctionTok{Options}\OperatorTok{[}\NormalTok{CheckDB}\OperatorTok{],} \StringTok{"test.s"}\OperatorTok{],} \StringTok{"Text"}\OperatorTok{]}
\end{Highlighting}
\end{Shaded}

\begin{dmath*}\breakingcomma
\text{$\{$1, 2$\}$}
\end{dmath*}

\begin{Shaded}
\begin{Highlighting}[]
\FunctionTok{DeleteFile}\OperatorTok{[}\NormalTok{ToFileName}\OperatorTok{[}\FunctionTok{Directory} \OtherTok{/.} \FunctionTok{Options}\OperatorTok{[}\NormalTok{CheckDB}\OperatorTok{],} \StringTok{"test.s"}\OperatorTok{]]}
\end{Highlighting}
\end{Shaded}

\end{document}
