% !TeX program = pdflatex
% !TeX root = ChiralityProjector.tex

\documentclass[../FeynCalcManual.tex]{subfiles}
\begin{document}
\hypertarget{chiralityprojector}{%
\section{ChiralityProjector}\label{chiralityprojector}}

\texttt{ChiralityProjector[\allowbreak{}+1]} denotes
\(1/2\left(1+\gamma^5\right)\).

ChiralityProjector{[}-1{]} denotes \(1/2\left(1+\gamma ^5\right)\).

The shortcut \texttt{ChiralityProjector} is deprecated, please use
\texttt{GA[\allowbreak{}6]} and \texttt{GA[\allowbreak{}7]} instead!

\subsection{See also}

\hyperlink{toc}{Overview}, \hyperlink{ga}{GA}, \hyperlink{fci}{FCI}.

\subsection{Examples}

\begin{Shaded}
\begin{Highlighting}[]
\OperatorTok{\{}\NormalTok{ChiralityProjector}\OperatorTok{[}\SpecialCharTok{+}\DecValTok{1}\OperatorTok{],}\NormalTok{ ChiralityProjector}\OperatorTok{[}\SpecialCharTok{{-}}\DecValTok{1}\OperatorTok{]\}}
\NormalTok{DiracSimplify}\OperatorTok{[}\NormalTok{\#}\OperatorTok{,}\NormalTok{ DiracSubstitute67 }\OtherTok{{-}\textgreater{}} \ConstantTok{True}\OperatorTok{]}\NormalTok{ \& }\SpecialCharTok{/}\NormalTok{@ }\SpecialCharTok{\%}
\end{Highlighting}
\end{Shaded}

\begin{dmath*}\breakingcomma
\left\{\bar{\gamma }^6,\bar{\gamma }^7\right\}
\end{dmath*}

\begin{dmath*}\breakingcomma
\left\{\frac{\bar{\gamma }^5}{2}+\frac{1}{2},\frac{1}{2}-\frac{\bar{\gamma }^5}{2}\right\}
\end{dmath*}

\texttt{ChiralityProjector} is scheduled for removal in the future
versions of FeynCalc. The safe alternative is to use GA{[}6{]} and
GA{[}7{]}.

\begin{Shaded}
\begin{Highlighting}[]
\OperatorTok{\{}\NormalTok{GA}\OperatorTok{[}\DecValTok{6}\OperatorTok{],}\NormalTok{ GA}\OperatorTok{[}\DecValTok{7}\OperatorTok{]\}}
\end{Highlighting}
\end{Shaded}

\begin{dmath*}\breakingcomma
\left\{\bar{\gamma }^6,\bar{\gamma }^7\right\}
\end{dmath*}

\begin{Shaded}
\begin{Highlighting}[]
\NormalTok{FCI}\OperatorTok{[}\NormalTok{GA}\OperatorTok{[}\DecValTok{6}\OperatorTok{]]} \ExtensionTok{===}\NormalTok{ ChiralityProjector}\OperatorTok{[}\SpecialCharTok{+}\DecValTok{1}\OperatorTok{]}
\end{Highlighting}
\end{Shaded}

\begin{dmath*}\breakingcomma
\text{True}
\end{dmath*}

\begin{Shaded}
\begin{Highlighting}[]
\NormalTok{FCI}\OperatorTok{[}\NormalTok{GA}\OperatorTok{[}\DecValTok{7}\OperatorTok{]]} \ExtensionTok{===}\NormalTok{ ChiralityProjector}\OperatorTok{[}\SpecialCharTok{{-}}\DecValTok{1}\OperatorTok{]}
\end{Highlighting}
\end{Shaded}

\begin{dmath*}\breakingcomma
\text{True}
\end{dmath*}
\end{document}
