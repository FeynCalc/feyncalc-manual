% !TeX program = pdflatex
% !TeX root = ColorAlgebra.tex

\documentclass[../FeynCalcManual.tex]{subfiles}
\begin{document}
\hypertarget{color algebra}{
\section{Color algebra}\label{color algebra}\index{Color algebra}}

\subsection{See also}

\hyperlink{toc}{Overview}.

\subsection{Notation for colored
objects}\label{notation-for-colored-objects}

FeynCalc objects relevant for the color algebra are

\begin{Shaded}
\begin{Highlighting}[]
\NormalTok{SUNT}\OperatorTok{[}\FunctionTok{a}\OperatorTok{]}
\end{Highlighting}
\end{Shaded}

\begin{dmath*}\breakingcomma
T^a
\end{dmath*}

\begin{Shaded}
\begin{Highlighting}[]
\NormalTok{SUNF}\OperatorTok{[}\FunctionTok{a}\OperatorTok{,} \FunctionTok{b}\OperatorTok{,} \FunctionTok{c}\OperatorTok{]}
\end{Highlighting}
\end{Shaded}

\begin{dmath*}\breakingcomma
f^{abc}
\end{dmath*}

\begin{Shaded}
\begin{Highlighting}[]
\NormalTok{SUND}\OperatorTok{[}\FunctionTok{a}\OperatorTok{,} \FunctionTok{b}\OperatorTok{,} \FunctionTok{c}\OperatorTok{]}
\end{Highlighting}
\end{Shaded}

\begin{dmath*}\breakingcomma
d^{abc}
\end{dmath*}

\begin{Shaded}
\begin{Highlighting}[]
\NormalTok{SUNDelta}\OperatorTok{[}\FunctionTok{a}\OperatorTok{,} \FunctionTok{b}\OperatorTok{]}
\end{Highlighting}
\end{Shaded}

\begin{dmath*}\breakingcomma
\delta ^{ab}
\end{dmath*}

\begin{Shaded}
\begin{Highlighting}[]
\NormalTok{SUNN}
\end{Highlighting}
\end{Shaded}

\begin{dmath*}\breakingcomma
N
\end{dmath*}

\begin{Shaded}
\begin{Highlighting}[]
\NormalTok{CA}
\end{Highlighting}
\end{Shaded}

\begin{dmath*}\breakingcomma
C_A
\end{dmath*}

\begin{Shaded}
\begin{Highlighting}[]
\NormalTok{CF}
\end{Highlighting}
\end{Shaded}

\begin{dmath*}\breakingcomma
C_F
\end{dmath*}

There are two main functions to deal with colored objects:
\texttt{SUNSimplify} and \texttt{SUNTrace}. In general,
\texttt{SUNSimplify} will also simplify color traces when possible

\begin{Shaded}
\begin{Highlighting}[]
\NormalTok{SUNT}\OperatorTok{[}\FunctionTok{a}\OperatorTok{,} \FunctionTok{a}\OperatorTok{]}
\NormalTok{SUNSimplify}\OperatorTok{[}\SpecialCharTok{\%}\OperatorTok{]}
\end{Highlighting}
\end{Shaded}

\begin{dmath*}\breakingcomma
T^a.T^a
\end{dmath*}

\begin{dmath*}\breakingcomma
C_F
\end{dmath*}

\begin{Shaded}
\begin{Highlighting}[]
\NormalTok{SUNT}\OperatorTok{[}\FunctionTok{a}\OperatorTok{,} \FunctionTok{b}\OperatorTok{,} \FunctionTok{a}\OperatorTok{,} \FunctionTok{b}\OperatorTok{]}
\NormalTok{SUNSimplify}\OperatorTok{[}\SpecialCharTok{\%}\OperatorTok{]}
\end{Highlighting}
\end{Shaded}

\begin{dmath*}\breakingcomma
T^a.T^b.T^a.T^b
\end{dmath*}

\begin{dmath*}\breakingcomma
-\frac{1}{2} C_F \left(C_A-2 C_F\right)
\end{dmath*}

\begin{Shaded}
\begin{Highlighting}[]
\NormalTok{SUNT}\OperatorTok{[}\FunctionTok{b}\OperatorTok{,} \FunctionTok{d}\OperatorTok{,} \FunctionTok{a}\OperatorTok{,} \FunctionTok{b}\OperatorTok{,} \FunctionTok{d}\OperatorTok{]}
\NormalTok{SUNSimplify}\OperatorTok{[}\SpecialCharTok{\%}\OperatorTok{]}
\end{Highlighting}
\end{Shaded}

\begin{dmath*}\breakingcomma
T^b.T^d.T^a.T^b.T^d
\end{dmath*}

\begin{dmath*}\breakingcomma
\frac{T^a \left(C_A^2+1\right)}{4 C_A^2}
\end{dmath*}

\begin{Shaded}
\begin{Highlighting}[]
\NormalTok{SUNF}\OperatorTok{[}\FunctionTok{a}\OperatorTok{,} \FunctionTok{r}\OperatorTok{,} \FunctionTok{s}\OperatorTok{]}\NormalTok{ SUNF}\OperatorTok{[}\FunctionTok{b}\OperatorTok{,} \FunctionTok{r}\OperatorTok{,} \FunctionTok{s}\OperatorTok{]}
\NormalTok{SUNSimplify}\OperatorTok{[}\SpecialCharTok{\%}\OperatorTok{]}
\end{Highlighting}
\end{Shaded}

\begin{dmath*}\breakingcomma
f^{ars} f^{brs}
\end{dmath*}

\begin{dmath*}\breakingcomma
C_A \delta ^{ab}
\end{dmath*}

\begin{Shaded}
\begin{Highlighting}[]
\NormalTok{SUNF}\OperatorTok{[}\FunctionTok{a}\OperatorTok{,} \FunctionTok{b}\OperatorTok{,} \FunctionTok{c}\OperatorTok{]}\NormalTok{  SUNF}\OperatorTok{[}\FunctionTok{a}\OperatorTok{,} \FunctionTok{b}\OperatorTok{,} \FunctionTok{c}\OperatorTok{]}
\NormalTok{SUNSimplify}\OperatorTok{[}\SpecialCharTok{\%}\OperatorTok{]}
\end{Highlighting}
\end{Shaded}

\begin{dmath*}\breakingcomma
\left(f^{abc}\right)^2
\end{dmath*}

\begin{dmath*}\breakingcomma
2 C_A^2 C_F
\end{dmath*}

\begin{Shaded}
\begin{Highlighting}[]
\NormalTok{SUNF}\OperatorTok{[}\FunctionTok{a}\OperatorTok{,} \FunctionTok{b}\OperatorTok{,} \FunctionTok{c}\OperatorTok{]}\NormalTok{ SUND}\OperatorTok{[}\FunctionTok{d}\OperatorTok{,} \FunctionTok{b}\OperatorTok{,} \FunctionTok{c}\OperatorTok{]} 
\NormalTok{SUNSimplify}\OperatorTok{[}\SpecialCharTok{\%}\OperatorTok{]}
\end{Highlighting}
\end{Shaded}

\begin{dmath*}\breakingcomma
d^{bcd} f^{abc}
\end{dmath*}

\begin{dmath*}\breakingcomma
0
\end{dmath*}

\begin{Shaded}
\begin{Highlighting}[]
\NormalTok{SUND}\OperatorTok{[}\FunctionTok{a}\OperatorTok{,} \FunctionTok{b}\OperatorTok{,} \FunctionTok{c}\OperatorTok{]}\NormalTok{ SUND}\OperatorTok{[}\FunctionTok{a}\OperatorTok{,} \FunctionTok{b}\OperatorTok{,} \FunctionTok{c}\OperatorTok{]} 
\NormalTok{SUNSimplify}\OperatorTok{[}\SpecialCharTok{\%}\OperatorTok{]}
\end{Highlighting}
\end{Shaded}

\begin{dmath*}\breakingcomma
\left(d^{abc}\right)^2
\end{dmath*}

\begin{dmath*}\breakingcomma
-2 \left(4-C_A^2\right) C_F
\end{dmath*}

The color factors \(C_A\) and \(C_F\) are reconstructed from \(N_c\)
using heuristics. The reconstruction can be disabled by setting the
option \texttt{SUNNToCACF} to \texttt{False}

\begin{Shaded}
\begin{Highlighting}[]
\NormalTok{SUNSimplify}\OperatorTok{[}\NormalTok{SUNT}\OperatorTok{[}\FunctionTok{b}\OperatorTok{,} \FunctionTok{d}\OperatorTok{,} \FunctionTok{a}\OperatorTok{,} \FunctionTok{b}\OperatorTok{,} \FunctionTok{d}\OperatorTok{],}\NormalTok{ SUNNToCACF }\OtherTok{{-}\textgreater{}} \ConstantTok{False}\OperatorTok{]}
\end{Highlighting}
\end{Shaded}

\begin{dmath*}\breakingcomma
\frac{\left(N^2+1\right) T^a}{4 N^2}
\end{dmath*}

The color traces are not evaluated by default. The evaluation can be
forced either by applying \texttt{SUNSimplify} or setting the option
\texttt{SUNTraceEvaluate} to \texttt{True}

\begin{Shaded}
\begin{Highlighting}[]
\NormalTok{SUNTrace}\OperatorTok{[}\NormalTok{SUNT}\OperatorTok{[}\FunctionTok{a}\OperatorTok{,} \FunctionTok{b}\OperatorTok{]]}
\end{Highlighting}
\end{Shaded}

\begin{dmath*}\breakingcomma
\text{tr}\left(T^a.T^b\right)
\end{dmath*}

\begin{Shaded}
\begin{Highlighting}[]
\NormalTok{SUNTrace}\OperatorTok{[}\NormalTok{SUNT}\OperatorTok{[}\FunctionTok{a}\OperatorTok{,} \FunctionTok{b}\OperatorTok{,} \FunctionTok{b}\OperatorTok{,} \FunctionTok{a}\OperatorTok{]]}
\end{Highlighting}
\end{Shaded}

\begin{dmath*}\breakingcomma
\text{tr}\left(T^a.T^b.T^b.T^a\right)
\end{dmath*}

\begin{Shaded}
\begin{Highlighting}[]
\NormalTok{SUNTrace}\OperatorTok{[}\NormalTok{SUNT}\OperatorTok{[}\FunctionTok{a}\OperatorTok{,} \FunctionTok{b}\OperatorTok{]]} \SpecialCharTok{//}\NormalTok{ SUNSimplify}
\end{Highlighting}
\end{Shaded}

\begin{dmath*}\breakingcomma
\frac{\delta ^{ab}}{2}
\end{dmath*}

\begin{Shaded}
\begin{Highlighting}[]
\NormalTok{SUNTrace}\OperatorTok{[}\NormalTok{SUNT}\OperatorTok{[}\FunctionTok{a}\OperatorTok{,} \FunctionTok{b}\OperatorTok{,} \FunctionTok{b}\OperatorTok{,} \FunctionTok{a}\OperatorTok{]]} \SpecialCharTok{//}\NormalTok{ SUNSimplify}
\end{Highlighting}
\end{Shaded}

\begin{dmath*}\breakingcomma
C_A C_F^2
\end{dmath*}

\begin{Shaded}
\begin{Highlighting}[]
\NormalTok{SUNTrace}\OperatorTok{[}\NormalTok{SUNT}\OperatorTok{[}\FunctionTok{a}\OperatorTok{,} \FunctionTok{b}\OperatorTok{],}\NormalTok{ SUNTraceEvaluate }\OtherTok{{-}\textgreater{}} \ConstantTok{True}\OperatorTok{]}
\end{Highlighting}
\end{Shaded}

\begin{dmath*}\breakingcomma
\frac{\delta ^{ab}}{2}
\end{dmath*}

Use \texttt{SUNTF} to get color matrices with explicit fundamental
indices

\begin{Shaded}
\begin{Highlighting}[]
\NormalTok{SUNTF}\OperatorTok{[\{}\FunctionTok{a}\OperatorTok{,} \FunctionTok{b}\OperatorTok{,} \FunctionTok{c}\OperatorTok{\},} \FunctionTok{i}\OperatorTok{,} \FunctionTok{j}\OperatorTok{]}\NormalTok{ SUNTrace}\OperatorTok{[}\NormalTok{SUNT}\OperatorTok{[}\FunctionTok{b}\OperatorTok{,} \FunctionTok{a}\OperatorTok{]]}
\SpecialCharTok{\%} \SpecialCharTok{//}\NormalTok{ SUNSimplify}
\end{Highlighting}
\end{Shaded}

\begin{dmath*}\breakingcomma
\text{tr}\left(T^b.T^a\right) \left(T^aT^bT^c\right){}_{ij}
\end{dmath*}

\begin{dmath*}\breakingcomma
\frac{1}{2} C_F T_{ij}^c
\end{dmath*}

\begin{Shaded}
\begin{Highlighting}[]
\NormalTok{SUNDelta}\OperatorTok{[}\FunctionTok{a}\OperatorTok{,} \FunctionTok{b}\OperatorTok{]}\NormalTok{ SUNTF}\OperatorTok{[\{}\FunctionTok{a}\OperatorTok{,} \FunctionTok{b}\OperatorTok{\},} \FunctionTok{i}\OperatorTok{,} \FunctionTok{j}\OperatorTok{]}\NormalTok{ SUNTF}\OperatorTok{[\{}\FunctionTok{c}\OperatorTok{,} \FunctionTok{d}\OperatorTok{\},} \FunctionTok{j}\OperatorTok{,} \FunctionTok{i}\OperatorTok{]}
\NormalTok{SUNSimplify}\OperatorTok{[}\SpecialCharTok{\%}\OperatorTok{]}
\end{Highlighting}
\end{Shaded}

\begin{dmath*}\breakingcomma
\delta ^{ab} \left(T^aT^b\right){}_{ij} \left(T^cT^d\right){}_{ji}
\end{dmath*}

\begin{dmath*}\breakingcomma
\frac{1}{2} C_F \delta ^{cd}
\end{dmath*}

Color traces with more than 3 distinct matrices are not evaluated by
default (assuming that no other simplifications are possible). The
evaluation can be forced using the option \texttt{SUNTraceEvaluate} set
to \texttt{True}

\begin{Shaded}
\begin{Highlighting}[]
\NormalTok{SUNTrace}\OperatorTok{[}\NormalTok{SUNT}\OperatorTok{[}\FunctionTok{a}\OperatorTok{,} \FunctionTok{b}\OperatorTok{,} \FunctionTok{c}\OperatorTok{,} \FunctionTok{d}\OperatorTok{]]} \SpecialCharTok{//}\NormalTok{ SUNSimplify}
\end{Highlighting}
\end{Shaded}

\begin{dmath*}\breakingcomma
\text{tr}\left(T^a.T^b.T^c.T^d\right)
\end{dmath*}

\begin{Shaded}
\begin{Highlighting}[]
\NormalTok{SUNTrace}\OperatorTok{[}\NormalTok{SUNT}\OperatorTok{[}\FunctionTok{a}\OperatorTok{,} \FunctionTok{b}\OperatorTok{,} \FunctionTok{c}\OperatorTok{,} \FunctionTok{d}\OperatorTok{]]} \SpecialCharTok{//}\NormalTok{ SUNSimplify}\OperatorTok{[}\NormalTok{\#}\OperatorTok{,}\NormalTok{ SUNTraceEvaluate }\OtherTok{{-}\textgreater{}} \ConstantTok{True}\OperatorTok{]}\NormalTok{ \&}
\end{Highlighting}
\end{Shaded}

\begin{dmath*}\breakingcomma
\frac{1}{4} \delta ^{ad} \left(C_A-2 C_F\right) \delta ^{bc}-\frac{1}{4} \delta ^{ac} \left(C_A-2 C_F\right) \delta ^{bd}+\frac{1}{4} \delta ^{ab} \left(C_A-2 C_F\right) \delta ^{cd}-\frac{1}{8} i f^{ad\text{FCGV}(\text{sun1521})} d^{bc\text{FCGV}(\text{sun1521})}+\frac{1}{8} i d^{ad\text{FCGV}(\text{sun1521})} f^{bc\text{FCGV}(\text{sun1521})}+\frac{1}{8} d^{ad\text{FCGV}(\text{sun1521})} d^{bc\text{FCGV}(\text{sun1521})}-\frac{1}{8} d^{bd\text{FCGV}(\text{sun1521})} d^{ac\text{FCGV}(\text{sun1521})}+\frac{1}{8} d^{cd\text{FCGV}(\text{sun1521})} d^{ab\text{FCGV}(\text{sun1521})}
\end{dmath*}

One can automatically rename dummy indices using the
\texttt{SUNIndexNames} option

\begin{Shaded}
\begin{Highlighting}[]
\NormalTok{SUNTrace}\OperatorTok{[}\NormalTok{SUNT}\OperatorTok{[}\FunctionTok{a}\OperatorTok{,} \FunctionTok{b}\OperatorTok{,} \FunctionTok{c}\OperatorTok{,} \FunctionTok{d}\OperatorTok{]]} \SpecialCharTok{//}\NormalTok{ SUNSimplify}\OperatorTok{[}\NormalTok{\#}\OperatorTok{,}\NormalTok{ SUNTraceEvaluate }\OtherTok{{-}\textgreater{}} \ConstantTok{True}\OperatorTok{,}\NormalTok{ SUNIndexNames }\OtherTok{{-}\textgreater{}} \OperatorTok{\{}\FunctionTok{j}\OperatorTok{\}]}\NormalTok{ \&}
\end{Highlighting}
\end{Shaded}

\begin{dmath*}\breakingcomma
\frac{1}{4} \delta ^{ad} \left(C_A-2 C_F\right) \delta ^{bc}-\frac{1}{4} \delta ^{ac} \left(C_A-2 C_F\right) \delta ^{bd}+\frac{1}{4} \delta ^{ab} \left(C_A-2 C_F\right) \delta ^{cd}-\frac{1}{8} i f^{adj} d^{bcj}+\frac{1}{8} i d^{adj} f^{bcj}+\frac{1}{8} d^{adj} d^{bcj}-\frac{1}{8} d^{bdj} d^{acj}+\frac{1}{8} d^{cdj} d^{abj}
\end{dmath*}
\end{document}
