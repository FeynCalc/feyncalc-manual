% !TeX program = pdflatex
% !TeX root = Combine.tex

\documentclass[../FeynCalcManual.tex]{subfiles}
\begin{document}
\hypertarget{combine}{
\section{Combine}\label{combine}\index{Combine}}

\texttt{Combine[\allowbreak{}expr]} puts terms in a sum over a common
denominator and cancels factors in the result. \texttt{Combine} is
similar to \texttt{Together}, but accepts the option \texttt{Expanding}
and works usually better than \texttt{Together} for polynomials
involving rationals with sums in the denominator.

\subsection{See also}

\hyperlink{toc}{Overview}, \hyperlink{factor2}{Factor2}.

\subsection{Examples}

\begin{Shaded}
\begin{Highlighting}[]
\NormalTok{Combine}\OperatorTok{[}\NormalTok{((}\FunctionTok{a} \SpecialCharTok{{-}} \FunctionTok{b}\NormalTok{) (}\FunctionTok{c} \SpecialCharTok{{-}} \FunctionTok{d}\NormalTok{))}\SpecialCharTok{/}\FunctionTok{e} \SpecialCharTok{+} \FunctionTok{g}\OperatorTok{]}
\end{Highlighting}
\end{Shaded}

\begin{dmath*}\breakingcomma
\frac{(a-b) (c-d)+e g}{e}
\end{dmath*}

Here the result from \texttt{Together} where the numerator is
automatically expanded.

\begin{Shaded}
\begin{Highlighting}[]
\FunctionTok{Together}\OperatorTok{[}\NormalTok{((}\FunctionTok{a} \SpecialCharTok{{-}} \FunctionTok{b}\NormalTok{) (}\FunctionTok{c} \SpecialCharTok{{-}} \FunctionTok{d}\NormalTok{))}\SpecialCharTok{/}\FunctionTok{e} \SpecialCharTok{+} \FunctionTok{g}\OperatorTok{]}
\end{Highlighting}
\end{Shaded}

\begin{dmath*}\breakingcomma
\frac{a c-a d-b c+b d+e g}{e}
\end{dmath*}

If the option \texttt{Expanding} is set to \texttt{True}, the result of
\texttt{Combine} is the same as \texttt{Together}, but uses a slightly
different algorithm.

\begin{Shaded}
\begin{Highlighting}[]
\NormalTok{Combine}\OperatorTok{[}\NormalTok{((}\FunctionTok{a} \SpecialCharTok{{-}} \FunctionTok{b}\NormalTok{) (}\FunctionTok{c} \SpecialCharTok{{-}} \FunctionTok{d}\NormalTok{))}\SpecialCharTok{/}\FunctionTok{e} \SpecialCharTok{+} \FunctionTok{g}\OperatorTok{,}\NormalTok{ Expanding }\OtherTok{{-}\textgreater{}} \ConstantTok{True}\OperatorTok{]}
\end{Highlighting}
\end{Shaded}

\begin{dmath*}\breakingcomma
\frac{a c-a d-b c+b d+e g}{e}
\end{dmath*}
\end{document}
