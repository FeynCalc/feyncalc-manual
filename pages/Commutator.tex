% !TeX program = pdflatex
% !TeX root = Commutator.tex

\documentclass[../FeynCalcManual.tex]{subfiles}
\begin{document}
\hypertarget{commutator}{%
\section{Commutator}\label{commutator}}

\texttt{Commutator[\allowbreak{}x,\ \allowbreak{}y] = c} defines the
commutator between the (non-commuting) objects \texttt{x} and
\texttt{y}.

\subsection{See also}

\hyperlink{toc}{Overview}, \hyperlink{anticommutator}{AntiCommutator},
\hyperlink{commutatorexplicit}{CommutatorExplicit},
\hyperlink{declarenoncommutative}{DeclareNonCommutative},
\hyperlink{dotsimplify}{DotSimplify}.

\subsection{Examples}

\begin{Shaded}
\begin{Highlighting}[]
\NormalTok{DeclareNonCommutative}\OperatorTok{[}\FunctionTok{a}\OperatorTok{,} \FunctionTok{b}\OperatorTok{,} \FunctionTok{c}\OperatorTok{,} \FunctionTok{d}\OperatorTok{]}
\end{Highlighting}
\end{Shaded}

\begin{Shaded}
\begin{Highlighting}[]
\NormalTok{Commutator}\OperatorTok{[}\FunctionTok{a}\OperatorTok{,} \FunctionTok{b}\OperatorTok{]} 
 
\NormalTok{CommutatorExplicit}\OperatorTok{[}\SpecialCharTok{\%}\OperatorTok{]}
\end{Highlighting}
\end{Shaded}

\begin{dmath*}\breakingcomma
[a,b]
\end{dmath*}

\begin{dmath*}\breakingcomma
a.b-b.a
\end{dmath*}

\begin{Shaded}
\begin{Highlighting}[]
\NormalTok{DotSimplify}\OperatorTok{[}\NormalTok{Commutator}\OperatorTok{[}\FunctionTok{a} \SpecialCharTok{+} \FunctionTok{b}\OperatorTok{,} \FunctionTok{c} \SpecialCharTok{+} \FunctionTok{d}\OperatorTok{]]} 
 
\NormalTok{UnDeclareNonCommutative}\OperatorTok{[}\FunctionTok{a}\OperatorTok{,} \FunctionTok{b}\OperatorTok{,} \FunctionTok{c}\OperatorTok{,} \FunctionTok{d}\OperatorTok{]}
\end{Highlighting}
\end{Shaded}

\begin{dmath*}\breakingcomma
a.c-c.a+a.d-d.a+b.c-c.b+b.d-d.b
\end{dmath*}

Verify the Jacobi identity.

\begin{Shaded}
\begin{Highlighting}[]
\SpecialCharTok{\textbackslash{}}\OperatorTok{[}\NormalTok{Chi}\OperatorTok{]} \ExtensionTok{=}\NormalTok{ Commutator; DeclareNonCommutative}\OperatorTok{[}\FunctionTok{x}\OperatorTok{,} \FunctionTok{y}\OperatorTok{,} \FunctionTok{z}\OperatorTok{]}\NormalTok{;}
\end{Highlighting}
\end{Shaded}

\begin{Shaded}
\begin{Highlighting}[]
\SpecialCharTok{\textbackslash{}}\OperatorTok{[}\NormalTok{Chi}\OperatorTok{][}\FunctionTok{x}\OperatorTok{,} \SpecialCharTok{\textbackslash{}}\OperatorTok{[}\NormalTok{Chi}\OperatorTok{][}\FunctionTok{y}\OperatorTok{,} \FunctionTok{z}\OperatorTok{]]} \SpecialCharTok{+} \SpecialCharTok{\textbackslash{}}\OperatorTok{[}\NormalTok{Chi}\OperatorTok{][}\FunctionTok{y}\OperatorTok{,} \SpecialCharTok{\textbackslash{}}\OperatorTok{[}\NormalTok{Chi}\OperatorTok{][}\FunctionTok{z}\OperatorTok{,} \FunctionTok{x}\OperatorTok{]]} \SpecialCharTok{+} \SpecialCharTok{\textbackslash{}}\OperatorTok{[}\NormalTok{Chi}\OperatorTok{][}\FunctionTok{z}\OperatorTok{,} \SpecialCharTok{\textbackslash{}}\OperatorTok{[}\NormalTok{Chi}\OperatorTok{][}\FunctionTok{x}\OperatorTok{,} \FunctionTok{y}\OperatorTok{]]} 
 
\NormalTok{DotSimplify}\OperatorTok{[}\SpecialCharTok{\%}\OperatorTok{]}
\end{Highlighting}
\end{Shaded}

\begin{dmath*}\breakingcomma
[x,[y,z]]+[y,[z,x]]+[z,[x,y]]
\end{dmath*}

\begin{dmath*}\breakingcomma
0
\end{dmath*}

\begin{Shaded}
\begin{Highlighting}[]
\FunctionTok{Clear}\OperatorTok{[}\SpecialCharTok{\textbackslash{}}\OperatorTok{[}\NormalTok{Chi}\OperatorTok{]]} 
 
\NormalTok{UnDeclareNonCommutative}\OperatorTok{[}\FunctionTok{x}\OperatorTok{,} \FunctionTok{y}\OperatorTok{,} \FunctionTok{z}\OperatorTok{]}
\end{Highlighting}
\end{Shaded}

\end{document}
