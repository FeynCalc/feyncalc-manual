% !TeX program = pdflatex
% !TeX root = CommutatorExplicit.tex

\documentclass[../FeynCalcManual.tex]{subfiles}
\begin{document}
\hypertarget{commutatorexplicit}{%
\section{CommutatorExplicit}\label{commutatorexplicit}}

\texttt{CommutatorExplicit[\allowbreak{}exp]} substitutes any
\texttt{Commutator} and \texttt{AntiCommutator} in \texttt{exp} by their
definitions.

\subsection{See also}

\hyperlink{toc}{Overview}, \hyperlink{calc}{Calc},
\hyperlink{dotsimplify}{DotSimplify}.

\subsection{Examples}

\begin{Shaded}
\begin{Highlighting}[]
\NormalTok{DeclareNonCommutative}\OperatorTok{[}\FunctionTok{a}\OperatorTok{,} \FunctionTok{b}\OperatorTok{,} \FunctionTok{c}\OperatorTok{,} \FunctionTok{d}\OperatorTok{]}
\end{Highlighting}
\end{Shaded}

\begin{Shaded}
\begin{Highlighting}[]
\NormalTok{Commutator}\OperatorTok{[}\FunctionTok{a}\OperatorTok{,} \FunctionTok{b}\OperatorTok{]} 
 
\NormalTok{CommutatorExplicit}\OperatorTok{[}\SpecialCharTok{\%}\OperatorTok{]}
\end{Highlighting}
\end{Shaded}

\begin{dmath*}\breakingcomma
[a,b]
\end{dmath*}

\begin{dmath*}\breakingcomma
a.b-b.a
\end{dmath*}

\begin{Shaded}
\begin{Highlighting}[]
\NormalTok{AntiCommutator}\OperatorTok{[}\FunctionTok{a} \SpecialCharTok{{-}} \FunctionTok{c}\OperatorTok{,} \FunctionTok{b} \SpecialCharTok{{-}} \FunctionTok{d}\OperatorTok{]} 
 
\NormalTok{CommutatorExplicit}\OperatorTok{[}\SpecialCharTok{\%}\OperatorTok{]}
\end{Highlighting}
\end{Shaded}

\begin{dmath*}\breakingcomma
\{a-c,\medspace b-d\}
\end{dmath*}

\begin{dmath*}\breakingcomma
(a-c).(b-d)+(b-d).(a-c)
\end{dmath*}

\begin{Shaded}
\begin{Highlighting}[]
\NormalTok{CommutatorExplicit}\OperatorTok{[}\NormalTok{AntiCommutator}\OperatorTok{[}\FunctionTok{a} \SpecialCharTok{{-}} \FunctionTok{c}\OperatorTok{,} \FunctionTok{b} \SpecialCharTok{{-}} \FunctionTok{d}\OperatorTok{]]} \SpecialCharTok{//}\NormalTok{ DotSimplify}
\end{Highlighting}
\end{Shaded}

\begin{dmath*}\breakingcomma
a.b+b.a-a.d-d.a-b.c-c.b+c.d+d.c
\end{dmath*}

\begin{Shaded}
\begin{Highlighting}[]
\NormalTok{UnDeclareNonCommutative}\OperatorTok{[}\FunctionTok{a}\OperatorTok{,} \FunctionTok{b}\OperatorTok{,} \FunctionTok{c}\OperatorTok{,} \FunctionTok{d}\OperatorTok{]}
\end{Highlighting}
\end{Shaded}

\end{document}
