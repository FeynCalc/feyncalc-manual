% !TeX program = pdflatex
% !TeX root = Complement1.tex

\documentclass[../FeynCalcManual.tex]{subfiles}
\begin{document}
\hypertarget{complement1}{
\section{Complement1}\label{complement1}\index{Complement1}}

\texttt{Complement1[\allowbreak{}l1,\ \allowbreak{}l2]} where
\texttt{l1} and \texttt{l2} are lists returns a list of elements from
\texttt{l1} not in\texttt{l2}. Multiple occurrences of an element in
\texttt{l1} are kept and multiple occurrences of an element in
\texttt{l2} are dropped if present in \texttt{l1}.

\subsection{See also}

\hyperlink{toc}{Overview}

\subsection{Examples}

\begin{Shaded}
\begin{Highlighting}[]
\FunctionTok{Complement}\OperatorTok{[\{}\FunctionTok{a}\OperatorTok{,} \FunctionTok{b}\OperatorTok{,} \FunctionTok{c}\OperatorTok{,} \FunctionTok{d}\OperatorTok{,} \FunctionTok{e}\OperatorTok{,} \FunctionTok{f}\OperatorTok{,} \FunctionTok{e}\OperatorTok{\},} \OperatorTok{\{}\FunctionTok{a}\OperatorTok{,} \FunctionTok{b}\OperatorTok{,} \FunctionTok{c}\OperatorTok{,} \FunctionTok{d}\OperatorTok{\}]}
\end{Highlighting}
\end{Shaded}

\begin{dmath*}\breakingcomma
\{e,f\}
\end{dmath*}

\begin{Shaded}
\begin{Highlighting}[]
\NormalTok{Complement1}\OperatorTok{[\{}\FunctionTok{a}\OperatorTok{,} \FunctionTok{b}\OperatorTok{,} \FunctionTok{c}\OperatorTok{,} \FunctionTok{d}\OperatorTok{,} \FunctionTok{e}\OperatorTok{,} \FunctionTok{f}\OperatorTok{,} \FunctionTok{e}\OperatorTok{\},} \OperatorTok{\{}\FunctionTok{a}\OperatorTok{,} \FunctionTok{b}\OperatorTok{,} \FunctionTok{c}\OperatorTok{,} \FunctionTok{d}\OperatorTok{\}]}
\end{Highlighting}
\end{Shaded}

\begin{dmath*}\breakingcomma
\{e,f,e\}
\end{dmath*}
\end{document}
