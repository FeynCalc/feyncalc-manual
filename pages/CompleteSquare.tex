% !TeX program = pdflatex
% !TeX root = CompleteSquare.tex

\documentclass[../FeynCalcManual.tex]{subfiles}
\begin{document}
\hypertarget{completesquare}{
\section{CompleteSquare}\label{completesquare}\index{CompleteSquare}}

\texttt{CompleteSquarep[\allowbreak{}exp,\ \allowbreak{}x]} completes
the square of a second order polynomial in the momentum x.

\subsection{See also}

\hyperlink{toc}{Overview},
\hyperlink{expandscalarproduct}{ExpandScalarProduct}.

\subsection{Examples}

\begin{Shaded}
\begin{Highlighting}[]
\NormalTok{CompleteSquare}\OperatorTok{[}\DecValTok{4}\NormalTok{ SP}\OperatorTok{[}\FunctionTok{p}\OperatorTok{]} \SpecialCharTok{+}\NormalTok{ SP}\OperatorTok{[}\FunctionTok{b}\OperatorTok{,} \FunctionTok{p}\OperatorTok{]} \SpecialCharTok{+} \FunctionTok{c}\OperatorTok{,} \FunctionTok{p}\OperatorTok{]}
\end{Highlighting}
\end{Shaded}

\begin{dmath*}\breakingcomma
4 (\frac{\overline{b}}{8}+\overline{p})^2-\frac{\overline{b}^2}{16}+c
\end{dmath*}

\begin{Shaded}
\begin{Highlighting}[]
\NormalTok{CompleteSquare}\OperatorTok{[}\DecValTok{4}\NormalTok{ SP}\OperatorTok{[}\FunctionTok{p}\OperatorTok{]} \SpecialCharTok{+}\NormalTok{ SP}\OperatorTok{[}\FunctionTok{b}\OperatorTok{,} \FunctionTok{p}\OperatorTok{]} \SpecialCharTok{+} \FunctionTok{c}\OperatorTok{,} \FunctionTok{p}\OperatorTok{,} \FunctionTok{q}\OperatorTok{]}
\end{Highlighting}
\end{Shaded}

\begin{dmath*}\breakingcomma
\left\{-\frac{\overline{b}^2}{16}+4 \overline{q}^2+c,\overline{q}\to \frac{\overline{b}}{8}+\overline{p}\right\}
\end{dmath*}

\begin{Shaded}
\begin{Highlighting}[]
\NormalTok{ex1 }\ExtensionTok{=} \DecValTok{5}\NormalTok{ SP}\OperatorTok{[}\DecValTok{2} \FunctionTok{p} \SpecialCharTok{+} \DecValTok{3} \FunctionTok{r}\OperatorTok{,} \FunctionTok{p} \SpecialCharTok{+} \FunctionTok{r}\OperatorTok{]}
\end{Highlighting}
\end{Shaded}

\begin{dmath*}\breakingcomma
5 \left((\overline{p}+\overline{r})\cdot (2 \overline{p}+3 \overline{r})\right)
\end{dmath*}

\begin{Shaded}
\begin{Highlighting}[]
\NormalTok{ex2 }\ExtensionTok{=}\NormalTok{ CompleteSquare}\OperatorTok{[}\NormalTok{ex1}\OperatorTok{,} \FunctionTok{p}\OperatorTok{]}
\end{Highlighting}
\end{Shaded}

\begin{dmath*}\breakingcomma
10 (\overline{p}+\frac{5 \overline{r}}{4})^2-\frac{5 \overline{r}^2}{8}
\end{dmath*}

\begin{Shaded}
\begin{Highlighting}[]
\NormalTok{ex1 }\SpecialCharTok{{-}}\NormalTok{ ex2 }\SpecialCharTok{//}\NormalTok{ ScalarProductExpand }\SpecialCharTok{//} \FunctionTok{Expand}
\end{Highlighting}
\end{Shaded}

\begin{dmath*}\breakingcomma
0
\end{dmath*}

\begin{Shaded}
\begin{Highlighting}[]
\NormalTok{CompleteSquare}\OperatorTok{[}\DecValTok{5}\NormalTok{ SP}\OperatorTok{[}\DecValTok{2} \FunctionTok{p} \SpecialCharTok{+} \DecValTok{3} \FunctionTok{r}\OperatorTok{,} \FunctionTok{p} \SpecialCharTok{+} \FunctionTok{r}\OperatorTok{],} \FunctionTok{p}\OperatorTok{,} \FunctionTok{q}\OperatorTok{]}
\end{Highlighting}
\end{Shaded}

\begin{dmath*}\breakingcomma
\left\{10 \overline{q}^2-\frac{5 \overline{r}^2}{8},\overline{q}\to \overline{p}+\frac{5 \overline{r}}{4}\right\}
\end{dmath*}

\begin{Shaded}
\begin{Highlighting}[]
\NormalTok{SPD}\OperatorTok{[}\FunctionTok{a}\OperatorTok{]} \SpecialCharTok{+} \DecValTok{2}\NormalTok{ SPD}\OperatorTok{[}\FunctionTok{a}\OperatorTok{,} \FunctionTok{b}\OperatorTok{]} 
 
\NormalTok{ex }\ExtensionTok{=}\NormalTok{ CompleteSquare}\OperatorTok{[}\SpecialCharTok{\%}\OperatorTok{,} \FunctionTok{a}\OperatorTok{]}
\end{Highlighting}
\end{Shaded}

\begin{dmath*}\breakingcomma
2 (a\cdot b)+a^2
\end{dmath*}

\begin{dmath*}\breakingcomma
(a+b)^2-b^2
\end{dmath*}

\begin{Shaded}
\begin{Highlighting}[]
\NormalTok{ex }\SpecialCharTok{//} \FunctionTok{StandardForm}

\CommentTok{(*{-}Pair[Momentum[b, D], Momentum[b, D]] + Pair[Momentum[a, D] + Momentum[b, D], Momentum[a, D] + Momentum[b, D]]*)}
\end{Highlighting}
\end{Shaded}

\end{document}
