% !TeX program = pdflatex
% !TeX root = Contract.tex

\documentclass[../FeynCalcManual.tex]{subfiles}
\begin{document}
\hypertarget{contract}{
\section{Contract}\label{contract}\index{Contract}}

\texttt{Contract[\allowbreak{}expr]} contracts pairs of Lorentz or
Cartesian indices of metric tensors, vectors and (depending on the value
of the option \texttt{EpsContract}) of Levi-Civita tensors in
\texttt{expr}.

For contractions of Dirac matrices with each other use
\texttt{DiracSimplify}.

\texttt{Contract[\allowbreak{}exp1,\ \allowbreak{}exp2]} contracts
\texttt{(exp1*exp2)}, where \texttt{exp1} and \texttt{exp2} may be
larger products of sums of metric tensors and 4-vectors. This can be
also useful when evaluating polarization sums, where \texttt{exp2}
should be the product (or expanded sum) of the polarization sums for the
vector bosons.

\subsection{See also}

\hyperlink{toc}{Overview}, \hyperlink{pair}{Pair},
\hyperlink{cartesianpair}{CartesianPair},
\hyperlink{diracsimplify}{DiracSimplify},
\hyperlink{momentumcombine}{MomentumCombine}.

\subsection{Examples}

\begin{Shaded}
\begin{Highlighting}[]
\NormalTok{MT}\OperatorTok{[}\SpecialCharTok{\textbackslash{}}\OperatorTok{[}\NormalTok{Mu}\OperatorTok{],} \SpecialCharTok{\textbackslash{}}\OperatorTok{[}\NormalTok{Nu}\OperatorTok{]]}\NormalTok{ FV}\OperatorTok{[}\FunctionTok{p}\OperatorTok{,} \SpecialCharTok{\textbackslash{}}\OperatorTok{[}\NormalTok{Mu}\OperatorTok{]]} 
 
\NormalTok{Contract}\OperatorTok{[}\SpecialCharTok{\%}\OperatorTok{]}
\end{Highlighting}
\end{Shaded}

\begin{dmath*}\breakingcomma
\overline{p}^{\mu } \bar{g}^{\mu \nu }
\end{dmath*}

\begin{dmath*}\breakingcomma
\overline{p}^{\nu }
\end{dmath*}

\begin{Shaded}
\begin{Highlighting}[]
\NormalTok{FV}\OperatorTok{[}\FunctionTok{p}\OperatorTok{,} \SpecialCharTok{\textbackslash{}}\OperatorTok{[}\NormalTok{Mu}\OperatorTok{]]}\NormalTok{ GA}\OperatorTok{[}\SpecialCharTok{\textbackslash{}}\OperatorTok{[}\NormalTok{Mu}\OperatorTok{]]} 
 
\NormalTok{Contract}\OperatorTok{[}\SpecialCharTok{\%}\OperatorTok{]}
\end{Highlighting}
\end{Shaded}

\begin{dmath*}\breakingcomma
\bar{\gamma }^{\mu } \overline{p}^{\mu }
\end{dmath*}

\begin{dmath*}\breakingcomma
\bar{\gamma }\cdot \overline{p}
\end{dmath*}

The default dimension for a metric tensor is 4.

\begin{Shaded}
\begin{Highlighting}[]
\NormalTok{MT}\OperatorTok{[}\SpecialCharTok{\textbackslash{}}\OperatorTok{[}\NormalTok{Mu}\OperatorTok{],} \SpecialCharTok{\textbackslash{}}\OperatorTok{[}\NormalTok{Mu}\OperatorTok{]]} 
 
\NormalTok{Contract}\OperatorTok{[}\SpecialCharTok{\%}\OperatorTok{]}
\end{Highlighting}
\end{Shaded}

\begin{dmath*}\breakingcomma
\bar{g}^{\mu \mu }
\end{dmath*}

\begin{dmath*}\breakingcomma
4
\end{dmath*}

A quick way to enter \(D\)-dimensional metric tensors is given by
\texttt{MTD}.

\begin{Shaded}
\begin{Highlighting}[]
\NormalTok{MTD}\OperatorTok{[}\SpecialCharTok{\textbackslash{}}\OperatorTok{[}\NormalTok{Mu}\OperatorTok{],} \SpecialCharTok{\textbackslash{}}\OperatorTok{[}\NormalTok{Nu}\OperatorTok{]]}\NormalTok{  MTD}\OperatorTok{[}\SpecialCharTok{\textbackslash{}}\OperatorTok{[}\NormalTok{Mu}\OperatorTok{],} \SpecialCharTok{\textbackslash{}}\OperatorTok{[}\NormalTok{Nu}\OperatorTok{]]} 
 
\NormalTok{Contract}\OperatorTok{[}\SpecialCharTok{\%}\OperatorTok{]}
\end{Highlighting}
\end{Shaded}

\begin{dmath*}\breakingcomma
(g^{\mu \nu})^2
\end{dmath*}

\begin{dmath*}\breakingcomma
D
\end{dmath*}

\begin{Shaded}
\begin{Highlighting}[]
\NormalTok{FV}\OperatorTok{[}\FunctionTok{p}\OperatorTok{,} \SpecialCharTok{\textbackslash{}}\OperatorTok{[}\NormalTok{Mu}\OperatorTok{]]}\NormalTok{ FV}\OperatorTok{[}\FunctionTok{q}\OperatorTok{,} \SpecialCharTok{\textbackslash{}}\OperatorTok{[}\NormalTok{Mu}\OperatorTok{]]} 
 
\NormalTok{Contract}\OperatorTok{[}\SpecialCharTok{\%} \OperatorTok{]}
\end{Highlighting}
\end{Shaded}

\begin{dmath*}\breakingcomma
\overline{p}^{\mu } \overline{q}^{\mu }
\end{dmath*}

\begin{dmath*}\breakingcomma
\overline{p}\cdot \overline{q}
\end{dmath*}

\begin{Shaded}
\begin{Highlighting}[]
\NormalTok{FV}\OperatorTok{[}\FunctionTok{p} \SpecialCharTok{{-}} \FunctionTok{q}\OperatorTok{,} \SpecialCharTok{\textbackslash{}}\OperatorTok{[}\NormalTok{Mu}\OperatorTok{]]}\NormalTok{ FV}\OperatorTok{[}\FunctionTok{a} \SpecialCharTok{{-}} \FunctionTok{b}\OperatorTok{,} \SpecialCharTok{\textbackslash{}}\OperatorTok{[}\NormalTok{Mu}\OperatorTok{]]} 
 
\NormalTok{Contract}\OperatorTok{[}\SpecialCharTok{\%}\OperatorTok{]}
\end{Highlighting}
\end{Shaded}

\begin{dmath*}\breakingcomma
\left(\overline{a}-\overline{b}\right)^{\mu } \left(\overline{p}-\overline{q}\right)^{\mu }
\end{dmath*}

\begin{dmath*}\breakingcomma
\overline{a}\cdot \overline{p}-\overline{a}\cdot \overline{q}-\overline{b}\cdot \overline{p}+\overline{b}\cdot \overline{q}
\end{dmath*}

\begin{Shaded}
\begin{Highlighting}[]
\NormalTok{FVD}\OperatorTok{[}\FunctionTok{p} \SpecialCharTok{{-}} \FunctionTok{q}\OperatorTok{,} \SpecialCharTok{\textbackslash{}}\OperatorTok{[}\NormalTok{Nu}\OperatorTok{]]}\NormalTok{ FVD}\OperatorTok{[}\FunctionTok{a} \SpecialCharTok{{-}} \FunctionTok{b}\OperatorTok{,} \SpecialCharTok{\textbackslash{}}\OperatorTok{[}\NormalTok{Nu}\OperatorTok{]]} 
 
\NormalTok{Contract}\OperatorTok{[}\SpecialCharTok{\%}\OperatorTok{]}
\end{Highlighting}
\end{Shaded}

\begin{dmath*}\breakingcomma
(a-b)^{\nu } (p-q)^{\nu }
\end{dmath*}

\begin{dmath*}\breakingcomma
a\cdot p-a\cdot q-b\cdot p+b\cdot q
\end{dmath*}

\begin{Shaded}
\begin{Highlighting}[]
\NormalTok{LC}\OperatorTok{[}\SpecialCharTok{\textbackslash{}}\OperatorTok{[}\NormalTok{Mu}\OperatorTok{],} \SpecialCharTok{\textbackslash{}}\OperatorTok{[}\NormalTok{Nu}\OperatorTok{],} \SpecialCharTok{\textbackslash{}}\OperatorTok{[}\NormalTok{Alpha}\OperatorTok{],} \SpecialCharTok{\textbackslash{}}\OperatorTok{[}\NormalTok{Sigma}\OperatorTok{]]}\NormalTok{ FV}\OperatorTok{[}\FunctionTok{p}\OperatorTok{,} \SpecialCharTok{\textbackslash{}}\OperatorTok{[}\NormalTok{Sigma}\OperatorTok{]]} 
 
\NormalTok{Contract}\OperatorTok{[}\SpecialCharTok{\%}\OperatorTok{]}
\end{Highlighting}
\end{Shaded}

\begin{dmath*}\breakingcomma
\overline{p}^{\sigma } \bar{\epsilon }^{\mu \nu \alpha \sigma }
\end{dmath*}

\begin{dmath*}\breakingcomma
\bar{\epsilon }^{\alpha \mu \nu \overline{p}}
\end{dmath*}

\begin{Shaded}
\begin{Highlighting}[]
\NormalTok{LC}\OperatorTok{[}\SpecialCharTok{\textbackslash{}}\OperatorTok{[}\NormalTok{Mu}\OperatorTok{],} \SpecialCharTok{\textbackslash{}}\OperatorTok{[}\NormalTok{Nu}\OperatorTok{],} \SpecialCharTok{\textbackslash{}}\OperatorTok{[}\NormalTok{Alpha}\OperatorTok{],} \SpecialCharTok{\textbackslash{}}\OperatorTok{[}\FunctionTok{Beta}\OperatorTok{]]}\NormalTok{ LC}\OperatorTok{[}\SpecialCharTok{\textbackslash{}}\OperatorTok{[}\NormalTok{Mu}\OperatorTok{],} \SpecialCharTok{\textbackslash{}}\OperatorTok{[}\NormalTok{Nu}\OperatorTok{],} \SpecialCharTok{\textbackslash{}}\OperatorTok{[}\NormalTok{Alpha}\OperatorTok{],} \SpecialCharTok{\textbackslash{}}\OperatorTok{[}\NormalTok{Sigma}\OperatorTok{]]} 
 
\NormalTok{Contract}\OperatorTok{[}\SpecialCharTok{\%}\OperatorTok{]}
\end{Highlighting}
\end{Shaded}

\begin{dmath*}\breakingcomma
\bar{\epsilon }^{\mu \nu \alpha \beta } \bar{\epsilon }^{\mu \nu \alpha \sigma }
\end{dmath*}

\begin{dmath*}\breakingcomma
-6 \bar{g}^{\beta \sigma }
\end{dmath*}

\begin{Shaded}
\begin{Highlighting}[]
\NormalTok{LCD}\OperatorTok{[}\SpecialCharTok{\textbackslash{}}\OperatorTok{[}\NormalTok{Mu}\OperatorTok{],} \SpecialCharTok{\textbackslash{}}\OperatorTok{[}\NormalTok{Nu}\OperatorTok{],} \SpecialCharTok{\textbackslash{}}\OperatorTok{[}\NormalTok{Alpha}\OperatorTok{],} \SpecialCharTok{\textbackslash{}}\OperatorTok{[}\FunctionTok{Beta}\OperatorTok{]]}\NormalTok{ LCD}\OperatorTok{[}\SpecialCharTok{\textbackslash{}}\OperatorTok{[}\NormalTok{Mu}\OperatorTok{],} \SpecialCharTok{\textbackslash{}}\OperatorTok{[}\NormalTok{Nu}\OperatorTok{],} \SpecialCharTok{\textbackslash{}}\OperatorTok{[}\NormalTok{Alpha}\OperatorTok{],} \SpecialCharTok{\textbackslash{}}\OperatorTok{[}\NormalTok{Sigma}\OperatorTok{]]} 
 
\NormalTok{Contract}\OperatorTok{[}\SpecialCharTok{\%}\OperatorTok{]} \SpecialCharTok{//}\NormalTok{ Factor2}
\end{Highlighting}
\end{Shaded}

\begin{dmath*}\breakingcomma
\overset{\text{}}{\epsilon }^{\mu \nu \alpha \beta } \overset{\text{}}{\epsilon }^{\mu \nu \alpha \sigma }
\end{dmath*}

\begin{dmath*}\breakingcomma
(1-D) (2-D) (3-D) g^{\beta \sigma }
\end{dmath*}

Contractions of Cartesian tensors are also possible. They can live in
\(3\), \(D-1\) or \(D-4\) dimensions.

\begin{Shaded}
\begin{Highlighting}[]
\NormalTok{KD}\OperatorTok{[}\FunctionTok{i}\OperatorTok{,} \FunctionTok{j}\OperatorTok{]}\NormalTok{ CV}\OperatorTok{[}\FunctionTok{p}\OperatorTok{,} \FunctionTok{i}\OperatorTok{]} 
 
\NormalTok{Contract}\OperatorTok{[}\SpecialCharTok{\%}\OperatorTok{]}
\end{Highlighting}
\end{Shaded}

\begin{dmath*}\breakingcomma
\overline{p}^i \bar{\delta }^{ij}
\end{dmath*}

\begin{dmath*}\breakingcomma
\overline{p}^j
\end{dmath*}

\begin{Shaded}
\begin{Highlighting}[]
\NormalTok{CV}\OperatorTok{[}\FunctionTok{p}\OperatorTok{,} \FunctionTok{i}\OperatorTok{]}\NormalTok{ CGA}\OperatorTok{[}\FunctionTok{i}\OperatorTok{]} 
 
\NormalTok{Contract}\OperatorTok{[}\SpecialCharTok{\%}\OperatorTok{]}
\end{Highlighting}
\end{Shaded}

\begin{dmath*}\breakingcomma
\overline{\gamma }^i \overline{p}^i
\end{dmath*}

\begin{dmath*}\breakingcomma
\overline{\gamma }\cdot \overline{p}
\end{dmath*}

\begin{Shaded}
\begin{Highlighting}[]
\NormalTok{KD}\OperatorTok{[}\FunctionTok{i}\OperatorTok{,} \FunctionTok{i}\OperatorTok{]} 
 
\NormalTok{Contract}\OperatorTok{[}\SpecialCharTok{\%}\OperatorTok{]}
\end{Highlighting}
\end{Shaded}

\begin{dmath*}\breakingcomma
\bar{\delta }^{ii}
\end{dmath*}

\begin{dmath*}\breakingcomma
3
\end{dmath*}

\begin{Shaded}
\begin{Highlighting}[]
\NormalTok{KD}\OperatorTok{[}\FunctionTok{i}\OperatorTok{,} \FunctionTok{j}\OperatorTok{]}\SpecialCharTok{\^{}}\DecValTok{2} 
 
\NormalTok{Contract}\OperatorTok{[}\SpecialCharTok{\%}\OperatorTok{]}
\end{Highlighting}
\end{Shaded}

\begin{dmath*}\breakingcomma
(\bar{\delta}^{ij})^2
\end{dmath*}

\begin{dmath*}\breakingcomma
3
\end{dmath*}

\begin{Shaded}
\begin{Highlighting}[]
\NormalTok{CV}\OperatorTok{[}\FunctionTok{p} \SpecialCharTok{{-}} \FunctionTok{q}\OperatorTok{,} \FunctionTok{j}\OperatorTok{]}\NormalTok{ CV}\OperatorTok{[}\FunctionTok{a} \SpecialCharTok{{-}} \FunctionTok{b}\OperatorTok{,} \FunctionTok{j}\OperatorTok{]} 
 
\NormalTok{Contract}\OperatorTok{[}\SpecialCharTok{\%}\OperatorTok{]}
\end{Highlighting}
\end{Shaded}

\begin{dmath*}\breakingcomma
\left(\overline{a}-\overline{b}\right)^j \left(\overline{p}-\overline{q}\right)^j
\end{dmath*}

\begin{dmath*}\breakingcomma
(\overline{a}-\overline{b})\cdot (\overline{p}-\overline{q})
\end{dmath*}

\begin{Shaded}
\begin{Highlighting}[]
\NormalTok{CLC}\OperatorTok{[}\FunctionTok{i}\OperatorTok{,} \FunctionTok{j}\OperatorTok{,} \FunctionTok{k}\OperatorTok{]}\NormalTok{ CV}\OperatorTok{[}\FunctionTok{p}\OperatorTok{,} \FunctionTok{k}\OperatorTok{]} 
 
\NormalTok{Contract}\OperatorTok{[}\SpecialCharTok{\%}\OperatorTok{]}
\end{Highlighting}
\end{Shaded}

\begin{dmath*}\breakingcomma
\overline{p}^k \bar{\epsilon }^{ijk}
\end{dmath*}

\begin{dmath*}\breakingcomma
\bar{\epsilon }^{ij\overline{p}}
\end{dmath*}

\begin{Shaded}
\begin{Highlighting}[]
\NormalTok{CLC}\OperatorTok{[}\FunctionTok{i}\OperatorTok{,} \FunctionTok{j}\OperatorTok{,} \FunctionTok{k}\OperatorTok{]}\NormalTok{ CLC}\OperatorTok{[}\FunctionTok{i}\OperatorTok{,} \FunctionTok{j}\OperatorTok{,} \FunctionTok{l}\OperatorTok{]} 
 
\NormalTok{Contract}\OperatorTok{[}\SpecialCharTok{\%}\OperatorTok{]}
\end{Highlighting}
\end{Shaded}

\begin{dmath*}\breakingcomma
\bar{\epsilon }^{ijk} \bar{\epsilon }^{ijl}
\end{dmath*}

\begin{dmath*}\breakingcomma
2 \bar{\delta }^{kl}
\end{dmath*}

\begin{Shaded}
\begin{Highlighting}[]
\NormalTok{CLCD}\OperatorTok{[}\FunctionTok{i}\OperatorTok{,} \FunctionTok{j}\OperatorTok{,} \FunctionTok{k}\OperatorTok{]}\NormalTok{ CLCD}\OperatorTok{[}\FunctionTok{i}\OperatorTok{,} \FunctionTok{j}\OperatorTok{,} \FunctionTok{l}\OperatorTok{]} 
 
\NormalTok{Contract}\OperatorTok{[}\SpecialCharTok{\%}\OperatorTok{]} \SpecialCharTok{//}\NormalTok{ Factor2}
\end{Highlighting}
\end{Shaded}

\begin{dmath*}\breakingcomma
\overset{\text{}}{\epsilon }^{ijk} \overset{\text{}}{\epsilon }^{ijl}
\end{dmath*}

\begin{dmath*}\breakingcomma
(2-D) (3-D) \delta ^{kl}
\end{dmath*}
\end{document}
