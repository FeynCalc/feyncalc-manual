% !TeX program = pdflatex
% !TeX root = Contractions.tex

\documentclass[../FeynCalcManual.tex]{subfiles}
\begin{document}
\hypertarget{contractions}{
\section{Contractions}\label{contractions}\index{Contractions}}

\subsection{See also}

\hyperlink{toc}{Overview}.

\subsection{Simplifications of tensorial
expressions}\label{simplifications-of-tensorial-expressions}

Now that we have some basic understanding of FeynCalc objects, let us do
something with them. Contractions of Lorentz indices are one of the most
essential operations in symbolic QFT calculations. In FeynCalc the
corresponding function is called \texttt{Contract}

\begin{Shaded}
\begin{Highlighting}[]
\NormalTok{FV}\OperatorTok{[}\FunctionTok{p}\OperatorTok{,} \SpecialCharTok{\textbackslash{}}\OperatorTok{[}\NormalTok{Mu}\OperatorTok{]]}\NormalTok{ MT}\OperatorTok{[}\SpecialCharTok{\textbackslash{}}\OperatorTok{[}\NormalTok{Mu}\OperatorTok{],} \SpecialCharTok{\textbackslash{}}\OperatorTok{[}\NormalTok{Nu}\OperatorTok{]]}
\NormalTok{Contract}\OperatorTok{[}\SpecialCharTok{\%}\OperatorTok{]}
\end{Highlighting}
\end{Shaded}

\begin{dmath*}\breakingcomma
\text{FV}(p,\mu ) \;\text{MT}(\mu ,\nu )
\end{dmath*}

\begin{dmath*}\breakingcomma
\text{Contract}(\text{FV}(p,\mu ) \;\text{MT}(\mu ,\nu ))
\end{dmath*}

\begin{Shaded}
\begin{Highlighting}[]
\NormalTok{FV}\OperatorTok{[}\FunctionTok{p}\OperatorTok{,} \SpecialCharTok{\textbackslash{}}\OperatorTok{[}\NormalTok{Alpha}\OperatorTok{]]}\NormalTok{ FV}\OperatorTok{[}\FunctionTok{q}\OperatorTok{,} \SpecialCharTok{\textbackslash{}}\OperatorTok{[}\NormalTok{Alpha}\OperatorTok{]]}
\NormalTok{Contract}\OperatorTok{[}\SpecialCharTok{\%}\OperatorTok{]}
\end{Highlighting}
\end{Shaded}

\begin{dmath*}\breakingcomma
\text{FV}(p,\alpha ) \;\text{FV}(q,\alpha )
\end{dmath*}

\begin{dmath*}\breakingcomma
\text{Contract}(\text{FV}(p,\alpha ) \;\text{FV}(q,\alpha ))
\end{dmath*}

Notice that when we enter noncommutative objects, such as Dirac
matrices, we use \texttt{Dot} (\texttt{.}) and not \texttt{Times}
(\texttt{*})

\begin{Shaded}
\begin{Highlighting}[]
\NormalTok{FV}\OperatorTok{[}\FunctionTok{p}\OperatorTok{,} \SpecialCharTok{\textbackslash{}}\OperatorTok{[}\NormalTok{Alpha}\OperatorTok{]]}\NormalTok{ MT}\OperatorTok{[}\SpecialCharTok{\textbackslash{}}\OperatorTok{[}\FunctionTok{Beta}\OperatorTok{],} \SpecialCharTok{\textbackslash{}}\OperatorTok{[}\FunctionTok{Gamma}\OperatorTok{]]}\NormalTok{ GA}\OperatorTok{[}\SpecialCharTok{\textbackslash{}}\OperatorTok{[}\NormalTok{Alpha}\OperatorTok{]]}\NormalTok{ . GA}\OperatorTok{[}\SpecialCharTok{\textbackslash{}}\OperatorTok{[}\FunctionTok{Beta}\OperatorTok{]]}\NormalTok{ . GA}\OperatorTok{[}\SpecialCharTok{\textbackslash{}}\OperatorTok{[}\FunctionTok{Gamma}\OperatorTok{]]}
\NormalTok{Contract}\OperatorTok{[}\SpecialCharTok{\%}\OperatorTok{]}
\end{Highlighting}
\end{Shaded}

\begin{dmath*}\breakingcomma
\text{FV}(p,\alpha ) \;\text{MT}(\beta ,\gamma ) \;\text{GA}(\alpha ).\text{GA}(\beta ).\text{GA}(\gamma )
\end{dmath*}

\begin{dmath*}\breakingcomma
\text{Contract}(\text{FV}(p,\alpha ) \;\text{MT}(\beta ,\gamma ) \;\text{GA}(\alpha ).\text{GA}(\beta ).\text{GA}(\gamma ))
\end{dmath*}

This is because \texttt{Times} is commutative, so writing something like

\begin{Shaded}
\begin{Highlighting}[]
\NormalTok{GA}\OperatorTok{[}\SpecialCharTok{\textbackslash{}}\OperatorTok{[}\NormalTok{Delta}\OperatorTok{]]}\NormalTok{ GA}\OperatorTok{[}\SpecialCharTok{\textbackslash{}}\OperatorTok{[}\FunctionTok{Beta}\OperatorTok{]]}\NormalTok{ GA}\OperatorTok{[}\SpecialCharTok{\textbackslash{}}\OperatorTok{[}\NormalTok{Alpha}\OperatorTok{]]}
\end{Highlighting}
\end{Shaded}

\begin{dmath*}\breakingcomma
\text{GA}(\alpha ) \;\text{GA}(\beta ) \;\text{GA}(\delta )
\end{dmath*}

will give you completely wrong results. It is also a very common
beginner's mistake!

It might be surprising that FeynCalc does not seem to distinguish
between upper and lower Lorentz indices.

In fact, FeynCalc tacitly assumes that all your expressions with Lorentz
indices are manifestly Lorentz covariant and respect Einstein's
summation. In particular, this implies that

In an equality, if a free Lorentz index appears upstairs on the right
hand side, it must also appear upstairs on the left hand side. Something
like \(p^{\mu} = c q_{\mu}\) would violate manifest Lorentz covariance.
Hence,

\begin{Shaded}
\begin{Highlighting}[]
\NormalTok{FV}\OperatorTok{[}\FunctionTok{p}\OperatorTok{,} \SpecialCharTok{\textbackslash{}}\OperatorTok{[}\NormalTok{Mu}\OperatorTok{]]} \ExtensionTok{==} \FunctionTok{c}\NormalTok{ FV}\OperatorTok{[}\FunctionTok{q}\OperatorTok{,} \SpecialCharTok{\textbackslash{}}\OperatorTok{[}\NormalTok{Mu}\OperatorTok{]]}
\end{Highlighting}
\end{Shaded}

\begin{dmath*}\breakingcomma
\text{FV}(p,\mu )=c \;\text{FV}(q,\mu )
\end{dmath*}

could equally stand for \(p^{\mu} = c q^{\mu}\) or
\(p_{\mu} = c q_{\mu}\).

For the sake of definiteness, we impose that a free Lorentz should be
always understood to be an upper index. This becomes important when
dealing with nonrelativistic expressions involving Cartesian indices,
where there's no manifest Lorentz covariance.

Since FeynCalc assumes that the expressions you enter are mathematically
sensible, it will not check your input or complain, even if the
expression you provided is obviously incorrect

\begin{Shaded}
\begin{Highlighting}[]
\NormalTok{MT}\OperatorTok{[}\SpecialCharTok{\textbackslash{}}\OperatorTok{[}\NormalTok{Mu}\OperatorTok{],} \SpecialCharTok{\textbackslash{}}\OperatorTok{[}\NormalTok{Nu}\OperatorTok{]]}\NormalTok{ FV}\OperatorTok{[}\FunctionTok{p}\OperatorTok{,} \SpecialCharTok{\textbackslash{}}\OperatorTok{[}\NormalTok{Mu}\OperatorTok{]]}\NormalTok{ FV}\OperatorTok{[}\FunctionTok{q}\OperatorTok{,} \SpecialCharTok{\textbackslash{}}\OperatorTok{[}\NormalTok{Mu}\OperatorTok{]]}
\NormalTok{Contract}\OperatorTok{[}\SpecialCharTok{\%}\OperatorTok{]}
\end{Highlighting}
\end{Shaded}

\begin{dmath*}\breakingcomma
\text{FV}(p,\mu ) \;\text{FV}(q,\mu ) \;\text{MT}(\mu ,\nu )
\end{dmath*}

\begin{dmath*}\breakingcomma
\text{Contract}(\text{FV}(p,\mu ) \;\text{FV}(q,\mu ) \;\text{MT}(\mu ,\nu ))
\end{dmath*}

When it comes to products of Levi-Civita tensors (\texttt{Eps}),
\texttt{Contract} will by default apply the product formula with the
determinant of metric tensors

\begin{Shaded}
\begin{Highlighting}[]
\NormalTok{LC}\OperatorTok{[}\SpecialCharTok{\textbackslash{}}\OperatorTok{[}\NormalTok{Mu}\OperatorTok{],} \SpecialCharTok{\textbackslash{}}\OperatorTok{[}\NormalTok{Nu}\OperatorTok{]][}\FunctionTok{p}\OperatorTok{,} \FunctionTok{q}\OperatorTok{]}\NormalTok{ LC}\OperatorTok{[}\SpecialCharTok{\textbackslash{}}\OperatorTok{[}\NormalTok{Rho}\OperatorTok{],} \SpecialCharTok{\textbackslash{}}\OperatorTok{[}\NormalTok{Sigma}\OperatorTok{]][}\FunctionTok{r}\OperatorTok{,} \FunctionTok{s}\OperatorTok{]}\NormalTok{ FV}\OperatorTok{[}\FunctionTok{x}\OperatorTok{,} \SpecialCharTok{\textbackslash{}}\OperatorTok{[}\NormalTok{Mu}\OperatorTok{]]}
\NormalTok{Contract}\OperatorTok{[}\SpecialCharTok{\%}\OperatorTok{]}
\end{Highlighting}
\end{Shaded}

\begin{dmath*}\breakingcomma
\text{FV}(x,\mu ) \;\text{LC}(\mu ,\nu )(p,q) \;\text{LC}(\rho ,\sigma )(r,s)
\end{dmath*}

\begin{dmath*}\breakingcomma
\text{Contract}(\text{FV}(x,\mu ) \;\text{LC}(\mu ,\nu )(p,q) \;\text{LC}(\rho ,\sigma )(r,s))
\end{dmath*}

This is, however, not always what we want and can be inhibited via the
option \texttt{EpsContract}

\begin{Shaded}
\begin{Highlighting}[]
\NormalTok{LC}\OperatorTok{[}\SpecialCharTok{\textbackslash{}}\OperatorTok{[}\NormalTok{Mu}\OperatorTok{],} \SpecialCharTok{\textbackslash{}}\OperatorTok{[}\NormalTok{Nu}\OperatorTok{]][}\FunctionTok{p}\OperatorTok{,} \FunctionTok{q}\OperatorTok{]}\NormalTok{ LC}\OperatorTok{[}\SpecialCharTok{\textbackslash{}}\OperatorTok{[}\NormalTok{Rho}\OperatorTok{],} \SpecialCharTok{\textbackslash{}}\OperatorTok{[}\NormalTok{Sigma}\OperatorTok{]][}\FunctionTok{r}\OperatorTok{,} \FunctionTok{s}\OperatorTok{]}\NormalTok{ FV}\OperatorTok{[}\FunctionTok{x}\OperatorTok{,} \SpecialCharTok{\textbackslash{}}\OperatorTok{[}\NormalTok{Mu}\OperatorTok{]]}
\NormalTok{Contract}\OperatorTok{[}\SpecialCharTok{\%}\OperatorTok{,}\NormalTok{ EpsContract }\OtherTok{{-}\textgreater{}} \ConstantTok{False}\OperatorTok{]}
\end{Highlighting}
\end{Shaded}

\begin{dmath*}\breakingcomma
\text{FV}(x,\mu ) \;\text{LC}(\mu ,\nu )(p,q) \;\text{LC}(\rho ,\sigma )(r,s)
\end{dmath*}

\begin{dmath*}\breakingcomma
\text{Contract}(\text{FV}(x,\mu ) \;\text{LC}(\mu ,\nu )(p,q) \;\text{LC}(\rho ,\sigma )(r,s),\text{EpsContract}\to \;\text{False})
\end{dmath*}
\end{document}
