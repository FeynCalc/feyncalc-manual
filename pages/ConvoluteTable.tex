% !TeX program = pdflatex
% !TeX root = ConvoluteTable.tex

\documentclass[../FeynCalcManual.tex]{subfiles}
\begin{document}
\hypertarget{convolutetable}{
\section{ConvoluteTable}\label{convolutetable}\index{ConvoluteTable}}

\texttt{ConvoluteTable[\allowbreak{}f,\ \allowbreak{}g,\ \allowbreak{}x]}
yields the convolution of \texttt{f} and \texttt{g}.
\texttt{ConvoluteTable} is called by \texttt{Convolute}.

\subsection{See also}

\hyperlink{toc}{Overview},
\hyperlink{plusdistribution}{PlusDistribution},
\hyperlink{convolute}{Convolute}.

\subsection{Examples}

\begin{Shaded}
\begin{Highlighting}[]
\NormalTok{ConvoluteTable}\OperatorTok{[}\DecValTok{1}\OperatorTok{,} \DecValTok{1}\OperatorTok{,} \FunctionTok{x}\OperatorTok{]}
\end{Highlighting}
\end{Shaded}

\begin{dmath*}\breakingcomma
-\log (x)
\end{dmath*}

\begin{Shaded}
\begin{Highlighting}[]
\NormalTok{ConvoluteTable}\OperatorTok{[}\FunctionTok{x}\OperatorTok{,} \FunctionTok{x}\OperatorTok{]}
\end{Highlighting}
\end{Shaded}

\begin{dmath*}\breakingcomma
\text{False}[x,x]
\end{dmath*}

\begin{Shaded}
\begin{Highlighting}[]
\NormalTok{ConvoluteTable}\OperatorTok{[}\DecValTok{1}\OperatorTok{,} \FunctionTok{x}\OperatorTok{,} \FunctionTok{x}\OperatorTok{]}
\end{Highlighting}
\end{Shaded}

\begin{dmath*}\breakingcomma
1-x
\end{dmath*}
\end{document}
