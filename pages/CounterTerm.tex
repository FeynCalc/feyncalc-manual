% !TeX program = pdflatex
% !TeX root = CounterTerm.tex

\documentclass[../FeynCalcManual.tex]{subfiles}
\begin{document}
\hypertarget{counterterm}{%
\section{CounterTerm}\label{counterterm}}

\texttt{CounterTerm[\allowbreak{}name]} is a database of counter terms.
\texttt{CounterTerm} is also an option for the Feynman rule functions
\texttt{QuarkGluonVertex}, \texttt{GluonPropagator},
\texttt{QuarkPropagator}.

\subsection{See also}

\hyperlink{toc}{Overview}, \hyperlink{countert}{CounterT},
\hyperlink{quarkgluonvertex}{QuarkGluonVertex},
\hyperlink{gluonpropagator}{GluonPropagator},
\hyperlink{quarkpropagator}{QuarkPropagator}.

\subsection{Examples}

\begin{Shaded}
\begin{Highlighting}[]
\NormalTok{CounterTerm}\OperatorTok{[}\NormalTok{Zm}\OperatorTok{]}
\end{Highlighting}
\end{Shaded}

\begin{dmath*}\breakingcomma
\frac{C_F g_s^4 \left(\frac{4 \left(\frac{11 C_A}{2}+\frac{9 C_F}{2}-2 N_f T_f\right)}{\varepsilon ^2}+\frac{2 \left(\frac{97 C_A}{12}+\frac{3 C_F}{4}-\frac{5 N_f T_f}{3}\right)}{\varepsilon }\right)}{256 \pi ^4}+\frac{3 C_F g_s^2}{8 \pi ^2 \varepsilon }+1
\end{dmath*}
\end{document}
