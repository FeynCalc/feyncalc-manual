% !TeX program = pdflatex
% !TeX root = CovariantD.tex

\documentclass[../FeynCalcManual.tex]{subfiles}
\begin{document}
\hypertarget{covariantd}{%
\section{CovariantD}\label{covariantd}}

\texttt{CovariantD[\allowbreak{}mu]} is a generic covariant derivative
with Lorentz index \(\mu\).

\texttt{CovariantD[\allowbreak{}x,\ \allowbreak{}mu]} is a generic
covariant derivative with respect to \(x^{\mu }\).

\texttt{CovariantD[\allowbreak{}mu,\ \allowbreak{}a,\ \allowbreak{}b]}
is a covariant derivative for a bosonic field that acts on
\texttt{QuantumField[\allowbreak{}f,\ \allowbreak{}\{\allowbreak{}\},\ \allowbreak{}\{\allowbreak{}a,\ \allowbreak{}b\}]},
where \texttt{f} is some field name and \texttt{a} and \texttt{b} are
two \(SU(N)\) indices in the adjoint representation.

\texttt{CovariantD[\allowbreak{}OPEDelta,\ \allowbreak{}a,\ \allowbreak{}b]}
is a short form for
\texttt{CovariantD[\allowbreak{}mu,\ \allowbreak{}a,\ \allowbreak{}b] FV[\allowbreak{}OPEDelta,\ \allowbreak{}mu]}.

\texttt{CovariantD[\allowbreak{}\{\allowbreak{}OPEDelta,\ \allowbreak{}a,\ \allowbreak{}b\},\ \allowbreak{}\{\allowbreak{}n\}]}
yields the product of \texttt{n} operators, where \texttt{n} is an
integer.

\texttt{CovariantD[\allowbreak{}OPEDelta,\ \allowbreak{}a,\ \allowbreak{}b,\ \allowbreak{}\{\allowbreak{}m,\ \allowbreak{}n\}]}
gives the expanded form of
\texttt{CovariantD[\allowbreak{}OPEDelta,\ \allowbreak{}a,\ \allowbreak{}b]^m}
up to order \(g^n\) for the gluon, where \(n\) is an integer and \(g\)
the coupling constant indicated by the setting of the option
\texttt{CouplingConstant}.

\texttt{CovariantD[\allowbreak{}OPEDelta,\ \allowbreak{}\{\allowbreak{}m,\ \allowbreak{}n\}]}
gives the expanded form of \texttt{CovariantD[\allowbreak{}OPEDelta]^m}
up to order \(g^n\) of the fermionic field. To obtain the explicit
expression for a particular covariant derivative, the option
\texttt{Explicit} must be set to \texttt{True}.

\subsection{See also}

\hyperlink{toc}{Overview}

\subsection{Examples}

\begin{Shaded}
\begin{Highlighting}[]
\NormalTok{CovariantD}\OperatorTok{[}\SpecialCharTok{\textbackslash{}}\OperatorTok{[}\NormalTok{Mu}\OperatorTok{]]}
\end{Highlighting}
\end{Shaded}

\begin{dmath*}\breakingcomma
D_{\mu }
\end{dmath*}

\begin{Shaded}
\begin{Highlighting}[]
\NormalTok{CovariantD}\OperatorTok{[}\SpecialCharTok{\textbackslash{}}\OperatorTok{[}\NormalTok{Mu}\OperatorTok{],} \FunctionTok{a}\OperatorTok{,} \FunctionTok{b}\OperatorTok{]}
\end{Highlighting}
\end{Shaded}

\begin{dmath*}\breakingcomma
D_{\mu }^{ab}
\end{dmath*}

\begin{Shaded}
\begin{Highlighting}[]
\NormalTok{CovariantD}\OperatorTok{[}\SpecialCharTok{\textbackslash{}}\OperatorTok{[}\NormalTok{Mu}\OperatorTok{],}\NormalTok{ Explicit }\OtherTok{{-}\textgreater{}} \ConstantTok{True}\OperatorTok{]}
\end{Highlighting}
\end{Shaded}

\begin{dmath*}\breakingcomma
\vec{\partial }_{\mu }-i g_s T^{\text{c19}}.A_{\mu }^{\text{c19}}
\end{dmath*}

The first argument of \texttt{CovariantD} is interpreted as type
\texttt{LorentzIndex}, except for \texttt{OPEDelta}, which is type
\texttt{Momentum}.

\begin{Shaded}
\begin{Highlighting}[]
\NormalTok{CovariantD}\OperatorTok{[}\NormalTok{OPEDelta}\OperatorTok{]}
\end{Highlighting}
\end{Shaded}

\begin{dmath*}\breakingcomma
D_{\Delta }
\end{dmath*}

\begin{Shaded}
\begin{Highlighting}[]
\NormalTok{CovariantD}\OperatorTok{[}\NormalTok{OPEDelta}\OperatorTok{,} \FunctionTok{a}\OperatorTok{,} \FunctionTok{b}\OperatorTok{]}
\end{Highlighting}
\end{Shaded}

\begin{dmath*}\breakingcomma
D_{\Delta }^{ab}
\end{dmath*}

\begin{Shaded}
\begin{Highlighting}[]
\NormalTok{CovariantD}\OperatorTok{[}\NormalTok{OPEDelta}\OperatorTok{,} \FunctionTok{a}\OperatorTok{,} \FunctionTok{b}\OperatorTok{,}\NormalTok{ Explicit }\OtherTok{{-}\textgreater{}} \ConstantTok{True}\OperatorTok{]}
\end{Highlighting}
\end{Shaded}

\begin{dmath*}\breakingcomma
\delta ^{ab} \vec{\partial }_{\Delta }-g_s A_{\Delta }^{\text{c20}} f^{ab\text{c20}}
\end{dmath*}

\begin{Shaded}
\begin{Highlighting}[]
\NormalTok{CovariantD}\OperatorTok{[}\NormalTok{OPEDelta}\OperatorTok{,}\NormalTok{ Explicit }\OtherTok{{-}\textgreater{}} \ConstantTok{True}\OperatorTok{]}
\end{Highlighting}
\end{Shaded}

\begin{dmath*}\breakingcomma
\vec{\partial }_{\Delta }-i g_s T^{\text{c21}}.A_{\Delta }^{\text{c21}}
\end{dmath*}

\begin{Shaded}
\begin{Highlighting}[]
\NormalTok{CovariantD}\OperatorTok{[}\NormalTok{OPEDelta}\OperatorTok{,} \FunctionTok{a}\OperatorTok{,} \FunctionTok{b}\OperatorTok{,} \OperatorTok{\{}\DecValTok{2}\OperatorTok{\}]}
\end{Highlighting}
\end{Shaded}

\begin{dmath*}\breakingcomma
\left(\delta ^{a\text{c22}} \vec{\partial }_{\Delta }-g_s A_{\Delta }^{\text{e23}} f^{a\text{c22}\;\text{e23}}\right).\left(\delta ^{b\text{c22}} \vec{\partial }_{\Delta }-g_s A_{\Delta }^{\text{e24}} f^{\text{c22}b\text{e24}}\right)
\end{dmath*}

This gives \(m * \vec{\partial}_{\Delta}\), the partial derivative
\(\vec{\partial}_{\mu }\) contracted with \(\Delta ^{\mu }\)

\begin{Shaded}
\begin{Highlighting}[]
\NormalTok{CovariantD}\OperatorTok{[}\NormalTok{OPEDelta}\OperatorTok{,} \FunctionTok{a}\OperatorTok{,} \FunctionTok{b}\OperatorTok{,} \OperatorTok{\{}\NormalTok{OPEm}\OperatorTok{,} \DecValTok{0}\OperatorTok{\}]}
\end{Highlighting}
\end{Shaded}

\begin{dmath*}\breakingcomma
\delta ^{ab} \left(\vec{\partial }_{\Delta }\right){}^m
\end{dmath*}

The expansion up to first order in the coupling constant \(g_s\) (the
sum is the \texttt{FeynCalcOPESum})

\begin{Shaded}
\begin{Highlighting}[]
\NormalTok{CovariantD}\OperatorTok{[}\NormalTok{OPEDelta}\OperatorTok{,} \FunctionTok{a}\OperatorTok{,} \FunctionTok{b}\OperatorTok{,} \OperatorTok{\{}\NormalTok{OPEm}\OperatorTok{,} \DecValTok{1}\OperatorTok{\}]}
\end{Highlighting}
\end{Shaded}

\begin{dmath*}\breakingcomma
\delta ^{ab} \left(\vec{\partial }_{\Delta }\right){}^m-g_s \left(\sum _{i=0}^{-1+m} \left(\vec{\partial }_{\Delta }\right){}^i.A_{\Delta }^{\text{c34}_1}.\left(\vec{\partial }_{\Delta }\right){}^{-1-i+m} f^{ab\text{c34}_1}\right)
\end{dmath*}

The expansion up to second order in the \(g_s\)

\begin{Shaded}
\begin{Highlighting}[]
\NormalTok{CovariantD}\OperatorTok{[}\NormalTok{OPEDelta}\OperatorTok{,} \FunctionTok{a}\OperatorTok{,} \FunctionTok{b}\OperatorTok{,} \OperatorTok{\{}\NormalTok{OPEm}\OperatorTok{,} \DecValTok{2}\OperatorTok{\}]}
\end{Highlighting}
\end{Shaded}

\begin{dmath*}\breakingcomma
-g_s \left(\sum _{i=0}^{-1+m} \left(\vec{\partial }_{\Delta }\right){}^i.A_{\Delta }^{\text{c42}_1}.\left(\vec{\partial }_{\Delta }\right){}^{-1-i+m} f^{ab\text{c42}_1}\right)-g_s^2 \left(\sum _{j=0}^{-2+m} \left(\sum _{i=0}^j \left(\vec{\partial }_{\Delta }\right){}^i.A_{\Delta }^{\text{c46}_1}.\left(\vec{\partial }_{\Delta }\right){}^{-i+j}.A_{\Delta }^{\text{c46}_2}.\left(\vec{\partial }_{\Delta }\right){}^{-2-j+m} f^{a\text{c46}_1\text{e45}_1} f^{b\text{c46}_2\text{e45}_1}\right)\right)+\delta ^{ab} \left(\vec{\partial }_{\Delta }\right){}^m
\end{dmath*}

\begin{Shaded}
\begin{Highlighting}[]
\NormalTok{CovariantD}\OperatorTok{[}\NormalTok{OPEDelta}\OperatorTok{,} \FunctionTok{a}\OperatorTok{,} \FunctionTok{b}\OperatorTok{]}\SpecialCharTok{\^{}}\NormalTok{OPEm}
\end{Highlighting}
\end{Shaded}

\begin{dmath*}\breakingcomma
\left(D_{\Delta }^{ab}\right){}^m
\end{dmath*}

\begin{Shaded}
\begin{Highlighting}[]
\NormalTok{CovariantD}\OperatorTok{[}\NormalTok{OPEDelta}\OperatorTok{,} \OperatorTok{\{}\NormalTok{OPEm}\OperatorTok{,} \DecValTok{2}\OperatorTok{\}]}
\end{Highlighting}
\end{Shaded}

\begin{dmath*}\breakingcomma
-i g_s \left(\sum _{i=0}^{-1+m} \left(\vec{\partial }_{\Delta }\right){}^i.A_{\Delta }^{\text{c55}_1}.\left(\vec{\partial }_{\Delta }\right){}^{-1-i+m} T^{\text{c55}_1}\right)-g_s^2 \left(\sum _{j=0}^{-2+m} \left(\sum _{i=0}^j T^{\text{c59}_1}.T^{\text{c59}_2} \left(\vec{\partial }_{\Delta }\right){}^i.A_{\Delta }^{\text{c59}_1}.\left(\vec{\partial }_{\Delta }\right){}^{-i+j}.A_{\Delta }^{\text{c59}_2}.\left(\vec{\partial }_{\Delta }\right){}^{-2-j+m}\right)\right)+\left(\vec{\partial }_{\Delta }\right){}^m
\end{dmath*}

\begin{Shaded}
\begin{Highlighting}[]
\NormalTok{CovariantD}\OperatorTok{[}\NormalTok{OPEDelta}\OperatorTok{,}\NormalTok{ Explicit }\OtherTok{{-}\textgreater{}} \ConstantTok{True}\OperatorTok{]} \SpecialCharTok{//} \FunctionTok{StandardForm}

\CommentTok{(*RightPartialD[Momentum[OPEDelta]] {-} I SUNT[SUNIndex[c62]] . QuantumField[GaugeField, Momentum[OPEDelta], SUNIndex[c62]] SMP["g\_s"]*)}
\end{Highlighting}
\end{Shaded}

\begin{Shaded}
\begin{Highlighting}[]
\NormalTok{CovariantD}\OperatorTok{[}\SpecialCharTok{\textbackslash{}}\OperatorTok{[}\NormalTok{Mu}\OperatorTok{],} \FunctionTok{a}\OperatorTok{,} \FunctionTok{b}\OperatorTok{,}\NormalTok{ Explicit }\OtherTok{{-}\textgreater{}} \ConstantTok{True}\OperatorTok{]} \SpecialCharTok{//} \FunctionTok{StandardForm}

\CommentTok{(*RightPartialD[LorentzIndex[\textbackslash{}[Mu]]] SUNDelta[a, b] {-} QuantumField[GaugeField, LorentzIndex[\textbackslash{}[Mu]], SUNIndex[c63]] SMP["g\_s"] SUNF[a, b, c63]*)}
\end{Highlighting}
\end{Shaded}

\end{document}
