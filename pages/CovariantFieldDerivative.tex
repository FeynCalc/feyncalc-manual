% !TeX program = pdflatex
% !TeX root = CovariantFieldDerivative.tex

\documentclass[../FeynCalcManual.tex]{subfiles}
\begin{document}
\hypertarget{covariantfieldderivative}{
\section{CovariantFieldDerivative}\label{covariantfieldderivative}\index{CovariantFieldDerivative}}

\texttt{CovariantFieldDerivative[\allowbreak{}f[\allowbreak{}x],\ \allowbreak{}x,\ \allowbreak{}\{\allowbreak{}li1,\ \allowbreak{}li2,\ \allowbreak{}...\}]}
is a covariant derivative of \texttt{f[\allowbreak{}x]} with respect to
space-time variables \texttt{x} and with Lorentz indices
\texttt{li1,\ \allowbreak{}li2,\ \allowbreak{}...}.
\texttt{CovariantFieldDerivative} has only typesetting definitions by
default. The user is must supply his/her own definition of the actual
function.

\subsection{See also}

\hyperlink{toc}{Overview}, \hyperlink{covariantd}{CovariantD},
\hyperlink{expandpartiald}{ExpandPartialD},
\hyperlink{fieldderivative}{FieldDerivative}.

\subsection{Examples}

\begin{Shaded}
\begin{Highlighting}[]
\NormalTok{CovariantFieldDerivative}\OperatorTok{[}\NormalTok{QuantumField}\OperatorTok{[}\FunctionTok{A}\OperatorTok{,} \OperatorTok{\{}\SpecialCharTok{\textbackslash{}}\OperatorTok{[}\NormalTok{Mu}\OperatorTok{]\}][}\FunctionTok{x}\OperatorTok{],} \FunctionTok{x}\OperatorTok{,} \OperatorTok{\{}\SpecialCharTok{\textbackslash{}}\OperatorTok{[}\NormalTok{Mu}\OperatorTok{]\}]}
\end{Highlighting}
\end{Shaded}

\begin{dmath*}\breakingcomma
\text{\textit{$\mathcal{D}$}}_{\mu }\left(A_{\mu }(x)\right)
\end{dmath*}
\end{document}
