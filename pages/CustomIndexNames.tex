% !TeX program = pdflatex
% !TeX root = CustomIndexNames.tex

\documentclass[../FeynCalcManual.tex]{subfiles}
\begin{document}
\hypertarget{customindexnames}{
\section{CustomIndexNames}\label{customindexnames}\index{CustomIndexNames}}

\texttt{CustomIndexNames} is an option of
\texttt{FCCanonicalizeDummyIndices}. It allows to specify custom names
for canonicalized dummy indices of custom index heads.

\subsection{See also}

\hyperlink{toc}{Overview}, \hyperlink{fcfaconvert}{FCFAConvert},
\hyperlink{fccanonicalizedummyindices}{FCCanonicalizeDummyIndices},
\hyperlink{lorentzindexnames}{LorentzIndexNames},
\hyperlink{cartesianindexnames}{CartesianIndexNames}.

\subsection{Examples}

\begin{Shaded}
\begin{Highlighting}[]
\NormalTok{ex }\ExtensionTok{=}\NormalTok{ T1}\OperatorTok{[}\NormalTok{MyIndex2}\OperatorTok{[}\FunctionTok{a}\OperatorTok{],}\NormalTok{ MyIndex1}\OperatorTok{[}\FunctionTok{b}\OperatorTok{]]}\NormalTok{ v1}\OperatorTok{[}\NormalTok{MyIndex1}\OperatorTok{[}\FunctionTok{b}\OperatorTok{]]}\NormalTok{ v2}\OperatorTok{[}\NormalTok{MyIndex2}\OperatorTok{[}\FunctionTok{a}\OperatorTok{]]} \SpecialCharTok{+} 
\NormalTok{   T1}\OperatorTok{[}\NormalTok{MyIndex2}\OperatorTok{[}\FunctionTok{c}\OperatorTok{],}\NormalTok{ MyIndex1}\OperatorTok{[}\FunctionTok{f}\OperatorTok{]]}\NormalTok{ v1}\OperatorTok{[}\NormalTok{MyIndex1}\OperatorTok{[}\FunctionTok{f}\OperatorTok{]]}\NormalTok{ v2}\OperatorTok{[}\NormalTok{MyIndex2}\OperatorTok{[}\FunctionTok{c}\OperatorTok{]]}
\end{Highlighting}
\end{Shaded}

\begin{dmath*}\breakingcomma
\text{v2}(\text{MyIndex2}(a)) \;\text{v1}(\text{MyIndex1}(b)) \;\text{T1}(\text{MyIndex2}(a),\text{MyIndex1}(b))+\text{v2}(\text{MyIndex2}(c)) \;\text{v1}(\text{MyIndex1}(f)) \;\text{T1}(\text{MyIndex2}(c),\text{MyIndex1}(f))
\end{dmath*}

\begin{Shaded}
\begin{Highlighting}[]
\NormalTok{FCCanonicalizeDummyIndices}\OperatorTok{[}\NormalTok{ex }\OperatorTok{,} \FunctionTok{Head} \OtherTok{{-}\textgreater{}} \OperatorTok{\{}\NormalTok{MyIndex1}\OperatorTok{,}\NormalTok{ MyIndex2}\OperatorTok{\},} 
\NormalTok{  CustomIndexNames }\OtherTok{{-}\textgreater{}} \OperatorTok{\{\{}\NormalTok{MyIndex1}\OperatorTok{,} \OperatorTok{\{}\NormalTok{i1}\OperatorTok{\}\},} \OperatorTok{\{}\NormalTok{MyIndex2}\OperatorTok{,} \OperatorTok{\{}\NormalTok{i2}\OperatorTok{\}\}\}]}
\end{Highlighting}
\end{Shaded}

\begin{dmath*}\breakingcomma
2 \;\text{v1}(\text{MyIndex1}(\text{i1})) \;\text{v2}(\text{MyIndex2}(\text{i2})) \;\text{T1}(\text{MyIndex2}(\text{i2}),\text{MyIndex1}(\text{i1}))
\end{dmath*}
\end{document}
