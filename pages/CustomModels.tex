% !TeX program = pdflatex
% !TeX root = CustomModels.tex

\documentclass[../FeynCalcManual.tex]{subfiles}
\begin{document}
\hypertarget{custom feynrules models}{
\section{Custom FeynRules models}\label{custom feynrules models}\index{Custom FeynRules models}}

\subsection{See also}

\hyperlink{toc}{Overview}.

If you want to create new FeynArts models using
\href{https://feynrules.irmp.ucl.ac.be/}{FeynRules}, please keep in mind
that those models must be patched for compatibility with FeynCalc before
they can be used for calculations.

The patching is done by using the function \texttt{FAPatch} with the
option \texttt{PatchModelsOnly} set to \texttt{True}

\begin{itemize}
\tightlist
\item
  If the model is located inside the \texttt{Models} directory of
  \texttt{\$FeynArtsDirectory}, it is sufficient to evaluate
  \texttt{FAPatch[\allowbreak{}PatchModelsOnly -> True]} after loading
  FeynCalc and FeynArts in the usual way.
\item
  If the model has been placed into a separate directory, use
  \texttt{FAPatch[\allowbreak{}PatchModelsOnly -> True,\ \allowbreak{}FAModelsDirectory -> "fullPathToMyModelDir"]}
\end{itemize}

The patching has to be done only once for each new model. However,
rerunning \texttt{FAPatch[\allowbreak{}PatchModelsOnly -> True]} would
not do any harm, so if you often modify existing or add new models, you
might want to keep this command in your working notebooks.
\end{document}
