% !TeX program = pdflatex
% !TeX root = D0.tex

\documentclass[../FeynCalcManual.tex]{subfiles}
\begin{document}
\hypertarget{d0}{
\section{D0}\label{d0}\index{D0}}

\texttt{D0[\allowbreak{}p10,\ \allowbreak{}p12,\ \allowbreak{}p23,\ \allowbreak{}p30,\ \allowbreak{}p20,\ \allowbreak{}p13,\ \allowbreak{}m1^2,\ \allowbreak{}m2^2,\ \allowbreak{}m3^2,\ \allowbreak{}m4^2 ]}
is the Passarino-Veltman \(D_0\) function. The convention for the
arguments is that if the denominator of the integrand has the form
\(([q^2-m1^2] [(q+p1)^2-m2^2] [(q+p2)^2-m3^2] [(q+p3)^2-m4^2])\), the
first six arguments of \texttt{D0} are the scalar products
\(p10 = p1^2\), \(p12 = (p1-p2)^2\), \(p23 = (p2-p3)^2\),
\(p30 = p3^2\), \(p20 = p2^2\), \(p13 = (p1-p3)^2\).

\subsection{See also}

\hyperlink{toc}{Overview}, \hyperlink{b0}{B0}, \hyperlink{c0}{C0},
\hyperlink{pave}{PaVe}, \hyperlink{paveorder}{PaVeOrder}.

\subsection{Examples}

\begin{Shaded}
\begin{Highlighting}[]
\NormalTok{D0}\OperatorTok{[}\NormalTok{p10}\OperatorTok{,}\NormalTok{ p12}\OperatorTok{,}\NormalTok{ p23}\OperatorTok{,}\NormalTok{ p30}\OperatorTok{,}\NormalTok{ p20}\OperatorTok{,}\NormalTok{ p13}\OperatorTok{,}\NormalTok{ m1}\SpecialCharTok{\^{}}\DecValTok{2}\OperatorTok{,}\NormalTok{ m2}\SpecialCharTok{\^{}}\DecValTok{2}\OperatorTok{,}\NormalTok{ m3}\SpecialCharTok{\^{}}\DecValTok{2}\OperatorTok{,}\NormalTok{ m4}\SpecialCharTok{\^{}}\DecValTok{2}\OperatorTok{]}
\end{Highlighting}
\end{Shaded}

\begin{dmath*}\breakingcomma
\text{D}_0\left(\text{p10},\text{p12},\text{p23},\text{p30},\text{p20},\text{p13},\text{m1}^2,\text{m2}^2,\text{m3}^2,\text{m4}^2\right)
\end{dmath*}

\begin{Shaded}
\begin{Highlighting}[]
\NormalTok{PaVeOrder}\OperatorTok{[}\NormalTok{D0}\OperatorTok{[}\NormalTok{p10}\OperatorTok{,}\NormalTok{ p12}\OperatorTok{,}\NormalTok{ p23}\OperatorTok{,}\NormalTok{ p30}\OperatorTok{,}\NormalTok{ p20}\OperatorTok{,}\NormalTok{ p13}\OperatorTok{,}\NormalTok{ m1}\SpecialCharTok{\^{}}\DecValTok{2}\OperatorTok{,}\NormalTok{ m2}\SpecialCharTok{\^{}}\DecValTok{2}\OperatorTok{,}\NormalTok{ m3}\SpecialCharTok{\^{}}\DecValTok{2}\OperatorTok{,}\NormalTok{ m4}\SpecialCharTok{\^{}}\DecValTok{2}\OperatorTok{],}\NormalTok{ PaVeOrderList }\OtherTok{{-}\textgreater{}} \OperatorTok{\{}\NormalTok{p13}\OperatorTok{,}\NormalTok{ p20}\OperatorTok{\}]}
\end{Highlighting}
\end{Shaded}

\begin{dmath*}\breakingcomma
\text{D}_0\left(\text{p10},\text{p30},\text{p23},\text{p12},\text{p13},\text{p20},\text{m2}^2,\text{m1}^2,\text{m4}^2,\text{m3}^2\right)
\end{dmath*}

\begin{Shaded}
\begin{Highlighting}[]
\NormalTok{PaVeOrder}\OperatorTok{[}\SpecialCharTok{\%}\OperatorTok{]}
\end{Highlighting}
\end{Shaded}

\begin{dmath*}\breakingcomma
\text{D}_0\left(\text{p10},\text{p12},\text{p23},\text{p30},\text{p20},\text{p13},\text{m1}^2,\text{m2}^2,\text{m3}^2,\text{m4}^2\right)
\end{dmath*}
\end{document}
