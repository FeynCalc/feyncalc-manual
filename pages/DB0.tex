% !TeX program = pdflatex
% !TeX root = DB0.tex

\documentclass[../FeynCalcManual.tex]{subfiles}
\begin{document}
\hypertarget{db0}{
\section{DB0}\label{db0}\index{DB0}}

\texttt{DB0[\allowbreak{}p2,\ \allowbreak{}m1^2,\ \allowbreak{}m2^2]} is
the derivative of the two-point function
\texttt{B0[\allowbreak{}p2,\ \allowbreak{}m1^2,\ \allowbreak{}m2^2]}
with respect to \texttt{p2}.

\subsection{See also}

\hyperlink{toc}{Overview}, \hyperlink{b0}{B0}.

\subsection{Examples}

\begin{Shaded}
\begin{Highlighting}[]
\FunctionTok{D}\OperatorTok{[}\NormalTok{B0}\OperatorTok{[}\FunctionTok{Subscript}\OperatorTok{[}\FunctionTok{p}\OperatorTok{,} \DecValTok{2}\OperatorTok{],} \FunctionTok{Subscript}\OperatorTok{[}\FunctionTok{m}\OperatorTok{,} \DecValTok{1}\OperatorTok{]}\SpecialCharTok{\^{}}\DecValTok{2}\OperatorTok{,} \FunctionTok{Subscript}\OperatorTok{[}\FunctionTok{m}\OperatorTok{,} \DecValTok{2}\OperatorTok{]}\SpecialCharTok{\^{}}\DecValTok{2}\OperatorTok{],} \FunctionTok{Subscript}\OperatorTok{[}\FunctionTok{p}\OperatorTok{,} \DecValTok{2}\OperatorTok{]]}
\end{Highlighting}
\end{Shaded}

\begin{dmath*}\breakingcomma
\text{DB0}\left(p_2,m_1^2,m_2^2\right)
\end{dmath*}
\end{document}
