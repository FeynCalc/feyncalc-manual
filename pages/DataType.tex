% !TeX program = pdflatex
% !TeX root = DataType.tex

\documentclass[../FeynCalcManual.tex]{subfiles}
\begin{document}
\hypertarget{datatype}{
\section{DataType}\label{datatype}\index{DataType}}

\texttt{DataType[\allowbreak{}exp,\ \allowbreak{}type] = True} defines
the object \texttt{exp} to have data-type \texttt{type}.

\texttt{DataType[\allowbreak{}exp1,\ \allowbreak{}exp2,\ \allowbreak{}...,\ \allowbreak{}type]}
defines the objects \texttt{exp1,\ \allowbreak{}exp2,\ \allowbreak{}...}
to have data-type \texttt{type}.

The default setting is
\texttt{DataType[\allowbreak{}__,\ \allowbreak{}_] := False}.

To assign a certain data-type, do, e.g.,
\texttt{DataType[\allowbreak{}x,\ \allowbreak{}PositiveInteger] = True}.
Currently used \texttt{DataTypes}:

\begin{itemize}
\item
  \texttt{NonCommutative}
\item
  \texttt{PositiveInteger}
\item
  \texttt{NegativeInteger}
\item
  \texttt{PositiveNumber}
\item
  \texttt{FreeIndex}
\item
  \texttt{GrassmannParity}
\end{itemize}

If loaded, PHI adds the \texttt{DataTypes}: \texttt{UMatrix},
\texttt{UScalar}.

\subsection{See also}

\hyperlink{toc}{Overview},
\hyperlink{declarenoncommutative}{DeclareNonCommutative}.

\subsection{Examples}

\texttt{NonCommutative} is just a data-type.

\begin{Shaded}
\begin{Highlighting}[]
\NormalTok{DataType}\OperatorTok{[}\FunctionTok{f}\OperatorTok{,} \FunctionTok{g}\OperatorTok{,}\NormalTok{ NonCommutative}\OperatorTok{]} \ExtensionTok{=} \ConstantTok{True}\NormalTok{; }
 
\FunctionTok{t} \ExtensionTok{=} \FunctionTok{f}\NormalTok{ . }\FunctionTok{g} \SpecialCharTok{{-}} \FunctionTok{g}\NormalTok{ . (}\DecValTok{2} \FunctionTok{a}\NormalTok{) . }\FunctionTok{f}
\end{Highlighting}
\end{Shaded}

\begin{dmath*}\breakingcomma
f.g-g.(2 a).f
\end{dmath*}

Since \texttt{f} and \texttt{g} have \texttt{DataType}
\texttt{NonCommutative}, the function \texttt{DotSimplify} extracts only
\texttt{a} out of the noncommutative product.

\begin{Shaded}
\begin{Highlighting}[]
\NormalTok{DotSimplify}\OperatorTok{[}\FunctionTok{t}\OperatorTok{]}
\end{Highlighting}
\end{Shaded}

\begin{dmath*}\breakingcomma
f.g-2 a g.f
\end{dmath*}

\begin{Shaded}
\begin{Highlighting}[]
\NormalTok{DataType}\OperatorTok{[}\FunctionTok{m}\OperatorTok{,}\NormalTok{ odd}\OperatorTok{]} \ExtensionTok{=}\NormalTok{ DataType}\OperatorTok{[}\FunctionTok{a}\OperatorTok{,}\NormalTok{ even}\OperatorTok{]} \ExtensionTok{=} \ConstantTok{True}\NormalTok{; }
 
\NormalTok{ptest1}\OperatorTok{[}\AttributeTok{x\_}\OperatorTok{]} \ExtensionTok{:=} \FunctionTok{x} \OtherTok{/.}\NormalTok{ (}\SpecialCharTok{{-}}\DecValTok{1}\NormalTok{)}\SpecialCharTok{\^{}}\AttributeTok{n\_} \SpecialCharTok{/}\NormalTok{; DataType}\OperatorTok{[}\FunctionTok{n}\OperatorTok{,}\NormalTok{ odd}\OperatorTok{]}\NormalTok{ :\textgreater{} }\SpecialCharTok{{-}}\DecValTok{1}\NormalTok{; }
 
\NormalTok{ptest2}\OperatorTok{[}\AttributeTok{x\_}\OperatorTok{]} \ExtensionTok{:=} \FunctionTok{x} \OtherTok{/.}\NormalTok{ (}\SpecialCharTok{{-}}\DecValTok{1}\NormalTok{)}\SpecialCharTok{\^{}}\AttributeTok{n\_} \SpecialCharTok{/}\NormalTok{; DataType}\OperatorTok{[}\FunctionTok{n}\OperatorTok{,}\NormalTok{ even}\OperatorTok{]}\NormalTok{ :\textgreater{} }\DecValTok{1}\NormalTok{; }
 
\FunctionTok{t} \ExtensionTok{=}\NormalTok{ (}\SpecialCharTok{{-}}\DecValTok{1}\NormalTok{)}\SpecialCharTok{\^{}}\FunctionTok{m} \SpecialCharTok{+}\NormalTok{ (}\SpecialCharTok{{-}}\DecValTok{1}\NormalTok{)}\SpecialCharTok{\^{}}\FunctionTok{a} \SpecialCharTok{+}\NormalTok{ (}\SpecialCharTok{{-}}\DecValTok{1}\NormalTok{)}\SpecialCharTok{\^{}}\FunctionTok{z}
\end{Highlighting}
\end{Shaded}

\begin{dmath*}\breakingcomma
(-1)^a+(-1)^m+(-1)^z
\end{dmath*}

\begin{Shaded}
\begin{Highlighting}[]
\NormalTok{ptest1}\OperatorTok{[}\FunctionTok{t}\OperatorTok{]} 
 
\NormalTok{ptest2}\OperatorTok{[}\SpecialCharTok{\%}\OperatorTok{]}
\end{Highlighting}
\end{Shaded}

\begin{dmath*}\breakingcomma
(-1)^a+(-1)^z-1
\end{dmath*}

\begin{dmath*}\breakingcomma
(-1)^z
\end{dmath*}

\begin{Shaded}
\begin{Highlighting}[]
\FunctionTok{Clear}\OperatorTok{[}\NormalTok{ptest1}\OperatorTok{,}\NormalTok{ ptest2}\OperatorTok{,} \FunctionTok{t}\OperatorTok{,} \FunctionTok{a}\OperatorTok{,} \FunctionTok{m}\OperatorTok{]}\NormalTok{;}
\end{Highlighting}
\end{Shaded}

\begin{Shaded}
\begin{Highlighting}[]
\NormalTok{DataType}\OperatorTok{[}\FunctionTok{m}\OperatorTok{,} \FunctionTok{integer}\OperatorTok{]} \ExtensionTok{=} \ConstantTok{True}\NormalTok{; }
 
\FunctionTok{f}\OperatorTok{[}\AttributeTok{x\_}\OperatorTok{]} \ExtensionTok{:=} \FunctionTok{x} \OtherTok{/.} \OperatorTok{\{}\NormalTok{(}\SpecialCharTok{{-}}\DecValTok{1}\NormalTok{)}\SpecialCharTok{\^{}}\AttributeTok{p\_} \SpecialCharTok{/}\NormalTok{; DataType}\OperatorTok{[}\FunctionTok{p}\OperatorTok{,} \FunctionTok{integer}\OperatorTok{]}\NormalTok{ :\textgreater{} }\DecValTok{1}\OperatorTok{\}}\NormalTok{;}
\end{Highlighting}
\end{Shaded}

\begin{Shaded}
\begin{Highlighting}[]
\NormalTok{test }\ExtensionTok{=}\NormalTok{ (}\SpecialCharTok{{-}}\DecValTok{1}\NormalTok{)}\SpecialCharTok{\^{}}\FunctionTok{m} \SpecialCharTok{+}\NormalTok{ (}\SpecialCharTok{{-}}\DecValTok{1}\NormalTok{)}\SpecialCharTok{\^{}}\FunctionTok{n} \FunctionTok{x}
\end{Highlighting}
\end{Shaded}

\begin{dmath*}\breakingcomma
(-1)^m+(-1)^n x
\end{dmath*}

\begin{Shaded}
\begin{Highlighting}[]
\FunctionTok{f}\OperatorTok{[}\NormalTok{test}\OperatorTok{]}
\end{Highlighting}
\end{Shaded}

\begin{dmath*}\breakingcomma
(-1)^n x+1
\end{dmath*}

\begin{Shaded}
\begin{Highlighting}[]
\FunctionTok{Clear}\OperatorTok{[}\FunctionTok{f}\OperatorTok{,}\NormalTok{ test}\OperatorTok{]}\NormalTok{; }
 
\NormalTok{DataType}\OperatorTok{[}\FunctionTok{f}\OperatorTok{,} \FunctionTok{g}\OperatorTok{,}\NormalTok{ NonCommutative}\OperatorTok{]} \ExtensionTok{=} \ConstantTok{False}\NormalTok{; }
 
\NormalTok{DataType}\OperatorTok{[}\FunctionTok{m}\OperatorTok{,}\NormalTok{ odd}\OperatorTok{]} \ExtensionTok{=}\NormalTok{ DataType}\OperatorTok{[}\FunctionTok{a}\OperatorTok{,}\NormalTok{ even}\OperatorTok{]} \ExtensionTok{=} \ConstantTok{False}\NormalTok{;}
\end{Highlighting}
\end{Shaded}

Certain FeynCalc objects have \texttt{DataType} \texttt{PositiveInteger}
set to \texttt{True}.

\begin{Shaded}
\begin{Highlighting}[]
\NormalTok{DataType}\OperatorTok{[}\NormalTok{OPEm}\OperatorTok{,}\NormalTok{ PositiveInteger}\OperatorTok{]}
\end{Highlighting}
\end{Shaded}

\begin{dmath*}\breakingcomma
\text{True}
\end{dmath*}

PowerSimplify uses the DataType information.

\begin{Shaded}
\begin{Highlighting}[]
\NormalTok{PowerSimplify}\OperatorTok{[}\NormalTok{ (}\SpecialCharTok{{-}}\DecValTok{1}\NormalTok{)}\SpecialCharTok{\^{}}\NormalTok{(}\DecValTok{2}\NormalTok{ OPEm)}\OperatorTok{]}
\end{Highlighting}
\end{Shaded}

\begin{dmath*}\breakingcomma
1
\end{dmath*}

\begin{Shaded}
\begin{Highlighting}[]
\NormalTok{PowerSimplify}\OperatorTok{[}\NormalTok{ (}\SpecialCharTok{{-}}\NormalTok{ SO}\OperatorTok{[}\FunctionTok{q}\OperatorTok{]}\NormalTok{)}\SpecialCharTok{\^{}}\NormalTok{OPEm}\OperatorTok{]}
\end{Highlighting}
\end{Shaded}

\begin{dmath*}\breakingcomma
(\Delta \cdot q)^m e^{2 i \pi  m \left\lfloor -\frac{\arg (\Delta \cdot q)}{2 \pi }\right\rfloor +i \pi  m}
\end{dmath*}
\end{document}
