% !TeX program = pdflatex
% !TeX root = DeclareNonCommutative.tex

\documentclass[../FeynCalcManual.tex]{subfiles}
\begin{document}
\hypertarget{declarenoncommutative}{%
\section{DeclareNonCommutative}\label{declarenoncommutative}}

\texttt{DeclareNonCommutative[\allowbreak{}a,\ \allowbreak{}b,\ \allowbreak{}...]}
declares \texttt{a,\ \allowbreak{}b,\ \allowbreak{}...} to be
non-commutative, i.e.,
\texttt{DataType[\allowbreak{}a,\ \allowbreak{}b,\ \allowbreak{}...,\ \allowbreak{}NonCommutative]}
is set to \texttt{True}.

\subsection{See also}

\hyperlink{toc}{Overview}, \hyperlink{datatype}{DataType},
\hyperlink{undeclarenoncommutative}{UnDeclareNonCommutative}.

\subsection{Examples}

As a side effect of \texttt{DeclareNonCommutative}, \texttt{x} is
declared to be of data type \texttt{NonCommutative}.

\begin{Shaded}
\begin{Highlighting}[]
\NormalTok{DeclareNonCommutative}\OperatorTok{[}\FunctionTok{x}\OperatorTok{]}
\end{Highlighting}
\end{Shaded}

\begin{Shaded}
\begin{Highlighting}[]
\NormalTok{DataType}\OperatorTok{[}\FunctionTok{x}\OperatorTok{,}\NormalTok{ NonCommutative}\OperatorTok{]}
\end{Highlighting}
\end{Shaded}

\begin{dmath*}\breakingcomma
\text{True}
\end{dmath*}

\begin{Shaded}
\begin{Highlighting}[]
\NormalTok{DeclareNonCommutative}\OperatorTok{[}\FunctionTok{y}\OperatorTok{,} \FunctionTok{z}\OperatorTok{]} 
 
\NormalTok{DataType}\OperatorTok{[}\FunctionTok{a}\OperatorTok{,} \FunctionTok{x}\OperatorTok{,} \FunctionTok{y}\OperatorTok{,} \FunctionTok{z}\OperatorTok{,}\NormalTok{ NonCommutative}\OperatorTok{]}
\end{Highlighting}
\end{Shaded}

\begin{dmath*}\breakingcomma
\{\text{False},\text{True},\text{True},\text{True}\}
\end{dmath*}

\begin{Shaded}
\begin{Highlighting}[]
\NormalTok{UnDeclareNonCommutative}\OperatorTok{[}\FunctionTok{x}\OperatorTok{,} \FunctionTok{y}\OperatorTok{,} \FunctionTok{z}\OperatorTok{]} 
 
\NormalTok{DataType}\OperatorTok{[}\FunctionTok{a}\OperatorTok{,} \FunctionTok{x}\OperatorTok{,} \FunctionTok{y}\OperatorTok{,} \FunctionTok{z}\OperatorTok{,}\NormalTok{ NonCommutative}\OperatorTok{]}
\end{Highlighting}
\end{Shaded}

\begin{dmath*}\breakingcomma
\{\text{False},\text{False},\text{False},\text{False}\}
\end{dmath*}
\end{document}
