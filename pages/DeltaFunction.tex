% !TeX program = pdflatex
% !TeX root = DeltaFunction.tex

\documentclass[../FeynCalcManual.tex]{subfiles}
\begin{document}
\hypertarget{deltafunction}{
\section{DeltaFunction}\label{deltafunction}\index{DeltaFunction}}

\texttt{DeltaFunction[\allowbreak{}x]} is the Dirac delta-function
\(\delta (x)\).

Mathematica also provides a built-in function \texttt{DiracDelta} with
comparable properties.

\subsection{See also}

\hyperlink{toc}{Overview}, \hyperlink{convolute}{Convolute},
\hyperlink{deltafunctionprime}{DeltaFunctionPrime},
\hyperlink{integrate2}{Integrate2},
\hyperlink{simplifydeltafunction}{SimplifyDeltaFunction}.

\subsection{Examples}

\begin{Shaded}
\begin{Highlighting}[]
\NormalTok{DeltaFunction}\OperatorTok{[}\DecValTok{1} \SpecialCharTok{{-}} \FunctionTok{x}\OperatorTok{]}
\end{Highlighting}
\end{Shaded}

\begin{dmath*}\breakingcomma
\delta (1-x)
\end{dmath*}

\begin{Shaded}
\begin{Highlighting}[]
\NormalTok{Integrate2}\OperatorTok{[}\NormalTok{DeltaFunction}\OperatorTok{[}\DecValTok{1} \SpecialCharTok{{-}} \FunctionTok{x}\OperatorTok{]} \FunctionTok{f}\OperatorTok{[}\FunctionTok{x}\OperatorTok{],} \OperatorTok{\{}\FunctionTok{x}\OperatorTok{,} \DecValTok{0}\OperatorTok{,} \DecValTok{1}\OperatorTok{\}]}
\end{Highlighting}
\end{Shaded}

\begin{dmath*}\breakingcomma
f(1)
\end{dmath*}

\begin{Shaded}
\begin{Highlighting}[]
\NormalTok{Integrate2}\OperatorTok{[}\NormalTok{DeltaFunction}\OperatorTok{[}\FunctionTok{x}\OperatorTok{]} \FunctionTok{f}\OperatorTok{[}\FunctionTok{x}\OperatorTok{],} \OperatorTok{\{}\FunctionTok{x}\OperatorTok{,} \DecValTok{0}\OperatorTok{,} \DecValTok{1}\OperatorTok{\}]}
\end{Highlighting}
\end{Shaded}

\begin{dmath*}\breakingcomma
f(0)
\end{dmath*}

\begin{Shaded}
\begin{Highlighting}[]
\NormalTok{Integrate2}\OperatorTok{[}\NormalTok{DeltaFunction}\OperatorTok{[}\DecValTok{1} \SpecialCharTok{{-}} \FunctionTok{x}\OperatorTok{]} \FunctionTok{f}\OperatorTok{[}\FunctionTok{x}\OperatorTok{],} \OperatorTok{\{}\FunctionTok{x}\OperatorTok{,} \DecValTok{0}\OperatorTok{,} \DecValTok{1}\OperatorTok{\}]}
\end{Highlighting}
\end{Shaded}

\begin{dmath*}\breakingcomma
f(1)
\end{dmath*}

\begin{Shaded}
\begin{Highlighting}[]
\NormalTok{Convolute}\OperatorTok{[}\NormalTok{DeltaFunction}\OperatorTok{[}\DecValTok{1} \SpecialCharTok{{-}} \FunctionTok{x}\OperatorTok{],} \FunctionTok{x}\OperatorTok{]} \OtherTok{/.}\NormalTok{ FCGV}\OperatorTok{[}\AttributeTok{z\_}\OperatorTok{]}\NormalTok{ :\textgreater{} }\FunctionTok{ToExpression}\OperatorTok{[}\FunctionTok{z}\OperatorTok{]}
\end{Highlighting}
\end{Shaded}

\begin{dmath*}\breakingcomma
-x \delta (1-x) \log (x)
\end{dmath*}
\end{document}
