% !TeX program = pdflatex
% !TeX root = DeltaFunctionPrime.tex

\documentclass[../FeynCalcManual.tex]{subfiles}
\begin{document}
\hypertarget{deltafunctionprime}{%
\section{DeltaFunctionPrime}\label{deltafunctionprime}}

\texttt{DeltaFunctionPrime[\allowbreak{}1 - x]} is the derivative of the
Dirac delta-function \(\delta (x)\).

\subsection{See also}

\hyperlink{toc}{Overview}, \hyperlink{convolute}{Convolute},
\hyperlink{deltafunction}{DeltaFunction},
\hyperlink{deltafunctiondoubleprime}{DeltaFunctionDoublePrime},
\hyperlink{integrate2}{Integrate2},
\hyperlink{simplifydeltafunction}{SimplifyDeltaFunction}.

\subsection{Examples}

\begin{Shaded}
\begin{Highlighting}[]
\NormalTok{DeltaFunctionPrime}\OperatorTok{[}\DecValTok{1} \SpecialCharTok{{-}} \FunctionTok{x}\OperatorTok{]}
\end{Highlighting}
\end{Shaded}

\begin{dmath*}\breakingcomma
\delta '(1-x)
\end{dmath*}

\begin{Shaded}
\begin{Highlighting}[]
\NormalTok{Integrate2}\OperatorTok{[}\NormalTok{DeltaFunctionPrime}\OperatorTok{[}\DecValTok{1} \SpecialCharTok{{-}} \FunctionTok{x}\OperatorTok{]} \FunctionTok{f}\OperatorTok{[}\FunctionTok{x}\OperatorTok{],} \OperatorTok{\{}\FunctionTok{x}\OperatorTok{,} \DecValTok{0}\OperatorTok{,} \DecValTok{1}\OperatorTok{\}]}
\end{Highlighting}
\end{Shaded}

\begin{dmath*}\breakingcomma
f'(1)
\end{dmath*}

\begin{Shaded}
\begin{Highlighting}[]
\NormalTok{Integrate2}\OperatorTok{[}\NormalTok{DeltaFunctionPrime}\OperatorTok{[}\DecValTok{1} \SpecialCharTok{{-}} \FunctionTok{x}\OperatorTok{]} \FunctionTok{x}\SpecialCharTok{\^{}}\DecValTok{2}\OperatorTok{,} \OperatorTok{\{}\FunctionTok{x}\OperatorTok{,} \DecValTok{0}\OperatorTok{,} \DecValTok{1}\OperatorTok{\}]}
\end{Highlighting}
\end{Shaded}

\begin{dmath*}\breakingcomma
2
\end{dmath*}
\end{document}
