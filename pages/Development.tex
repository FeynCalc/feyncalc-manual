% !TeX program = pdflatex
% !TeX root = Development.tex

\documentclass[../FeynCalcManual.tex]{subfiles}
\begin{document}
\hypertarget{development}{
\section{Development}\label{development}\index{Development}}

\subsection{See also}

\hyperlink{toc}{Overview}.

If you are using FeynCalc in your research activities on a regular
basis, sooner or later you will encounter situations, where FeynCalc is
missing something that you need. Provided that you have some knowledge
in Mathematica programming, you can in principle extend FeynCalc with
whatever you want, where the only limits are your skills and the amount
of time you want to invest into it.

Here we collect some tips and recommendations for extending/modifying
FeynCalc

\hypertarget{unit-tests}{%
\subsection{Unit tests}\label{unit-tests}}

To improve the code quality of FeynCalc and avoid new bugs or
regressions we employ a large number of
\href{https://en.wikipedia.org/wiki/Unit_testing}{unit tests}. The tests
are written using Mathematica's native framework \texttt{MUnit} and can
be found in the
\href{https://github.com/FeynCalc/feyncalc/tree/master/Tests}{Tests}
directory of the FeynCalc repository

The unit tests are useful not only for the developers. If you are
modifying the source code of FeynCalc for your own needs or using
FeynCalc together with other packages on the same kernel, it might be
useful to check if FeynCalc is still working properly. This can be done
by running the test suite and analyzing the results.

\hypertarget{modifying-the-source-code}{%
\subsection{Modifying the source code}\label{modifying-the-source-code}}

In general, it is not a very good idea to modify the source code of
FeynCalc directly, since this may give you a lot of troubles when
updating to a newer version. Also, there is a fair chance to break
existing functionality which might lead to erroneous results. However,
for some deep modifications this might be sometimes necessary.

Still, if you think that the feature that you want to implement could be
useful also for other people, it won't harm to publish your request on
the mailing list.

\begin{itemize}
\tightlist
\item
  The source code of FeynCalc is located in the
  \texttt{\$FeynCalcDirectory} directory. The package is loaded via the
  \texttt{FeynCalc.m} file.
\item
  In \texttt{FeynCalc.m} there is a code snippet that starts with
  \texttt{boostrappingList = ...}. This is where different parts of the
  FeynCalc package are loaded. Those parts are contained in different
  directories inside \texttt{\$FeynCalcDirectory} (e.g.~\texttt{Shared},
  \texttt{NonCommAlgebra}, \texttt{Lorentz}) depending on their purpose.
  Every .m file from those directories is loaded into Mathematica.
\end{itemize}

\hypertarget{add-ons}{%
\subsection{Add-ons}\label{add-ons}}

Starting with the version 9.0, FeynCalc features a new add-on system.
The purpose of add-ons is to extend FeynCalc with new features without
the need to modify FeynCalc itself.

The add-ons are placed in the directory \texttt{AddOns} inside
\texttt{\$FeynCalcDirectory}. They are loaded in the same fashion as
regular FeynCalc objects and functions and live in the context
\texttt{FeynCalc\textasciigrave NameOfTheAddon\textasciigrave }.

A toy add-on
\href{https://github.com/FeynCalc/feyncalc/tree/master/FeynCalc/AddOns/FVProjection}{FVProjection}
that represents a minimal working example is shipped together with
FeynCalc. You can use its source code as a template for writing your own
addons. The general structure of the main .m file
(e.g.~\texttt{MyAddon.m}) for an add-on looks like

\begin{Shaded}
\begin{Highlighting}[]

\NormalTok{Foo::usage}\ExtensionTok{=}
\StringTok{"Description of Foo"}\NormalTok{;}

\NormalTok{Bar::usage}\ExtensionTok{=}
\StringTok{"Description of Bar"}\NormalTok{;}

\NormalTok{$MyAddonVersion::usage}\ExtensionTok{=}
\StringTok{"$MyAddonVersion is the string that represents the version of MyAddon"}\NormalTok{;}

\NormalTok{$MyAddonDirectory::usage}\ExtensionTok{=}
\StringTok{"$MyAddonDirectory is the string that represents the full path to the MyAddon directory"}\NormalTok{;}

\FunctionTok{Begin}\OperatorTok{[}\StringTok{"\textasciigrave{}Package\textasciigrave{}"}\OperatorTok{]}
\FunctionTok{End}\OperatorTok{[]}

\NormalTok{$MyAddonVersion}\ExtensionTok{=}\StringTok{"1.0.0"}\NormalTok{;}

\NormalTok{$MyAddonDirectory }\ExtensionTok{=}
\NormalTok{ToFileName}\OperatorTok{[\{}\NormalTok{$FeynCalcDirectory}\OperatorTok{,} \StringTok{"AddOns"}\OperatorTok{,} \StringTok{"MyAddon"}\OperatorTok{\}]}\NormalTok{;}

\FunctionTok{Begin}\OperatorTok{[}\StringTok{"\textasciigrave{}MyAddon\textasciigrave{}Private\textasciigrave{}"}\OperatorTok{]}\NormalTok{;}

\NormalTok{Foo}\OperatorTok{[}\AttributeTok{x\_}\OperatorTok{,}\AttributeTok{y\_}\OperatorTok{]}\ExtensionTok{:=}\NormalTok{ Pair}\OperatorTok{[}\NormalTok{Momentum}\OperatorTok{[}\FunctionTok{x}\OperatorTok{],}\NormalTok{Momentum}\OperatorTok{[}\FunctionTok{y}\OperatorTok{]]}\NormalTok{;}
\NormalTok{Bar}\OperatorTok{[}\AttributeTok{x\_}\OperatorTok{,}\AttributeTok{y\_}\OperatorTok{]}\ExtensionTok{:=}\NormalTok{ DiracGamma}\OperatorTok{[}\NormalTok{LorentzIndex}\OperatorTok{[}\FunctionTok{x}\OperatorTok{]]}\NormalTok{.DiracGamma}\OperatorTok{[}\NormalTok{Momentum}\OperatorTok{[}\FunctionTok{y}\OperatorTok{]]}\NormalTok{;}

\CommentTok{(* Print startup message *)}
\FunctionTok{If}\OperatorTok{[}\NormalTok{ Global\textasciigrave{}$FeynCalcStartupMessages }\ExtensionTok{=}\NormalTok{!}\ExtensionTok{=} \ConstantTok{False}\OperatorTok{,}
    \FunctionTok{Print}\OperatorTok{[}\FunctionTok{Style}\OperatorTok{[}\StringTok{"MyAddon "}\OperatorTok{,} \StringTok{"Text"}\OperatorTok{,} \FunctionTok{Bold}\OperatorTok{],} \FunctionTok{Style}\OperatorTok{[}\NormalTok{$MyAddonVersion \textless{}\textgreater{} }\StringTok{" loaded."}\OperatorTok{,} \StringTok{"Text"}\OperatorTok{]]}
\OperatorTok{]}\NormalTok{;}


\FunctionTok{End}\OperatorTok{[]}
\end{Highlighting}
\end{Shaded}

\hypertarget{packages-that-use-feyncalc}{%
\subsection{Packages that use
FeynCalc}\label{packages-that-use-feyncalc}}

Starting with FeynCalc 9.0 one can easily write something like

\begin{Shaded}
\begin{Highlighting}[]
\FunctionTok{BeginPackage}\OperatorTok{[}\StringTok{"MyPackage\textasciigrave{}"}\OperatorTok{,\{}\StringTok{"FeynCalc\textasciigrave{}"}\OperatorTok{\}]}

\NormalTok{MySP::usage}\ExtensionTok{=}
\StringTok{"Description of MySP"}\NormalTok{;}

\FunctionTok{Begin}\OperatorTok{[}\StringTok{"\textasciigrave{}Private\textasciigrave{}"}\OperatorTok{]}\NormalTok{;}

\NormalTok{MySP}\OperatorTok{[}\AttributeTok{a\_}\OperatorTok{,}\AttributeTok{b\_}\OperatorTok{]}\ExtensionTok{:=}\NormalTok{ FCI}\OperatorTok{[}\NormalTok{SP}\OperatorTok{[}\FunctionTok{a}\OperatorTok{,}\FunctionTok{b}\OperatorTok{]]}\NormalTok{;}

\FunctionTok{End}\OperatorTok{[]}
\FunctionTok{EndPackage}\OperatorTok{[]}
\end{Highlighting}
\end{Shaded}

\hypertarget{package-or-add-on}{%
\subsection{Package or add-on?}\label{package-or-add-on}}

The decision, whether you want to write an add-on or a separate package
that requires FeynCalc should be made depending on your working style.

\begin{itemize}
\tightlist
\item
  If you use FeynCalc as your primary tool for doing computations, an
  add-on would be the best way to add things that you're missing.
\item
  If for you FeynCalc is only a part of a bigger framework, then a
  package that combines different tools (including FeynCalc) would
  probably be a better idea.
\end{itemize}
\end{document}
