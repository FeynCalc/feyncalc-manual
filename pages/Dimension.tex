% !TeX program = pdflatex
% !TeX root = Dimension.tex

\documentclass[../FeynCalcManual.tex]{subfiles}
\begin{document}
\hypertarget{dimension}{%
\section{Dimension}\label{dimension}}

\texttt{Dimension} is an option of several functions and denotes the
number of space-time dimensions. Possible settings are: \texttt{4},
\texttt{n}, \texttt{d}, \texttt{D}, \ldots{} , the variable does not
matter, but it should have head Symbol.

\subsection{See also}

\hyperlink{toc}{Overview}, \hyperlink{scalarproduct}{ScalarProduct}.

\subsection{Examples}

\begin{Shaded}
\begin{Highlighting}[]
\FunctionTok{Options}\OperatorTok{[}\NormalTok{ScalarProduct}\OperatorTok{]}
\end{Highlighting}
\end{Shaded}

\begin{dmath*}\breakingcomma
\{\text{Dimension}\to 4,\text{FeynCalcInternal}\to \;\text{True},\text{SetDimensions}\to \{4,D\}\}
\end{dmath*}

\begin{Shaded}
\begin{Highlighting}[]
\NormalTok{ex }\ExtensionTok{=}\NormalTok{ ScalarProduct}\OperatorTok{[}\FunctionTok{m}\OperatorTok{,} \FunctionTok{n}\OperatorTok{,}\NormalTok{ Dimension }\OtherTok{{-}\textgreater{}} \FunctionTok{d}\OperatorTok{]}
\end{Highlighting}
\end{Shaded}

\begin{dmath*}\breakingcomma
m\cdot n
\end{dmath*}

\begin{Shaded}
\begin{Highlighting}[]
\NormalTok{ex }\SpecialCharTok{//} \FunctionTok{StandardForm}

\CommentTok{(*Pair[Momentum[m, d], Momentum[n, d]]*)}
\end{Highlighting}
\end{Shaded}

\end{document}
