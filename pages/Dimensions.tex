% !TeX program = pdflatex
% !TeX root = Dimensions.tex

\documentclass[../FeynCalcManual.tex]{subfiles}
\begin{document}
\hypertarget{dimensions}{
\section{Dimensions}\label{dimensions}\index{Dimensions}}

\subsection{See also}

\hyperlink{toc}{Overview}.

\subsection{Notation for tensors living in different
dimensions}\label{notation-for-tensors-living-in-different-dimensions}

You might have wondered why 4-vectors, scalar products and Dirac
matrices all have a bar, like \(\bar{p}^\mu\) or
\(\bar{p} \cdot \bar{q}\). The bar is there to specify that they are
4-dimensional objects. Objects that live in \(D\) dimensions do not have
a bar, cf.

\begin{Shaded}
\begin{Highlighting}[]
\NormalTok{FVD}\OperatorTok{[}\FunctionTok{p}\OperatorTok{,} \SpecialCharTok{\textbackslash{}}\OperatorTok{[}\NormalTok{Mu}\OperatorTok{]]}
\SpecialCharTok{\%} \SpecialCharTok{//}\NormalTok{ FCI }\SpecialCharTok{//} \FunctionTok{StandardForm}
\end{Highlighting}
\end{Shaded}

\begin{dmath*}\breakingcomma
p^{\mu }
\end{dmath*}

\begin{Shaded}
\begin{Highlighting}[]
\CommentTok{(*Pair[LorentzIndex[\textbackslash{}[Mu], D], Momentum[p, D]]*)}
\end{Highlighting}
\end{Shaded}

\begin{Shaded}
\begin{Highlighting}[]
\NormalTok{MTD}\OperatorTok{[}\SpecialCharTok{\textbackslash{}}\OperatorTok{[}\NormalTok{Mu}\OperatorTok{],} \SpecialCharTok{\textbackslash{}}\OperatorTok{[}\NormalTok{Nu}\OperatorTok{]]}
\SpecialCharTok{\%} \SpecialCharTok{//}\NormalTok{ FCI }\SpecialCharTok{//} \FunctionTok{StandardForm}
\end{Highlighting}
\end{Shaded}

\begin{dmath*}\breakingcomma
g^{\mu \nu }
\end{dmath*}

\begin{Shaded}
\begin{Highlighting}[]
\CommentTok{(*Pair[LorentzIndex[\textbackslash{}[Mu], D], LorentzIndex[\textbackslash{}[Nu], D]]*)}
\end{Highlighting}
\end{Shaded}

This origin of this notation is a
\href{https://inspirehep.net/record/124212}{publication} by
Breitenlohner and Maison on the treatment of \(\gamma^5\) in \(D\)
dimensions in the t'Hooft-Veltman scheme. The main idea was that we can
decompose indexed objects into \(4\)- and \(D-4\)-dimensional pieces,
e.g.~\(p^\mu = \bar{p}^\mu + \hat{p}^\mu\). Consequently, in FeynCalc we
can also enter \(D-4\)-dimensional objects

\begin{Shaded}
\begin{Highlighting}[]
\NormalTok{FVE}\OperatorTok{[}\FunctionTok{p}\OperatorTok{,} \SpecialCharTok{\textbackslash{}}\OperatorTok{[}\NormalTok{Mu}\OperatorTok{]]}
\SpecialCharTok{\%} \SpecialCharTok{//}\NormalTok{ FCI }\SpecialCharTok{//} \FunctionTok{StandardForm}
\end{Highlighting}
\end{Shaded}

\begin{dmath*}\breakingcomma
\hat{p}^{\mu }
\end{dmath*}

\begin{Shaded}
\begin{Highlighting}[]
\CommentTok{(*Pair[LorentzIndex[\textbackslash{}[Mu], {-}4 + D], Momentum[p, {-}4 + D]]*)}
\end{Highlighting}
\end{Shaded}

\begin{Shaded}
\begin{Highlighting}[]
\NormalTok{MTE}\OperatorTok{[}\FunctionTok{p}\OperatorTok{,} \FunctionTok{q}\OperatorTok{]}
\SpecialCharTok{\%} \SpecialCharTok{//}\NormalTok{ FCI }\SpecialCharTok{//} \FunctionTok{StandardForm}
\end{Highlighting}
\end{Shaded}

\begin{dmath*}\breakingcomma
\hat{g}^{pq}
\end{dmath*}

\begin{Shaded}
\begin{Highlighting}[]
\CommentTok{(*Pair[LorentzIndex[p, {-}4 + D], LorentzIndex[q, {-}4 + D]]*)}
\end{Highlighting}
\end{Shaded}

When we contract Lorentz tensors from different dimensions, the
contractions are resolved according to the rules from the paper of
Breitenlohner and Maison, e. g.

\begin{Shaded}
\begin{Highlighting}[]
\NormalTok{FVD}\OperatorTok{[}\FunctionTok{p}\OperatorTok{,} \SpecialCharTok{\textbackslash{}}\OperatorTok{[}\NormalTok{Mu}\OperatorTok{]]}\NormalTok{ FV}\OperatorTok{[}\FunctionTok{q}\OperatorTok{,} \SpecialCharTok{\textbackslash{}}\OperatorTok{[}\NormalTok{Mu}\OperatorTok{]]}
\NormalTok{Contract}\OperatorTok{[}\SpecialCharTok{\%}\OperatorTok{]}
\end{Highlighting}
\end{Shaded}

\begin{dmath*}\breakingcomma
p^{\mu } \overline{q}^{\mu }
\end{dmath*}

\begin{dmath*}\breakingcomma
\overline{p}\cdot \overline{q}
\end{dmath*}

\begin{Shaded}
\begin{Highlighting}[]
\NormalTok{FV}\OperatorTok{[}\FunctionTok{p}\OperatorTok{,} \SpecialCharTok{\textbackslash{}}\OperatorTok{[}\NormalTok{Mu}\OperatorTok{]]}\NormalTok{ FVE}\OperatorTok{[}\FunctionTok{q}\OperatorTok{,} \SpecialCharTok{\textbackslash{}}\OperatorTok{[}\NormalTok{Mu}\OperatorTok{]]}
\NormalTok{Contract}\OperatorTok{[}\SpecialCharTok{\%}\OperatorTok{]}
\end{Highlighting}
\end{Shaded}

\begin{dmath*}\breakingcomma
\hat{q}^{\mu } \overline{p}^{\mu }
\end{dmath*}

\begin{dmath*}\breakingcomma
0
\end{dmath*}

\begin{Shaded}
\begin{Highlighting}[]
\NormalTok{(FVD}\OperatorTok{[}\FunctionTok{p}\OperatorTok{,} \SpecialCharTok{\textbackslash{}}\OperatorTok{[}\NormalTok{Mu}\OperatorTok{]]} \SpecialCharTok{+}\NormalTok{ FVE}\OperatorTok{[}\FunctionTok{p}\OperatorTok{,} \SpecialCharTok{\textbackslash{}}\OperatorTok{[}\NormalTok{Mu}\OperatorTok{]]}\NormalTok{) (FVD}\OperatorTok{[}\FunctionTok{q}\OperatorTok{,} \SpecialCharTok{\textbackslash{}}\OperatorTok{[}\NormalTok{Mu}\OperatorTok{]]} \SpecialCharTok{+}\NormalTok{ FVE}\OperatorTok{[}\FunctionTok{q}\OperatorTok{,} \SpecialCharTok{\textbackslash{}}\OperatorTok{[}\NormalTok{Mu}\OperatorTok{]]}\NormalTok{)}
\NormalTok{Contract}\OperatorTok{[}\SpecialCharTok{\%}\OperatorTok{]}
\end{Highlighting}
\end{Shaded}

\begin{dmath*}\breakingcomma
\left(p^{\mu }+\hat{p}^{\mu }\right) \left(q^{\mu }+\hat{q}^{\mu }\right)
\end{dmath*}

\begin{dmath*}\breakingcomma
3 \left(\hat{p}\cdot \hat{q}\right)+p\cdot q
\end{dmath*}

Sometimes we need to switch from one dimension to another, e.g.~to
convert a 4-dimensional object to a \(D\)-dimensional one or vice versa.
This is done via

\begin{Shaded}
\begin{Highlighting}[]
\NormalTok{FVD}\OperatorTok{[}\FunctionTok{p}\OperatorTok{,} \SpecialCharTok{\textbackslash{}}\OperatorTok{[}\NormalTok{Mu}\OperatorTok{]]}
\NormalTok{ChangeDimension}\OperatorTok{[}\SpecialCharTok{\%}\OperatorTok{,} \DecValTok{4}\OperatorTok{]}
\end{Highlighting}
\end{Shaded}

\begin{dmath*}\breakingcomma
p^{\mu }
\end{dmath*}

\begin{dmath*}\breakingcomma
\overline{p}^{\mu }
\end{dmath*}

The second argument of \texttt{ChangeDimension} is the new dimension .
The most common choices are \(4\), \(D\) or \(D-4\)

\begin{Shaded}
\begin{Highlighting}[]
\NormalTok{FVD}\OperatorTok{[}\FunctionTok{p}\OperatorTok{,} \SpecialCharTok{\textbackslash{}}\OperatorTok{[}\NormalTok{Mu}\OperatorTok{]]}
\NormalTok{ChangeDimension}\OperatorTok{[}\SpecialCharTok{\%}\OperatorTok{,} \FunctionTok{D} \SpecialCharTok{{-}} \DecValTok{4}\OperatorTok{]}
\end{Highlighting}
\end{Shaded}

\begin{dmath*}\breakingcomma
p^{\mu }
\end{dmath*}

\begin{dmath*}\breakingcomma
\hat{p}^{\mu }
\end{dmath*}

\begin{Shaded}
\begin{Highlighting}[]
\NormalTok{SP}\OperatorTok{[}\FunctionTok{p}\OperatorTok{,} \FunctionTok{q}\OperatorTok{]}
\NormalTok{ChangeDimension}\OperatorTok{[}\SpecialCharTok{\%}\OperatorTok{,} \FunctionTok{D}\OperatorTok{]}
\end{Highlighting}
\end{Shaded}

\begin{dmath*}\breakingcomma
\overline{p}\cdot \overline{q}
\end{dmath*}

\begin{dmath*}\breakingcomma
p\cdot q
\end{dmath*}

To check the dimension of the given expression one can use
\texttt{FCGetDimensions}

\begin{Shaded}
\begin{Highlighting}[]
\NormalTok{FVD}\OperatorTok{[}\FunctionTok{p}\OperatorTok{,} \SpecialCharTok{\textbackslash{}}\OperatorTok{[}\NormalTok{Mu}\OperatorTok{]]}\NormalTok{ FV}\OperatorTok{[}\FunctionTok{q}\OperatorTok{,} \SpecialCharTok{\textbackslash{}}\OperatorTok{[}\NormalTok{Mu}\OperatorTok{]]}
\NormalTok{FCGetDimensions}\OperatorTok{[}\SpecialCharTok{\%}\OperatorTok{,} \OperatorTok{\{\}]}
\end{Highlighting}
\end{Shaded}

\begin{dmath*}\breakingcomma
p^{\mu } \overline{q}^{\mu }
\end{dmath*}

\begin{dmath*}\breakingcomma
\{4,D\}
\end{dmath*}

If one needs to replace the dimensional symbols \texttt{D} in the
prefactors of the Lorentz tensors, it is better to use
\texttt{FCReplaceD} instead of a replacement rule. Otherwise, the
dimensions of the tensors will get messed up

\begin{Shaded}
\begin{Highlighting}[]
\NormalTok{FCI}\OperatorTok{[}\NormalTok{(}\FunctionTok{D} \SpecialCharTok{+} \DecValTok{2}\NormalTok{) MTD}\OperatorTok{[}\SpecialCharTok{\textbackslash{}}\OperatorTok{[}\NormalTok{Mu}\OperatorTok{],} \SpecialCharTok{\textbackslash{}}\OperatorTok{[}\NormalTok{Nu}\OperatorTok{]]]} 
\SpecialCharTok{\%} \OtherTok{/.} \FunctionTok{D} \OtherTok{{-}\textgreater{}} \DecValTok{4} \SpecialCharTok{{-}} \DecValTok{2}\NormalTok{ Epsilon}
\end{Highlighting}
\end{Shaded}

\begin{dmath*}\breakingcomma
(D+2) g^{\mu \nu }
\end{dmath*}

\begin{dmath*}\breakingcomma
(6-2 \varepsilon ) g_{\{4-2 \varepsilon ,4-2 \varepsilon \}}{}^{\mu \nu }
\end{dmath*}

```mathematica (D + 2) MTD{[}{[}Mu{]}, {[}Nu{]}{]} FCReplaceD{[}\%, D
-\textgreater{} 4 - 2 Epsilon{]}

```mathematica

\begin{dmath*}\breakingcomma
(D+2) g^{\mu \nu }
\end{dmath*}

\begin{dmath*}\breakingcomma
(6-2 \varepsilon ) g^{\mu \nu }
\end{dmath*}
\end{document}
