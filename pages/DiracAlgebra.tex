% !TeX program = pdflatex
% !TeX root = DiracAlgebra.tex

\documentclass[../FeynCalcManual.tex]{subfiles}
\begin{document}
\begin{Shaded}
\begin{Highlighting}[]
 
\end{Highlighting}
\end{Shaded}

\hypertarget{dirac algebra}{
\section{Dirac algebra}\label{dirac algebra}\index{Dirac algebra}}

\subsection{See also}

\hyperlink{toc}{Overview}.

\hypertarget{simplifications}{%
\subsection{Simplifications}\label{simplifications}}

The two most relevant functions for the manipulations of Dirac matrices
are \texttt{DiracSimplify} and \texttt{DiracTrace}.

The goal of \texttt{DiracSimplify} is to eliminate all pairs of Dirac
matrices with the equal indices or contracted with the same
\(4\)-vectors

\begin{Shaded}
\begin{Highlighting}[]
\NormalTok{GA}\OperatorTok{[}\SpecialCharTok{\textbackslash{}}\OperatorTok{[}\NormalTok{Mu}\OperatorTok{]]}\NormalTok{ . GS}\OperatorTok{[}\FunctionTok{p} \SpecialCharTok{+} \FunctionTok{m}\OperatorTok{]}\NormalTok{ . GA}\OperatorTok{[}\SpecialCharTok{\textbackslash{}}\OperatorTok{[}\NormalTok{Mu}\OperatorTok{]]}
\NormalTok{DiracSimplify}\OperatorTok{[}\SpecialCharTok{\%}\OperatorTok{]}
\end{Highlighting}
\end{Shaded}

\begin{dmath*}\breakingcomma
\bar{\gamma }^{\mu }.\left(\bar{\gamma }\cdot \left(\overline{m}+\overline{p}\right)\right).\bar{\gamma }^{\mu }
\end{dmath*}

\begin{dmath*}\breakingcomma
-2 \bar{\gamma }\cdot \overline{m}-2 \bar{\gamma }\cdot \overline{p}
\end{dmath*}

\begin{Shaded}
\begin{Highlighting}[]
\NormalTok{GA}\OperatorTok{[}\SpecialCharTok{\textbackslash{}}\OperatorTok{[}\NormalTok{Mu}\OperatorTok{]]}\NormalTok{ . GS}\OperatorTok{[}\FunctionTok{p} \SpecialCharTok{+}\NormalTok{ m1}\OperatorTok{]}\NormalTok{ . GA}\OperatorTok{[}\SpecialCharTok{\textbackslash{}}\OperatorTok{[}\NormalTok{Nu}\OperatorTok{]]}\NormalTok{ . GS}\OperatorTok{[}\FunctionTok{p} \SpecialCharTok{+}\NormalTok{ m2}\OperatorTok{]}
\NormalTok{DiracSimplify}\OperatorTok{[}\SpecialCharTok{\%}\OperatorTok{]}
\end{Highlighting}
\end{Shaded}

\begin{dmath*}\breakingcomma
\bar{\gamma }^{\mu }.\left(\bar{\gamma }\cdot \left(\overline{\text{m1}}+\overline{p}\right)\right).\bar{\gamma }^{\nu }.\left(\bar{\gamma }\cdot \left(\overline{\text{m2}}+\overline{p}\right)\right)
\end{dmath*}

\begin{dmath*}\breakingcomma
\bar{\gamma }^{\mu }.\left(\bar{\gamma }\cdot \overline{\text{m1}}\right).\bar{\gamma }^{\nu }.\left(\bar{\gamma }\cdot \overline{\text{m2}}\right)+\bar{\gamma }^{\mu }.\left(\bar{\gamma }\cdot \overline{\text{m1}}\right).\bar{\gamma }^{\nu }.\left(\bar{\gamma }\cdot \overline{p}\right)+\bar{\gamma }^{\mu }.\left(\bar{\gamma }\cdot \overline{p}\right).\bar{\gamma }^{\nu }.\left(\bar{\gamma }\cdot \overline{\text{m2}}\right)-\overline{p}^2 \bar{\gamma }^{\mu }.\bar{\gamma }^{\nu }+2 \overline{p}^{\nu } \bar{\gamma }^{\mu }.\left(\bar{\gamma }\cdot \overline{p}\right)
\end{dmath*}

\texttt{DiracTrace} is used for the evaluation of Dirac traces. The
trace is not evaluated by default

\begin{Shaded}
\begin{Highlighting}[]
\NormalTok{DiracTrace}\OperatorTok{[}\NormalTok{GA}\OperatorTok{[}\SpecialCharTok{\textbackslash{}}\OperatorTok{[}\NormalTok{Mu}\OperatorTok{],} \SpecialCharTok{\textbackslash{}}\OperatorTok{[}\NormalTok{Nu}\OperatorTok{]]]}
\end{Highlighting}
\end{Shaded}

\begin{dmath*}\breakingcomma
\text{tr}\left(\bar{\gamma }^{\mu }.\bar{\gamma }^{\nu }\right)
\end{dmath*}

To obtain the result we can either use the option
\texttt{DiracTraceEvaluate}

\begin{Shaded}
\begin{Highlighting}[]
\NormalTok{DiracTrace}\OperatorTok{[}\NormalTok{GA}\OperatorTok{[}\SpecialCharTok{\textbackslash{}}\OperatorTok{[}\NormalTok{Mu}\OperatorTok{],} \SpecialCharTok{\textbackslash{}}\OperatorTok{[}\NormalTok{Nu}\OperatorTok{]],}\NormalTok{ DiracTraceEvaluate }\OtherTok{{-}\textgreater{}} \ConstantTok{True}\OperatorTok{]}
\end{Highlighting}
\end{Shaded}

\begin{dmath*}\breakingcomma
4 \bar{g}^{\mu \nu }
\end{dmath*}

or use \texttt{DiracSimplify} instead.

By default FeynCalc refuses to compute a \(D\)-dimensional trace that
contains \(\gamma^5\)

\begin{Shaded}
\begin{Highlighting}[]
\NormalTok{DiracTrace}\OperatorTok{[}\NormalTok{GAD}\OperatorTok{[}\SpecialCharTok{\textbackslash{}}\OperatorTok{[}\NormalTok{Alpha}\OperatorTok{],} \SpecialCharTok{\textbackslash{}}\OperatorTok{[}\FunctionTok{Beta}\OperatorTok{],} \SpecialCharTok{\textbackslash{}}\OperatorTok{[}\NormalTok{Mu}\OperatorTok{],} \SpecialCharTok{\textbackslash{}}\OperatorTok{[}\NormalTok{Nu}\OperatorTok{],} \SpecialCharTok{\textbackslash{}}\OperatorTok{[}\NormalTok{Rho}\OperatorTok{],} \SpecialCharTok{\textbackslash{}}\OperatorTok{[}\NormalTok{Sigma}\OperatorTok{],} \DecValTok{5}\OperatorTok{]]} \SpecialCharTok{//}\NormalTok{ DiracSimplify}
\end{Highlighting}
\end{Shaded}

\begin{dmath*}\breakingcomma
\text{tr}\left(\gamma ^{\alpha }.\gamma ^{\beta }.\gamma ^{\mu }.\gamma ^{\nu }.\gamma ^{\rho }.\gamma ^{\sigma }.\bar{\gamma }^5\right)
\end{dmath*}

This is because by default FeynCalc is using anticommuting \(\gamma^5\)
in \(D\)-dimensions, a scheme known as Naive Dimensional Regularization
(NDR)

\begin{Shaded}
\begin{Highlighting}[]
\NormalTok{DiracSimplify}\OperatorTok{[}\NormalTok{GAD}\OperatorTok{[}\SpecialCharTok{\textbackslash{}}\OperatorTok{[}\NormalTok{Mu}\OperatorTok{]]}\NormalTok{ . GA}\OperatorTok{[}\DecValTok{5}\OperatorTok{]}\NormalTok{ . GAD}\OperatorTok{[}\SpecialCharTok{\textbackslash{}}\OperatorTok{[}\NormalTok{Nu}\OperatorTok{]]]}
\end{Highlighting}
\end{Shaded}

\begin{dmath*}\breakingcomma
-\gamma ^{\mu }.\gamma ^{\nu }.\bar{\gamma }^5
\end{dmath*}

In general, a chiral trace is a very ambiguous object in NDR. The
results depends on the position of \(\gamma^5\) inside the trace, so
that we chose not to produce results that might be potentially
inconsistent. However, FeynCalc can also be told to use the
Breitenlohner-Maison-t'Hooft-Veltman scheme (BMHV), which is an
algebraically consistent scheme (but has other issues, e.g.~it breaks
Ward identities)

\begin{Shaded}
\begin{Highlighting}[]
\NormalTok{FCSetDiracGammaScheme}\OperatorTok{[}\StringTok{"BMHV"}\OperatorTok{]}\NormalTok{;}
\end{Highlighting}
\end{Shaded}

Notice that now FeynCalc anticommutes \(\gamma^5\) according to the BMHV
algebra, which leads to the appearance of \(D-4\)-dimensional Dirac
matrices

\begin{Shaded}
\begin{Highlighting}[]
\NormalTok{DiracSimplify}\OperatorTok{[}\NormalTok{GAD}\OperatorTok{[}\SpecialCharTok{\textbackslash{}}\OperatorTok{[}\NormalTok{Mu}\OperatorTok{]]}\NormalTok{ . GA}\OperatorTok{[}\DecValTok{5}\OperatorTok{]}\NormalTok{ . GAD}\OperatorTok{[}\SpecialCharTok{\textbackslash{}}\OperatorTok{[}\NormalTok{Nu}\OperatorTok{]]]}
\end{Highlighting}
\end{Shaded}

\begin{dmath*}\breakingcomma
2 \gamma ^{\mu }.\hat{\gamma }^{\nu }.\bar{\gamma }^5-\gamma ^{\mu }.\gamma ^{\nu }.\bar{\gamma }^5
\end{dmath*}

Also Dirac traces are not an issue now

\begin{Shaded}
\begin{Highlighting}[]
\NormalTok{DiracTrace}\OperatorTok{[}\NormalTok{GAD}\OperatorTok{[}\SpecialCharTok{\textbackslash{}}\OperatorTok{[}\NormalTok{Alpha}\OperatorTok{],} \SpecialCharTok{\textbackslash{}}\OperatorTok{[}\FunctionTok{Beta}\OperatorTok{],} \SpecialCharTok{\textbackslash{}}\OperatorTok{[}\NormalTok{Mu}\OperatorTok{],} \SpecialCharTok{\textbackslash{}}\OperatorTok{[}\NormalTok{Nu}\OperatorTok{],} \SpecialCharTok{\textbackslash{}}\OperatorTok{[}\NormalTok{Rho}\OperatorTok{],} \SpecialCharTok{\textbackslash{}}\OperatorTok{[}\NormalTok{Sigma}\OperatorTok{]]}\NormalTok{ . GA}\OperatorTok{[}\DecValTok{5}\OperatorTok{]]} \SpecialCharTok{//}\NormalTok{ DiracSimplify}
\end{Highlighting}
\end{Shaded}

\begin{dmath*}\breakingcomma
-4 i g^{\alpha \beta } \bar{\epsilon }^{\mu \nu \rho \sigma }+4 i g^{\alpha \mu } \bar{\epsilon }^{\beta \nu \rho \sigma }-4 i g^{\alpha \nu } \bar{\epsilon }^{\beta \mu \rho \sigma }+4 i g^{\alpha \rho } \bar{\epsilon }^{\beta \mu \nu \sigma }-4 i g^{\alpha \sigma } \bar{\epsilon }^{\beta \mu \nu \rho }-4 i g^{\beta \mu } \bar{\epsilon }^{\alpha \nu \rho \sigma }+4 i g^{\beta \nu } \bar{\epsilon }^{\alpha \mu \rho \sigma }-4 i g^{\beta \rho } \bar{\epsilon }^{\alpha \mu \nu \sigma }+4 i g^{\beta \sigma } \bar{\epsilon }^{\alpha \mu \nu \rho }-4 i g^{\mu \nu } \bar{\epsilon }^{\alpha \beta \rho \sigma }+4 i g^{\mu \rho } \bar{\epsilon }^{\alpha \beta \nu \sigma }-4 i g^{\mu \sigma } \bar{\epsilon }^{\alpha \beta \nu \rho }-4 i g^{\nu \rho } \bar{\epsilon }^{\alpha \beta \mu \sigma }+4 i g^{\nu \sigma } \bar{\epsilon }^{\alpha \beta \mu \rho }-4 i g^{\rho \sigma } \bar{\epsilon }^{\alpha \beta \mu \nu }
\end{dmath*}

To compute chiral traces in the BMHV scheme, FeynCalc uses
\href{https://inspirehep.net/record/31057}{West's formula}. Still, NDR
is the default scheme in FeynCalc.

In tree-level calculation a useful operation is the so-called
SPVAT-decomposition of Dirac chains. This is done using
\texttt{DiracReduce}

\begin{Shaded}
\begin{Highlighting}[]
\NormalTok{GA}\OperatorTok{[}\SpecialCharTok{\textbackslash{}}\OperatorTok{[}\NormalTok{Mu}\OperatorTok{],} \SpecialCharTok{\textbackslash{}}\OperatorTok{[}\NormalTok{Nu}\OperatorTok{],} \SpecialCharTok{\textbackslash{}}\OperatorTok{[}\NormalTok{Rho}\OperatorTok{]]}\NormalTok{ . GS}\OperatorTok{[}\FunctionTok{p}\OperatorTok{]}\NormalTok{ . GA}\OperatorTok{[}\SpecialCharTok{\textbackslash{}}\OperatorTok{[}\NormalTok{Alpha}\OperatorTok{]]}
\NormalTok{DiracReduce}\OperatorTok{[}\SpecialCharTok{\%}\OperatorTok{]}
\end{Highlighting}
\end{Shaded}

\begin{dmath*}\breakingcomma
\bar{\gamma }^{\mu }.\bar{\gamma }^{\nu }.\bar{\gamma }^{\rho }.\left(\bar{\gamma }\cdot \overline{p}\right).\bar{\gamma }^{\alpha }
\end{dmath*}

\begin{dmath*}\breakingcomma
-i \bar{g}^{\mu \nu } \bar{\gamma }^{\text{\$MU}(\text{\$68})}.\bar{\gamma }^5 \bar{\epsilon }^{\alpha \rho \;\text{\$MU}(\text{\$68})\overline{p}}+i \bar{g}^{\alpha \rho } \bar{\gamma }^{\text{\$MU}(\text{\$70})}.\bar{\gamma }^5 \bar{\epsilon }^{\mu \nu \;\text{\$MU}(\text{\$70})\overline{p}}+i \overline{p}^{\alpha } \bar{\gamma }^{\text{\$MU}(\text{\$71})}.\bar{\gamma }^5 \bar{\epsilon }^{\mu \nu \rho \;\text{\$MU}(\text{\$71})}+i \overline{p}^{\rho } \bar{\gamma }^{\text{\$MU}(\text{\$72})}.\bar{\gamma }^5 \bar{\epsilon }^{\alpha \mu \nu \;\text{\$MU}(\text{\$72})}+\bar{\gamma }^{\rho } \overline{p}^{\alpha } \bar{g}^{\mu \nu }-\bar{\gamma }^{\nu } \overline{p}^{\alpha } \bar{g}^{\mu \rho }-\bar{\gamma }^{\rho } \overline{p}^{\mu } \bar{g}^{\alpha \nu }+\bar{\gamma }^{\nu } \overline{p}^{\mu } \bar{g}^{\alpha \rho }+\bar{\gamma }^{\mu } \overline{p}^{\alpha } \bar{g}^{\nu \rho }+\bar{\gamma }^{\alpha } \overline{p}^{\mu } \bar{g}^{\nu \rho }+\bar{\gamma }^{\rho } \overline{p}^{\nu } \bar{g}^{\alpha \mu }-\bar{\gamma }^{\mu } \overline{p}^{\nu } \bar{g}^{\alpha \rho }-\bar{\gamma }^{\alpha } \overline{p}^{\nu } \bar{g}^{\mu \rho }-\bar{\gamma }^{\nu } \overline{p}^{\rho } \bar{g}^{\alpha \mu }+\bar{\gamma }^{\mu } \overline{p}^{\rho } \bar{g}^{\alpha \nu }+\bar{\gamma }^{\alpha } \overline{p}^{\rho } \bar{g}^{\mu \nu }-\bar{g}^{\alpha \rho } \bar{g}^{\mu \nu } \bar{\gamma }\cdot \overline{p}+\bar{g}^{\alpha \nu } \bar{g}^{\mu \rho } \bar{\gamma }\cdot \overline{p}-\bar{g}^{\alpha \mu } \bar{g}^{\nu \rho } \bar{\gamma }\cdot \overline{p}-i \bar{\gamma }^{\nu }.\bar{\gamma }^5 \bar{\epsilon }^{\alpha \mu \rho \overline{p}}+i \bar{\gamma }^{\mu }.\bar{\gamma }^5 \bar{\epsilon }^{\alpha \nu \rho \overline{p}}
\end{dmath*}

Gordon's identities are implemented via \texttt{GordonSimplify}

\begin{Shaded}
\begin{Highlighting}[]
\NormalTok{SpinorUBar}\OperatorTok{[}\NormalTok{p1}\OperatorTok{,}\NormalTok{ m1}\OperatorTok{]}\NormalTok{ . GA}\OperatorTok{[}\SpecialCharTok{\textbackslash{}}\OperatorTok{[}\NormalTok{Mu}\OperatorTok{]]}\NormalTok{ . SpinorU}\OperatorTok{[}\NormalTok{p2}\OperatorTok{,}\NormalTok{ m2}\OperatorTok{]}
\NormalTok{GordonSimplify}\OperatorTok{[}\SpecialCharTok{\%}\OperatorTok{]}
\end{Highlighting}
\end{Shaded}

\begin{dmath*}\breakingcomma
\bar{u}(\text{p1},\text{m1}).\bar{\gamma }^{\mu }.u(\text{p2},\text{m2})
\end{dmath*}

\begin{dmath*}\breakingcomma
\frac{\left(\overline{\text{p1}}+\overline{\text{p2}}\right)^{\mu } \left(\varphi (\overline{\text{p1}},\text{m1})\right).\left(\varphi (\overline{\text{p2}},\text{m2})\right)}{\text{m1}+\text{m2}}+\frac{i \left(\varphi (\overline{\text{p1}},\text{m1})\right).\sigma ^{\mu \overline{\text{p1}}-\overline{\text{p2}}}.\left(\varphi (\overline{\text{p2}},\text{m2})\right)}{\text{m1}+\text{m2}}
\end{dmath*}
\end{document}
