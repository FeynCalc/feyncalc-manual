% !TeX program = pdflatex
% !TeX root = DiracChain.tex

\documentclass[../FeynCalcManual.tex]{subfiles}
\begin{document}
\hypertarget{diracchain}{%
\section{DiracChain}\label{diracchain}}

\texttt{DiracChain[\allowbreak{}x,\ \allowbreak{}i,\ \allowbreak{}j]}
denotes a chain of Dirac matrices \texttt{x}, where the Dirac indices
\texttt{i} and \texttt{j} are explicit.

\subsection{See also}

\hyperlink{toc}{Overview}, \hyperlink{diracchain}{DiracChain},
\hyperlink{dchn}{DCHN}, \hyperlink{diracindex}{DiracIndex},
\hyperlink{diracindexdelta}{DiracIndexDelta},
\hyperlink{diracchainjoin}{DiracChainJoin},
\hyperlink{diracchainexpand}{DiracChainExpand},
\hyperlink{diracchainfactor}{DiracChainFactor}.

\subsection{Examples}

A standalone Dirac matrix

\begin{Shaded}
\begin{Highlighting}[]
\NormalTok{DiracChain}\OperatorTok{[}\NormalTok{DiracGamma}\OperatorTok{[}\NormalTok{LorentzIndex}\OperatorTok{[}\SpecialCharTok{\textbackslash{}}\OperatorTok{[}\NormalTok{Mu}\OperatorTok{]]],}\NormalTok{ DiracIndex}\OperatorTok{[}\FunctionTok{i}\OperatorTok{],}\NormalTok{ DiracIndex}\OperatorTok{[}\FunctionTok{j}\OperatorTok{]]}
\end{Highlighting}
\end{Shaded}

\begin{dmath*}\breakingcomma
\left(\bar{\gamma }^{\mu }\right){}_{ij}
\end{dmath*}

A chain of Dirac matrices with open indices

\begin{Shaded}
\begin{Highlighting}[]
\NormalTok{DiracChain}\OperatorTok{[}\NormalTok{DiracGamma}\OperatorTok{[}\NormalTok{LorentzIndex}\OperatorTok{[}\SpecialCharTok{\textbackslash{}}\OperatorTok{[}\NormalTok{Mu}\OperatorTok{],} \FunctionTok{D}\OperatorTok{],} \FunctionTok{D}\OperatorTok{]}\NormalTok{ . DiracGamma}\OperatorTok{[}\NormalTok{LorentzIndex}\OperatorTok{[}\SpecialCharTok{\textbackslash{}}\OperatorTok{[}\NormalTok{Nu}\OperatorTok{],} \FunctionTok{D}\OperatorTok{],} \FunctionTok{D}\OperatorTok{],}\NormalTok{ DiracIndex}\OperatorTok{[}\FunctionTok{i}\OperatorTok{],}\NormalTok{ DiracIndex}\OperatorTok{[}\FunctionTok{j}\OperatorTok{]]}
\end{Highlighting}
\end{Shaded}

\begin{dmath*}\breakingcomma
\left(\gamma ^{\mu }.\gamma ^{\nu }\right){}_{ij}
\end{dmath*}

A DiracChain with only two arguments denotes a spinor component

\begin{Shaded}
\begin{Highlighting}[]
\NormalTok{DiracChain}\OperatorTok{[}\NormalTok{Spinor}\OperatorTok{[}\NormalTok{Momentum}\OperatorTok{[}\FunctionTok{p}\OperatorTok{],} \FunctionTok{m}\OperatorTok{],}\NormalTok{ DiracIndex}\OperatorTok{[}\FunctionTok{i}\OperatorTok{]]}
\end{Highlighting}
\end{Shaded}

\begin{dmath*}\breakingcomma
\left(\varphi (\overline{p},m)\right)_i
\end{dmath*}

\begin{Shaded}
\begin{Highlighting}[]
\NormalTok{DiracChain}\OperatorTok{[}\NormalTok{Spinor}\OperatorTok{[}\NormalTok{Momentum}\OperatorTok{[}\SpecialCharTok{{-}}\FunctionTok{p}\OperatorTok{],} \FunctionTok{m}\OperatorTok{],}\NormalTok{ DiracIndex}\OperatorTok{[}\FunctionTok{i}\OperatorTok{]]}
\end{Highlighting}
\end{Shaded}

\begin{dmath*}\breakingcomma
\left(\varphi (-\overline{p},m)\right)_i
\end{dmath*}

\begin{Shaded}
\begin{Highlighting}[]
\NormalTok{DiracChain}\OperatorTok{[}\NormalTok{DiracIndex}\OperatorTok{[}\FunctionTok{i}\OperatorTok{],}\NormalTok{ Spinor}\OperatorTok{[}\NormalTok{Momentum}\OperatorTok{[}\FunctionTok{p}\OperatorTok{],} \FunctionTok{m}\OperatorTok{]]}
\end{Highlighting}
\end{Shaded}

\begin{dmath*}\breakingcomma
\left(\varphi (\overline{p},m)\right)_i
\end{dmath*}

\begin{Shaded}
\begin{Highlighting}[]
\NormalTok{DiracChain}\OperatorTok{[}\NormalTok{DiracIndex}\OperatorTok{[}\FunctionTok{i}\OperatorTok{],}\NormalTok{ Spinor}\OperatorTok{[}\NormalTok{Momentum}\OperatorTok{[}\SpecialCharTok{{-}}\FunctionTok{p}\OperatorTok{],} \FunctionTok{m}\OperatorTok{]]}
\end{Highlighting}
\end{Shaded}

\begin{dmath*}\breakingcomma
\left(\varphi (-\overline{p},m)\right)_i
\end{dmath*}

The chain may also be partially open or closed

\begin{Shaded}
\begin{Highlighting}[]
\NormalTok{DiracChain}\OperatorTok{[}\NormalTok{DiracGamma}\OperatorTok{[}\NormalTok{LorentzIndex}\OperatorTok{[}\SpecialCharTok{\textbackslash{}}\OperatorTok{[}\NormalTok{Mu}\OperatorTok{]]]}\NormalTok{ . (}\FunctionTok{m} \SpecialCharTok{+}\NormalTok{ DiracGamma}\OperatorTok{[}\NormalTok{Momentum}\OperatorTok{[}\FunctionTok{p}\OperatorTok{]]}\NormalTok{) . DiracGamma}\OperatorTok{[}\NormalTok{LorentzIndex}\OperatorTok{[}\SpecialCharTok{\textbackslash{}}\OperatorTok{[}\NormalTok{Nu}\OperatorTok{]]],}\NormalTok{ Spinor}\OperatorTok{[}\NormalTok{Momentum}\OperatorTok{[}\FunctionTok{p}\OperatorTok{],} \FunctionTok{m}\OperatorTok{,} \DecValTok{1}\OperatorTok{],}\NormalTok{ DiracIndex}\OperatorTok{[}\FunctionTok{j}\OperatorTok{]]}
\end{Highlighting}
\end{Shaded}

\begin{dmath*}\breakingcomma
\left(\varphi (\overline{p},m).\bar{\gamma }^{\mu }.\left(\bar{\gamma }\cdot \overline{p}+m\right).\bar{\gamma }^{\nu }\right){}_j
\end{dmath*}

\begin{Shaded}
\begin{Highlighting}[]
\NormalTok{DiracChain}\OperatorTok{[}\NormalTok{DiracGamma}\OperatorTok{[}\NormalTok{LorentzIndex}\OperatorTok{[}\SpecialCharTok{\textbackslash{}}\OperatorTok{[}\NormalTok{Mu}\OperatorTok{]]]}\NormalTok{ . (}\FunctionTok{m} \SpecialCharTok{+}\NormalTok{ DiracGamma}\OperatorTok{[}\NormalTok{Momentum}\OperatorTok{[}\FunctionTok{p}\OperatorTok{]]}\NormalTok{) . DiracGamma}\OperatorTok{[}\NormalTok{LorentzIndex}\OperatorTok{[}\SpecialCharTok{\textbackslash{}}\OperatorTok{[}\NormalTok{Nu}\OperatorTok{]]],}\NormalTok{ DiracIndex}\OperatorTok{[}\FunctionTok{i}\OperatorTok{],}\NormalTok{ Spinor}\OperatorTok{[}\NormalTok{Momentum}\OperatorTok{[}\FunctionTok{p}\OperatorTok{],} \FunctionTok{m}\OperatorTok{,} \DecValTok{1}\OperatorTok{]]}
\end{Highlighting}
\end{Shaded}

\begin{dmath*}\breakingcomma
\left(\bar{\gamma }^{\mu }.\left(\bar{\gamma }\cdot \overline{p}+m\right).\bar{\gamma }^{\nu }.\varphi (\overline{p},m)\right){}_i
\end{dmath*}

\begin{Shaded}
\begin{Highlighting}[]
\NormalTok{DiracChain}\OperatorTok{[}\NormalTok{DiracGamma}\OperatorTok{[}\NormalTok{LorentzIndex}\OperatorTok{[}\SpecialCharTok{\textbackslash{}}\OperatorTok{[}\NormalTok{Mu}\OperatorTok{]]]}\NormalTok{ . (}\FunctionTok{m} \SpecialCharTok{+}\NormalTok{ DiracGamma}\OperatorTok{[}\NormalTok{Momentum}\OperatorTok{[}\FunctionTok{p}\OperatorTok{]]}\NormalTok{) . DiracGamma}\OperatorTok{[}\NormalTok{LorentzIndex}\OperatorTok{[}\SpecialCharTok{\textbackslash{}}\OperatorTok{[}\NormalTok{Nu}\OperatorTok{]]],}\NormalTok{ Spinor}\OperatorTok{[}\NormalTok{Momentum}\OperatorTok{[}\NormalTok{p1}\OperatorTok{],}\NormalTok{ m1}\OperatorTok{,} \DecValTok{1}\OperatorTok{],}\NormalTok{ Spinor}\OperatorTok{[}\NormalTok{Momentum}\OperatorTok{[}\NormalTok{p2}\OperatorTok{],}\NormalTok{ m2}\OperatorTok{,} \DecValTok{1}\OperatorTok{]]}
\end{Highlighting}
\end{Shaded}

\begin{dmath*}\breakingcomma
\left(\varphi (\overline{\text{p1}},\text{m1}).\bar{\gamma }^{\mu }.\left(\bar{\gamma }\cdot \overline{p}+m\right).\bar{\gamma }^{\nu }.\varphi (\overline{\text{p2}},\text{m2})\right)
\end{dmath*}

\begin{Shaded}
\begin{Highlighting}[]
\NormalTok{DiracChain}\OperatorTok{[}\DecValTok{1}\OperatorTok{,}\NormalTok{ Spinor}\OperatorTok{[}\NormalTok{Momentum}\OperatorTok{[}\NormalTok{p1}\OperatorTok{],}\NormalTok{ m1}\OperatorTok{,} \DecValTok{1}\OperatorTok{],}\NormalTok{ Spinor}\OperatorTok{[}\NormalTok{Momentum}\OperatorTok{[}\NormalTok{p2}\OperatorTok{],}\NormalTok{ m2}\OperatorTok{,} \DecValTok{1}\OperatorTok{]]}
\end{Highlighting}
\end{Shaded}

\begin{dmath*}\breakingcomma
\left(\varphi (\overline{\text{p1}},\text{m1}).\varphi (\overline{\text{p2}},\text{m2})\right)
\end{dmath*}
\end{document}
