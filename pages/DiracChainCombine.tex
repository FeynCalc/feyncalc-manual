% !TeX program = pdflatex
% !TeX root = DiracChainCombine.tex

\documentclass[../FeynCalcManual.tex]{subfiles}
\begin{document}
\hypertarget{diracchaincombine}{%
\section{DiracChainCombine}\label{diracchaincombine}}

\texttt{DiracChainCombine[\allowbreak{}exp]} is (nearly) the inverse
operation to \texttt{DiracChainExpand}.

\subsection{See also}

\hyperlink{toc}{Overview}, \hyperlink{diracchain}{DiracChain},
\hyperlink{dchn}{DCHN}, \hyperlink{diracindex}{DiracIndex},
\hyperlink{diracindexdelta}{DiracIndexDelta},
\hyperlink{didelta}{DIDelta},
\hyperlink{diracchainjoin}{DiracChainJoin},
\hyperlink{diracchainexpand}{DiracChainExpand},
\hyperlink{diracchainfactor}{DiracChainFactor}.

\subsection{Examples}

\begin{Shaded}
\begin{Highlighting}[]
\NormalTok{(DCHN}\OperatorTok{[}\NormalTok{GSD}\OperatorTok{[}\FunctionTok{q}\OperatorTok{],}\NormalTok{ Dir3}\OperatorTok{,}\NormalTok{ Dir4}\OperatorTok{]}\NormalTok{ FAD}\OperatorTok{[\{}\FunctionTok{k}\OperatorTok{,}\NormalTok{ me}\OperatorTok{\}]}\NormalTok{)}\SpecialCharTok{/}\NormalTok{(}\DecValTok{2}\NormalTok{ SPD}\OperatorTok{[}\FunctionTok{q}\OperatorTok{,} \FunctionTok{q}\OperatorTok{]}\NormalTok{) }\SpecialCharTok{+} \DecValTok{1}\SpecialCharTok{/}\NormalTok{(}\DecValTok{2}\NormalTok{ SPD}\OperatorTok{[}\FunctionTok{q}\OperatorTok{,} \FunctionTok{q}\OperatorTok{]}\NormalTok{) FAD}\OperatorTok{[}\FunctionTok{k}\OperatorTok{,} 
     \OperatorTok{\{}\FunctionTok{k} \SpecialCharTok{{-}} \FunctionTok{q}\OperatorTok{,}\NormalTok{ me}\OperatorTok{\}]}\NormalTok{ (}\SpecialCharTok{{-}}\DecValTok{2}\NormalTok{ DCHN}\OperatorTok{[}\NormalTok{GSD}\OperatorTok{[}\FunctionTok{q}\OperatorTok{],}\NormalTok{ Dir3}\OperatorTok{,}\NormalTok{ Dir4}\OperatorTok{]}\NormalTok{ SPD}\OperatorTok{[}\FunctionTok{q}\OperatorTok{,} \FunctionTok{q}\OperatorTok{]} \SpecialCharTok{+} \DecValTok{2}\NormalTok{ DCHN}\OperatorTok{[}\DecValTok{1}\OperatorTok{,}\NormalTok{ Dir3}\OperatorTok{,}\NormalTok{ Dir4}\OperatorTok{]}\NormalTok{ me SPD}\OperatorTok{[}\FunctionTok{q}\OperatorTok{,} \FunctionTok{q}\OperatorTok{]} \SpecialCharTok{+} 
\NormalTok{      DCHN}\OperatorTok{[}\NormalTok{GSD}\OperatorTok{[}\FunctionTok{q}\OperatorTok{],}\NormalTok{ Dir3}\OperatorTok{,}\NormalTok{ Dir4}\OperatorTok{]}\NormalTok{ (}\SpecialCharTok{{-}}\NormalTok{me}\SpecialCharTok{\^{}}\DecValTok{2} \SpecialCharTok{+}\NormalTok{ SPD}\OperatorTok{[}\FunctionTok{q}\OperatorTok{,} \FunctionTok{q}\OperatorTok{]}\NormalTok{)) }
 
\NormalTok{DiracChainCombine}\OperatorTok{[}\SpecialCharTok{\%}\OperatorTok{]}
\end{Highlighting}
\end{Shaded}

\begin{dmath*}\breakingcomma
\frac{\left(q^2-\text{me}^2\right) (\gamma \cdot q)_{\text{Dir3}\;\text{Dir4}}+2 \;\text{me} q^2 (1)_{\text{Dir3}\;\text{Dir4}}-2 q^2 (\gamma \cdot q)_{\text{Dir3}\;\text{Dir4}}}{2 q^2 k^2.\left((k-q)^2-\text{me}^2\right)}+\frac{(\gamma \cdot q)_{\text{Dir3}\;\text{Dir4}}}{2 q^2 \left(k^2-\text{me}^2\right)}
\end{dmath*}

\begin{dmath*}\breakingcomma
\frac{\left(\left(q^2-\text{me}^2\right) \gamma \cdot q+2 \;\text{me} q^2-2 q^2 \gamma \cdot q\right){}_{\text{Dir3}\;\text{Dir4}}}{2 q^2 k^2.\left((k-q)^2-\text{me}^2\right)}+\frac{(\gamma \cdot q)_{\text{Dir3}\;\text{Dir4}}}{2 q^2 \left(k^2-\text{me}^2\right)}
\end{dmath*}
\end{document}
