% !TeX program = pdflatex
% !TeX root = DiracChainExpand.tex

\documentclass[../FeynCalcManual.tex]{subfiles}
\begin{document}
\hypertarget{diracchainexpand}{%
\section{DiracChainExpand}\label{diracchainexpand}}

\texttt{DiracChainExpand[\allowbreak{}exp]} expands all Dirac chains
with explicit indices using linearity,
e.g.~\texttt{DCHN[\allowbreak{}GA[\allowbreak{}p1]+GA[\allowbreak{}p2]+m,\ \allowbreak{}i,\ \allowbreak{}j]}
becomes
\texttt{DCHN[\allowbreak{}GA[\allowbreak{}p1],\ \allowbreak{}i,\ \allowbreak{}j]+DCHN[\allowbreak{}GA[\allowbreak{}p2],\ \allowbreak{}i,\ \allowbreak{}j]+m*DCHN[\allowbreak{}1,\ \allowbreak{}i,\ \allowbreak{}j]}.

\subsection{See also}

\hyperlink{toc}{Overview}, \hyperlink{diracchain}{DiracChain},
\hyperlink{dchn}{DCHN}, \hyperlink{diracindex}{DiracIndex},
\hyperlink{diracindexdelta}{DiracIndexDelta},
\hyperlink{didelta}{DIDelta},
\hyperlink{diracchainjoin}{DiracChainJoin},
\hyperlink{diracchaincombine}{DiracChainCombine},
\hyperlink{diracchainfactor}{DiracChainFactor}.

\subsection{Examples}

\begin{Shaded}
\begin{Highlighting}[]
\NormalTok{DCHN}\OperatorTok{[}\NormalTok{(GS}\OperatorTok{[}\FunctionTok{p}\OperatorTok{]} \SpecialCharTok{+} \FunctionTok{m}\NormalTok{) . GA}\OperatorTok{[}\NormalTok{mu}\OperatorTok{],} \FunctionTok{i}\OperatorTok{,} \FunctionTok{j}\OperatorTok{]} 
 
\NormalTok{DiracChainExpand}\OperatorTok{[}\SpecialCharTok{\%}\OperatorTok{]}
\end{Highlighting}
\end{Shaded}

\begin{dmath*}\breakingcomma
\left(\left(\bar{\gamma }\cdot \overline{p}+m\right).\bar{\gamma }^{\text{mu}}\right){}_{ij}
\end{dmath*}

\begin{dmath*}\breakingcomma
m \left(\bar{\gamma }^{\text{mu}}\right){}_{ij}+\left(\left(\bar{\gamma }\cdot \overline{p}\right).\bar{\gamma }^{\text{mu}}\right){}_{ij}
\end{dmath*}
\end{document}
