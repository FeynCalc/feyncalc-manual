% !TeX program = pdflatex
% !TeX root = DiracChainJoin.tex

\documentclass[../FeynCalcManual.tex]{subfiles}
\begin{document}
\hypertarget{diracchainjoin}{%
\section{DiracChainJoin}\label{diracchainjoin}}

\texttt{DiracChainJoin[\allowbreak{}exp]} joins chains of Dirac matrices
with explicit Dirac indices wrapped with a head \texttt{DiracChain}.
Notice that \texttt{DiracChainJoin} is not suitable for creating closed
Dirac chains out of the FeynArts output with explicit Dirac indices,
e.g.~when the model contains 4-fermion operators. Use
\texttt{FCFADiracChainJoin} for that.

\subsection{See also}

\hyperlink{toc}{Overview}, \hyperlink{diracchain}{DiracChain},
\hyperlink{dchn}{DCHN}, \hyperlink{diracindex}{DiracIndex},
\hyperlink{diracindexdelta}{DiracIndexDelta},
\hyperlink{didelta}{DIDelta},
\hyperlink{diracchaincombine}{DiracChainCombine},
\hyperlink{diracchainexpand}{DiracChainExpand},
\hyperlink{diracchainfactor}{DiracChainFactor},
\hyperlink{fcfadiracchainjoin}{FCFADiracChainJoin}.

\subsection{Examples}

\begin{Shaded}
\begin{Highlighting}[]
\NormalTok{DCHN}\OperatorTok{[}\NormalTok{SpinorUBar}\OperatorTok{[}\NormalTok{p1}\OperatorTok{,}\NormalTok{ m1}\OperatorTok{],} \FunctionTok{i}\OperatorTok{]}\NormalTok{ DCHN}\OperatorTok{[}\NormalTok{GAD}\OperatorTok{[}\SpecialCharTok{\textbackslash{}}\OperatorTok{[}\NormalTok{Mu}\OperatorTok{]]}\NormalTok{ . GAD}\OperatorTok{[}\SpecialCharTok{\textbackslash{}}\OperatorTok{[}\NormalTok{Nu}\OperatorTok{]],} \FunctionTok{i}\OperatorTok{,} \FunctionTok{j}\OperatorTok{]}\NormalTok{ DCHN}\OperatorTok{[}\FunctionTok{j}\OperatorTok{,}\NormalTok{ SpinorV}\OperatorTok{[}\NormalTok{p2}\OperatorTok{,}\NormalTok{ m2}\OperatorTok{]]} 
 
\NormalTok{DiracChainJoin}\OperatorTok{[}\SpecialCharTok{\%}\OperatorTok{]}
\end{Highlighting}
\end{Shaded}

\begin{dmath*}\breakingcomma
(v(\text{p2},\text{m2}))_j \left(\gamma ^{\mu }.\gamma ^{\nu }\right){}_{ij} \left(\bar{u}(\text{p1},\text{m1})\right)_i
\end{dmath*}

\begin{dmath*}\breakingcomma
\left(\varphi (\overline{\text{p1}},\text{m1})\right).\gamma ^{\mu }.\gamma ^{\nu }.\left(\varphi (-\overline{\text{p2}},\text{m2})\right)
\end{dmath*}
\end{document}
