% !TeX program = pdflatex
% !TeX root = DiracEquation.tex

\documentclass[../FeynCalcManual.tex]{subfiles}
\begin{document}
\hypertarget{diracequation}{%
\section{DiracEquation}\label{diracequation}}

\texttt{DiracEquation[\allowbreak{}exp]} applies the Dirac equation
without expanding exp. If expansions are necessary, use
\texttt{DiracSimplify}.

\subsection{See also}

\hyperlink{toc}{Overview}

\subsection{Examples}

\begin{Shaded}
\begin{Highlighting}[]
\NormalTok{GS}\OperatorTok{[}\FunctionTok{p}\OperatorTok{]}\NormalTok{ . SpinorU}\OperatorTok{[}\FunctionTok{p}\OperatorTok{,} \FunctionTok{m}\OperatorTok{]} 
 
\NormalTok{DiracSimplify}\OperatorTok{[}\SpecialCharTok{\%}\OperatorTok{]}
\end{Highlighting}
\end{Shaded}

\begin{dmath*}\breakingcomma
\left(\bar{\gamma }\cdot \overline{p}\right).u(p,m)
\end{dmath*}

\begin{dmath*}\breakingcomma
m \left(\varphi (\overline{p},m)\right)
\end{dmath*}

\begin{Shaded}
\begin{Highlighting}[]
\NormalTok{GS}\OperatorTok{[}\FunctionTok{p}\OperatorTok{]}\NormalTok{ . SpinorU}\OperatorTok{[}\FunctionTok{p}\OperatorTok{,} \FunctionTok{m}\OperatorTok{]} 
 
\NormalTok{DiracEquation}\OperatorTok{[}\SpecialCharTok{\%}\OperatorTok{]}
\end{Highlighting}
\end{Shaded}

\begin{dmath*}\breakingcomma
\left(\bar{\gamma }\cdot \overline{p}\right).u(p,m)
\end{dmath*}

\begin{dmath*}\breakingcomma
m \left(\varphi (\overline{p},m)\right)
\end{dmath*}

\begin{Shaded}
\begin{Highlighting}[]
\NormalTok{GS}\OperatorTok{[}\FunctionTok{p}\OperatorTok{]}\NormalTok{ . SpinorV}\OperatorTok{[}\FunctionTok{p}\OperatorTok{,} \FunctionTok{m}\OperatorTok{]} 
 
\NormalTok{DiracEquation}\OperatorTok{[}\SpecialCharTok{\%}\OperatorTok{]}
\end{Highlighting}
\end{Shaded}

\begin{dmath*}\breakingcomma
\left(\bar{\gamma }\cdot \overline{p}\right).v(p,m)
\end{dmath*}

\begin{dmath*}\breakingcomma
-m \left(\varphi (-\overline{p},m)\right)
\end{dmath*}

\begin{Shaded}
\begin{Highlighting}[]
\NormalTok{SpinorUBar}\OperatorTok{[}\FunctionTok{p}\OperatorTok{,} \DecValTok{0}\OperatorTok{]}\NormalTok{ . GS}\OperatorTok{[}\FunctionTok{p}\OperatorTok{]} 
 
\NormalTok{DiracEquation}\OperatorTok{[}\SpecialCharTok{\%}\OperatorTok{]}
\end{Highlighting}
\end{Shaded}

\begin{dmath*}\breakingcomma
\bar{u}(p).\left(\bar{\gamma }\cdot \overline{p}\right)
\end{dmath*}

\begin{dmath*}\breakingcomma
0
\end{dmath*}

\texttt{DiracEquation} also works in \(D\)-dimensions

\begin{Shaded}
\begin{Highlighting}[]
\NormalTok{SpinorVBarD}\OperatorTok{[}\FunctionTok{p}\OperatorTok{,} \FunctionTok{m}\OperatorTok{]}\NormalTok{ . GSD}\OperatorTok{[}\FunctionTok{p}\OperatorTok{]} 
 
\NormalTok{DiracEquation}\OperatorTok{[}\SpecialCharTok{\%}\OperatorTok{]}
\end{Highlighting}
\end{Shaded}

\begin{dmath*}\breakingcomma
\bar{v}(p,m).(\gamma \cdot p)
\end{dmath*}

\begin{dmath*}\breakingcomma
-m (\varphi (-p,m))
\end{dmath*}
\end{document}
