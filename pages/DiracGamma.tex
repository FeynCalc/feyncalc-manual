% !TeX program = pdflatex
% !TeX root = DiracGamma.tex

\documentclass[../FeynCalcManual.tex]{subfiles}
\begin{document}
\hypertarget{diracgamma}{
\section{DiracGamma}\label{diracgamma}\index{DiracGamma}}

\texttt{DiracGamma[\allowbreak{}x,\ \allowbreak{}dim]} is the head of
all Dirac matrices and slashes (in the internal representation). Use
\texttt{GA}, \texttt{GAD}, \texttt{GS} or \texttt{GSD} for manual
(short) input.

\texttt{DiracGamma[\allowbreak{}x,\ \allowbreak{}4]} simplifies to
\texttt{DiracGamma[\allowbreak{}x]}.

\texttt{DiracGamma[\allowbreak{}5]} is \(\gamma ^5\).

\texttt{DiracGamma[\allowbreak{}6]} is \((1+\gamma ^5)/2\).

\texttt{DiracGamma[\allowbreak{}7]} is \((1-\gamma ^5)/2\).

\subsection{See also}

\hyperlink{toc}{Overview},
\hyperlink{diracgammaexpand}{DiracGammaExpand}, \hyperlink{ga}{GA},
\hyperlink{diracsimplify}{DiracSimplify}, \hyperlink{gs}{GS},
\hyperlink{diractrick}{DiracTrick}.

\subsection{Examples}

\begin{Shaded}
\begin{Highlighting}[]
\NormalTok{DiracGamma}\OperatorTok{[}\DecValTok{5}\OperatorTok{]}
\end{Highlighting}
\end{Shaded}

\begin{dmath*}\breakingcomma
\bar{\gamma }^5
\end{dmath*}

\begin{Shaded}
\begin{Highlighting}[]
\NormalTok{DiracGamma}\OperatorTok{[}\NormalTok{LorentzIndex}\OperatorTok{[}\SpecialCharTok{\textbackslash{}}\OperatorTok{[}\NormalTok{Alpha}\OperatorTok{]]]}
\end{Highlighting}
\end{Shaded}

\begin{dmath*}\breakingcomma
\bar{\gamma }^{\alpha }
\end{dmath*}

A Dirac-slash, i.e., \(\gamma ^{\mu }q_{\mu}\), is displayed as
\(\gamma \cdot q\).

\begin{Shaded}
\begin{Highlighting}[]
\NormalTok{DiracGamma}\OperatorTok{[}\NormalTok{Momentum}\OperatorTok{[}\FunctionTok{q}\OperatorTok{]]} 
\end{Highlighting}
\end{Shaded}

\begin{dmath*}\breakingcomma
\bar{\gamma }\cdot \overline{q}
\end{dmath*}

\begin{Shaded}
\begin{Highlighting}[]
\NormalTok{DiracGamma}\OperatorTok{[}\NormalTok{Momentum}\OperatorTok{[}\FunctionTok{q}\OperatorTok{]]}\NormalTok{ . DiracGamma}\OperatorTok{[}\NormalTok{Momentum}\OperatorTok{[}\FunctionTok{p} \SpecialCharTok{{-}} \FunctionTok{q}\OperatorTok{]]}
\end{Highlighting}
\end{Shaded}

\begin{dmath*}\breakingcomma
\left(\bar{\gamma }\cdot \overline{q}\right).\left(\bar{\gamma }\cdot \left(\overline{p}-\overline{q}\right)\right)
\end{dmath*}

\begin{Shaded}
\begin{Highlighting}[]
\NormalTok{DiracGamma}\OperatorTok{[}\NormalTok{Momentum}\OperatorTok{[}\FunctionTok{q}\OperatorTok{,} \FunctionTok{D}\OperatorTok{],} \FunctionTok{D}\OperatorTok{]} 
\end{Highlighting}
\end{Shaded}

\begin{dmath*}\breakingcomma
\gamma \cdot q
\end{dmath*}

\begin{Shaded}
\begin{Highlighting}[]
\NormalTok{GS}\OperatorTok{[}\FunctionTok{p} \SpecialCharTok{{-}} \FunctionTok{q}\OperatorTok{]}\NormalTok{ . GS}\OperatorTok{[}\FunctionTok{p}\OperatorTok{]} 
 
\NormalTok{DiracGammaExpand}\OperatorTok{[}\SpecialCharTok{\%}\OperatorTok{]}
\end{Highlighting}
\end{Shaded}

\begin{dmath*}\breakingcomma
\left(\bar{\gamma }\cdot \left(\overline{p}-\overline{q}\right)\right).\left(\bar{\gamma }\cdot \overline{p}\right)
\end{dmath*}

\begin{dmath*}\breakingcomma
\left(\bar{\gamma }\cdot \overline{p}-\bar{\gamma }\cdot \overline{q}\right).\left(\bar{\gamma }\cdot \overline{p}\right)
\end{dmath*}

\begin{Shaded}
\begin{Highlighting}[]
\NormalTok{ex }\ExtensionTok{=}\NormalTok{ GAD}\OperatorTok{[}\SpecialCharTok{\textbackslash{}}\OperatorTok{[}\NormalTok{Mu}\OperatorTok{]]}\NormalTok{ . GSD}\OperatorTok{[}\FunctionTok{p} \SpecialCharTok{{-}} \FunctionTok{q}\OperatorTok{]}\NormalTok{ . GSD}\OperatorTok{[}\FunctionTok{q}\OperatorTok{]}\NormalTok{ . GAD}\OperatorTok{[}\SpecialCharTok{\textbackslash{}}\OperatorTok{[}\NormalTok{Mu}\OperatorTok{]]}
\end{Highlighting}
\end{Shaded}

\begin{dmath*}\breakingcomma
\gamma ^{\mu }.(\gamma \cdot (p-q)).(\gamma \cdot q).\gamma ^{\mu }
\end{dmath*}

\begin{Shaded}
\begin{Highlighting}[]
\NormalTok{DiracTrick}\OperatorTok{[}\NormalTok{ex}\OperatorTok{]}
\end{Highlighting}
\end{Shaded}

\begin{dmath*}\breakingcomma
4 ((p-q)\cdot q)+(D-4) (\gamma \cdot (p-q)).(\gamma \cdot q)
\end{dmath*}

\begin{Shaded}
\begin{Highlighting}[]
\NormalTok{DiracSimplify}\OperatorTok{[}\NormalTok{ex}\OperatorTok{]}
\end{Highlighting}
\end{Shaded}

\begin{dmath*}\breakingcomma
D (\gamma \cdot p).(\gamma \cdot q)-D q^2-4 (\gamma \cdot p).(\gamma \cdot q)+4 (p\cdot q)
\end{dmath*}

\texttt{DiracGamma} may also carry Cartesian indices or appear
contracted with Cartesian momenta.

\begin{Shaded}
\begin{Highlighting}[]
\NormalTok{DiracGamma}\OperatorTok{[}\NormalTok{CartesianIndex}\OperatorTok{[}\FunctionTok{i}\OperatorTok{]]}
\end{Highlighting}
\end{Shaded}

\begin{dmath*}\breakingcomma
\overline{\gamma }^i
\end{dmath*}

\begin{Shaded}
\begin{Highlighting}[]
\NormalTok{DiracGamma}\OperatorTok{[}\NormalTok{CartesianIndex}\OperatorTok{[}\FunctionTok{i}\OperatorTok{,} \FunctionTok{D} \SpecialCharTok{{-}} \DecValTok{1}\OperatorTok{],} \FunctionTok{D}\OperatorTok{]}
\end{Highlighting}
\end{Shaded}

\begin{dmath*}\breakingcomma
\gamma ^i
\end{dmath*}

\begin{Shaded}
\begin{Highlighting}[]
\NormalTok{DiracGamma}\OperatorTok{[}\NormalTok{CartesianMomentum}\OperatorTok{[}\FunctionTok{p}\OperatorTok{]]}
\end{Highlighting}
\end{Shaded}

\begin{dmath*}\breakingcomma
\overline{\gamma }\cdot \overline{p}
\end{dmath*}

\begin{Shaded}
\begin{Highlighting}[]
\NormalTok{DiracGamma}\OperatorTok{[}\NormalTok{CartesianMomentum}\OperatorTok{[}\FunctionTok{p}\OperatorTok{,} \FunctionTok{D} \SpecialCharTok{{-}} \DecValTok{1}\OperatorTok{],} \FunctionTok{D}\OperatorTok{]}
\end{Highlighting}
\end{Shaded}

\begin{dmath*}\breakingcomma
\gamma \cdot p
\end{dmath*}

Temporal indices are represented using
\texttt{ExplicitLorentzIndex[\allowbreak{}0]}

\begin{Shaded}
\begin{Highlighting}[]
\NormalTok{DiracGamma}\OperatorTok{[}\NormalTok{ExplicitLorentzIndex}\OperatorTok{[}\DecValTok{0}\OperatorTok{]]}
\end{Highlighting}
\end{Shaded}

\begin{dmath*}\breakingcomma
\bar{\gamma }^0
\end{dmath*}
\end{document}
