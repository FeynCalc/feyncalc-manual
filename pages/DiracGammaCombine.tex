% !TeX program = pdflatex
% !TeX root = DiracGammaCombine.tex

\documentclass[../FeynCalcManual.tex]{subfiles}
\begin{document}
\hypertarget{diracgammacombine}{
\section{DiracGammaCombine}\label{diracgammacombine}\index{DiracGammaCombine}}

\texttt{DiracGammaCombine[\allowbreak{}exp]} is (nearly) the inverse
operation to \texttt{DiracGammaExpand}.

\subsection{See also}

\hyperlink{toc}{Overview}, \hyperlink{diracgamma}{DiracGamma},
\hyperlink{diracgammaexpand}{DiracGammaExpand},
\hyperlink{diracsimplify}{DiracSimplify},
\hyperlink{diractrick}{DiracTrick}.

\subsection{Examples}

\begin{Shaded}
\begin{Highlighting}[]
\NormalTok{GS}\OperatorTok{[}\FunctionTok{p}\OperatorTok{]} \SpecialCharTok{+}\NormalTok{ GS}\OperatorTok{[}\FunctionTok{q}\OperatorTok{]} 
 
\NormalTok{ex }\ExtensionTok{=}\NormalTok{ DiracGammaCombine}\OperatorTok{[}\SpecialCharTok{\%}\OperatorTok{]}
\end{Highlighting}
\end{Shaded}

\begin{dmath*}\breakingcomma
\bar{\gamma }\cdot \overline{p}+\bar{\gamma }\cdot \overline{q}
\end{dmath*}

\begin{dmath*}\breakingcomma
\bar{\gamma }\cdot \left(\overline{p}+\overline{q}\right)
\end{dmath*}

\begin{Shaded}
\begin{Highlighting}[]
\NormalTok{ex }\SpecialCharTok{//} \FunctionTok{StandardForm}

\CommentTok{(*DiracGamma[Momentum[p + q]]*)}
\end{Highlighting}
\end{Shaded}

\begin{Shaded}
\begin{Highlighting}[]
\DecValTok{2}\NormalTok{ GSD}\OperatorTok{[}\FunctionTok{p}\OperatorTok{]} \SpecialCharTok{{-}} \DecValTok{3}\NormalTok{ GSD}\OperatorTok{[}\FunctionTok{q}\OperatorTok{]} 
 
\NormalTok{ex }\ExtensionTok{=}\NormalTok{ DiracGammaCombine}\OperatorTok{[}\SpecialCharTok{\%}\OperatorTok{]}
\end{Highlighting}
\end{Shaded}

\begin{dmath*}\breakingcomma
2 \gamma \cdot p-3 \gamma \cdot q
\end{dmath*}

\begin{dmath*}\breakingcomma
\gamma \cdot (2 p-3 q)
\end{dmath*}

\begin{Shaded}
\begin{Highlighting}[]
\NormalTok{ex }\SpecialCharTok{//} \FunctionTok{StandardForm}

\CommentTok{(*DiracGamma[Momentum[2 p {-} 3 q, D], D]*)}
\end{Highlighting}
\end{Shaded}

\begin{Shaded}
\begin{Highlighting}[]
\NormalTok{DiracGammaCombine}\OperatorTok{[}\DecValTok{2}\NormalTok{ GSD}\OperatorTok{[}\FunctionTok{p}\OperatorTok{]} \SpecialCharTok{{-}} \DecValTok{3}\NormalTok{ GSD}\OperatorTok{[}\FunctionTok{q}\OperatorTok{]]} 
 
\NormalTok{DiracGammaExpand}\OperatorTok{[}\SpecialCharTok{\%}\OperatorTok{]}
\end{Highlighting}
\end{Shaded}

\begin{dmath*}\breakingcomma
\gamma \cdot (2 p-3 q)
\end{dmath*}

\begin{dmath*}\breakingcomma
2 \gamma \cdot p-3 \gamma \cdot q
\end{dmath*}
\end{document}
