% !TeX program = pdflatex
% !TeX root = DiracGammaExpand.tex

\documentclass[../FeynCalcManual.tex]{subfiles}
\begin{document}
\hypertarget{diracgammaexpand}{
\section{DiracGammaExpand}\label{diracgammaexpand}\index{DiracGammaExpand}}

\texttt{DiracGammaExpand[\allowbreak{}exp]} expands Dirac matrices
contracted to linear combinations of \(4\)-vectors. All
\texttt{DiracGamma[\allowbreak{}Momentum[\allowbreak{}a+b+ ...]]} will
be expanded to
\texttt{DiracGamma[\allowbreak{}Momentum[\allowbreak{}a]] + DiracGamma[\allowbreak{}Momentum[\allowbreak{}b]] + DiracGamma[\allowbreak{}Momentum[\allowbreak{}...]]}
.

\subsection{See also}

\hyperlink{toc}{Overview}, \hyperlink{diracgamma}{DiracGamma},
\hyperlink{diracgammacombine}{DiracGammaCombine},
\hyperlink{diracsimplify}{DiracSimplify},
\hyperlink{diractrick}{DiracTrick}.

\subsection{Examples}

\begin{Shaded}
\begin{Highlighting}[]
\NormalTok{GS}\OperatorTok{[}\FunctionTok{q}\OperatorTok{]}\NormalTok{ . GS}\OperatorTok{[}\FunctionTok{p} \SpecialCharTok{{-}} \FunctionTok{q}\OperatorTok{]} 
 
\NormalTok{ex }\ExtensionTok{=}\NormalTok{ DiracGammaExpand}\OperatorTok{[}\SpecialCharTok{\%}\OperatorTok{]}
\end{Highlighting}
\end{Shaded}

\begin{dmath*}\breakingcomma
\left(\bar{\gamma }\cdot \overline{q}\right).\left(\bar{\gamma }\cdot \left(\overline{p}-\overline{q}\right)\right)
\end{dmath*}

\begin{dmath*}\breakingcomma
\left(\bar{\gamma }\cdot \overline{q}\right).\left(\bar{\gamma }\cdot \overline{p}-\bar{\gamma }\cdot \overline{q}\right)
\end{dmath*}

\begin{Shaded}
\begin{Highlighting}[]
\NormalTok{ex }\SpecialCharTok{//} \FunctionTok{StandardForm}

\CommentTok{(*DiracGamma[Momentum[q]] . (DiracGamma[Momentum[p]] {-} DiracGamma[Momentum[q]])*)}
\end{Highlighting}
\end{Shaded}

\texttt{DiracGammaExpand} rewrites \(\gamma^{\mu } (p-q)_{\mu }\) as
\(\gamma^{mu } p_{mu } - \gamma^{\mu } q_{\mu }\).

The inverse operation is \texttt{DiracGammaCombine}.

\begin{Shaded}
\begin{Highlighting}[]
\NormalTok{GS}\OperatorTok{[}\FunctionTok{q}\OperatorTok{]}\NormalTok{ . (GS}\OperatorTok{[}\FunctionTok{p}\OperatorTok{]} \SpecialCharTok{{-}}\NormalTok{ GS}\OperatorTok{[}\FunctionTok{q}\OperatorTok{]}\NormalTok{) }
 
\NormalTok{ex }\ExtensionTok{=}\NormalTok{ DiracGammaCombine}\OperatorTok{[}\SpecialCharTok{\%}\OperatorTok{]}
\end{Highlighting}
\end{Shaded}

\begin{dmath*}\breakingcomma
\left(\bar{\gamma }\cdot \overline{q}\right).\left(\bar{\gamma }\cdot \overline{p}-\bar{\gamma }\cdot \overline{q}\right)
\end{dmath*}

\begin{dmath*}\breakingcomma
\left(\bar{\gamma }\cdot \overline{q}\right).\left(\bar{\gamma }\cdot \left(\overline{p}-\overline{q}\right)\right)
\end{dmath*}

\begin{Shaded}
\begin{Highlighting}[]
\NormalTok{ex }\SpecialCharTok{//} \FunctionTok{StandardForm}

\CommentTok{(*DiracGamma[Momentum[q]] . DiracGamma[Momentum[p {-} q]]*)}
\end{Highlighting}
\end{Shaded}

It is possible to perform the expansions only on Dirac matrices
contracted with particular momenta.

\begin{Shaded}
\begin{Highlighting}[]
\NormalTok{c1 GAD}\OperatorTok{[}\SpecialCharTok{\textbackslash{}}\OperatorTok{[}\NormalTok{Mu}\OperatorTok{]]}\NormalTok{ . (GSD}\OperatorTok{[}\NormalTok{p1 }\SpecialCharTok{+}\NormalTok{ p2}\OperatorTok{]} \SpecialCharTok{+} \FunctionTok{m}\NormalTok{) . GAD}\OperatorTok{[}\SpecialCharTok{\textbackslash{}}\OperatorTok{[}\NormalTok{Nu}\OperatorTok{]]} \SpecialCharTok{+}\NormalTok{ c2 GAD}\OperatorTok{[}\SpecialCharTok{\textbackslash{}}\OperatorTok{[}\NormalTok{Mu}\OperatorTok{]]}\NormalTok{ . (GSD}\OperatorTok{[}\NormalTok{q1 }\SpecialCharTok{+}\NormalTok{ q2}\OperatorTok{]} \SpecialCharTok{+} \FunctionTok{m}\NormalTok{) . GAD}\OperatorTok{[}\SpecialCharTok{\textbackslash{}}\OperatorTok{[}\NormalTok{Nu}\OperatorTok{]]} 
 
\NormalTok{DiracGammaExpand}\OperatorTok{[}\SpecialCharTok{\%}\OperatorTok{,}\NormalTok{ Momentum }\OtherTok{{-}\textgreater{}} \OperatorTok{\{}\NormalTok{q1}\OperatorTok{\}]}
\end{Highlighting}
\end{Shaded}

\begin{dmath*}\breakingcomma
\text{c1} \gamma ^{\mu }.(m+\gamma \cdot (\text{p1}+\text{p2})).\gamma ^{\nu }+\text{c2} \gamma ^{\mu }.(m+\gamma \cdot (\text{q1}+\text{q2})).\gamma ^{\nu }
\end{dmath*}

\begin{dmath*}\breakingcomma
\text{c1} \gamma ^{\mu }.(m+\gamma \cdot (\text{p1}+\text{p2})).\gamma ^{\nu }+\text{c2} \gamma ^{\mu }.(m+\gamma \cdot \;\text{q1}+\gamma \cdot \;\text{q2}).\gamma ^{\nu }
\end{dmath*}

If the input expression contains \texttt{DiracSigma},
\texttt{DiracGammaExpand} will expand Feynman slashes inside
\texttt{DiracSigma} and call \texttt{DiracSigmaExpand}.

\begin{Shaded}
\begin{Highlighting}[]
\NormalTok{DiracSigma}\OperatorTok{[}\NormalTok{GSD}\OperatorTok{[}\FunctionTok{p} \SpecialCharTok{+} \FunctionTok{q}\OperatorTok{],}\NormalTok{ GSD}\OperatorTok{[}\FunctionTok{r}\OperatorTok{]]} 
 
\NormalTok{DiracGammaExpand}\OperatorTok{[}\SpecialCharTok{\%}\OperatorTok{]}
\end{Highlighting}
\end{Shaded}

\begin{dmath*}\breakingcomma
\sigma ^{p+qr}
\end{dmath*}

\begin{dmath*}\breakingcomma
\sigma ^{pr}+\sigma ^{qr}
\end{dmath*}

The call to \texttt{DiracSigmaExpand} can be inhibited by disabling the
corresponding option.

\begin{Shaded}
\begin{Highlighting}[]
\NormalTok{DiracGammaExpand}\OperatorTok{[}\NormalTok{DiracSigma}\OperatorTok{[}\NormalTok{GSD}\OperatorTok{[}\FunctionTok{p} \SpecialCharTok{+} \FunctionTok{q}\OperatorTok{],}\NormalTok{ GSD}\OperatorTok{[}\FunctionTok{r}\OperatorTok{]],}\NormalTok{ DiracSigmaExpand }\OtherTok{{-}\textgreater{}} \ConstantTok{False}\OperatorTok{]}
\end{Highlighting}
\end{Shaded}

\begin{dmath*}\breakingcomma
\text{DiracSigma}(\gamma \cdot p+\gamma \cdot q,\gamma \cdot r)
\end{dmath*}

Use \texttt{DiracSimplify} for noncommutative expansions with the
corresponding simplifications.

\begin{Shaded}
\begin{Highlighting}[]
\NormalTok{DiracSimplify}\OperatorTok{[}\NormalTok{GS}\OperatorTok{[}\FunctionTok{q}\OperatorTok{]}\NormalTok{ . (GS}\OperatorTok{[}\FunctionTok{p} \SpecialCharTok{{-}} \FunctionTok{q}\OperatorTok{]}\NormalTok{)}\OperatorTok{]}
\end{Highlighting}
\end{Shaded}

\begin{dmath*}\breakingcomma
\left(\bar{\gamma }\cdot \overline{q}\right).\left(\bar{\gamma }\cdot \overline{p}\right)-\overline{q}^2
\end{dmath*}

If simplifications are not required, you may also combine
\texttt{DiracGammaExpand} with \texttt{DotSimplify}.

\begin{Shaded}
\begin{Highlighting}[]
\NormalTok{DotSimplify}\OperatorTok{[}\NormalTok{DiracGammaExpand}\OperatorTok{[}\NormalTok{GS}\OperatorTok{[}\FunctionTok{q}\OperatorTok{]}\NormalTok{ . (GS}\OperatorTok{[}\FunctionTok{p} \SpecialCharTok{{-}} \FunctionTok{q}\OperatorTok{]}\NormalTok{)}\OperatorTok{]]}
\end{Highlighting}
\end{Shaded}

\begin{dmath*}\breakingcomma
\left(\bar{\gamma }\cdot \overline{q}\right).\left(\bar{\gamma }\cdot \overline{p}\right)-\left(\bar{\gamma }\cdot \overline{q}\right).\left(\bar{\gamma }\cdot \overline{q}\right)
\end{dmath*}
\end{document}
