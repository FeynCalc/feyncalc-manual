% !TeX program = pdflatex
% !TeX root = DiracIndex.tex

\documentclass[../FeynCalcManual.tex]{subfiles}
\begin{document}
\hypertarget{diracindex}{
\section{DiracIndex}\label{diracindex}\index{DiracIndex}}

\texttt{DiracIndex} is the head of Dirac indices. The internal
representation of a four-dimensional spinorial index \texttt{i} is
\texttt{DiracIndex[\allowbreak{}i]}.

If the first argument is an integer, \texttt{DiracIndex[\allowbreak{}i]}
turns into \texttt{ExplicitDiracIndex[\allowbreak{}i]}.

Dirac indices are the indices that denote the components of Dirac
matrices or spinors. They should not be confused with the Lorentz
indices attached to the Dirac matrices. For example in the case of
\(\gamma_{ij}^{\mu}\), \(\mu\) is a Lorentz index, while \(i\) and \(j\)
are Dirac (spinorial) indices.

\subsection{See also}

\hyperlink{toc}{Overview}, \hyperlink{diracchain}{DiracChain},
\hyperlink{dchn}{DCHN},
\hyperlink{explicitdiracindex}{ExplicitDiracIndex},
\hyperlink{diracindexdelta}{DiracIndexDelta},
\hyperlink{didelta}{DIDelta},
\hyperlink{diracchainjoin}{DiracChainJoin},
\hyperlink{diracchaincombine}{DiracChainCombine},
\hyperlink{diracchainexpand}{DiracChainExpand},
\hyperlink{diracchainfactor}{DiracChainFactor}.

\subsection{Examples}

\begin{Shaded}
\begin{Highlighting}[]
\NormalTok{DiracIndex}\OperatorTok{[}\FunctionTok{i}\OperatorTok{]}
\end{Highlighting}
\end{Shaded}

\begin{dmath*}\breakingcomma
i
\end{dmath*}

\begin{Shaded}
\begin{Highlighting}[]
\NormalTok{DiracIndex}\OperatorTok{[}\FunctionTok{i}\OperatorTok{]} \SpecialCharTok{//} \FunctionTok{StandardForm}

\CommentTok{(*DiracIndex[i]*)}
\end{Highlighting}
\end{Shaded}

\begin{Shaded}
\begin{Highlighting}[]
\NormalTok{DiracIndex}\OperatorTok{[}\DecValTok{2}\OperatorTok{]}
\end{Highlighting}
\end{Shaded}

\begin{dmath*}\breakingcomma
2
\end{dmath*}

\begin{Shaded}
\begin{Highlighting}[]
\NormalTok{DiracIndex}\OperatorTok{[}\DecValTok{2}\OperatorTok{]} \SpecialCharTok{//} \FunctionTok{StandardForm}

\CommentTok{(*ExplicitDiracIndex[2]*)}
\end{Highlighting}
\end{Shaded}

\begin{Shaded}
\begin{Highlighting}[]
\NormalTok{DIDelta}\OperatorTok{[}\FunctionTok{i}\OperatorTok{,} \FunctionTok{j}\OperatorTok{]} \SpecialCharTok{//}\NormalTok{ FCI }\SpecialCharTok{//} \FunctionTok{StandardForm}

\CommentTok{(*DiracIndexDelta[DiracIndex[i], DiracIndex[j]]*)}
\end{Highlighting}
\end{Shaded}

\end{document}
