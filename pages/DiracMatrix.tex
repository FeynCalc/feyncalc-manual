% !TeX program = pdflatex
% !TeX root = DiracMatrix.tex

\documentclass[../FeynCalcManual.tex]{subfiles}
\begin{document}
\hypertarget{diracmatrix}{
\section{DiracMatrix}\label{diracmatrix}\index{DiracMatrix}}

\texttt{DiracMatrix[\allowbreak{}mu]} denotes a Dirac gamma matrix with
Lorentz index \(\mu\).

\texttt{DiracMatrix[\allowbreak{}mu ,\ \allowbreak{}nu ,\ \allowbreak{}...]}
is a product of \(\gamma\) matrices with Lorentz indices
\texttt{mu ,\ \allowbreak{}nu ,\ \allowbreak{}...}

\texttt{DiracMatrix[\allowbreak{}5]} is \(\gamma ^5\).

\texttt{DiracMatrix[\allowbreak{}6]} is \((1 + \gamma^5)/2\).

\texttt{DiracMatrix[\allowbreak{}7]} is \((1 - \gamma^5)/2\).

The shortcut \texttt{DiracMatrix} is deprecated, please use \texttt{GA}
instead!

\subsection{See also}

\hyperlink{toc}{Overview}, \hyperlink{ga}{GA}, \hyperlink{fci}{FCI}.

\subsection{Examples}

\begin{Shaded}
\begin{Highlighting}[]
\NormalTok{DiracMatrix}\OperatorTok{[}\SpecialCharTok{\textbackslash{}}\OperatorTok{[}\NormalTok{Mu}\OperatorTok{]]}
\end{Highlighting}
\end{Shaded}

\begin{dmath*}\breakingcomma
\bar{\gamma }^{\mu }
\end{dmath*}

This is how to enter the non-commutative product of two. The Mathematica
Dot ``.'' is used as non-commutative multiplication operator.

\begin{Shaded}
\begin{Highlighting}[]
\NormalTok{DiracMatrix}\OperatorTok{[}\SpecialCharTok{\textbackslash{}}\OperatorTok{[}\NormalTok{Mu}\OperatorTok{]]}\NormalTok{ . DiracMatrix}\OperatorTok{[}\SpecialCharTok{\textbackslash{}}\OperatorTok{[}\NormalTok{Nu}\OperatorTok{]]}
\end{Highlighting}
\end{Shaded}

\begin{dmath*}\breakingcomma
\bar{\gamma }^{\mu }.\bar{\gamma }^{\nu }
\end{dmath*}

\begin{Shaded}
\begin{Highlighting}[]
\NormalTok{DiracMatrix}\OperatorTok{[}\SpecialCharTok{\textbackslash{}}\OperatorTok{[}\NormalTok{Alpha}\OperatorTok{]]} \SpecialCharTok{//} \FunctionTok{StandardForm}

\CommentTok{(*DiracGamma[LorentzIndex[\textbackslash{}[Alpha]]]*)}
\end{Highlighting}
\end{Shaded}

\texttt{DiracMatrix} is scheduled for removal in the future versions of
FeynCalc. The safe alternative is to use \texttt{GA}.

\begin{Shaded}
\begin{Highlighting}[]
\NormalTok{GA}\OperatorTok{[}\SpecialCharTok{\textbackslash{}}\OperatorTok{[}\NormalTok{Mu}\OperatorTok{]]}
\end{Highlighting}
\end{Shaded}

\begin{dmath*}\breakingcomma
\bar{\gamma }^{\mu }
\end{dmath*}

\begin{Shaded}
\begin{Highlighting}[]
\NormalTok{GAD}\OperatorTok{[}\SpecialCharTok{\textbackslash{}}\OperatorTok{[}\NormalTok{Mu}\OperatorTok{]]}
\end{Highlighting}
\end{Shaded}

\begin{dmath*}\breakingcomma
\gamma ^{\mu }
\end{dmath*}

\begin{Shaded}
\begin{Highlighting}[]
\NormalTok{FCI}\OperatorTok{[}\NormalTok{GA}\OperatorTok{[}\SpecialCharTok{\textbackslash{}}\OperatorTok{[}\NormalTok{Mu}\OperatorTok{]]]} \ExtensionTok{===}\NormalTok{ DiracMatrix}\OperatorTok{[}\SpecialCharTok{\textbackslash{}}\OperatorTok{[}\NormalTok{Mu}\OperatorTok{]]}
\end{Highlighting}
\end{Shaded}

\begin{dmath*}\breakingcomma
\text{True}
\end{dmath*}

\begin{Shaded}
\begin{Highlighting}[]
\NormalTok{FCI}\OperatorTok{[}\NormalTok{GAD}\OperatorTok{[}\SpecialCharTok{\textbackslash{}}\OperatorTok{[}\NormalTok{Mu}\OperatorTok{]]]} \ExtensionTok{===}\NormalTok{ DiracMatrix}\OperatorTok{[}\SpecialCharTok{\textbackslash{}}\OperatorTok{[}\NormalTok{Mu}\OperatorTok{],}\NormalTok{ Dimension }\OtherTok{{-}\textgreater{}} \FunctionTok{D}\OperatorTok{]}
\end{Highlighting}
\end{Shaded}

\begin{dmath*}\breakingcomma
\text{True}
\end{dmath*}
\end{document}
