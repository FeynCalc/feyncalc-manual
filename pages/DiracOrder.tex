% !TeX program = pdflatex
% !TeX root = DiracOrder.tex

\documentclass[../FeynCalcManual.tex]{subfiles}
\begin{document}
\hypertarget{diracorder}{
\section{DiracOrder}\label{diracorder}\index{DiracOrder}}

\texttt{DiracOrder[\allowbreak{}exp]} orders the Dirac matrices in
\texttt{exp} lexicographically.
\texttt{DiracOrder[\allowbreak{}exp,\ \allowbreak{}orderlist]} orders
the Dirac matrices in \texttt{exp} according to \texttt{orderlist}.
\texttt{DiracOrder} is also an option of \texttt{DiracSimplify} and some
other functions dealing with Dirac algebra. If set to \texttt{True}, the
function \texttt{DiracOrder} will be applied to the intermediate result
to reorder the Dirac matrices lexicographically.

\subsection{See also}

\hyperlink{toc}{Overview}, \hyperlink{diracsimplify}{DiracSimplify},
\hyperlink{diractrick}{DiracTrick}.

\subsection{Examples}

\begin{Shaded}
\begin{Highlighting}[]
\NormalTok{GA}\OperatorTok{[}\SpecialCharTok{\textbackslash{}}\OperatorTok{[}\FunctionTok{Beta}\OperatorTok{],} \SpecialCharTok{\textbackslash{}}\OperatorTok{[}\NormalTok{Alpha}\OperatorTok{]]} 
 
\NormalTok{DiracOrder}\OperatorTok{[}\SpecialCharTok{\%}\OperatorTok{]}
\end{Highlighting}
\end{Shaded}

\begin{dmath*}\breakingcomma
\bar{\gamma }^{\beta }.\bar{\gamma }^{\alpha }
\end{dmath*}

\begin{dmath*}\breakingcomma
2 \bar{g}^{\alpha \beta }-\bar{\gamma }^{\alpha }.\bar{\gamma }^{\beta }
\end{dmath*}

\texttt{DiracOrder} also works with Dirac matrices in \(D\)-dimensions.

\begin{Shaded}
\begin{Highlighting}[]
\NormalTok{GAD}\OperatorTok{[}\SpecialCharTok{\textbackslash{}}\OperatorTok{[}\NormalTok{Rho}\OperatorTok{],} \SpecialCharTok{\textbackslash{}}\OperatorTok{[}\NormalTok{Nu}\OperatorTok{],} \SpecialCharTok{\textbackslash{}}\OperatorTok{[}\NormalTok{Mu}\OperatorTok{],} \SpecialCharTok{\textbackslash{}}\OperatorTok{[}\NormalTok{Nu}\OperatorTok{]]} 
 
\NormalTok{DiracOrder}\OperatorTok{[}\SpecialCharTok{\%}\OperatorTok{]}
\end{Highlighting}
\end{Shaded}

\begin{dmath*}\breakingcomma
\gamma ^{\rho }.\gamma ^{\nu }.\gamma ^{\mu }.\gamma ^{\nu }
\end{dmath*}

\begin{dmath*}\breakingcomma
(D-2) \gamma ^{\mu }.\gamma ^{\rho }+2 (2-D) g^{\mu \rho }
\end{dmath*}

By default \(\gamma^5\) is moved to the right.

\begin{Shaded}
\begin{Highlighting}[]
\NormalTok{GA}\OperatorTok{[}\DecValTok{5}\OperatorTok{,} \SpecialCharTok{\textbackslash{}}\OperatorTok{[}\NormalTok{Mu}\OperatorTok{],} \SpecialCharTok{\textbackslash{}}\OperatorTok{[}\NormalTok{Nu}\OperatorTok{]]} 
 
\NormalTok{DiracOrder}\OperatorTok{[}\SpecialCharTok{\%}\OperatorTok{]}
\end{Highlighting}
\end{Shaded}

\begin{dmath*}\breakingcomma
\bar{\gamma }^5.\bar{\gamma }^{\mu }.\bar{\gamma }^{\nu }
\end{dmath*}

\begin{dmath*}\breakingcomma
\bar{\gamma }^{\mu }.\bar{\gamma }^{\nu }.\bar{\gamma }^5
\end{dmath*}

\begin{Shaded}
\begin{Highlighting}[]
\NormalTok{GA}\OperatorTok{[}\DecValTok{6}\OperatorTok{,} \SpecialCharTok{\textbackslash{}}\OperatorTok{[}\NormalTok{Mu}\OperatorTok{],} \DecValTok{7}\OperatorTok{]} 
 
\NormalTok{DiracOrder}\OperatorTok{[}\SpecialCharTok{\%}\OperatorTok{]}
\end{Highlighting}
\end{Shaded}

\begin{dmath*}\breakingcomma
\bar{\gamma }^6.\bar{\gamma }^{\mu }.\bar{\gamma }^7
\end{dmath*}

\begin{dmath*}\breakingcomma
\bar{\gamma }^{\mu }.\bar{\gamma }^7
\end{dmath*}

\texttt{orderlist} comes into play when we need an ordering that is not
lexicographic

\begin{Shaded}
\begin{Highlighting}[]
\NormalTok{GA}\OperatorTok{[}\SpecialCharTok{\textbackslash{}}\OperatorTok{[}\NormalTok{Alpha}\OperatorTok{],} \SpecialCharTok{\textbackslash{}}\OperatorTok{[}\FunctionTok{Beta}\OperatorTok{],} \SpecialCharTok{\textbackslash{}}\OperatorTok{[}\NormalTok{Delta}\OperatorTok{]]} 
 
\NormalTok{DiracOrder}\OperatorTok{[}\SpecialCharTok{\%}\OperatorTok{]}
\end{Highlighting}
\end{Shaded}

\begin{dmath*}\breakingcomma
\bar{\gamma }^{\alpha }.\bar{\gamma }^{\beta }.\bar{\gamma }^{\delta }
\end{dmath*}

\begin{dmath*}\breakingcomma
\bar{\gamma }^{\alpha }.\bar{\gamma }^{\beta }.\bar{\gamma }^{\delta }
\end{dmath*}

\begin{Shaded}
\begin{Highlighting}[]
\NormalTok{DiracOrder}\OperatorTok{[}\NormalTok{GA}\OperatorTok{[}\SpecialCharTok{\textbackslash{}}\OperatorTok{[}\NormalTok{Alpha}\OperatorTok{],} \SpecialCharTok{\textbackslash{}}\OperatorTok{[}\FunctionTok{Beta}\OperatorTok{],} \SpecialCharTok{\textbackslash{}}\OperatorTok{[}\NormalTok{Delta}\OperatorTok{]],} \OperatorTok{\{}\SpecialCharTok{\textbackslash{}}\OperatorTok{[}\NormalTok{Delta}\OperatorTok{],} \SpecialCharTok{\textbackslash{}}\OperatorTok{[}\FunctionTok{Beta}\OperatorTok{],} \SpecialCharTok{\textbackslash{}}\OperatorTok{[}\NormalTok{Alpha}\OperatorTok{]\}]}
\end{Highlighting}
\end{Shaded}

\begin{dmath*}\breakingcomma
-\bar{\gamma }^{\delta }.\bar{\gamma }^{\beta }.\bar{\gamma }^{\alpha }+2 \bar{\gamma }^{\delta } \bar{g}^{\alpha \beta }-2 \bar{\gamma }^{\beta } \bar{g}^{\alpha \delta }+2 \bar{\gamma }^{\alpha } \bar{g}^{\beta \delta }
\end{dmath*}

Reordering of Dirac matrices in long chains is expensive, so that
\texttt{DiracSimplify} does not do it by default.

\begin{Shaded}
\begin{Highlighting}[]
\NormalTok{DiracSimplify}\OperatorTok{[}\NormalTok{GAD}\OperatorTok{[}\SpecialCharTok{\textbackslash{}}\OperatorTok{[}\NormalTok{Mu}\OperatorTok{],} \SpecialCharTok{\textbackslash{}}\OperatorTok{[}\NormalTok{Nu}\OperatorTok{]]} \SpecialCharTok{+}\NormalTok{ GAD}\OperatorTok{[}\SpecialCharTok{\textbackslash{}}\OperatorTok{[}\NormalTok{Nu}\OperatorTok{],} \SpecialCharTok{\textbackslash{}}\OperatorTok{[}\NormalTok{Mu}\OperatorTok{]]]}
\end{Highlighting}
\end{Shaded}

\begin{dmath*}\breakingcomma
\gamma ^{\mu }.\gamma ^{\nu }+\gamma ^{\nu }.\gamma ^{\mu }
\end{dmath*}

However, if you know that it can lead to simpler expressions, you can
activate the reordering via the option \texttt{DiracOrder}.

\begin{Shaded}
\begin{Highlighting}[]
\NormalTok{DiracSimplify}\OperatorTok{[}\NormalTok{GAD}\OperatorTok{[}\SpecialCharTok{\textbackslash{}}\OperatorTok{[}\NormalTok{Mu}\OperatorTok{],} \SpecialCharTok{\textbackslash{}}\OperatorTok{[}\NormalTok{Nu}\OperatorTok{]]} \SpecialCharTok{+}\NormalTok{ GAD}\OperatorTok{[}\SpecialCharTok{\textbackslash{}}\OperatorTok{[}\NormalTok{Nu}\OperatorTok{],} \SpecialCharTok{\textbackslash{}}\OperatorTok{[}\NormalTok{Mu}\OperatorTok{]],}\NormalTok{ DiracOrder }\OtherTok{{-}\textgreater{}} \ConstantTok{True}\OperatorTok{]}
\end{Highlighting}
\end{Shaded}

\begin{dmath*}\breakingcomma
2 g^{\mu \nu }
\end{dmath*}
\end{document}
