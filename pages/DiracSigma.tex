% !TeX program = pdflatex
% !TeX root = DiracSigma.tex

\documentclass[../FeynCalcManual.tex]{subfiles}
\begin{document}
\hypertarget{diracsigma}{
\section{DiracSigma}\label{diracsigma}\index{DiracSigma}}

\texttt{DiracSigma[\allowbreak{}a,\ \allowbreak{}b]} stands for
\(I/2(a.b-b.a)\) in 4 dimensions.

\texttt{a} and \texttt{b} must have head \texttt{DiracGamma},
\texttt{GA} or \texttt{GS}. Only antisymmetry is implemented.

\subsection{See also}

\hyperlink{toc}{Overview},
\hyperlink{diracsigmaexplicit}{DiracSigmaExplicit}.

\subsection{Examples}

\begin{Shaded}
\begin{Highlighting}[]
\NormalTok{DiracSigma}\OperatorTok{[}\NormalTok{GA}\OperatorTok{[}\SpecialCharTok{\textbackslash{}}\OperatorTok{[}\NormalTok{Alpha}\OperatorTok{]],}\NormalTok{ GA}\OperatorTok{[}\SpecialCharTok{\textbackslash{}}\OperatorTok{[}\FunctionTok{Beta}\OperatorTok{]]]} 
 
\NormalTok{DiracSigmaExplicit}\OperatorTok{[}\SpecialCharTok{\%}\OperatorTok{]}
\end{Highlighting}
\end{Shaded}

\begin{dmath*}\breakingcomma
\sigma ^{\alpha \beta }
\end{dmath*}

\begin{dmath*}\breakingcomma
\frac{1}{2} i \left(\bar{\gamma }^{\alpha }.\bar{\gamma }^{\beta }-\bar{\gamma }^{\beta }.\bar{\gamma }^{\alpha }\right)
\end{dmath*}

\begin{Shaded}
\begin{Highlighting}[]
\NormalTok{DiracSigma}\OperatorTok{[}\NormalTok{GA}\OperatorTok{[}\SpecialCharTok{\textbackslash{}}\OperatorTok{[}\FunctionTok{Beta}\OperatorTok{]],}\NormalTok{ GA}\OperatorTok{[}\SpecialCharTok{\textbackslash{}}\OperatorTok{[}\NormalTok{Alpha}\OperatorTok{]]]}
\end{Highlighting}
\end{Shaded}

\begin{dmath*}\breakingcomma
-\sigma ^{\alpha \beta }
\end{dmath*}

\begin{Shaded}
\begin{Highlighting}[]
\NormalTok{DiracSigma}\OperatorTok{[}\NormalTok{GS}\OperatorTok{[}\FunctionTok{p}\OperatorTok{],}\NormalTok{ GS}\OperatorTok{[}\FunctionTok{q}\OperatorTok{]]} 
 
\NormalTok{DiracSigmaExplicit}\OperatorTok{[}\SpecialCharTok{\%}\OperatorTok{]}
\end{Highlighting}
\end{Shaded}

\begin{dmath*}\breakingcomma
\sigma ^{pq}
\end{dmath*}

\begin{dmath*}\breakingcomma
\frac{1}{2} i \left(\left(\bar{\gamma }\cdot \overline{p}\right).\left(\bar{\gamma }\cdot \overline{q}\right)-\left(\bar{\gamma }\cdot \overline{q}\right).\left(\bar{\gamma }\cdot \overline{p}\right)\right)
\end{dmath*}

The antisymmetry property is built-in

\begin{Shaded}
\begin{Highlighting}[]
\NormalTok{DiracSigma}\OperatorTok{[}\NormalTok{GA}\OperatorTok{[}\SpecialCharTok{\textbackslash{}}\OperatorTok{[}\NormalTok{Alpha}\OperatorTok{]],}\NormalTok{ GA}\OperatorTok{[}\SpecialCharTok{\textbackslash{}}\OperatorTok{[}\NormalTok{Alpha}\OperatorTok{]]]}
\end{Highlighting}
\end{Shaded}

\begin{dmath*}\breakingcomma
0
\end{dmath*}
\end{document}
