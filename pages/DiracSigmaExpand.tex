% !TeX program = pdflatex
% !TeX root = DiracSigmaExpand.tex

\documentclass[../FeynCalcManual.tex]{subfiles}
\begin{document}
\hypertarget{diracsigmaexpand}{
\section{DiracSigmaExpand}\label{diracsigmaexpand}\index{DiracSigmaExpand}}

\texttt{DiracSigmaExpand[\allowbreak{}exp]} applies linearity to the
arguments of \texttt{DiracSigma}.

\subsection{See also}

\hyperlink{toc}{Overview}, \hyperlink{diracgamma}{DiracGamma},
\hyperlink{diracsigma}{DiracSigma}.

\subsection{Examples}

\begin{Shaded}
\begin{Highlighting}[]
\NormalTok{DiracSigma}\OperatorTok{[}\NormalTok{GSD}\OperatorTok{[}\FunctionTok{p}\OperatorTok{]} \SpecialCharTok{+}\NormalTok{ GSD}\OperatorTok{[}\FunctionTok{q}\OperatorTok{],}\NormalTok{ GSD}\OperatorTok{[}\FunctionTok{r}\OperatorTok{]]} 
 
\NormalTok{ex }\ExtensionTok{=} \SpecialCharTok{\%} \SpecialCharTok{//}\NormalTok{ DiracSigmaExpand}
\end{Highlighting}
\end{Shaded}

\begin{dmath*}\breakingcomma
\text{DiracSigma}(\gamma \cdot p+\gamma \cdot q,\gamma \cdot r)
\end{dmath*}

\begin{dmath*}\breakingcomma
\sigma ^{pr}+\sigma ^{qr}
\end{dmath*}

\begin{Shaded}
\begin{Highlighting}[]
\NormalTok{ex }\SpecialCharTok{//}\NormalTok{ FCE }\SpecialCharTok{//} \FunctionTok{StandardForm}

\CommentTok{(*DiracSigma[GSD[p], GSD[r]] + DiracSigma[GSD[q], GSD[r]]*)}
\end{Highlighting}
\end{Shaded}

Notice that DiracSigmaExpand does not expand Dirac matrices contracted
to linear combinations of \(4\)-vectors by default.

\begin{Shaded}
\begin{Highlighting}[]
\NormalTok{DiracSigma}\OperatorTok{[}\NormalTok{GSD}\OperatorTok{[}\FunctionTok{p} \SpecialCharTok{+} \FunctionTok{q}\OperatorTok{],}\NormalTok{ GSD}\OperatorTok{[}\FunctionTok{r}\OperatorTok{]]} 
 
\NormalTok{DiracSigmaExpand}\OperatorTok{[}\SpecialCharTok{\%}\OperatorTok{]}
\end{Highlighting}
\end{Shaded}

\begin{dmath*}\breakingcomma
\sigma ^{p+qr}
\end{dmath*}

\begin{dmath*}\breakingcomma
\sigma ^{p+qr}
\end{dmath*}

If such expansions are required, use the option
\texttt{DiracGammaExpand}.

\begin{Shaded}
\begin{Highlighting}[]
\NormalTok{DiracSigmaExpand}\OperatorTok{[}\NormalTok{DiracSigma}\OperatorTok{[}\NormalTok{GSD}\OperatorTok{[}\FunctionTok{p} \SpecialCharTok{+} \FunctionTok{q}\OperatorTok{],}\NormalTok{ GSD}\OperatorTok{[}\FunctionTok{r}\OperatorTok{]],}\NormalTok{ DiracGammaExpand }\OtherTok{{-}\textgreater{}} \ConstantTok{True}\OperatorTok{]}
\end{Highlighting}
\end{Shaded}

\begin{dmath*}\breakingcomma
\sigma ^{pr}+\sigma ^{qr}
\end{dmath*}

The option Momentum allows us to perform more fine-grained expansions of
\texttt{DiracSigma}.

\begin{Shaded}
\begin{Highlighting}[]
\NormalTok{DiracSigma}\OperatorTok{[}\NormalTok{GSD}\OperatorTok{[}\FunctionTok{p}\OperatorTok{],}\NormalTok{ GSD}\OperatorTok{[}\FunctionTok{r}\OperatorTok{]} \SpecialCharTok{+}\NormalTok{ GSD}\OperatorTok{[}\FunctionTok{t}\OperatorTok{]]} \SpecialCharTok{+}\NormalTok{ DiracSigma}\OperatorTok{[}\NormalTok{GSD}\OperatorTok{[}\FunctionTok{l}\OperatorTok{]} \SpecialCharTok{+}\NormalTok{ GSD}\OperatorTok{[}\FunctionTok{n}\OperatorTok{],}\NormalTok{ GSD}\OperatorTok{[}\FunctionTok{p}\OperatorTok{]]} 
 
\NormalTok{DiracSigmaExpand}\OperatorTok{[}\SpecialCharTok{\%}\OperatorTok{,}\NormalTok{ Momentum }\OtherTok{{-}\textgreater{}} \OperatorTok{\{}\FunctionTok{r}\OperatorTok{\}]}
\end{Highlighting}
\end{Shaded}

\begin{dmath*}\breakingcomma
\text{DiracSigma}(\gamma \cdot l+\gamma \cdot n,\gamma \cdot p)+\text{DiracSigma}(\gamma \cdot p,\gamma \cdot r+\gamma \cdot t)
\end{dmath*}

\begin{dmath*}\breakingcomma
\text{DiracSigma}(\gamma \cdot l+\gamma \cdot n,\gamma \cdot p)+\sigma ^{pr}+\sigma ^{pt}
\end{dmath*}
\end{document}
