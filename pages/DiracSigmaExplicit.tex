% !TeX program = pdflatex
% !TeX root = DiracSigmaExplicit.tex

\documentclass[../FeynCalcManual.tex]{subfiles}
\begin{document}
\hypertarget{diracsigmaexplicit}{
\section{DiracSigmaExplicit}\label{diracsigmaexplicit}\index{DiracSigmaExplicit}}

\texttt{DiracSigmaExplicit[\allowbreak{}exp]} inserts in exp for all
\texttt{DiracSigma} its definition. \texttt{DiracSigmaExplict} is also
an option of \texttt{DiracSimplify}. \texttt{DiracSigmaExplict} is also
an option of various FeynCalc functions that handle the Dirac algebra.

\subsection{See also}

\hyperlink{toc}{Overview}, \hyperlink{diracgamma}{DiracGamma},
\hyperlink{diracsigma}{DiracSigma}.

\subsection{Examples}

\begin{Shaded}
\begin{Highlighting}[]
\NormalTok{DiracSigma}\OperatorTok{[}\NormalTok{GA}\OperatorTok{[}\SpecialCharTok{\textbackslash{}}\OperatorTok{[}\NormalTok{Alpha}\OperatorTok{]],}\NormalTok{ GA}\OperatorTok{[}\SpecialCharTok{\textbackslash{}}\OperatorTok{[}\FunctionTok{Beta}\OperatorTok{]]]} 
 
\NormalTok{DiracSigmaExplicit}\OperatorTok{[}\SpecialCharTok{\%}\OperatorTok{]}
\end{Highlighting}
\end{Shaded}

\begin{dmath*}\breakingcomma
\sigma ^{\alpha \beta }
\end{dmath*}

\begin{dmath*}\breakingcomma
\frac{1}{2} i \left(\bar{\gamma }^{\alpha }.\bar{\gamma }^{\beta }-\bar{\gamma }^{\beta }.\bar{\gamma }^{\alpha }\right)
\end{dmath*}

\begin{Shaded}
\begin{Highlighting}[]
\NormalTok{GSD}\OperatorTok{[}\FunctionTok{p}\OperatorTok{]}\NormalTok{ . DiracSigma}\OperatorTok{[}\NormalTok{GAD}\OperatorTok{[}\SpecialCharTok{\textbackslash{}}\OperatorTok{[}\NormalTok{Mu}\OperatorTok{]],}\NormalTok{ GAD}\OperatorTok{[}\SpecialCharTok{\textbackslash{}}\OperatorTok{[}\NormalTok{Nu}\OperatorTok{]]]}\NormalTok{ . GSD}\OperatorTok{[}\FunctionTok{q}\OperatorTok{]} 
 
\NormalTok{DiracSigmaExplicit}\OperatorTok{[}\SpecialCharTok{\%}\OperatorTok{]}
\end{Highlighting}
\end{Shaded}

\begin{dmath*}\breakingcomma
(\gamma \cdot p).\sigma ^{\mu \nu }.(\gamma \cdot q)
\end{dmath*}

\begin{dmath*}\breakingcomma
\frac{1}{2} i \left((\gamma \cdot p).\gamma ^{\mu }.\gamma ^{\nu }.(\gamma \cdot q)-(\gamma \cdot p).\gamma ^{\nu }.\gamma ^{\mu }.(\gamma \cdot q)\right)
\end{dmath*}
\end{document}
