% !TeX program = pdflatex
% !TeX root = DiracSlash.tex

\documentclass[../FeynCalcManual.tex]{subfiles}
\begin{document}
\hypertarget{diracslash}{%
\section{DiracSlash}\label{diracslash}}

\texttt{DiracSlash[\allowbreak{}p]} is the contraction
\(p^{\mu } \gamma _{\mu }\)
(\texttt{FV[\allowbreak{}p,\ \allowbreak{}mu] GA[\allowbreak{}mu]}).

Products of those can be entered in the form GS{[}p1, p2, \ldots{]}.

The shortcut DiracSlash is deprecated, please use GS instead!

\subsection{See also}

\hyperlink{toc}{Overview}, \hyperlink{gs}{GS}, \hyperlink{fci}{FCI}.

\subsection{Examples}

This is \(q\)-slash, i.e., \(\gamma^{\mu} q_{\mu }\)

\begin{Shaded}
\begin{Highlighting}[]
\NormalTok{DiracSlash}\OperatorTok{[}\FunctionTok{q}\OperatorTok{]}
\end{Highlighting}
\end{Shaded}

\begin{dmath*}\breakingcomma
\bar{\gamma }\cdot \overline{q}
\end{dmath*}

\begin{Shaded}
\begin{Highlighting}[]
\NormalTok{DiracSlash}\OperatorTok{[}\FunctionTok{p}\OperatorTok{]}\NormalTok{ . DiracSlash}\OperatorTok{[}\FunctionTok{q}\OperatorTok{]}
\end{Highlighting}
\end{Shaded}

\begin{dmath*}\breakingcomma
\left(\bar{\gamma }\cdot \overline{p}\right).\left(\bar{\gamma }\cdot \overline{q}\right)
\end{dmath*}

\begin{Shaded}
\begin{Highlighting}[]
\NormalTok{DiracSlash}\OperatorTok{[}\FunctionTok{p}\OperatorTok{,} \FunctionTok{q}\OperatorTok{]}
\end{Highlighting}
\end{Shaded}

\begin{dmath*}\breakingcomma
\left(\bar{\gamma }\cdot \overline{p}\right).\left(\bar{\gamma }\cdot \overline{q}\right)
\end{dmath*}

DiracSlash is scheduled for removal in the future versions of FeynCalc.
The safe alternative is to use GS.

\begin{Shaded}
\begin{Highlighting}[]
\NormalTok{GS}\OperatorTok{[}\FunctionTok{p}\OperatorTok{]}
\end{Highlighting}
\end{Shaded}

\begin{dmath*}\breakingcomma
\bar{\gamma }\cdot \overline{p}
\end{dmath*}

\begin{Shaded}
\begin{Highlighting}[]
\NormalTok{GSD}\OperatorTok{[}\FunctionTok{p}\OperatorTok{]}
\end{Highlighting}
\end{Shaded}

\begin{dmath*}\breakingcomma
\gamma \cdot p
\end{dmath*}

\begin{Shaded}
\begin{Highlighting}[]
\NormalTok{FCI}\OperatorTok{[}\NormalTok{GS}\OperatorTok{[}\FunctionTok{p}\OperatorTok{]]} \ExtensionTok{===}\NormalTok{ DiracSlash}\OperatorTok{[}\FunctionTok{p}\OperatorTok{]}
\end{Highlighting}
\end{Shaded}

\begin{dmath*}\breakingcomma
\text{True}
\end{dmath*}

\begin{Shaded}
\begin{Highlighting}[]
\NormalTok{FCI}\OperatorTok{[}\NormalTok{GSD}\OperatorTok{[}\FunctionTok{p}\OperatorTok{]]} \ExtensionTok{===}\NormalTok{ DiracSlash}\OperatorTok{[}\FunctionTok{p}\OperatorTok{,}\NormalTok{ Dimension }\OtherTok{{-}\textgreater{}} \FunctionTok{D}\OperatorTok{]}
\end{Highlighting}
\end{Shaded}

\begin{dmath*}\breakingcomma
\text{True}
\end{dmath*}
\end{document}
