% !TeX program = pdflatex
% !TeX root = DiracSpinorNormalization.tex

\documentclass[../FeynCalcManual.tex]{subfiles}
\begin{document}
\hypertarget{diracspinornormalization}{
\section{DiracSpinorNormalization}\label{diracspinornormalization}\index{DiracSpinorNormalization}}

\texttt{DiracSpinorNormalization} is an option for
\texttt{SpinorChainEvaluate}, \texttt{DiracSimplify} and other
functions. It specifies the normalization of the spinor inner products
\(\bar{u}(p) u(p)\) and \(\bar{v}(p) v(p)\). Following values are
supported:

\begin{itemize}
\item
  \texttt{"Relativistic"} - this is the standard value corresponding to
  \(\bar{u}(p) u(p) = 2 m\), \(\bar{v}(p) v(p) = - 2 m\).
\item
  \texttt{"Rest"} - this sets \(\bar{u}(p) u(p) = 1\),
  \(\bar{v}(p) v(p) = - 1\).
\item
  \texttt{"Nonrelativistic"} - this sets
  \(\bar{u}(p) u(p) = \frac{m}{p^0}\),
  \(\bar{v}(p) v(p) = - \frac{m}{p^0}\).
\end{itemize}

\subsection{See also}

\hyperlink{toc}{Overview}, \hyperlink{diracsimplify}{DiracSimplify},
\hyperlink{spinorchainevaluate}{SpinorChainEvaluate}.

\subsection{Examples}

\begin{Shaded}
\begin{Highlighting}[]
\NormalTok{SpinorUBar}\OperatorTok{[}\FunctionTok{p}\OperatorTok{,} \FunctionTok{m}\OperatorTok{]}\NormalTok{ . SpinorU}\OperatorTok{[}\FunctionTok{p}\OperatorTok{,} \FunctionTok{m}\OperatorTok{]} 
 
\NormalTok{DiracSimplify}\OperatorTok{[}\SpecialCharTok{\%}\OperatorTok{]}
\end{Highlighting}
\end{Shaded}

\begin{dmath*}\breakingcomma
\bar{u}(p,m).u(p,m)
\end{dmath*}

\begin{dmath*}\breakingcomma
2 m
\end{dmath*}

\begin{Shaded}
\begin{Highlighting}[]
\NormalTok{SpinorUBar}\OperatorTok{[}\FunctionTok{p}\OperatorTok{,} \FunctionTok{m}\OperatorTok{]}\NormalTok{ . SpinorU}\OperatorTok{[}\FunctionTok{p}\OperatorTok{,} \FunctionTok{m}\OperatorTok{]} 
 
\NormalTok{DiracSimplify}\OperatorTok{[}\SpecialCharTok{\%}\OperatorTok{,}\NormalTok{ DiracSpinorNormalization }\OtherTok{{-}\textgreater{}} \StringTok{"Rest"}\OperatorTok{]}
\end{Highlighting}
\end{Shaded}

\begin{dmath*}\breakingcomma
\bar{u}(p,m).u(p,m)
\end{dmath*}

\begin{dmath*}\breakingcomma
1
\end{dmath*}

\begin{Shaded}
\begin{Highlighting}[]
\NormalTok{SpinorUBar}\OperatorTok{[}\FunctionTok{p}\OperatorTok{,} \FunctionTok{m}\OperatorTok{]}\NormalTok{ . SpinorU}\OperatorTok{[}\FunctionTok{p}\OperatorTok{,} \FunctionTok{m}\OperatorTok{]} 
 
\NormalTok{DiracSimplify}\OperatorTok{[}\SpecialCharTok{\%}\OperatorTok{,}\NormalTok{ DiracSpinorNormalization }\OtherTok{{-}\textgreater{}} \StringTok{"Nonrelativistic"}\OperatorTok{]}
\end{Highlighting}
\end{Shaded}

\begin{dmath*}\breakingcomma
\bar{u}(p,m).u(p,m)
\end{dmath*}

\begin{dmath*}\breakingcomma
\frac{m}{p^0}
\end{dmath*}
\end{document}
