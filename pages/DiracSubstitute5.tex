% !TeX program = pdflatex
% !TeX root = DiracSubstitute5.tex

\documentclass[../FeynCalcManual.tex]{subfiles}
\begin{document}
\hypertarget{diracsubstitute5}{%
\section{DiracSubstitute5}\label{diracsubstitute5}}

DiracSubstitute5{[}exp{]} rewrites \(\gamma^5\) in terms of the
chirality projectors \(\gamma^6\) and \(\gamma^7\).
\texttt{DiracSubstitute5} is also an option of various FeynCalc
functions that handle Dirac algebra.

\subsection{See also}

\hyperlink{toc}{Overview},
\hyperlink{diracsubstitute67}{DiracSubstitute67},
\hyperlink{diracgamma}{DiracGamma},
\hyperlink{todiracgamma67}{ToDiracGamma67}.

\subsection{Examples}

\begin{Shaded}
\begin{Highlighting}[]
\NormalTok{GA}\OperatorTok{[}\DecValTok{5}\OperatorTok{]} 
 
\NormalTok{DiracSubstitute5}\OperatorTok{[}\SpecialCharTok{\%}\OperatorTok{]}
\end{Highlighting}
\end{Shaded}

\begin{dmath*}\breakingcomma
\bar{\gamma }^5
\end{dmath*}

\begin{dmath*}\breakingcomma
\bar{\gamma }^6-\bar{\gamma }^7
\end{dmath*}

\begin{Shaded}
\begin{Highlighting}[]
\NormalTok{SpinorUBar}\OperatorTok{[}\FunctionTok{Subscript}\OperatorTok{[}\FunctionTok{p}\OperatorTok{,} \DecValTok{1}\OperatorTok{]]}\NormalTok{ . GA}\OperatorTok{[}\SpecialCharTok{\textbackslash{}}\OperatorTok{[}\NormalTok{Mu}\OperatorTok{]]}\NormalTok{ . GA}\OperatorTok{[}\DecValTok{5}\OperatorTok{]}\NormalTok{ . GA}\OperatorTok{[}\SpecialCharTok{\textbackslash{}}\OperatorTok{[}\NormalTok{Nu}\OperatorTok{]]}\NormalTok{ . SpinorU}\OperatorTok{[}\FunctionTok{Subscript}\OperatorTok{[}\FunctionTok{p}\OperatorTok{,} \DecValTok{2}\OperatorTok{]]} 
 
\NormalTok{DiracSubstitute5}\OperatorTok{[}\SpecialCharTok{\%}\OperatorTok{]}
\end{Highlighting}
\end{Shaded}

\begin{dmath*}\breakingcomma
\bar{u}\left(p_1\right).\bar{\gamma }^{\mu }.\bar{\gamma }^5.\bar{\gamma }^{\nu }.u\left(p_2\right)
\end{dmath*}

\begin{dmath*}\breakingcomma
\left(\varphi (\overline{p}_1)\right).\bar{\gamma }^{\mu }.\left(\bar{\gamma }^6-\bar{\gamma }^7\right).\bar{\gamma }^{\nu }.\left(\varphi (\overline{p}_2)\right)
\end{dmath*}
\end{document}
