% !TeX program = pdflatex
% !TeX root = DiracSubstitute67.tex

\documentclass[../FeynCalcManual.tex]{subfiles}
\begin{document}
\hypertarget{diracsubstitute67}{
\section{DiracSubstitute67}\label{diracsubstitute67}\index{DiracSubstitute67}}

\texttt{DiracSubstitute67[\allowbreak{}exp]} inserts the explicit
definitions of the chirality projectors \(\gamma^6\) and \(\gamma^7\).
\texttt{DiracSubstitute67} is also an option of various FeynCalc
functions that handle Dirac algebra.

\subsection{See also}

\hyperlink{toc}{Overview},
\hyperlink{diracsubstitute5}{DiracSubstitute5},
\hyperlink{diracgamma}{DiracGamma},
\hyperlink{todiracgamma67}{ToDiracGamma67}.

\subsection{Examples}

\begin{Shaded}
\begin{Highlighting}[]
\NormalTok{DiracGamma}\OperatorTok{[}\DecValTok{6}\OperatorTok{]} 
 
\NormalTok{DiracSubstitute67}\OperatorTok{[}\SpecialCharTok{\%}\OperatorTok{]}
\end{Highlighting}
\end{Shaded}

\begin{dmath*}\breakingcomma
\bar{\gamma }^6
\end{dmath*}

\begin{dmath*}\breakingcomma
\frac{\bar{\gamma }^5}{2}+\frac{1}{2}
\end{dmath*}

\begin{Shaded}
\begin{Highlighting}[]
\NormalTok{DiracGamma}\OperatorTok{[}\DecValTok{7}\OperatorTok{]} 
 
\NormalTok{DiracSubstitute67}\OperatorTok{[}\SpecialCharTok{\%}\OperatorTok{]}
\end{Highlighting}
\end{Shaded}

\begin{dmath*}\breakingcomma
\bar{\gamma }^7
\end{dmath*}

\begin{dmath*}\breakingcomma
\frac{1}{2}-\frac{\bar{\gamma }^5}{2}
\end{dmath*}

\begin{Shaded}
\begin{Highlighting}[]
\NormalTok{SpinorUBar}\OperatorTok{[}\FunctionTok{Subscript}\OperatorTok{[}\FunctionTok{p}\OperatorTok{,} \DecValTok{1}\OperatorTok{]]}\NormalTok{ . GA}\OperatorTok{[}\DecValTok{6}\OperatorTok{]}\NormalTok{ . SpinorU}\OperatorTok{[}\FunctionTok{Subscript}\OperatorTok{[}\FunctionTok{p}\OperatorTok{,} \DecValTok{2}\OperatorTok{]]} 
 
\NormalTok{DiracSubstitute67}\OperatorTok{[}\SpecialCharTok{\%}\OperatorTok{]}
\end{Highlighting}
\end{Shaded}

\begin{dmath*}\breakingcomma
\bar{u}\left(p_1\right).\bar{\gamma }^6.u\left(p_2\right)
\end{dmath*}

\begin{dmath*}\breakingcomma
\left(\varphi (\overline{p}_1)\right).\left(\frac{\bar{\gamma }^5}{2}+\frac{1}{2}\right).\left(\varphi (\overline{p}_2)\right)
\end{dmath*}

\begin{Shaded}
\begin{Highlighting}[]
\NormalTok{SpinorUBar}\OperatorTok{[}\FunctionTok{Subscript}\OperatorTok{[}\FunctionTok{p}\OperatorTok{,} \DecValTok{1}\OperatorTok{]]}\NormalTok{ . GA}\OperatorTok{[}\DecValTok{7}\OperatorTok{]}\NormalTok{ . SpinorU}\OperatorTok{[}\FunctionTok{Subscript}\OperatorTok{[}\FunctionTok{p}\OperatorTok{,} \DecValTok{2}\OperatorTok{]]} 
 
\NormalTok{DiracSubstitute67}\OperatorTok{[}\SpecialCharTok{\%}\OperatorTok{]}
\end{Highlighting}
\end{Shaded}

\begin{dmath*}\breakingcomma
\bar{u}\left(p_1\right).\bar{\gamma }^7.u\left(p_2\right)
\end{dmath*}

\begin{dmath*}\breakingcomma
\left(\varphi (\overline{p}_1)\right).\left(\frac{1}{2}-\frac{\bar{\gamma }^5}{2}\right).\left(\varphi (\overline{p}_2)\right)
\end{dmath*}
\end{document}
