% !TeX program = pdflatex
% !TeX root = DiracTrick.tex

\documentclass[../FeynCalcManual.tex]{subfiles}
\begin{document}
\hypertarget{diractrick}{%
\section{DiracTrick}\label{diractrick}}

\texttt{DiracTrick[\allowbreak{}exp]} contracts Dirac matrices with each
other and performs several simplifications but no expansions.There are
not many cases when a user will need to call this function directly. Use
\texttt{DiracSimplify} to achieve maximal simplification of Dirac matrix
chains. Regarding the treatment of \(\gamma^5\) in \(D\)-dimensional
expressions or the evaluation of expressions with tensors living in
different dimensions, see the explanations on the help pages for
\texttt{DiracSimplify} and \texttt{DiracTrace}.

\subsection{See also}

\hyperlink{toc}{Overview}, \hyperlink{contract}{Contract},
\hyperlink{diracequation}{DiracEquation},
\hyperlink{diracgamma}{DiracGamma},
\hyperlink{diracgammaexpand}{DiracGammaExpand},
\hyperlink{diractrick}{DiracTrick},
\hyperlink{sirlinsimplify}{SirlinSimplify},
\hyperlink{spinorchaintrick}{SpinorChainTrick}.

\subsection{Examples}

When applied to chains of Dirac matrices that do not require
noncommutative expansions, contractions with other tensors,
simplifications of spinor chains or evaluations of Dirac traces,
\texttt{DiracTrick} will produce results similar to those of
\texttt{DiracSimplify}.

\begin{Shaded}
\begin{Highlighting}[]
\NormalTok{GA}\OperatorTok{[}\SpecialCharTok{\textbackslash{}}\OperatorTok{[}\NormalTok{Mu}\OperatorTok{],} \SpecialCharTok{\textbackslash{}}\OperatorTok{[}\NormalTok{Nu}\OperatorTok{],} \SpecialCharTok{\textbackslash{}}\OperatorTok{[}\NormalTok{Mu}\OperatorTok{]]} 
 
\NormalTok{DiracTrick}\OperatorTok{[}\SpecialCharTok{\%}\OperatorTok{]}
\end{Highlighting}
\end{Shaded}

\begin{dmath*}\breakingcomma
\bar{\gamma }^{\mu }.\bar{\gamma }^{\nu }.\bar{\gamma }^{\mu }
\end{dmath*}

\begin{dmath*}\breakingcomma
-2 \bar{\gamma }^{\nu }
\end{dmath*}

\begin{Shaded}
\begin{Highlighting}[]
\NormalTok{GS}\OperatorTok{[}\FunctionTok{p}\OperatorTok{]}\NormalTok{ . GS}\OperatorTok{[}\FunctionTok{p}\OperatorTok{]} 
 
\NormalTok{DiracTrick}\OperatorTok{[}\SpecialCharTok{\%}\OperatorTok{]}
\end{Highlighting}
\end{Shaded}

\begin{dmath*}\breakingcomma
\left(\bar{\gamma }\cdot \overline{p}\right).\left(\bar{\gamma }\cdot \overline{p}\right)
\end{dmath*}

\begin{dmath*}\breakingcomma
\overline{p}^2
\end{dmath*}

\begin{Shaded}
\begin{Highlighting}[]
\NormalTok{GA}\OperatorTok{[}\DecValTok{5}\OperatorTok{,} \SpecialCharTok{\textbackslash{}}\OperatorTok{[}\NormalTok{Mu}\OperatorTok{],} \SpecialCharTok{\textbackslash{}}\OperatorTok{[}\NormalTok{Nu}\OperatorTok{]]} 
 
\NormalTok{DiracTrick}\OperatorTok{[}\SpecialCharTok{\%}\OperatorTok{]}
\end{Highlighting}
\end{Shaded}

\begin{dmath*}\breakingcomma
\bar{\gamma }^5.\bar{\gamma }^{\mu }.\bar{\gamma }^{\nu }
\end{dmath*}

\begin{dmath*}\breakingcomma
\bar{\gamma }^{\mu }.\bar{\gamma }^{\nu }.\bar{\gamma }^5
\end{dmath*}

\begin{Shaded}
\begin{Highlighting}[]
\NormalTok{(}\DecValTok{1}\SpecialCharTok{/}\DecValTok{2} \SpecialCharTok{{-}}\NormalTok{ GA}\OperatorTok{[}\DecValTok{5}\OperatorTok{]}\SpecialCharTok{/}\DecValTok{2}\NormalTok{) . (}\SpecialCharTok{{-}}\NormalTok{((}\FunctionTok{a} \SpecialCharTok{+}\NormalTok{ GS}\OperatorTok{[}\FunctionTok{p} \SpecialCharTok{+} \FunctionTok{q}\OperatorTok{]}\NormalTok{)}\SpecialCharTok{/}\FunctionTok{b}\NormalTok{)) . (}\DecValTok{1}\SpecialCharTok{/}\DecValTok{2} \SpecialCharTok{+}\NormalTok{ GA}\OperatorTok{[}\DecValTok{5}\OperatorTok{]}\SpecialCharTok{/}\DecValTok{2}\NormalTok{) }
 
\NormalTok{DiracTrick}\OperatorTok{[}\SpecialCharTok{\%}\OperatorTok{]}
\end{Highlighting}
\end{Shaded}

\begin{dmath*}\breakingcomma
\left(\frac{1}{2}-\frac{\bar{\gamma }^5}{2}\right).\left(-\frac{\bar{\gamma }\cdot \left(\overline{p}+\overline{q}\right)+a}{b}\right).\left(\frac{\bar{\gamma }^5}{2}+\frac{1}{2}\right)
\end{dmath*}

\begin{dmath*}\breakingcomma
-\frac{\left(\bar{\gamma }\cdot \left(\overline{p}+\overline{q}\right)\right).\bar{\gamma }^6}{b}
\end{dmath*}

Dirac traces are not evaluated by \texttt{DiracTrick}

\begin{Shaded}
\begin{Highlighting}[]
\NormalTok{DiracTrace}\OperatorTok{[}\NormalTok{GAD}\OperatorTok{[}\SpecialCharTok{\textbackslash{}}\OperatorTok{[}\NormalTok{Mu}\OperatorTok{],} \SpecialCharTok{\textbackslash{}}\OperatorTok{[}\NormalTok{Nu}\OperatorTok{]]]} 
 
\NormalTok{DiracTrick}\OperatorTok{[}\SpecialCharTok{\%}\OperatorTok{]}
\end{Highlighting}
\end{Shaded}

\begin{dmath*}\breakingcomma
\text{tr}\left(\gamma ^{\mu }.\gamma ^{\nu }\right)
\end{dmath*}

\begin{dmath*}\breakingcomma
\text{tr}\left(\gamma ^{\mu }.\gamma ^{\nu }\right)
\end{dmath*}
\end{document}
