% !TeX program = pdflatex
% !TeX root = DollarAL.tex

\documentclass[../FeynCalcManual.tex]{subfiles}
\begin{document}
\hypertarget{dollaral}{
\section{\$AL}\label{dollaral}\index{\$AL}}

\texttt{\$AL} is the head for dummy indices which may be introduced by
\texttt{Amputate} and \texttt{Uncontract}. By default it is unset, but
may be set to anything.

\subsection{See also}

\hyperlink{toc}{Overview}, \hyperlink{amputate}{Amputate},
\hyperlink{uncontract}{Uncontract}.

\subsection{Examples}

\begin{Shaded}
\begin{Highlighting}[]
\NormalTok{Uncontract}\OperatorTok{[}\NormalTok{ScalarProduct}\OperatorTok{[}\FunctionTok{p}\OperatorTok{,} \FunctionTok{q}\OperatorTok{],} \FunctionTok{q}\OperatorTok{,}\NormalTok{ Pair }\OtherTok{{-}\textgreater{}} \ConstantTok{All}\OperatorTok{]}
\end{Highlighting}
\end{Shaded}

\begin{dmath*}\breakingcomma
\overline{p}^{\text{\$AL}(\text{\$19})} \overline{q}^{\text{\$AL}(\text{\$19})}
\end{dmath*}

\begin{Shaded}
\begin{Highlighting}[]
\NormalTok{$AL }\ExtensionTok{=} \SpecialCharTok{\textbackslash{}}\OperatorTok{[}\NormalTok{Mu}\OperatorTok{]}\NormalTok{; }
 
\NormalTok{Uncontract}\OperatorTok{[}\NormalTok{ScalarProduct}\OperatorTok{[}\FunctionTok{p}\OperatorTok{,} \FunctionTok{q}\OperatorTok{],} \FunctionTok{q}\OperatorTok{,}\NormalTok{ Pair }\OtherTok{{-}\textgreater{}} \ConstantTok{All}\OperatorTok{]}
\end{Highlighting}
\end{Shaded}

\begin{dmath*}\breakingcomma
\overline{p}^{\mu (\text{\$20})} \overline{q}^{\mu (\text{\$20})}
\end{dmath*}

\begin{Shaded}
\begin{Highlighting}[]
\NormalTok{$AL }\ExtensionTok{=}\NormalTok{.;}
\end{Highlighting}
\end{Shaded}

\end{document}
