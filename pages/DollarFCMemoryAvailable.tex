% !TeX program = pdflatex
% !TeX root = DollarFCMemoryAvailable.tex

\documentclass[../FeynCalcManual.tex]{subfiles}
\begin{document}
\hypertarget{fcmemoryavailable}{%
\section{\$FCMemoryAvailable}\label{fcmemoryavailable}}

\texttt{\$FCMemoryAvailable} is a global variable which is set to an
integer \texttt{n}, where \texttt{n} is the available amount of main
memory in MB. The default is \texttt{1/4} of \texttt{\$SystemMemory}. It
should be increased if possible. The higher \$FCMemoryAvailable can be,
the more intermediate steps do not have to be repeated by FeynCalc.

\subsection{See also}

\hyperlink{toc}{Overview}, \hyperlink{memset}{MemSet},
\hyperlink{fcmemoryavailable}{FCMemoryAvailable}.

\subsection{Examples}

\begin{Shaded}
\begin{Highlighting}[]
\NormalTok{$SystemMemory}
\end{Highlighting}
\end{Shaded}

\begin{dmath*}\breakingcomma
32898408448
\end{dmath*}

\begin{Shaded}
\begin{Highlighting}[]
\FunctionTok{Floor}\OperatorTok{[}\NormalTok{$SystemMemory}\SpecialCharTok{/}\DecValTok{10}\SpecialCharTok{\^{}}\DecValTok{6}\SpecialCharTok{/}\DecValTok{4}\OperatorTok{]}
\end{Highlighting}
\end{Shaded}

\begin{dmath*}\breakingcomma
8224
\end{dmath*}

\begin{Shaded}
\begin{Highlighting}[]
\NormalTok{$FCMemoryAvailable}
\end{Highlighting}
\end{Shaded}

\begin{dmath*}\breakingcomma
8224
\end{dmath*}
\end{document}
