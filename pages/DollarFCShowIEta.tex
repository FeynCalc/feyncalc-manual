% !TeX program = pdflatex
% !TeX root = DollarFCShowIEta.tex

\documentclass[../FeynCalcManual.tex]{subfiles}
\begin{document}
\hypertarget{dollarfcshowieta}{
\section{\$FCShowIEta}\label{dollarfcshowieta}\index{\$FCShowIEta}}

The Boolean setting of \texttt{\$FCShowIEta} determines whether
\(i \eta\) should be displayed in the typesetting of propagator objects
(except for \texttt{FAD}s) or not. This setting affects only the
TraditionalForm typesetting and has absolutely no influence on the
internal handling of propagator denominators in FeynCalc.

\subsection{See also}

\hyperlink{toc}{Overview}, \hyperlink{sfad}{SFAD},
\hyperlink{cfad}{CFAD}, \hyperlink{gfad}{GFAD}.

\subsection{Examples}

\begin{Shaded}
\begin{Highlighting}[]
\NormalTok{$FCShowIEta }
 
\NormalTok{SFAD}\OperatorTok{[\{}\FunctionTok{p}\OperatorTok{,} \FunctionTok{m}\SpecialCharTok{\^{}}\DecValTok{2}\OperatorTok{\}]}
\end{Highlighting}
\end{Shaded}

\begin{dmath*}\breakingcomma
\text{True}
\end{dmath*}

\begin{dmath*}\breakingcomma
\frac{1}{(p^2-m^2+i \eta )}
\end{dmath*}

\begin{Shaded}
\begin{Highlighting}[]
\NormalTok{$FCShowIEta }\ExtensionTok{=} \ConstantTok{False} 
 
\NormalTok{SFAD}\OperatorTok{[\{}\FunctionTok{p}\OperatorTok{,} \FunctionTok{m}\SpecialCharTok{\^{}}\DecValTok{2}\OperatorTok{\}]}
\end{Highlighting}
\end{Shaded}

\begin{dmath*}\breakingcomma
\text{False}
\end{dmath*}

\begin{dmath*}\breakingcomma
\frac{1}{(p^2-m^2)}
\end{dmath*}

\begin{Shaded}
\begin{Highlighting}[]
\NormalTok{$FCShowIEta }\ExtensionTok{=} \ConstantTok{True}
\end{Highlighting}
\end{Shaded}

\begin{dmath*}\breakingcomma
\text{True}
\end{dmath*}
\end{document}
