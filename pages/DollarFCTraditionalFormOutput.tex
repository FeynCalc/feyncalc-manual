% !TeX program = pdflatex
% !TeX root = DollarFCTraditionalFormOutput.tex

\documentclass[../FeynCalcManual.tex]{subfiles}
\begin{document}
\hypertarget{fctraditionalformoutput}{%
\section{\$FCTraditionalFormOutput}\label{fctraditionalformoutput}}

The Boolean setting of \texttt{\$FCTraditionalFormOutput} determines
which output format type should be used in the notebook front end when
FeynCalc is loaded. If set to \texttt{True}, FeynCalc will activate the
\texttt{TraditionalForm} output. Otherwise, the \texttt{StandardForm}
output (Mathematica's default) will be used.

This setting only changes the output format of the current notebook,
i.e.~it is not persistent and will not modify the global options of
Mathematica.

If unsure, it is recommended to set \texttt{\$FCTraditionalFormOutput}
to \texttt{True}, so that you can benefit from the nice FeynCalc
typesetting for various QFT quantities.

\subsection{See also}

\hyperlink{toc}{Overview},
\hyperlink{fcenabletraditionalformoutput}{FCEnableTraditionalFormOutput},
\hyperlink{fcdisabletraditionalformoutput}{FCDisableTraditionalFormOutput}.

\subsection{Examples}
\end{document}
