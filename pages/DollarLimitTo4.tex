% !TeX program = pdflatex
% !TeX root = DollarLimitTo4.tex

\documentclass[../FeynCalcManual.tex]{subfiles}
\begin{document}
\hypertarget{dollarlimitto4}{
\section{\$LimitTo4}\label{dollarlimitto4}\index{\$LimitTo4}}

\texttt{\$LimitTo4} is a variable with default setting \texttt{False}.
If set to \texttt{True}, the limit \texttt{Dimension -> 4} is performed
after tensor integral decomposition.

\$LimitTo4 is a global variable that determines whether UV-divergent
Passarino-Veltman functions are simplified by taking the limit
\(D-4 \to 0\).

A generic IR-finite Passarino-Veltman function \(X\) can be written as
\(X = \frac{a}{D-4} + b + \mathcal{O}(\varepsilon)\), with \(a\) being
the prefactor of the pole and \(b\) being the finite part. Therefore,
products of such functions with coefficients that are rational functions
of \(D\) with \(f(D) = f(4) + (D-4) f'(4) + \mathcal{O}(\varepsilon^2)\)
can be simplified to
\(f(D) X = f(4) X + a f'(4) + \mathcal{O}(\varepsilon)\), whenever such
products appear in the reduction.

This relation is correct only if the Passarino-Veltman functions have no
IR divergences, or if such divergences are regulated without using
dimensional regularization.

For this reason, even when \$LimitTo4 is set to \texttt{True}, the
simplifications are applied only to \(A\) and \(B\) functions. Although
\(B\) functions can exhibit an IR divergence, such integrals are zero in
dimensional regularization, so that no mixing of \(\varepsilon\)-terms
from IR and UV can occur.

The default value of \texttt{\$LimitTo4} is \texttt{False}. Notice that
even when the switch is set to \texttt{True}, it will essentially affect
only the Passarino-Veltman reduction via \texttt{PaVeReduce}.

The modern and more flexible way to simplify amplitudes involving
IR-finite \texttt{PaVe} functions is to use the special routine
\texttt{PaVeLimitTo4}.

\subsection{See also}

\hyperlink{toc}{Overview}, \hyperlink{pave}{PaVe},
\hyperlink{pavereduce}{PaVeReduce}, \hyperlink{oneloop}{OneLoop},
\hyperlink{dollarlimitto4irunsafe}{\$LimitTo4IRUnsafe},
\hyperlink{pavelimitto4}{PaVeLimitTo4}.

\subsection{Examples}

\begin{Shaded}
\begin{Highlighting}[]
\NormalTok{$LimitTo4}
\end{Highlighting}
\end{Shaded}

\begin{dmath*}\breakingcomma
\text{False}
\end{dmath*}
\end{document}
