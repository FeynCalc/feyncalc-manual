% !TeX program = pdflatex
% !TeX root = Eps.tex

\documentclass[../FeynCalcManual.tex]{subfiles}
\begin{document}
\hypertarget{eps}{
\section{Eps}\label{eps}\index{Eps}}

\texttt{Eps[\allowbreak{}a,\ \allowbreak{}b,\ \allowbreak{}c,\ \allowbreak{}d]}
is the head of the totally antisymmetric \(\epsilon\) (Levi-Civita)
tensor. The \texttt{a,\ \allowbreak{}b,\ \allowbreak{}...} may have head
\texttt{LorentzIndex} or \texttt{Momentum}.

\subsection{See also}

\hyperlink{toc}{Overview}, \hyperlink{epscontract}{EpsContract},
\hyperlink{epsevaluate}{EpsEvaluate}, \hyperlink{lc}{LC},
\hyperlink{lcd}{LCD}.

\subsection{Examples}

\begin{Shaded}
\begin{Highlighting}[]
\NormalTok{Eps}\OperatorTok{[}\NormalTok{LorentzIndex}\OperatorTok{[}\SpecialCharTok{\textbackslash{}}\OperatorTok{[}\NormalTok{Mu}\OperatorTok{]],}\NormalTok{ LorentzIndex}\OperatorTok{[}\SpecialCharTok{\textbackslash{}}\OperatorTok{[}\NormalTok{Nu}\OperatorTok{]],}\NormalTok{ LorentzIndex}\OperatorTok{[}\SpecialCharTok{\textbackslash{}}\OperatorTok{[}\NormalTok{Rho}\OperatorTok{]],}\NormalTok{ LorentzIndex}\OperatorTok{[}\SpecialCharTok{\textbackslash{}}\OperatorTok{[}\NormalTok{Sigma}\OperatorTok{]]]}
\end{Highlighting}
\end{Shaded}

\begin{dmath*}\breakingcomma
\bar{\epsilon }^{\mu \nu \rho \sigma }
\end{dmath*}

\begin{Shaded}
\begin{Highlighting}[]
\NormalTok{Eps}\OperatorTok{[}\NormalTok{Momentum}\OperatorTok{[}\FunctionTok{p}\OperatorTok{],}\NormalTok{ LorentzIndex}\OperatorTok{[}\SpecialCharTok{\textbackslash{}}\OperatorTok{[}\NormalTok{Nu}\OperatorTok{]],}\NormalTok{ LorentzIndex}\OperatorTok{[}\SpecialCharTok{\textbackslash{}}\OperatorTok{[}\NormalTok{Rho}\OperatorTok{]],}\NormalTok{ LorentzIndex}\OperatorTok{[}\SpecialCharTok{\textbackslash{}}\OperatorTok{[}\NormalTok{Sigma}\OperatorTok{]]]}
\end{Highlighting}
\end{Shaded}

\begin{dmath*}\breakingcomma
\bar{\epsilon }^{\overline{p}\nu \rho \sigma }
\end{dmath*}

\begin{Shaded}
\begin{Highlighting}[]
\NormalTok{Eps}\OperatorTok{[}\FunctionTok{b}\OperatorTok{,} \FunctionTok{a}\OperatorTok{,} \FunctionTok{c}\OperatorTok{,} \FunctionTok{d}\OperatorTok{]} \SpecialCharTok{//} \FunctionTok{StandardForm}

\CommentTok{(*Eps[b, a, c, d]*)}
\end{Highlighting}
\end{Shaded}

\begin{Shaded}
\begin{Highlighting}[]
\NormalTok{Eps}\OperatorTok{[}\NormalTok{ExplicitLorentzIndex}\OperatorTok{[}\DecValTok{0}\OperatorTok{],}\NormalTok{ ExplicitLorentzIndex}\OperatorTok{[}\DecValTok{1}\OperatorTok{],}\NormalTok{ ExplicitLorentzIndex}\OperatorTok{[}\DecValTok{2}\OperatorTok{],} 
\NormalTok{  ExplicitLorentzIndex}\OperatorTok{[}\DecValTok{3}\OperatorTok{]]}
\end{Highlighting}
\end{Shaded}

\begin{dmath*}\breakingcomma
\bar{\epsilon }^{0123}
\end{dmath*}

\begin{Shaded}
\begin{Highlighting}[]
\NormalTok{Eps}\OperatorTok{[}\NormalTok{LorentzIndex}\OperatorTok{[}\SpecialCharTok{\textbackslash{}}\OperatorTok{[}\NormalTok{Mu}\OperatorTok{]],}\NormalTok{ LorentzIndex}\OperatorTok{[}\SpecialCharTok{\textbackslash{}}\OperatorTok{[}\NormalTok{Nu}\OperatorTok{]],}\NormalTok{ LorentzIndex}\OperatorTok{[}\SpecialCharTok{\textbackslash{}}\OperatorTok{[}\NormalTok{Rho}\OperatorTok{]],}\NormalTok{ LorentzIndex}\OperatorTok{[}\SpecialCharTok{\textbackslash{}}\OperatorTok{[}\NormalTok{Sigma}\OperatorTok{]]]} 
 
\NormalTok{Contract}\OperatorTok{[}\SpecialCharTok{\%} \SpecialCharTok{\%}\OperatorTok{]}
\end{Highlighting}
\end{Shaded}

\begin{dmath*}\breakingcomma
\bar{\epsilon }^{\mu \nu \rho \sigma }
\end{dmath*}

\begin{dmath*}\breakingcomma
-24
\end{dmath*}

\begin{Shaded}
\begin{Highlighting}[]
\NormalTok{Eps}\OperatorTok{[}\NormalTok{LorentzIndex}\OperatorTok{[}\SpecialCharTok{\textbackslash{}}\OperatorTok{[}\NormalTok{Mu}\OperatorTok{],} \FunctionTok{D}\OperatorTok{],}\NormalTok{ LorentzIndex}\OperatorTok{[}\SpecialCharTok{\textbackslash{}}\OperatorTok{[}\NormalTok{Nu}\OperatorTok{],} \FunctionTok{D}\OperatorTok{],}\NormalTok{ LorentzIndex}\OperatorTok{[}\SpecialCharTok{\textbackslash{}}\OperatorTok{[}\NormalTok{Rho}\OperatorTok{],} \FunctionTok{D}\OperatorTok{],}\NormalTok{ LorentzIndex}\OperatorTok{[}\SpecialCharTok{\textbackslash{}}\OperatorTok{[}\NormalTok{Sigma}\OperatorTok{],} \FunctionTok{D}\OperatorTok{]]} 
 
\NormalTok{Contract}\OperatorTok{[}\SpecialCharTok{\%} \SpecialCharTok{\%}\OperatorTok{]} \SpecialCharTok{//}\NormalTok{ Factor2}
\end{Highlighting}
\end{Shaded}

\begin{dmath*}\breakingcomma
\overset{\text{}}{\epsilon }^{\mu \nu \rho \sigma }
\end{dmath*}

\begin{dmath*}\breakingcomma
(1-D) (2-D) (3-D) D
\end{dmath*}

\begin{Shaded}
\begin{Highlighting}[]
\NormalTok{ex1 }\ExtensionTok{=} \SpecialCharTok{{-}}\NormalTok{(}\FunctionTok{I}\SpecialCharTok{/}\DecValTok{24}\NormalTok{) LCD}\OperatorTok{[}\SpecialCharTok{\textbackslash{}}\OperatorTok{[}\NormalTok{Mu}\OperatorTok{],} \SpecialCharTok{\textbackslash{}}\OperatorTok{[}\NormalTok{Nu}\OperatorTok{],} \SpecialCharTok{\textbackslash{}}\OperatorTok{[}\NormalTok{Rho}\OperatorTok{],} \SpecialCharTok{\textbackslash{}}\OperatorTok{[}\NormalTok{Alpha}\OperatorTok{]]}\NormalTok{ . GAD}\OperatorTok{[}\SpecialCharTok{\textbackslash{}}\OperatorTok{[}\NormalTok{Mu}\OperatorTok{],} \SpecialCharTok{\textbackslash{}}\OperatorTok{[}\NormalTok{Nu}\OperatorTok{],} \SpecialCharTok{\textbackslash{}}\OperatorTok{[}\NormalTok{Rho}\OperatorTok{],} \SpecialCharTok{\textbackslash{}}\OperatorTok{[}\NormalTok{Alpha}\OperatorTok{]]} \SpecialCharTok{//}\NormalTok{ FCI}
\end{Highlighting}
\end{Shaded}

\begin{dmath*}\breakingcomma
-\frac{1}{24} i \overset{\text{}}{\epsilon }^{\mu \nu \rho \alpha }.\gamma ^{\mu }.\gamma ^{\nu }.\gamma ^{\rho }.\gamma ^{\alpha }
\end{dmath*}

\begin{Shaded}
\begin{Highlighting}[]
\NormalTok{ex2 }\ExtensionTok{=} \SpecialCharTok{{-}}\NormalTok{(}\FunctionTok{I}\SpecialCharTok{/}\DecValTok{24}\NormalTok{) LCD}\OperatorTok{[}\SpecialCharTok{\textbackslash{}}\OperatorTok{[}\NormalTok{Mu}\OperatorTok{]}\NormalTok{\textquotesingle{}}\OperatorTok{,} \SpecialCharTok{\textbackslash{}}\OperatorTok{[}\NormalTok{Nu}\OperatorTok{]}\NormalTok{\textquotesingle{}}\OperatorTok{,} \SpecialCharTok{\textbackslash{}}\OperatorTok{[}\NormalTok{Rho}\OperatorTok{]}\NormalTok{\textquotesingle{}}\OperatorTok{,} \SpecialCharTok{\textbackslash{}}\OperatorTok{[}\NormalTok{Alpha}\OperatorTok{]}\NormalTok{\textquotesingle{}}\OperatorTok{]}\NormalTok{ . GAD}\OperatorTok{[}\SpecialCharTok{\textbackslash{}}\OperatorTok{[}\NormalTok{Mu}\OperatorTok{]}\NormalTok{\textquotesingle{}}\OperatorTok{,} \SpecialCharTok{\textbackslash{}}\OperatorTok{[}\NormalTok{Nu}\OperatorTok{]}\NormalTok{\textquotesingle{}}\OperatorTok{,} \SpecialCharTok{\textbackslash{}}\OperatorTok{[}\NormalTok{Rho}\OperatorTok{]}\NormalTok{\textquotesingle{}}\OperatorTok{,} \SpecialCharTok{\textbackslash{}}\OperatorTok{[}\NormalTok{Alpha}\OperatorTok{]}\NormalTok{\textquotesingle{}}\OperatorTok{]} \SpecialCharTok{//}\NormalTok{ FCI}
\end{Highlighting}
\end{Shaded}

\begin{dmath*}\breakingcomma
-\frac{1}{24} i \overset{\text{}}{\epsilon }^{\mu '\nu '\rho '\alpha '}.\gamma ^{\mu '}.\gamma ^{\nu '}.\gamma ^{\rho '}.\gamma ^{\alpha '}
\end{dmath*}

\begin{Shaded}
\begin{Highlighting}[]
\NormalTok{DiracSimplify}\OperatorTok{[}\NormalTok{ex1 . ex2}\OperatorTok{]} \SpecialCharTok{//}\NormalTok{ Factor2 }
 
\SpecialCharTok{\%} \OtherTok{/.} \FunctionTok{D} \OtherTok{{-}\textgreater{}} \DecValTok{4} 
  
 
\end{Highlighting}
\end{Shaded}

\begin{dmath*}\breakingcomma
-\frac{1}{24} (1-D) (2-D) (3-D) D
\end{dmath*}

\begin{dmath*}\breakingcomma
1
\end{dmath*}
\end{document}
