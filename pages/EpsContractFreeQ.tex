% !TeX program = pdflatex
% !TeX root = EpsContractFreeQ.tex

\documentclass[../FeynCalcManual.tex]{subfiles}
\begin{document}
\hypertarget{epscontractfreeq}{
\section{EpsContractFreeQ}\label{epscontractfreeq}\index{EpsContractFreeQ}}

\texttt{EpsContractFreeQ[\allowbreak{}exp]} returns \texttt{True} if the
expression contains epsilon tensors that can be contracted with each
other. The function is optimized for large expressions, i.e.~it is not
so good as a criterion in e.g.~\texttt{Select}.

\subsection{See also}

\hyperlink{toc}{Overview}, \hyperlink{contract}{Contract},
\hyperlink{epscontract}{EpsContract}.

\subsection{Examples}

\begin{Shaded}
\begin{Highlighting}[]
\NormalTok{FCI}\OperatorTok{[}\NormalTok{LC}\OperatorTok{[}\NormalTok{p1}\OperatorTok{,}\NormalTok{ p2}\OperatorTok{,}\NormalTok{ p3}\OperatorTok{,}\NormalTok{ p4}\OperatorTok{]]} 
 
\NormalTok{EpsContractFreeQ}\OperatorTok{[}\SpecialCharTok{\%}\OperatorTok{]}
\end{Highlighting}
\end{Shaded}

\begin{dmath*}\breakingcomma
\bar{\epsilon }^{\text{p1}\;\text{p2}\;\text{p3}\;\text{p4}}
\end{dmath*}

\begin{dmath*}\breakingcomma
\text{True}
\end{dmath*}

\begin{Shaded}
\begin{Highlighting}[]
\NormalTok{FCI}\OperatorTok{[}\NormalTok{LC}\OperatorTok{[}\NormalTok{p1}\OperatorTok{,}\NormalTok{ p2}\OperatorTok{,}\NormalTok{ p3}\OperatorTok{,}\NormalTok{ mu}\OperatorTok{]}\NormalTok{ LC}\OperatorTok{[}\NormalTok{q1}\OperatorTok{,}\NormalTok{ q2}\OperatorTok{,}\NormalTok{ q3}\OperatorTok{,}\NormalTok{ q4}\OperatorTok{]]} 
 
\NormalTok{EpsContractFreeQ}\OperatorTok{[}\SpecialCharTok{\%}\OperatorTok{]} 
  
 
\end{Highlighting}
\end{Shaded}

\begin{dmath*}\breakingcomma
\bar{\epsilon }^{\text{p1}\;\text{p2}\;\text{p3}\;\text{mu}} \bar{\epsilon }^{\text{q1}\;\text{q2}\;\text{q3}\;\text{q4}}
\end{dmath*}

\begin{dmath*}\breakingcomma
\text{False}
\end{dmath*}
\end{document}
