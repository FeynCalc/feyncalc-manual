% !TeX program = pdflatex
% !TeX root = EpsExpand.tex

\documentclass[../FeynCalcManual.tex]{subfiles}
\begin{document}
\hypertarget{epsexpand}{
\section{EpsExpand}\label{epsexpand}\index{EpsExpand}}

\texttt{EpsExpand} is an option for \texttt{EpsEvaluate} and other
functions that use \texttt{EpsEvaluate} internally. When set to
\texttt{False}, sums of momenta in the \texttt{Eps} tensor will not be
rewritten as a sum of \texttt{Eps} tensors.

\subsection{See also}

\hyperlink{toc}{Overview}, \hyperlink{epsevaluate}{EpsEvaluate}.

\subsection{Examples}

\begin{Shaded}
\begin{Highlighting}[]
\NormalTok{LC}\OperatorTok{[}\NormalTok{mu}\OperatorTok{,}\NormalTok{ nu}\OperatorTok{][}\NormalTok{q1 }\SpecialCharTok{+}\NormalTok{ q2}\OperatorTok{,}\NormalTok{ p1 }\SpecialCharTok{+}\NormalTok{ p2}\OperatorTok{]} 
 
\NormalTok{EpsEvaluate}\OperatorTok{[}\SpecialCharTok{\%}\OperatorTok{]}
\end{Highlighting}
\end{Shaded}

\begin{dmath*}\breakingcomma
\bar{\epsilon }^{\text{mu}\;\text{nu}\;\overline{\text{q1}}+\overline{\text{q2}}\;\overline{\text{p1}}+\overline{\text{p2}}}
\end{dmath*}

\begin{dmath*}\breakingcomma
-\bar{\epsilon }^{\text{mu}\;\text{nu}\;\overline{\text{p1}}\;\overline{\text{q1}}}-\bar{\epsilon }^{\text{mu}\;\text{nu}\;\overline{\text{p1}}\;\overline{\text{q2}}}-\bar{\epsilon }^{\text{mu}\;\text{nu}\;\overline{\text{p2}}\;\overline{\text{q1}}}-\bar{\epsilon }^{\text{mu}\;\text{nu}\;\overline{\text{p2}}\;\overline{\text{q2}}}
\end{dmath*}

\begin{Shaded}
\begin{Highlighting}[]
\NormalTok{LC}\OperatorTok{[}\NormalTok{mu}\OperatorTok{,}\NormalTok{ nu}\OperatorTok{][}\NormalTok{q1 }\SpecialCharTok{+}\NormalTok{ q2}\OperatorTok{,}\NormalTok{ p1 }\SpecialCharTok{+}\NormalTok{ p2}\OperatorTok{]} 
 
\NormalTok{EpsEvaluate}\OperatorTok{[}\SpecialCharTok{\%}\OperatorTok{,}\NormalTok{ EpsExpand }\OtherTok{{-}\textgreater{}} \ConstantTok{False}\OperatorTok{]}
\end{Highlighting}
\end{Shaded}

\begin{dmath*}\breakingcomma
\bar{\epsilon }^{\text{mu}\;\text{nu}\;\overline{\text{q1}}+\overline{\text{q2}}\;\overline{\text{p1}}+\overline{\text{p2}}}
\end{dmath*}

\begin{dmath*}\breakingcomma
-\bar{\epsilon }^{\text{mu}\;\text{nu}\;\overline{\text{p1}}+\overline{\text{p2}}\;\overline{\text{q1}}+\overline{\text{q2}}}
\end{dmath*}
\end{document}
