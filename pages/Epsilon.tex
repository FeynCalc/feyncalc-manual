% !TeX program = pdflatex
% !TeX root = Epsilon.tex

\documentclass[../FeynCalcManual.tex]{subfiles}
\begin{document}
\hypertarget{epsilon}{%
\section{Epsilon}\label{epsilon}}

Epsilon is \((n-4)\), where \(n\) is the space-time dimension.

\texttt{Epsilon} stands for a small positive number.

\subsection{See also}

\hyperlink{toc}{Overview}, \hyperlink{series2}{Series2}.

\subsection{Examples}

\begin{Shaded}
\begin{Highlighting}[]
\NormalTok{Epsilon}
\end{Highlighting}
\end{Shaded}

\begin{dmath*}\breakingcomma
\varepsilon
\end{dmath*}

Epsilon has no functional properties, but some upvalues are changed:

\begin{Shaded}
\begin{Highlighting}[]
\OperatorTok{\{}\FunctionTok{Re}\OperatorTok{[}\NormalTok{Epsilon}\OperatorTok{]}\NormalTok{ \textgreater{} }\SpecialCharTok{{-}}\DecValTok{4}\OperatorTok{,} \FunctionTok{Re}\OperatorTok{[}\NormalTok{Epsilon}\OperatorTok{]}\NormalTok{ \textgreater{} }\SpecialCharTok{{-}}\DecValTok{3}\OperatorTok{,} \FunctionTok{Re}\OperatorTok{[}\NormalTok{Epsilon}\OperatorTok{]}\NormalTok{ \textgreater{} }\SpecialCharTok{{-}}\DecValTok{2}\OperatorTok{,} \FunctionTok{Re}\OperatorTok{[}\NormalTok{Epsilon}\OperatorTok{]}\NormalTok{ \textgreater{} }\SpecialCharTok{{-}}\DecValTok{1}\OperatorTok{,} \FunctionTok{Re}\OperatorTok{[}\NormalTok{Epsilon}\OperatorTok{]}\NormalTok{ \textgreater{} }\DecValTok{0}\OperatorTok{\}}
\end{Highlighting}
\end{Shaded}

\begin{dmath*}\breakingcomma
\{\text{True},\text{True},\text{True},\text{True},\text{True}\}
\end{dmath*}
\end{document}
