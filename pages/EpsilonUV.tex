% !TeX program = pdflatex
% !TeX root = EpsilonUV.tex

\documentclass[../FeynCalcManual.tex]{subfiles}
\begin{document}
\hypertarget{epsilonuv}{
\section{EpsilonUV}\label{epsilonuv}\index{EpsilonUV}}

\texttt{EpsilonUV} denotes \((D-4)\), where \(D\) is the number of
space-time dimensions.

\texttt{EpsilonUV} stands for a small positive number that explicitly
regulates only UV divergences.

\subsection{See also}

\hyperlink{toc}{Overview}, \hyperlink{epsilon}{Epsilon},
\hyperlink{epsilonir}{EpsilonIR}.

\subsection{Examples}

\begin{Shaded}
\begin{Highlighting}[]
\NormalTok{EpsilonUV}
\end{Highlighting}
\end{Shaded}

\begin{dmath*}\breakingcomma
\varepsilon _{\text{UV}}
\end{dmath*}
\end{document}
