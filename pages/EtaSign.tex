% !TeX program = pdflatex
% !TeX root = EtaSign.tex

\documentclass[../FeynCalcManual.tex]{subfiles}
\begin{document}
\hypertarget{etasign}{%
\section{EtaSign}\label{etasign}}

\texttt{EtaSign} is an option for \texttt{SFAD}, \texttt{GFAD},
\texttt{CFAD} and other objects representing propagators. It specifies
the default sign of the \(i \eta\) prescription in the propagators,
e.g.~for standard Feynman propagators the value \texttt{1} corresponds
to \(\frac{1}{p^2-m^2 + i \eta}\), while the value \texttt{-1} sets
\(\frac{1}{p^2-m^2 - i \eta}\).

\subsection{See also}

\hyperlink{toc}{Overview}, \hyperlink{sfad}{SFAD},
\hyperlink{cfad}{CFAD}, \hyperlink{gfad}{GFAD}.

\subsection{Examples}

Notice that if the sign of \(i \eta\) is already specified in the
propagator, e.g.

\begin{Shaded}
\begin{Highlighting}[]
\NormalTok{CFAD}\OperatorTok{[\{}\FunctionTok{q}\OperatorTok{,} \OperatorTok{\{}\FunctionTok{m}\SpecialCharTok{\^{}}\DecValTok{2}\OperatorTok{,} \DecValTok{1}\OperatorTok{\}\}]}
\end{Highlighting}
\end{Shaded}

\begin{dmath*}\breakingcomma
\frac{1}{(q^2+m^2+i \eta )}
\end{dmath*}

then this specification always overrides the EtaSign option. Hence

\begin{Shaded}
\begin{Highlighting}[]
\NormalTok{CFAD}\OperatorTok{[\{}\FunctionTok{q}\OperatorTok{,} \OperatorTok{\{}\FunctionTok{m}\SpecialCharTok{\^{}}\DecValTok{2}\OperatorTok{,} \DecValTok{1}\OperatorTok{\}\},}\NormalTok{ EtaSign }\OtherTok{{-}\textgreater{}} \SpecialCharTok{{-}}\DecValTok{1}\OperatorTok{]}
\end{Highlighting}
\end{Shaded}

\begin{dmath*}\breakingcomma
\frac{1}{(q^2+m^2+i \eta )}
\end{dmath*}

still has the positive \(i \eta\).
\end{document}
