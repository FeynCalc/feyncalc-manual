% !TeX program = pdflatex
% !TeX root = Explicit.tex

\documentclass[../FeynCalcManual.tex]{subfiles}
\begin{document}
\hypertarget{explicit}{
\section{Explicit}\label{explicit}\index{Explicit}}

\texttt{Explicit[\allowbreak{}exp]} inserts explicit expressions of
\texttt{GluonVertex}, \texttt{Twist2GluonOperator}, \texttt{SUNF} etc.
in \texttt{exp}.

To rewrite the \(SU(N)\) structure constants in terms of traces, please
set the corresponding options \texttt{SUNF} or \texttt{SUND} to
\texttt{True}.

The color traces are left untouched unless the option \texttt{SUNTrace}
is set to \texttt{True}. In this case they will be rewritten in terms of
structure constants.

\texttt{Explicit} is also an option for \texttt{FieldStrength},
\texttt{GluonVertex}, \texttt{SUNF}, \texttt{Twist2GluonOperator} etc.
If set to \texttt{True} the full form of the operator is inserted.

\subsection{See also}

\hyperlink{toc}{Overview}, \hyperlink{gluonvertex}{GluonVertex},
\hyperlink{twist2gluonoperator}{Twist2GluonOperator}.

\subsection{Examples}

\begin{Shaded}
\begin{Highlighting}[]
\NormalTok{gv }\ExtensionTok{=}\NormalTok{ GluonVertex}\OperatorTok{[}\FunctionTok{p}\OperatorTok{,} \SpecialCharTok{\textbackslash{}}\OperatorTok{[}\NormalTok{Mu}\OperatorTok{],} \FunctionTok{a}\OperatorTok{,} \FunctionTok{q}\OperatorTok{,} \SpecialCharTok{\textbackslash{}}\OperatorTok{[}\NormalTok{Nu}\OperatorTok{],} \FunctionTok{b}\OperatorTok{,} \FunctionTok{r}\OperatorTok{,} \SpecialCharTok{\textbackslash{}}\OperatorTok{[}\NormalTok{Rho}\OperatorTok{],} \FunctionTok{c}\OperatorTok{]}
\end{Highlighting}
\end{Shaded}

\begin{dmath*}\breakingcomma
f^{abc} V^{\mu \nu \rho }(p\text{, }q\text{, }r)
\end{dmath*}

\begin{Shaded}
\begin{Highlighting}[]
\NormalTok{Explicit}\OperatorTok{[}\NormalTok{gv}\OperatorTok{]}
\end{Highlighting}
\end{Shaded}

\begin{dmath*}\breakingcomma
g_s f^{abc} \left(g^{\mu \nu } \left(p^{\rho }-q^{\rho }\right)+g^{\mu \rho } \left(r^{\nu }-p^{\nu }\right)+g^{\nu \rho } \left(q^{\mu }-r^{\mu }\right)\right)
\end{dmath*}

\begin{Shaded}
\begin{Highlighting}[]
\NormalTok{Explicit}\OperatorTok{[}\NormalTok{gv}\OperatorTok{,}\NormalTok{ SUNF }\OtherTok{{-}\textgreater{}} \ConstantTok{True}\OperatorTok{]}
\end{Highlighting}
\end{Shaded}

\begin{dmath*}\breakingcomma
2 i g_s \left(\text{tr}(T^a.T^c.T^b)-\text{tr}(T^a.T^b.T^c)\right) \left(g^{\mu \nu } \left(p^{\rho }-q^{\rho }\right)+g^{\mu \rho } \left(r^{\nu }-p^{\nu }\right)+g^{\nu \rho } \left(q^{\mu }-r^{\mu }\right)\right)
\end{dmath*}

\begin{Shaded}
\begin{Highlighting}[]
\NormalTok{Twist2GluonOperator}\OperatorTok{[}\FunctionTok{p}\OperatorTok{,} \SpecialCharTok{\textbackslash{}}\OperatorTok{[}\NormalTok{Mu}\OperatorTok{],} \FunctionTok{a}\OperatorTok{,} \SpecialCharTok{\textbackslash{}}\OperatorTok{[}\NormalTok{Nu}\OperatorTok{],} \FunctionTok{b}\OperatorTok{]} 
 
\NormalTok{Explicit}\OperatorTok{[}\SpecialCharTok{\%}\OperatorTok{]}
\end{Highlighting}
\end{Shaded}

\begin{dmath*}\breakingcomma
\frac{1}{2} \left((-1)^m+1\right) \delta ^{ab} \left(O_{\mu \, \nu }^{\text{G2}}(p)\right)
\end{dmath*}

\begin{dmath*}\breakingcomma
\frac{1}{2} \left((-1)^m+1\right) \delta ^{ab} (\Delta \cdot p)^{m-2} \left(g^{\mu \nu } (\Delta \cdot p)^2+p^2 \Delta ^{\mu } \Delta ^{\nu }-(\Delta \cdot p) \left(\Delta ^{\nu } p^{\mu }+\Delta ^{\mu } p^{\nu }\right)\right)
\end{dmath*}

\begin{Shaded}
\begin{Highlighting}[]
\NormalTok{FieldStrength}\OperatorTok{[}\SpecialCharTok{\textbackslash{}}\OperatorTok{[}\NormalTok{Mu}\OperatorTok{],} \SpecialCharTok{\textbackslash{}}\OperatorTok{[}\NormalTok{Nu}\OperatorTok{],} \FunctionTok{a}\OperatorTok{]} 
 
\NormalTok{Explicit}\OperatorTok{[}\SpecialCharTok{\%}\OperatorTok{]}
\end{Highlighting}
\end{Shaded}

\begin{dmath*}\breakingcomma
F_{\mu \nu }^a
\end{dmath*}

\begin{dmath*}\breakingcomma
g_s f^{a\text{b19}\;\text{c20}} A_{\mu }^{\text{b19}}.A_{\nu }^{\text{c20}}+\left(\partial _{\mu }A_{\nu }^a\right)-\left(\partial _{\nu }A_{\mu }^a\right)
\end{dmath*}

\begin{Shaded}
\begin{Highlighting}[]
\NormalTok{Explicit}\OperatorTok{[}\NormalTok{SUNF}\OperatorTok{[}\FunctionTok{a}\OperatorTok{,} \FunctionTok{b}\OperatorTok{,} \FunctionTok{c}\OperatorTok{]]}
\end{Highlighting}
\end{Shaded}

\begin{dmath*}\breakingcomma
f^{abc}
\end{dmath*}

\begin{Shaded}
\begin{Highlighting}[]
\NormalTok{Explicit}\OperatorTok{[}\NormalTok{SUNF}\OperatorTok{[}\FunctionTok{a}\OperatorTok{,} \FunctionTok{b}\OperatorTok{,} \FunctionTok{c}\OperatorTok{],}\NormalTok{ SUNF }\OtherTok{{-}\textgreater{}} \ConstantTok{True}\OperatorTok{]}
\end{Highlighting}
\end{Shaded}

\begin{dmath*}\breakingcomma
2 i \left(\text{tr}(T^a.T^c.T^b)-\text{tr}(T^a.T^b.T^c)\right)
\end{dmath*}

\begin{Shaded}
\begin{Highlighting}[]
\NormalTok{Explicit}\OperatorTok{[}\NormalTok{SUND}\OperatorTok{[}\FunctionTok{a}\OperatorTok{,} \FunctionTok{b}\OperatorTok{,} \FunctionTok{c}\OperatorTok{]]}
\end{Highlighting}
\end{Shaded}

\begin{dmath*}\breakingcomma
d^{abc}
\end{dmath*}

\begin{Shaded}
\begin{Highlighting}[]
\NormalTok{Explicit}\OperatorTok{[}\NormalTok{SUND}\OperatorTok{[}\FunctionTok{a}\OperatorTok{,} \FunctionTok{b}\OperatorTok{,} \FunctionTok{c}\OperatorTok{],}\NormalTok{ SUND }\OtherTok{{-}\textgreater{}} \ConstantTok{True}\OperatorTok{]}
\end{Highlighting}
\end{Shaded}

\begin{dmath*}\breakingcomma
2 \left(\text{tr}(T^a.T^b.T^c)\right)+2 \left(\text{tr}(T^b.T^a.T^c)\right)
\end{dmath*}

\begin{Shaded}
\begin{Highlighting}[]
\NormalTok{Explicit}\OperatorTok{[}\NormalTok{SUNTrace}\OperatorTok{[}\NormalTok{SUNT}\OperatorTok{[}\FunctionTok{a}\OperatorTok{,} \FunctionTok{b}\OperatorTok{,} \FunctionTok{c}\OperatorTok{]]]}
\end{Highlighting}
\end{Shaded}

\begin{dmath*}\breakingcomma
\text{tr}(T^a.T^b.T^c)
\end{dmath*}

\begin{Shaded}
\begin{Highlighting}[]
\NormalTok{Explicit}\OperatorTok{[}\NormalTok{SUNTrace}\OperatorTok{[}\NormalTok{SUNT}\OperatorTok{[}\FunctionTok{a}\OperatorTok{,} \FunctionTok{b}\OperatorTok{,} \FunctionTok{c}\OperatorTok{]],}\NormalTok{ SUNTrace }\OtherTok{{-}\textgreater{}} \ConstantTok{True}\OperatorTok{]}
\end{Highlighting}
\end{Shaded}

\begin{dmath*}\breakingcomma
\frac{d^{abc}}{4}+\frac{1}{4} i f^{abc}
\end{dmath*}
\end{document}
