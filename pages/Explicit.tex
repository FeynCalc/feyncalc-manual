% !TeX program = pdflatex
% !TeX root = Explicit.tex

\documentclass[../FeynCalcManual.tex]{subfiles}
\begin{document}
\hypertarget{explicit}{
\section{Explicit}\label{explicit}\index{Explicit}}

\texttt{Explicit} is an option for \texttt{FieldStrength},
\texttt{GluonVertex}, \texttt{SUNF}, and \texttt{Twist2GluonOperator}.
If set to \texttt{True} the full form of the operator is inserted.

\texttt{Explicit[\allowbreak{}exp]} inserts explicit expressions of
\texttt{GluonVertex}, \texttt{Twist2GluonOperator} etc. in \texttt{exp}.
\texttt{SUNF}s are replaced by \texttt{SUNTrace} objects.

\subsection{See also}

\hyperlink{toc}{Overview}, \hyperlink{gluonvertex}{GluonVertex},
\hyperlink{twist2gluonoperator}{Twist2GluonOperator}.

\subsection{Examples}

\begin{Shaded}
\begin{Highlighting}[]
\NormalTok{GluonVertex}\OperatorTok{[}\FunctionTok{p}\OperatorTok{,} \SpecialCharTok{\textbackslash{}}\OperatorTok{[}\NormalTok{Mu}\OperatorTok{],} \FunctionTok{a}\OperatorTok{,} \FunctionTok{q}\OperatorTok{,} \SpecialCharTok{\textbackslash{}}\OperatorTok{[}\NormalTok{Nu}\OperatorTok{],} \FunctionTok{b}\OperatorTok{,} \FunctionTok{r}\OperatorTok{,} \SpecialCharTok{\textbackslash{}}\OperatorTok{[}\NormalTok{Rho}\OperatorTok{],} \FunctionTok{c}\OperatorTok{]} 
 
\NormalTok{Explicit}\OperatorTok{[}\SpecialCharTok{\%}\OperatorTok{]}
\end{Highlighting}
\end{Shaded}

\begin{dmath*}\breakingcomma
f^{abc} V^{\mu \nu \rho }(p\text{, }q\text{, }r)
\end{dmath*}

\begin{dmath*}\breakingcomma
g_s f^{abc} \left(g^{\mu \nu } \left(p^{\rho }-q^{\rho }\right)+g^{\mu \rho } \left(r^{\nu }-p^{\nu }\right)+g^{\nu \rho } \left(q^{\mu }-r^{\mu }\right)\right)
\end{dmath*}

\begin{Shaded}
\begin{Highlighting}[]
\NormalTok{Twist2GluonOperator}\OperatorTok{[}\FunctionTok{p}\OperatorTok{,} \SpecialCharTok{\textbackslash{}}\OperatorTok{[}\NormalTok{Mu}\OperatorTok{],} \FunctionTok{a}\OperatorTok{,} \SpecialCharTok{\textbackslash{}}\OperatorTok{[}\NormalTok{Nu}\OperatorTok{],} \FunctionTok{b}\OperatorTok{]} 
 
\NormalTok{Explicit}\OperatorTok{[}\SpecialCharTok{\%}\OperatorTok{]}
\end{Highlighting}
\end{Shaded}

\begin{dmath*}\breakingcomma
\frac{1}{2} \left((-1)^m+1\right) \delta ^{ab} \left(O_{\mu \, \nu }^{\text{G2}}(p)\right)
\end{dmath*}

\begin{dmath*}\breakingcomma
\frac{1}{2} \left((-1)^m+1\right) \delta ^{ab} (\Delta \cdot p)^{m-2} \left(g^{\mu \nu } (\Delta \cdot p)^2+p^2 \Delta ^{\mu } \Delta ^{\nu }-(\Delta \cdot p) \left(\Delta ^{\nu } p^{\mu }+\Delta ^{\mu } p^{\nu }\right)\right)
\end{dmath*}

\begin{Shaded}
\begin{Highlighting}[]
\NormalTok{FieldStrength}\OperatorTok{[}\SpecialCharTok{\textbackslash{}}\OperatorTok{[}\NormalTok{Mu}\OperatorTok{],} \SpecialCharTok{\textbackslash{}}\OperatorTok{[}\NormalTok{Nu}\OperatorTok{],} \FunctionTok{a}\OperatorTok{]} 
 
\NormalTok{Explicit}\OperatorTok{[}\SpecialCharTok{\%}\OperatorTok{]} 
  
 
\end{Highlighting}
\end{Shaded}

\begin{dmath*}\breakingcomma
F_{\mu \nu }^a
\end{dmath*}

\begin{dmath*}\breakingcomma
g_s f^{a\text{b19}\;\text{c20}} A_{\mu }^{\text{b19}}.A_{\nu }^{\text{c20}}+\left.(\partial _{\mu }A_{\nu }^a\right)-\left.(\partial _{\nu }A_{\mu }^a\right)
\end{dmath*}
\end{document}
