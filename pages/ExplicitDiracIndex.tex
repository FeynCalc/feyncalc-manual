% !TeX program = pdflatex
% !TeX root = ExplicitDiracIndex.tex

\documentclass[../FeynCalcManual.tex]{subfiles}
\begin{document}
\hypertarget{explicitdiracindex}{
\section{ExplicitDiracIndex}\label{explicitdiracindex}\index{ExplicitDiracIndex}}

\texttt{ExplicitDiracIndex[\allowbreak{}ind]} is an explicit Dirac
index, i.e., \texttt{ind} is an integer.

\subsection{See also}

\hyperlink{toc}{Overview}, \hyperlink{diracchain}{DiracChain},
\hyperlink{dchn}{DCHN}, \hyperlink{diracindex}{DiracIndex},
\hyperlink{diracindexdelta}{DiracIndexDelta},
\hyperlink{didelta}{DIDelta},
\hyperlink{diracchainjoin}{DiracChainJoin},
\hyperlink{diracchaincombine}{DiracChainCombine},
\hyperlink{diracchainexpand}{DiracChainExpand},
\hyperlink{diracchainfactor}{DiracChainFactor}.

\subsection{Examples}

\begin{Shaded}
\begin{Highlighting}[]
\NormalTok{DCHN}\OperatorTok{[}\NormalTok{GA}\OperatorTok{[}\SpecialCharTok{\textbackslash{}}\OperatorTok{[}\NormalTok{Mu}\OperatorTok{]],} \DecValTok{1}\OperatorTok{,} \DecValTok{2}\OperatorTok{]}
\end{Highlighting}
\end{Shaded}

\begin{dmath*}\breakingcomma
\left(\bar{\gamma }^{\mu }\right){}_{12}
\end{dmath*}

\begin{Shaded}
\begin{Highlighting}[]
\NormalTok{DCHN}\OperatorTok{[}\NormalTok{GA}\OperatorTok{[}\SpecialCharTok{\textbackslash{}}\OperatorTok{[}\NormalTok{Mu}\OperatorTok{]],} \DecValTok{1}\OperatorTok{,} \DecValTok{2}\OperatorTok{]} \SpecialCharTok{//}\NormalTok{ FCI }\SpecialCharTok{//} \FunctionTok{StandardForm}

\CommentTok{(*DiracChain[DiracGamma[LorentzIndex[\textbackslash{}[Mu]]], ExplicitDiracIndex[1], ExplicitDiracIndex[2]]*)}
\end{Highlighting}
\end{Shaded}

\end{document}
