% !TeX program = pdflatex
% !TeX root = ExplicitLorentzIndex.tex

\documentclass[../FeynCalcManual.tex]{subfiles}
\begin{document}
\hypertarget{explicitlorentzindex}{
\section{ExplicitLorentzIndex}\label{explicitlorentzindex}\index{ExplicitLorentzIndex}}

\texttt{ExplicitLorentzIndex[\allowbreak{}ind]} is an explicit Lorentz
index, i.e., \texttt{ind} is an integer.

\subsection{See also}

\hyperlink{toc}{Overview}, \hyperlink{lorentzindex}{LorentzIndex},
\hyperlink{pair}{Pair}.

\subsection{Examples}

\begin{Shaded}
\begin{Highlighting}[]
\NormalTok{Pair}\OperatorTok{[}\NormalTok{LorentzIndex}\OperatorTok{[}\DecValTok{1}\OperatorTok{],}\NormalTok{ LorentzIndex}\OperatorTok{[}\SpecialCharTok{\textbackslash{}}\OperatorTok{[}\NormalTok{Mu}\OperatorTok{]]]}
\end{Highlighting}
\end{Shaded}

\begin{dmath*}\breakingcomma
\bar{g}^{1\mu }
\end{dmath*}

\begin{Shaded}
\begin{Highlighting}[]
\NormalTok{Pair}\OperatorTok{[}\NormalTok{LorentzIndex}\OperatorTok{[}\DecValTok{1}\OperatorTok{],}\NormalTok{ LorentzIndex}\OperatorTok{[}\SpecialCharTok{\textbackslash{}}\OperatorTok{[}\NormalTok{Mu}\OperatorTok{]]]} \SpecialCharTok{//} \FunctionTok{StandardForm}

\CommentTok{(*Pair[ExplicitLorentzIndex[1], LorentzIndex[\textbackslash{}[Mu]]]*)}
\end{Highlighting}
\end{Shaded}

\end{document}
