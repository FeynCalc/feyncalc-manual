% !TeX program = pdflatex
% !TeX root = ExplicitPartialD.tex

\documentclass[../FeynCalcManual.tex]{subfiles}
\begin{document}
\hypertarget{explicitpartiald}{
\section{ExplicitPartialD}\label{explicitpartiald}\index{ExplicitPartialD}}

\texttt{ExplicitPartialD[\allowbreak{}exp]} inserts the definitions for
\texttt{LeftRightPartialD}, \texttt{LeftRightPartialD2},
\texttt{LeftRightNablaD}, \texttt{LeftRightNablaD2}, \texttt{LeftNablaD}
and \texttt{RightNablaD}

\subsection{See also}

\hyperlink{toc}{Overview}, \hyperlink{expandpartiald}{ExpandPartialD},
\hyperlink{leftrightpartiald}{LeftRightPartialD},
\hyperlink{leftrightpartiald2}{LeftRightPartialD2},
\hyperlink{leftrightnablald}{LeftRightNablalD},
\hyperlink{leftrightnablald2}{LeftRightNablalD2},
\hyperlink{leftnablald}{LeftNablalD},
\hyperlink{rightnablald}{RightNablalD}.

\subsection{Examples}

\begin{Shaded}
\begin{Highlighting}[]
\NormalTok{LeftRightPartialD}\OperatorTok{[}\SpecialCharTok{\textbackslash{}}\OperatorTok{[}\NormalTok{Mu}\OperatorTok{]]} 
 
\NormalTok{ExplicitPartialD}\OperatorTok{[}\SpecialCharTok{\%}\OperatorTok{]}
\end{Highlighting}
\end{Shaded}

\begin{dmath*}\breakingcomma
\overleftrightarrow{\partial }_{\mu }
\end{dmath*}

\begin{dmath*}\breakingcomma
\frac{1}{2} \left(\vec{\partial }_{\mu }-\overleftarrow{\partial }_{\mu }\right)
\end{dmath*}

\begin{Shaded}
\begin{Highlighting}[]
\NormalTok{LeftRightPartialD2}\OperatorTok{[}\SpecialCharTok{\textbackslash{}}\OperatorTok{[}\NormalTok{Mu}\OperatorTok{]]} 
 
\NormalTok{ExplicitPartialD}\OperatorTok{[}\SpecialCharTok{\%}\OperatorTok{]}
\end{Highlighting}
\end{Shaded}

\begin{dmath*}\breakingcomma
\overleftrightarrow{\partial }_{\mu }
\end{dmath*}

\begin{dmath*}\breakingcomma
\overleftarrow{\partial }_{\mu }+\vec{\partial }_{\mu }
\end{dmath*}

\begin{Shaded}
\begin{Highlighting}[]
\NormalTok{LeftRightPartialD}\OperatorTok{[}\NormalTok{OPEDelta}\OperatorTok{]} 
 
\NormalTok{ExplicitPartialD}\OperatorTok{[}\SpecialCharTok{\%}\OperatorTok{]}
\end{Highlighting}
\end{Shaded}

\begin{dmath*}\breakingcomma
\overleftrightarrow{\partial }_{\Delta }
\end{dmath*}

\begin{dmath*}\breakingcomma
\frac{1}{2} \left(\vec{\partial }_{\Delta }-\overleftarrow{\partial }_{\Delta }\right)
\end{dmath*}

\begin{Shaded}
\begin{Highlighting}[]
\DecValTok{16}\NormalTok{ LeftRightPartialD}\OperatorTok{[}\NormalTok{OPEDelta}\OperatorTok{]}\SpecialCharTok{\^{}}\DecValTok{4} 
 
\NormalTok{ExplicitPartialD}\OperatorTok{[}\SpecialCharTok{\%}\OperatorTok{]}
\end{Highlighting}
\end{Shaded}

\begin{dmath*}\breakingcomma
16 \overleftrightarrow{\partial }_{\Delta }^4
\end{dmath*}

\begin{dmath*}\breakingcomma
\left(\vec{\partial }_{\Delta }-\overleftarrow{\partial }_{\Delta }\right){}^4
\end{dmath*}

Notice that by definition \(\nabla^i = \partial_i = - \partial^i\),
where the last equality depends on the metric signature.

\begin{Shaded}
\begin{Highlighting}[]
\NormalTok{LeftNablaD}\OperatorTok{[}\FunctionTok{i}\OperatorTok{]} 
 
\NormalTok{ExplicitPartialD}\OperatorTok{[}\SpecialCharTok{\%}\OperatorTok{]}
\end{Highlighting}
\end{Shaded}

\begin{dmath*}\breakingcomma
\overleftarrow{\nabla }^i
\end{dmath*}

\begin{dmath*}\breakingcomma
-\overleftarrow{\partial }_i
\end{dmath*}

\begin{Shaded}
\begin{Highlighting}[]
\NormalTok{RightNablaD}\OperatorTok{[}\FunctionTok{i}\OperatorTok{]} 
 
\NormalTok{ExplicitPartialD}\OperatorTok{[}\SpecialCharTok{\%}\OperatorTok{]}
\end{Highlighting}
\end{Shaded}

\begin{dmath*}\breakingcomma
\vec{\nabla }^i
\end{dmath*}

\begin{dmath*}\breakingcomma
-\vec{\partial }_i
\end{dmath*}

\begin{Shaded}
\begin{Highlighting}[]
\NormalTok{LeftRightNablaD}\OperatorTok{[}\FunctionTok{i}\OperatorTok{]} 
 
\NormalTok{ExplicitPartialD}\OperatorTok{[}\SpecialCharTok{\%}\OperatorTok{]}
\end{Highlighting}
\end{Shaded}

\begin{dmath*}\breakingcomma
\overleftrightarrow{\nabla }_i
\end{dmath*}

\begin{dmath*}\breakingcomma
\frac{1}{2} \overleftarrow{\partial }_i-\vec{\partial }_i
\end{dmath*}

\begin{Shaded}
\begin{Highlighting}[]
\NormalTok{LeftRightNablaD2}\OperatorTok{[}\SpecialCharTok{\textbackslash{}}\OperatorTok{[}\NormalTok{Mu}\OperatorTok{]]} 
 
\NormalTok{ExplicitPartialD}\OperatorTok{[}\SpecialCharTok{\%}\OperatorTok{]} 
  
 
\end{Highlighting}
\end{Shaded}

\begin{dmath*}\breakingcomma
\overleftrightarrow{\nabla }_{\mu }
\end{dmath*}

\begin{dmath*}\breakingcomma
-\overleftarrow{\partial }_{\mu }-\vec{\partial }_{\mu }
\end{dmath*}
\end{document}
