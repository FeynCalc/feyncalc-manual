% !TeX program = pdflatex
% !TeX root = ExplicitPauliIndex.tex

\documentclass[../FeynCalcManual.tex]{subfiles}
\begin{document}
\hypertarget{explicitpauliindex}{
\section{ExplicitPauliIndex}\label{explicitpauliindex}\index{ExplicitPauliIndex}}

\texttt{ExplicitPauliIndex[\allowbreak{}ind]} is an explicit Pauli
index, i.e., \texttt{ind} is an integer.

\subsection{See also}

\hyperlink{toc}{Overview}, \hyperlink{paulichain}{PauliChain},
\hyperlink{pchn}{PCHN}, \hyperlink{pauliindex}{PauliIndex},
\hyperlink{pauliindexdelta}{PauliIndexDelta},
\hyperlink{pidelta}{PIDelta},
\hyperlink{paulichainjoin}{PauliChainJoin},
\hyperlink{paulichaincombine}{PauliChainCombine},
\hyperlink{paulichainexpand}{PauliChainExpand},
\hyperlink{paulichainfactor}{PauliChainFactor}.

\subsection{Examples}

\begin{Shaded}
\begin{Highlighting}[]
\NormalTok{PCHN}\OperatorTok{[}\NormalTok{SI}\OperatorTok{[}\FunctionTok{i}\OperatorTok{],} \DecValTok{1}\OperatorTok{,} \DecValTok{2}\OperatorTok{]}
\end{Highlighting}
\end{Shaded}

\begin{dmath*}\breakingcomma
\left(\bar{\sigma }^i\right){}_{12}
\end{dmath*}

\begin{Shaded}
\begin{Highlighting}[]
\NormalTok{PCHN}\OperatorTok{[}\NormalTok{SI}\OperatorTok{[}\FunctionTok{i}\OperatorTok{],} \DecValTok{1}\OperatorTok{,} \DecValTok{2}\OperatorTok{]} \SpecialCharTok{//}\NormalTok{ FCI }\SpecialCharTok{//} \FunctionTok{StandardForm}

\CommentTok{(*PauliChain[PauliSigma[LorentzIndex[i]], ExplicitPauliIndex[1], ExplicitPauliIndex[2]]*)}
\end{Highlighting}
\end{Shaded}

\end{document}
