% !TeX program = pdflatex
% !TeX root = ExplicitSUNFIndex.tex

\documentclass[../FeynCalcManual.tex]{subfiles}
\begin{document}
\hypertarget{explicitsunfindex}{
\section{ExplicitSUNFIndex}\label{explicitsunfindex}\index{ExplicitSUNFIndex}}

\texttt{ExplicitSUNFIndex[\allowbreak{}ind]} is a specific \(SU(N)\)
index in the fundamental representation, i.e.~\texttt{ind} is an
integer.

\subsection{See also}

\hyperlink{toc}{Overview}, \hyperlink{sunindex}{SUNIndex},
\hyperlink{sunfindex}{SUNFIndex}.

\subsection{Examples}

\begin{Shaded}
\begin{Highlighting}[]
\NormalTok{ExplicitSUNFIndex}\OperatorTok{[}\DecValTok{1}\OperatorTok{]}
\end{Highlighting}
\end{Shaded}

\begin{dmath*}\breakingcomma
1
\end{dmath*}

\begin{Shaded}
\begin{Highlighting}[]
\NormalTok{SUNTF}\OperatorTok{[}\FunctionTok{a}\OperatorTok{,} \DecValTok{1}\OperatorTok{,} \DecValTok{2}\OperatorTok{]}
\end{Highlighting}
\end{Shaded}

\begin{dmath*}\breakingcomma
T_{12}^a
\end{dmath*}

\begin{Shaded}
\begin{Highlighting}[]
\NormalTok{SUNTF}\OperatorTok{[}\FunctionTok{a}\OperatorTok{,} \DecValTok{1}\OperatorTok{,} \DecValTok{2}\OperatorTok{]} \SpecialCharTok{//}\NormalTok{ FCI }\SpecialCharTok{//} \FunctionTok{StandardForm}

\CommentTok{(*SUNTF[\{SUNIndex[a]\}, ExplicitSUNFIndex[1], ExplicitSUNFIndex[2]]*)}
\end{Highlighting}
\end{Shaded}

\end{document}
