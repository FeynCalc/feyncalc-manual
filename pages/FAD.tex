% !TeX program = pdflatex
% !TeX root = FAD.tex

\documentclass[../FeynCalcManual.tex]{subfiles}
\begin{document}
\hypertarget{fad}{
\section{FAD}\label{fad}\index{FAD}}

\texttt{FAD} is the FeynCalc external form of
\texttt{FeynAmpDenominator} and denotes an inverse propagator.

\texttt{FAD[\allowbreak{}q,\ \allowbreak{}q-p,\ \allowbreak{}...]} is
\(\frac{1}{q^2 (q-p)^2 \ldots}\).

\texttt{FAD[\allowbreak{}\{\allowbreak{}q1,\ \allowbreak{}m\},\ \allowbreak{}\{\allowbreak{}q1-p,\ \allowbreak{}m\},\ \allowbreak{}q2,\ \allowbreak{}...]}
is \(\frac{1}{[q1^2 - m^2][(q1-p)^2 - m^2] q2^2}\). Translation into
FeynCalc internal form is performed by \texttt{FeynCalcInternal}.

\subsection{See also}

\hyperlink{toc}{Overview}, \hyperlink{fad}{FAD}, \hyperlink{fce}{FCE},
\hyperlink{fci}{FCI},
\hyperlink{feynampdenominator}{FeynAmpDenominator},
\hyperlink{feynampdenominatorsimplify}{FeynAmpDenominatorSimplify},
\hyperlink{propagatordenominator}{PropagatorDenominator}.

\subsection{Examples}

\begin{Shaded}
\begin{Highlighting}[]
\NormalTok{FAD}\OperatorTok{[}\FunctionTok{q}\OperatorTok{,} \FunctionTok{p} \SpecialCharTok{{-}} \FunctionTok{q}\OperatorTok{]}
\end{Highlighting}
\end{Shaded}

\begin{dmath*}\breakingcomma
\frac{1}{q^2.(p-q)^2}
\end{dmath*}

\begin{Shaded}
\begin{Highlighting}[]
\NormalTok{FAD}\OperatorTok{[}\FunctionTok{p}\OperatorTok{,} \OperatorTok{\{}\FunctionTok{p} \SpecialCharTok{{-}} \FunctionTok{q}\OperatorTok{,} \FunctionTok{m}\OperatorTok{\}]}
\end{Highlighting}
\end{Shaded}

\begin{dmath*}\breakingcomma
\frac{1}{p^2.\left((p-q)^2-m^2\right)}
\end{dmath*}

\begin{Shaded}
\begin{Highlighting}[]
\NormalTok{FAD}\OperatorTok{[\{}\FunctionTok{p}\OperatorTok{,} \DecValTok{0}\OperatorTok{,} \DecValTok{2}\OperatorTok{\},} \OperatorTok{\{}\FunctionTok{p} \SpecialCharTok{{-}} \FunctionTok{q}\OperatorTok{,} \FunctionTok{m}\OperatorTok{,} \DecValTok{3}\OperatorTok{\}]}
\end{Highlighting}
\end{Shaded}

\begin{dmath*}\breakingcomma
\frac{1}{\left(p^2\right)^2.\left((p-q)^2-m^2\right)^3}
\end{dmath*}

\begin{Shaded}
\begin{Highlighting}[]
\NormalTok{FAD}\OperatorTok{[}\FunctionTok{q}\OperatorTok{,} \FunctionTok{p} \SpecialCharTok{{-}} \FunctionTok{q}\OperatorTok{]} \SpecialCharTok{//}\NormalTok{ FCI }\SpecialCharTok{//} \FunctionTok{StandardForm}

\CommentTok{(*FeynAmpDenominator[PropagatorDenominator[Momentum[q, D], 0], PropagatorDenominator[Momentum[p, D] {-} Momentum[q, D], 0]]*)}
\end{Highlighting}
\end{Shaded}

\begin{Shaded}
\begin{Highlighting}[]
\NormalTok{FAD}\OperatorTok{[}\FunctionTok{p}\OperatorTok{]}\NormalTok{ FAD}\OperatorTok{[}\FunctionTok{p} \SpecialCharTok{{-}} \FunctionTok{q}\OperatorTok{]} \SpecialCharTok{//}\NormalTok{ FeynAmpDenominatorCombine}\OperatorTok{[}\NormalTok{\#}\OperatorTok{,}\NormalTok{ FCE }\OtherTok{{-}\textgreater{}} \ConstantTok{True}\OperatorTok{]}\NormalTok{ \& }\SpecialCharTok{//} \FunctionTok{StandardForm}

\CommentTok{(*FAD[p, p {-} q]*)}
\end{Highlighting}
\end{Shaded}

\end{document}
