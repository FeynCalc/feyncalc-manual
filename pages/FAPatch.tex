% !TeX program = pdflatex
% !TeX root = FAPatch.tex

\documentclass[../FeynCalcManual.tex]{subfiles}
\begin{document}
\hypertarget{fapatch}{
\section{FAPatch}\label{fapatch}\index{FAPatch}}

FAPatch{[}{]} is an auxiliary function that patches FeynArts to be
compatible with FeynCalc. If an unpatched copy of FeynArts is present in
\$FeynArtsDirectory, evaluating FAPatch{[}{]} will start the patching
process.

\subsection{See also}

\hyperlink{toc}{Overview}, \hyperlink{patchmodelsonly}{PatchModelsOnly},
\hyperlink{famodelsdirectory}{FAModelsDirectory}.

\subsection{Examples}

Setting the option \texttt{Quiet} to \texttt{True} will suppress the
\texttt{ChoiceDialog} asking whether you really want to patch FeynArts.

\begin{Shaded}
\begin{Highlighting}[]
\CommentTok{(*FAPatch[Quiet{-}\textgreater{}True]*)}
\end{Highlighting}
\end{Shaded}

If you just want to patch some new models (e.g.~generated with
FeynRules), while your FeynArts version is already patched, use the
option \texttt{PatchModelsOnly}.

\begin{Shaded}
\begin{Highlighting}[]
\CommentTok{(*FAPatch[PatchModelsOnly{-}\textgreater{}True]*)}
\end{Highlighting}
\end{Shaded}

The model files do not necessarily have to be located inside
\texttt{FileNameJoin[\allowbreak{}\{\allowbreak{}\$FeynArtsDirectory,\ \allowbreak{}"Models"\}]}.
A custom location can be specified via the option
\texttt{FAModelsDirectory} as in

\begin{Shaded}
\begin{Highlighting}[]
\CommentTok{(*FAPatch[PatchModelsOnly{-}\textgreater{}True,FAModelsDirectory{-}\textgreater{}}
\CommentTok{FileNameJoin[\{ParentDirectory@NotebookDirectory[],"FeynArts","MyModel"\}]]*)}
\end{Highlighting}
\end{Shaded}

\end{document}
