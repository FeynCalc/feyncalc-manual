% !TeX program = pdflatex
% !TeX root = FC.tex

\documentclass[../FeynCalcManual.tex]{subfiles}
\begin{document}
\hypertarget{fc}{
\section{FC}\label{fc}\index{FC}}

\texttt{FC} changes the output format to \texttt{FeynCalcForm}. To
change to \texttt{InputForm} use \texttt{FI}.

\subsection{See also}

\hyperlink{toc}{Overview}, \hyperlink{feyncalcform}{FeynCalcForm},
\hyperlink{fi}{FI}, \hyperlink{feyncalcexternal}{FeynCalcExternal},
\hyperlink{feyncalcinternal}{FeynCalcInternal}.

\subsection{Examples}

\begin{Shaded}
\begin{Highlighting}[]
\NormalTok{FI }
 
\OperatorTok{\{}\NormalTok{DiracGamma}\OperatorTok{[}\DecValTok{5}\OperatorTok{],}\NormalTok{ DiracGamma}\OperatorTok{[}\NormalTok{Momentum}\OperatorTok{[}\FunctionTok{p}\OperatorTok{]]\}}
\end{Highlighting}
\end{Shaded}

\begin{Shaded}
\begin{Highlighting}[]
\OperatorTok{\{}\NormalTok{DiracGamma}\OperatorTok{[}\DecValTok{5}\OperatorTok{],}\NormalTok{ DiracGamma}\OperatorTok{[}\NormalTok{Momentum}\OperatorTok{[}\FunctionTok{p}\OperatorTok{]]\}}
\end{Highlighting}
\end{Shaded}

\begin{Shaded}
\begin{Highlighting}[]
\NormalTok{FC }
 
\OperatorTok{\{}\NormalTok{DiracGamma}\OperatorTok{[}\DecValTok{5}\OperatorTok{],}\NormalTok{ DiracGamma}\OperatorTok{[}\NormalTok{Momentum}\OperatorTok{[}\FunctionTok{p}\OperatorTok{]]\}}
\end{Highlighting}
\end{Shaded}

\begin{dmath*}\breakingcomma
\left\{\bar{\gamma }^5,\bar{\gamma }\cdot \overline{p}\right\}
\end{dmath*}
\end{document}
