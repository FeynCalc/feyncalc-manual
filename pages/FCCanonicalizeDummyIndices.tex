% !TeX program = pdflatex
% !TeX root = FCCanonicalizeDummyIndices.tex

\documentclass[../FeynCalcManual.tex]{subfiles}
\begin{document}
\hypertarget{fccanonicalizedummyindices}{
\section{FCCanonicalizeDummyIndices}\label{fccanonicalizedummyindices}\index{FCCanonicalizeDummyIndices}}

\texttt{FCCanonicalizeDummyIndices[\allowbreak{}expr]} canonicalizes
dummy indices in the expression.

Following index types are supported: \texttt{LorentzIndex},
\texttt{CartesianIndex}, \texttt{SUNIndex}, \texttt{SUNFIndex},
\texttt{DiracIndex}, \texttt{PauliIndex}

In the case of Lorentz indices the option \texttt{Momentum} provides a
possibility to limit the canonicalization only to particular
\texttt{Momenta}. The option \texttt{LorentzIndexNames} can be used to
assign specific names to the canonicalized indices, to have say \(\mu\),
\(\nu\), \(\rho\) etc. instead of some random names.

For other index types the corresponding options are called
\texttt{CartesianIndexNames}, \texttt{SUNIndexNames},
\texttt{SUNFIndexNames}, \texttt{DiracIndexNames} and
\texttt{PauliIndexNames}.

\subsection{See also}

\hyperlink{toc}{Overview},
\hyperlink{fcrenamedummyindices}{FCRenameDummyIndices}.

\subsection{Examples}

Canonicalization of Lorentz indices

\begin{Shaded}
\begin{Highlighting}[]
\NormalTok{FVD}\OperatorTok{[}\FunctionTok{q}\OperatorTok{,}\NormalTok{ mu}\OperatorTok{]}\NormalTok{ FVD}\OperatorTok{[}\FunctionTok{p}\OperatorTok{,}\NormalTok{ mu}\OperatorTok{]} \SpecialCharTok{+}\NormalTok{ FVD}\OperatorTok{[}\FunctionTok{q}\OperatorTok{,}\NormalTok{ nu}\OperatorTok{]}\NormalTok{ FVD}\OperatorTok{[}\FunctionTok{p}\OperatorTok{,}\NormalTok{ nu}\OperatorTok{]} \SpecialCharTok{+}\NormalTok{ FVD}\OperatorTok{[}\FunctionTok{q}\OperatorTok{,}\NormalTok{ si}\OperatorTok{]}\NormalTok{ FVD}\OperatorTok{[}\FunctionTok{r}\OperatorTok{,}\NormalTok{ si}\OperatorTok{]} 
 
\NormalTok{FCCanonicalizeDummyIndices}\OperatorTok{[}\SpecialCharTok{\%}\OperatorTok{]} \SpecialCharTok{//}\NormalTok{ Factor2}
\end{Highlighting}
\end{Shaded}

\begin{dmath*}\breakingcomma
p^{\text{mu}} q^{\text{mu}}+p^{\text{nu}} q^{\text{nu}}+q^{\text{si}} r^{\text{si}}
\end{dmath*}

\begin{dmath*}\breakingcomma
q^{\text{FCGV}(\text{li191})} \left(2 p^{\text{FCGV}(\text{li191})}+r^{\text{FCGV}(\text{li191})}\right)
\end{dmath*}

\begin{Shaded}
\begin{Highlighting}[]
\NormalTok{Uncontract}\OperatorTok{[}\NormalTok{SPD}\OperatorTok{[}\FunctionTok{q}\OperatorTok{,} \FunctionTok{p}\OperatorTok{]}\SpecialCharTok{\^{}}\DecValTok{2}\OperatorTok{,} \FunctionTok{q}\OperatorTok{,} \FunctionTok{p}\OperatorTok{,}\NormalTok{ Pair }\OtherTok{{-}\textgreater{}} \ConstantTok{All}\OperatorTok{]} 
 
\NormalTok{FCCanonicalizeDummyIndices}\OperatorTok{[}\SpecialCharTok{\%}\OperatorTok{,}\NormalTok{ LorentzIndexNames }\OtherTok{{-}\textgreater{}} \OperatorTok{\{}\SpecialCharTok{\textbackslash{}}\OperatorTok{[}\NormalTok{Mu}\OperatorTok{],} \SpecialCharTok{\textbackslash{}}\OperatorTok{[}\NormalTok{Nu}\OperatorTok{]\}]}
\end{Highlighting}
\end{Shaded}

\begin{dmath*}\breakingcomma
p^{\text{\$AL}(\text{\$28})} p^{\text{\$AL}(\text{\$29})} q^{\text{\$AL}(\text{\$28})} q^{\text{\$AL}(\text{\$29})}
\end{dmath*}

\begin{dmath*}\breakingcomma
p^{\mu } p^{\nu } q^{\mu } q^{\nu }
\end{dmath*}

Canonicalization of Cartesian indices

\begin{Shaded}
\begin{Highlighting}[]
\NormalTok{CVD}\OperatorTok{[}\FunctionTok{p}\OperatorTok{,} \FunctionTok{i}\OperatorTok{]}\NormalTok{ CVD}\OperatorTok{[}\FunctionTok{q}\OperatorTok{,} \FunctionTok{i}\OperatorTok{]} \SpecialCharTok{+}\NormalTok{ CVD}\OperatorTok{[}\FunctionTok{p}\OperatorTok{,} \FunctionTok{j}\OperatorTok{]}\NormalTok{ CVD}\OperatorTok{[}\FunctionTok{r}\OperatorTok{,} \FunctionTok{j}\OperatorTok{]} 
 
\NormalTok{FCCanonicalizeDummyIndices}\OperatorTok{[}\SpecialCharTok{\%}\OperatorTok{]} \SpecialCharTok{//}\NormalTok{ Factor2}
\end{Highlighting}
\end{Shaded}

\begin{dmath*}\breakingcomma
p^i q^i+p^j r^j
\end{dmath*}

\begin{dmath*}\breakingcomma
p^{\text{FCGV}(\text{ci391})} \left(q^{\text{FCGV}(\text{ci391})}+r^{\text{FCGV}(\text{ci391})}\right)
\end{dmath*}

\begin{Shaded}
\begin{Highlighting}[]
\NormalTok{CVD}\OperatorTok{[}\FunctionTok{p}\OperatorTok{,} \FunctionTok{i}\OperatorTok{]}\NormalTok{ CVD}\OperatorTok{[}\FunctionTok{q}\OperatorTok{,} \FunctionTok{i}\OperatorTok{]} \SpecialCharTok{+}\NormalTok{ CVD}\OperatorTok{[}\FunctionTok{p}\OperatorTok{,} \FunctionTok{j}\OperatorTok{]}\NormalTok{ CVD}\OperatorTok{[}\FunctionTok{r}\OperatorTok{,} \FunctionTok{j}\OperatorTok{]} 
 
\NormalTok{FCCanonicalizeDummyIndices}\OperatorTok{[}\SpecialCharTok{\%}\OperatorTok{,}\NormalTok{ CartesianIndexNames }\OtherTok{{-}\textgreater{}} \OperatorTok{\{}\FunctionTok{a}\OperatorTok{\}]} \SpecialCharTok{//}\NormalTok{ Factor2}
\end{Highlighting}
\end{Shaded}

\begin{dmath*}\breakingcomma
p^i q^i+p^j r^j
\end{dmath*}

\begin{dmath*}\breakingcomma
p^a \left(q^a+r^a\right)
\end{dmath*}

Canonicalization of color indices

\begin{Shaded}
\begin{Highlighting}[]
\NormalTok{SUNT}\OperatorTok{[}\FunctionTok{a}\OperatorTok{,} \FunctionTok{b}\OperatorTok{,} \FunctionTok{a}\OperatorTok{]} \SpecialCharTok{+}\NormalTok{ SUNT}\OperatorTok{[}\FunctionTok{c}\OperatorTok{,} \FunctionTok{b}\OperatorTok{,} \FunctionTok{c}\OperatorTok{]} 
 
\NormalTok{FCCanonicalizeDummyIndices}\OperatorTok{[}\SpecialCharTok{\%}\OperatorTok{]}
\end{Highlighting}
\end{Shaded}

\begin{dmath*}\breakingcomma
T^a.T^b.T^a+T^c.T^b.T^c
\end{dmath*}

\begin{dmath*}\breakingcomma
2 T^{\text{FCGV}(\text{sun601})}.T^b.T^{\text{FCGV}(\text{sun601})}
\end{dmath*}

\begin{Shaded}
\begin{Highlighting}[]
\NormalTok{SUNT}\OperatorTok{[}\FunctionTok{a}\OperatorTok{,} \FunctionTok{b}\OperatorTok{,} \FunctionTok{a}\OperatorTok{]} \SpecialCharTok{+}\NormalTok{ SUNT}\OperatorTok{[}\FunctionTok{c}\OperatorTok{,} \FunctionTok{b}\OperatorTok{,} \FunctionTok{c}\OperatorTok{]} 
 
\NormalTok{FCCanonicalizeDummyIndices}\OperatorTok{[}\SpecialCharTok{\%}\OperatorTok{,}\NormalTok{ SUNIndexNames }\OtherTok{{-}\textgreater{}} \OperatorTok{\{}\FunctionTok{u}\OperatorTok{\}]}
\end{Highlighting}
\end{Shaded}

\begin{dmath*}\breakingcomma
T^a.T^b.T^a+T^c.T^b.T^c
\end{dmath*}

\begin{dmath*}\breakingcomma
2 T^u.T^b.T^u
\end{dmath*}

Canonicalization of Dirac indices

\begin{Shaded}
\begin{Highlighting}[]
\NormalTok{DCHN}\OperatorTok{[}\NormalTok{GA}\OperatorTok{[}\NormalTok{mu}\OperatorTok{],} \FunctionTok{i}\OperatorTok{,} \FunctionTok{j}\OperatorTok{]}\NormalTok{ DCHN}\OperatorTok{[}\NormalTok{GA}\OperatorTok{[}\NormalTok{nu}\OperatorTok{],} \FunctionTok{j}\OperatorTok{,} \FunctionTok{k}\OperatorTok{]} 
 
\NormalTok{FCCanonicalizeDummyIndices}\OperatorTok{[}\SpecialCharTok{\%}\OperatorTok{]}
\end{Highlighting}
\end{Shaded}

\begin{dmath*}\breakingcomma
\left(\bar{\gamma }^{\text{mu}}\right){}_{ij} \left(\bar{\gamma }^{\text{nu}}\right){}_{jk}
\end{dmath*}

\begin{dmath*}\breakingcomma
\left(\bar{\gamma }^{\text{mu}}\right){}_{i\text{FCGV}(\text{di771})} \left(\bar{\gamma }^{\text{nu}}\right){}_{\text{FCGV}(\text{di771})k}
\end{dmath*}

\begin{Shaded}
\begin{Highlighting}[]
\NormalTok{DCHN}\OperatorTok{[}\NormalTok{GA}\OperatorTok{[}\NormalTok{mu}\OperatorTok{],} \FunctionTok{i}\OperatorTok{,} \FunctionTok{j}\OperatorTok{]}\NormalTok{ DCHN}\OperatorTok{[}\NormalTok{GA}\OperatorTok{[}\NormalTok{nu}\OperatorTok{],} \FunctionTok{j}\OperatorTok{,} \FunctionTok{k}\OperatorTok{]} 
 
\NormalTok{FCCanonicalizeDummyIndices}\OperatorTok{[}\SpecialCharTok{\%}\OperatorTok{,}\NormalTok{ DiracIndexNames }\OtherTok{{-}\textgreater{}} \OperatorTok{\{}\FunctionTok{a}\OperatorTok{\}]}
\end{Highlighting}
\end{Shaded}

\begin{dmath*}\breakingcomma
\left(\bar{\gamma }^{\text{mu}}\right){}_{ij} \left(\bar{\gamma }^{\text{nu}}\right){}_{jk}
\end{dmath*}

\begin{dmath*}\breakingcomma
\left(\bar{\gamma }^{\text{mu}}\right){}_{ia} \left(\bar{\gamma }^{\text{nu}}\right){}_{ak}
\end{dmath*}

Canonicalization of Pauli indices

\begin{Shaded}
\begin{Highlighting}[]
\NormalTok{PCHN}\OperatorTok{[}\NormalTok{CSI}\OperatorTok{[}\FunctionTok{a}\OperatorTok{],} \FunctionTok{i}\OperatorTok{,} \FunctionTok{j}\OperatorTok{]}\NormalTok{ PCHN}\OperatorTok{[}\NormalTok{CSI}\OperatorTok{[}\FunctionTok{b}\OperatorTok{],} \FunctionTok{j}\OperatorTok{,} \FunctionTok{k}\OperatorTok{]} 
 
\NormalTok{FCCanonicalizeDummyIndices}\OperatorTok{[}\SpecialCharTok{\%}\OperatorTok{]}
\end{Highlighting}
\end{Shaded}

\begin{dmath*}\breakingcomma
\left(\overline{\sigma }^a\right){}_{ij} \left(\overline{\sigma }^b\right){}_{jk}
\end{dmath*}

\begin{dmath*}\breakingcomma
\left(\overline{\sigma }^a\right){}_{i\text{FCGV}(\text{pi921})} \left(\overline{\sigma }^b\right){}_{\text{FCGV}(\text{pi921})k}
\end{dmath*}

\begin{Shaded}
\begin{Highlighting}[]
\NormalTok{PCHN}\OperatorTok{[}\NormalTok{CSI}\OperatorTok{[}\FunctionTok{a}\OperatorTok{],} \FunctionTok{i}\OperatorTok{,} \FunctionTok{j}\OperatorTok{]}\NormalTok{ PCHN}\OperatorTok{[}\NormalTok{CSI}\OperatorTok{[}\FunctionTok{b}\OperatorTok{],} \FunctionTok{j}\OperatorTok{,} \FunctionTok{k}\OperatorTok{]} 
 
\NormalTok{FCCanonicalizeDummyIndices}\OperatorTok{[}\SpecialCharTok{\%}\OperatorTok{,}\NormalTok{ PauliIndexNames }\OtherTok{{-}\textgreater{}} \OperatorTok{\{}\FunctionTok{l}\OperatorTok{\}]}
\end{Highlighting}
\end{Shaded}

\begin{dmath*}\breakingcomma
\left(\overline{\sigma }^a\right){}_{ij} \left(\overline{\sigma }^b\right){}_{jk}
\end{dmath*}

\begin{dmath*}\breakingcomma
\left(\overline{\sigma }^a\right){}_{il} \left(\overline{\sigma }^b\right){}_{lk}
\end{dmath*}

Using the option \texttt{Head} one can specify which index heads should
be canonicalized, while the rest will be ignored.

\begin{Shaded}
\begin{Highlighting}[]
\NormalTok{(QuantumField}\OperatorTok{[}\FunctionTok{Superscript}\OperatorTok{[}\SpecialCharTok{\textbackslash{}}\OperatorTok{[}\NormalTok{Phi}\OperatorTok{],} \StringTok{"+"}\OperatorTok{],}\NormalTok{ PauliIndex}\OperatorTok{[}\NormalTok{k1}\OperatorTok{],}\NormalTok{ PauliIndex}\OperatorTok{[}\NormalTok{k2}\OperatorTok{],} 
     \FunctionTok{R}\OperatorTok{,} \FunctionTok{r}\OperatorTok{]}\NormalTok{ . QuantumField}\OperatorTok{[}\NormalTok{FCPartialD}\OperatorTok{[\{}\NormalTok{CartesianIndex}\OperatorTok{[}\FunctionTok{i}\OperatorTok{],} \FunctionTok{r}\OperatorTok{\}],} 
\NormalTok{     FCPartialD}\OperatorTok{[\{}\NormalTok{CartesianIndex}\OperatorTok{[}\FunctionTok{i}\OperatorTok{],} \FunctionTok{r}\OperatorTok{\}],} \SpecialCharTok{\textbackslash{}}\OperatorTok{[}\NormalTok{Phi}\OperatorTok{],}\NormalTok{ PauliIndex}\OperatorTok{[}\NormalTok{k2}\OperatorTok{],}\NormalTok{ PauliIndex}\OperatorTok{[}\NormalTok{k1}\OperatorTok{],} \FunctionTok{R}\OperatorTok{,} \FunctionTok{r}\OperatorTok{]}\NormalTok{) }
 
\NormalTok{FCCanonicalizeDummyIndices}\OperatorTok{[}\SpecialCharTok{\%}\OperatorTok{,}\NormalTok{ CartesianIndexNames }\OtherTok{{-}\textgreater{}} \OperatorTok{\{}\FunctionTok{j}\OperatorTok{\},} \FunctionTok{Head} \OtherTok{{-}\textgreater{}} \OperatorTok{\{}\NormalTok{CartesianIndex}\OperatorTok{\}]}
\end{Highlighting}
\end{Shaded}

\begin{dmath*}\breakingcomma
\phi ^{+\text{k1}\;\text{k2}Rr}.\left(\partial _{\{i,r\}}\partial _{\{i,r\}}\phi ^{\text{k2}\;\text{k1}Rr}\right)
\end{dmath*}

\begin{dmath*}\breakingcomma
\phi ^{+\text{k1}\;\text{k2}Rr}.\left(\partial _{\{j,r\}}\partial _{\{j,r\}}\phi ^{\text{k2}\;\text{k1}Rr}\right)
\end{dmath*}
\end{document}
