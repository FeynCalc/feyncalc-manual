% !TeX program = pdflatex
% !TeX root = FCChargeConjugateTransposed.tex

\documentclass[../FeynCalcManual.tex]{subfiles}
\begin{document}
\hypertarget{fcchargeconjugatetransposed}{
\section{FCChargeConjugateTransposed}\label{fcchargeconjugatetransposed}\index{FCChargeConjugateTransposed}}

\texttt{FCChargeConjugateTransposed[\allowbreak{}exp]} represents the
application of the charge conjugation operator to the transposed of
\texttt{exp}, i.e.~\(C^{-1} \;\text{exp}^T C\). Here \texttt{exp} is
understood to be a single Dirac matrix or a chain thereof. The option
setting \texttt{Explicit} determines whether the explicit result is
returned or whether it is left in the unevaluated form.The unevaluated
form will be also maintained if the function does not know how to obtain
\(C^{-1} \;\text{exp}^T C\) from the given exp.

The shortcut for \texttt{FCChargeConjugateTransposed} is \texttt{FCCCT}.

\subsection{See also}

\hyperlink{toc}{Overview},
\hyperlink{spinorchaintranspose}{SpinorChainTranspose},
\hyperlink{diracgamma}{DiracGamma}, \hyperlink{spinor}{Spinor}.

\subsection{Examples}

\begin{Shaded}
\begin{Highlighting}[]
\NormalTok{GA}\OperatorTok{[}\SpecialCharTok{\textbackslash{}}\OperatorTok{[}\NormalTok{Mu}\OperatorTok{],} \SpecialCharTok{\textbackslash{}}\OperatorTok{[}\NormalTok{Nu}\OperatorTok{],} \SpecialCharTok{\textbackslash{}}\OperatorTok{[}\NormalTok{Rho}\OperatorTok{]]} 
 
\NormalTok{FCChargeConjugateTransposed}\OperatorTok{[}\SpecialCharTok{\%}\OperatorTok{]}
\end{Highlighting}
\end{Shaded}

\begin{dmath*}\breakingcomma
\bar{\gamma }^{\mu }.\bar{\gamma }^{\nu }.\bar{\gamma }^{\rho }
\end{dmath*}

\begin{dmath*}\breakingcomma
C\left(\bar{\gamma }^{\mu }.\bar{\gamma }^{\nu }.\bar{\gamma }^{\rho }\right)^TC^{-1}
\end{dmath*}

\begin{Shaded}
\begin{Highlighting}[]
\NormalTok{FCChargeConjugateTransposed}\OperatorTok{[}\NormalTok{GA}\OperatorTok{[}\SpecialCharTok{\textbackslash{}}\OperatorTok{[}\NormalTok{Mu}\OperatorTok{],} \SpecialCharTok{\textbackslash{}}\OperatorTok{[}\NormalTok{Nu}\OperatorTok{],} \SpecialCharTok{\textbackslash{}}\OperatorTok{[}\NormalTok{Rho}\OperatorTok{]],}\NormalTok{ Explicit }\OtherTok{{-}\textgreater{}} \ConstantTok{True}\OperatorTok{]}
\end{Highlighting}
\end{Shaded}

\begin{dmath*}\breakingcomma
-\bar{\gamma }^{\rho }.\bar{\gamma }^{\nu }.\bar{\gamma }^{\mu }
\end{dmath*}

\begin{Shaded}
\begin{Highlighting}[]
\NormalTok{GA}\OperatorTok{[}\DecValTok{5}\OperatorTok{]} 
 
\NormalTok{FCCCT}\OperatorTok{[}\SpecialCharTok{\%}\OperatorTok{]} 
 
\SpecialCharTok{\%} \SpecialCharTok{//}\NormalTok{ Explicit}
\end{Highlighting}
\end{Shaded}

\begin{dmath*}\breakingcomma
\bar{\gamma }^5
\end{dmath*}

\begin{dmath*}\breakingcomma
C\left(\bar{\gamma }^5\right)^TC^{-1}
\end{dmath*}

\begin{dmath*}\breakingcomma
\bar{\gamma }^5
\end{dmath*}
\end{document}
