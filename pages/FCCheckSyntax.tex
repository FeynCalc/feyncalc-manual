% !TeX program = pdflatex
% !TeX root = FCCheckSyntax.tex

\documentclass[../FeynCalcManual.tex]{subfiles}
\begin{document}
\hypertarget{fcchecksyntax}{
\section{FCCheckSyntax}\label{fcchecksyntax}\index{FCCheckSyntax}}

\texttt{FCCheckSyntax[\allowbreak{}exp]} attempts to detect mistakes and
inconsistencies in the user input. The function returns the original
expression but will abort the evaluation if it thinks that the input is
incorrect. Notice that false positives are possible and it is not
guaranteed that the input which passes \texttt{FCCheckSyntax} is indeed
fully correct.

\texttt{FCCheckSyntax} is also an option for several FeynCalc routines.
If set to \texttt{True}, those functions will try to check the syntax of
the input expressions to detect possible inconsistencies. However, on
large expressions such checks may cost a lot of performance, which is
why this option is set to \texttt{False} by default.

\subsection{See also}

\hyperlink{toc}{Overview}

\subsection{Examples}

Typical mistake, using \texttt{Times} instead of \texttt{Dot} in
noncommutative products

\begin{Shaded}
\begin{Highlighting}[]
\NormalTok{FCCheckSyntax}\OperatorTok{[}\NormalTok{GA}\OperatorTok{[}\NormalTok{mu}\OperatorTok{]}\SpecialCharTok{*}\NormalTok{GA}\OperatorTok{[}\NormalTok{nu}\OperatorTok{]]}
\end{Highlighting}
\end{Shaded}

\FloatBarrier
\begin{figure}[!ht]
\centering
\includegraphics[width=0.6\linewidth]{img/10x7yb8v4z8tt.pdf}
\end{figure}
\FloatBarrier

\begin{dmath*}\breakingcomma
\text{\$Aborted}
\end{dmath*}

Another common mistake, Einstein summation convention is violated

\begin{Shaded}
\begin{Highlighting}[]
\NormalTok{FCCheckSyntax}\OperatorTok{[}\NormalTok{FV}\OperatorTok{[}\FunctionTok{p}\OperatorTok{,} \SpecialCharTok{\textbackslash{}}\OperatorTok{[}\NormalTok{Mu}\OperatorTok{]]}\NormalTok{ FV}\OperatorTok{[}\FunctionTok{q}\OperatorTok{,} \SpecialCharTok{\textbackslash{}}\OperatorTok{[}\NormalTok{Mu}\OperatorTok{]]}\NormalTok{ FV}\OperatorTok{[}\FunctionTok{r}\OperatorTok{,} \SpecialCharTok{\textbackslash{}}\OperatorTok{[}\NormalTok{Mu}\OperatorTok{]]]}
\end{Highlighting}
\end{Shaded}

\FloatBarrier
\begin{figure}[!ht]
\centering
\includegraphics[width=0.6\linewidth]{img/1pck0pnu8c08i.pdf}
\end{figure}
\FloatBarrier

\begin{dmath*}\breakingcomma
\text{\$Aborted}
\end{dmath*}
\end{document}
