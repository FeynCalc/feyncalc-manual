% !TeX program = pdflatex
% !TeX root = FCDiffEqSolve.tex

\documentclass[../FeynCalcManual.tex]{subfiles}
\begin{document}
\hypertarget{fcdiffeqsolve}{
\section{FCDiffEqSolve}\label{fcdiffeqsolve}\index{FCDiffEqSolve}}

\texttt{FCDiffEqSolve[\allowbreak{}mat,\ \allowbreak{}var,\ \allowbreak{}eps,\ \allowbreak{}n]}
constructs a solution for a single-variable differential equation
\(G' = \varepsilon \mathcal{B} G\) in the canonical form, where
\texttt{mat} is \(B\), \texttt{var} is the variable w.r.t. which \(G\)
was differentiated and \texttt{n} is the required order in \texttt{eps}.

The output consists of iterated integrals written in terms of
\texttt{FCIteratedIntegral} objects.

\subsection{See also}

\hyperlink{toc}{Overview},
\hyperlink{fciteratedintegral}{FCIteratedIntegral},
\hyperlink{fcdiffeqchangevariables}{FCDiffEqChangeVariables}

\subsection{Examples}

\begin{Shaded}
\begin{Highlighting}[]
\NormalTok{mat }\ExtensionTok{=} \OperatorTok{\{\{}\SpecialCharTok{{-}}\DecValTok{2}\SpecialCharTok{/}\FunctionTok{x}\OperatorTok{,} \DecValTok{0}\OperatorTok{,} \DecValTok{0}\OperatorTok{\},} \OperatorTok{\{}\DecValTok{0}\OperatorTok{,} \DecValTok{0}\OperatorTok{,} \DecValTok{0}\OperatorTok{\},} \OperatorTok{\{}\SpecialCharTok{{-}}\FunctionTok{x}\SpecialCharTok{\^{}}\NormalTok{(}\SpecialCharTok{{-}}\DecValTok{1}\NormalTok{)}\OperatorTok{,} \DecValTok{3}\SpecialCharTok{/}\FunctionTok{x}\OperatorTok{,} \SpecialCharTok{{-}}\DecValTok{2}\SpecialCharTok{/}\FunctionTok{x}\OperatorTok{\}\}}
\end{Highlighting}
\end{Shaded}

\begin{dmath*}\breakingcomma
\left(
\begin{array}{ccc}
 -\frac{2}{x} & 0 & 0 \\
 0 & 0 & 0 \\
 -\frac{1}{x} & \frac{3}{x} & -\frac{2}{x} \\
\end{array}
\right)
\end{dmath*}

\begin{Shaded}
\begin{Highlighting}[]
\NormalTok{FCDiffEqSolve}\OperatorTok{[}\NormalTok{mat}\OperatorTok{,} \FunctionTok{x}\OperatorTok{,}\NormalTok{ ep}\OperatorTok{,} \DecValTok{1}\OperatorTok{]}
\end{Highlighting}
\end{Shaded}

\begin{dmath*}\breakingcomma
\left\{\text{ep} \left(C[1,0] \;\text{FCIteratedIntegral}\left(\text{FCPartialFractionForm}\left(0,\left(
\begin{array}{cc}
 \{x,-1\} & -2 \\
\end{array}
\right),x\right),x,0,x\right)+C[1,1]\right)+C[1,0],\text{ep} C[2,1]+C[2,0],\text{ep} \left(C[3,0] \;\text{FCIteratedIntegral}\left(\text{FCPartialFractionForm}\left(0,\left(
\begin{array}{cc}
 \{x,-1\} & -2 \\
\end{array}
\right),x\right),x,0,x\right)+C[1,0] \;\text{FCIteratedIntegral}\left(\text{FCPartialFractionForm}\left(0,\left(
\begin{array}{cc}
 \{x,-1\} & -1 \\
\end{array}
\right),x\right),x,0,x\right)+C[2,0] \;\text{FCIteratedIntegral}\left(\text{FCPartialFractionForm}\left(0,\left(
\begin{array}{cc}
 \{x,-1\} & 3 \\
\end{array}
\right),x\right),x,0,x\right)+C[3,1]\right)+C[3,0]\right\}
\end{dmath*}
\end{document}
