% !TeX program = pdflatex
% !TeX root = FCDuplicateFreeQ.tex

\documentclass[../FeynCalcManual.tex]{subfiles}
\begin{document}
\hypertarget{fcduplicatefreeq}{%
\section{FCDuplicateFreeQ}\label{fcduplicatefreeq}}

\texttt{FCDuplicateFreeQ[\allowbreak{}list]} yields \texttt{True} if
list contains no duplicates and \texttt{False} otherwise.

\texttt{FCDuplicateFreeQ[\allowbreak{}list,\ \allowbreak{}test]} uses
test to determine whether two objects should be considered duplicates.

\texttt{FCDuplicateFreeQ} returns the same results as the standard
\texttt{DuplicateFreeQ}. The only reason for introducing
\texttt{FCDuplicateFreeQ} is that \texttt{DuplicateFreeQ} is not
available in Mathematica 8 and 9, which are still supported by FeynCalc.

\subsection{See also}

\hyperlink{toc}{Overview}, \hyperlink{fcsubsetq}{FCSubsetQ}.

\subsection{Examples}

\begin{Shaded}
\begin{Highlighting}[]
\NormalTok{FCDuplicateFreeQ}\OperatorTok{[\{}\FunctionTok{a}\OperatorTok{,} \FunctionTok{b}\OperatorTok{,} \FunctionTok{c}\OperatorTok{\}]}
\end{Highlighting}
\end{Shaded}

\begin{dmath*}\breakingcomma
\text{True}
\end{dmath*}

\begin{Shaded}
\begin{Highlighting}[]
\NormalTok{FCDuplicateFreeQ}\OperatorTok{[\{}\FunctionTok{a}\OperatorTok{,} \FunctionTok{b}\OperatorTok{,} \FunctionTok{c}\OperatorTok{,} \FunctionTok{a}\OperatorTok{\}]}
\end{Highlighting}
\end{Shaded}

\begin{dmath*}\breakingcomma
\text{False}
\end{dmath*}

\begin{Shaded}
\begin{Highlighting}[]
\NormalTok{FCDuplicateFreeQ}\OperatorTok{[\{\{}\FunctionTok{a}\OperatorTok{,} \FunctionTok{b}\OperatorTok{\},} \OperatorTok{\{}\FunctionTok{a}\OperatorTok{,} \FunctionTok{c}\OperatorTok{\}\}]}
\end{Highlighting}
\end{Shaded}

\begin{dmath*}\breakingcomma
\text{True}
\end{dmath*}

\begin{Shaded}
\begin{Highlighting}[]
\NormalTok{FCDuplicateFreeQ}\OperatorTok{[\{\{}\FunctionTok{a}\OperatorTok{,} \FunctionTok{b}\OperatorTok{\},} \OperatorTok{\{}\FunctionTok{a}\OperatorTok{,} \FunctionTok{c}\OperatorTok{\}\},} \FunctionTok{Function}\OperatorTok{[\{}\FunctionTok{x}\OperatorTok{,} \FunctionTok{y}\OperatorTok{\},} \FunctionTok{First}\OperatorTok{[}\FunctionTok{x}\OperatorTok{]} \ExtensionTok{===} \FunctionTok{First}\OperatorTok{[}\FunctionTok{y}\OperatorTok{]]]}
\end{Highlighting}
\end{Shaded}

\begin{dmath*}\breakingcomma
\text{False}
\end{dmath*}
\end{document}
