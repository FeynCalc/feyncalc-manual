% !TeX program = pdflatex
% !TeX root = FCFeynmanFindDivergences.tex

\documentclass[../FeynCalcManual.tex]{subfiles}
\begin{document}
\hypertarget{fcfeynmanfinddivergences}{
\section{FCFeynmanFindDivergences}\label{fcfeynmanfinddivergences}\index{FCFeynmanFindDivergences}}

\texttt{FCFeynmanFindDivergences[\allowbreak{}exp,\ \allowbreak{}vars]}
identifies UV and IR divergences of the given Feynman parametric
integral that arise when different parametric variables approach zero or
infinity.

This function employs the analytic regularization algorithm introduced
by Erik Panzer in \href{https://arxiv.org/abs/1403.3385}{1403.3385},
\href{https://arxiv.org/abs/1401.4361}{1401.4361} and
\href{https://arxiv.org/abs/1506.07243}{1506.07243}. Its current
implementation is very much based on the code of the
\texttt{findDivergences} routine from the Maple package
\href{https://bitbucket.org/PanzerErik/hyperint/}{HyperInt} by Erik
Panzer.

The function returns a list of lists of the form
\texttt{\{\allowbreak{}\{\allowbreak{}\{\allowbreak{}x[\allowbreak{}i],\ \allowbreak{}x[\allowbreak{}j],\ \allowbreak{}...\},\ \allowbreak{}\{\allowbreak{}x[\allowbreak{}k],\ \allowbreak{}x[\allowbreak{}l],\ \allowbreak{}...\},\ \allowbreak{}sdd\},\ \allowbreak{}...\}},
where
\texttt{\{\allowbreak{}x[\allowbreak{}i],\ \allowbreak{}x[\allowbreak{}j],\ \allowbreak{}...\}}
need to approach zero, while
\texttt{\{\allowbreak{}x[\allowbreak{}k],\ \allowbreak{}x[\allowbreak{}l],\ \allowbreak{}...\}}
must tend towards infinity to generate the superficial degree of
divergence \texttt{sdd}.

It is important to apply the function directly to the Feynman parametric
integrand obtained e.g.~from \texttt{FCFeynmanParametrize}. If the
integrand has already been modified using variable transformations or
the Cheng-Wu theorem, the algorithm may not work properly.

Furthermore, divergences that arise inside the integration domain cannot
be identified using this method.

The identified divergences can be regularized using the function
\texttt{FCFeynmanRegularizeDivergence}.

\subsection{See also}

\hyperlink{toc}{Overview},
\hyperlink{fcfeynmanparametrize}{FCFeynmanParametrize},
\hyperlink{fcfeynmanprojectivize}{FCFeynmanProjectivize},
\hyperlink{fcfeynmanregularizedivergence}{FCFeynmanRegularizeDivergence}.

\subsection{Examples}

\begin{Shaded}
\begin{Highlighting}[]
\NormalTok{int }\ExtensionTok{=}\NormalTok{ SFAD}\OperatorTok{[}\FunctionTok{l}\OperatorTok{,} \FunctionTok{k} \SpecialCharTok{+} \FunctionTok{l}\OperatorTok{,} \OperatorTok{\{\{}\FunctionTok{k}\OperatorTok{,} \SpecialCharTok{{-}}\DecValTok{2} \FunctionTok{k}\NormalTok{ . }\FunctionTok{q}\OperatorTok{\}\}]} 
 
\NormalTok{fpar }\ExtensionTok{=}\NormalTok{ FCFeynmanParametrize}\OperatorTok{[}\NormalTok{int}\OperatorTok{,} \OperatorTok{\{}\FunctionTok{k}\OperatorTok{,} \FunctionTok{l}\OperatorTok{\},} \FunctionTok{Names} \OtherTok{{-}\textgreater{}} \FunctionTok{x}\OperatorTok{,}\NormalTok{ FCReplaceD }\OtherTok{{-}\textgreater{}} \OperatorTok{\{}\FunctionTok{D} \OtherTok{{-}\textgreater{}} \DecValTok{4} \SpecialCharTok{{-}} \DecValTok{2}\NormalTok{ Epsilon}\OperatorTok{\}]}
\end{Highlighting}
\end{Shaded}

\begin{dmath*}\breakingcomma
\frac{1}{(l^2+i \eta ).((k+l)^2+i \eta ).(k^2-2 (k\cdot q)+i \eta )}
\end{dmath*}

\begin{dmath*}\breakingcomma
\left\{(x(1) x(2)+x(3) x(2)+x(1) x(3))^{3 \varepsilon -3} \left(q^2 x(1)^2 (x(2)+x(3))\right)^{1-2 \varepsilon },-\Gamma (2 \varepsilon -1),\{x(1),x(2),x(3)\}\right\}
\end{dmath*}

This Feynman parametric integral contains logarithmic divergences for
\(x_1 \to \infty\) and \(x_{2,3} \to 0\)

\begin{Shaded}
\begin{Highlighting}[]
\NormalTok{FCFeynmanFindDivergences}\OperatorTok{[}\NormalTok{fpar}\OperatorTok{[[}\DecValTok{1}\OperatorTok{]],} \FunctionTok{x}\OperatorTok{]}
\end{Highlighting}
\end{Shaded}

\begin{dmath*}\breakingcomma
\left(
\begin{array}{cc}
 \{\{\},\{x(1)\}\} & \varepsilon  \\
 \{\{x(2),x(3)\},\{\}\} & \varepsilon  \\
\end{array}
\right)
\end{dmath*}
\end{document}
