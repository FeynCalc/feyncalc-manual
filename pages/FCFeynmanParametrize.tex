% !TeX program = pdflatex
% !TeX root = FCFeynmanParametrize.tex

\documentclass[../FeynCalcManual.tex]{subfiles}
\begin{document}
\hypertarget{fcfeynmanparametrize}{
\section{FCFeynmanParametrize}\label{fcfeynmanparametrize}\index{FCFeynmanParametrize}}

\texttt{FCFeynmanParametrize[\allowbreak{}int,\ \allowbreak{}\{\allowbreak{}q1,\ \allowbreak{}q2,\ \allowbreak{}...\}]}
introduces Feynman parameters for the multi-loop integral int.

The function returns
\texttt{\{\allowbreak{}fpInt,\ \allowbreak{}pref,\ \allowbreak{}vars\}},
where \texttt{fpInt} is the integrand in Feynman parameters,
\texttt{pref} is the prefactor free of Feynman parameters and
\texttt{vars} is the list of integration variables.

If the chosen parametrization contains a Dirac delta multiplying the
integrand, it will be omitted unless the option \texttt{DiracDelta} is
set to True.

By default \texttt{FCFeynmanParametrize} uses normalization that is
common in multi-loop calculations, i.e.~\(\frac{1}{i \pi^{D/2}}\) or
\(\frac{1}{\pi^{D/2}}\) per loop for Minkowski or Euclidean/Cartesian
integrals respectively.

If you want to have the standard \(\frac{1}{(2 \pi)^D}\) normalization
or yet another value, please set the option
\texttt{FeynmanIntegralPrefactor} accordingly. Following values are
available

\begin{itemize}
\tightlist
\item
  ``MultiLoop1'' - default value explained above
\item
  ``MultiLoop2'' - like the default value but with an extra
  \(e^{\gamma_E \frac{4-D}{2}}\) per loop
\item
  ``Textbook'' - \(\frac{1}{(2 \pi)^D}\) per loop
\item
  ``Unity'' - no extra prefactor multiplying the integral measure
\item
  ``LoopTools'' - overall prefactor
  \(\frac{1}{i (\pi)^{D/2} r_{\Gamma}}\) with
  \(r_{\Gamma} = \frac{\Gamma(3-D/2) \Gamma^2 (D/2-1)}{\Gamma(D-3)}\) at
  1 loop. This matches the the normalization of 1-loop integrals in
  LoopTools. For 2 loops and above an extra \(\frac{1}{i \pi^{D/2}}\) is
  added per loop.
\end{itemize}

The calculation of \(D\)-dimensional Minkowski integrals and
\(D-1\)-dimensional Cartesian integrals is straightforward.

To calculate a \(D\)-dimensional Euclidean integral (i.e.~an integral
defined with the Euclidean metric signature \((1,1,1,1)\) you need to
write it in terms of \texttt{FVD}, \texttt{SPD}, \texttt{FAD},
\texttt{SFAD} etc. and set the option \texttt{"Euclidean"} to
\texttt{True}.

The function can derive different representations of a loop integral.
The choice of the representation is controlled by the option
\texttt{Method}. Following representations are available

\begin{itemize}
\tightlist
\item
  ``Feynman'' - the standard Feynman representation (default value).
  Both tensor integrals and integrals with scalar products in the
  numerator are supported.
\item
  ``Lee-Pomeransky'' - this representation was first introduced in
  \href{https://arxiv.org/abs/1308.6676}{1308.6676} by Roman Lee and
  Andrei Pomeransky. Currently, only scalar integrals without numerators
  are supported.
\end{itemize}

\texttt{FCFeynmanParametrize} can also be employed in conjunction with
\texttt{FCFeynmanParameterJoin}, where one first joins suitable
propagators using auxiliary Feynman parameters and then finally
integrates out loop momenta.

For a proper analysis of a loop integral one usually needs the
\texttt{U} and \texttt{F} polynomials separately. Since internally
\texttt{FCFeynmanParametrize} uses \texttt{FCFeynmanPrepare}, the
information available from the latter is also accessible to
\texttt{FCFeynmanParametrize}.

By setting the option \texttt{FCFeynmanPrepare} to \texttt{True}, the
output of \texttt{FCFeynmanPrepare} will be added the the output of
\texttt{FCFeynmanParametrize} as the 4th list element.

\subsection{See also}

\hyperlink{toc}{Overview},
\hyperlink{fcfeynmanprepare}{FCFeynmanPrepare},
\hyperlink{fcfeynmanprojectivize}{FCFeynmanProjectivize},
\hyperlink{fcfeynmanparameterjoin}{FCFeynmanParameterJoin},
\hyperlink{splitsymbolicpowers}{SplitSymbolicPowers}.

\subsection{Examples}

\hypertarget{feynman-representation}{%
\subsubsection{Feynman representation}\label{feynman-representation}}

1-loop tadpole

\begin{Shaded}
\begin{Highlighting}[]
\NormalTok{FCFeynmanParametrize}\OperatorTok{[}\NormalTok{FAD}\OperatorTok{[\{}\FunctionTok{q}\OperatorTok{,} \FunctionTok{m}\OperatorTok{\}],} \OperatorTok{\{}\FunctionTok{q}\OperatorTok{\},} \FunctionTok{Names} \OtherTok{{-}\textgreater{}} \FunctionTok{x}\OperatorTok{]}
\end{Highlighting}
\end{Shaded}

\begin{dmath*}\breakingcomma
\left\{1,-\left(m^2\right)^{\frac{D}{2}-1} \Gamma \left(1-\frac{D}{2}\right),\{\}\right\}
\end{dmath*}

Massless 1-loop 2-point function

\begin{Shaded}
\begin{Highlighting}[]
\NormalTok{FCFeynmanParametrize}\OperatorTok{[}\NormalTok{FAD}\OperatorTok{[}\FunctionTok{q}\OperatorTok{,} \FunctionTok{q} \SpecialCharTok{{-}} \FunctionTok{p}\OperatorTok{],} \OperatorTok{\{}\FunctionTok{q}\OperatorTok{\},} \FunctionTok{Names} \OtherTok{{-}\textgreater{}} \FunctionTok{x}\OperatorTok{]}
\end{Highlighting}
\end{Shaded}

\begin{dmath*}\breakingcomma
\left\{(x(1)+x(2))^{2-D} \left(-p^2 x(1) x(2)\right)^{\frac{D}{2}-2},\Gamma \left(2-\frac{D}{2}\right),\{x(1),x(2)\}\right\}
\end{dmath*}

With \(p^2\) replaced by \texttt{pp} and \texttt{D} set to
\texttt{4 - 2 Epsilon}

\begin{Shaded}
\begin{Highlighting}[]
\NormalTok{FCFeynmanParametrize}\OperatorTok{[}\NormalTok{FAD}\OperatorTok{[}\FunctionTok{q}\OperatorTok{,} \FunctionTok{q} \SpecialCharTok{{-}} \FunctionTok{p}\OperatorTok{],} \OperatorTok{\{}\FunctionTok{q}\OperatorTok{\},} \FunctionTok{Names} \OtherTok{{-}\textgreater{}} \FunctionTok{x}\OperatorTok{,}\NormalTok{ FinalSubstitutions }\OtherTok{{-}\textgreater{}}\NormalTok{ SPD}\OperatorTok{[}\FunctionTok{p}\OperatorTok{]} \OtherTok{{-}\textgreater{}}\NormalTok{ pp}\OperatorTok{,} 
\NormalTok{  FCReplaceD }\OtherTok{{-}\textgreater{}} \OperatorTok{\{}\FunctionTok{D} \OtherTok{{-}\textgreater{}} \DecValTok{4} \SpecialCharTok{{-}} \DecValTok{2}\NormalTok{ Epsilon}\OperatorTok{\}]}
\end{Highlighting}
\end{Shaded}

\begin{dmath*}\breakingcomma
\left\{(x(1)+x(2))^{2 \varepsilon -2} (-\text{pp} x(1) x(2))^{-\varepsilon },\Gamma (\varepsilon ),\{x(1),x(2)\}\right\}
\end{dmath*}

Standard text-book prefactor of the loop integral measure

\begin{Shaded}
\begin{Highlighting}[]
\NormalTok{FCFeynmanParametrize}\OperatorTok{[}\NormalTok{FAD}\OperatorTok{[}\FunctionTok{q}\OperatorTok{,} \FunctionTok{q} \SpecialCharTok{{-}} \FunctionTok{p}\OperatorTok{],} \OperatorTok{\{}\FunctionTok{q}\OperatorTok{\},} \FunctionTok{Names} \OtherTok{{-}\textgreater{}} \FunctionTok{x}\OperatorTok{,}\NormalTok{ FinalSubstitutions }\OtherTok{{-}\textgreater{}}\NormalTok{ SPD}\OperatorTok{[}\FunctionTok{p}\OperatorTok{]} \OtherTok{{-}\textgreater{}}\NormalTok{ pp}\OperatorTok{,} 
\NormalTok{  FCReplaceD }\OtherTok{{-}\textgreater{}} \OperatorTok{\{}\FunctionTok{D} \OtherTok{{-}\textgreater{}} \DecValTok{4} \SpecialCharTok{{-}} \DecValTok{2}\NormalTok{ Epsilon}\OperatorTok{\},}\NormalTok{ FeynmanIntegralPrefactor }\OtherTok{{-}\textgreater{}} \StringTok{"Textbook"}\OperatorTok{]}
\end{Highlighting}
\end{Shaded}

\begin{dmath*}\breakingcomma
\left\{(x(1)+x(2))^{2 \varepsilon -2} (-\text{pp} x(1) x(2))^{-\varepsilon },i 2^{2 \varepsilon -4} \pi ^{\varepsilon -2} \Gamma (\varepsilon ),\{x(1),x(2)\}\right\}
\end{dmath*}

Same integral but with the Euclidean metric signature

\begin{Shaded}
\begin{Highlighting}[]
\NormalTok{FCFeynmanParametrize}\OperatorTok{[}\NormalTok{FAD}\OperatorTok{[}\FunctionTok{q}\OperatorTok{,} \FunctionTok{q} \SpecialCharTok{{-}} \FunctionTok{p}\OperatorTok{],} \OperatorTok{\{}\FunctionTok{q}\OperatorTok{\},} \FunctionTok{Names} \OtherTok{{-}\textgreater{}} \FunctionTok{x}\OperatorTok{,}\NormalTok{ FinalSubstitutions }\OtherTok{{-}\textgreater{}}\NormalTok{ SPD}\OperatorTok{[}\FunctionTok{p}\OperatorTok{]} \OtherTok{{-}\textgreater{}}\NormalTok{ pp}\OperatorTok{,} 
\NormalTok{  FCReplaceD }\OtherTok{{-}\textgreater{}} \OperatorTok{\{}\FunctionTok{D} \OtherTok{{-}\textgreater{}} \DecValTok{4} \SpecialCharTok{{-}} \DecValTok{2}\NormalTok{ Epsilon}\OperatorTok{\},}\NormalTok{ FeynmanIntegralPrefactor }\OtherTok{{-}\textgreater{}} \StringTok{"Textbook"}\OperatorTok{,} \StringTok{"Euclidean"} \OtherTok{{-}\textgreater{}} \ConstantTok{True}\OperatorTok{]}
\end{Highlighting}
\end{Shaded}

\begin{dmath*}\breakingcomma
\left\{(x(1)+x(2))^{2 \varepsilon -2} (\text{pp} x(1) x(2))^{-\varepsilon },2^{2 \varepsilon -4} \pi ^{\varepsilon -2} \Gamma (\varepsilon ),\{x(1),x(2)\}\right\}
\end{dmath*}

A tensor integral

\begin{Shaded}
\begin{Highlighting}[]
\NormalTok{FCFeynmanParametrize}\OperatorTok{[}\NormalTok{FAD}\OperatorTok{[\{}\FunctionTok{q}\OperatorTok{,} \FunctionTok{m}\OperatorTok{\}]}\NormalTok{ FAD}\OperatorTok{[\{}\FunctionTok{q} \SpecialCharTok{{-}} \FunctionTok{p}\OperatorTok{,}\NormalTok{ m2}\OperatorTok{\}]}\NormalTok{ FVD}\OperatorTok{[}\FunctionTok{q}\OperatorTok{,}\NormalTok{ mu}\OperatorTok{]}\NormalTok{ FVD}\OperatorTok{[}\FunctionTok{q}\OperatorTok{,}\NormalTok{ nu}\OperatorTok{],} \OperatorTok{\{}\FunctionTok{q}\OperatorTok{\},} 
  \FunctionTok{Names} \OtherTok{{-}\textgreater{}} \FunctionTok{x}\OperatorTok{,}\NormalTok{ FCE }\OtherTok{{-}\textgreater{}} \ConstantTok{True}\OperatorTok{]}
\end{Highlighting}
\end{Shaded}

\begin{dmath*}\breakingcomma
\left\{(x(1)+x(2))^{-D} \left(m^2 x(1)^2+m^2 x(1) x(2)+\text{m2}^2 x(2)^2+\text{m2}^2 x(1) x(2)-p^2 x(1) x(2)\right)^{\frac{D}{2}-2} \left(x(2)^2 \Gamma \left(2-\frac{D}{2}\right) p^{\text{mu}} p^{\text{nu}}-\frac{1}{2} \Gamma \left(1-\frac{D}{2}\right) g^{\text{mu}\;\text{nu}} \left(m^2 x(1)^2+m^2 x(1) x(2)+\text{m2}^2 x(2)^2+\text{m2}^2 x(1) x(2)-p^2 x(1) x(2)\right)\right),1,\{x(1),x(2)\}\right\}
\end{dmath*}

1-loop master formulas for Minkowski integrals (cf.~Eq. 9.49b in
Sterman's An introduction to QFT)

\begin{Shaded}
\begin{Highlighting}[]
\NormalTok{SFAD}\OperatorTok{[\{\{}\FunctionTok{k}\OperatorTok{,} \DecValTok{2} \FunctionTok{p}\NormalTok{ . }\FunctionTok{k}\OperatorTok{\},} \FunctionTok{M}\SpecialCharTok{\^{}}\DecValTok{2}\OperatorTok{,} \FunctionTok{s}\OperatorTok{\}]} 
 
\NormalTok{FCFeynmanParametrize}\OperatorTok{[}\SpecialCharTok{\%}\OperatorTok{,} \OperatorTok{\{}\FunctionTok{k}\OperatorTok{\},} \FunctionTok{Names} \OtherTok{{-}\textgreater{}} \FunctionTok{x}\OperatorTok{,}\NormalTok{ FCE }\OtherTok{{-}\textgreater{}} \ConstantTok{True}\OperatorTok{,}\NormalTok{ FeynmanIntegralPrefactor }\OtherTok{{-}\textgreater{}} \DecValTok{1}\OperatorTok{,} 
\NormalTok{  FCReplaceD }\OtherTok{{-}\textgreater{}} \OperatorTok{\{}\FunctionTok{D} \OtherTok{{-}\textgreater{}} \FunctionTok{n}\OperatorTok{\}]}
\end{Highlighting}
\end{Shaded}

\begin{dmath*}\breakingcomma
(k^2+2 (k\cdot p)-M^2+i \eta )^{-s}
\end{dmath*}

\begin{dmath*}\breakingcomma
\left\{1,\frac{i \pi ^{n/2} (-1)^s \Gamma \left(s-\frac{n}{2}\right) \left(M^2+p^2\right)^{\frac{n}{2}-s}}{\Gamma (s)},\{\}\right\}
\end{dmath*}

\begin{Shaded}
\begin{Highlighting}[]
\NormalTok{FVD}\OperatorTok{[}\FunctionTok{k}\OperatorTok{,} \SpecialCharTok{\textbackslash{}}\OperatorTok{[}\NormalTok{Mu}\OperatorTok{]]}\NormalTok{ SFAD}\OperatorTok{[\{\{}\FunctionTok{k}\OperatorTok{,} \DecValTok{2} \FunctionTok{p}\NormalTok{ . }\FunctionTok{k}\OperatorTok{\},} \FunctionTok{M}\SpecialCharTok{\^{}}\DecValTok{2}\OperatorTok{,} \FunctionTok{s}\OperatorTok{\}]} 
 
\NormalTok{FCFeynmanParametrize}\OperatorTok{[}\SpecialCharTok{\%}\OperatorTok{,} \OperatorTok{\{}\FunctionTok{k}\OperatorTok{\},} \FunctionTok{Names} \OtherTok{{-}\textgreater{}} \FunctionTok{x}\OperatorTok{,}\NormalTok{ FCE }\OtherTok{{-}\textgreater{}} \ConstantTok{True}\OperatorTok{,}\NormalTok{ FeynmanIntegralPrefactor }\OtherTok{{-}\textgreater{}} \DecValTok{1}\OperatorTok{,} 
\NormalTok{  FCReplaceD }\OtherTok{{-}\textgreater{}} \OperatorTok{\{}\FunctionTok{D} \OtherTok{{-}\textgreater{}} \FunctionTok{n}\OperatorTok{\}]}
\end{Highlighting}
\end{Shaded}

\begin{dmath*}\breakingcomma
k^{\mu } (k^2+2 (k\cdot p)-M^2+i \eta )^{-s}
\end{dmath*}

\begin{dmath*}\breakingcomma
\left\{1,-\frac{i \pi ^{n/2} (-1)^s p^{\mu } \Gamma \left(s-\frac{n}{2}\right) \left(M^2+p^2\right)^{\frac{n}{2}-s}}{\Gamma (s)},\{\}\right\}
\end{dmath*}

\begin{Shaded}
\begin{Highlighting}[]
\NormalTok{FVD}\OperatorTok{[}\FunctionTok{k}\OperatorTok{,} \SpecialCharTok{\textbackslash{}}\OperatorTok{[}\NormalTok{Mu}\OperatorTok{]]}\NormalTok{ FVD}\OperatorTok{[}\FunctionTok{k}\OperatorTok{,} \SpecialCharTok{\textbackslash{}}\OperatorTok{[}\NormalTok{Nu}\OperatorTok{]]}\NormalTok{ SFAD}\OperatorTok{[\{\{}\FunctionTok{k}\OperatorTok{,} \DecValTok{2} \FunctionTok{p}\NormalTok{ . }\FunctionTok{k}\OperatorTok{\},} \FunctionTok{M}\SpecialCharTok{\^{}}\DecValTok{2}\OperatorTok{,} \FunctionTok{s}\OperatorTok{\}]} 
 
\NormalTok{FCFeynmanParametrize}\OperatorTok{[}\SpecialCharTok{\%}\OperatorTok{,} \OperatorTok{\{}\FunctionTok{k}\OperatorTok{\},} \FunctionTok{Names} \OtherTok{{-}\textgreater{}} \FunctionTok{x}\OperatorTok{,}\NormalTok{ FCE }\OtherTok{{-}\textgreater{}} \ConstantTok{True}\OperatorTok{,}\NormalTok{ FeynmanIntegralPrefactor }\OtherTok{{-}\textgreater{}} \DecValTok{1}\OperatorTok{,} 
\NormalTok{  FCReplaceD }\OtherTok{{-}\textgreater{}} \OperatorTok{\{}\FunctionTok{D} \OtherTok{{-}\textgreater{}} \FunctionTok{n}\OperatorTok{\}]}
\end{Highlighting}
\end{Shaded}

\begin{dmath*}\breakingcomma
k^{\mu } k^{\nu } (k^2+2 (k\cdot p)-M^2+i \eta )^{-s}
\end{dmath*}

\begin{dmath*}\breakingcomma
\left\{1,\frac{i \pi ^{n/2} (-1)^s \left(M^2+p^2\right)^{\frac{n}{2}-s} \left(p^{\mu } p^{\nu } \Gamma \left(s-\frac{n}{2}\right)-\frac{1}{2} \left(M^2+p^2\right) g^{\mu \nu } \Gamma \left(-\frac{n}{2}+s-1\right)\right)}{\Gamma (s)},\{\}\right\}
\end{dmath*}

1-loop master formulas for Euclidean integrals (cf.~Eq. 9.49a in
Sterman's An introduction to QFT)

\begin{Shaded}
\begin{Highlighting}[]
\NormalTok{SFAD}\OperatorTok{[\{\{}\FunctionTok{k}\OperatorTok{,} \DecValTok{2} \FunctionTok{p}\NormalTok{ . }\FunctionTok{k}\OperatorTok{\},} \SpecialCharTok{{-}}\FunctionTok{M}\SpecialCharTok{\^{}}\DecValTok{2}\OperatorTok{,} \FunctionTok{s}\OperatorTok{\}]} 
 
\NormalTok{FCFeynmanParametrize}\OperatorTok{[}\SpecialCharTok{\%}\OperatorTok{,} \OperatorTok{\{}\FunctionTok{k}\OperatorTok{\},} \FunctionTok{Names} \OtherTok{{-}\textgreater{}} \FunctionTok{x}\OperatorTok{,}\NormalTok{ FCE }\OtherTok{{-}\textgreater{}} \ConstantTok{True}\OperatorTok{,} \StringTok{"Euclidean"} \OtherTok{{-}\textgreater{}} \ConstantTok{True}\OperatorTok{,} 
\NormalTok{  FeynmanIntegralPrefactor }\OtherTok{{-}\textgreater{}} \FunctionTok{I}\OperatorTok{]}
\end{Highlighting}
\end{Shaded}

\begin{dmath*}\breakingcomma
(k^2+2 (k\cdot p)+M^2+i \eta )^{-s}
\end{dmath*}

\begin{dmath*}\breakingcomma
\left\{1,\frac{i \pi ^{D/2} \Gamma \left(s-\frac{D}{2}\right) \left(M^2-p^2\right)^{\frac{D}{2}-s}}{\Gamma (s)},\{\}\right\}
\end{dmath*}

\begin{Shaded}
\begin{Highlighting}[]
\NormalTok{FVD}\OperatorTok{[}\FunctionTok{k}\OperatorTok{,} \SpecialCharTok{\textbackslash{}}\OperatorTok{[}\NormalTok{Mu}\OperatorTok{]]}\NormalTok{ SFAD}\OperatorTok{[\{\{}\FunctionTok{k}\OperatorTok{,} \DecValTok{2} \FunctionTok{p}\NormalTok{ . }\FunctionTok{k}\OperatorTok{\},} \SpecialCharTok{{-}}\FunctionTok{M}\SpecialCharTok{\^{}}\DecValTok{2}\OperatorTok{,} \FunctionTok{s}\OperatorTok{\}]} 
 
\NormalTok{FCFeynmanParametrize}\OperatorTok{[}\SpecialCharTok{\%}\OperatorTok{,} \OperatorTok{\{}\FunctionTok{k}\OperatorTok{\},} \FunctionTok{Names} \OtherTok{{-}\textgreater{}} \FunctionTok{x}\OperatorTok{,}\NormalTok{ FCE }\OtherTok{{-}\textgreater{}} \ConstantTok{True}\OperatorTok{,}\NormalTok{ FeynmanIntegralPrefactor }\OtherTok{{-}\textgreater{}} \FunctionTok{I}\OperatorTok{,} 
\NormalTok{  FCReplaceD }\OtherTok{{-}\textgreater{}} \OperatorTok{\{}\FunctionTok{D} \OtherTok{{-}\textgreater{}} \FunctionTok{n}\OperatorTok{\},} \StringTok{"Euclidean"} \OtherTok{{-}\textgreater{}} \ConstantTok{True}\OperatorTok{]}
\end{Highlighting}
\end{Shaded}

\begin{dmath*}\breakingcomma
k^{\mu } (k^2+2 (k\cdot p)+M^2+i \eta )^{-s}
\end{dmath*}

\begin{dmath*}\breakingcomma
\left\{1,-\frac{i \pi ^{n/2} p^{\mu } \Gamma \left(s-\frac{n}{2}\right) \left(M^2-p^2\right)^{\frac{n}{2}-s}}{\Gamma (s)},\{\}\right\}
\end{dmath*}

\begin{Shaded}
\begin{Highlighting}[]
\NormalTok{FVD}\OperatorTok{[}\FunctionTok{k}\OperatorTok{,} \SpecialCharTok{\textbackslash{}}\OperatorTok{[}\NormalTok{Mu}\OperatorTok{]]}\NormalTok{ FVD}\OperatorTok{[}\FunctionTok{k}\OperatorTok{,} \SpecialCharTok{\textbackslash{}}\OperatorTok{[}\NormalTok{Nu}\OperatorTok{]]}\NormalTok{ SFAD}\OperatorTok{[\{\{}\FunctionTok{k}\OperatorTok{,} \DecValTok{2} \FunctionTok{p}\NormalTok{ . }\FunctionTok{k}\OperatorTok{\},} \SpecialCharTok{{-}}\FunctionTok{M}\SpecialCharTok{\^{}}\DecValTok{2}\OperatorTok{,} \FunctionTok{s}\OperatorTok{\}]} 
 
\NormalTok{FCFeynmanParametrize}\OperatorTok{[}\SpecialCharTok{\%}\OperatorTok{,} \OperatorTok{\{}\FunctionTok{k}\OperatorTok{\},} \FunctionTok{Names} \OtherTok{{-}\textgreater{}} \FunctionTok{x}\OperatorTok{,}\NormalTok{ FCE }\OtherTok{{-}\textgreater{}} \ConstantTok{True}\OperatorTok{,}\NormalTok{ FeynmanIntegralPrefactor }\OtherTok{{-}\textgreater{}} \FunctionTok{I}\OperatorTok{,} 
\NormalTok{  FCReplaceD }\OtherTok{{-}\textgreater{}} \OperatorTok{\{}\FunctionTok{D} \OtherTok{{-}\textgreater{}} \FunctionTok{n}\OperatorTok{\},} \StringTok{"Euclidean"} \OtherTok{{-}\textgreater{}} \ConstantTok{True}\OperatorTok{]}
\end{Highlighting}
\end{Shaded}

\begin{dmath*}\breakingcomma
k^{\mu } k^{\nu } (k^2+2 (k\cdot p)+M^2+i \eta )^{-s}
\end{dmath*}

\begin{dmath*}\breakingcomma
\left\{1,\frac{i \pi ^{n/2} \left(M^2-p^2\right)^{\frac{n}{2}-s} \left(\frac{1}{2} \left(M^2-p^2\right) g^{\mu \nu } \Gamma \left(-\frac{n}{2}+s-1\right)+p^{\mu } p^{\nu } \Gamma \left(s-\frac{n}{2}\right)\right)}{\Gamma (s)},\{\}\right\}
\end{dmath*}

1-loop massless box

\begin{Shaded}
\begin{Highlighting}[]
\NormalTok{FAD}\OperatorTok{[}\FunctionTok{p}\OperatorTok{,} \FunctionTok{p} \SpecialCharTok{+}\NormalTok{ q1}\OperatorTok{,} \FunctionTok{p} \SpecialCharTok{+}\NormalTok{ q1 }\SpecialCharTok{+}\NormalTok{ q2}\OperatorTok{,} \FunctionTok{p} \SpecialCharTok{+}\NormalTok{ q1 }\SpecialCharTok{+}\NormalTok{ q2 }\SpecialCharTok{+}\NormalTok{ q3}\OperatorTok{]} 
 
\NormalTok{FCFeynmanParametrize}\OperatorTok{[}\SpecialCharTok{\%}\OperatorTok{,} \OperatorTok{\{}\FunctionTok{p}\OperatorTok{\},} \FunctionTok{Names} \OtherTok{{-}\textgreater{}} \FunctionTok{x}\OperatorTok{,}\NormalTok{ FCReplaceD }\OtherTok{{-}\textgreater{}} \OperatorTok{\{}\FunctionTok{D} \OtherTok{{-}\textgreater{}} \DecValTok{4} \SpecialCharTok{{-}} \DecValTok{2}\NormalTok{ Epsilon}\OperatorTok{\}]}
\end{Highlighting}
\end{Shaded}

\begin{dmath*}\breakingcomma
\frac{1}{p^2.(p+\text{q1})^2.(p+\text{q1}+\text{q2})^2.(p+\text{q1}+\text{q2}+\text{q3})^2}
\end{dmath*}

\begin{dmath*}\breakingcomma
\left\{(x(1)+x(2)+x(3)+x(4))^{2 \varepsilon } \left(-2 x(1) x(3) (\text{q1}\cdot \;\text{q2})-2 x(1) x(4) (\text{q1}\cdot \;\text{q2})-2 x(1) x(4) (\text{q1}\cdot \;\text{q3})-\text{q1}^2 x(1) x(2)-\text{q1}^2 x(1) x(3)-\text{q1}^2 x(1) x(4)-2 x(4) x(2) (\text{q2}\cdot \;\text{q3})-2 x(1) x(4) (\text{q2}\cdot \;\text{q3})-\text{q2}^2 x(3) x(2)-\text{q2}^2 x(4) x(2)-\text{q2}^2 x(1) x(3)-\text{q2}^2 x(1) x(4)-\text{q3}^2 x(4) x(2)-\text{q3}^2 x(1) x(4)-\text{q3}^2 x(3) x(4)\right)^{-\varepsilon -2},\Gamma (\varepsilon +2),\{x(1),x(2),x(3),x(4)\}\right\}
\end{dmath*}

3-loop self-energy with two massive lines

\begin{Shaded}
\begin{Highlighting}[]
\NormalTok{SFAD}\OperatorTok{[\{\{}\NormalTok{p1}\OperatorTok{,} \DecValTok{0}\OperatorTok{\},} \OperatorTok{\{}\FunctionTok{m}\SpecialCharTok{\^{}}\DecValTok{2}\OperatorTok{,} \DecValTok{1}\OperatorTok{\},} \DecValTok{1}\OperatorTok{\},} \OperatorTok{\{\{}\NormalTok{p2}\OperatorTok{,} \DecValTok{0}\OperatorTok{\},} \OperatorTok{\{}\DecValTok{0}\OperatorTok{,} \DecValTok{1}\OperatorTok{\},} \DecValTok{1}\OperatorTok{\},} \OperatorTok{\{\{}\NormalTok{p3}\OperatorTok{,} \DecValTok{0}\OperatorTok{\},} \OperatorTok{\{}\DecValTok{0}\OperatorTok{,} \DecValTok{1}\OperatorTok{\},} \DecValTok{1}\OperatorTok{\},} 
   \OperatorTok{\{\{}\NormalTok{p2 }\SpecialCharTok{+}\NormalTok{ p3}\OperatorTok{,} \DecValTok{0}\OperatorTok{\},} \OperatorTok{\{}\DecValTok{0}\OperatorTok{,} \DecValTok{1}\OperatorTok{\},} \DecValTok{1}\OperatorTok{\},} \OperatorTok{\{\{}\NormalTok{p1 }\SpecialCharTok{{-}} \FunctionTok{Q}\OperatorTok{,} \DecValTok{0}\OperatorTok{\},} \OperatorTok{\{}\FunctionTok{m}\SpecialCharTok{\^{}}\DecValTok{2}\OperatorTok{,} \DecValTok{1}\OperatorTok{\},} \DecValTok{1}\OperatorTok{\},} \OperatorTok{\{\{}\NormalTok{p2 }\SpecialCharTok{{-}} \FunctionTok{Q}\OperatorTok{,} \DecValTok{0}\OperatorTok{\},} \OperatorTok{\{}\DecValTok{0}\OperatorTok{,} \DecValTok{1}\OperatorTok{\},} \DecValTok{1}\OperatorTok{\},} 
   \OperatorTok{\{\{}\NormalTok{p2 }\SpecialCharTok{+}\NormalTok{ p3 }\SpecialCharTok{{-}} \FunctionTok{Q}\OperatorTok{,} \DecValTok{0}\OperatorTok{\},} \OperatorTok{\{}\DecValTok{0}\OperatorTok{,} \DecValTok{1}\OperatorTok{\},} \DecValTok{1}\OperatorTok{\},} \OperatorTok{\{\{}\NormalTok{p1 }\SpecialCharTok{+}\NormalTok{ p2 }\SpecialCharTok{+}\NormalTok{ p3 }\SpecialCharTok{{-}} \FunctionTok{Q}\OperatorTok{,} \DecValTok{0}\OperatorTok{\},} \OperatorTok{\{}\DecValTok{0}\OperatorTok{,} \DecValTok{1}\OperatorTok{\},} \DecValTok{1}\OperatorTok{\}]} 
 
\NormalTok{FCFeynmanParametrize}\OperatorTok{[}\SpecialCharTok{\%}\OperatorTok{,} \OperatorTok{\{}\NormalTok{p1}\OperatorTok{,}\NormalTok{ p2}\OperatorTok{,}\NormalTok{ p3}\OperatorTok{\},} \FunctionTok{Names} \OtherTok{{-}\textgreater{}} \FunctionTok{x}\OperatorTok{,}\NormalTok{ FCReplaceD }\OtherTok{{-}\textgreater{}} \OperatorTok{\{}\FunctionTok{D} \OtherTok{{-}\textgreater{}} \DecValTok{4} \SpecialCharTok{{-}} \DecValTok{2}\NormalTok{ Epsilon}\OperatorTok{\}]}
\end{Highlighting}
\end{Shaded}

\begin{dmath*}\breakingcomma
\frac{1}{(\text{p1}^2-m^2+i \eta ).(\text{p2}^2+i \eta ).(\text{p3}^2+i \eta ).((\text{p2}+\text{p3})^2+i \eta ).((\text{p1}-Q)^2-m^2+i \eta ).((\text{p2}-Q)^2+i \eta ).((\text{p2}+\text{p3}-Q)^2+i \eta ).((\text{p1}+\text{p2}+\text{p3}-Q)^2+i \eta )}
\end{dmath*}

\begin{dmath*}\breakingcomma
\left\{(x(1) x(2) x(3)+x(1) x(4) x(3)+x(2) x(5) x(3)+x(4) x(5) x(3)+x(1) x(6) x(3)+x(5) x(6) x(3)+x(1) x(7) x(3)+x(5) x(7) x(3)+x(1) x(8) x(3)+x(2) x(8) x(3)+x(4) x(8) x(3)+x(5) x(8) x(3)+x(6) x(8) x(3)+x(7) x(8) x(3)+x(1) x(2) x(4)+x(2) x(4) x(5)+x(1) x(4) x(6)+x(4) x(5) x(6)+x(1) x(2) x(7)+x(2) x(5) x(7)+x(1) x(6) x(7)+x(5) x(6) x(7)+x(1) x(2) x(8)+x(2) x(4) x(8)+x(2) x(5) x(8)+x(1) x(6) x(8)+x(4) x(6) x(8)+x(5) x(6) x(8)+x(2) x(7) x(8)+x(6) x(7) x(8))^{4 \varepsilon } \left(x(2) x(3) x(5)^2 m^2+x(2) x(4) x(5)^2 m^2+x(3) x(4) x(5)^2 m^2+x(1)^2 x(2) x(3) m^2+x(1)^2 x(2) x(4) m^2+x(1)^2 x(3) x(4) m^2+2 x(1) x(2) x(3) x(5) m^2+2 x(1) x(2) x(4) x(5) m^2+2 x(1) x(3) x(4) x(5) m^2+x(3) x(5)^2 x(6) m^2+x(4) x(5)^2 x(6) m^2+x(1)^2 x(3) x(6) m^2+x(1)^2 x(4) x(6) m^2+2 x(1) x(3) x(5) x(6) m^2+2 x(1) x(4) x(5) x(6) m^2+x(2) x(5)^2 x(7) m^2+x(3) x(5)^2 x(7) m^2+x(1)^2 x(2) x(7) m^2+x(1)^2 x(3) x(7) m^2+2 x(1) x(2) x(5) x(7) m^2+2 x(1) x(3) x(5) x(7) m^2+x(1)^2 x(6) x(7) m^2+x(5)^2 x(6) x(7) m^2+2 x(1) x(5) x(6) x(7) m^2+x(2) x(5)^2 x(8) m^2+x(3) x(5)^2 x(8) m^2+x(1)^2 x(2) x(8) m^2+x(1)^2 x(3) x(8) m^2+x(1) x(2) x(3) x(8) m^2+x(1) x(2) x(4) x(8) m^2+x(1) x(3) x(4) x(8) m^2+2 x(1) x(2) x(5) x(8) m^2+2 x(1) x(3) x(5) x(8) m^2+x(2) x(3) x(5) x(8) m^2+x(2) x(4) x(5) x(8) m^2+x(3) x(4) x(5) x(8) m^2+x(1)^2 x(6) x(8) m^2+x(5)^2 x(6) x(8) m^2+x(1) x(3) x(6) x(8) m^2+x(1) x(4) x(6) x(8) m^2+2 x(1) x(5) x(6) x(8) m^2+x(3) x(5) x(6) x(8) m^2+x(4) x(5) x(6) x(8) m^2+x(1) x(2) x(7) x(8) m^2+x(1) x(3) x(7) x(8) m^2+x(2) x(5) x(7) x(8) m^2+x(3) x(5) x(7) x(8) m^2+x(1) x(6) x(7) x(8) m^2+x(5) x(6) x(7) x(8) m^2-Q^2 x(1) x(2) x(3) x(5)-Q^2 x(1) x(2) x(4) x(5)-Q^2 x(1) x(3) x(4) x(5)-Q^2 x(1) x(2) x(3) x(6)-Q^2 x(1) x(2) x(4) x(6)-Q^2 x(1) x(3) x(4) x(6)-Q^2 x(1) x(3) x(5) x(6)-Q^2 x(2) x(3) x(5) x(6)-Q^2 x(1) x(4) x(5) x(6)-Q^2 x(2) x(4) x(5) x(6)-Q^2 x(3) x(4) x(5) x(6)-Q^2 x(1) x(2) x(3) x(7)-Q^2 x(1) x(2) x(4) x(7)-Q^2 x(1) x(3) x(4) x(7)-Q^2 x(1) x(2) x(5) x(7)-Q^2 x(1) x(3) x(5) x(7)-Q^2 x(2) x(3) x(5) x(7)-Q^2 x(2) x(4) x(5) x(7)-Q^2 x(3) x(4) x(5) x(7)-Q^2 x(1) x(2) x(6) x(7)-Q^2 x(1) x(4) x(6) x(7)-Q^2 x(1) x(5) x(6) x(7)-Q^2 x(2) x(5) x(6) x(7)-Q^2 x(4) x(5) x(6) x(7)-Q^2 x(1) x(2) x(3) x(8)-Q^2 x(1) x(2) x(4) x(8)-Q^2 x(1) x(3) x(4) x(8)-Q^2 x(1) x(2) x(5) x(8)-Q^2 x(1) x(3) x(5) x(8)-Q^2 x(1) x(2) x(6) x(8)-Q^2 x(2) x(3) x(6) x(8)-Q^2 x(1) x(4) x(6) x(8)-Q^2 x(2) x(4) x(6) x(8)-Q^2 x(3) x(4) x(6) x(8)-Q^2 x(1) x(5) x(6) x(8)-Q^2 x(2) x(5) x(6) x(8)-Q^2 x(3) x(5) x(6) x(8)-Q^2 x(2) x(3) x(7) x(8)-Q^2 x(2) x(4) x(7) x(8)-Q^2 x(3) x(4) x(7) x(8)-Q^2 x(2) x(5) x(7) x(8)-Q^2 x(3) x(5) x(7) x(8)-Q^2 x(2) x(6) x(7) x(8)-Q^2 x(4) x(6) x(7) x(8)-Q^2 x(5) x(6) x(7) x(8)\right)^{-3 \varepsilon -2},\Gamma (3 \varepsilon +2),\{x(1),x(2),x(3),x(4),x(5),x(6),x(7),x(8)\}\right\}
\end{dmath*}

An example of using \texttt{FCFeynmanParametrize} together with
\texttt{FCFeynmanParameterJoin}

\begin{Shaded}
\begin{Highlighting}[]
\NormalTok{props }\ExtensionTok{=} \OperatorTok{\{}\NormalTok{SFAD}\OperatorTok{[\{}\NormalTok{p1}\OperatorTok{,} \FunctionTok{m}\SpecialCharTok{\^{}}\DecValTok{2}\OperatorTok{\}],}\NormalTok{ SFAD}\OperatorTok{[\{}\NormalTok{p3}\OperatorTok{,} \FunctionTok{m}\SpecialCharTok{\^{}}\DecValTok{2}\OperatorTok{\}],}\NormalTok{ SFAD}\OperatorTok{[\{\{}\DecValTok{0}\OperatorTok{,} \DecValTok{2}\NormalTok{ p1 . }\FunctionTok{n}\OperatorTok{\}\}],} 
\NormalTok{   SFAD}\OperatorTok{[\{\{}\DecValTok{0}\OperatorTok{,} \DecValTok{2}\NormalTok{ (p1 }\SpecialCharTok{+}\NormalTok{ p3) . }\FunctionTok{n}\OperatorTok{\}\}]\}}
\end{Highlighting}
\end{Shaded}

\begin{dmath*}\breakingcomma
\left\{\frac{1}{(\text{p1}^2-m^2+i \eta )},\frac{1}{(\text{p3}^2-m^2+i \eta )},\frac{1}{(2 (n\cdot \;\text{p1})+i \eta )},\frac{1}{(2 (n\cdot (\text{p1}+\text{p3}))+i \eta )}\right\}
\end{dmath*}

\begin{Shaded}
\begin{Highlighting}[]
\NormalTok{intT }\ExtensionTok{=}\NormalTok{ FCFeynmanParameterJoin}\OperatorTok{[\{\{}\NormalTok{props}\OperatorTok{[[}\DecValTok{1}\OperatorTok{]]}\NormalTok{ props}\OperatorTok{[[}\DecValTok{2}\OperatorTok{]],} \DecValTok{1}\OperatorTok{,} \FunctionTok{x}\OperatorTok{\},} 
\NormalTok{    props}\OperatorTok{[[}\DecValTok{3}\OperatorTok{]]}\NormalTok{ props}\OperatorTok{[[}\DecValTok{4}\OperatorTok{]],} \FunctionTok{y}\OperatorTok{\},} \OperatorTok{\{}\NormalTok{p1}\OperatorTok{,}\NormalTok{ p3}\OperatorTok{\}]}
\end{Highlighting}
\end{Shaded}

\begin{dmath*}\breakingcomma
\left\{\frac{1}{(\left(-x(1) m^2-x(2) m^2+\text{p1}^2 x(1)+\text{p3}^2 x(2)\right) y(1)+2 (n\cdot \;\text{p1}) y(2)+(2 (n\cdot \;\text{p1})+2 (n\cdot \;\text{p3})) y(3)+i \eta )^4},6 y(1),\{x(1),x(2),y(1),y(2),y(3)\}\right\}
\end{dmath*}

Here the Feynman parameter variables \(x_i\) and \(y_i\) are independent
from each other, i.e.~we have
\(\delta(1-x_1-x_2-x_3) \times \delta(1-y_1-y_2-y_3)\). This gives us
much more freedom when exploiting the Cheng-Wu theorem.

\begin{Shaded}
\begin{Highlighting}[]
\NormalTok{FCFeynmanParametrize}\OperatorTok{[}\NormalTok{intT}\OperatorTok{[[}\DecValTok{1}\OperatorTok{]],}\NormalTok{ intT}\OperatorTok{[[}\DecValTok{2}\OperatorTok{]],} \OperatorTok{\{}\NormalTok{p1}\OperatorTok{,}\NormalTok{ p3}\OperatorTok{\},}\NormalTok{ Indexed }\OtherTok{{-}\textgreater{}} \ConstantTok{True}\OperatorTok{,} 
\NormalTok{  FCReplaceD }\OtherTok{{-}\textgreater{}} \OperatorTok{\{}\FunctionTok{D} \OtherTok{{-}\textgreater{}} \DecValTok{4} \SpecialCharTok{{-}} \DecValTok{2}\NormalTok{ ep}\OperatorTok{\},}\NormalTok{ FinalSubstitutions }\OtherTok{{-}\textgreater{}} \OperatorTok{\{}\NormalTok{SPD}\OperatorTok{[}\FunctionTok{n}\OperatorTok{]} \OtherTok{{-}\textgreater{}} \DecValTok{1}\OperatorTok{,} \FunctionTok{m} \OtherTok{{-}\textgreater{}} \DecValTok{1}\OperatorTok{\},} \FunctionTok{Variables} \OtherTok{{-}\textgreater{}}\NormalTok{ intT}\OperatorTok{[[}\DecValTok{3}\OperatorTok{]]]}
\end{Highlighting}
\end{Shaded}

\begin{dmath*}\breakingcomma
\left\{y(1) \left(x(1) x(2) y(1)^2\right)^{3 \;\text{ep}-2} \left(y(1) \left(x(1) x(2)^2 y(1)^2+x(1)^2 x(2) y(1)^2+x(2) y(2)^2+x(1) y(3)^2+x(2) y(3)^2+2 x(2) y(2) y(3)\right)\right)^{-2 \;\text{ep}},\Gamma (2 \;\text{ep}),\{x(1),x(2),y(1),y(2),y(3)\}\right\}
\end{dmath*}

In the case that we need \texttt{U} and \texttt{F} polynomials in
addition to the normal output (e.g.~for HyperInt)

\begin{Shaded}
\begin{Highlighting}[]
\NormalTok{(SFAD}\OperatorTok{[\{\{}\DecValTok{0}\OperatorTok{,} \DecValTok{2}\SpecialCharTok{*}\NormalTok{k1 . }\FunctionTok{n}\OperatorTok{\}\}]}\SpecialCharTok{*}\NormalTok{SFAD}\OperatorTok{[\{\{}\DecValTok{0}\OperatorTok{,} \DecValTok{2}\SpecialCharTok{*}\NormalTok{k2 . }\FunctionTok{n}\OperatorTok{\}\}]}\SpecialCharTok{*}\NormalTok{SFAD}\OperatorTok{[\{}\NormalTok{k1}\OperatorTok{,} \FunctionTok{m}\SpecialCharTok{\^{}}\DecValTok{2}\OperatorTok{\}]}\SpecialCharTok{*}
\NormalTok{    SFAD}\OperatorTok{[\{}\NormalTok{k2}\OperatorTok{,} \FunctionTok{m}\SpecialCharTok{\^{}}\DecValTok{2}\OperatorTok{\}]}\SpecialCharTok{*}\NormalTok{SFAD}\OperatorTok{[\{}\NormalTok{k1 }\SpecialCharTok{{-}}\NormalTok{ k2}\OperatorTok{,} \FunctionTok{m}\SpecialCharTok{\^{}}\DecValTok{2}\OperatorTok{\}]}\NormalTok{) }
 
\FunctionTok{out} \ExtensionTok{=}\NormalTok{ FCFeynmanParametrize}\OperatorTok{[}\SpecialCharTok{\%}\OperatorTok{,} \OperatorTok{\{}\NormalTok{k1}\OperatorTok{,}\NormalTok{ k2}\OperatorTok{\},} \FunctionTok{Names} \OtherTok{{-}\textgreater{}} \FunctionTok{x}\OperatorTok{,}\NormalTok{ FCReplaceD }\OtherTok{{-}\textgreater{}} \OperatorTok{\{}\FunctionTok{D} \OtherTok{{-}\textgreater{}} \DecValTok{4} \SpecialCharTok{{-}} \DecValTok{2}\NormalTok{ Epsilon}\OperatorTok{\},} 
\NormalTok{   FCFeynmanPrepare }\OtherTok{{-}\textgreater{}} \ConstantTok{True}\OperatorTok{]}
\end{Highlighting}
\end{Shaded}

\begin{dmath*}\breakingcomma
\frac{1}{(\text{k1}^2-m^2+i \eta ) (\text{k2}^2-m^2+i \eta ) ((\text{k1}-\text{k2})^2-m^2+i \eta ) (2 (\text{k1}\cdot n)+i \eta ) (2 (\text{k2}\cdot n)+i \eta )}
\end{dmath*}

\begin{dmath*}\breakingcomma
\left\{(x(3) x(4)+x(5) x(4)+x(3) x(5))^{3 \varepsilon -1} \left(m^2 x(3) x(4)^2+m^2 x(3) x(5)^2+m^2 x(4) x(5)^2+m^2 x(3)^2 x(4)+m^2 x(3)^2 x(5)+m^2 x(4)^2 x(5)+3 m^2 x(3) x(4) x(5)+n^2 x(2)^2 x(3)+n^2 x(1)^2 x(4)+n^2 x(2)^2 x(4)+2 n^2 x(1) x(2) x(4)+n^2 x(1)^2 x(5)\right)^{-2 \varepsilon -1},-\Gamma (2 \varepsilon +1),\{x(1),x(2),x(3),x(4),x(5)\},\left\{x(3) x(4)+x(5) x(4)+x(3) x(5),m^2 x(3) x(4)^2+m^2 x(3) x(5)^2+m^2 x(4) x(5)^2+m^2 x(3)^2 x(4)+m^2 x(3)^2 x(5)+m^2 x(4)^2 x(5)+3 m^2 x(3) x(4) x(5)+n^2 x(2)^2 x(3)+n^2 x(1)^2 x(4)+n^2 x(2)^2 x(4)+2 n^2 x(1) x(2) x(4)+n^2 x(1)^2 x(5),\left(
\begin{array}{ccc}
 x(1) & \frac{1}{(2 (\text{k1}\cdot n)+i \eta )} & 1 \\
 x(2) & \frac{1}{(2 (\text{k2}\cdot n)+i \eta )} & 1 \\
 x(3) & \frac{1}{(\text{k1}^2-m^2+i \eta )} & 1 \\
 x(4) & \frac{1}{((\text{k1}-\text{k2})^2-m^2+i \eta )} & 1 \\
 x(5) & \frac{1}{(\text{k2}^2-m^2+i \eta )} & 1 \\
\end{array}
\right),\left(
\begin{array}{cc}
 x(3)+x(4) & -x(4) \\
 -x(4) & x(4)+x(5) \\
\end{array}
\right),\left\{x(1) \left(-n^{\text{FCGV}(\text{mu})}\right),x(2) \left(-n^{\text{FCGV}(\text{mu})}\right)\right\},-m^2 (x(3)+x(4)+x(5)),1,0\right\}\right\}
\end{dmath*}

From this output we can easily extract the integrand, its
\(x_i\)-independent prefactor and the two Symanzik polynomials

\begin{Shaded}
\begin{Highlighting}[]
\OperatorTok{\{}\NormalTok{integrand}\OperatorTok{,}\NormalTok{ pref}\OperatorTok{\}} \ExtensionTok{=} \FunctionTok{out}\OperatorTok{[[}\DecValTok{1}\NormalTok{ ;; }\DecValTok{2}\OperatorTok{]]} 
 
\OperatorTok{\{}\NormalTok{uPoly}\OperatorTok{,}\NormalTok{ fPoly}\OperatorTok{\}} \ExtensionTok{=} \FunctionTok{out}\OperatorTok{[[}\DecValTok{4}\OperatorTok{]][[}\DecValTok{1}\NormalTok{ ;; }\DecValTok{2}\OperatorTok{]]}
\end{Highlighting}
\end{Shaded}

\begin{dmath*}\breakingcomma
\left\{(x(3) x(4)+x(5) x(4)+x(3) x(5))^{3 \varepsilon -1} \left(m^2 x(3) x(4)^2+m^2 x(3) x(5)^2+m^2 x(4) x(5)^2+m^2 x(3)^2 x(4)+m^2 x(3)^2 x(5)+m^2 x(4)^2 x(5)+3 m^2 x(3) x(4) x(5)+n^2 x(2)^2 x(3)+n^2 x(1)^2 x(4)+n^2 x(2)^2 x(4)+2 n^2 x(1) x(2) x(4)+n^2 x(1)^2 x(5)\right)^{-2 \varepsilon -1},-\Gamma (2 \varepsilon +1)\right\}
\end{dmath*}

\begin{dmath*}\breakingcomma
\left\{x(3) x(4)+x(5) x(4)+x(3) x(5),m^2 x(3) x(4)^2+m^2 x(3) x(5)^2+m^2 x(4) x(5)^2+m^2 x(3)^2 x(4)+m^2 x(3)^2 x(5)+m^2 x(4)^2 x(5)+3 m^2 x(3) x(4) x(5)+n^2 x(2)^2 x(3)+n^2 x(1)^2 x(4)+n^2 x(2)^2 x(4)+2 n^2 x(1) x(2) x(4)+n^2 x(1)^2 x(5)\right\}
\end{dmath*}

Symbolic propagator powers are fully supported

\begin{Shaded}
\begin{Highlighting}[]
\NormalTok{SFAD}\OperatorTok{[\{}\FunctionTok{I} \FunctionTok{k}\OperatorTok{,} \DecValTok{0}\OperatorTok{,} \SpecialCharTok{{-}}\DecValTok{1}\SpecialCharTok{/}\DecValTok{2} \SpecialCharTok{+}\NormalTok{ ep}\OperatorTok{\},} \OperatorTok{\{}\FunctionTok{I}\NormalTok{ (}\FunctionTok{k} \SpecialCharTok{+} \FunctionTok{p}\NormalTok{)}\OperatorTok{,} \DecValTok{0}\OperatorTok{,} \DecValTok{1}\OperatorTok{\},}\NormalTok{ EtaSign }\OtherTok{{-}\textgreater{}} \SpecialCharTok{{-}}\DecValTok{1}\OperatorTok{]} 
 
\NormalTok{v1 }\ExtensionTok{=}\NormalTok{ FCFeynmanParametrize}\OperatorTok{[}\SpecialCharTok{\%}\OperatorTok{,} \OperatorTok{\{}\FunctionTok{k}\OperatorTok{\},} \FunctionTok{Names} \OtherTok{{-}\textgreater{}} \FunctionTok{x}\OperatorTok{,}\NormalTok{ FCReplaceD }\OtherTok{{-}\textgreater{}} \OperatorTok{\{}\FunctionTok{D} \OtherTok{{-}\textgreater{}} \DecValTok{4} \SpecialCharTok{{-}} \DecValTok{2}\NormalTok{ ep}\OperatorTok{\},} 
\NormalTok{   FinalSubstitutions }\OtherTok{{-}\textgreater{}} \OperatorTok{\{}\NormalTok{SPD}\OperatorTok{[}\FunctionTok{p}\OperatorTok{]} \OtherTok{{-}\textgreater{}} \DecValTok{1}\OperatorTok{\}]}
\end{Highlighting}
\end{Shaded}

\begin{dmath*}\breakingcomma
\frac{1}{(-k^2-i \eta )^{\text{ep}-\frac{1}{2}}.(-(k+p)^2-i \eta )}
\end{dmath*}

\begin{dmath*}\breakingcomma
\left\{(-x(1)-x(2))^{3 \;\text{ep}-\frac{7}{2}} x(2)^{\text{ep}-\frac{3}{2}} (-x(1) x(2))^{\frac{3}{2}-2 \;\text{ep}},\frac{(-1)^{\text{ep}+\frac{1}{2}} \Gamma \left(2 \;\text{ep}-\frac{3}{2}\right)}{\Gamma \left(\text{ep}-\frac{1}{2}\right)},\{x(1),x(2)\}\right\}
\end{dmath*}

An alternative representation for symbolic powers can be obtained using
the option \texttt{SplitSymbolicPowers}

\begin{Shaded}
\begin{Highlighting}[]
\NormalTok{SFAD}\OperatorTok{[\{}\FunctionTok{I} \FunctionTok{k}\OperatorTok{,} \DecValTok{0}\OperatorTok{,} \SpecialCharTok{{-}}\DecValTok{1}\SpecialCharTok{/}\DecValTok{2} \SpecialCharTok{+}\NormalTok{ ep}\OperatorTok{\},} \OperatorTok{\{}\FunctionTok{I}\NormalTok{ (}\FunctionTok{k} \SpecialCharTok{+} \FunctionTok{p}\NormalTok{)}\OperatorTok{,} \DecValTok{0}\OperatorTok{,} \DecValTok{1}\OperatorTok{\},}\NormalTok{ EtaSign }\OtherTok{{-}\textgreater{}} \SpecialCharTok{{-}}\DecValTok{1}\OperatorTok{]} 
 
\NormalTok{v2 }\ExtensionTok{=}\NormalTok{ FCFeynmanParametrize}\OperatorTok{[}\SpecialCharTok{\%}\OperatorTok{,} \OperatorTok{\{}\FunctionTok{k}\OperatorTok{\},} \FunctionTok{Names} \OtherTok{{-}\textgreater{}} \FunctionTok{x}\OperatorTok{,}\NormalTok{ FCReplaceD }\OtherTok{{-}\textgreater{}} \OperatorTok{\{}\FunctionTok{D} \OtherTok{{-}\textgreater{}} \DecValTok{4} \SpecialCharTok{{-}} \DecValTok{2}\NormalTok{ ep}\OperatorTok{\},} 
\NormalTok{   FinalSubstitutions }\OtherTok{{-}\textgreater{}} \OperatorTok{\{}\NormalTok{SPD}\OperatorTok{[}\FunctionTok{p}\OperatorTok{]} \OtherTok{{-}\textgreater{}} \DecValTok{1}\OperatorTok{\},}\NormalTok{ SplitSymbolicPowers }\OtherTok{{-}\textgreater{}} \ConstantTok{True}\OperatorTok{]}
\end{Highlighting}
\end{Shaded}

\begin{dmath*}\breakingcomma
\frac{1}{(-k^2-i \eta )^{\text{ep}-\frac{1}{2}}.(-(k+p)^2-i \eta )}
\end{dmath*}

\begin{dmath*}\breakingcomma
\left\{x(2)^{\text{ep}-\frac{1}{2}} \left(\left(\frac{1}{2} (1-2 \;\text{ep})+\frac{1}{2} (4-2 \;\text{ep})-1\right) x(1) (-x(1)-x(2))^{3 \;\text{ep}-\frac{7}{2}} (-x(1) x(2))^{\frac{1}{2}-2 \;\text{ep}}+\left(2 \;\text{ep}+\frac{1}{2} (2 \;\text{ep}-1)-3\right) (-x(1)-x(2))^{3 \;\text{ep}-\frac{9}{2}} (-x(1) x(2))^{\frac{3}{2}-2 \;\text{ep}}\right),\frac{(-1)^{\text{ep}+\frac{1}{2}} \Gamma \left(2 \;\text{ep}-\frac{3}{2}\right)}{\Gamma \left(\text{ep}+\frac{1}{2}\right)},\{x(1),x(2)\}\right\}
\end{dmath*}

Even though the parametric integrals evaluate to different values, the
product of the integral and its prefactor remains the same

\begin{Shaded}
\begin{Highlighting}[]
\FunctionTok{Integrate}\OperatorTok{[}\FunctionTok{Normal}\OperatorTok{[}\FunctionTok{Series}\OperatorTok{[}\NormalTok{v1}\OperatorTok{[[}\DecValTok{1}\OperatorTok{]]} \OtherTok{/.} \FunctionTok{x}\OperatorTok{[}\DecValTok{1}\OperatorTok{]} \OtherTok{{-}\textgreater{}} \DecValTok{1}\OperatorTok{,} \OperatorTok{\{}\NormalTok{ep}\OperatorTok{,} \DecValTok{0}\OperatorTok{,} \DecValTok{0}\OperatorTok{\}]]} \OtherTok{/.} \FunctionTok{x}\OperatorTok{[}\DecValTok{1}\OperatorTok{]} \OtherTok{{-}\textgreater{}} \DecValTok{1}\OperatorTok{,} \OperatorTok{\{}\FunctionTok{x}\OperatorTok{[}\DecValTok{2}\OperatorTok{],} \DecValTok{0}\OperatorTok{,} \FunctionTok{Infinity}\OperatorTok{\}]} 
 
\NormalTok{Normal@Series}\OperatorTok{[}\NormalTok{v1}\OperatorTok{[[}\DecValTok{2}\OperatorTok{]]} \SpecialCharTok{\%}\OperatorTok{,} \OperatorTok{\{}\NormalTok{ep}\OperatorTok{,} \DecValTok{0}\OperatorTok{,} \DecValTok{0}\OperatorTok{\}]}
\end{Highlighting}
\end{Shaded}

\begin{dmath*}\breakingcomma
\frac{2}{5}
\end{dmath*}

\begin{dmath*}\breakingcomma
-\frac{4 i}{15}
\end{dmath*}

\begin{Shaded}
\begin{Highlighting}[]
\FunctionTok{Integrate}\OperatorTok{[}\FunctionTok{Normal}\OperatorTok{[}\FunctionTok{Series}\OperatorTok{[}\NormalTok{v2}\OperatorTok{[[}\DecValTok{1}\OperatorTok{]]} \OtherTok{/.} \FunctionTok{x}\OperatorTok{[}\DecValTok{1}\OperatorTok{]} \OtherTok{{-}\textgreater{}} \DecValTok{1}\OperatorTok{,} \OperatorTok{\{}\NormalTok{ep}\OperatorTok{,} \DecValTok{0}\OperatorTok{,} \DecValTok{0}\OperatorTok{\}]]} \OtherTok{/.} \FunctionTok{x}\OperatorTok{[}\DecValTok{1}\OperatorTok{]} \OtherTok{{-}\textgreater{}} \DecValTok{1}\OperatorTok{,} \OperatorTok{\{}\FunctionTok{x}\OperatorTok{[}\DecValTok{2}\OperatorTok{],} \DecValTok{0}\OperatorTok{,} \FunctionTok{Infinity}\OperatorTok{\}]} 
 
\NormalTok{Normal@Series}\OperatorTok{[}\NormalTok{v2}\OperatorTok{[[}\DecValTok{2}\OperatorTok{]]} \SpecialCharTok{\%}\OperatorTok{,} \OperatorTok{\{}\NormalTok{ep}\OperatorTok{,} \DecValTok{0}\OperatorTok{,} \DecValTok{0}\OperatorTok{\}]}
\end{Highlighting}
\end{Shaded}

\begin{dmath*}\breakingcomma
-\frac{1}{5}
\end{dmath*}

\begin{dmath*}\breakingcomma
-\frac{4 i}{15}
\end{dmath*}

Calculate the simplest divergent triangle integral from
(https://qcdloop.fnal.gov/tridiv1.pdf)

\begin{Shaded}
\begin{Highlighting}[]
\NormalTok{FCClearScalarProducts}\OperatorTok{[]}\NormalTok{;}
\NormalTok{SPD}\OperatorTok{[}\FunctionTok{r}\OperatorTok{]} \ExtensionTok{=} \DecValTok{0}\NormalTok{;}
\NormalTok{SPD}\OperatorTok{[}\FunctionTok{s}\OperatorTok{]} \ExtensionTok{=} \DecValTok{0}\NormalTok{;}
\NormalTok{SPD}\OperatorTok{[}\FunctionTok{r}\OperatorTok{,} \FunctionTok{s}\OperatorTok{]} \ExtensionTok{=} \SpecialCharTok{{-}}\DecValTok{1}\SpecialCharTok{/}\DecValTok{2}\NormalTok{;}
\NormalTok{int }\ExtensionTok{=}\NormalTok{ FAD}\OperatorTok{[\{}\FunctionTok{q}\OperatorTok{,} \DecValTok{0}\OperatorTok{\},} \OperatorTok{\{}\FunctionTok{q} \SpecialCharTok{{-}} \FunctionTok{r}\OperatorTok{,} \DecValTok{0}\OperatorTok{\},} \OperatorTok{\{}\FunctionTok{q} \SpecialCharTok{{-}} \FunctionTok{s}\OperatorTok{,} \DecValTok{0}\OperatorTok{\}]}
\end{Highlighting}
\end{Shaded}

\begin{dmath*}\breakingcomma
\frac{1}{q^2.(q-r)^2.(q-s)^2}
\end{dmath*}

\begin{Shaded}
\begin{Highlighting}[]
\NormalTok{ToPaVe}\OperatorTok{[}\NormalTok{int}\OperatorTok{,} \FunctionTok{q}\OperatorTok{]}
\end{Highlighting}
\end{Shaded}

\begin{dmath*}\breakingcomma
i \pi ^2 \;\text{C}_0(0,0,1,0,0,0)
\end{dmath*}

\begin{Shaded}
\begin{Highlighting}[]
\NormalTok{fp }\ExtensionTok{=}\NormalTok{ FCFeynmanParametrize}\OperatorTok{[}\NormalTok{int}\OperatorTok{,} \OperatorTok{\{}\FunctionTok{q}\OperatorTok{\},} \FunctionTok{Names} \OtherTok{{-}\textgreater{}} \FunctionTok{x}\OperatorTok{,}\NormalTok{ FCReplaceD }\OtherTok{{-}\textgreater{}} \OperatorTok{\{}\FunctionTok{D} \OtherTok{{-}\textgreater{}} \DecValTok{4} \SpecialCharTok{{-}} \DecValTok{2}\NormalTok{ ep}\OperatorTok{\},}\NormalTok{ FeynmanIntegralPrefactor }\OtherTok{{-}\textgreater{}} \StringTok{"LoopTools"}\OperatorTok{]}
\end{Highlighting}
\end{Shaded}

\begin{dmath*}\breakingcomma
\left\{(-x(2) x(3))^{-\text{ep}-1} (x(1)+x(2)+x(3))^{2 \;\text{ep}-1},-\frac{\Gamma (1-2 \;\text{ep})}{\Gamma (1-\text{ep})^2},\{x(1),x(2),x(3)\}\right\}
\end{dmath*}

\begin{Shaded}
\begin{Highlighting}[]
\NormalTok{intRaw }\ExtensionTok{=} \FunctionTok{Integrate}\OperatorTok{[}\NormalTok{fp}\OperatorTok{[[}\DecValTok{1}\OperatorTok{]]} \OtherTok{/.} \FunctionTok{x}\OperatorTok{[}\DecValTok{2}\OperatorTok{]} \OtherTok{{-}\textgreater{}} \DecValTok{1}\OperatorTok{,} \OperatorTok{\{}\FunctionTok{x}\OperatorTok{[}\DecValTok{1}\OperatorTok{],} \DecValTok{0}\OperatorTok{,} \FunctionTok{Infinity}\OperatorTok{\},} \FunctionTok{Assumptions} \OtherTok{{-}\textgreater{}} \OperatorTok{\{}\NormalTok{ep \textless{} }\DecValTok{0}\OperatorTok{,} \FunctionTok{x}\OperatorTok{[}\DecValTok{3}\OperatorTok{]}\NormalTok{ \textgreater{}}\ExtensionTok{=} \DecValTok{0}\OperatorTok{\}]}
\end{Highlighting}
\end{Shaded}

\begin{dmath*}\breakingcomma
-\frac{(-x(3))^{-\text{ep}-1} (x(3)+1)^{2 \;\text{ep}}}{2 \;\text{ep}}
\end{dmath*}

Reintroduce the correct \(i \eta\)-prescription to get the imaginary
part right

\begin{Shaded}
\begin{Highlighting}[]
\NormalTok{intRes }\ExtensionTok{=} \FunctionTok{Integrate}\OperatorTok{[}\NormalTok{intRaw}\OperatorTok{,} \OperatorTok{\{}\FunctionTok{x}\OperatorTok{[}\DecValTok{3}\OperatorTok{],} \DecValTok{0}\OperatorTok{,} \FunctionTok{Infinity}\OperatorTok{\},} \FunctionTok{Assumptions} \OtherTok{{-}\textgreater{}} \OperatorTok{\{}\NormalTok{ep \textless{} }\DecValTok{0}\OperatorTok{\}]} \OtherTok{/.}\NormalTok{ (}\SpecialCharTok{{-}}\DecValTok{1}\NormalTok{)}\SpecialCharTok{\^{}}\NormalTok{(}\SpecialCharTok{{-}}\NormalTok{ep) }\OtherTok{{-}\textgreater{}}\NormalTok{ (}\SpecialCharTok{{-}}\DecValTok{1} \SpecialCharTok{{-}} \FunctionTok{I}\NormalTok{ eta)}\SpecialCharTok{\^{}}\NormalTok{(}\SpecialCharTok{{-}}\NormalTok{ep)}
\end{Highlighting}
\end{Shaded}

\begin{dmath*}\breakingcomma
\frac{(-1-i \;\text{eta})^{-\text{ep}} \Gamma (-\text{ep})^2}{2 \;\text{ep} \Gamma (-2 \;\text{ep})}
\end{dmath*}

\begin{Shaded}
\begin{Highlighting}[]
\NormalTok{res }\ExtensionTok{=}\NormalTok{ (}\FunctionTok{Series}\OperatorTok{[}\NormalTok{fp}\OperatorTok{[[}\DecValTok{2}\OperatorTok{]]}\NormalTok{ intRes}\OperatorTok{,} \OperatorTok{\{}\NormalTok{ep}\OperatorTok{,} \DecValTok{0}\OperatorTok{,} \DecValTok{0}\OperatorTok{\}]} \SpecialCharTok{//} \FunctionTok{Normal}\NormalTok{) }\OtherTok{/.} \FunctionTok{Log}\OperatorTok{[}\SpecialCharTok{{-}}\DecValTok{1} \SpecialCharTok{{-}} \FunctionTok{I}\NormalTok{ eta}\OperatorTok{]} \OtherTok{{-}\textgreater{}} \FunctionTok{Log}\OperatorTok{[}\DecValTok{1}\OperatorTok{]} \SpecialCharTok{{-}} \FunctionTok{I} \FunctionTok{Pi}
\end{Highlighting}
\end{Shaded}

\begin{dmath*}\breakingcomma
\frac{1}{\text{ep}^2}+\frac{i \pi }{\text{ep}}-\frac{\pi ^2}{2}
\end{dmath*}

Compare to the known result

\begin{Shaded}
\begin{Highlighting}[]
\NormalTok{resLit }\ExtensionTok{=} \FunctionTok{Series}\OperatorTok{[}\NormalTok{ScaleMu}\SpecialCharTok{\^{}}\NormalTok{(}\DecValTok{2}\NormalTok{ ep)}\SpecialCharTok{/}\NormalTok{ep}\SpecialCharTok{\^{}}\DecValTok{2} \DecValTok{1}\SpecialCharTok{/}\NormalTok{pp}\SpecialCharTok{\^{}}\DecValTok{2}\NormalTok{ (}\SpecialCharTok{{-}}\NormalTok{pp }\SpecialCharTok{{-}} \FunctionTok{I}\NormalTok{ eta)}\SpecialCharTok{\^{}}\NormalTok{(}\SpecialCharTok{{-}}\NormalTok{ep)}\OperatorTok{,} \OperatorTok{\{}\NormalTok{ep}\OperatorTok{,} \DecValTok{0}\OperatorTok{,} \DecValTok{0}\OperatorTok{\}]} \OtherTok{/.} \FunctionTok{Log}\OperatorTok{[}\SpecialCharTok{{-}}\NormalTok{pp }\SpecialCharTok{{-}} \FunctionTok{I}\NormalTok{ eta}\OperatorTok{]} \OtherTok{{-}\textgreater{}} \FunctionTok{Log}\OperatorTok{[}\NormalTok{pp}\OperatorTok{]} \SpecialCharTok{{-}} \FunctionTok{I} \FunctionTok{Pi} \SpecialCharTok{//} \FunctionTok{Normal}
\end{Highlighting}
\end{Shaded}

\begin{dmath*}\breakingcomma
\frac{1}{\text{ep}^2 \;\text{pp}^2}+\frac{2 \log (\mu )-\log (\text{pp})+i \pi }{\text{ep} \;\text{pp}^2}+\frac{4 \log ^2(\mu )-4 \log (\mu ) (\log (\text{pp})-i \pi )+(\log (\text{pp})-i \pi )^2}{2 \;\text{pp}^2}
\end{dmath*}

\begin{Shaded}
\begin{Highlighting}[]
\NormalTok{(res }\SpecialCharTok{{-}}\NormalTok{ resLit) }\OtherTok{/.}\NormalTok{ pp }\SpecialCharTok{|}\NormalTok{ ScaleMu }\OtherTok{{-}\textgreater{}} \DecValTok{1}
\end{Highlighting}
\end{Shaded}

\begin{dmath*}\breakingcomma
0
\end{dmath*}

\hypertarget{lee-pomeransky-representation}{%
\subsubsection{Lee-Pomeransky
representation}\label{lee-pomeransky-representation}}

1-loop tadpole

\begin{Shaded}
\begin{Highlighting}[]
\NormalTok{FCFeynmanParametrize}\OperatorTok{[}\NormalTok{FAD}\OperatorTok{[\{}\FunctionTok{q}\OperatorTok{,} \FunctionTok{m}\OperatorTok{\}],} \OperatorTok{\{}\FunctionTok{q}\OperatorTok{\},} \FunctionTok{Names} \OtherTok{{-}\textgreater{}} \FunctionTok{x}\OperatorTok{,} \FunctionTok{Method} \OtherTok{{-}\textgreater{}} \StringTok{"Lee{-}Pomeransky"}\OperatorTok{]}
\end{Highlighting}
\end{Shaded}

\begin{dmath*}\breakingcomma
\left\{\left(m^2 x(1)^2+x(1)\right)^{-D/2},-\frac{\Gamma \left(\frac{D}{2}\right)}{\Gamma (D-1)},\{x(1)\}\right\}
\end{dmath*}

Massless 1-loop 2-point function

\begin{Shaded}
\begin{Highlighting}[]
\NormalTok{FCFeynmanParametrize}\OperatorTok{[}\NormalTok{FAD}\OperatorTok{[}\FunctionTok{q}\OperatorTok{,} \FunctionTok{q} \SpecialCharTok{{-}} \FunctionTok{p}\OperatorTok{],} \OperatorTok{\{}\FunctionTok{q}\OperatorTok{\},} \FunctionTok{Names} \OtherTok{{-}\textgreater{}} \FunctionTok{x}\OperatorTok{,} \FunctionTok{Method} \OtherTok{{-}\textgreater{}} \StringTok{"Lee{-}Pomeransky"}\OperatorTok{]}
\end{Highlighting}
\end{Shaded}

\begin{dmath*}\breakingcomma
\left\{\left(-p^2 x(2) x(1)+x(1)+x(2)\right)^{-D/2},\frac{\Gamma \left(\frac{D}{2}\right)}{\Gamma (D-2)},\{x(1),x(2)\}\right\}
\end{dmath*}

2-loop self-energy with 3 massive lines and two eikonal propagators

\begin{Shaded}
\begin{Highlighting}[]
\NormalTok{FCFeynmanParametrize}\OperatorTok{[\{}\NormalTok{SFAD}\OperatorTok{[\{}\NormalTok{ p1}\OperatorTok{,} \FunctionTok{m}\SpecialCharTok{\^{}}\DecValTok{2}\OperatorTok{\}],}\NormalTok{ SFAD}\OperatorTok{[\{}\NormalTok{ p3}\OperatorTok{,} \FunctionTok{m}\SpecialCharTok{\^{}}\DecValTok{2}\OperatorTok{\}],} 
\NormalTok{   SFAD}\OperatorTok{[\{}\NormalTok{(p3 }\SpecialCharTok{{-}}\NormalTok{ p1)}\OperatorTok{,} \FunctionTok{m}\SpecialCharTok{\^{}}\DecValTok{2}\OperatorTok{\}],}\NormalTok{ SFAD}\OperatorTok{[\{\{}\DecValTok{0}\OperatorTok{,} \DecValTok{2}\NormalTok{ p1 . }\FunctionTok{n}\OperatorTok{\}\}],}\NormalTok{ SFAD}\OperatorTok{[\{\{}\DecValTok{0}\OperatorTok{,} \DecValTok{2}\NormalTok{ p3 . }\FunctionTok{n}\OperatorTok{\}\}]\},} \OperatorTok{\{}\NormalTok{p1}\OperatorTok{,}\NormalTok{ p3}\OperatorTok{\},} 
  \FunctionTok{Names} \OtherTok{{-}\textgreater{}} \FunctionTok{x}\OperatorTok{,} \FunctionTok{Method} \OtherTok{{-}\textgreater{}} \StringTok{"Lee{-}Pomeransky"}\OperatorTok{,}\NormalTok{ FCReplaceD }\OtherTok{{-}\textgreater{}} \OperatorTok{\{}\FunctionTok{D} \OtherTok{{-}\textgreater{}} \DecValTok{4} \SpecialCharTok{{-}} \DecValTok{2}\NormalTok{ ep}\OperatorTok{\},} 
\NormalTok{  FinalSubstitutions }\OtherTok{{-}\textgreater{}} \OperatorTok{\{}\NormalTok{SPD}\OperatorTok{[}\FunctionTok{n}\OperatorTok{]} \OtherTok{{-}\textgreater{}} \DecValTok{1}\OperatorTok{,} \FunctionTok{m} \OtherTok{{-}\textgreater{}} \DecValTok{1}\OperatorTok{\}]}
\end{Highlighting}
\end{Shaded}

\begin{dmath*}\breakingcomma
\left\{\left(x(4) x(1)^2+x(5) x(1)^2+2 x(2) x(5) x(1)+x(3) x(4)^2+x(3) x(5)^2+x(4) x(5)^2+x(2)^2 x(3)+x(3)^2 x(4)+x(3) x(4)+x(2)^2 x(5)+x(3)^2 x(5)+x(4)^2 x(5)+x(3) x(5)+3 x(3) x(4) x(5)+x(4) x(5)\right)^{\text{ep}-2},-\frac{\Gamma (2-\text{ep})}{\Gamma (1-3 \;\text{ep})},\{x(1),x(2),x(3),x(4),x(5)\}\right\}
\end{dmath*}
\end{document}
