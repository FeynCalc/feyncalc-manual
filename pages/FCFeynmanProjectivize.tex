% !TeX program = pdflatex
% !TeX root = FCFeynmanProjectivize.tex

\documentclass[../FeynCalcManual.tex]{subfiles}
\begin{document}
\hypertarget{fcfeynmanprojectivize}{
\section{FCFeynmanProjectivize}\label{fcfeynmanprojectivize}\index{FCFeynmanProjectivize}}

\texttt{FCFeynmanProjectivize[\allowbreak{}int,\ \allowbreak{}x]} checks
if the given Feynman parameter integral (without prefactors) depending
on x{[}1{]}, x{[}2{]}, \ldots{} is a projective form. If this is not the
case, the integral will be projectivized.

Projectivity is a necessary condition for computing the integral with
the aid of the Cheng-Wu theorem

\subsection{See also}

\hyperlink{toc}{Overview},
\hyperlink{fcfeynmanparametrize}{FCFeynmanParametrize},
\hyperlink{fcfeynmanprepare}{FCFeynmanPrepare},
\hyperlink{fcfeynmanprojectiveq}{FCFeynmanProjectiveQ}.

\subsection{Examples}

\begin{Shaded}
\begin{Highlighting}[]
\NormalTok{int }\ExtensionTok{=}\NormalTok{ SFAD}\OperatorTok{[\{}\NormalTok{p3}\OperatorTok{,}\NormalTok{ mg}\SpecialCharTok{\^{}}\DecValTok{2}\OperatorTok{\}]}\NormalTok{ SFAD}\OperatorTok{[\{}\NormalTok{p3 }\SpecialCharTok{{-}}\NormalTok{ p1}\OperatorTok{,}\NormalTok{ mg}\SpecialCharTok{\^{}}\DecValTok{2}\OperatorTok{\}]}\NormalTok{ SFAD}\OperatorTok{[\{\{}\DecValTok{0}\OperatorTok{,} \SpecialCharTok{{-}}\DecValTok{2}\NormalTok{ p1 . }\FunctionTok{q}\OperatorTok{\}\}]}
\end{Highlighting}
\end{Shaded}

\begin{dmath*}\breakingcomma
\frac{1}{(\text{p3}^2-\text{mg}^2+i \eta ) ((\text{p3}-\text{p1})^2-\text{mg}^2+i \eta ) (-2 (\text{p1}\cdot q)+i \eta )}
\end{dmath*}

\begin{Shaded}
\begin{Highlighting}[]
\NormalTok{fp }\ExtensionTok{=}\NormalTok{ FCFeynmanParametrize}\OperatorTok{[}\NormalTok{int}\OperatorTok{,} \OperatorTok{\{}\NormalTok{p1}\OperatorTok{,}\NormalTok{ p3}\OperatorTok{\},} \FunctionTok{Names} \OtherTok{{-}\textgreater{}} \FunctionTok{x}\OperatorTok{,}\NormalTok{ Indexed }\OtherTok{{-}\textgreater{}} \ConstantTok{True}\OperatorTok{,}\NormalTok{ FCReplaceD }\OtherTok{{-}\textgreater{}} \OperatorTok{\{}\FunctionTok{D} \OtherTok{{-}\textgreater{}} \DecValTok{4} \SpecialCharTok{{-}} \DecValTok{2}\NormalTok{ ep}\OperatorTok{\},} 
   \FunctionTok{Simplify} \OtherTok{{-}\textgreater{}} \ConstantTok{True}\OperatorTok{,} \FunctionTok{Assumptions} \OtherTok{{-}\textgreater{}} \OperatorTok{\{}\NormalTok{mg \textgreater{} }\DecValTok{0}\OperatorTok{,}\NormalTok{ ep \textgreater{} }\DecValTok{0}\OperatorTok{\},}\NormalTok{ FinalSubstitutions }\OtherTok{{-}\textgreater{}} \OperatorTok{\{}\NormalTok{SPD}\OperatorTok{[}\FunctionTok{q}\OperatorTok{]} \OtherTok{{-}\textgreater{}}\NormalTok{ qq}\OperatorTok{,}\NormalTok{ mg}\SpecialCharTok{\^{}}\DecValTok{2} \OtherTok{{-}\textgreater{}}\NormalTok{ mg2}\OperatorTok{\}]}
\end{Highlighting}
\end{Shaded}

\begin{dmath*}\breakingcomma
\left\{(x(2) x(3))^{3 \;\text{ep}-3} \left((x(2)+x(3)) \left(\text{mg2} x(2) x(3)+\text{qq} x(1)^2\right)\right)^{1-2 \;\text{ep}},-\Gamma (2 \;\text{ep}-1),\{x(1),x(2),x(3)\}\right\}
\end{dmath*}

\begin{Shaded}
\begin{Highlighting}[]
\NormalTok{FCFeynmanProjectivize}\OperatorTok{[}\NormalTok{fp}\OperatorTok{[[}\DecValTok{1}\OperatorTok{]],} \FunctionTok{x}\OperatorTok{]}
\end{Highlighting}
\end{Shaded}

\begin{dmath*}\breakingcomma
\text{FCFeynmanProjectivize: The integral is already projective, no further transformations are required.}
\end{dmath*}

\begin{dmath*}\breakingcomma
(x(2) x(3))^{3 \;\text{ep}-3} \left((x(2)+x(3)) \left(\text{mg2} x(2) x(3)+\text{qq} x(1)^2\right)\right)^{1-2 \;\text{ep}}
\end{dmath*}

\begin{Shaded}
\begin{Highlighting}[]
\NormalTok{FCFeynmanProjectivize}\OperatorTok{[}\NormalTok{(}\FunctionTok{x}\OperatorTok{[}\DecValTok{1}\OperatorTok{]} \SpecialCharTok{+} \FunctionTok{x}\OperatorTok{[}\DecValTok{2}\OperatorTok{]}\NormalTok{)}\SpecialCharTok{\^{}}\NormalTok{(}\SpecialCharTok{{-}}\DecValTok{2} \SpecialCharTok{+} \DecValTok{2}\SpecialCharTok{*}\NormalTok{ep)}\SpecialCharTok{/}\NormalTok{(mb2}\SpecialCharTok{*}\NormalTok{(}\FunctionTok{x}\OperatorTok{[}\DecValTok{1}\OperatorTok{]}\SpecialCharTok{\^{}}\DecValTok{2} \SpecialCharTok{+} \FunctionTok{x}\OperatorTok{[}\DecValTok{1}\OperatorTok{]}\SpecialCharTok{*}\FunctionTok{x}\OperatorTok{[}\DecValTok{2}\OperatorTok{]} \SpecialCharTok{+} 
        \FunctionTok{x}\OperatorTok{[}\DecValTok{2}\OperatorTok{]}\SpecialCharTok{\^{}}\DecValTok{2}\NormalTok{))}\SpecialCharTok{\^{}}\NormalTok{ep}\OperatorTok{,} \FunctionTok{x}\OperatorTok{]}
\end{Highlighting}
\end{Shaded}

\begin{dmath*}\breakingcomma
\text{FCFeynmanProjectivize: The integral is already projective, no further transformations are required.}
\end{dmath*}

\begin{dmath*}\breakingcomma
(x(1)+x(2))^{2 \;\text{ep}-2} \left(\text{mb2} \left(x(1)^2+x(2) x(1)+x(2)^2\right)\right)^{-\text{ep}}
\end{dmath*}

Feynman parametrizations derived from propagator representations should
be projective in most cases. However, arbitrary Feynman parameter
integrals do not necessarily have this property.

\begin{Shaded}
\begin{Highlighting}[]
\NormalTok{FCFeynmanProjectivize}\OperatorTok{[}\FunctionTok{x}\OperatorTok{[}\DecValTok{1}\OperatorTok{]}\SpecialCharTok{\^{}}\NormalTok{(}\FunctionTok{x} \SpecialCharTok{{-}} \DecValTok{1}\NormalTok{) (}\FunctionTok{x}\OperatorTok{[}\DecValTok{2}\OperatorTok{]}\NormalTok{)}\SpecialCharTok{\^{}}\NormalTok{(}\FunctionTok{y} \SpecialCharTok{{-}} \DecValTok{1}\NormalTok{)}\OperatorTok{,} \FunctionTok{x}\OperatorTok{]}
\end{Highlighting}
\end{Shaded}

\begin{dmath*}\breakingcomma
\text{FCFeynmanProjectivize: The integral is not projective, trying to projectivize.}
\end{dmath*}

\begin{dmath*}\breakingcomma
\text{FCFeynmanProjectivize: Projective transformation successful, the integral is now projective.}
\end{dmath*}

\begin{dmath*}\breakingcomma
\frac{\left(\frac{x(1)}{x(1)+x(2)}\right)^{x-1} \left(\frac{x(2)}{x(1)+x(2)}\right)^{y-1}}{(x(1)+x(2))^2}
\end{dmath*}
\end{document}
