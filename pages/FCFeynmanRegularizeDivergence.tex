% !TeX program = pdflatex
% !TeX root = FCFeynmanRegularizeDivergence.tex

\documentclass[../FeynCalcManual.tex]{subfiles}
\begin{document}
\hypertarget{fcfeynmanregularizedivergence}{
\section{FCFeynmanRegularizeDivergence}\label{fcfeynmanregularizedivergence}\index{FCFeynmanRegularizeDivergence}}

\texttt{FCFeynmanRegularizeDivergence[\allowbreak{}exp,\ \allowbreak{}div]}
regularizes the divergence \texttt{div} in the Feynman parametric
integral \texttt{exp}. Provided that all divergences have been
regularized in this fashion, upon expanding the integrand around
\(\varepsilon = 0\) one can safely integrate in the Feynman parameters.

This function uses the method of analytic regularization introduced by
Erik Panzer in \href{https://arxiv.org/abs/1403.3385}{1403.3385},
\href{https://arxiv.org/abs/1401.4361}{1401.4361} and
\href{https://arxiv.org/abs/1506.07243}{1506.07243}.

Its current implementation is very much based on the code of the
\texttt{dimregPartial} routine from the Maple package
\href{https://bitbucket.org/PanzerErik/hyperint/}{HyperInt} by Erik
Panzer.

Here \texttt{div} must be of the form
\texttt{\{\allowbreak{}\{\allowbreak{}x[\allowbreak{}i],\ \allowbreak{}x[\allowbreak{}j],\ \allowbreak{}...\},\ \allowbreak{}\{\allowbreak{}x[\allowbreak{}k],\ \allowbreak{}x[\allowbreak{}l],\ \allowbreak{}...\},\ \allowbreak{}sdd\}},
where
\texttt{\{\allowbreak{}x[\allowbreak{}i],\ \allowbreak{}x[\allowbreak{}j],\ \allowbreak{}...\}}
need to approach zero, while
\texttt{\{\allowbreak{}x[\allowbreak{}k],\ \allowbreak{}x[\allowbreak{}l],\ \allowbreak{}...\}}
must tend towards infinity to generate the superficial degree of
divergence \texttt{sdd}.

\subsection{See also}

\hyperlink{toc}{Overview},
\hyperlink{fcfeynmanparametrize}{FCFeynmanParametrize},
\hyperlink{fcfeynmanprojectivize}{FCFeynmanProjectivize},
\hyperlink{fcfeynmanfinddivergences}{FCFeynmanFindDivergences}.

\subsection{Examples}

\begin{Shaded}
\begin{Highlighting}[]
\NormalTok{int }\ExtensionTok{=}\NormalTok{ SFAD}\OperatorTok{[}\FunctionTok{l}\OperatorTok{,} \FunctionTok{k} \SpecialCharTok{+} \FunctionTok{l}\OperatorTok{,} \OperatorTok{\{\{}\FunctionTok{k}\OperatorTok{,} \SpecialCharTok{{-}}\DecValTok{2} \FunctionTok{k}\NormalTok{ . }\FunctionTok{q}\OperatorTok{\}\}]}
\NormalTok{fpar }\ExtensionTok{=}\NormalTok{ FCFeynmanParametrize}\OperatorTok{[}\NormalTok{int}\OperatorTok{,} \OperatorTok{\{}\FunctionTok{k}\OperatorTok{,} \FunctionTok{l}\OperatorTok{\},} \FunctionTok{Names} \OtherTok{{-}\textgreater{}} \FunctionTok{x}\OperatorTok{,}\NormalTok{ FCReplaceD }\OtherTok{{-}\textgreater{}} \OperatorTok{\{}\FunctionTok{D} \OtherTok{{-}\textgreater{}} \DecValTok{4} \SpecialCharTok{{-}} \DecValTok{2}\NormalTok{ Epsilon}\OperatorTok{\}]}
\end{Highlighting}
\end{Shaded}

\begin{dmath*}\breakingcomma
\frac{1}{(l^2+i \eta ).((k+l)^2+i \eta ).(k^2-2 (k\cdot q)+i \eta )}
\end{dmath*}

\begin{dmath*}\breakingcomma
\left\{(x(1) x(2)+x(3) x(2)+x(1) x(3))^{3 \varepsilon -3} \left(q^2 x(1)^2 (x(2)+x(3))\right)^{1-2 \varepsilon },-\Gamma (2 \varepsilon -1),\{x(1),x(2),x(3)\}\right\}
\end{dmath*}

This Feynman parametric integral integrand contains logarithmic
divergences for \(x_1 \to \infty\) and \(x_{2,3} \to 0\)

\begin{Shaded}
\begin{Highlighting}[]
\NormalTok{divs }\ExtensionTok{=}\NormalTok{ FCFeynmanFindDivergences}\OperatorTok{[}\NormalTok{fpar}\OperatorTok{[[}\DecValTok{1}\OperatorTok{]],} \FunctionTok{x}\OperatorTok{]}
\end{Highlighting}
\end{Shaded}

\begin{dmath*}\breakingcomma
\left(
\begin{array}{cc}
 \{\{\},\{x(1)\}\} & \varepsilon  \\
 \{\{x(2),x(3)\},\{\}\} & \varepsilon  \\
\end{array}
\right)
\end{dmath*}

Regularizing the first divergence we obtain

\begin{Shaded}
\begin{Highlighting}[]
\NormalTok{intReg }\ExtensionTok{=}\NormalTok{ FCFeynmanRegularizeDivergence}\OperatorTok{[}\NormalTok{fpar}\OperatorTok{[[}\DecValTok{1}\OperatorTok{]],}\NormalTok{ divs}\OperatorTok{[[}\DecValTok{1}\OperatorTok{]]]}
\end{Highlighting}
\end{Shaded}

\begin{dmath*}\breakingcomma
-\frac{3 (\varepsilon -1) q^2 x(1)^2 x(2) x(3) (x(2)+x(3)) (x(1) x(2)+x(3) x(2)+x(1) x(3))^{3 \varepsilon -4} \left(q^2 x(1)^2 (x(2)+x(3))\right)^{-2 \varepsilon }}{\varepsilon }
\end{dmath*}

It turns out that there are no further divergences left

\begin{Shaded}
\begin{Highlighting}[]
\NormalTok{FCFeynmanFindDivergences}\OperatorTok{[}\NormalTok{intReg}\OperatorTok{,} \FunctionTok{x}\OperatorTok{]}
\end{Highlighting}
\end{Shaded}

\begin{dmath*}\breakingcomma
\{\}
\end{dmath*}

Now one can expand the integrand in \texttt{Epsilon} and perform the
integration in Feynman parameters order by order in \texttt{Epsilon}

\begin{Shaded}
\begin{Highlighting}[]
\FunctionTok{Series}\OperatorTok{[}\NormalTok{intReg}\OperatorTok{,} \OperatorTok{\{}\NormalTok{Epsilon}\OperatorTok{,} \DecValTok{0}\OperatorTok{,} \DecValTok{0}\OperatorTok{\}]} \SpecialCharTok{//} \FunctionTok{Normal}
\end{Highlighting}
\end{Shaded}

\begin{dmath*}\breakingcomma
\frac{3 q^2 x(1)^2 x(2) x(3) (x(2)+x(3))}{\varepsilon  (x(1) x(2)+x(3) x(2)+x(1) x(3))^4}-\frac{3 q^2 x(1)^2 x(2) x(3) (x(2)+x(3)) \left(2 \log \left(q^2 x(1)^2 (x(2)+x(3))\right)-3 \log (x(1) x(2)+x(3) x(2)+x(1) x(3))+1\right)}{(x(1) x(2)+x(3) x(2)+x(1) x(3))^4}
\end{dmath*}
\end{document}
