% !TeX program = pdflatex
% !TeX root = FCGVToSymbol.tex

\documentclass[../FeynCalcManual.tex]{subfiles}
\begin{document}
\hypertarget{fcgvtosymbol}{
\section{FCGVToSymbol}\label{fcgvtosymbol}\index{FCGVToSymbol}}

\texttt{FCGVToSymbol[\allowbreak{}exp]} converts objects of type
\texttt{FCGV[\allowbreak{}"sth"]} in \texttt{exp} to symbols using
\texttt{ToExpression[\allowbreak{}"sth"]}.

The option \texttt{StringReplace} can be used to specify string
replacement rules that will take care of special characters
(e.g.~\texttt{^} or \texttt{_}) that cannot appear in valid Mathematica
expressions. \texttt{SMPToSymbol} is useful when exporting FeynCalc
expressions to other tools, e.g.~FORM.

\subsection{See also}

\hyperlink{toc}{Overview}, \hyperlink{fcgv}{FCGV},
\hyperlink{smptosymbol}{SMPToSymbol}.

\subsection{Examples}

\begin{Shaded}
\begin{Highlighting}[]
\NormalTok{FCGV}\OperatorTok{[}\StringTok{"a"}\OperatorTok{]} \SpecialCharTok{//}\NormalTok{ FCGVToSymbol }
 
\SpecialCharTok{\%} \SpecialCharTok{//} \FunctionTok{InputForm}
\end{Highlighting}
\end{Shaded}

\begin{dmath*}\breakingcomma
a
\end{dmath*}

\begin{Shaded}
\begin{Highlighting}[]
\FunctionTok{a}
\end{Highlighting}
\end{Shaded}

\begin{Shaded}
\begin{Highlighting}[]
\NormalTok{FCGV}\OperatorTok{[}\StringTok{"$MU"}\OperatorTok{]} \SpecialCharTok{//}\NormalTok{ FCGVToSymbol }
 
\SpecialCharTok{\%} \SpecialCharTok{//} \FunctionTok{InputForm}
\end{Highlighting}
\end{Shaded}

\begin{dmath*}\breakingcomma
\text{\$MU}
\end{dmath*}

\begin{Shaded}
\begin{Highlighting}[]
\NormalTok{$MU}
\end{Highlighting}
\end{Shaded}

\end{document}
