% !TeX program = pdflatex
% !TeX root = FCGetDimensions.tex

\documentclass[../FeynCalcManual.tex]{subfiles}
\begin{document}
\hypertarget{fcgetdimensions}{%
\section{FCGetDimensions}\label{fcgetdimensions}}

\texttt{FCGetDimensions[\allowbreak{}expr]} is an auxiliary function
that determines the dimensions in which 4-momenta and Dirac matrices of
the given expression are defined. The result is returned as a list,
e.g.~\texttt{\{\allowbreak{}4\}}, \texttt{\{\allowbreak{}D\}} or
\texttt{\{\allowbreak{}4,\ \allowbreak{}D,\ \allowbreak{}D-4\}} etc.

This is useful if one wants to be sure that all quantities inside a
particular expression are purely \(4\)-dimensional or purely
\(D\)-dimensional.

\subsection{See also}

\hyperlink{toc}{Overview}, \hyperlink{changedimension}{ChangeDimension}.

\subsection{Examples}

\begin{Shaded}
\begin{Highlighting}[]
\NormalTok{FCGetDimensions}\OperatorTok{[}\NormalTok{GA}\OperatorTok{[}\FunctionTok{i}\OperatorTok{]]}
\end{Highlighting}
\end{Shaded}

\begin{dmath*}\breakingcomma
\{4\}
\end{dmath*}

\begin{Shaded}
\begin{Highlighting}[]
\NormalTok{FCGetDimensions}\OperatorTok{[}\NormalTok{GSD}\OperatorTok{[}\FunctionTok{p}\OperatorTok{]]}
\end{Highlighting}
\end{Shaded}

\begin{dmath*}\breakingcomma
\{D\}
\end{dmath*}

\begin{Shaded}
\begin{Highlighting}[]
\NormalTok{FCGetDimensions}\OperatorTok{[}\NormalTok{FVE}\OperatorTok{[}\FunctionTok{q}\OperatorTok{,} \SpecialCharTok{\textbackslash{}}\OperatorTok{[}\NormalTok{Mu}\OperatorTok{]]}\NormalTok{ GS}\OperatorTok{[}\FunctionTok{p}\OperatorTok{]]}
\end{Highlighting}
\end{Shaded}

\begin{dmath*}\breakingcomma
\{4,D-4\}
\end{dmath*}
\end{document}
