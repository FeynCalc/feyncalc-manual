% !TeX program = pdflatex
% !TeX root = FCGetDiracGammaScheme.tex

\documentclass[../FeynCalcManual.tex]{subfiles}
\begin{document}
\hypertarget{fcgetdiracgammascheme}{
\section{FCGetDiracGammaScheme}\label{fcgetdiracgammascheme}\index{FCGetDiracGammaScheme}}

\texttt{FCGetDiracGammaScheme[\allowbreak{}]} shows the currently used
scheme for handling Dirac matrices in \(D\) dimensions.

\subsection{See also}

\hyperlink{toc}{Overview},
\hyperlink{fcsetdiracgammascheme}{FCSetDiracGammaScheme},
\hyperlink{diractrace}{DiracTrace}.

\subsection{Examples}

\begin{Shaded}
\begin{Highlighting}[]
\NormalTok{FCSetDiracGammaScheme}\OperatorTok{[}\StringTok{"BMHV"}\OperatorTok{]} 
 
\NormalTok{FCGetDiracGammaScheme}\OperatorTok{[]} 
 
\SpecialCharTok{\%} \SpecialCharTok{//} \FunctionTok{FullForm}
\end{Highlighting}
\end{Shaded}

\begin{dmath*}\breakingcomma
\text{BMHV}
\end{dmath*}

\begin{dmath*}\breakingcomma
\text{BMHV}
\end{dmath*}

\begin{dmath*}\breakingcomma
\text{BMHV}
\end{dmath*}

\begin{Shaded}
\begin{Highlighting}[]
\NormalTok{FCSetDiracGammaScheme}\OperatorTok{[}\StringTok{"NDR"}\OperatorTok{]} 
 
\NormalTok{FCGetDiracGammaScheme}\OperatorTok{[]} 
 
\SpecialCharTok{\%} \SpecialCharTok{//} \FunctionTok{FullForm}
\end{Highlighting}
\end{Shaded}

\begin{dmath*}\breakingcomma
\text{NDR}
\end{dmath*}

\begin{dmath*}\breakingcomma
\text{NDR}
\end{dmath*}

\begin{dmath*}\breakingcomma
\text{NDR}
\end{dmath*}
\end{document}
