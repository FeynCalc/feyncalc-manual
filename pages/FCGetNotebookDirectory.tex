% !TeX program = pdflatex
% !TeX root = FCGetNotebookDirectory.tex

\documentclass[../FeynCalcManual.tex]{subfiles}
\begin{document}
\hypertarget{fcgetnotebookdirectory}{%
\section{FCGetNotebookDirectory}\label{fcgetnotebookdirectory}}

\texttt{FCGetNotebookDirectory[\allowbreak{}]} is a convenience function
that returns the directory in which the current notebook or .m file is
located. It also works when the FrontEnd is not available.

\subsection{See also}

\hyperlink{toc}{Overview}

\subsection{Examples}

\begin{Shaded}
\begin{Highlighting}[]
\NormalTok{FCGetNotebookDirectory}\OperatorTok{[]}
\end{Highlighting}
\end{Shaded}

\begin{dmath*}\breakingcomma
\text{/media/Data/Projects/VS/FeynCalc/FeynCalc/Documentation/Mathematica/Shared/Tools/}
\end{dmath*}
\end{document}
