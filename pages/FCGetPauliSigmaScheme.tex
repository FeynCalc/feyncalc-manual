% !TeX program = pdflatex
% !TeX root = FCGetPauliSigmaScheme.tex

\documentclass[../FeynCalcManual.tex]{subfiles}
\begin{document}
\hypertarget{fcgetpaulisigmascheme}{
\section{FCGetPauliSigmaScheme}\label{fcgetpaulisigmascheme}\index{FCGetPauliSigmaScheme}}

\texttt{FCGetPauliSigmaScheme[\allowbreak{}]} shows the currently used
scheme for handling Pauli matrices in \(D-1\) dimensions. For more
details see the documentation of \texttt{FCSetPauliSigmaScheme}.

\subsection{See also}

\hyperlink{toc}{Overview}, \hyperlink{paulisigma}{PauliSigma},
\hyperlink{fcsetpaulisigmascheme}{FCSetPauliSigmaScheme}.

\subsection{Examples}

\begin{Shaded}
\begin{Highlighting}[]
\NormalTok{FCGetPauliSigmaScheme}\OperatorTok{[]}
\end{Highlighting}
\end{Shaded}

\begin{dmath*}\breakingcomma
\text{None}
\end{dmath*}
\end{document}
