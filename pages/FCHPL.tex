% !TeX program = pdflatex
% !TeX root = FCHPL.tex

\documentclass[../FeynCalcManual.tex]{subfiles}
\begin{document}
\begin{Shaded}
\begin{Highlighting}[]
 
\end{Highlighting}
\end{Shaded}

\hypertarget{fchpl}{
\section{FCHPL}\label{fchpl}\index{FCHPL}}

\texttt{FCHPL[\allowbreak{}\{\allowbreak{}inds_\},\ \allowbreak{}var]}
represents a harmonic polylogarithm (HPL) with indices \texttt{inds} and
argument \texttt{var}.

As of know this symbol merely acts a placeholder and does not implement
any mathematical properties of HPLs.

\subsection{See also}

\hyperlink{toc}{Overview}, \hyperlink{fcgpl}{FCGPL}

\subsection{Examples}

\begin{Shaded}
\begin{Highlighting}[]
\NormalTok{FCHPL}\OperatorTok{[\{}\DecValTok{1}\OperatorTok{\},} \FunctionTok{x}\OperatorTok{]}
\end{Highlighting}
\end{Shaded}

\begin{dmath*}\breakingcomma
H(1; x)
\end{dmath*}
\end{document}
