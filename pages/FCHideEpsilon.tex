% !TeX program = pdflatex
% !TeX root = FCHideEpsilon.tex

\documentclass[../FeynCalcManual.tex]{subfiles}
\begin{document}
\hypertarget{fchideepsilon}{%
\section{FCHideEpsilon}\label{fchideepsilon}}

\texttt{FCHideEpsilon[\allowbreak{}expr]} substitutes
\texttt{1/Epsilon - EulerGamma + Log[\allowbreak{}4 Pi]} with
\texttt{SMP[\allowbreak{}"Delta"]}.

\subsection{See also}

\hyperlink{toc}{Overview}, \hyperlink{fcshowepsilon}{FCShowEpsilon}.

\subsection{Examples}

\begin{Shaded}
\begin{Highlighting}[]
\DecValTok{1}\SpecialCharTok{/}\NormalTok{Epsilon }\SpecialCharTok{+} \FunctionTok{Log}\OperatorTok{[}\DecValTok{4} \FunctionTok{Pi}\OperatorTok{]} \SpecialCharTok{{-}} \FunctionTok{EulerGamma} 
 
\NormalTok{FCHideEpsilon}\OperatorTok{[}\SpecialCharTok{\%}\OperatorTok{]}
\end{Highlighting}
\end{Shaded}

\begin{dmath*}\breakingcomma
\frac{1}{\varepsilon }-\gamma +\log (4 \pi )
\end{dmath*}

\begin{dmath*}\breakingcomma
\Delta
\end{dmath*}

\begin{Shaded}
\begin{Highlighting}[]
\DecValTok{1}\SpecialCharTok{/}\NormalTok{EpsilonUV }\SpecialCharTok{+} \FunctionTok{Log}\OperatorTok{[}\DecValTok{4} \FunctionTok{Pi}\OperatorTok{]} \SpecialCharTok{{-}} \FunctionTok{EulerGamma} 
 
\NormalTok{FCHideEpsilon}\OperatorTok{[}\SpecialCharTok{\%}\OperatorTok{]}
\end{Highlighting}
\end{Shaded}

\begin{dmath*}\breakingcomma
\frac{1}{\varepsilon _{\text{UV}}}-\gamma +\log (4 \pi )
\end{dmath*}

\begin{dmath*}\breakingcomma
\Delta _{\text{UV}}
\end{dmath*}

\begin{Shaded}
\begin{Highlighting}[]
\DecValTok{1}\SpecialCharTok{/}\NormalTok{EpsilonIR }\SpecialCharTok{+} \FunctionTok{Log}\OperatorTok{[}\DecValTok{4} \FunctionTok{Pi}\OperatorTok{]} \SpecialCharTok{{-}} \FunctionTok{EulerGamma} 
 
\NormalTok{FCHideEpsilon}\OperatorTok{[}\SpecialCharTok{\%}\OperatorTok{]}
\end{Highlighting}
\end{Shaded}

\begin{dmath*}\breakingcomma
\frac{1}{\varepsilon _{\text{IR}}}-\gamma +\log (4 \pi )
\end{dmath*}

\begin{dmath*}\breakingcomma
\Delta _{\text{IR}}
\end{dmath*}
\end{document}
