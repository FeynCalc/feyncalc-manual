% !TeX program = pdflatex
% !TeX root = FCHighlight.tex

\documentclass[../FeynCalcManual.tex]{subfiles}
\begin{document}
\hypertarget{fchighlight}{%
\section{FCHighlight}\label{fchighlight}}

\texttt{FCHighlight[\allowbreak{}exp,\ \allowbreak{}\{\allowbreak{}\{\allowbreak{}symbol1,\ \allowbreak{}color1\},\ \allowbreak{}\{\allowbreak{}symbol2,\ \allowbreak{}color2\},\ \allowbreak{}...\}]}
highlights the given set of symbols in the output using \texttt{Style}
and the provided colors. This works only in the frontend and alters the
input expression in such a way, that it cannot be processed further
(because of the introduced \texttt{Style} heads).

\subsection{See also}

\hyperlink{toc}{Overview}

\subsection{Examples}

\begin{Shaded}
\begin{Highlighting}[]
\NormalTok{FCHighlight}\OperatorTok{[}\FunctionTok{Expand}\OperatorTok{[}\NormalTok{(}\FunctionTok{a} \SpecialCharTok{+} \FunctionTok{b} \SpecialCharTok{+} \FunctionTok{c}\NormalTok{)}\SpecialCharTok{\^{}}\DecValTok{2}\OperatorTok{],} \OperatorTok{\{\{}\FunctionTok{a}\OperatorTok{,} \FunctionTok{Red}\OperatorTok{\},} \OperatorTok{\{}\FunctionTok{b}\OperatorTok{,} \FunctionTok{Blue}\OperatorTok{\}\}]}
\end{Highlighting}
\end{Shaded}

\begin{dmath*}\breakingcomma
2 a b+2 a c+a^2+2 b c+b^2+c^2
\end{dmath*}
\end{document}
