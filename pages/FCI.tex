% !TeX program = pdflatex
% !TeX root = FCI.tex

\documentclass[../FeynCalcManual.tex]{subfiles}
\begin{document}
\hypertarget{fci}{
\section{FCI}\label{fci}\index{FCI}}

\texttt{FCI[\allowbreak{}exp]} translates exp into the internal FeynCalc
(datatype-)representation.

\texttt{FCI} is equivalent to \texttt{FeynCalcInternal}.

\subsection{See also}

\hyperlink{toc}{Overview},
\hyperlink{feyncalcexternal}{FeynCalcExternal},
\hyperlink{feyncalcinternal}{FeynCalcInternal}, \hyperlink{fce}{FCE}.

\subsection{Examples}

\begin{Shaded}
\begin{Highlighting}[]
\NormalTok{ex }\ExtensionTok{=} \OperatorTok{\{}\NormalTok{GA}\OperatorTok{[}\SpecialCharTok{\textbackslash{}}\OperatorTok{[}\NormalTok{Mu}\OperatorTok{]],}\NormalTok{ GAD}\OperatorTok{[}\SpecialCharTok{\textbackslash{}}\OperatorTok{[}\NormalTok{Rho}\OperatorTok{]],}\NormalTok{ GS}\OperatorTok{[}\FunctionTok{p}\OperatorTok{],}\NormalTok{ SP}\OperatorTok{[}\FunctionTok{p}\OperatorTok{,} \FunctionTok{q}\OperatorTok{],}\NormalTok{ MT}\OperatorTok{[}\SpecialCharTok{\textbackslash{}}\OperatorTok{[}\NormalTok{Alpha}\OperatorTok{],} \SpecialCharTok{\textbackslash{}}\OperatorTok{[}\FunctionTok{Beta}\OperatorTok{]],}\NormalTok{ FV}\OperatorTok{[}\FunctionTok{p}\OperatorTok{,} \SpecialCharTok{\textbackslash{}}\OperatorTok{[}\NormalTok{Mu}\OperatorTok{]]\}}
\end{Highlighting}
\end{Shaded}

\begin{dmath*}\breakingcomma
\left\{\bar{\gamma }^{\mu },\gamma ^{\rho },\bar{\gamma }\cdot \overline{p},\overline{p}\cdot \overline{q},\bar{g}^{\alpha \beta },\overline{p}^{\mu }\right\}
\end{dmath*}

\begin{Shaded}
\begin{Highlighting}[]
\NormalTok{ex }\SpecialCharTok{//} \FunctionTok{StandardForm}

\CommentTok{(*\{GA[\textbackslash{}[Mu]], GAD[\textbackslash{}[Rho]], GS[p], SP[p, q], MT[\textbackslash{}[Alpha], \textbackslash{}[Beta]], FV[p, \textbackslash{}[Mu]]\}*)}
\end{Highlighting}
\end{Shaded}

\begin{Shaded}
\begin{Highlighting}[]
\NormalTok{ex }\SpecialCharTok{//}\NormalTok{ FCI}
\end{Highlighting}
\end{Shaded}

\begin{dmath*}\breakingcomma
\left\{\bar{\gamma }^{\mu },\gamma ^{\rho },\bar{\gamma }\cdot \overline{p},\overline{p}\cdot \overline{q},\bar{g}^{\alpha \beta },\overline{p}^{\mu }\right\}
\end{dmath*}

\begin{Shaded}
\begin{Highlighting}[]
\NormalTok{ex }\SpecialCharTok{//}\NormalTok{ FCI }\SpecialCharTok{//} \FunctionTok{StandardForm}

\CommentTok{(*\{DiracGamma[LorentzIndex[\textbackslash{}[Mu]]], DiracGamma[LorentzIndex[\textbackslash{}[Rho], D], D], DiracGamma[Momentum[p]], Pair[Momentum[p], Momentum[q]], Pair[LorentzIndex[\textbackslash{}[Alpha]], LorentzIndex[\textbackslash{}[Beta]]], Pair[LorentzIndex[\textbackslash{}[Mu]], Momentum[p]]\}*)}
\end{Highlighting}
\end{Shaded}

\begin{Shaded}
\begin{Highlighting}[]
\NormalTok{ex }\SpecialCharTok{//}\NormalTok{ FCE}
\end{Highlighting}
\end{Shaded}

\begin{dmath*}\breakingcomma
\left\{\bar{\gamma }^{\mu },\gamma ^{\rho },\bar{\gamma }\cdot \overline{p},\overline{p}\cdot \overline{q},\bar{g}^{\alpha \beta },\overline{p}^{\mu }\right\}
\end{dmath*}

\begin{Shaded}
\begin{Highlighting}[]
\NormalTok{ex }\SpecialCharTok{//}\NormalTok{ FCE }\SpecialCharTok{//} \FunctionTok{StandardForm}

\CommentTok{(*\{GA[\textbackslash{}[Mu]], GAD[\textbackslash{}[Rho]], GS[p], SP[p, q], MT[\textbackslash{}[Alpha], \textbackslash{}[Beta]], FV[p, \textbackslash{}[Mu]]\}*)}
\end{Highlighting}
\end{Shaded}

\end{document}
