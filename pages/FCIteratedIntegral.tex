% !TeX program = pdflatex
% !TeX root = FCIteratedIntegral.tex

\documentclass[../FeynCalcManual.tex]{subfiles}
\begin{document}
\begin{Shaded}
\begin{Highlighting}[]
 
\end{Highlighting}
\end{Shaded}

\hypertarget{fciteratedintegral}{
\section{FCIteratedIntegral}\label{fciteratedintegral}\index{FCIteratedIntegral}}

\texttt{FCIteratedIntegral[\allowbreak{}f,\ \allowbreak{}x,\ \allowbreak{}a,\ \allowbreak{}b]}
is a special head indicating that the function \(f\) represents an
iterated integral or a linear combination thereof and that it should be
integrated in \(x\) from \(a\) to \(b\). This notation is understood by
the function \texttt{FCIteratedIntegralEvaluate} that does the actual
integration.

Notice that before applying \texttt{FCIteratedIntegralEvaluate} all
rational functions of \(x\) in \(f\) should be converted to the
\texttt{FCPartialFractionForm}representation.

\subsection{See also}

\hyperlink{toc}{Overview},
\hyperlink{fciteratedintegralevaluate}{FCIteratedIntegralEvaluate},
\hyperlink{tofcpartialfractionform}{ToFCPartialFractionForm}

\subsection{Examples}

\begin{Shaded}
\begin{Highlighting}[]
\NormalTok{fun }\ExtensionTok{=} \DecValTok{1}\SpecialCharTok{/}\NormalTok{(}\DecValTok{1} \SpecialCharTok{+} \FunctionTok{x}\NormalTok{)}
\end{Highlighting}
\end{Shaded}

\begin{dmath*}\breakingcomma
\frac{1}{x+1}
\end{dmath*}

\begin{Shaded}
\begin{Highlighting}[]
\NormalTok{int }\ExtensionTok{=}\NormalTok{ FCIteratedIntegral}\OperatorTok{[}\NormalTok{ToFCPartialFractionForm}\OperatorTok{[}\NormalTok{fun}\OperatorTok{,} \FunctionTok{x}\OperatorTok{],} \FunctionTok{x}\OperatorTok{,} \FunctionTok{a}\OperatorTok{,} \FunctionTok{b}\OperatorTok{]}
\end{Highlighting}
\end{Shaded}

\begin{dmath*}\breakingcomma
\text{FCIteratedIntegral}\left(\text{FCPartialFractionForm}\left(0,\left(
\begin{array}{cc}
 \{x+1,-1\} & 1 \\
\end{array}
\right),x\right),x,a,b\right)
\end{dmath*}

\begin{Shaded}
\begin{Highlighting}[]
\NormalTok{FCIteratedIntegralEvaluate}\OperatorTok{[}\NormalTok{int}\OperatorTok{]}
\end{Highlighting}
\end{Shaded}

\begin{dmath*}\breakingcomma
G(-1; b)-G(-1; a)
\end{dmath*}
\end{document}
