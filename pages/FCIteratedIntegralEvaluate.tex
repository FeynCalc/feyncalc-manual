% !TeX program = pdflatex
% !TeX root = FCIteratedIntegralEvaluate.tex

\documentclass[../FeynCalcManual.tex]{subfiles}
\begin{document}
\hypertarget{fciteratedintegralevaluate}{
\section{FCIteratedIntegralEvaluate}\label{fciteratedintegralevaluate}\index{FCIteratedIntegralEvaluate}}

\texttt{FCIteratedIntegralEvaluate[\allowbreak{}ex]} evaluates iterated
integrals in ex in terms of multiple polylogarithms.

To that aim the \texttt{ex} must contain ration functions (in the
\texttt{FCPartialFractionForm} notation) and possibly \texttt{FCGPL}s
wrapped with \texttt{FCIteratedIntegral} heads

\subsection{See also}

\hyperlink{toc}{Overview},
\hyperlink{fciteratedintegral}{FCIteratedIntegral},
\hyperlink{fciteratedintegralsimplify}{FCIteratedIntegralSimplify},
\hyperlink{fcgpl}{FCGPL}.

\subsection{Examples}

\begin{Shaded}
\begin{Highlighting}[]
\NormalTok{int }\ExtensionTok{=}\NormalTok{ FCPartialFractionForm}\OperatorTok{[}\DecValTok{0}\OperatorTok{,} \OperatorTok{\{\{\{}\SpecialCharTok{{-}}\FunctionTok{a} \SpecialCharTok{+} \FunctionTok{x}\OperatorTok{[}\DecValTok{2}\OperatorTok{],} \SpecialCharTok{{-}}\DecValTok{1}\OperatorTok{\},}\NormalTok{ (}\DecValTok{1} \SpecialCharTok{+} \FunctionTok{a} \SpecialCharTok{+} \FunctionTok{x}\OperatorTok{[}\DecValTok{3}\OperatorTok{]}\NormalTok{)}\SpecialCharTok{\^{}}\NormalTok{(}\SpecialCharTok{{-}}\DecValTok{2}\NormalTok{)}\OperatorTok{\},} 
    \OperatorTok{\{\{}\DecValTok{1} \SpecialCharTok{+} \FunctionTok{x}\OperatorTok{[}\DecValTok{2}\OperatorTok{]} \SpecialCharTok{+} \FunctionTok{x}\OperatorTok{[}\DecValTok{3}\OperatorTok{],} \SpecialCharTok{{-}}\DecValTok{2}\OperatorTok{\},} \SpecialCharTok{{-}}\NormalTok{(}\DecValTok{1} \SpecialCharTok{+} \FunctionTok{a} \SpecialCharTok{+} \FunctionTok{x}\OperatorTok{[}\DecValTok{3}\OperatorTok{]}\NormalTok{)}\SpecialCharTok{\^{}}\NormalTok{(}\SpecialCharTok{{-}}\DecValTok{1}\NormalTok{)}\OperatorTok{\},} \OperatorTok{\{\{}\DecValTok{1} \SpecialCharTok{+} \FunctionTok{x}\OperatorTok{[}\DecValTok{2}\OperatorTok{]} \SpecialCharTok{+} \FunctionTok{x}\OperatorTok{[}\DecValTok{3}\OperatorTok{],} \SpecialCharTok{{-}}\DecValTok{1}\OperatorTok{\},} \SpecialCharTok{{-}}\NormalTok{(}\DecValTok{1} \SpecialCharTok{+} \FunctionTok{a} \SpecialCharTok{+} \FunctionTok{x}\OperatorTok{[}\DecValTok{3}\OperatorTok{]}\NormalTok{)}\SpecialCharTok{\^{}}\NormalTok{(}\SpecialCharTok{{-}}\DecValTok{2}\NormalTok{)}\OperatorTok{\}\},} \FunctionTok{x}\OperatorTok{[}\DecValTok{2}\OperatorTok{]]}
\end{Highlighting}
\end{Shaded}

\begin{dmath*}\breakingcomma
\text{FCPartialFractionForm}\left(0,\left(
\begin{array}{cc}
 \{x(2)-a,-1\} & \frac{1}{(a+x(3)+1)^2} \\
 \{x(2)+x(3)+1,-2\} & -\frac{1}{a+x(3)+1} \\
 \{x(2)+x(3)+1,-1\} & -\frac{1}{(a+x(3)+1)^2} \\
\end{array}
\right),x(2)\right)
\end{dmath*}

\begin{Shaded}
\begin{Highlighting}[]
\NormalTok{FCIteratedIntegralEvaluate}\OperatorTok{[}\NormalTok{FCIteratedIntegral}\OperatorTok{[}\NormalTok{int}\OperatorTok{,} \FunctionTok{x}\OperatorTok{[}\DecValTok{2}\OperatorTok{],} \DecValTok{0}\OperatorTok{,} \FunctionTok{Infinity}\OperatorTok{]]}
\end{Highlighting}
\end{Shaded}

\begin{dmath*}\breakingcomma
-\text{FCPartialFractionForm}\left(0,\left(
\begin{array}{cc}
 \{\infty ,-1\} & -\frac{1}{a+x(3)+1} \\
\end{array}
\right),\infty \right)+\text{FCPartialFractionForm}\left(0,\left(
\begin{array}{cc}
 \{x(3)+1,-1\} & -\frac{1}{a+x(3)+1} \\
\end{array}
\right),0\right)-\frac{G(-x[3]-1; \infty )}{(a+x(3)+1)^2}+\frac{G(a; \infty )}{(a+x(3)+1)^2}
\end{dmath*}

\begin{Shaded}
\begin{Highlighting}[]
\NormalTok{FCIteratedIntegralEvaluate}\OperatorTok{[}\NormalTok{FCIteratedIntegral}\OperatorTok{[}\NormalTok{int}\OperatorTok{,} \FunctionTok{x}\OperatorTok{[}\DecValTok{2}\OperatorTok{],} \DecValTok{0}\OperatorTok{,} \FunctionTok{x}\OperatorTok{[}\DecValTok{2}\OperatorTok{]]]}
\end{Highlighting}
\end{Shaded}

\begin{dmath*}\breakingcomma
\text{FCPartialFractionForm}\left(0,\left(
\begin{array}{cc}
 \{x(3)+1,-1\} & -\frac{1}{a+x(3)+1} \\
\end{array}
\right),0\right)-\text{FCPartialFractionForm}\left(0,\left(
\begin{array}{cc}
 \{x(2)+x(3)+1,-1\} & -\frac{1}{a+x(3)+1} \\
\end{array}
\right),x(2)\right)+\frac{G(a; x[2])}{(a+x(3)+1)^2}-\frac{G(-x[3]-1; x[2])}{(a+x(3)+1)^2}
\end{dmath*}
\end{document}
