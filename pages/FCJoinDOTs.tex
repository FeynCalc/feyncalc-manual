% !TeX program = pdflatex
% !TeX root = FCJoinDOTs.tex

\documentclass[../FeynCalcManual.tex]{subfiles}
\begin{document}
\hypertarget{fcjoindots}{
\section{FCJoinDOTs}\label{fcjoindots}\index{FCJoinDOTs}}

\texttt{FCJoinDOTs} is an option for \texttt{DotSimplify} and other
functions that use \texttt{DotSimplify} internally. When set to
\texttt{True}, \texttt{DotSimplify} will try to rewrite expressions like
\texttt{A.X.B + A.Y.B} as \texttt{A.(X+Y).B}.

Notice that although the default value of \texttt{FCJoinDOTs} is
\texttt{True}, the corresponding transformations will occur only if the
option \texttt{Expanding} is set to \texttt{False} (default:
\texttt{True})

\subsection{See also}

\hyperlink{toc}{Overview}, \hyperlink{dotsimplify}{DotSimplify}.

\subsection{Examples}

\begin{Shaded}
\begin{Highlighting}[]
\NormalTok{DeclareNonCommutative}\OperatorTok{[}\FunctionTok{A}\OperatorTok{,} \FunctionTok{B}\OperatorTok{,} \FunctionTok{X}\OperatorTok{,} \FunctionTok{Y}\OperatorTok{]}
\end{Highlighting}
\end{Shaded}

\begin{Shaded}
\begin{Highlighting}[]
\NormalTok{DotSimplify}\OperatorTok{[}\FunctionTok{A}\NormalTok{ . }\FunctionTok{X}\NormalTok{ . }\FunctionTok{B} \SpecialCharTok{+} \FunctionTok{A}\NormalTok{ . }\FunctionTok{Y}\NormalTok{ . }\FunctionTok{B}\OperatorTok{]}
\end{Highlighting}
\end{Shaded}

\begin{dmath*}\breakingcomma
A.X.B+A.Y.B
\end{dmath*}

\begin{Shaded}
\begin{Highlighting}[]
\NormalTok{DotSimplify}\OperatorTok{[}\FunctionTok{A}\NormalTok{ . }\FunctionTok{X}\NormalTok{ . }\FunctionTok{B} \SpecialCharTok{+} \FunctionTok{A}\NormalTok{ . }\FunctionTok{Y}\NormalTok{ . }\FunctionTok{B}\OperatorTok{,}\NormalTok{ FCJoinDOTs }\OtherTok{{-}\textgreater{}} \ConstantTok{True}\OperatorTok{]}
\end{Highlighting}
\end{Shaded}

\begin{dmath*}\breakingcomma
A.X.B+A.Y.B
\end{dmath*}

\begin{Shaded}
\begin{Highlighting}[]
\NormalTok{DotSimplify}\OperatorTok{[}\FunctionTok{A}\NormalTok{ . }\FunctionTok{X}\NormalTok{ . }\FunctionTok{B} \SpecialCharTok{+} \FunctionTok{A}\NormalTok{ . }\FunctionTok{Y}\NormalTok{ . }\FunctionTok{B}\OperatorTok{,}\NormalTok{ FCJoinDOTs }\OtherTok{{-}\textgreater{}} \ConstantTok{True}\OperatorTok{,}\NormalTok{ Expanding }\OtherTok{{-}\textgreater{}} \ConstantTok{False}\OperatorTok{]}
\end{Highlighting}
\end{Shaded}

\begin{dmath*}\breakingcomma
A.(X+Y).B
\end{dmath*}

\begin{Shaded}
\begin{Highlighting}[]
\NormalTok{DotSimplify}\OperatorTok{[}\NormalTok{GA}\OperatorTok{[}\NormalTok{mu}\OperatorTok{,} \DecValTok{6}\OperatorTok{,}\NormalTok{ nu}\OperatorTok{]} \SpecialCharTok{+}\NormalTok{ GA}\OperatorTok{[}\NormalTok{mu}\OperatorTok{,} \DecValTok{7}\OperatorTok{,}\NormalTok{ nu}\OperatorTok{],}\NormalTok{ FCJoinDOTs }\OtherTok{{-}\textgreater{}} \ConstantTok{True}\OperatorTok{,}\NormalTok{ Expanding }\OtherTok{{-}\textgreater{}} \ConstantTok{False}\OperatorTok{]}
\end{Highlighting}
\end{Shaded}

\begin{dmath*}\breakingcomma
\bar{\gamma }^{\text{mu}}.\left(\bar{\gamma }^6+\bar{\gamma }^7\right).\bar{\gamma }^{\text{nu}}
\end{dmath*}
\end{document}
