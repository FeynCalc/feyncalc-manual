% !TeX program = pdflatex
% !TeX root = FCLoopAddMissingHigherOrdersWarning.tex

\documentclass[../FeynCalcManual.tex]{subfiles}
\begin{document}
\hypertarget{fcloopaddmissinghigherorderswarning}{
\section{FCLoopAddMissingHigherOrdersWarning}\label{fcloopaddmissinghigherorderswarning}\index{FCLoopAddMissingHigherOrdersWarning}}

\texttt{FCLoopAddMissingHigherOrdersWarning[\allowbreak{}expr,\ \allowbreak{}ep,\ \allowbreak{}fun]}
determines the highest \texttt{ep}-power \(n\) in the given expression
and adds a warning flag of order \(\textrm{ep}^n+1\). This is meant to
prevent incorrect results stemming insufficient high expansions of
\texttt{expr} in \texttt{ep}

\subsection{See also}

\hyperlink{toc}{Overview},
\hyperlink{fcfeynmanparametrize}{FCFeynmanParametrize},
\hyperlink{fcfeynmanprepare}{FCFeynmanPrepare}.

\subsection{Examples}

\begin{Shaded}
\begin{Highlighting}[]
\NormalTok{FCLoopAddMissingHigherOrdersWarning}\OperatorTok{[}\DecValTok{1}\SpecialCharTok{/}\NormalTok{ep}\SpecialCharTok{\^{}}\DecValTok{2}\NormalTok{ cc1 }\SpecialCharTok{+} \DecValTok{1}\SpecialCharTok{/}\NormalTok{ep cc2}\OperatorTok{,}\NormalTok{ ep}\OperatorTok{,}\NormalTok{ epHelp}\OperatorTok{]}
\end{Highlighting}
\end{Shaded}

\begin{dmath*}\breakingcomma
\frac{\text{cc1}}{\text{ep}^2}+\frac{\text{cc2}}{\text{ep}}+(1+i) \;\text{epHelp}
\end{dmath*}

\begin{Shaded}
\begin{Highlighting}[]
\NormalTok{FCLoopAddMissingHigherOrdersWarning}\OperatorTok{[}\NormalTok{cc1}\OperatorTok{,}\NormalTok{ ep}\OperatorTok{,}\NormalTok{ epHelp}\OperatorTok{]}
\end{Highlighting}
\end{Shaded}

\begin{dmath*}\breakingcomma
\text{cc1}+(1+i) \;\text{ep} \;\text{epHelp}
\end{dmath*}

\begin{Shaded}
\begin{Highlighting}[]
\NormalTok{FCLoopAddMissingHigherOrdersWarning}\OperatorTok{[}\NormalTok{cc1}\OperatorTok{,}\NormalTok{ ep}\OperatorTok{,}\NormalTok{ epHelp}\OperatorTok{,} \FunctionTok{Complex} \OtherTok{{-}\textgreater{}} \ConstantTok{False}\OperatorTok{]}
\end{Highlighting}
\end{Shaded}

\begin{dmath*}\breakingcomma
\text{cc1}+\text{ep} \;\text{epHelp}
\end{dmath*}

\begin{Shaded}
\begin{Highlighting}[]
\NormalTok{FCLoopAddMissingHigherOrdersWarning}\OperatorTok{[}\NormalTok{cc1}\OperatorTok{,}\NormalTok{ ep}\OperatorTok{,}\NormalTok{ epHelp}\OperatorTok{,} \FunctionTok{Names} \OtherTok{{-}\textgreater{}} \ConstantTok{False}\OperatorTok{]}
\end{Highlighting}
\end{Shaded}

\begin{dmath*}\breakingcomma
\text{cc1}+(1+i) \;\text{ep} \;\text{epHelp}
\end{dmath*}

\begin{Shaded}
\begin{Highlighting}[]
\NormalTok{FCLoopAddMissingHigherOrdersWarning}\OperatorTok{[}\NormalTok{GLI}\OperatorTok{[}\NormalTok{topo1}\OperatorTok{,} \OperatorTok{\{}\DecValTok{1}\OperatorTok{,} \DecValTok{1}\OperatorTok{,} \DecValTok{1}\OperatorTok{,} \DecValTok{1}\OperatorTok{,} \DecValTok{1}\OperatorTok{\}]} \OtherTok{{-}\textgreater{}}\NormalTok{ cc1}\SpecialCharTok{/}\NormalTok{ep}\SpecialCharTok{\^{}}\DecValTok{2} \SpecialCharTok{+}\NormalTok{ cc2}\SpecialCharTok{/}\NormalTok{ep }\SpecialCharTok{+}\NormalTok{ cc3 }\OperatorTok{,}\NormalTok{ ep}\OperatorTok{,}\NormalTok{ epHelp}\OperatorTok{]}
\end{Highlighting}
\end{Shaded}

\begin{dmath*}\breakingcomma
G^{\text{topo1}}(1,1,1,1,1)\to \frac{\text{cc1}}{\text{ep}^2}+\frac{\text{cc2}}{\text{ep}}+\text{cc3}+(1+i) \;\text{ep} \;\text{epHelp}(\text{topo1X11111})
\end{dmath*}
\end{document}
