% !TeX program = pdflatex
% !TeX root = FCLoopAugmentTopology.tex

\documentclass[../FeynCalcManual.tex]{subfiles}
\begin{document}
\hypertarget{fcloopaugmenttopology}{
\section{FCLoopAugmentTopology}\label{fcloopaugmenttopology}\index{FCLoopAugmentTopology}}

\texttt{FCLoopAugmentTopology[\allowbreak{}topo,\ \allowbreak{}\{\allowbreak{}extraProps\}]}
augments the topology \texttt{topo} by adding new propagators
\texttt{extraProps} to the basis. This is usually needed when a tensor
reduction requires us to introduce an auxiliary vector that will appear
in scalar products involving loop momenta.

The input topologies do not have to be complete.

The output of this routine contains augmented topologies and a list of
replacement rules for converting \texttt{GLI}s depending on the old
topologies into new ones.

\subsection{See also}

\hyperlink{toc}{Overview},
\hyperlink{fclooptensorreduce}{FCLoopTensorReduce}.

\subsection{Examples}

\begin{Shaded}
\begin{Highlighting}[]
\NormalTok{topo }\ExtensionTok{=}\NormalTok{ FCTopology}\OperatorTok{[}\StringTok{"topo1"}\OperatorTok{,} \OperatorTok{\{}\NormalTok{SFAD}\OperatorTok{[\{}\NormalTok{q1}\OperatorTok{,} \FunctionTok{m}\SpecialCharTok{\^{}}\DecValTok{2}\OperatorTok{\}],}\NormalTok{ SFAD}\OperatorTok{[\{}\NormalTok{q1 }\SpecialCharTok{+}\NormalTok{ p1}\OperatorTok{\}],} 
\NormalTok{    SFAD}\OperatorTok{[\{}\NormalTok{q1 }\SpecialCharTok{+}\NormalTok{ p2}\OperatorTok{\}]\},} \OperatorTok{\{}\NormalTok{q1}\OperatorTok{\},} \OperatorTok{\{}\NormalTok{p1}\OperatorTok{,}\NormalTok{ p2}\OperatorTok{\},} \OperatorTok{\{}\FunctionTok{Hold}\OperatorTok{[}\NormalTok{SPD}\OperatorTok{][}\NormalTok{p1}\OperatorTok{]} \OtherTok{{-}\textgreater{}} \DecValTok{0}\OperatorTok{,} \FunctionTok{Hold}\OperatorTok{[}\NormalTok{SPD}\OperatorTok{][}\NormalTok{p2}\OperatorTok{]} \OtherTok{{-}\textgreater{}} \DecValTok{0}\OperatorTok{,} 
    \FunctionTok{Hold}\OperatorTok{[}\NormalTok{SPD}\OperatorTok{][}\NormalTok{p1}\OperatorTok{,}\NormalTok{ p2}\OperatorTok{]} \OtherTok{{-}\textgreater{}} \DecValTok{0}\OperatorTok{\},} \OperatorTok{\{\}]}
\end{Highlighting}
\end{Shaded}

\begin{dmath*}\breakingcomma
\text{FCTopology}\left(\text{topo1},\left\{\frac{1}{(\text{q1}^2-m^2+i \eta )},\frac{1}{((\text{p1}+\text{q1})^2+i \eta )},\frac{1}{((\text{p2}+\text{q1})^2+i \eta )}\right\},\{\text{q1}\},\{\text{p1},\text{p2}\},\{\text{Hold}[\text{SPD}][\text{p1}]\to 0,\text{Hold}[\text{SPD}][\text{p2}]\to 0,\text{Hold}[\text{SPD}][\text{p1},\text{p2}]\to 0\},\{\}\right)
\end{dmath*}

The option \texttt{AugmentedTopologyMarker} denotes a symbol that is
usually introduced by \texttt{FCLoopTensorReduce} when the reduction
requires an auxiliary vector. Therefore, it will appear on the right
hand side of the \texttt{GLI}-replacement rules. This can be disabled by
setting this option to \texttt{False}

\begin{Shaded}
\begin{Highlighting}[]
\NormalTok{FCLoopAugmentTopology}\OperatorTok{[}\NormalTok{topo}\OperatorTok{,} \OperatorTok{\{}\NormalTok{SFAD}\OperatorTok{[\{\{}\DecValTok{0}\OperatorTok{,}\NormalTok{ q1 . }\FunctionTok{n}\OperatorTok{\}\}]\},} 
\NormalTok{  FinalSubstitutions }\OtherTok{{-}\textgreater{}} \OperatorTok{\{}\FunctionTok{Hold}\OperatorTok{[}\NormalTok{SPD}\OperatorTok{][}\FunctionTok{n}\OperatorTok{]} \OtherTok{{-}\textgreater{}} \DecValTok{0}\OperatorTok{,} \FunctionTok{Hold}\OperatorTok{[}\NormalTok{SPD}\OperatorTok{][}\FunctionTok{n}\OperatorTok{,}\NormalTok{ p1}\OperatorTok{]} \OtherTok{{-}\textgreater{}}\NormalTok{ np1}\OperatorTok{,} 
    \FunctionTok{Hold}\OperatorTok{[}\NormalTok{SPD}\OperatorTok{][}\FunctionTok{n}\OperatorTok{,}\NormalTok{ p2}\OperatorTok{]} \OtherTok{{-}\textgreater{}}\NormalTok{ np2}\OperatorTok{\}]}
\end{Highlighting}
\end{Shaded}

\begin{dmath*}\breakingcomma
\left\{\text{FCTopology}\left(\text{topo1A},\left\{\frac{1}{(\text{q1}^2-m^2+i \eta )},\frac{1}{((\text{p1}+\text{q1})^2+i \eta )},\frac{1}{((\text{p2}+\text{q1})^2+i \eta )},\frac{1}{(n\cdot \;\text{q1}+i \eta )}\right\},\{\text{q1}\},\{\text{p1},\text{p2},n\},\{\text{Hold}[\text{SPD}][\text{p1}]\to 0,\text{Hold}[\text{SPD}][\text{p2}]\to 0,\text{Hold}[\text{SPD}][\text{p1},\text{p2}]\to 0,\text{Hold}[\text{SPD}][n]\to 0,\text{Hold}[\text{SPD}][n,\text{p1}]\to \;\text{np1},\text{Hold}[\text{SPD}][n,\text{p2}]\to \;\text{np2}\},\{\}\right),\text{FCGV}(\text{AddPropagators})(\{n\}) G^{\text{topo1}}(\text{n1$\_$},\text{n2$\_$},\text{n3$\_$}):\to G^{\text{topo1A}}(\text{n1},\text{n2},\text{n3},0)\right\}
\end{dmath*}

\begin{Shaded}
\begin{Highlighting}[]
\NormalTok{FCLoopAugmentTopology}\OperatorTok{[}\NormalTok{topo}\OperatorTok{,} \OperatorTok{\{}\NormalTok{SFAD}\OperatorTok{[\{\{}\DecValTok{0}\OperatorTok{,}\NormalTok{ q1 . }\FunctionTok{n}\OperatorTok{\}\}]\},} 
\NormalTok{  FinalSubstitutions }\OtherTok{{-}\textgreater{}} \OperatorTok{\{}\FunctionTok{Hold}\OperatorTok{[}\NormalTok{SPD}\OperatorTok{][}\FunctionTok{n}\OperatorTok{]} \OtherTok{{-}\textgreater{}} \DecValTok{0}\OperatorTok{,} \FunctionTok{Hold}\OperatorTok{[}\NormalTok{SPD}\OperatorTok{][}\FunctionTok{n}\OperatorTok{,}\NormalTok{ p1}\OperatorTok{]} \OtherTok{{-}\textgreater{}}\NormalTok{ np1}\OperatorTok{,} 
    \FunctionTok{Hold}\OperatorTok{[}\NormalTok{SPD}\OperatorTok{][}\FunctionTok{n}\OperatorTok{,}\NormalTok{ p2}\OperatorTok{]} \OtherTok{{-}\textgreater{}}\NormalTok{ np2}\OperatorTok{\},}\NormalTok{ AugmentedTopologyMarker }\OtherTok{{-}\textgreater{}} \ConstantTok{False}\OperatorTok{]}
\end{Highlighting}
\end{Shaded}

\begin{dmath*}\breakingcomma
\left\{\text{FCTopology}\left(\text{topo1A},\left\{\frac{1}{(\text{q1}^2-m^2+i \eta )},\frac{1}{((\text{p1}+\text{q1})^2+i \eta )},\frac{1}{((\text{p2}+\text{q1})^2+i \eta )},\frac{1}{(n\cdot \;\text{q1}+i \eta )}\right\},\{\text{q1}\},\{\text{p1},\text{p2},n\},\{\text{Hold}[\text{SPD}][\text{p1}]\to 0,\text{Hold}[\text{SPD}][\text{p2}]\to 0,\text{Hold}[\text{SPD}][\text{p1},\text{p2}]\to 0,\text{Hold}[\text{SPD}][n]\to 0,\text{Hold}[\text{SPD}][n,\text{p1}]\to \;\text{np1},\text{Hold}[\text{SPD}][n,\text{p2}]\to \;\text{np2}\},\{\}\right),G^{\text{topo1}}(\text{n1$\_$},\text{n2$\_$},\text{n3$\_$}):\to G^{\text{topo1A}}(\text{n1},\text{n2},\text{n3},0)\right\}
\end{dmath*}
\end{document}
