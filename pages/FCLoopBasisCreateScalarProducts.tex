% !TeX program = pdflatex
% !TeX root = FCLoopBasisCreateScalarProducts.tex

\documentclass[../FeynCalcManual.tex]{subfiles}
\begin{document}
\hypertarget{fcloopbasiscreatescalarproducts}{
\section{FCLoopBasisCreateScalarProducts}\label{fcloopbasiscreatescalarproducts}\index{FCLoopBasisCreateScalarProducts}}

\texttt{FCLoopBasisCreateScalarProducts \{\allowbreak{}q1,\ \allowbreak{}q2,\ \allowbreak{}...\},\ \allowbreak{}\{\allowbreak{}p1,\ \allowbreak{}p2,\ \allowbreak{}...\},\ \allowbreak{}\{\allowbreak{}d1,\ \allowbreak{}d2,\ \allowbreak{}...\},\ \allowbreak{}head]}
generates a list of all loop-momentum dependent scalar products made out
of the loop momenta \texttt{q1,\ \allowbreak{}q2,\ \allowbreak{}...} and
external momenta \texttt{p1,\ \allowbreak{}p2,\ \allowbreak{}...} in the
space-time dimensions \texttt{d1,\ \allowbreak{}d2,\ \allowbreak{}...}.
The argument \texttt{head} can be \texttt{Pair} to generate Lorentzian
scalar products or \texttt{CartesianPair} to generate Cartesian scalar
products.

\subsection{See also}

\hyperlink{toc}{Overview}

\subsection{Examples}

\begin{Shaded}
\begin{Highlighting}[]
\NormalTok{FCLoopBasisCreateScalarProducts}\OperatorTok{[\{}\FunctionTok{l}\OperatorTok{\},} \OperatorTok{\{\},} \OperatorTok{\{}\FunctionTok{D}\OperatorTok{\},}\NormalTok{ Pair}\OperatorTok{]}
\end{Highlighting}
\end{Shaded}

\begin{dmath*}\breakingcomma
\left\{l^2\right\}
\end{dmath*}

\begin{Shaded}
\begin{Highlighting}[]
\NormalTok{FCLoopBasisCreateScalarProducts}\OperatorTok{[\{}\FunctionTok{l}\OperatorTok{\},} \OperatorTok{\{}\NormalTok{p1}\OperatorTok{,}\NormalTok{ p2}\OperatorTok{\},} \OperatorTok{\{}\DecValTok{4}\OperatorTok{\},}\NormalTok{ Pair}\OperatorTok{]}
\end{Highlighting}
\end{Shaded}

\begin{dmath*}\breakingcomma
\left\{\overline{l}^2,\overline{l}\cdot \overline{\text{p1}},\overline{l}\cdot \overline{\text{p2}}\right\}
\end{dmath*}

\begin{Shaded}
\begin{Highlighting}[]
\NormalTok{FCLoopBasisCreateScalarProducts}\OperatorTok{[\{}\FunctionTok{l}\OperatorTok{\},} \OperatorTok{\{\},} \OperatorTok{\{}\FunctionTok{D} \SpecialCharTok{{-}} \DecValTok{1}\OperatorTok{\},}\NormalTok{ CartesianPair}\OperatorTok{]}
\end{Highlighting}
\end{Shaded}

\begin{dmath*}\breakingcomma
\left\{l^2\right\}
\end{dmath*}
\end{document}
