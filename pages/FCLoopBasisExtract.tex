% !TeX program = pdflatex
% !TeX root = FCLoopBasisExtract.tex

\documentclass[../FeynCalcManual.tex]{subfiles}
\begin{document}
\hypertarget{fcloopbasisextract}{
\section{FCLoopBasisExtract}\label{fcloopbasisextract}\index{FCLoopBasisExtract}}

\texttt{FCLoopBasisExtract[\allowbreak{}int,\ \allowbreak{}\{\allowbreak{}q1,\ \allowbreak{}q2,\ \allowbreak{}...\}]}
is an auxiliary function that extracts the scalar products that form the
basis of the loop integral in int. It needs to know the loop momenta on
which the integral depends and the dimensions of the momenta that may
occur in the integral.

\subsection{See also}

\hyperlink{toc}{Overview}

\subsection{Examples}

\begin{Shaded}
\begin{Highlighting}[]
\NormalTok{SPD}\OperatorTok{[}\FunctionTok{q}\OperatorTok{,} \FunctionTok{p}\OperatorTok{]}\NormalTok{ SFAD}\OperatorTok{[}\FunctionTok{q}\OperatorTok{,} \FunctionTok{q} \SpecialCharTok{{-}} \FunctionTok{p}\OperatorTok{,} \FunctionTok{q} \SpecialCharTok{{-}} \FunctionTok{p}\OperatorTok{]} 
 
\NormalTok{FCLoopBasisExtract}\OperatorTok{[}\SpecialCharTok{\%}\OperatorTok{,} \OperatorTok{\{}\FunctionTok{q}\OperatorTok{\},}\NormalTok{ SetDimensions }\OtherTok{{-}\textgreater{}} \OperatorTok{\{}\DecValTok{4}\OperatorTok{,} \FunctionTok{D}\OperatorTok{\}]}
\end{Highlighting}
\end{Shaded}

\begin{dmath*}\breakingcomma
\frac{p\cdot q}{(q^2+i \eta ).((q-p)^2+i \eta )^2}
\end{dmath*}

\begin{dmath*}\breakingcomma
\left\{\left\{p\cdot q,q^2,-2 (p\cdot q)+p^2+q^2\right\},\left\{p\cdot q,q^2\right\},\{-1,1,2\},\left\{p\cdot q,\frac{1}{(q^2+i \eta )},\frac{1}{((q-p)^2+i \eta )}\right\}\right\}
\end{dmath*}

\begin{Shaded}
\begin{Highlighting}[]
\NormalTok{SFAD}\OperatorTok{[}\NormalTok{p1}\OperatorTok{]} 
 
\NormalTok{FCLoopBasisExtract}\OperatorTok{[}\SpecialCharTok{\%}\OperatorTok{,} \OperatorTok{\{}\NormalTok{p1}\OperatorTok{,}\NormalTok{ p2}\OperatorTok{,}\NormalTok{ p3}\OperatorTok{\},}\NormalTok{ FCTopology }\OtherTok{{-}\textgreater{}} \ConstantTok{True}\OperatorTok{,}\NormalTok{ FCE }\OtherTok{{-}\textgreater{}} \ConstantTok{True}\OperatorTok{]}
\end{Highlighting}
\end{Shaded}

\begin{dmath*}\breakingcomma
\frac{1}{(\text{p1}^2+i \eta )}
\end{dmath*}

\begin{dmath*}\breakingcomma
\left\{\left\{\text{p1}^2\right\},\left\{\text{p1}^2,\text{p1}\cdot \;\text{p2},\text{p1}\cdot \;\text{p3},\text{p2}^2,\text{p2}\cdot \;\text{p3},\text{p3}^2\right\},\{1\},\left\{\frac{1}{(\text{p1}^2+i \eta )}\right\}\right\}
\end{dmath*}
\end{document}
