% !TeX program = pdflatex
% !TeX root = FCLoopBasisGetSize.tex

\documentclass[../FeynCalcManual.tex]{subfiles}
\begin{document}
\hypertarget{fcloopbasisgetsize}{
\section{FCLoopBasisGetSize}\label{fcloopbasisgetsize}\index{FCLoopBasisGetSize}}

\texttt{FCLoopBasisGetSize[\allowbreak{}n1,\ \allowbreak{}n2]} returns
the number of linearly independent propagators for a topology that
contains \texttt{n1} loop momenta and \texttt{n2} external momenta.

\subsection{See also}

\hyperlink{toc}{Overview}

\subsection{Examples}

\begin{Shaded}
\begin{Highlighting}[]
\NormalTok{FCLoopBasisGetSize}\OperatorTok{[}\DecValTok{1}\OperatorTok{,} \DecValTok{0}\OperatorTok{]}
\end{Highlighting}
\end{Shaded}

\begin{dmath*}\breakingcomma
1
\end{dmath*}

\begin{Shaded}
\begin{Highlighting}[]
\NormalTok{FCLoopBasisGetSize}\OperatorTok{[}\DecValTok{2}\OperatorTok{,} \DecValTok{1}\OperatorTok{]}
\end{Highlighting}
\end{Shaded}

\begin{dmath*}\breakingcomma
5
\end{dmath*}

\begin{Shaded}
\begin{Highlighting}[]
\NormalTok{FCLoopBasisGetSize}\OperatorTok{[}\DecValTok{3}\OperatorTok{,} \DecValTok{2}\OperatorTok{]}
\end{Highlighting}
\end{Shaded}

\begin{dmath*}\breakingcomma
12
\end{dmath*}

\begin{Shaded}
\begin{Highlighting}[]
\NormalTok{FCLoopBasisGetSize}\OperatorTok{[}\DecValTok{4}\OperatorTok{,} \DecValTok{1}\OperatorTok{]}
\end{Highlighting}
\end{Shaded}

\begin{dmath*}\breakingcomma
14
\end{dmath*}

The third argument (if given) is simply added to the final result.

\begin{Shaded}
\begin{Highlighting}[]
\NormalTok{FCLoopBasisGetSize}\OperatorTok{[}\DecValTok{4}\OperatorTok{,} \DecValTok{1}\OperatorTok{,} \DecValTok{1}\OperatorTok{]}
\end{Highlighting}
\end{Shaded}

\begin{dmath*}\breakingcomma
15
\end{dmath*}
\end{document}
