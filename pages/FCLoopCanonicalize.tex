% !TeX program = pdflatex
% !TeX root = FCLoopCanonicalize.tex

\documentclass[../FeynCalcManual.tex]{subfiles}
\begin{document}
\hypertarget{fcloopcanonicalize}{
\section{FCLoopCanonicalize}\label{fcloopcanonicalize}\index{FCLoopCanonicalize}}

\texttt{FCLoopCanonicalize[\allowbreak{}exp,\ \allowbreak{}q,\ \allowbreak{}loopHead]}
is an auxiliary internal function that canonicalizes indices of 1-loop
integrals with loop momentum \texttt{q} that are wrapped with
\texttt{loopHead}. The output is given as a list of 4 entries, of which
the last one contains a list of all the unique 1-loop integrals in the
given expression. After those are simplified, the original output of
\texttt{FCLoopCanonicalize} together with the list of the simplified
unique integrals should be inserted into \texttt{FCLoopSolutionList} to
obtain the final replacement list that will be applied to the original
expression.

\subsection{See also}

\hyperlink{toc}{Overview},
\hyperlink{fcloopsolutionlist}{FCLoopSolutionList}.

\subsection{Examples}

\begin{Shaded}
\begin{Highlighting}[]
\NormalTok{FCLoopCanonicalize}\OperatorTok{[}\NormalTok{myHead}\OperatorTok{[}\NormalTok{FVD}\OperatorTok{[}\FunctionTok{q}\OperatorTok{,} \SpecialCharTok{\textbackslash{}}\OperatorTok{[}\NormalTok{Mu}\OperatorTok{]]],} \FunctionTok{q}\OperatorTok{,}\NormalTok{ myHead}\OperatorTok{]}
\end{Highlighting}
\end{Shaded}

\begin{dmath*}\breakingcomma
\left(
\begin{array}{c}
 \;\text{myHead}\left(q^{\mu }\right) \\
 \{\text{FCGV}(\text{cli191})\to \mu \} \\
 \;\text{myHead}\left(q^{\text{FCGV}(\text{cli191})}\right) \\
 \;\text{myHead}\left(q^{\text{FCGV}(\text{cli191})}\right) \\
\end{array}
\right)
\end{dmath*}

\begin{Shaded}
\begin{Highlighting}[]
\NormalTok{FCLoopCanonicalize}\OperatorTok{[}\NormalTok{myHead}\OperatorTok{[}\NormalTok{FVD}\OperatorTok{[}\FunctionTok{q}\OperatorTok{,} \SpecialCharTok{\textbackslash{}}\OperatorTok{[}\NormalTok{Mu}\OperatorTok{]]}\NormalTok{ FVD}\OperatorTok{[}\FunctionTok{q}\OperatorTok{,} \SpecialCharTok{\textbackslash{}}\OperatorTok{[}\NormalTok{Nu}\OperatorTok{]]}\NormalTok{ FAD}\OperatorTok{[}\FunctionTok{q}\OperatorTok{,} \OperatorTok{\{}\FunctionTok{q} \SpecialCharTok{+} \FunctionTok{p}\OperatorTok{,} \FunctionTok{m}\OperatorTok{\}]]} \SpecialCharTok{+}\NormalTok{ myHead}\OperatorTok{[}\NormalTok{FVD}\OperatorTok{[}\FunctionTok{q}\OperatorTok{,} \SpecialCharTok{\textbackslash{}}\OperatorTok{[}\NormalTok{Rho}\OperatorTok{]]}\NormalTok{ FVD}\OperatorTok{[}\FunctionTok{q}\OperatorTok{,} \SpecialCharTok{\textbackslash{}}\OperatorTok{[}\NormalTok{Sigma}\OperatorTok{]]}\NormalTok{ FAD}\OperatorTok{[}\FunctionTok{q}\OperatorTok{,} \OperatorTok{\{}\FunctionTok{q} \SpecialCharTok{+} \FunctionTok{p}\OperatorTok{,} \FunctionTok{m}\OperatorTok{\}]],} \FunctionTok{q}\OperatorTok{,}\NormalTok{ myHead}\OperatorTok{]}
\end{Highlighting}
\end{Shaded}

\begin{dmath*}\breakingcomma
\left\{\left\{\text{myHead}\left(\frac{q^{\mu } q^{\nu }}{q^2.\left((p+q)^2-m^2\right)}\right),\text{myHead}\left(\frac{q^{\rho } q^{\sigma }}{q^2.\left((p+q)^2-m^2\right)}\right)\right\},\{\{\text{FCGV}(\text{cli201})\to \mu ,\text{FCGV}(\text{cli202})\to \nu \},\{\text{FCGV}(\text{cli201})\to \rho ,\text{FCGV}(\text{cli202})\to \sigma \}\},\left\{\text{myHead}\left(\frac{q^{\text{FCGV}(\text{cli201})} q^{\text{FCGV}(\text{cli202})}}{q^2.\left((p+q)^2-m^2\right)}\right),\text{myHead}\left(\frac{q^{\text{FCGV}(\text{cli201})} q^{\text{FCGV}(\text{cli202})}}{q^2.\left((p+q)^2-m^2\right)}\right)\right\},\left\{\text{myHead}\left(\frac{q^{\text{FCGV}(\text{cli201})} q^{\text{FCGV}(\text{cli202})}}{q^2.\left((p+q)^2-m^2\right)}\right)\right\}\right\}
\end{dmath*}
\end{document}
