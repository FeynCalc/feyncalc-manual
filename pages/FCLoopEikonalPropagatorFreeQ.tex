% !TeX program = pdflatex
% !TeX root = FCLoopEikonalPropagatorFreeQ.tex

\documentclass[../FeynCalcManual.tex]{subfiles}
\begin{document}
\hypertarget{fcloopeikonalpropagatorfreeq}{%
\section{FCLoopEikonalPropagatorFreeQ}\label{fcloopeikonalpropagatorfreeq}}

\texttt{FCLoopEikonalPropagatorFreeQ[\allowbreak{}exp]} checks if the
integral is free of eikonal propagators \(\frac{1}{p \cdot q+x}\). If
the option \texttt{First} is set to \texttt{False}, propagators that
have both a quadratic and linear piece,
e.g.~\(\frac{1}{p^2 + p \cdot q+x}\) will also count as eikonal
propagators. The option \texttt{Momentum} can be used to check for the
presence of eikonal propagators only with respect to particular momenta.
The check is performed only for \texttt{StandardPropagatorDenominator}
and \texttt{CartesianPropagatorDenominator}.

\subsection{See also}

\hyperlink{toc}{Overview}

\subsection{Examples}

\begin{Shaded}
\begin{Highlighting}[]
\NormalTok{FCI@SFAD}\OperatorTok{[}\FunctionTok{p}\OperatorTok{,} \FunctionTok{p} \SpecialCharTok{{-}} \FunctionTok{q}\OperatorTok{]} 
 
\NormalTok{FCLoopEikonalPropagatorFreeQ}\OperatorTok{[}\SpecialCharTok{\%}\OperatorTok{]}
\end{Highlighting}
\end{Shaded}

\begin{dmath*}\breakingcomma
\frac{1}{(p^2+i \eta ).((p-q)^2+i \eta )}
\end{dmath*}

\begin{dmath*}\breakingcomma
\text{True}
\end{dmath*}

\begin{Shaded}
\begin{Highlighting}[]
\NormalTok{FCI@SFAD}\OperatorTok{[\{\{}\DecValTok{0}\OperatorTok{,} \FunctionTok{p}\NormalTok{ . }\FunctionTok{q}\OperatorTok{\}\}]} 
 
\NormalTok{FCLoopEikonalPropagatorFreeQ}\OperatorTok{[}\SpecialCharTok{\%}\OperatorTok{]}
\end{Highlighting}
\end{Shaded}

\begin{dmath*}\breakingcomma
\frac{1}{(p\cdot q+i \eta )}
\end{dmath*}

\begin{dmath*}\breakingcomma
\text{False}
\end{dmath*}

\begin{Shaded}
\begin{Highlighting}[]
\NormalTok{FCI@CFAD}\OperatorTok{[\{\{}\DecValTok{0}\OperatorTok{,} \FunctionTok{p}\NormalTok{ . }\FunctionTok{q}\OperatorTok{\}\}]} 
 
\NormalTok{FCLoopEikonalPropagatorFreeQ}\OperatorTok{[}\SpecialCharTok{\%}\OperatorTok{,}\NormalTok{ Momentum }\OtherTok{{-}\textgreater{}} \OperatorTok{\{}\FunctionTok{q}\OperatorTok{\}]}
\end{Highlighting}
\end{Shaded}

\begin{dmath*}\breakingcomma
\frac{1}{(p\cdot q-i \eta )}
\end{dmath*}

\begin{dmath*}\breakingcomma
\text{False}
\end{dmath*}

\begin{Shaded}
\begin{Highlighting}[]
\NormalTok{FCI@SFAD}\OperatorTok{[\{\{}\FunctionTok{q}\OperatorTok{,} \FunctionTok{q}\NormalTok{ . }\FunctionTok{p}\OperatorTok{\}\}]} 
 
\NormalTok{FCLoopEikonalPropagatorFreeQ}\OperatorTok{[}\SpecialCharTok{\%}\OperatorTok{,} \FunctionTok{First} \OtherTok{{-}\textgreater{}} \ConstantTok{False}\OperatorTok{]}
\end{Highlighting}
\end{Shaded}

\begin{dmath*}\breakingcomma
\frac{1}{(q^2+p\cdot q+i \eta )}
\end{dmath*}

\begin{dmath*}\breakingcomma
\text{False}
\end{dmath*}
\end{document}
