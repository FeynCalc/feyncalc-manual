% !TeX program = pdflatex
% !TeX root = FCLoopExtract.tex

\documentclass[../FeynCalcManual.tex]{subfiles}
\begin{document}
\hypertarget{fcloopextract}{
\section{FCLoopExtract}\label{fcloopextract}\index{FCLoopExtract}}

\texttt{FCLoopExtract[\allowbreak{}expr,\ \allowbreak{}\{\allowbreak{}q1,\ \allowbreak{}q2,\ \allowbreak{}...\},\ \allowbreak{}loopHead]}
exctracts loop integrals that depend on
\texttt{q1,\ \allowbreak{}q2,\ \allowbreak{}...} from the given
expression. The output is given as a list of three entries. The first
one contains part of the original expression that consists of irrelevant
loop integrals and terms that are free of any loop integrals. The second
entry contains relevant loop integrals, where each integral is wrapped
into \texttt{loopHead}. The third entry is a list of all the unique loop
integrals from the second entry and can be used as an input to another
function. Note that if loop integrals contain free indices, those will
not be canonicalized.

\subsection{See also}

\hyperlink{toc}{Overview}, \hyperlink{fcloopisolate}{FCLoopIsolate}.

\subsection{Examples}

\begin{Shaded}
\begin{Highlighting}[]
\NormalTok{FCI}\OperatorTok{[}\NormalTok{GSD}\OperatorTok{[}\FunctionTok{q} \SpecialCharTok{{-}}\NormalTok{ p1}\OperatorTok{]}\NormalTok{ . (GSD}\OperatorTok{[}\FunctionTok{q} \SpecialCharTok{{-}}\NormalTok{ p2}\OperatorTok{]} \SpecialCharTok{+} \FunctionTok{M}\NormalTok{) . GSD}\OperatorTok{[}\NormalTok{p3}\OperatorTok{]}\NormalTok{ SPD}\OperatorTok{[}\FunctionTok{q}\OperatorTok{,}\NormalTok{ p2}\OperatorTok{]}\NormalTok{ FAD}\OperatorTok{[}\FunctionTok{q}\OperatorTok{,} \FunctionTok{q} \SpecialCharTok{{-}}\NormalTok{ p1}\OperatorTok{,} \OperatorTok{\{}\FunctionTok{q} \SpecialCharTok{{-}}\NormalTok{ p2}\OperatorTok{,} \FunctionTok{m}\OperatorTok{\}]]} 
 
\NormalTok{FCLoopExtract}\OperatorTok{[}\SpecialCharTok{\%}\OperatorTok{,} \OperatorTok{\{}\FunctionTok{q}\OperatorTok{\},}\NormalTok{ loopInt}\OperatorTok{]}
\end{Highlighting}
\end{Shaded}

\begin{dmath*}\breakingcomma
\frac{(\text{p2}\cdot q) (\gamma \cdot (q-\text{p1})).(M+\gamma \cdot (q-\text{p2})).(\gamma \cdot \;\text{p3})}{q^2.(q-\text{p1})^2.\left((q-\text{p2})^2-m^2\right)}
\end{dmath*}

\begin{dmath*}\breakingcomma
\left\{0,((\gamma \cdot \;\text{p1}).(\gamma \cdot \;\text{p2}).(\gamma \cdot \;\text{p3})-M (\gamma \cdot \;\text{p1}).(\gamma \cdot \;\text{p3})) \;\text{loopInt}\left(\frac{\text{p2}\cdot q}{q^2.(q-\text{p1})^2.\left((q-\text{p2})^2-m^2\right)}\right)+M \;\text{loopInt}\left(\frac{(\text{p2}\cdot q) (\gamma \cdot q).(\gamma \cdot \;\text{p3})}{q^2.(q-\text{p1})^2.\left((q-\text{p2})^2-m^2\right)}\right)-\text{loopInt}\left(\frac{(\text{p2}\cdot q) (\gamma \cdot \;\text{p1}).(\gamma \cdot q).(\gamma \cdot \;\text{p3})}{q^2.(q-\text{p1})^2.\left((q-\text{p2})^2-m^2\right)}\right)-\text{loopInt}\left(\frac{(\text{p2}\cdot q) (\gamma \cdot q).(\gamma \cdot \;\text{p2}).(\gamma \cdot \;\text{p3})}{q^2.(q-\text{p1})^2.\left((q-\text{p2})^2-m^2\right)}\right)+\text{loopInt}\left(\frac{(\text{p2}\cdot q) (\gamma \cdot q).(\gamma \cdot q).(\gamma \cdot \;\text{p3})}{q^2.(q-\text{p1})^2.\left((q-\text{p2})^2-m^2\right)}\right),\left\{\text{loopInt}\left(\frac{\text{p2}\cdot q}{q^2.(q-\text{p1})^2.\left((q-\text{p2})^2-m^2\right)}\right),\text{loopInt}\left(\frac{(\text{p2}\cdot q) (\gamma \cdot q).(\gamma \cdot \;\text{p3})}{q^2.(q-\text{p1})^2.\left((q-\text{p2})^2-m^2\right)}\right),\text{loopInt}\left(\frac{(\text{p2}\cdot q) (\gamma \cdot \;\text{p1}).(\gamma \cdot q).(\gamma \cdot \;\text{p3})}{q^2.(q-\text{p1})^2.\left((q-\text{p2})^2-m^2\right)}\right),\text{loopInt}\left(\frac{(\text{p2}\cdot q) (\gamma \cdot q).(\gamma \cdot \;\text{p2}).(\gamma \cdot \;\text{p3})}{q^2.(q-\text{p1})^2.\left((q-\text{p2})^2-m^2\right)}\right),\text{loopInt}\left(\frac{(\text{p2}\cdot q) (\gamma \cdot q).(\gamma \cdot q).(\gamma \cdot \;\text{p3})}{q^2.(q-\text{p1})^2.\left((q-\text{p2})^2-m^2\right)}\right)\right\}\right\}
\end{dmath*}
\end{document}
