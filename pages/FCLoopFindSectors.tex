% !TeX program = pdflatex
% !TeX root = FCLoopFindSectors.tex

\documentclass[../FeynCalcManual.tex]{subfiles}
\begin{document}
\begin{Shaded}
\begin{Highlighting}[]
 
\end{Highlighting}
\end{Shaded}

\hypertarget{fcloopfindsectors}{
\section{FCLoopFindSectors}\label{fcloopfindsectors}\index{FCLoopFindSectors}}

\texttt{FCLoopFindSectors[\allowbreak{}\{\allowbreak{}GLI[\allowbreak{}...],\ \allowbreak{}...\}]}
analyzes the indices of the GLI integrals in the given list and
identifies sectors to which they belong. Notice that only \texttt{GLI}s
with integer indices are supported.

If the option \texttt{GatherBy} is set to \texttt{True} (default), the
output will be a list of two lists, where the former contains the
original integrals sorted w.r.t the identified sectors, while the latter
is a list of all available sectors.

For \texttt{GatherBy->False}, the output is a list containing all
identified sectors without the original integrals.

Setting the option \texttt{Last} to \texttt{True}will return only the
top sector.

\subsection{See also}

\hyperlink{toc}{Overview}.

\subsection{Examples}

\begin{Shaded}
\begin{Highlighting}[]
\NormalTok{ints }\ExtensionTok{=} \OperatorTok{\{}
\NormalTok{    GLI}\OperatorTok{[}\NormalTok{topo1}\OperatorTok{,} \OperatorTok{\{}\DecValTok{1}\OperatorTok{,} \DecValTok{1}\OperatorTok{,} \DecValTok{1}\OperatorTok{,} \DecValTok{1}\OperatorTok{\}],} 
\NormalTok{    GLI}\OperatorTok{[}\NormalTok{topo1}\OperatorTok{,} \OperatorTok{\{}\DecValTok{2}\OperatorTok{,} \DecValTok{1}\OperatorTok{,} \DecValTok{2}\OperatorTok{,} \DecValTok{1}\OperatorTok{\}],} 
\NormalTok{    GLI}\OperatorTok{[}\NormalTok{topo2}\OperatorTok{,} \OperatorTok{\{}\DecValTok{1}\OperatorTok{,} \DecValTok{0}\OperatorTok{,} \DecValTok{1}\OperatorTok{,} \DecValTok{1}\OperatorTok{\}],} 
\NormalTok{    GLI}\OperatorTok{[}\NormalTok{topo3}\OperatorTok{,} \OperatorTok{\{}\DecValTok{1}\OperatorTok{,} \DecValTok{0}\OperatorTok{,} \DecValTok{1}\OperatorTok{,} \SpecialCharTok{{-}}\DecValTok{1}\OperatorTok{\}]} 
   \OperatorTok{\}}\NormalTok{;}
\end{Highlighting}
\end{Shaded}

\begin{Shaded}
\begin{Highlighting}[]
\NormalTok{FCLoopFindSectors}\OperatorTok{[}\NormalTok{ints}\OperatorTok{]}
\end{Highlighting}
\end{Shaded}

\begin{dmath*}\breakingcomma
\left(
\begin{array}{ccc}
 \left\{\{1,0,1,0\},\left\{G^{\text{topo3}}(1,0,1,-1)\right\}\right\} & \left\{\{1,0,1,1\},\left\{G^{\text{topo2}}(1,0,1,1)\right\}\right\} & \left\{\{1,1,1,1\},\left\{G^{\text{topo1}}(1,1,1,1),G^{\text{topo1}}(2,1,2,1)\right\}\right\} \\
 \{1,0,1,0\} & \{1,0,1,1\} & \{1,1,1,1\} \\
\end{array}
\right)
\end{dmath*}

\begin{Shaded}
\begin{Highlighting}[]
\NormalTok{FCLoopFindSectors}\OperatorTok{[}\NormalTok{ints}\OperatorTok{,} \FunctionTok{Last} \OtherTok{{-}\textgreater{}} \ConstantTok{True}\OperatorTok{]}
\end{Highlighting}
\end{Shaded}

\begin{dmath*}\breakingcomma
\left\{\{1,1,1,1\},\left\{G^{\text{topo1}}(1,1,1,1),G^{\text{topo1}}(2,1,2,1)\right\}\right\}
\end{dmath*}

\begin{Shaded}
\begin{Highlighting}[]
\NormalTok{FCLoopFindSectors}\OperatorTok{[}\NormalTok{ints}\OperatorTok{,} \FunctionTok{GatherBy} \OtherTok{{-}\textgreater{}} \ConstantTok{False}\OperatorTok{]}
\end{Highlighting}
\end{Shaded}

\begin{dmath*}\breakingcomma
\left(
\begin{array}{cccc}
 1 & 0 & 1 & 0 \\
 1 & 0 & 1 & 1 \\
 1 & 1 & 1 & 1 \\
\end{array}
\right)
\end{dmath*}

\begin{Shaded}
\begin{Highlighting}[]
\NormalTok{FCLoopFindSectors}\OperatorTok{[}\NormalTok{ints}\OperatorTok{,} \FunctionTok{GatherBy} \OtherTok{{-}\textgreater{}} \ConstantTok{False}\OperatorTok{,} \FunctionTok{Last} \OtherTok{{-}\textgreater{}} \ConstantTok{True}\OperatorTok{]}
\end{Highlighting}
\end{Shaded}

\begin{dmath*}\breakingcomma
\{1,1,1,1\}
\end{dmath*}
\end{document}
