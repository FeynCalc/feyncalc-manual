% !TeX program = pdflatex
% !TeX root = FCLoopFindTopologies.tex

\documentclass[../FeynCalcManual.tex]{subfiles}
\begin{document}
\hypertarget{fcloopfindtopologies}{%
\section{FCLoopFindTopologies}\label{fcloopfindtopologies}}

\texttt{FCLoopFindTopologies[\allowbreak{}exp,\ \allowbreak{}\{\allowbreak{}q1,\ \allowbreak{}q2,\ \allowbreak{}...\}]}
attempts to identify the loop integral topologies present in
\texttt{exp} by looking at the propagator denominators that depend on
the loop momenta \texttt{q1,\ \allowbreak{}q2,\ \allowbreak{}...} . It
returns a list of two entries, where the first one is the original
expression with the denominators rewritten as \texttt{GLI}s, and the
second one is the set of the identified topologies. Each of the
identified topologies must contain linearly independent propagators
(unless the option \texttt{FCLoopBasisOverdeterminedQ} is set to True),
but may lack propagators needed to form a complete basis.

\subsection{See also}

\hyperlink{toc}{Overview}, \hyperlink{fctopology}{FCTopology},
\hyperlink{gli}{GLI}.

\subsection{Examples}

Find topologies occurring in the 2-loop ghost self-energy amplitude

\begin{Shaded}
\begin{Highlighting}[]
\NormalTok{amp }\ExtensionTok{=} \FunctionTok{Get}\OperatorTok{[}\FunctionTok{FileNameJoin}\OperatorTok{[\{}\NormalTok{$FeynCalcDirectory}\OperatorTok{,} \StringTok{"Documentation"}\OperatorTok{,} \StringTok{"Examples"}\OperatorTok{,} 
      \StringTok{"Amplitudes"}\OperatorTok{,} \StringTok{"Gh{-}Gh{-}2L.m"}\OperatorTok{\}]]}\NormalTok{;}
\end{Highlighting}
\end{Shaded}

\begin{Shaded}
\begin{Highlighting}[]
\NormalTok{res }\ExtensionTok{=}\NormalTok{ FCLoopFindTopologies}\OperatorTok{[}\NormalTok{amp}\OperatorTok{,} \OperatorTok{\{}\NormalTok{q1}\OperatorTok{,}\NormalTok{ q2}\OperatorTok{\}]}\NormalTok{;}
\end{Highlighting}
\end{Shaded}

\begin{dmath*}\breakingcomma
\text{Number of the initial candidate topologies: }3
\end{dmath*}

\begin{dmath*}\breakingcomma
\text{Number of the identified unique topologies: }3
\end{dmath*}

\begin{dmath*}\breakingcomma
\text{Number of the preferred topologies among the unique topologies: }0
\end{dmath*}

\begin{dmath*}\breakingcomma
\text{Number of the identified subtopologies: }0
\end{dmath*}

\begin{Shaded}
\begin{Highlighting}[]
\NormalTok{res }\SpecialCharTok{//} \FunctionTok{Last}
\end{Highlighting}
\end{Shaded}

\begin{dmath*}\breakingcomma
\left\{\text{FCTopology}\left(\text{fctopology1},\left\{\frac{1}{(\text{q2}^2+i \eta )},\frac{1}{(\text{q1}^2+i \eta )},\frac{1}{((\text{q1}+\text{q2})^2+i \eta )},\frac{1}{((p+\text{q1})^2+i \eta )},\frac{1}{((p-\text{q2})^2+i \eta )}\right\},\{\text{q1},\text{q2}\},\{p\},\{\},\{\}\right),\text{FCTopology}\left(\text{fctopology2},\left\{\frac{1}{(\text{q2}^2+i \eta )},\frac{1}{(\text{q1}^2+i \eta )},\frac{1}{((p+\text{q2})^2+i \eta )},\frac{1}{((p-\text{q1})^2+i \eta )}\right\},\{\text{q1},\text{q2}\},\{p\},\{\},\{\}\right),\text{FCTopology}\left(\text{fctopology3},\left\{\frac{1}{(\text{q2}^2+i \eta )},\frac{1}{(\text{q1}^2+i \eta )},\frac{1}{((p-\text{q1})^2+i \eta )},\frac{1}{((p-\text{q1}+\text{q2})^2+i \eta )}\right\},\{\text{q1},\text{q2}\},\{p\},\{\},\{\}\right)\right\}
\end{dmath*}
\end{document}
