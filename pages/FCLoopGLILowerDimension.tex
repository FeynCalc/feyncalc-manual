% !TeX program = pdflatex
% !TeX root = FCLoopGLILowerDimension.tex

\documentclass[../FeynCalcManual.tex]{subfiles}
\begin{document}
\hypertarget{fcloopglilowerdimension}{
\section{FCLoopGLILowerDimension}\label{fcloopglilowerdimension}\index{FCLoopGLILowerDimension}}

\texttt{FCLoopGLILowerDimension[\allowbreak{}gli,\ \allowbreak{}topo]}
lowers the dimension of the given \texttt{GLI} from \texttt{D} to
\texttt{D-2} and expresses it in terms of \texttt{D}-dimensional loop
integrals returned in the output.

The algorithm is based on the code of the function \texttt{RaisingDRR}
from R. Lee's LiteRed

\subsection{See also}

\hyperlink{toc}{Overview},
\hyperlink{fcloopgliraisedimension}{FCLoopGLIRaiseDimension}.

\subsection{Examples}

\begin{Shaded}
\begin{Highlighting}[]
\NormalTok{topo }\ExtensionTok{=}\NormalTok{ FCTopology}\OperatorTok{[}
\NormalTok{   topo1}\OperatorTok{,} \OperatorTok{\{}\NormalTok{SFAD}\OperatorTok{[}\NormalTok{p1}\OperatorTok{],}\NormalTok{ SFAD}\OperatorTok{[}\NormalTok{p2}\OperatorTok{],}\NormalTok{ SFAD}\OperatorTok{[}\FunctionTok{Q} \SpecialCharTok{{-}}\NormalTok{ p1 }\SpecialCharTok{{-}}\NormalTok{ p2}\OperatorTok{],}\NormalTok{ SFAD}\OperatorTok{[}\FunctionTok{Q} \SpecialCharTok{{-}}\NormalTok{ p2}\OperatorTok{],} 
\NormalTok{    SFAD}\OperatorTok{[}\FunctionTok{Q} \SpecialCharTok{{-}}\NormalTok{ p1}\OperatorTok{]\},} \OperatorTok{\{}\NormalTok{p1}\OperatorTok{,}\NormalTok{ p2}\OperatorTok{\},} \OperatorTok{\{}\FunctionTok{Q}\OperatorTok{\},} \OperatorTok{\{}\FunctionTok{Hold}\OperatorTok{[}\NormalTok{SPD}\OperatorTok{[}\FunctionTok{Q}\OperatorTok{]]} \OtherTok{{-}\textgreater{}}\NormalTok{ qq}\OperatorTok{\},} \OperatorTok{\{\}]}
\end{Highlighting}
\end{Shaded}

\begin{dmath*}\breakingcomma
\text{FCTopology}\left(\text{topo1},\left\{\frac{1}{(\text{p1}^2+i \eta )},\frac{1}{(\text{p2}^2+i \eta )},\frac{1}{((-\text{p1}-\text{p2}+Q)^2+i \eta )},\frac{1}{((Q-\text{p2})^2+i \eta )},\frac{1}{((Q-\text{p1})^2+i \eta )}\right\},\{\text{p1},\text{p2}\},\{Q\},\{\text{Hold}[\text{SPD}(Q)]\to \;\text{qq}\},\{\}\right)
\end{dmath*}

\begin{Shaded}
\begin{Highlighting}[]
\NormalTok{FCLoopGLILowerDimension}\OperatorTok{[}\NormalTok{GLI}\OperatorTok{[}\NormalTok{topo1}\OperatorTok{,} \OperatorTok{\{}\DecValTok{1}\OperatorTok{,} \DecValTok{1}\OperatorTok{,} \DecValTok{1}\OperatorTok{,} \DecValTok{1}\OperatorTok{,} \DecValTok{1}\OperatorTok{\}],}\NormalTok{ topo}\OperatorTok{]}
\end{Highlighting}
\end{Shaded}

\begin{dmath*}\breakingcomma
G^{\text{topo1}}(1,1,1,2,2)+G^{\text{topo1}}(1,1,2,1,2)+G^{\text{topo1}}(1,1,2,2,1)+G^{\text{topo1}}(1,2,1,1,2)+G^{\text{topo1}}(1,2,2,1,1)+G^{\text{topo1}}(2,1,1,2,1)+G^{\text{topo1}}(2,1,2,1,1)+G^{\text{topo1}}(2,2,1,1,1)
\end{dmath*}

\begin{Shaded}
\begin{Highlighting}[]
\NormalTok{FCLoopGLILowerDimension}\OperatorTok{[}\NormalTok{GLI}\OperatorTok{[}\NormalTok{topo1}\OperatorTok{,} \OperatorTok{\{}\NormalTok{n1}\OperatorTok{,}\NormalTok{ n2}\OperatorTok{,}\NormalTok{ n3}\OperatorTok{,} \DecValTok{1}\OperatorTok{,} \DecValTok{1}\OperatorTok{\}],}\NormalTok{ topo}\OperatorTok{]}
\end{Highlighting}
\end{Shaded}

\begin{dmath*}\breakingcomma
G^{\text{topo1}}(\text{n1},\text{n2},\text{n3},2,2)+\text{n3} G^{\text{topo1}}(\text{n1},\text{n2},\text{n3}+1,1,2)+\text{n3} G^{\text{topo1}}(\text{n1},\text{n2},\text{n3}+1,2,1)+\text{n2} G^{\text{topo1}}(\text{n1},\text{n2}+1,\text{n3},1,2)+\text{n2} \;\text{n3} G^{\text{topo1}}(\text{n1},\text{n2}+1,\text{n3}+1,1,1)+\text{n1} G^{\text{topo1}}(\text{n1}+1,\text{n2},\text{n3},2,1)+\text{n1} \;\text{n3} G^{\text{topo1}}(\text{n1}+1,\text{n2},\text{n3}+1,1,1)+\text{n1} \;\text{n2} G^{\text{topo1}}(\text{n1}+1,\text{n2}+1,\text{n3},1,1)
\end{dmath*}
\end{document}
