% !TeX program = pdflatex
% !TeX root = FCLoopGetEtaSigns.tex

\documentclass[../FeynCalcManual.tex]{subfiles}
\begin{document}
\hypertarget{fcloopgetetasigns}{
\section{FCLoopGetEtaSigns}\label{fcloopgetetasigns}\index{FCLoopGetEtaSigns}}

\texttt{FCLoopGetEtaSigns[\allowbreak{}exp]} is an auxiliary function
that extracts the signs of \(i \eta\) from propagators present in the
input expression. The result is returned as a list,
e.g.~\texttt{\{\allowbreak{}\}}, \texttt{\{\allowbreak{}1\}},
\texttt{\{\allowbreak{}-1\}} or
\texttt{\{\allowbreak{}-1,\ \allowbreak{}1\}}.

This is useful if one wants ensure that all propagators of the given
integral or topology follow a particular \(i \eta\) sign convention.

\subsection{See also}

\hyperlink{toc}{Overview}, \hyperlink{fctopology}{FCTopology},
\hyperlink{fcloopswitchetasign}{FCLoopSwitchEtaSign}.

\subsection{Examples}

\begin{Shaded}
\begin{Highlighting}[]
\NormalTok{FAD}\OperatorTok{[\{}\FunctionTok{p}\OperatorTok{,} \FunctionTok{m}\OperatorTok{\}]} 
 
\NormalTok{FCLoopGetEtaSigns}\OperatorTok{[}\SpecialCharTok{\%}\OperatorTok{]}
\end{Highlighting}
\end{Shaded}

\begin{dmath*}\breakingcomma
\frac{1}{p^2-m^2}
\end{dmath*}

\begin{dmath*}\breakingcomma
\{1\}
\end{dmath*}

\begin{Shaded}
\begin{Highlighting}[]
\NormalTok{SFAD}\OperatorTok{[\{}\FunctionTok{p}\OperatorTok{,} \FunctionTok{m}\SpecialCharTok{\^{}}\DecValTok{2}\OperatorTok{\}]} 
 
\NormalTok{FCLoopGetEtaSigns}\OperatorTok{[}\SpecialCharTok{\%}\OperatorTok{]}
\end{Highlighting}
\end{Shaded}

\begin{dmath*}\breakingcomma
\frac{1}{(p^2-m^2+i \eta )}
\end{dmath*}

\begin{dmath*}\breakingcomma
\{1\}
\end{dmath*}

\begin{Shaded}
\begin{Highlighting}[]
\NormalTok{SFAD}\OperatorTok{[\{}\FunctionTok{I} \FunctionTok{p}\OperatorTok{,} \SpecialCharTok{{-}}\FunctionTok{m}\SpecialCharTok{\^{}}\DecValTok{2}\OperatorTok{\},}\NormalTok{ EtaSign }\OtherTok{{-}\textgreater{}} \SpecialCharTok{{-}}\DecValTok{1}\OperatorTok{]} 
 
\NormalTok{FCLoopGetEtaSigns}\OperatorTok{[}\SpecialCharTok{\%}\OperatorTok{]}
\end{Highlighting}
\end{Shaded}

\begin{dmath*}\breakingcomma
\frac{1}{(-p^2+m^2-i \eta )}
\end{dmath*}

\begin{dmath*}\breakingcomma
\{-1\}
\end{dmath*}

\begin{Shaded}
\begin{Highlighting}[]
\NormalTok{CFAD}\OperatorTok{[\{}\FunctionTok{p}\OperatorTok{,} \FunctionTok{m}\SpecialCharTok{\^{}}\DecValTok{2}\OperatorTok{\}]} 
 
\NormalTok{FCLoopGetEtaSigns}\OperatorTok{[}\SpecialCharTok{\%}\OperatorTok{]}
\end{Highlighting}
\end{Shaded}

\begin{dmath*}\breakingcomma
\frac{1}{(p^2+m^2-i \eta )}
\end{dmath*}

\begin{dmath*}\breakingcomma
\{-1\}
\end{dmath*}
\end{document}
