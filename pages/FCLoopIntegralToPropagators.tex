% !TeX program = pdflatex
% !TeX root = FCLoopIntegralToPropagators.tex

\documentclass[../FeynCalcManual.tex]{subfiles}
\begin{document}
\hypertarget{fcloopintegraltopropagators}{
\section{FCLoopIntegralToPropagators}\label{fcloopintegraltopropagators}\index{FCLoopIntegralToPropagators}}

\texttt{FCLoopIntegralToPropagators[\allowbreak{}int,\ \allowbreak{}\{\allowbreak{}q1,\ \allowbreak{}q2,\ \allowbreak{}...\}]}
is an auxiliary function that converts the loop integral \texttt{int}
that depends on the loop momenta
\texttt{q1,\ \allowbreak{}q2,\ \allowbreak{}...} to a list of
propagators and scalar products.

All propagators and scalar products that do not depend on the loop
momenta are discarded, unless the \texttt{Rest} option is set to
\texttt{True}.

\subsection{See also}

\hyperlink{toc}{Overview},
\hyperlink{fclooppropagatorstotopology}{FCLoopPropagatorsToTopology}

\subsection{Examples}

\begin{Shaded}
\begin{Highlighting}[]
\NormalTok{SFAD}\OperatorTok{[}\NormalTok{p1}\OperatorTok{]} 
 
\NormalTok{FCLoopIntegralToPropagators}\OperatorTok{[}\SpecialCharTok{\%}\OperatorTok{,} \OperatorTok{\{}\NormalTok{p1}\OperatorTok{\}]}
\end{Highlighting}
\end{Shaded}

\begin{dmath*}\breakingcomma
\frac{1}{(\text{p1}^2+i \eta )}
\end{dmath*}

\begin{dmath*}\breakingcomma
\left\{\frac{1}{(\text{p1}^2+i \eta )}\right\}
\end{dmath*}

\begin{Shaded}
\begin{Highlighting}[]
\NormalTok{SFAD}\OperatorTok{[}\NormalTok{p1}\OperatorTok{,}\NormalTok{ p2}\OperatorTok{]} 
 
\NormalTok{FCLoopIntegralToPropagators}\OperatorTok{[}\SpecialCharTok{\%}\OperatorTok{,} \OperatorTok{\{}\NormalTok{p1}\OperatorTok{,}\NormalTok{ p2}\OperatorTok{\}]}
\end{Highlighting}
\end{Shaded}

\begin{dmath*}\breakingcomma
\frac{1}{(\text{p1}^2+i \eta ).(\text{p2}^2+i \eta )}
\end{dmath*}

\begin{dmath*}\breakingcomma
\left\{\frac{1}{(\text{p1}^2+i \eta )},\frac{1}{(\text{p2}^2+i \eta )}\right\}
\end{dmath*}

If the integral contains propagators raised to integer powers, only one
propagator will appear in the output.

\begin{Shaded}
\begin{Highlighting}[]
\NormalTok{int }\ExtensionTok{=}\NormalTok{ SPD}\OperatorTok{[}\FunctionTok{q}\OperatorTok{,} \FunctionTok{p}\OperatorTok{]}\NormalTok{ SFAD}\OperatorTok{[}\FunctionTok{q}\OperatorTok{,} \FunctionTok{q} \SpecialCharTok{{-}} \FunctionTok{p}\OperatorTok{,} \FunctionTok{q} \SpecialCharTok{{-}} \FunctionTok{p}\OperatorTok{]}
\end{Highlighting}
\end{Shaded}

\begin{dmath*}\breakingcomma
\frac{p\cdot q}{(q^2+i \eta ).((q-p)^2+i \eta )^2}
\end{dmath*}

\begin{Shaded}
\begin{Highlighting}[]
\NormalTok{FCLoopIntegralToPropagators}\OperatorTok{[}\NormalTok{int}\OperatorTok{,} \OperatorTok{\{}\FunctionTok{q}\OperatorTok{\}]}
\end{Highlighting}
\end{Shaded}

\begin{dmath*}\breakingcomma
\left\{(p\cdot q+i \eta ),\frac{1}{(q^2+i \eta )},\frac{1}{((q-p)^2+i \eta )}\right\}
\end{dmath*}

However, setting the option \texttt{Tally} to \texttt{True} will count
the powers of the appearing propagators.

\begin{Shaded}
\begin{Highlighting}[]
\NormalTok{FCLoopIntegralToPropagators}\OperatorTok{[}\NormalTok{int}\OperatorTok{,} \OperatorTok{\{}\FunctionTok{q}\OperatorTok{\},} \FunctionTok{Tally} \OtherTok{{-}\textgreater{}} \ConstantTok{True}\OperatorTok{]}
\end{Highlighting}
\end{Shaded}

\begin{dmath*}\breakingcomma
\left(
\begin{array}{cc}
 \frac{1}{(q^2+i \eta )} & 1 \\
 (p\cdot q+i \eta ) & 1 \\
 \frac{1}{((q-p)^2+i \eta )} & 2 \\
\end{array}
\right)
\end{dmath*}

Here is a more realistic 3-loop example

\begin{Shaded}
\begin{Highlighting}[]
\NormalTok{int }\ExtensionTok{=}\NormalTok{ SFAD}\OperatorTok{[\{\{}\SpecialCharTok{{-}}\NormalTok{k1}\OperatorTok{,} \DecValTok{0}\OperatorTok{\},} \OperatorTok{\{}\NormalTok{mc}\SpecialCharTok{\^{}}\DecValTok{2}\OperatorTok{,} \DecValTok{1}\OperatorTok{\},} \DecValTok{1}\OperatorTok{\}]}\NormalTok{ SFAD}\OperatorTok{[\{\{}\SpecialCharTok{{-}}\NormalTok{k1 }\SpecialCharTok{{-}}\NormalTok{ k2}\OperatorTok{,} \DecValTok{0}\OperatorTok{\},} \OperatorTok{\{}\NormalTok{mc}\SpecialCharTok{\^{}}\DecValTok{2}\OperatorTok{,} \DecValTok{1}\OperatorTok{\},} \DecValTok{1}\OperatorTok{\}]}\NormalTok{ SFAD}\OperatorTok{[\{\{}\SpecialCharTok{{-}}\NormalTok{k2}\OperatorTok{,} \DecValTok{0}\OperatorTok{\},} \OperatorTok{\{}\DecValTok{0}\OperatorTok{,} \DecValTok{1}\OperatorTok{\},} \DecValTok{1}\OperatorTok{\}]}\NormalTok{ SFAD}\OperatorTok{[\{\{}\SpecialCharTok{{-}}\NormalTok{k2}\OperatorTok{,} \DecValTok{0}\OperatorTok{\},} \OperatorTok{\{}\DecValTok{0}\OperatorTok{,} \DecValTok{1}\OperatorTok{\},} \DecValTok{2}\OperatorTok{\}]}\NormalTok{ SFAD}\OperatorTok{[\{\{}\SpecialCharTok{{-}}\NormalTok{k3}\OperatorTok{,} \DecValTok{0}\OperatorTok{\},} \OperatorTok{\{}\NormalTok{mc}\SpecialCharTok{\^{}}\DecValTok{2}\OperatorTok{,} \DecValTok{1}\OperatorTok{\},} \DecValTok{1}\OperatorTok{\}]} \SpecialCharTok{*}\NormalTok{SFAD}\OperatorTok{[\{\{}\NormalTok{k1 }\SpecialCharTok{{-}}\NormalTok{ k3 }\SpecialCharTok{{-}}\NormalTok{ p1}\OperatorTok{,} \DecValTok{0}\OperatorTok{\},} \OperatorTok{\{}\DecValTok{0}\OperatorTok{,} \DecValTok{1}\OperatorTok{\},} \DecValTok{1}\OperatorTok{\}]}\NormalTok{ SFAD}\OperatorTok{[\{\{}\SpecialCharTok{{-}}\NormalTok{k1 }\SpecialCharTok{{-}}\NormalTok{ k2 }\SpecialCharTok{+}\NormalTok{ k3 }\SpecialCharTok{+}\NormalTok{ p1}\OperatorTok{,} \DecValTok{0}\OperatorTok{\},} \OperatorTok{\{}\DecValTok{0}\OperatorTok{,} \DecValTok{1}\OperatorTok{\},} \DecValTok{1}\OperatorTok{\}]}\NormalTok{ SFAD}\OperatorTok{[\{\{}\SpecialCharTok{{-}}\NormalTok{k1 }\SpecialCharTok{{-}}\NormalTok{ k2 }\SpecialCharTok{+}\NormalTok{ k3 }\SpecialCharTok{+}\NormalTok{ p1}\OperatorTok{,} \DecValTok{0}\OperatorTok{\},} \OperatorTok{\{}\DecValTok{0}\OperatorTok{,} \DecValTok{1}\OperatorTok{\},} \DecValTok{2}\OperatorTok{\}]}
\end{Highlighting}
\end{Shaded}

\begin{dmath*}\breakingcomma
\frac{1}{(\text{k2}^2+i \eta )^3 (\text{k1}^2-\text{mc}^2+i \eta ) (\text{k3}^2-\text{mc}^2+i \eta ) ((-\text{k1}-\text{k2})^2-\text{mc}^2+i \eta ) ((\text{k1}-\text{k3}-\text{p1})^2+i \eta ) ((-\text{k1}-\text{k2}+\text{k3}+\text{p1})^2+i \eta )^3}
\end{dmath*}

\begin{Shaded}
\begin{Highlighting}[]
\NormalTok{FCLoopIntegralToPropagators}\OperatorTok{[}\NormalTok{int}\OperatorTok{,} \OperatorTok{\{}\NormalTok{k1}\OperatorTok{,}\NormalTok{ k2}\OperatorTok{,}\NormalTok{ k3}\OperatorTok{\},} \FunctionTok{Tally} \OtherTok{{-}\textgreater{}} \ConstantTok{True}\OperatorTok{]}
\end{Highlighting}
\end{Shaded}

\begin{dmath*}\breakingcomma
\left(
\begin{array}{cc}
 \frac{1}{(\text{k2}^2+i \eta )} & 3 \\
 \frac{1}{(\text{k3}^2-\text{mc}^2+i \eta )} & 1 \\
 \frac{1}{(\text{k1}^2-\text{mc}^2+i \eta )} & 1 \\
 \frac{1}{((\text{k1}-\text{k3}-\text{p1})^2+i \eta )} & 1 \\
 \frac{1}{((-\text{k1}-\text{k2}+\text{k3}+\text{p1})^2+i \eta )} & 3 \\
 \frac{1}{((-\text{k1}-\text{k2})^2-\text{mc}^2+i \eta )} & 1 \\
\end{array}
\right)
\end{dmath*}
\end{document}
