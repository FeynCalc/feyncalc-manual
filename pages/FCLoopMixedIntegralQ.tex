% !TeX program = pdflatex
% !TeX root = FCLoopMixedIntegralQ.tex

\documentclass[../FeynCalcManual.tex]{subfiles}
\begin{document}
\hypertarget{fcloopmixedintegralq}{
\section{FCLoopMixedIntegralQ}\label{fcloopmixedintegralq}\index{FCLoopMixedIntegralQ}}

\texttt{FCLoopMixedIntegralQ[\allowbreak{}exp]} returns \texttt{True} if
the integral contains both Lorentz and Cartesian indices and momenta.

\subsection{See also}

\hyperlink{toc}{Overview}

\subsection{Examples}

\begin{Shaded}
\begin{Highlighting}[]
\NormalTok{FCI}\OperatorTok{[}\NormalTok{FVD}\OperatorTok{[}\FunctionTok{p}\OperatorTok{,}\NormalTok{ mu}\OperatorTok{]}\NormalTok{ CFAD}\OperatorTok{[}\FunctionTok{q}\OperatorTok{,} \FunctionTok{q} \SpecialCharTok{{-}} \FunctionTok{p}\OperatorTok{]]} 
 
\NormalTok{FCLoopMixedIntegralQ}\OperatorTok{[}\SpecialCharTok{\%}\OperatorTok{]}
\end{Highlighting}
\end{Shaded}

\begin{dmath*}\breakingcomma
\frac{p^{\text{mu}}}{(q^2-i \eta ).((q-p)^2-i \eta )}
\end{dmath*}

\begin{dmath*}\breakingcomma
\text{True}
\end{dmath*}

\begin{Shaded}
\begin{Highlighting}[]
\NormalTok{FCI}\OperatorTok{[}\NormalTok{FVD}\OperatorTok{[}\FunctionTok{p}\OperatorTok{,}\NormalTok{ mu}\OperatorTok{]}\NormalTok{ FAD}\OperatorTok{[}\FunctionTok{q}\OperatorTok{,} \FunctionTok{q} \SpecialCharTok{{-}} \FunctionTok{p}\OperatorTok{]]} 
 
\NormalTok{FCLoopMixedIntegralQ}\OperatorTok{[}\SpecialCharTok{\%}\OperatorTok{]}
\end{Highlighting}
\end{Shaded}

\begin{dmath*}\breakingcomma
\frac{p^{\text{mu}}}{q^2.(q-p)^2}
\end{dmath*}

\begin{dmath*}\breakingcomma
\text{False}
\end{dmath*}
\end{document}
