% !TeX program = pdflatex
% !TeX root = FCLoopNonIntegerPropagatorPowersFreeQ.tex

\documentclass[../FeynCalcManual.tex]{subfiles}
\begin{document}
\hypertarget{fcloopnonintegerpropagatorpowersfreeq}{
\section{FCLoopNonIntegerPropagatorPowersFreeQ}\label{fcloopnonintegerpropagatorpowersfreeq}\index{FCLoopNonIntegerPropagatorPowersFreeQ}}

\texttt{FCLoopNonIntegerPropagatorPowersFreeQ[\allowbreak{}int]} checks
if the integral contains propagators raised to noninteger
(i.e.~fractional or symbolic) powers.

\subsection{See also}

\hyperlink{toc}{Overview},
\hyperlink{fcloopremovenegativepropagatorpowers}{FCLoopRemoveNegativePropagatorPowers}.

\subsection{Examples}

\begin{Shaded}
\begin{Highlighting}[]
\NormalTok{SFAD}\OperatorTok{[\{}\FunctionTok{q} \SpecialCharTok{+} \FunctionTok{p}\OperatorTok{,} \FunctionTok{m}\SpecialCharTok{\^{}}\DecValTok{2}\OperatorTok{,} \DecValTok{2}\OperatorTok{\}]} 
 
\NormalTok{FCLoopNonIntegerPropagatorPowersFreeQ}\OperatorTok{[}\NormalTok{FCI}\OperatorTok{[}\SpecialCharTok{\%}\OperatorTok{]]}
\end{Highlighting}
\end{Shaded}

\begin{dmath*}\breakingcomma
\frac{1}{((p+q)^2-m^2+i \eta )^2}
\end{dmath*}

\begin{dmath*}\breakingcomma
\text{True}
\end{dmath*}

\begin{Shaded}
\begin{Highlighting}[]
\NormalTok{SFAD}\OperatorTok{[\{}\FunctionTok{q} \SpecialCharTok{+} \FunctionTok{p}\OperatorTok{,} \FunctionTok{m}\SpecialCharTok{\^{}}\DecValTok{2}\OperatorTok{,} \FunctionTok{n}\OperatorTok{\}]} 
 
\NormalTok{FCLoopNonIntegerPropagatorPowersFreeQ}\OperatorTok{[}\NormalTok{FCI}\OperatorTok{[}\SpecialCharTok{\%}\OperatorTok{]]}
\end{Highlighting}
\end{Shaded}

\begin{dmath*}\breakingcomma
((p+q)^2-m^2+i \eta )^{-n}
\end{dmath*}

\begin{dmath*}\breakingcomma
\text{False}
\end{dmath*}

\begin{Shaded}
\begin{Highlighting}[]
\NormalTok{CFAD}\OperatorTok{[\{}\FunctionTok{l}\OperatorTok{,} \FunctionTok{m}\SpecialCharTok{\^{}}\DecValTok{2}\OperatorTok{,} \DecValTok{1}\SpecialCharTok{/}\DecValTok{2}\OperatorTok{\}]} 
 
\NormalTok{FCLoopNonIntegerPropagatorPowersFreeQ}\OperatorTok{[}\NormalTok{FCI}\OperatorTok{[}\SpecialCharTok{\%}\OperatorTok{]]}
\end{Highlighting}
\end{Shaded}

\begin{dmath*}\breakingcomma
\frac{1}{\sqrt{(l^2+m^2-i \eta )}}
\end{dmath*}

\begin{dmath*}\breakingcomma
\text{False}
\end{dmath*}
\end{document}
