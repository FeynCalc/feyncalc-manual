% !TeX program = pdflatex
% !TeX root = FCLoopPakScalelessQ.tex

\documentclass[../FeynCalcManual.tex]{subfiles}
\begin{document}
\hypertarget{fclooppakscalelessq}{%
\section{FCLoopPakScalelessQ}\label{fclooppakscalelessq}}

\texttt{FCLoopPakScalelessQ[\allowbreak{}poly,\ \allowbreak{}x]} checks
whether the characteristic polynomial poly (in the \(U \times xF\) form)
with the Feynman parameters
\texttt{x[\allowbreak{}1],\ \allowbreak{}x[\allowbreak{}2],\ \allowbreak{}...}
corresponds to a scaleless loop integral or loop integral topology. The
polynomial does not need to be canonically ordered.

The function uses the algorithm of Alexey Pak
\href{https://arxiv.org/abs/1111.0868}{arXiv:1111.0868}. Cf. also the
PhD thesis of Jens Hoff
\href{https://doi.org/10.5445/IR/1000047447}{10.5445/IR/1000047447} for
the detailed description of a possible implementation.
\texttt{FCLoopPakScalelessQ} is a backend function used in
\texttt{FCLoopScalelessQ}, \texttt{FCLoopFindSubtopologies} etc.

\subsection{See also}

\hyperlink{toc}{Overview}, \hyperlink{fctopology}{FCTopology},
\hyperlink{gli}{GLI}, \hyperlink{fclooptopakform}{FCLoopToPakForm},
\hyperlink{fcloopscalelessq}{FCLoopScalelessQ}.

\subsection{Examples}

A scaleless 2-loop tadpole is a clear case, since here the
characteristic polynomial vanishes

\begin{Shaded}
\begin{Highlighting}[]
\NormalTok{int }\ExtensionTok{=}\NormalTok{ FAD}\OperatorTok{[}\NormalTok{p1}\OperatorTok{,}\NormalTok{ p2}\OperatorTok{,}\NormalTok{ p1 }\SpecialCharTok{{-}}\NormalTok{ p2}\OperatorTok{]} 
 
\NormalTok{pf }\ExtensionTok{=}\NormalTok{ FCLoopToPakForm}\OperatorTok{[}\NormalTok{int}\OperatorTok{,} \OperatorTok{\{}\NormalTok{p1}\OperatorTok{,}\NormalTok{ p2}\OperatorTok{\},} \FunctionTok{Names} \OtherTok{{-}\textgreater{}} \FunctionTok{x}\OperatorTok{,} 
     
     \FunctionTok{CharacteristicPolynomial} \OtherTok{{-}\textgreater{}} \FunctionTok{Function}\OperatorTok{[\{}\FunctionTok{u}\OperatorTok{,} \FunctionTok{f}\OperatorTok{\},} \FunctionTok{u} \FunctionTok{f}\OperatorTok{]][[}\DecValTok{2}\OperatorTok{]][[}\DecValTok{1}\OperatorTok{]]}
\end{Highlighting}
\end{Shaded}

\begin{dmath*}\breakingcomma
\frac{1}{\text{p1}^2.\text{p2}^2.(\text{p1}-\text{p2})^2}
\end{dmath*}

\begin{dmath*}\breakingcomma
0
\end{dmath*}

\begin{Shaded}
\begin{Highlighting}[]
\NormalTok{FCLoopPakScalelessQ}\OperatorTok{[}\NormalTok{pf}\OperatorTok{,} \FunctionTok{x}\OperatorTok{]}
\end{Highlighting}
\end{Shaded}

\begin{dmath*}\breakingcomma
\text{True}
\end{dmath*}

A somewhat less obvious (but still simple) case is this 1-loop eikonal
integral.

\begin{Shaded}
\begin{Highlighting}[]
\NormalTok{int }\ExtensionTok{=}\NormalTok{ SFAD}\OperatorTok{[\{\{}\DecValTok{0}\OperatorTok{,} \DecValTok{2} \FunctionTok{p}\NormalTok{ . }\FunctionTok{q}\OperatorTok{\},} \DecValTok{0}\OperatorTok{\},} \FunctionTok{p}\OperatorTok{]} 
 
\NormalTok{pf }\ExtensionTok{=}\NormalTok{ FCLoopToPakForm}\OperatorTok{[}\NormalTok{int}\OperatorTok{,} \OperatorTok{\{}\FunctionTok{p}\OperatorTok{\},} \FunctionTok{Names} \OtherTok{{-}\textgreater{}} \FunctionTok{x}\OperatorTok{,} 
     
     \FunctionTok{CharacteristicPolynomial} \OtherTok{{-}\textgreater{}} \FunctionTok{Function}\OperatorTok{[\{}\FunctionTok{u}\OperatorTok{,} \FunctionTok{f}\OperatorTok{\},} \FunctionTok{u} \FunctionTok{f}\OperatorTok{]][[}\DecValTok{2}\OperatorTok{]][[}\DecValTok{1}\OperatorTok{]]}
\end{Highlighting}
\end{Shaded}

\begin{dmath*}\breakingcomma
\frac{1}{(2 (p\cdot q)+i \eta ).(p^2+i \eta )}
\end{dmath*}

\begin{dmath*}\breakingcomma
q^2 x(1)^2 x(2)
\end{dmath*}

\begin{Shaded}
\begin{Highlighting}[]
\NormalTok{FCLoopPakScalelessQ}\OperatorTok{[}\NormalTok{pf}\OperatorTok{,} \FunctionTok{x}\OperatorTok{]}
\end{Highlighting}
\end{Shaded}

\begin{dmath*}\breakingcomma
\text{True}
\end{dmath*}

Adding a mass term to the quadratic propagator makes this integral
nonvanishing

\begin{Shaded}
\begin{Highlighting}[]
\NormalTok{int }\ExtensionTok{=}\NormalTok{ SFAD}\OperatorTok{[\{\{}\DecValTok{0}\OperatorTok{,} \DecValTok{2} \FunctionTok{p}\NormalTok{ . }\FunctionTok{q}\OperatorTok{\},} \DecValTok{0}\OperatorTok{\},} \OperatorTok{\{}\FunctionTok{p}\OperatorTok{,} \FunctionTok{m}\SpecialCharTok{\^{}}\DecValTok{2}\OperatorTok{\}]} 
 
\NormalTok{pf }\ExtensionTok{=}\NormalTok{ FCLoopToPakForm}\OperatorTok{[}\NormalTok{int}\OperatorTok{,} \OperatorTok{\{}\FunctionTok{p}\OperatorTok{\},} \FunctionTok{Names} \OtherTok{{-}\textgreater{}} \FunctionTok{x}\OperatorTok{,} 
     
     \FunctionTok{CharacteristicPolynomial} \OtherTok{{-}\textgreater{}} \FunctionTok{Function}\OperatorTok{[\{}\FunctionTok{u}\OperatorTok{,} \FunctionTok{f}\OperatorTok{\},} \FunctionTok{u} \FunctionTok{f}\OperatorTok{]][[}\DecValTok{2}\OperatorTok{]][[}\DecValTok{1}\OperatorTok{]]}
\end{Highlighting}
\end{Shaded}

\begin{dmath*}\breakingcomma
\frac{1}{(2 (p\cdot q)+i \eta ).(p^2-m^2+i \eta )}
\end{dmath*}

\begin{dmath*}\breakingcomma
m^2 x(1)^3+q^2 x(2)^2 x(1)
\end{dmath*}

\begin{Shaded}
\begin{Highlighting}[]
\NormalTok{FCLoopPakScalelessQ}\OperatorTok{[}\NormalTok{pf}\OperatorTok{,} \FunctionTok{x}\OperatorTok{]}
\end{Highlighting}
\end{Shaded}

\begin{dmath*}\breakingcomma
\text{False}
\end{dmath*}

Notice that \texttt{FCLoopPakScalelessQ} is more of an auxiliary
function. The corresponding end-user function is called
\texttt{FCLoopScalelessQ}

\begin{Shaded}
\begin{Highlighting}[]
\NormalTok{FCLoopScalelessQ}\OperatorTok{[}\NormalTok{SFAD}\OperatorTok{[\{\{}\DecValTok{0}\OperatorTok{,} \DecValTok{2} \FunctionTok{p}\NormalTok{ . }\FunctionTok{q}\OperatorTok{\},} \DecValTok{0}\OperatorTok{\},} \FunctionTok{p}\OperatorTok{],} \OperatorTok{\{}\FunctionTok{p}\OperatorTok{\}]}
\end{Highlighting}
\end{Shaded}

\begin{dmath*}\breakingcomma
\text{True}
\end{dmath*}
\end{document}
