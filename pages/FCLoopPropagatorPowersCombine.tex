% !TeX program = pdflatex
% !TeX root = FCLoopPropagatorPowersCombine.tex

\documentclass[../FeynCalcManual.tex]{subfiles}
\begin{document}
\hypertarget{fclooppropagatorpowerscombine}{
\section{FCLoopPropagatorPowersCombine}\label{fclooppropagatorpowerscombine}\index{FCLoopPropagatorPowersCombine}}

\texttt{FCLoopPropagatorPowersCombine[\allowbreak{}exp]} combines the
same propagators in a \texttt{FeynAmpDenominator} to one propagator
raised to an integer power.

\subsection{See also}

\hyperlink{toc}{Overview},
\hyperlink{fclooppropagatorpowersexpand}{FCLoopPropagatorPowersExpand}.

\subsection{Examples}

\begin{Shaded}
\begin{Highlighting}[]
\NormalTok{SFAD}\OperatorTok{[\{\{}\FunctionTok{q}\OperatorTok{,} \DecValTok{0}\OperatorTok{\},} \OperatorTok{\{}\FunctionTok{m}\OperatorTok{,} \DecValTok{1}\OperatorTok{\},} \DecValTok{1}\OperatorTok{\},} \OperatorTok{\{\{}\FunctionTok{q}\OperatorTok{,} \DecValTok{0}\OperatorTok{\},} \OperatorTok{\{}\FunctionTok{m}\OperatorTok{,} \DecValTok{1}\OperatorTok{\},} \DecValTok{1}\OperatorTok{\}]} 
 
\NormalTok{ex }\ExtensionTok{=}\NormalTok{ FCLoopPropagatorPowersCombine}\OperatorTok{[}\SpecialCharTok{\%}\OperatorTok{]}
\end{Highlighting}
\end{Shaded}

\begin{dmath*}\breakingcomma
\frac{1}{(q^2-m+i \eta )^2}
\end{dmath*}

\begin{dmath*}\breakingcomma
\frac{1}{(q^2-m+i \eta )^2}
\end{dmath*}

\begin{Shaded}
\begin{Highlighting}[]
\NormalTok{ex }\SpecialCharTok{//} \FunctionTok{StandardForm}

\CommentTok{(*FeynAmpDenominator[StandardPropagatorDenominator[Momentum[q, D], 0, {-}m, \{2, 1\}]]*)}
\end{Highlighting}
\end{Shaded}

\begin{Shaded}
\begin{Highlighting}[]
\NormalTok{SFAD}\OperatorTok{[\{\{}\FunctionTok{q}\OperatorTok{,} \DecValTok{0}\OperatorTok{\},} \OperatorTok{\{}\FunctionTok{m}\OperatorTok{,} \DecValTok{1}\OperatorTok{\},} \SpecialCharTok{{-}}\DecValTok{1}\OperatorTok{\},} \OperatorTok{\{\{}\FunctionTok{q}\OperatorTok{,} \DecValTok{0}\OperatorTok{\},} \OperatorTok{\{}\FunctionTok{m}\OperatorTok{,} \DecValTok{1}\OperatorTok{\},} \DecValTok{1}\OperatorTok{\}]} 
 
\NormalTok{ex }\ExtensionTok{=}\NormalTok{ FCLoopPropagatorPowersCombine}\OperatorTok{[}\SpecialCharTok{\%}\OperatorTok{]}
\end{Highlighting}
\end{Shaded}

\begin{dmath*}\breakingcomma
\frac{1}{\frac{1}{(q^2-m+i \eta )}.(q^2-m+i \eta )}
\end{dmath*}

\begin{dmath*}\breakingcomma
1
\end{dmath*}

\begin{Shaded}
\begin{Highlighting}[]
\NormalTok{ex }\SpecialCharTok{//} \FunctionTok{StandardForm}

\CommentTok{(*1*)}
\end{Highlighting}
\end{Shaded}

The function automatically employs \texttt{FeynAmpDenominatorCombine}.

\begin{Shaded}
\begin{Highlighting}[]
\NormalTok{int }\ExtensionTok{=}\NormalTok{ SFAD}\OperatorTok{[\{\{}\SpecialCharTok{{-}}\NormalTok{k1}\OperatorTok{,} \DecValTok{0}\OperatorTok{\},} \OperatorTok{\{}\NormalTok{mc}\SpecialCharTok{\^{}}\DecValTok{2}\OperatorTok{,} \DecValTok{1}\OperatorTok{\},} \DecValTok{1}\OperatorTok{\}]}\NormalTok{  SFAD}\OperatorTok{[\{\{}\SpecialCharTok{{-}}\NormalTok{k1 }\SpecialCharTok{{-}}\NormalTok{ k2 }\SpecialCharTok{+}\NormalTok{ k3 }\SpecialCharTok{+}\NormalTok{ p1}\OperatorTok{,} \DecValTok{0}\OperatorTok{\},} \OperatorTok{\{}\DecValTok{0}\OperatorTok{,} \DecValTok{1}\OperatorTok{\},} \DecValTok{1}\OperatorTok{\}]}\NormalTok{ SFAD}\OperatorTok{[\{\{}\SpecialCharTok{{-}}\NormalTok{k1 }\SpecialCharTok{{-}}\NormalTok{ k2 }\SpecialCharTok{+}\NormalTok{ k3 }\SpecialCharTok{+}\NormalTok{ p1}\OperatorTok{,} \DecValTok{0}\OperatorTok{\},} \OperatorTok{\{}\DecValTok{0}\OperatorTok{,} \DecValTok{1}\OperatorTok{\},} \DecValTok{2}\OperatorTok{\}]}
\end{Highlighting}
\end{Shaded}

\begin{dmath*}\breakingcomma
\frac{1}{(\text{k1}^2-\text{mc}^2+i \eta ) ((-\text{k1}-\text{k2}+\text{k3}+\text{p1})^2+i \eta )^3}
\end{dmath*}

\begin{Shaded}
\begin{Highlighting}[]
\NormalTok{int }\SpecialCharTok{//}\NormalTok{ FCI }\SpecialCharTok{//} \FunctionTok{StandardForm}

\CommentTok{(*FeynAmpDenominator[StandardPropagatorDenominator[{-}Momentum[k1, D], 0, {-}mc\^{}2, \{1, 1\}]] FeynAmpDenominator[StandardPropagatorDenominator[{-}Momentum[k1, D] {-} Momentum[k2, D] + Momentum[k3, D] + Momentum[p1, D], 0, 0, \{1, 1\}]] FeynAmpDenominator[StandardPropagatorDenominator[{-}Momentum[k1, D] {-} Momentum[k2, D] + Momentum[k3, D] + Momentum[p1, D], 0, 0, \{2, 1\}]]*)}
\end{Highlighting}
\end{Shaded}

\begin{Shaded}
\begin{Highlighting}[]
\NormalTok{res }\ExtensionTok{=}\NormalTok{ FCLoopPropagatorPowersCombine}\OperatorTok{[}\NormalTok{int}\OperatorTok{]}
\end{Highlighting}
\end{Shaded}

\begin{dmath*}\breakingcomma
\frac{1}{(\text{k1}^2-\text{mc}^2+i \eta ).((-\text{k1}-\text{k2}+\text{k3}+\text{p1})^2+i \eta )^3}
\end{dmath*}

\begin{Shaded}
\begin{Highlighting}[]
\NormalTok{res }\SpecialCharTok{//}\NormalTok{ FCI }\SpecialCharTok{//} \FunctionTok{StandardForm}

\CommentTok{(*FeynAmpDenominator[StandardPropagatorDenominator[{-}Momentum[k1, D], 0, {-}mc\^{}2, \{1, 1\}], StandardPropagatorDenominator[{-}Momentum[k1, D] {-} Momentum[k2, D] + Momentum[k3, D] + Momentum[p1, D], 0, 0, \{3, 1\}]]*)}
\end{Highlighting}
\end{Shaded}

\end{document}
