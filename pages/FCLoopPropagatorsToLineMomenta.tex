% !TeX program = pdflatex
% !TeX root = FCLoopPropagatorsToLineMomenta.tex

\documentclass[../FeynCalcManual.tex]{subfiles}
\begin{document}
\hypertarget{fclooppropagatorstolinemomenta}{
\section{FCLoopPropagatorsToLineMomenta}\label{fclooppropagatorstolinemomenta}\index{FCLoopPropagatorsToLineMomenta}}

\texttt{FCLoopPropagatorsToLineMomenta[\allowbreak{}\{\allowbreak{}prop1,\ \allowbreak{}prop2,\ \allowbreak{}...\}]}
is an auxiliary function that extracts line momenta flowing through the
given list of propagators.

\subsection{See also}

\hyperlink{toc}{Overview},
\hyperlink{fcloopintegraltograph}{FCLoopIntegralToGraph},
\hyperlink{auxiliarymomenta}{AuxiliaryMomenta}.

\subsection{Examples}

\begin{Shaded}
\begin{Highlighting}[]
\NormalTok{FCLoopPropagatorsToLineMomenta}\OperatorTok{[\{}\NormalTok{SFAD}\OperatorTok{[\{}\FunctionTok{q} \SpecialCharTok{+} \FunctionTok{l}\OperatorTok{,} \FunctionTok{m}\SpecialCharTok{\^{}}\DecValTok{2}\OperatorTok{\}],}\NormalTok{ SFAD}\OperatorTok{[\{}\FunctionTok{p}\OperatorTok{,} \SpecialCharTok{{-}}\FunctionTok{m}\SpecialCharTok{\^{}}\DecValTok{2}\OperatorTok{\}]\},}\NormalTok{ FCE }\OtherTok{{-}\textgreater{}} \ConstantTok{True}\OperatorTok{]}
\end{Highlighting}
\end{Shaded}

\begin{dmath*}\breakingcomma
\left(
\begin{array}{cc}
 l+q & p \\
 -m^2 & m^2 \\
 \frac{1}{((l+q)^2-m^2+i \eta )} & \frac{1}{(p^2+m^2+i \eta )} \\
\end{array}
\right)
\end{dmath*}

\begin{Shaded}
\begin{Highlighting}[]
\NormalTok{FCLoopPropagatorsToLineMomenta}\OperatorTok{[\{}\NormalTok{CFAD}\OperatorTok{[\{\{}\DecValTok{0}\OperatorTok{,} \DecValTok{2} \FunctionTok{v}\NormalTok{ . (}\FunctionTok{q} \SpecialCharTok{+} \FunctionTok{r}\NormalTok{)}\OperatorTok{\},} \FunctionTok{m}\SpecialCharTok{\^{}}\DecValTok{2}\OperatorTok{\}]\},}\NormalTok{ FCE }\OtherTok{{-}\textgreater{}} \ConstantTok{True}\OperatorTok{,} 
\NormalTok{  AuxiliaryMomenta }\OtherTok{{-}\textgreater{}} \OperatorTok{\{}\FunctionTok{v}\OperatorTok{\}]}
\end{Highlighting}
\end{Shaded}

\begin{dmath*}\breakingcomma
\left(
\begin{array}{c}
 q+r \\
 m^2 \\
 \frac{1}{(2 ((q+r)\cdot v)+m^2-i \eta )} \\
\end{array}
\right)
\end{dmath*}

Reversed signs are also supported

\begin{Shaded}
\begin{Highlighting}[]
\OperatorTok{\{}\NormalTok{SFAD}\OperatorTok{[\{}\FunctionTok{I}\NormalTok{ (}\FunctionTok{q} \SpecialCharTok{+} \FunctionTok{l}\NormalTok{)}\OperatorTok{,} \SpecialCharTok{{-}}\FunctionTok{m}\SpecialCharTok{\^{}}\DecValTok{2}\OperatorTok{\}],}\NormalTok{ SFAD}\OperatorTok{[\{}\FunctionTok{I} \FunctionTok{p}\OperatorTok{,} \SpecialCharTok{{-}}\FunctionTok{m}\SpecialCharTok{\^{}}\DecValTok{2}\OperatorTok{\}]\}} 
 
\NormalTok{FCLoopPropagatorsToLineMomenta}\OperatorTok{[}\SpecialCharTok{\%}\OperatorTok{,}\NormalTok{ FCE }\OtherTok{{-}\textgreater{}} \ConstantTok{True}\OperatorTok{]}
\end{Highlighting}
\end{Shaded}

\begin{dmath*}\breakingcomma
\left\{\frac{1}{(-(l+q)^2+m^2+i \eta )},\frac{1}{(-p^2+m^2+i \eta )}\right\}
\end{dmath*}

\begin{dmath*}\breakingcomma
\left(
\begin{array}{cc}
 l+q & p \\
 m^2 & m^2 \\
 \frac{1}{(-(l+q)^2+m^2+i \eta )} & \frac{1}{(-p^2+m^2+i \eta )} \\
\end{array}
\right)
\end{dmath*}

\begin{Shaded}
\begin{Highlighting}[]
\NormalTok{FCLoopPropagatorsToLineMomenta}\OperatorTok{[\{}\NormalTok{SFAD}\OperatorTok{[\{}\FunctionTok{I}\NormalTok{ (}\FunctionTok{q} \SpecialCharTok{+} \FunctionTok{l}\NormalTok{)}\OperatorTok{,} \SpecialCharTok{{-}}\FunctionTok{m}\SpecialCharTok{\^{}}\DecValTok{2}\OperatorTok{\}],}\NormalTok{ SFAD}\OperatorTok{[\{}\FunctionTok{I} \FunctionTok{p}\OperatorTok{,} \SpecialCharTok{{-}}\FunctionTok{m}\SpecialCharTok{\^{}}\DecValTok{2}\OperatorTok{\}]\},} 
\NormalTok{   FCE }\OtherTok{{-}\textgreater{}} \ConstantTok{True}\OperatorTok{]} \SpecialCharTok{//} \FunctionTok{InputForm}
\end{Highlighting}
\end{Shaded}

\begin{Shaded}
\begin{Highlighting}[]
\OperatorTok{\{\{}\FunctionTok{l} \SpecialCharTok{+} \FunctionTok{q}\OperatorTok{,} \FunctionTok{p}\OperatorTok{\},} \OperatorTok{\{}\FunctionTok{m}\SpecialCharTok{\^{}}\DecValTok{2}\OperatorTok{,} \FunctionTok{m}\SpecialCharTok{\^{}}\DecValTok{2}\OperatorTok{\},} \OperatorTok{\{}\NormalTok{SFAD}\OperatorTok{[\{\{}\FunctionTok{I}\SpecialCharTok{*}\NormalTok{(}\FunctionTok{l} \SpecialCharTok{+} \FunctionTok{q}\NormalTok{)}\OperatorTok{,} \DecValTok{0}\OperatorTok{\},} \OperatorTok{\{}\SpecialCharTok{{-}}\FunctionTok{m}\SpecialCharTok{\^{}}\DecValTok{2}\OperatorTok{,} \DecValTok{1}\OperatorTok{\},} \DecValTok{1}\OperatorTok{\}],} 
\NormalTok{  SFAD}\OperatorTok{[\{\{}\FunctionTok{I}\SpecialCharTok{*}\FunctionTok{p}\OperatorTok{,} \DecValTok{0}\OperatorTok{\},} \OperatorTok{\{}\SpecialCharTok{{-}}\FunctionTok{m}\SpecialCharTok{\^{}}\DecValTok{2}\OperatorTok{,} \DecValTok{1}\OperatorTok{\},} \DecValTok{1}\OperatorTok{\}]\}\}}
\end{Highlighting}
\end{Shaded}

\begin{Shaded}
\begin{Highlighting}[]
\NormalTok{FCLoopPropagatorsToLineMomenta}\OperatorTok{[\{}\NormalTok{SFAD}\OperatorTok{[\{\{}\FunctionTok{I}\NormalTok{ p1}\OperatorTok{,} \SpecialCharTok{{-}}\DecValTok{2}\NormalTok{ p1 . }\FunctionTok{q}\OperatorTok{\},} \OperatorTok{\{}\DecValTok{0}\OperatorTok{,} \DecValTok{1}\OperatorTok{\},} \DecValTok{1}\OperatorTok{\}]\},}\NormalTok{FCE }\OtherTok{{-}\textgreater{}} \ConstantTok{True}\OperatorTok{]}
\end{Highlighting}
\end{Shaded}

\begin{dmath*}\breakingcomma
\left(
\begin{array}{c}
 \;\text{p1}+q \\
 0 \\
 \frac{1}{(-\text{p1}^2-2 (\text{p1}\cdot q)+i \eta )} \\
\end{array}
\right)
\end{dmath*}
\end{document}
