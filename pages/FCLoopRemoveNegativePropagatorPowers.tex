% !TeX program = pdflatex
% !TeX root = FCLoopRemoveNegativePropagatorPowers.tex

\documentclass[../FeynCalcManual.tex]{subfiles}
\begin{document}
\hypertarget{fcloopremovenegativepropagatorpowers}{%
\section{FCLoopRemoveNegativePropagatorPowers}\label{fcloopremovenegativepropagatorpowers}}

\texttt{FCLoopRemoveNegativePropagatorPowers[\allowbreak{}exp]} rewrites
propagators raised to integer powers as products.

\subsection{See also}

\hyperlink{toc}{Overview}

\subsection{Examples}

\begin{Shaded}
\begin{Highlighting}[]
\NormalTok{SFAD}\OperatorTok{[\{}\FunctionTok{q}\OperatorTok{,} \FunctionTok{m}\OperatorTok{,} \SpecialCharTok{{-}}\DecValTok{1}\OperatorTok{\}]} 
 
\NormalTok{ex }\ExtensionTok{=}\NormalTok{ FCLoopRemoveNegativePropagatorPowers}\OperatorTok{[}\SpecialCharTok{\%}\OperatorTok{]}
\end{Highlighting}
\end{Shaded}

\begin{dmath*}\breakingcomma
(q^2-m+i \eta )
\end{dmath*}

\begin{dmath*}\breakingcomma
q^2-m
\end{dmath*}

\begin{Shaded}
\begin{Highlighting}[]
\NormalTok{ex }\SpecialCharTok{//} \FunctionTok{StandardForm}

\CommentTok{(*{-}m + Pair[Momentum[q, D], Momentum[q, D]]*)}
\end{Highlighting}
\end{Shaded}

\begin{Shaded}
\begin{Highlighting}[]
\NormalTok{SFAD}\OperatorTok{[\{}\FunctionTok{q}\OperatorTok{,} \FunctionTok{m}\OperatorTok{\},} \FunctionTok{q} \SpecialCharTok{+} \FunctionTok{p}\OperatorTok{,} \OperatorTok{\{}\FunctionTok{q}\OperatorTok{,} \FunctionTok{m}\OperatorTok{,} \SpecialCharTok{{-}}\DecValTok{2}\OperatorTok{\}]} 
 
\NormalTok{ex }\ExtensionTok{=}\NormalTok{ FCLoopRemoveNegativePropagatorPowers}\OperatorTok{[}\SpecialCharTok{\%}\OperatorTok{]}
\end{Highlighting}
\end{Shaded}

\begin{dmath*}\breakingcomma
\frac{1}{(q^2-m+i \eta ).((p+q)^2+i \eta ).\frac{1}{(q^2-m+i \eta )^2}}
\end{dmath*}

\begin{dmath*}\breakingcomma
\frac{q^2-m}{((p+q)^2+i \eta )}
\end{dmath*}

\begin{Shaded}
\begin{Highlighting}[]
\NormalTok{ex }\SpecialCharTok{//} \FunctionTok{StandardForm}

\CommentTok{(*FeynAmpDenominator[StandardPropagatorDenominator[Momentum[p, D] + Momentum[q, D], 0, 0, \{1, 1\}]] ({-}m + Pair[Momentum[q, D], Momentum[q, D]])*)}
\end{Highlighting}
\end{Shaded}

\end{document}
