% !TeX program = pdflatex
% !TeX root = FCLoopRemovePropagator.tex

\documentclass[../FeynCalcManual.tex]{subfiles}
\begin{document}
\hypertarget{fcloopremovepropagator}{
\section{FCLoopRemovePropagator}\label{fcloopremovepropagator}\index{FCLoopRemovePropagator}}

\texttt{FCLoopRemovePropagator[\allowbreak{}input,\ \allowbreak{}\{\allowbreak{}pos1,\ \allowbreak{}pos2,\ \allowbreak{}...\}]}
returns a new \texttt{FCTopology} or \texttt{GLI} obtained from input by
removing propagators at positions listed in
\texttt{\{\allowbreak{}pos1,\ \allowbreak{}pos2,\ \allowbreak{}...\}}.

\subsection{See also}

\hyperlink{toc}{Overview},
\hyperlink{fcloopcreatepartialfractioningrules}{FCLoopCreatePartialFractioningRules},
\hyperlink{fctopology}{FCTopology}, \hyperlink{gli}{GLI}.

\subsection{Examples}

A 2-loop topology with one external momentum \texttt{Q}

\begin{Shaded}
\begin{Highlighting}[]
\NormalTok{topo }\ExtensionTok{=}\NormalTok{ FCTopology}\OperatorTok{[}\NormalTok{topo1}\OperatorTok{,} \OperatorTok{\{}\NormalTok{SFAD}\OperatorTok{[}\NormalTok{p1}\OperatorTok{],}\NormalTok{ SFAD}\OperatorTok{[}\NormalTok{p2}\OperatorTok{],}\NormalTok{ SFAD}\OperatorTok{[}\FunctionTok{Q} \SpecialCharTok{{-}}\NormalTok{ p1 }\SpecialCharTok{{-}}\NormalTok{ p2}\OperatorTok{],}\NormalTok{ SFAD}\OperatorTok{[}\FunctionTok{Q} \SpecialCharTok{{-}}\NormalTok{ p2}\OperatorTok{],}\NormalTok{ SFAD}\OperatorTok{[}\FunctionTok{Q} \SpecialCharTok{{-}}\NormalTok{ p1}\OperatorTok{]\},} \OperatorTok{\{}\NormalTok{p1}\OperatorTok{,}\NormalTok{ p2}\OperatorTok{\},} \OperatorTok{\{}\FunctionTok{Q}\OperatorTok{\},} \OperatorTok{\{}
    \FunctionTok{Hold}\OperatorTok{[}\NormalTok{SPD}\OperatorTok{[}\FunctionTok{Q}\OperatorTok{]]} \OtherTok{{-}\textgreater{}}\NormalTok{ qq}\OperatorTok{\},} \OperatorTok{\{\}]}
\end{Highlighting}
\end{Shaded}

\begin{dmath*}\breakingcomma
\text{FCTopology}\left(\text{topo1},\left\{\frac{1}{(\text{p1}^2+i \eta )},\frac{1}{(\text{p2}^2+i \eta )},\frac{1}{((-\text{p1}-\text{p2}+Q)^2+i \eta )},\frac{1}{((Q-\text{p2})^2+i \eta )},\frac{1}{((Q-\text{p1})^2+i \eta )}\right\},\{\text{p1},\text{p2}\},\{Q\},\{\text{Hold}[\text{SPD}(Q)]\to \;\text{qq}\},\{\}\right)
\end{dmath*}

The same topology with the 1st and 3rd propagators removed. Notice that
the new name is created using the suffix specified via the option
\texttt{Names}

\begin{Shaded}
\begin{Highlighting}[]
\NormalTok{FCLoopRemovePropagator}\OperatorTok{[}\NormalTok{topo}\OperatorTok{,} \OperatorTok{\{}\DecValTok{1}\OperatorTok{,} \DecValTok{3}\OperatorTok{\}]}
\end{Highlighting}
\end{Shaded}

\begin{dmath*}\breakingcomma
\text{FCTopology}\left(\text{topo1PFR13},\left\{\frac{1}{(\text{p2}^2+i \eta )},\frac{1}{((Q-\text{p2})^2+i \eta )},\frac{1}{((Q-\text{p1})^2+i \eta )}\right\},\{\text{p1},\text{p2}\},\{Q\},\{\text{Hold}[\text{SPD}(Q)]\to \;\text{qq}\},\{\}\right)
\end{dmath*}

\begin{Shaded}
\begin{Highlighting}[]
\NormalTok{gli }\ExtensionTok{=}\NormalTok{ GLI}\OperatorTok{[}\NormalTok{topo2}\OperatorTok{,} \OperatorTok{\{}\DecValTok{1}\OperatorTok{,} \DecValTok{1}\OperatorTok{,} \DecValTok{1}\OperatorTok{,} \DecValTok{2}\OperatorTok{,} \DecValTok{0}\OperatorTok{,} \DecValTok{1}\OperatorTok{,} \DecValTok{1}\OperatorTok{\}]}
\end{Highlighting}
\end{Shaded}

\begin{dmath*}\breakingcomma
G^{\text{topo2}}(1,1,1,2,0,1,1)
\end{dmath*}

\begin{Shaded}
\begin{Highlighting}[]
\NormalTok{FCLoopRemovePropagator}\OperatorTok{[}\NormalTok{gli}\OperatorTok{,} \OperatorTok{\{}\DecValTok{2}\OperatorTok{,} \DecValTok{4}\OperatorTok{\}]}
\end{Highlighting}
\end{Shaded}

\begin{dmath*}\breakingcomma
G^{\text{topo2PFR24}}(1,1,0,1,1)
\end{dmath*}
\end{document}
