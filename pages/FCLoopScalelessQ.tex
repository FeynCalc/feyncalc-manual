% !TeX program = pdflatex
% !TeX root = FCLoopScalelessQ.tex

\documentclass[../FeynCalcManual.tex]{subfiles}
\begin{document}
\hypertarget{fcloopscalelessq}{
\section{FCLoopScalelessQ}\label{fcloopscalelessq}\index{FCLoopScalelessQ}}

\texttt{FCLoopScalelessQ[\allowbreak{}int,\ \allowbreak{}\{\allowbreak{}p1,\ \allowbreak{}p2,\ \allowbreak{}...\}]}
checks whether the loop integral \texttt{int} depending on the loop
momenta \texttt{p1,\ \allowbreak{}p2,\ \allowbreak{}...} is scaleless.
Only integrals that admit a Feynman parametrization with proper \(U\)
and \(F\) polynomials are supported.

The function uses the algorithm of Alexey Pak
\href{https://arxiv.org/abs/1111.0868}{arXiv:1111.0868}. Cf. also the
PhD thesis of Jens Hoff
\href{https://doi.org/10.5445/IR/1000047447}{10.5445/IR/1000047447} for
the detailed description of a possible implementation.

\subsection{See also}

\hyperlink{toc}{Overview}, \hyperlink{fctopology}{FCTopology},
\hyperlink{gli}{GLI}, \hyperlink{fclooptopakform}{FCLoopToPakForm},
\hyperlink{fclooppakorder}{FCLoopPakOrder},
\hyperlink{fcloopscalelessq}{FCLoopScalelessQ}.

\subsection{Examples}

A scaleless 2-loop tadpole

\begin{Shaded}
\begin{Highlighting}[]
\NormalTok{FCLoopScalelessQ}\OperatorTok{[}\NormalTok{FAD}\OperatorTok{[}\NormalTok{p1}\OperatorTok{,}\NormalTok{ p2}\OperatorTok{,}\NormalTok{ p1 }\SpecialCharTok{{-}}\NormalTok{ p2}\OperatorTok{],} \OperatorTok{\{}\NormalTok{p1}\OperatorTok{,}\NormalTok{ p2}\OperatorTok{\}]}
\end{Highlighting}
\end{Shaded}

\begin{dmath*}\breakingcomma
\text{True}
\end{dmath*}

A 1-loop massless eikonal integral.

\begin{Shaded}
\begin{Highlighting}[]
\NormalTok{FCLoopScalelessQ}\OperatorTok{[}\NormalTok{SFAD}\OperatorTok{[\{\{}\DecValTok{0}\OperatorTok{,} \DecValTok{2} \FunctionTok{p}\NormalTok{ . }\FunctionTok{q}\OperatorTok{\},} \DecValTok{0}\OperatorTok{\},} \FunctionTok{p}\OperatorTok{],} \OperatorTok{\{}\FunctionTok{p}\OperatorTok{\}]}
\end{Highlighting}
\end{Shaded}

\begin{dmath*}\breakingcomma
\text{True}
\end{dmath*}

A 1-loop massive eikonal integral.

\begin{Shaded}
\begin{Highlighting}[]
\NormalTok{FCLoopScalelessQ}\OperatorTok{[}\NormalTok{SFAD}\OperatorTok{[\{\{}\DecValTok{0}\OperatorTok{,} \DecValTok{2} \FunctionTok{p}\NormalTok{ . }\FunctionTok{q}\OperatorTok{\},} \DecValTok{0}\OperatorTok{\},} \FunctionTok{p}\OperatorTok{],} \OperatorTok{\{}\FunctionTok{p}\OperatorTok{\}]}
\end{Highlighting}
\end{Shaded}

\begin{dmath*}\breakingcomma
\text{True}
\end{dmath*}

A scaleless topology

\begin{Shaded}
\begin{Highlighting}[]
\NormalTok{FCTopology}\OperatorTok{[}\NormalTok{topo}\OperatorTok{,} \OperatorTok{\{}\NormalTok{SFAD}\OperatorTok{[\{\{}\FunctionTok{I}\NormalTok{ p3}\OperatorTok{,} \DecValTok{0}\OperatorTok{\},} \OperatorTok{\{}\DecValTok{0}\OperatorTok{,} \DecValTok{1}\OperatorTok{\},} \DecValTok{1}\OperatorTok{\}],}\NormalTok{ SFAD}\OperatorTok{[\{\{}\FunctionTok{I}\NormalTok{ p1}\OperatorTok{,} \DecValTok{0}\OperatorTok{\},} \OperatorTok{\{}\DecValTok{0}\OperatorTok{,} \DecValTok{1}\OperatorTok{\},} \DecValTok{1}\OperatorTok{\}],} 
\NormalTok{    SFAD}\OperatorTok{[\{\{}\DecValTok{0}\OperatorTok{,} \SpecialCharTok{{-}}\DecValTok{2}\NormalTok{ p1 . }\FunctionTok{q}\OperatorTok{\},} \OperatorTok{\{}\DecValTok{0}\OperatorTok{,} \DecValTok{1}\OperatorTok{\},} \DecValTok{1}\OperatorTok{\}],}\NormalTok{ SFAD}\OperatorTok{[\{\{}\FunctionTok{I}\NormalTok{ p3 }\SpecialCharTok{+} \FunctionTok{I} \FunctionTok{q}\OperatorTok{,} \DecValTok{0}\OperatorTok{\},} \OperatorTok{\{}\SpecialCharTok{{-}}\NormalTok{mb}\SpecialCharTok{\^{}}\DecValTok{2}\OperatorTok{,} \DecValTok{1}\OperatorTok{\},} \DecValTok{1}\OperatorTok{\}],} 
\NormalTok{    SFAD}\OperatorTok{[\{\{}\DecValTok{0}\OperatorTok{,}\NormalTok{ p1 . p3}\OperatorTok{\},} \OperatorTok{\{}\DecValTok{0}\OperatorTok{,} \DecValTok{1}\OperatorTok{\},} \DecValTok{1}\OperatorTok{\}]\},} \OperatorTok{\{}\NormalTok{p1}\OperatorTok{,}\NormalTok{ p3}\OperatorTok{\},} \OperatorTok{\{}\FunctionTok{q}\OperatorTok{\},} \OperatorTok{\{\},} \OperatorTok{\{\}]} 
 
\NormalTok{FCLoopScalelessQ}\OperatorTok{[}\SpecialCharTok{\%}\OperatorTok{]}
\end{Highlighting}
\end{Shaded}

\begin{dmath*}\breakingcomma
\text{FCTopology}\left(\text{topo},\left\{\frac{1}{(-\text{p3}^2+i \eta )},\frac{1}{(-\text{p1}^2+i \eta )},\frac{1}{(-2 (\text{p1}\cdot q)+i \eta )},\frac{1}{((i \;\text{p3}+i q)^2+\text{mb}^2+i \eta )},\frac{1}{(\text{p1}\cdot \;\text{p3}+i \eta )}\right\},\{\text{p1},\text{p3}\},\{q\},\{\},\{\}\right)
\end{dmath*}

\begin{dmath*}\breakingcomma
\text{True}
\end{dmath*}
\end{document}
