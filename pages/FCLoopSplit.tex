% !TeX program = pdflatex
% !TeX root = FCLoopSplit.tex

\documentclass[../FeynCalcManual.tex]{subfiles}
\begin{document}
\hypertarget{fcloopsplit}{
\section{FCLoopSplit}\label{fcloopsplit}\index{FCLoopSplit}}

\texttt{FCLoopSplit[\allowbreak{}exp,\ \allowbreak{}\{\allowbreak{}q1,\ \allowbreak{}q2,\ \allowbreak{}...\}]}
separates \texttt{exp} into the following four pieces:

\begin{enumerate}
\def\labelenumi{\arabic{enumi})}
\item
  terms that are free of loop integrals
\item
  terms with scalar loop integrals
\item
  terms with tensor loop integrals, where all loop momenta are
  contracted
\item
  terms with tensor loop integrals, where at least some loop momenta
  have free indices
\end{enumerate}

The result is returned as a list with the 4 above elements.

\subsection{See also}

\hyperlink{toc}{Overview}

\subsection{Examples}

\begin{Shaded}
\begin{Highlighting}[]
\NormalTok{FVD}\OperatorTok{[}\FunctionTok{q}\OperatorTok{,} \SpecialCharTok{\textbackslash{}}\OperatorTok{[}\NormalTok{Mu}\OperatorTok{]]}\NormalTok{ FAD}\OperatorTok{[\{}\FunctionTok{q}\OperatorTok{,} \FunctionTok{m}\OperatorTok{\}]} 
 
\NormalTok{FCLoopSplit}\OperatorTok{[}\SpecialCharTok{\%}\OperatorTok{,} \OperatorTok{\{}\FunctionTok{q}\OperatorTok{\}]}
\end{Highlighting}
\end{Shaded}

\begin{dmath*}\breakingcomma
\frac{q^{\mu }}{q^2-m^2}
\end{dmath*}

\begin{dmath*}\breakingcomma
\left\{0,0,0,\frac{q^{\mu }}{q^2-m^2}\right\}
\end{dmath*}

\begin{Shaded}
\begin{Highlighting}[]
\FunctionTok{x} \SpecialCharTok{+}\NormalTok{ GSD}\OperatorTok{[}\FunctionTok{p} \SpecialCharTok{+} \FunctionTok{q}\OperatorTok{]}\NormalTok{ FAD}\OperatorTok{[\{}\FunctionTok{q}\OperatorTok{,} \FunctionTok{m}\OperatorTok{\}]} 
 
\NormalTok{FCLoopSplit}\OperatorTok{[}\SpecialCharTok{\%}\OperatorTok{,} \OperatorTok{\{}\FunctionTok{q}\OperatorTok{\}]}
\end{Highlighting}
\end{Shaded}

\begin{dmath*}\breakingcomma
\frac{\gamma \cdot (p+q)}{q^2-m^2}+x
\end{dmath*}

\begin{dmath*}\breakingcomma
\left\{x,\frac{\gamma \cdot p}{q^2-m^2},0,\frac{\gamma \cdot q}{q^2-m^2}\right\}
\end{dmath*}
\end{document}
