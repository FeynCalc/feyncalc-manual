% !TeX program = pdflatex
% !TeX root = FCLoopSwitchEtaSign.tex

\documentclass[../FeynCalcManual.tex]{subfiles}
\begin{document}
\hypertarget{fcloopswitchetasign}{%
\section{FCLoopSwitchEtaSign}\label{fcloopswitchetasign}}

\texttt{FCLoopSwitchEtaSign[\allowbreak{}exp,\ \allowbreak{}s]} switches
the sign of \(i \eta\) in all integrals to \texttt{s}, where \texttt{s}
can be \texttt{+1} or \texttt{-1}.

Notice to change the sign of \(i \eta\) the function pulls out a factor
\(-1\) from the propagator.

\subsection{See also}

\hyperlink{toc}{Overview}, \hyperlink{fctopology}{FCTopology},
\hyperlink{fcloopgetetasigns}{FCLoopGetEtaSigns}.

\subsection{Examples}

\texttt{FAD}s are automatically converted to \texttt{SFAD}s, since
otherwise their \(i \eta\) prescription cannot be modified

\begin{Shaded}
\begin{Highlighting}[]
\NormalTok{FAD}\OperatorTok{[\{}\FunctionTok{p}\OperatorTok{,} \FunctionTok{m}\OperatorTok{\}]} 
 
\NormalTok{FCLoopSwitchEtaSign}\OperatorTok{[}\SpecialCharTok{\%}\OperatorTok{,} \DecValTok{1}\OperatorTok{]}
\end{Highlighting}
\end{Shaded}

\begin{dmath*}\breakingcomma
\frac{1}{p^2-m^2}
\end{dmath*}

\begin{dmath*}\breakingcomma
\frac{1}{(p^2-m^2+i \eta )}
\end{dmath*}

\begin{Shaded}
\begin{Highlighting}[]
\NormalTok{FAD}\OperatorTok{[\{}\FunctionTok{p}\OperatorTok{,} \FunctionTok{m}\OperatorTok{\}]} 
 
\NormalTok{FCLoopSwitchEtaSign}\OperatorTok{[}\SpecialCharTok{\%}\OperatorTok{,} \SpecialCharTok{{-}}\DecValTok{1}\OperatorTok{]}
\end{Highlighting}
\end{Shaded}

\begin{dmath*}\breakingcomma
\frac{1}{p^2-m^2}
\end{dmath*}

\begin{dmath*}\breakingcomma
-\frac{1}{(-p^2+m^2-i \eta )}
\end{dmath*}

\begin{Shaded}
\begin{Highlighting}[]
\NormalTok{SFAD}\OperatorTok{[\{}\FunctionTok{p}\OperatorTok{,} \FunctionTok{m}\SpecialCharTok{\^{}}\DecValTok{2}\OperatorTok{\}]} 
 
\NormalTok{FCLoopSwitchEtaSign}\OperatorTok{[}\SpecialCharTok{\%}\OperatorTok{,} \SpecialCharTok{{-}}\DecValTok{1}\OperatorTok{]}
\end{Highlighting}
\end{Shaded}

\begin{dmath*}\breakingcomma
\frac{1}{(p^2-m^2+i \eta )}
\end{dmath*}

\begin{dmath*}\breakingcomma
-\frac{1}{(-p^2+m^2-i \eta )}
\end{dmath*}

\begin{Shaded}
\begin{Highlighting}[]
\NormalTok{CFAD}\OperatorTok{[\{}\FunctionTok{p}\OperatorTok{,} \FunctionTok{m}\SpecialCharTok{\^{}}\DecValTok{2}\OperatorTok{\}]} 
 
\NormalTok{FCLoopSwitchEtaSign}\OperatorTok{[}\SpecialCharTok{\%}\OperatorTok{,} \DecValTok{1}\OperatorTok{]}
\end{Highlighting}
\end{Shaded}

\begin{dmath*}\breakingcomma
\frac{1}{(p^2+m^2-i \eta )}
\end{dmath*}

\begin{dmath*}\breakingcomma
-\frac{1}{(-p^2-m^2+i \eta )}
\end{dmath*}
\end{document}
