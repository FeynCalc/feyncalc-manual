% !TeX program = pdflatex
% !TeX root = FCLoopTensorReduce.tex

\documentclass[../FeynCalcManual.tex]{subfiles}
\begin{document}
\hypertarget{fclooptensorreduce}{
\section{FCLoopTensorReduce}\label{fclooptensorreduce}\index{FCLoopTensorReduce}}

\texttt{FCLoopTensorReduce[\allowbreak{}exp,\ \allowbreak{}topos]}
performs tensor reduction for the numerators of multi-loop integrals
present in \texttt{exp}. Notice that \texttt{exp} is expected to be the
output of \texttt{FCLoopFindTopologies}where all loop integrals have
been written as
\texttt{fun[\allowbreak{}num,\ \allowbreak{}GLI[\allowbreak{}...]]} with
\texttt{num} being the numerator to be acted upon.

The reduction is done only for loop momenta contracted with Dirac
matrices, polarization vectors or Levi-Civita tensors. Scalar products
with external momenta are left untouched. The goal is to rewrite
everything in terms of scalar products involving only loop momenta and
external momenta appearing in the given topology. These quantities can
be then rewritten in terms of inverse propagators (\texttt{GLI}s with
negative indices), so that the complete dependence on loop momenta will
go into the \texttt{GLI}s.

Unlike \texttt{FCMultiLoopTID}, this function does not perform any
partial fractioning or shifts in the loop momenta.

The default value for \texttt{fun} is FCGV{[}``GLIProduct''{]} set by
the option \texttt{Head}

\subsection{See also}

\hyperlink{toc}{Overview},
\hyperlink{fcloopfindtopologies}{FCLoopFindTopologies}.

\subsection{Examples}

1-loop tadpole topology

\begin{Shaded}
\begin{Highlighting}[]
\NormalTok{topo1 }\ExtensionTok{=}\NormalTok{ FCTopology}\OperatorTok{[}\StringTok{"tad1l"}\OperatorTok{,} \OperatorTok{\{}\NormalTok{SFAD}\OperatorTok{[\{}\FunctionTok{q}\OperatorTok{,} \FunctionTok{m}\SpecialCharTok{\^{}}\DecValTok{2}\OperatorTok{\}]\},} \OperatorTok{\{}\FunctionTok{q}\OperatorTok{\},} \OperatorTok{\{\},} \OperatorTok{\{\},} \OperatorTok{\{\}]}
\end{Highlighting}
\end{Shaded}

\begin{dmath*}\breakingcomma
\text{FCTopology}\left(\text{tad1l},\left\{\frac{1}{(q^2-m^2+i \eta )}\right\},\{q\},\{\},\{\},\{\}\right)
\end{dmath*}

\begin{Shaded}
\begin{Highlighting}[]
\NormalTok{amp1 }\ExtensionTok{=}\NormalTok{ FCGV}\OperatorTok{[}\StringTok{"GLIProduct"}\OperatorTok{][}\NormalTok{GSD}\OperatorTok{[}\FunctionTok{q}\OperatorTok{]}\NormalTok{ . GAD}\OperatorTok{[}\SpecialCharTok{\textbackslash{}}\OperatorTok{[}\NormalTok{Mu}\OperatorTok{]]}\NormalTok{ . GSD}\OperatorTok{[}\FunctionTok{q}\OperatorTok{],}\NormalTok{ GLI}\OperatorTok{[}\StringTok{"tad1l"}\OperatorTok{,} \OperatorTok{\{}\DecValTok{1}\OperatorTok{\}]]}
\end{Highlighting}
\end{Shaded}

\begin{dmath*}\breakingcomma
\text{FCGV}(\text{GLIProduct})\left((\gamma \cdot q).\gamma ^{\mu }.(\gamma \cdot q),G^{\text{tad1l}}(1)\right)
\end{dmath*}

\begin{Shaded}
\begin{Highlighting}[]
\NormalTok{amp1Red }\ExtensionTok{=}\NormalTok{ FCLoopTensorReduce}\OperatorTok{[}\NormalTok{amp1}\OperatorTok{,} \OperatorTok{\{}\NormalTok{topo1}\OperatorTok{\}]}
\end{Highlighting}
\end{Shaded}

\begin{dmath*}\breakingcomma
\text{FCGV}(\text{GLIProduct})\left(\frac{(2-D) q^2 \gamma ^{\mu }}{D},G^{\text{tad1l}}(1)\right)
\end{dmath*}

\begin{Shaded}
\begin{Highlighting}[]
\NormalTok{topo2 }\ExtensionTok{=}\NormalTok{ FCTopology}\OperatorTok{[}\NormalTok{prop1l}\OperatorTok{,} \OperatorTok{\{}\NormalTok{SFAD}\OperatorTok{[\{}\FunctionTok{q}\OperatorTok{,} \FunctionTok{m}\SpecialCharTok{\^{}}\DecValTok{2}\OperatorTok{\},} \OperatorTok{\{}\FunctionTok{q} \SpecialCharTok{{-}} \FunctionTok{p}\OperatorTok{,} \FunctionTok{m}\SpecialCharTok{\^{}}\DecValTok{2}\OperatorTok{\}]\},} \OperatorTok{\{}\FunctionTok{q}\OperatorTok{\},} \OperatorTok{\{}\FunctionTok{p}\OperatorTok{\},} \OperatorTok{\{\},} \OperatorTok{\{\}]}
\end{Highlighting}
\end{Shaded}

\begin{dmath*}\breakingcomma
\text{FCTopology}\left(\text{prop1l},\left\{\frac{1}{(q^2-m^2+i \eta ).((q-p)^2-m^2+i \eta )}\right\},\{q\},\{p\},\{\},\{\}\right)
\end{dmath*}

1-loop self-energy topology

\begin{Shaded}
\begin{Highlighting}[]
\NormalTok{amp2 }\ExtensionTok{=}\NormalTok{ gliProduct}\OperatorTok{[}\NormalTok{GSD}\OperatorTok{[}\FunctionTok{q}\OperatorTok{]}\NormalTok{ . GAD}\OperatorTok{[}\SpecialCharTok{\textbackslash{}}\OperatorTok{[}\NormalTok{Mu}\OperatorTok{]]}\NormalTok{ . GSD}\OperatorTok{[}\FunctionTok{q}\OperatorTok{],}\NormalTok{ GLI}\OperatorTok{[}\NormalTok{prop1l}\OperatorTok{,} \OperatorTok{\{}\DecValTok{1}\OperatorTok{,} \DecValTok{2}\OperatorTok{\}]]}
\end{Highlighting}
\end{Shaded}

\begin{dmath*}\breakingcomma
\text{gliProduct}\left((\gamma \cdot q).\gamma ^{\mu }.(\gamma \cdot q),G^{\text{prop1l}}(1,2)\right)
\end{dmath*}

\begin{Shaded}
\begin{Highlighting}[]
\NormalTok{amp2Red }\ExtensionTok{=}\NormalTok{ FCLoopTensorReduce}\OperatorTok{[}\NormalTok{amp2}\OperatorTok{,} \OperatorTok{\{}\NormalTok{topo2}\OperatorTok{\},} \FunctionTok{Head} \OtherTok{{-}\textgreater{}}\NormalTok{ gliProduct}\OperatorTok{]}
\end{Highlighting}
\end{Shaded}

\begin{dmath*}\breakingcomma
\text{gliProduct}\left(-\frac{2 D p^{\mu } \gamma \cdot p (p\cdot q)^2-2 p^2 \gamma ^{\mu } (p\cdot q)^2-D p^4 q^2 \gamma ^{\mu }+3 p^4 q^2 \gamma ^{\mu }-2 p^2 q^2 p^{\mu } \gamma \cdot p}{(1-D) p^4},G^{\text{prop1l}}(1,2)\right)
\end{dmath*}

If the loop momenta are contracted with some external momenta that do
not appear in the given integral topologies, they should be listed via
the option \texttt{Uncontract}

\begin{Shaded}
\begin{Highlighting}[]
\NormalTok{amp3 }\ExtensionTok{=}\NormalTok{ gliProduct}\OperatorTok{[}\NormalTok{SPD}\OperatorTok{[}\FunctionTok{q}\OperatorTok{,} \FunctionTok{x}\OperatorTok{],}\NormalTok{ GLI}\OperatorTok{[}\NormalTok{prop1l}\OperatorTok{,} \OperatorTok{\{}\DecValTok{1}\OperatorTok{,} \DecValTok{2}\OperatorTok{\}]]}
\end{Highlighting}
\end{Shaded}

\begin{dmath*}\breakingcomma
\text{gliProduct}\left(q\cdot x,G^{\text{prop1l}}(1,2)\right)
\end{dmath*}

\begin{Shaded}
\begin{Highlighting}[]
\NormalTok{FCLoopTensorReduce}\OperatorTok{[}\NormalTok{amp3}\OperatorTok{,} \OperatorTok{\{}\NormalTok{topo2}\OperatorTok{\},}\NormalTok{ Uncontract }\OtherTok{{-}\textgreater{}} \OperatorTok{\{}\FunctionTok{x}\OperatorTok{\},} \FunctionTok{Head} \OtherTok{{-}\textgreater{}}\NormalTok{ gliProduct}\OperatorTok{]}
\end{Highlighting}
\end{Shaded}

\begin{dmath*}\breakingcomma
\text{gliProduct}\left(\frac{(p\cdot q) (p\cdot x)}{p^2},G^{\text{prop1l}}(1,2)\right)
\end{dmath*}
\end{document}
