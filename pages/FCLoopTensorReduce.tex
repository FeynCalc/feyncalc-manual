% !TeX program = pdflatex
% !TeX root = FCLoopTensorReduce.tex

\documentclass[../FeynCalcManual.tex]{subfiles}
\begin{document}
\hypertarget{fclooptensorreduce}{
\section{FCLoopTensorReduce}\label{fclooptensorreduce}\index{FCLoopTensorReduce}}

\texttt{FCLoopTensorReduce[\allowbreak{}exp,\ \allowbreak{}topos]}
performs tensor reduction for the numerators of multi-loop integrals
present in \texttt{exp}. Notice that \texttt{exp} is expected to be the
output of \texttt{FCLoopFindTopologies}where all loop integrals have
been written as
\texttt{fun[\allowbreak{}num,\ \allowbreak{}GLI[\allowbreak{}...]]} with
\texttt{num} being the numerator to be acted upon.

Unlike \texttt{FCMultiLoopTID}, this function does not perform any
partial fractioning or shifts in the loop momenta.

\subsection{See also}

\hyperlink{toc}{Overview},
\hyperlink{fcloopfindtopologies}{FCLoopFindTopologies}.

\subsection{Examples}

\begin{Shaded}
\begin{Highlighting}[]
\NormalTok{FCI}\OperatorTok{[}\NormalTok{FVD}\OperatorTok{[}\NormalTok{q1}\OperatorTok{,} \SpecialCharTok{\textbackslash{}}\OperatorTok{[}\NormalTok{Mu}\OperatorTok{]]}\NormalTok{ FVD}\OperatorTok{[}\NormalTok{q2}\OperatorTok{,} \SpecialCharTok{\textbackslash{}}\OperatorTok{[}\NormalTok{Nu}\OperatorTok{]]}\NormalTok{ FAD}\OperatorTok{[}\NormalTok{q1}\OperatorTok{,}\NormalTok{ q2}\OperatorTok{,} \OperatorTok{\{}\NormalTok{q1 }\SpecialCharTok{{-}}\NormalTok{ p1}\OperatorTok{\},} \OperatorTok{\{}\NormalTok{q2 }\SpecialCharTok{{-}}\NormalTok{ p1}\OperatorTok{\},} \OperatorTok{\{}\NormalTok{q1 }\SpecialCharTok{{-}}\NormalTok{ q2}\OperatorTok{\}]]} 
 
\NormalTok{FCMultiLoopTID}\OperatorTok{[}\SpecialCharTok{\%}\OperatorTok{,} \OperatorTok{\{}\NormalTok{q1}\OperatorTok{,}\NormalTok{ q2}\OperatorTok{\}]}
\end{Highlighting}
\end{Shaded}

\begin{dmath*}\breakingcomma
\frac{\text{q1}^{\mu } \;\text{q2}^{\nu }}{\text{q1}^2.\text{q2}^2.(\text{q1}-\text{p1})^2.(\text{q2}-\text{p1})^2.(\text{q1}-\text{q2})^2}
\end{dmath*}

\begin{dmath*}\breakingcomma
\frac{\text{p1}^{\mu } \;\text{p1}^{\nu }-\text{p1}^2 g^{\mu \nu }}{(1-D) \;\text{p1}^2 \;\text{q1}^2.\text{q2}^2.(\text{q1}-\text{p1})^2.(\text{q1}-\text{q2})^2}-\frac{\text{p1}^{\mu } \;\text{p1}^{\nu }-\text{p1}^2 g^{\mu \nu }}{2 (1-D) \;\text{p1}^2 \;\text{q1}^2.\text{q2}^2.(\text{q1}-\text{p1})^2.(\text{q2}-\text{p1})^2}-\frac{D \;\text{p1}^{\mu } \;\text{p1}^{\nu }-\text{p1}^2 g^{\mu \nu }}{4 (1-D) \;\text{q1}^2.\text{q2}^2.(\text{q1}-\text{p1})^2.(\text{q1}-\text{q2})^2.(\text{q2}-\text{p1})^2}+\frac{D \;\text{p1}^{\mu } \;\text{p1}^{\nu }-\text{p1}^2 g^{\mu \nu }}{2 (1-D) \;\text{p1}^4 \;\text{q1}^2.(\text{q1}-\text{q2})^2.(\text{q2}-\text{p1})^2}
\end{dmath*}
\end{document}
