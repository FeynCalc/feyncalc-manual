% !TeX program = pdflatex
% !TeX root = FCMakeIndex.tex

\documentclass[../FeynCalcManual.tex]{subfiles}
\begin{document}
\hypertarget{fcmakeindex}{%
\section{FCMakeIndex}\label{fcmakeindex}}

\texttt{FCMakeIndex[\allowbreak{}str1,\ \allowbreak{}str2,\ \allowbreak{}head]}
generates an index with the given head out of the string \texttt{str1}
and \texttt{str2}. For example,
\texttt{FCMakeIndex[\allowbreak{}"Lor",\ \allowbreak{}"1",\ \allowbreak{}LorentzIndex]}
yields \texttt{LorentzIndex[\allowbreak{}Lor1]}. The second argument can
also be an integer. \texttt{FCMakeIndex} is useful for converting the
output of different diagram generators such as FeynArts or QGAF into the
FeynCalc notation. It uses memoization to improve the performance.

\subsection{See also}

\hyperlink{toc}{Overview}, \hyperlink{fcmakesymbols}{FCMakeSymbols}.

\subsection{Examples}

\begin{Shaded}
\begin{Highlighting}[]
\NormalTok{FCMakeIndex}\OperatorTok{[}\StringTok{"Lor"}\OperatorTok{,} \StringTok{"1"}\OperatorTok{]}
\end{Highlighting}
\end{Shaded}

\begin{dmath*}\breakingcomma
\text{Lor1}
\end{dmath*}

\begin{Shaded}
\begin{Highlighting}[]
\NormalTok{FCMakeIndex}\OperatorTok{[}\StringTok{"Lor"}\OperatorTok{,} \StringTok{"1"}\OperatorTok{]} \SpecialCharTok{//} \FunctionTok{StandardForm}

\CommentTok{(*Lor1*)}
\end{Highlighting}
\end{Shaded}

\begin{Shaded}
\begin{Highlighting}[]
\NormalTok{FCMakeIndex}\OperatorTok{[}\StringTok{"Lor"}\OperatorTok{,} \OperatorTok{\{}\DecValTok{3}\OperatorTok{,} \DecValTok{1}\OperatorTok{,} \DecValTok{4}\OperatorTok{\},}\NormalTok{ LorentzIndex}\OperatorTok{]}
\end{Highlighting}
\end{Shaded}

\begin{dmath*}\breakingcomma
\{\text{Lor3},\text{Lor1},\text{Lor4}\}
\end{dmath*}

\begin{Shaded}
\begin{Highlighting}[]
\NormalTok{FCMakeIndex}\OperatorTok{[}\StringTok{"Lor"}\OperatorTok{,} \OperatorTok{\{}\DecValTok{3}\OperatorTok{,} \DecValTok{1}\OperatorTok{,} \DecValTok{4}\OperatorTok{\},}\NormalTok{ LorentzIndex}\OperatorTok{]} \SpecialCharTok{//} \FunctionTok{StandardForm}

\CommentTok{(*\{LorentzIndex[Lor3], LorentzIndex[Lor1], LorentzIndex[Lor4]\}*)}
\end{Highlighting}
\end{Shaded}

\begin{Shaded}
\begin{Highlighting}[]
\NormalTok{FCMakeIndex}\OperatorTok{[}\StringTok{"Sun"}\OperatorTok{,} \OperatorTok{\{}\StringTok{"a"}\OperatorTok{,} \DecValTok{1}\OperatorTok{,} \SpecialCharTok{{-}}\DecValTok{4}\OperatorTok{\}]}
\end{Highlighting}
\end{Shaded}

\begin{dmath*}\breakingcomma
\{\text{Suna},\text{Sun1},\text{SunMinus4}\}
\end{dmath*}

\begin{Shaded}
\begin{Highlighting}[]
\NormalTok{FCMakeIndex}\OperatorTok{[}\StringTok{"Sun"}\OperatorTok{,} \OperatorTok{\{}\StringTok{"a"}\OperatorTok{,} \DecValTok{1}\OperatorTok{,} \SpecialCharTok{{-}}\DecValTok{4}\OperatorTok{\}]} \SpecialCharTok{//} \FunctionTok{StandardForm}

\CommentTok{(*\{Suna, Sun1, SunMinus4\}*)}
\end{Highlighting}
\end{Shaded}

\end{document}
