% !TeX program = pdflatex
% !TeX root = FCMakeSymbols.tex

\documentclass[../FeynCalcManual.tex]{subfiles}
\begin{document}
\hypertarget{fcmakesymbols}{%
\section{FCMakeSymbols}\label{fcmakesymbols}}

\texttt{FCMakeSymbols[\allowbreak{}name,\ \allowbreak{}range,\ \allowbreak{}type]}
generates a list or a sequence of symbols (depending on the value of
type) by attaching elements of the list range to name.

For example,
\texttt{FCMakeSymbols[\allowbreak{}mu,\ \allowbreak{}Range[\allowbreak{}1,\ \allowbreak{}3],\ \allowbreak{}List]}
returns
\texttt{\{\allowbreak{}mu1,\ \allowbreak{}mu2,\ \allowbreak{}mu3\}}.

\subsection{See also}

\hyperlink{toc}{Overview}, \hyperlink{fcmakeindex}{FCMakeIndex}.

\subsection{Examples}

\begin{Shaded}
\begin{Highlighting}[]
\NormalTok{FCMakeSymbols}\OperatorTok{[}\FunctionTok{a}\OperatorTok{,} \FunctionTok{Range}\OperatorTok{[}\DecValTok{1}\OperatorTok{,} \DecValTok{4}\OperatorTok{],} \FunctionTok{List}\OperatorTok{]}
\end{Highlighting}
\end{Shaded}

\begin{dmath*}\breakingcomma
\{\text{a1},\text{a2},\text{a3},\text{a4}\}
\end{dmath*}

\begin{Shaded}
\begin{Highlighting}[]
\FunctionTok{f}\OperatorTok{[}\NormalTok{FCMakeSymbols}\OperatorTok{[}\FunctionTok{a}\OperatorTok{,} \FunctionTok{Range}\OperatorTok{[}\DecValTok{1}\OperatorTok{,} \DecValTok{4}\OperatorTok{],} \FunctionTok{Sequence}\OperatorTok{]]}
\end{Highlighting}
\end{Shaded}

\begin{dmath*}\breakingcomma
f(\text{a1},\text{a2},\text{a3},\text{a4})
\end{dmath*}

\begin{Shaded}
\begin{Highlighting}[]
\FunctionTok{f}\OperatorTok{[}\NormalTok{FCMakeSymbols}\OperatorTok{[}\FunctionTok{a}\OperatorTok{,} \OperatorTok{\{}\DecValTok{1}\OperatorTok{,} \DecValTok{3}\OperatorTok{\},} \FunctionTok{Sequence}\OperatorTok{]]}
\end{Highlighting}
\end{Shaded}

\begin{dmath*}\breakingcomma
f(\text{a1},\text{a3})
\end{dmath*}
\end{document}
