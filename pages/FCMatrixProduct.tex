% !TeX program = pdflatex
% !TeX root = FCMatrixProduct.tex

\documentclass[../FeynCalcManual.tex]{subfiles}
\begin{document}
\hypertarget{fcmatrixproduct}{
\section{FCMatrixProduct}\label{fcmatrixproduct}\index{FCMatrixProduct}}

\texttt{FCMatrixProduct[\allowbreak{}mat1,\ \allowbreak{}mat2,\ \allowbreak{}...]}
can be used to obtain products of matrices with entries containing
noncommutative symbols. Using the usual \texttt{Dot} on such matrices
would otherwise destroy the original ordering.

The resulting expression can be then further simplified using
\texttt{DotSimplify}.

\subsection{See also}

\hyperlink{toc}{Overview}, \hyperlink{datatype}{DataType},
\hyperlink{declarenoncommutative}{DeclareNonCommutative},
\hyperlink{undeclarenoncommutative}{UnDeclareNonCommutative}.

\subsection{Examples}

\hypertarget{generic-matrices}{%
\subsubsection{Generic matrices}\label{generic-matrices}}

Consider two generic \(2 \times 2\)-matrices containing noncommutative
heads

\begin{Shaded}
\begin{Highlighting}[]
\NormalTok{DeclareNonCommutative}\OperatorTok{[}\NormalTok{opA}\OperatorTok{,}\NormalTok{ opB}\OperatorTok{,}\NormalTok{ opC}\OperatorTok{,}\NormalTok{ opD}\OperatorTok{]}
\end{Highlighting}
\end{Shaded}

\begin{Shaded}
\begin{Highlighting}[]
\NormalTok{mat}\OperatorTok{[}\DecValTok{1}\OperatorTok{]} \ExtensionTok{=} \OperatorTok{\{\{}\NormalTok{opA}\OperatorTok{[}\DecValTok{1}\OperatorTok{],}\NormalTok{ opB}\OperatorTok{[}\DecValTok{1}\OperatorTok{]\},} \OperatorTok{\{}\NormalTok{opC}\OperatorTok{[}\DecValTok{1}\OperatorTok{],}\NormalTok{ opD}\OperatorTok{[}\DecValTok{1}\OperatorTok{]\}\}} 
 
\NormalTok{mat}\OperatorTok{[}\DecValTok{2}\OperatorTok{]} \ExtensionTok{=} \OperatorTok{\{\{}\NormalTok{opA}\OperatorTok{[}\DecValTok{2}\OperatorTok{],}\NormalTok{ opB}\OperatorTok{[}\DecValTok{2}\OperatorTok{]\},} \OperatorTok{\{}\NormalTok{opC}\OperatorTok{[}\DecValTok{2}\OperatorTok{],}\NormalTok{ opD}\OperatorTok{[}\DecValTok{2}\OperatorTok{]\}\}}
\end{Highlighting}
\end{Shaded}

\begin{dmath*}\breakingcomma
\left(
\begin{array}{cc}
 \;\text{opA}(1) & \;\text{opB}(1) \\
 \;\text{opC}(1) & \;\text{opD}(1) \\
\end{array}
\right)
\end{dmath*}

\begin{dmath*}\breakingcomma
\left(
\begin{array}{cc}
 \;\text{opA}(2) & \;\text{opB}(2) \\
 \;\text{opC}(2) & \;\text{opD}(2) \\
\end{array}
\right)
\end{dmath*}

Using the usual \texttt{Dot} product the elements of the resulting
matrix are now multiplied with each other commutatively. Hence, the
result is incorrect.

\begin{Shaded}
\begin{Highlighting}[]
\NormalTok{mat}\OperatorTok{[}\DecValTok{1}\OperatorTok{]}\NormalTok{ . mat}\OperatorTok{[}\DecValTok{2}\OperatorTok{]}
\end{Highlighting}
\end{Shaded}

\begin{dmath*}\breakingcomma
\left(
\begin{array}{cc}
 \;\text{opA}(1) \;\text{opA}(2)+\text{opB}(1) \;\text{opC}(2) & \;\text{opA}(1) \;\text{opB}(2)+\text{opB}(1) \;\text{opD}(2) \\
 \;\text{opA}(2) \;\text{opC}(1)+\text{opC}(2) \;\text{opD}(1) & \;\text{opB}(2) \;\text{opC}(1)+\text{opD}(1) \;\text{opD}(2) \\
\end{array}
\right)
\end{dmath*}

With \texttt{FCMatrixProduct} the proper ordering is preserved

\begin{Shaded}
\begin{Highlighting}[]
\NormalTok{FCMatrixProduct}\OperatorTok{[}\NormalTok{mat}\OperatorTok{[}\DecValTok{1}\OperatorTok{],}\NormalTok{ mat}\OperatorTok{[}\DecValTok{2}\OperatorTok{]]}
\end{Highlighting}
\end{Shaded}

\begin{dmath*}\breakingcomma
\left(
\begin{array}{cc}
 \;\text{opA}(1).\text{opA}(2)+\text{opB}(1).\text{opC}(2) & \;\text{opA}(1).\text{opB}(2)+\text{opB}(1).\text{opD}(2) \\
 \;\text{opC}(1).\text{opA}(2)+\text{opD}(1).\text{opC}(2) & \;\text{opC}(1).\text{opB}(2)+\text{opD}(1).\text{opD}(2) \\
\end{array}
\right)
\end{dmath*}

We can also multiply more than two matrices at once

\begin{Shaded}
\begin{Highlighting}[]
\NormalTok{mat}\OperatorTok{[}\DecValTok{3}\OperatorTok{]} \ExtensionTok{=} \OperatorTok{\{\{}\NormalTok{opA}\OperatorTok{[}\DecValTok{3}\OperatorTok{],}\NormalTok{ opB}\OperatorTok{[}\DecValTok{3}\OperatorTok{]\},} \OperatorTok{\{}\NormalTok{opC}\OperatorTok{[}\DecValTok{3}\OperatorTok{],}\NormalTok{ opD}\OperatorTok{[}\DecValTok{3}\OperatorTok{]\}\}}
\end{Highlighting}
\end{Shaded}

\begin{dmath*}\breakingcomma
\left(
\begin{array}{cc}
 \;\text{opA}(3) & \;\text{opB}(3) \\
 \;\text{opC}(3) & \;\text{opD}(3) \\
\end{array}
\right)
\end{dmath*}

\begin{Shaded}
\begin{Highlighting}[]
\FunctionTok{out} \ExtensionTok{=}\NormalTok{ FCMatrixProduct}\OperatorTok{[}\NormalTok{mat}\OperatorTok{[}\DecValTok{1}\OperatorTok{],}\NormalTok{ mat}\OperatorTok{[}\DecValTok{2}\OperatorTok{],}\NormalTok{ mat}\OperatorTok{[}\DecValTok{3}\OperatorTok{]]}
\end{Highlighting}
\end{Shaded}

\begin{dmath*}\breakingcomma
\left(
\begin{array}{cc}
 \;\text{opB}(1).(\text{opC}(2).\text{opA}(3)+\text{opD}(2).\text{opC}(3))+\text{opA}(1).(\text{opA}(2).\text{opA}(3)+\text{opB}(2).\text{opC}(3)) & \;\text{opA}(1).(\text{opA}(2).\text{opB}(3)+\text{opB}(2).\text{opD}(3))+\text{opB}(1).(\text{opC}(2).\text{opB}(3)+\text{opD}(2).\text{opD}(3)) \\
 \;\text{opC}(1).(\text{opA}(2).\text{opA}(3)+\text{opB}(2).\text{opC}(3))+\text{opD}(1).(\text{opC}(2).\text{opA}(3)+\text{opD}(2).\text{opC}(3)) & \;\text{opC}(1).(\text{opA}(2).\text{opB}(3)+\text{opB}(2).\text{opD}(3))+\text{opD}(1).(\text{opC}(2).\text{opB}(3)+\text{opD}(2).\text{opD}(3)) \\
\end{array}
\right)
\end{dmath*}

Now use \texttt{DotSimplify} to expand noncommutative products

\begin{Shaded}
\begin{Highlighting}[]
\NormalTok{DotSimplify}\OperatorTok{[}\FunctionTok{out}\OperatorTok{]}
\end{Highlighting}
\end{Shaded}

\begin{dmath*}\breakingcomma
\left(
\begin{array}{cc}
 \;\text{opA}(1).\text{opB}(2).\text{opC}(3)+\text{opB}(1).\text{opC}(2).\text{opA}(3)+\text{opA}(1).\text{opA}(2).\text{opA}(3)+\text{opB}(1).\text{opD}(2).\text{opC}(3) & \;\text{opA}(1).\text{opB}(2).\text{opD}(3)+\text{opA}(1).\text{opA}(2).\text{opB}(3)+\text{opB}(1).\text{opC}(2).\text{opB}(3)+\text{opB}(1).\text{opD}(2).\text{opD}(3) \\
 \;\text{opD}(1).\text{opC}(2).\text{opA}(3)+\text{opC}(1).\text{opA}(2).\text{opA}(3)+\text{opC}(1).\text{opB}(2).\text{opC}(3)+\text{opD}(1).\text{opD}(2).\text{opC}(3) & \;\text{opC}(1).\text{opA}(2).\text{opB}(3)+\text{opC}(1).\text{opB}(2).\text{opD}(3)+\text{opD}(1).\text{opC}(2).\text{opB}(3)+\text{opD}(1).\text{opD}(2).\text{opD}(3) \\
\end{array}
\right)
\end{dmath*}

\hypertarget{dirac-matrices-in-terms-of-pauli-matrices}{%
\subsubsection{Dirac matrices in terms of Pauli
matrices}\label{dirac-matrices-in-terms-of-pauli-matrices}}

Let us define Dirac matrices in the Dirac basis in terms of Pauli
matrices

\begin{Shaded}
\begin{Highlighting}[]
\FunctionTok{gamma}\OperatorTok{[}\DecValTok{0}\OperatorTok{]} \ExtensionTok{=} \OperatorTok{\{\{}\DecValTok{1}\OperatorTok{,} \DecValTok{0}\OperatorTok{\},} \OperatorTok{\{}\DecValTok{0}\OperatorTok{,} \SpecialCharTok{{-}}\DecValTok{1}\OperatorTok{\}\}}\NormalTok{; }
 
\FunctionTok{gamma}\OperatorTok{[}\AttributeTok{i\_}\OperatorTok{]} \ExtensionTok{:=} \OperatorTok{\{\{}\DecValTok{0}\OperatorTok{,}\NormalTok{ CSI}\OperatorTok{[}\FunctionTok{i}\OperatorTok{]\},} \OperatorTok{\{}\SpecialCharTok{{-}}\NormalTok{CSI}\OperatorTok{[}\FunctionTok{i}\OperatorTok{],} \DecValTok{0}\OperatorTok{\}\}}\NormalTok{;}
\end{Highlighting}
\end{Shaded}

and express \(\gamma^i \gamma^j \gamma^i\) as a \(2 \times 2\)-matrix

\begin{Shaded}
\begin{Highlighting}[]
\NormalTok{FCMatrixProduct}\OperatorTok{[}\FunctionTok{gamma}\OperatorTok{[}\FunctionTok{i}\OperatorTok{],} \FunctionTok{gamma}\OperatorTok{[}\FunctionTok{j}\OperatorTok{],} \FunctionTok{gamma}\OperatorTok{[}\FunctionTok{i}\OperatorTok{]]} 
 
\NormalTok{DotSimplify}\OperatorTok{[}\SpecialCharTok{\%}\OperatorTok{]}
\end{Highlighting}
\end{Shaded}

\begin{dmath*}\breakingcomma
\left(
\begin{array}{cc}
 0.\left(\overline{\sigma }^j.\left(-\overline{\sigma }^i\right)+0.0\right)+\overline{\sigma }^i.\left(0.\left(-\overline{\sigma }^i\right)+\left(-\overline{\sigma }^j\right).0\right) & 0.\left(0.\overline{\sigma }^i+\overline{\sigma }^j.0\right)+\overline{\sigma }^i.\left(\left(-\overline{\sigma }^j\right).\overline{\sigma }^i+0.0\right) \\
 0.\left(0.\left(-\overline{\sigma }^i\right)+\left(-\overline{\sigma }^j\right).0\right)+\left(-\overline{\sigma }^i\right).\left(\overline{\sigma }^j.\left(-\overline{\sigma }^i\right)+0.0\right) & 0.\left(\left(-\overline{\sigma }^j\right).\overline{\sigma }^i+0.0\right)+\left(-\overline{\sigma }^i\right).\left(0.\overline{\sigma }^i+\overline{\sigma }^j.0\right) \\
\end{array}
\right)
\end{dmath*}

\begin{dmath*}\breakingcomma
\left(
\begin{array}{cc}
 0 & -\overline{\sigma }^i.\overline{\sigma }^j.\overline{\sigma }^i \\
 \overline{\sigma }^i.\overline{\sigma }^j.\overline{\sigma }^i & 0 \\
\end{array}
\right)
\end{dmath*}
\end{document}
