% !TeX program = pdflatex
% !TeX root = FCMellinJoin.tex

\documentclass[../FeynCalcManual.tex]{subfiles}
\begin{document}
\begin{Shaded}
\begin{Highlighting}[]
 
\end{Highlighting}
\end{Shaded}

\hypertarget{fcmellinjoin}{
\section{FCMellinJoin}\label{fcmellinjoin}\index{FCMellinJoin}}

\texttt{FCMellinJoin[\allowbreak{}int,\ \allowbreak{}\{\allowbreak{}q1,\ \allowbreak{}q2,\ \allowbreak{}...\},\ \allowbreak{}\{\allowbreak{}prop1,\ \allowbreak{}prop2,\ \allowbreak{}...\}]}
applies the standard formula for splitting propagators
\texttt{prop1,\ \allowbreak{}prop2,\ \allowbreak{}...} into summands by
introducing integrations along a contour in the complex space.

The main purpose of this routine is to convert massive propagators into
massless ones when using Mellin-Barnes integration techniques.

The output consists of a list containing two elements, the first one
being the prefactor and the second one the product of remaining
propagators. The second element (or, alternatively, the product of both
elements) can be then further processed using
\texttt{FCFeynmanParametrize}. Setting the option \texttt{List} to
\texttt{False} will return a product instead of a list.

The option \texttt{FCSplit} can be used to split a propagators in more
than 2 terms as it is done by default.

\subsection{See also}

\hyperlink{toc}{Overview},
\hyperlink{fcfeynmanparametrize}{FCFeynmanParametrize}.

\subsection{Examples}

\begin{Shaded}
\begin{Highlighting}[]
\NormalTok{FCMellinJoin}\OperatorTok{[}\NormalTok{FAD}\OperatorTok{[}\FunctionTok{q}\OperatorTok{,} \OperatorTok{\{}\SpecialCharTok{{-}}\NormalTok{p1 }\SpecialCharTok{+} \FunctionTok{q}\OperatorTok{,} \FunctionTok{m}\OperatorTok{\},} \OperatorTok{\{}\SpecialCharTok{{-}}\NormalTok{p1 }\SpecialCharTok{+} \FunctionTok{q}\OperatorTok{,} \FunctionTok{m}\OperatorTok{\},} \OperatorTok{\{}\SpecialCharTok{{-}}\NormalTok{p2 }\SpecialCharTok{+} \FunctionTok{q}\OperatorTok{,} \FunctionTok{m}\OperatorTok{\}],} \OperatorTok{\{}\FunctionTok{q}\OperatorTok{\},} \OperatorTok{\{}\NormalTok{SFAD}\OperatorTok{[\{}\FunctionTok{q} \SpecialCharTok{{-}}\NormalTok{ p1}\OperatorTok{,} \FunctionTok{m}\SpecialCharTok{\^{}}\DecValTok{2}\OperatorTok{\}],} 
\NormalTok{   SFAD}\OperatorTok{[\{}\FunctionTok{q} \SpecialCharTok{{-}}\NormalTok{ p2}\OperatorTok{,} \FunctionTok{m}\SpecialCharTok{\^{}}\DecValTok{2}\OperatorTok{\}]\},} \FunctionTok{Names} \OtherTok{{-}\textgreater{}} \FunctionTok{z}\OperatorTok{]}
\end{Highlighting}
\end{Shaded}

\begin{dmath*}\breakingcomma
\left\{-\frac{\Gamma (-z(1)) \Gamma (z(1)+2) \Gamma (-z(2)) \Gamma (z(2)+1)}{4 \pi ^2},\frac{\left(-m^2+i \eta \right)^{z(1)+z(2)}}{(q^2+i \eta ).(\text{p1}^2-2 (\text{p1}\cdot q)+q^2+i \eta )^{z(1)+2}.(\text{p2}^2-2 (\text{p2}\cdot q)+q^2+i \eta )^{z(2)+1}}\right\}
\end{dmath*}

\begin{Shaded}
\begin{Highlighting}[]
\NormalTok{FCMellinJoin}\OperatorTok{[}\NormalTok{FAD}\OperatorTok{[}\FunctionTok{q}\OperatorTok{,} \OperatorTok{\{}\SpecialCharTok{{-}}\NormalTok{p1 }\SpecialCharTok{+} \FunctionTok{q}\OperatorTok{,} \FunctionTok{m}\OperatorTok{\},} \OperatorTok{\{}\SpecialCharTok{{-}}\NormalTok{p1 }\SpecialCharTok{+} \FunctionTok{q}\OperatorTok{,} \FunctionTok{m}\OperatorTok{\},} \OperatorTok{\{}\SpecialCharTok{{-}}\NormalTok{p2 }\SpecialCharTok{+} \FunctionTok{q}\OperatorTok{,} \FunctionTok{m}\OperatorTok{\}],} \OperatorTok{\{}\FunctionTok{q}\OperatorTok{\},} \OperatorTok{\{}\NormalTok{SFAD}\OperatorTok{[\{}\FunctionTok{q} \SpecialCharTok{{-}}\NormalTok{ p1}\OperatorTok{,} \FunctionTok{m}\SpecialCharTok{\^{}}\DecValTok{2}\OperatorTok{\}],} 
\NormalTok{   SFAD}\OperatorTok{[\{}\FunctionTok{q} \SpecialCharTok{{-}}\NormalTok{ p2}\OperatorTok{,} \FunctionTok{m}\SpecialCharTok{\^{}}\DecValTok{2}\OperatorTok{\}]\},} \FunctionTok{Names} \OtherTok{{-}\textgreater{}} \FunctionTok{z}\OperatorTok{,}\NormalTok{ FCSplit }\OtherTok{{-}\textgreater{}} \OperatorTok{\{\{}\FunctionTok{q}\OperatorTok{,} \FunctionTok{m}\OperatorTok{,}\NormalTok{ p1}\OperatorTok{\},} \OperatorTok{\{}\FunctionTok{q}\OperatorTok{,} \FunctionTok{m}\OperatorTok{,}\NormalTok{ p2}\OperatorTok{\}\}]}
\end{Highlighting}
\end{Shaded}

\begin{dmath*}\breakingcomma
\left\{-\frac{\Gamma (-z(1)(1)) \Gamma (-z(1)(2)) \Gamma (-z(1)(3)) \Gamma (z(1)(1)+z(1)(2)+z(1)(3)+2) \Gamma (-z(2)(1)) \Gamma (-z(2)(2)) \Gamma (-z(2)(3)) \Gamma (z(2)(1)+z(2)(2)+z(2)(3)+1)}{64 \pi ^6},\frac{\left(-m^2+i \eta \right)^{z(1)(3)+z(2)(3)} \left(q^2+i \eta \right)^{z(1)(2)+z(2)(2)} (-2 (\text{p1}\cdot q)+i \eta )^{z(1)(1)} (-2 (\text{p2}\cdot q)+i \eta )^{z(2)(1)}}{(q^2+i \eta ).(\text{p1}^2+i \eta )^{z(1)(1)+z(1)(2)+z(1)(3)+2}.(\text{p2}^2+i \eta )^{z(2)(1)+z(2)(2)+z(2)(3)+1}}\right\}
\end{dmath*}

\begin{Shaded}
\begin{Highlighting}[]
\NormalTok{FCMellinJoin}\OperatorTok{[}\NormalTok{SFAD}\OperatorTok{[\{}\FunctionTok{k}\OperatorTok{,} \FunctionTok{m}\SpecialCharTok{\^{}}\DecValTok{2}\OperatorTok{,}\NormalTok{ nu1}\OperatorTok{\},} \OperatorTok{\{}\FunctionTok{p} \SpecialCharTok{{-}} \FunctionTok{k}\OperatorTok{,} \DecValTok{0}\OperatorTok{,}\NormalTok{ nu2}\OperatorTok{\}],} \OperatorTok{\{}\FunctionTok{k}\OperatorTok{\},} \OperatorTok{\{}\NormalTok{SFAD}\OperatorTok{[\{}\FunctionTok{k}\OperatorTok{,} \FunctionTok{m}\SpecialCharTok{\^{}}\DecValTok{2}\OperatorTok{\}]\},} 
\NormalTok{  FCLoopSwitchEtaSign }\OtherTok{{-}\textgreater{}} \SpecialCharTok{{-}}\DecValTok{1}\OperatorTok{,} \FunctionTok{Names} \OtherTok{{-}\textgreater{}} \FunctionTok{z}\OperatorTok{]}
\end{Highlighting}
\end{Shaded}

\begin{dmath*}\breakingcomma
\left\{-\frac{i (-1)^{-\text{nu1}-\text{nu2}} \Gamma (-z(1)) \Gamma (\text{nu1}+z(1))}{2 \pi  \Gamma (\text{nu1})},\left(m^2-i \eta \right)^{z(1)} (-k^2-i \eta )^{-\text{nu1}-z(1)} \frac{1}{(-(p-k)^2-i \eta )}^{\text{nu2}}\right\}
\end{dmath*}
\end{document}
