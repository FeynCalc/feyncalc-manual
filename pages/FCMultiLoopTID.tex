% !TeX program = pdflatex
% !TeX root = FCMultiLoopTID.tex

\documentclass[../FeynCalcManual.tex]{subfiles}
\begin{document}
\hypertarget{fcmultilooptid}{
\section{FCMultiLoopTID}\label{fcmultilooptid}\index{FCMultiLoopTID}}

\texttt{FCMultiLoopTID[\allowbreak{}amp,\ \allowbreak{}\{\allowbreak{}q1,\ \allowbreak{}q2,\ \allowbreak{}...\}]}
does a multi-loop tensor integral decomposition, transforming the
Lorentz indices away from the loop momenta
\texttt{q1,\ \allowbreak{}q2,\ \allowbreak{}...} The decomposition is
applied only to the loop integrals where loop momenta are contracted
with Dirac matrices or epsilon tensors.

\subsection{See also}

\hyperlink{toc}{Overview}, \hyperlink{tid}{TID}.

\subsection{Examples}

\begin{Shaded}
\begin{Highlighting}[]
\NormalTok{FCI}\OperatorTok{[}\NormalTok{FVD}\OperatorTok{[}\NormalTok{q1}\OperatorTok{,} \SpecialCharTok{\textbackslash{}}\OperatorTok{[}\NormalTok{Mu}\OperatorTok{]]}\NormalTok{ FVD}\OperatorTok{[}\NormalTok{q2}\OperatorTok{,} \SpecialCharTok{\textbackslash{}}\OperatorTok{[}\NormalTok{Nu}\OperatorTok{]]}\NormalTok{ FAD}\OperatorTok{[}\NormalTok{q1}\OperatorTok{,}\NormalTok{ q2}\OperatorTok{,} \OperatorTok{\{}\NormalTok{q1 }\SpecialCharTok{{-}}\NormalTok{ p1}\OperatorTok{\},} \OperatorTok{\{}\NormalTok{q2 }\SpecialCharTok{{-}}\NormalTok{ p1}\OperatorTok{\},} \OperatorTok{\{}\NormalTok{q1 }\SpecialCharTok{{-}}\NormalTok{ q2}\OperatorTok{\}]]} 
 
\NormalTok{FCMultiLoopTID}\OperatorTok{[}\SpecialCharTok{\%}\OperatorTok{,} \OperatorTok{\{}\NormalTok{q1}\OperatorTok{,}\NormalTok{ q2}\OperatorTok{\}]}
\end{Highlighting}
\end{Shaded}

\begin{dmath*}\breakingcomma
\frac{\text{q1}^{\mu } \;\text{q2}^{\nu }}{\text{q1}^2.\text{q2}^2.(\text{q1}-\text{p1})^2.(\text{q2}-\text{p1})^2.(\text{q1}-\text{q2})^2}
\end{dmath*}

\begin{dmath*}\breakingcomma
\frac{\text{p1}^{\mu } \;\text{p1}^{\nu }-\text{p1}^2 g^{\mu \nu }}{(1-D) \;\text{p1}^2 \;\text{q1}^2.\text{q2}^2.(\text{q1}-\text{p1})^2.(\text{q1}-\text{q2})^2}-\frac{\text{p1}^{\mu } \;\text{p1}^{\nu }-\text{p1}^2 g^{\mu \nu }}{2 (1-D) \;\text{p1}^2 \;\text{q1}^2.\text{q2}^2.(\text{q1}-\text{p1})^2.(\text{q2}-\text{p1})^2}-\frac{D \;\text{p1}^{\mu } \;\text{p1}^{\nu }-\text{p1}^2 g^{\mu \nu }}{4 (1-D) \;\text{q1}^2.\text{q2}^2.(\text{q1}-\text{p1})^2.(\text{q1}-\text{q2})^2.(\text{q2}-\text{p1})^2}+\frac{D \;\text{p1}^{\mu } \;\text{p1}^{\nu }-\text{p1}^2 g^{\mu \nu }}{2 (1-D) \;\text{p1}^4 \;\text{q1}^2.(\text{q1}-\text{q2})^2.(\text{q2}-\text{p1})^2}
\end{dmath*}
\end{document}
