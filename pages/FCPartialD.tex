% !TeX program = pdflatex
% !TeX root = FCPartialD.tex

\documentclass[../FeynCalcManual.tex]{subfiles}
\begin{document}
\hypertarget{fcpartiald}{%
\section{FCPartialD}\label{fcpartiald}}

\texttt{FCPartialD[\allowbreak{}ind]} denotes a partial derivative of a
field. It is an internal object that may appear only inside a
\texttt{QuantumField}.

\texttt{FCPartialD[\allowbreak{}LorentzIndex[\allowbreak{}mu]]} denotes
\(\partial_{\mu }\).

\texttt{FCPartialD[\allowbreak{}LorentzIndex[\allowbreak{}mu ,\ \allowbreak{}D]]}
denotes the \(D\)-dimensional \(\partial_{\mu }\).

\texttt{FCPartialD[\allowbreak{}CartesianIndex[\allowbreak{}i]]} denotes
\(\partial^{i} = - \nabla^i\).

If you need to specify a derivative with respect to a particular
variable it also possible to use
\texttt{FCPartialD[\allowbreak{}\{\allowbreak{}LorentzIndex[\allowbreak{}mu],\ \allowbreak{}y\}]}
or
\texttt{FCPartialD[\allowbreak{}\{\allowbreak{}CartesianIndex[\allowbreak{}i],\ \allowbreak{}x\}]}although
this notation is still somewhat experimental

\subsection{See also}

\hyperlink{toc}{Overview}, \hyperlink{expandpartiald}{ExpandPartialD},
\hyperlink{leftpartiald}{LeftPartialD},
\hyperlink{leftrightpartiald}{LeftRightPartialD},
\hyperlink{rightpartiald}{RightPartialD}.

\subsection{Examples}

\begin{Shaded}
\begin{Highlighting}[]
\NormalTok{QuantumField}\OperatorTok{[}\FunctionTok{A}\OperatorTok{,} \OperatorTok{\{}\SpecialCharTok{\textbackslash{}}\OperatorTok{[}\NormalTok{Mu}\OperatorTok{]\}]}\NormalTok{ . LeftPartialD}\OperatorTok{[}\SpecialCharTok{\textbackslash{}}\OperatorTok{[}\NormalTok{Nu}\OperatorTok{]]} 
 
\NormalTok{ex }\ExtensionTok{=}\NormalTok{ ExpandPartialD}\OperatorTok{[}\SpecialCharTok{\%}\OperatorTok{]}
\end{Highlighting}
\end{Shaded}

\begin{dmath*}\breakingcomma
A_{\mu }.\overleftarrow{\partial }_{\nu }
\end{dmath*}

\begin{dmath*}\breakingcomma
\left.(\partial _{\nu }A_{\mu }\right)
\end{dmath*}

\begin{Shaded}
\begin{Highlighting}[]
\NormalTok{ex }\SpecialCharTok{//} \FunctionTok{StandardForm}

\CommentTok{(*QuantumField[FCPartialD[LorentzIndex[\textbackslash{}[Nu]]], A, LorentzIndex[\textbackslash{}[Mu]]]*)}
\end{Highlighting}
\end{Shaded}

\begin{Shaded}
\begin{Highlighting}[]
\NormalTok{RightPartialD}\OperatorTok{[\{}\NormalTok{CartesianIndex}\OperatorTok{[}\FunctionTok{i}\OperatorTok{],} \FunctionTok{x}\OperatorTok{\}]}\NormalTok{ . QuantumField}\OperatorTok{[}\FunctionTok{S}\OperatorTok{,} \FunctionTok{x}\OperatorTok{]} 
 
\NormalTok{ex }\ExtensionTok{=}\NormalTok{ ExpandPartialD}\OperatorTok{[}\SpecialCharTok{\%}\OperatorTok{]}
\end{Highlighting}
\end{Shaded}

\begin{dmath*}\breakingcomma
\vec{\partial }_{\{i,x\}}.S^x
\end{dmath*}

\begin{dmath*}\breakingcomma
\left.(\partial _{\{i,x\}}S^x\right)
\end{dmath*}

\begin{Shaded}
\begin{Highlighting}[]
\NormalTok{ex }\SpecialCharTok{//} \FunctionTok{StandardForm}

\CommentTok{(*QuantumField[FCPartialD[\{CartesianIndex[i], x\}], S, x]*)}
\end{Highlighting}
\end{Shaded}

\end{document}
