% !TeX program = pdflatex
% !TeX root = FCPartialFractionForm.tex

\documentclass[../FeynCalcManual.tex]{subfiles}
\begin{document}
\begin{Shaded}
\begin{Highlighting}[]
 
\end{Highlighting}
\end{Shaded}

\hypertarget{fcpartialfractionform}{
\section{FCPartialFractionForm}\label{fcpartialfractionform}\index{FCPartialFractionForm}}

\texttt{FCPartialFractionForm[\allowbreak{}n,\ \allowbreak{}\{\allowbreak{}\{\allowbreak{}f1,\ \allowbreak{}x-r1,\ \allowbreak{}p1\},\ \allowbreak{}\{\allowbreak{}f2,\ \allowbreak{}x-r2,\ \allowbreak{}p2\},\ \allowbreak{}...\},\ \allowbreak{}x]}
is a special way of representing sums of rational functions of
\texttt{x} given by
\(n + \frac{f_1}{[x-r_1]^p_1} + \frac{f_2}{[x-r_2]^p_2} + \ldots\)

It is inspired by the \texttt{parfrac}form from Maple and its usage in
E. Panzer's HyperInt for the integration of multiple polylogarithms.

Use \texttt{ToFCPartialFractionForm} to convert the given expression to
this notation and \texttt{FromFCPartialFractionForm} to return back to
the usual representation.

\subsection{See also}

\hyperlink{toc}{Overview},
\hyperlink{tofcpartialfractionform}{ToFCPartialFractionForm},
\hyperlink{fromfcpartialfractionform}{FromFCPartialFractionForm}.

\subsection{Examples}

\begin{Shaded}
\begin{Highlighting}[]
\FunctionTok{Apart}\OperatorTok{[}\FunctionTok{c} \SpecialCharTok{+} \FunctionTok{x}\SpecialCharTok{\^{}}\DecValTok{2}\SpecialCharTok{/}\NormalTok{(}\FunctionTok{x} \SpecialCharTok{{-}} \DecValTok{1}\NormalTok{)}\OperatorTok{,} \FunctionTok{x}\OperatorTok{]}
\end{Highlighting}
\end{Shaded}

\begin{dmath*}\breakingcomma
c+x+\frac{1}{x-1}+1
\end{dmath*}

\begin{Shaded}
\begin{Highlighting}[]
\NormalTok{ex1 }\ExtensionTok{=}\NormalTok{ ToFCPartialFractionForm}\OperatorTok{[}\FunctionTok{c} \SpecialCharTok{+} \FunctionTok{x}\SpecialCharTok{\^{}}\DecValTok{2}\SpecialCharTok{/}\NormalTok{(}\FunctionTok{x} \SpecialCharTok{{-}} \DecValTok{1}\NormalTok{)}\OperatorTok{,} \FunctionTok{x}\OperatorTok{]}
\end{Highlighting}
\end{Shaded}

\begin{dmath*}\breakingcomma
\text{FCPartialFractionForm}\left(c+x+1,\left(
\begin{array}{cc}
 \{x-1,-1\} & 1 \\
\end{array}
\right),x\right)
\end{dmath*}

\begin{Shaded}
\begin{Highlighting}[]
\NormalTok{FromFCPartialFractionForm}\OperatorTok{[}\NormalTok{ex1}\OperatorTok{]}
\end{Highlighting}
\end{Shaded}

\begin{dmath*}\breakingcomma
c+x+\frac{1}{x-1}+1
\end{dmath*}

\begin{Shaded}
\begin{Highlighting}[]
\NormalTok{ex2 }\ExtensionTok{=}\NormalTok{ ToFCPartialFractionForm}\OperatorTok{[}\NormalTok{(}\SpecialCharTok{{-}}\DecValTok{64}\SpecialCharTok{*}\NormalTok{(}\SpecialCharTok{{-}}\DecValTok{1} \SpecialCharTok{+} \FunctionTok{z}\SpecialCharTok{\^{}}\DecValTok{2}\NormalTok{))}\SpecialCharTok{/}\NormalTok{(}\DecValTok{15}\SpecialCharTok{*}\NormalTok{(}\DecValTok{1} \SpecialCharTok{+} \FunctionTok{z}\SpecialCharTok{\^{}}\DecValTok{2} \SpecialCharTok{+} \FunctionTok{z}\SpecialCharTok{\^{}}\DecValTok{4}\NormalTok{))}\OperatorTok{,} \FunctionTok{z}\OperatorTok{]}
\end{Highlighting}
\end{Shaded}

\begin{dmath*}\breakingcomma
\text{FCPartialFractionForm}\left(0,\left(
\begin{array}{cc}
 \left\{z-\sqrt[3]{-1},-1\right\} & -\frac{32}{15} \\
 \left\{z+\sqrt[3]{-1},-1\right\} & \frac{32}{15} \\
 \left\{z-(-1)^{2/3},-1\right\} & \frac{32}{15} \\
 \left\{z+(-1)^{2/3},-1\right\} & -\frac{32}{15} \\
\end{array}
\right),z\right)
\end{dmath*}

\begin{Shaded}
\begin{Highlighting}[]
\NormalTok{FromFCPartialFractionForm}\OperatorTok{[}\NormalTok{ex2}\OperatorTok{]}
\end{Highlighting}
\end{Shaded}

\begin{dmath*}\breakingcomma
\frac{32}{15 \left(z+\sqrt[3]{-1}\right)}+\frac{32}{15 \left(z-(-1)^{2/3}\right)}-\frac{32}{15 \left(z+(-1)^{2/3}\right)}-\frac{32}{15 \left(z-\sqrt[3]{-1}\right)}
\end{dmath*}

\begin{Shaded}
\begin{Highlighting}[]
\NormalTok{FromFCPartialFractionForm}\OperatorTok{[}\NormalTok{ex2}\OperatorTok{,}\NormalTok{ Factoring }\OtherTok{{-}\textgreater{}} \FunctionTok{Simplify}\OperatorTok{]}
\end{Highlighting}
\end{Shaded}

\begin{dmath*}\breakingcomma
-\frac{64 \left(z^2-1\right)}{15 \left(z^4+z^2+1\right)}
\end{dmath*}
\end{document}
