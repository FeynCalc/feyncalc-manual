% !TeX program = pdflatex
% !TeX root = FCPauliIsolate.tex

\documentclass[../FeynCalcManual.tex]{subfiles}
\begin{document}
\hypertarget{fcpauliisolate}{
\section{FCPauliIsolate}\label{fcpauliisolate}\index{FCPauliIsolate}}

\texttt{FCPauliIsolate[\allowbreak{}exp]} wraps chains of Pauli matrices
into heads specified by the user.

\subsection{See also}

\hyperlink{toc}{Overview}

\subsection{Examples}

\begin{Shaded}
\begin{Highlighting}[]
\NormalTok{FCPauliIsolate}\OperatorTok{[}\FunctionTok{y}\NormalTok{ SI}\OperatorTok{[}\FunctionTok{i}\OperatorTok{]} \SpecialCharTok{+} \FunctionTok{x}\NormalTok{ PauliXi}\OperatorTok{[}\SpecialCharTok{{-}}\FunctionTok{I}\OperatorTok{]}\NormalTok{ . SIS}\OperatorTok{[}\NormalTok{p1}\OperatorTok{]}\NormalTok{ . PauliEta}\OperatorTok{[}\FunctionTok{I}\OperatorTok{]}\NormalTok{ . PauliEta}\OperatorTok{[}\SpecialCharTok{{-}}\FunctionTok{I}\OperatorTok{]}\NormalTok{ . SIS}\OperatorTok{[}\NormalTok{p2}\OperatorTok{]}\NormalTok{ . PauliXi}\OperatorTok{[}\FunctionTok{I}\OperatorTok{],} \FunctionTok{Head} \OtherTok{{-}\textgreater{}}\NormalTok{ pChain}\OperatorTok{]}
\end{Highlighting}
\end{Shaded}

\begin{dmath*}\breakingcomma
y \;\text{pChain}\left(\bar{\sigma }^i\right)+x \;\text{pChain}\left(\xi ^{\dagger }.\left(\bar{\sigma }\cdot \overline{\text{p1}}\right).\eta \right) \;\text{pChain}\left(\eta ^{\dagger }.\left(\bar{\sigma }\cdot \overline{\text{p2}}\right).\xi \right)
\end{dmath*}
\end{document}
