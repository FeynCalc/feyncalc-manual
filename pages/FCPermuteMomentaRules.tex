% !TeX program = pdflatex
% !TeX root = FCPermuteMomentaRules.tex

\documentclass[../FeynCalcManual.tex]{subfiles}
\begin{document}
\hypertarget{fcpermutemomentarules}{%
\section{FCPermuteMomentaRules}\label{fcpermutemomentarules}}

\texttt{FCPermuteMomentaRules[\allowbreak{}\{\allowbreak{}p1,\ \allowbreak{}p2,\ \allowbreak{}...\}]}
returns a set of rules that contain all possible permutations of the
momenta \texttt{p1}, \texttt{p2}, \ldots{} . This can be useful when
working with amplitudes that exhibit a symmetry in some or all of the
final state momenta or when trying to find mappings between loop
integrals from different topologies.

\subsection{See also}

\hyperlink{toc}{Overview},
\hyperlink{fcreplacemomenta}{FCReplaceMomenta}.

\subsection{Examples}

\begin{Shaded}
\begin{Highlighting}[]
\NormalTok{FCPermuteMomentaRules}\OperatorTok{[\{}\NormalTok{p1}\OperatorTok{,}\NormalTok{ p2}\OperatorTok{\}]} 
 
\FunctionTok{f}\OperatorTok{[}\NormalTok{p1}\OperatorTok{,}\NormalTok{ p2}\OperatorTok{]} \OtherTok{/.} \SpecialCharTok{\%}
\end{Highlighting}
\end{Shaded}

\begin{dmath*}\breakingcomma
\{\{\},\{\text{p1}\to \;\text{p2},\text{p2}\to \;\text{p1}\}\}
\end{dmath*}

\begin{dmath*}\breakingcomma
\{f(\text{p1},\text{p2}),f(\text{p2},\text{p1})\}
\end{dmath*}

\begin{Shaded}
\begin{Highlighting}[]
\NormalTok{FCPermuteMomentaRules}\OperatorTok{[\{}\NormalTok{p1}\OperatorTok{,}\NormalTok{ p2}\OperatorTok{,}\NormalTok{ p3}\OperatorTok{\}]} 
 
\FunctionTok{f}\OperatorTok{[}\NormalTok{p1}\OperatorTok{,}\NormalTok{ p2}\OperatorTok{,}\NormalTok{ p3}\OperatorTok{]} \OtherTok{/.} \SpecialCharTok{\%} 
  
 
\end{Highlighting}
\end{Shaded}

\begin{dmath*}\breakingcomma
\{\{\},\{\text{p1}\to \;\text{p2},\text{p2}\to \;\text{p1}\},\{\text{p1}\to \;\text{p3},\text{p3}\to \;\text{p1}\},\{\text{p2}\to \;\text{p3},\text{p3}\to \;\text{p2}\},\{\text{p1}\to \;\text{p2},\text{p2}\to \;\text{p3},\text{p3}\to \;\text{p1}\},\{\text{p1}\to \;\text{p3},\text{p2}\to \;\text{p1},\text{p3}\to \;\text{p2}\}\}
\end{dmath*}

\begin{dmath*}\breakingcomma
\{f(\text{p1},\text{p2},\text{p3}),f(\text{p2},\text{p1},\text{p3}),f(\text{p3},\text{p2},\text{p1}),f(\text{p1},\text{p3},\text{p2}),f(\text{p2},\text{p3},\text{p1}),f(\text{p3},\text{p1},\text{p2})\}
\end{dmath*}
\end{document}
