% !TeX program = pdflatex
% !TeX root = FCProductSplit.tex

\documentclass[../FeynCalcManual.tex]{subfiles}
\begin{document}
\hypertarget{fcproductsplit}{
\section{FCProductSplit}\label{fcproductsplit}\index{FCProductSplit}}

\texttt{FCProductSplit[\allowbreak{}exp,\ \allowbreak{}\{\allowbreak{}v1,\ \allowbreak{}v2,\ \allowbreak{}...\}]}
splits expr into pieces that are free of any occurrence of
\texttt{v1,\ \allowbreak{}v2,\ \allowbreak{}...} and pieces that contain
those variables. This works both on sums and products. The output is
provided in the form of a two element list. One can recover the original
expression by applying \texttt{Total} to that list.

\subsection{See also}

\hyperlink{toc}{Overview}, \hyperlink{fcsplit}{FCSplit}.

\subsection{Examples}

\begin{Shaded}
\begin{Highlighting}[]
\NormalTok{FCProductSplit}\OperatorTok{[}\FunctionTok{c}\SpecialCharTok{\^{}}\DecValTok{2}\OperatorTok{,} \OperatorTok{\{}\FunctionTok{a}\OperatorTok{\}]}
\end{Highlighting}
\end{Shaded}

\begin{dmath*}\breakingcomma
\left\{c^2,1\right\}
\end{dmath*}

\begin{Shaded}
\begin{Highlighting}[]
\NormalTok{FCProductSplit}\OperatorTok{[}\FunctionTok{a}\SpecialCharTok{\^{}}\DecValTok{2}\SpecialCharTok{*}\FunctionTok{b}\OperatorTok{,} \OperatorTok{\{}\FunctionTok{a}\OperatorTok{\}]}
\end{Highlighting}
\end{Shaded}

\begin{dmath*}\breakingcomma
\left\{b,a^2\right\}
\end{dmath*}

\begin{Shaded}
\begin{Highlighting}[]
\NormalTok{FCProductSplit}\OperatorTok{[}\NormalTok{(}\FunctionTok{a}\SpecialCharTok{\^{}}\DecValTok{2} \SpecialCharTok{+} \FunctionTok{b}\NormalTok{)}\SpecialCharTok{*}\FunctionTok{b}\SpecialCharTok{*}\NormalTok{(}\FunctionTok{c} \SpecialCharTok{+} \FunctionTok{d}\NormalTok{)}\OperatorTok{,} \OperatorTok{\{}\FunctionTok{a}\OperatorTok{,} \FunctionTok{c}\OperatorTok{\}]}
\end{Highlighting}
\end{Shaded}

\begin{dmath*}\breakingcomma
\left\{b,\left(a^2+b\right) (c+d)\right\}
\end{dmath*}
\end{document}
