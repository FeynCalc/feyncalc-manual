% !TeX program = pdflatex
% !TeX root = FCScalarProductsSynchronizedQ.tex

\documentclass[../FeynCalcManual.tex]{subfiles}
\begin{document}
\hypertarget{fcscalarproductssynchronizedq}{
\section{FCScalarProductsSynchronizedQ}\label{fcscalarproductssynchronizedq}\index{FCScalarProductsSynchronizedQ}}

\texttt{FCScalarProductsSynchronizedQ[\allowbreak{}]} compares up and
down values of scalar products and other kinematic-related symbols such
as \texttt{Momentum}, \texttt{CartesianMomentum}, \texttt{TC} etc.
between the master kernel and the subkernels and returns \texttt{True}
if all of them identical.

This routine is only relevant in the parallel mode of FeynCalc. It helps
to avoid inconsistencies through definitions that were introduced before
activating the parallel mode and not correctly propagated to the
subkernels

\subsection{See also}

\hyperlink{toc}{Overview}, \hyperlink{scalarproduct}{ScalarProduct},
\hyperlink{pair}{Pair}, \hyperlink{sp}{SP}, \hyperlink{spd}{SPD}.

\subsection{Examples}

\begin{Shaded}
\begin{Highlighting}[]
\NormalTok{FCScalarProductsSynchronizedQ}\OperatorTok{[]}
\end{Highlighting}
\end{Shaded}

\FloatBarrier
\begin{figure}[!ht]
\centering
\includegraphics[width=0.6\linewidth]{img/1baac185a4bv4.pdf}
\end{figure}
\FloatBarrier

\begin{dmath*}\breakingcomma
\text{True}
\end{dmath*}
\end{document}
