% !TeX program = pdflatex
% !TeX root = FCSetDiracGammaScheme.tex

\documentclass[../FeynCalcManual.tex]{subfiles}
\begin{document}
\hypertarget{fcsetdiracgammascheme}{
\section{FCSetDiracGammaScheme}\label{fcsetdiracgammascheme}\index{FCSetDiracGammaScheme}}

\texttt{FCSetDiracGammaScheme[\allowbreak{}scheme]} allows you to
specify how Dirac matrices will be handled in \texttt{D} dimensions.
This is mainly relevant to the treatment of the 5th Dirac matrix
\(\gamma^5\), which is not well-defined in dimensional regularization.

Following schemes are supported:

``NDR'' - This is the default value. In the naive dimensional
regularization (also known as conventional dimensional regularization or
CDR) \(\gamma^5\) is assumed to anticommute with all Dirac matrices in
\(D\) dimensions. Hence, every Dirac trace can be rewritten in such a
way, that it contains either just one or not a single \(\gamma^5\)
matrix. The latter traces are obviously unambiguous. The traces with one
\(\gamma^5\) are not well-defined in this scheme. It usually depends on
the physics of the process, whether and how they can contribute to the
final result. Therefore, FeynCalc will keep such traces unevaluated,
leaving it to the user to decide how to treat them. Notice that traces
with an odd number of the usual Dirac matrices and one \(\gamma^5\),
that vanish in \(4\) dimensions, will be also put to zero in this
scheme.

``NDR-Discard'' - This is a special version of the NDR scheme. The Dirac
algebra is evaluated in the same way as with ``NDR'', but the remaining
traces with one \(\gamma^5\) are put to zero. This assumes that such
traces do not contribute to the final result, which is obviously true
only for specific calculations.

``BMHV'' - The Breitenlohner-Maison extension of the t'Hooft-Veltman
scheme. This scheme introduces Dirac and Lorentz tensors living in
\(4\), \(D\) or \(D-4\) dimensions, while \(\gamma^5\) is a purely
\(4\)-dimensional object. BMHV is algebraically consistent but often
suffers from nonconservation of currents in the final results. The
conservation must be then enforced by introducing finite counter-terms.
The counter-terms are to be supplied by the user, since FeynCalc does
not do this automatically.

``Larin'' - Special prescription developed by S. Larin, also known as
the Larin-Gorishny-Atkyampo-DelBurgo scheme. Essentially, it is a
shortcut (mostly used in QCD) for obtaining the same results as in BMHV
but without the necessity to deal with tensors from different
dimensions. In this scheme \(\gamma^5\) is treated as nonanticommuting,
while Dirac traces are still cyclic. If a chain of Dirac matrices
contains a single \(\gamma^5\), it is essentially left untouched. When
computing the trace of such a chain, the cyclicity is used to put
\(\gamma^5\) to the very end of the chain. Then, the trace is evaluated
using the Moch-Vermaseren-Vogt formula, Eq.(10) from
\href{https://arxiv.org/pdf/1506.04517.pdf}{arXiv:1506.04517}. If a
chain contains more than one \(\gamma^5\), all but one \(\gamma^5\) will
be eliminated using the replacement
\(\gamma_\mu \gamma^5 \to i/6 \varepsilon_{\mu \nu \rho \sigma} \gamma^\nu \gamma^\rho \gamma^\sigma\).
This way every trace with multiple occurrences of \(\gamma^5\) can be
converted to a linear combination of traces with a single \(\gamma^5\).
Such traces are then treated as described above. Notice that Levi-Civita
tensors generated during the calculation of traces are
\(D\)-dimensional. For example, a product of two such tensors with all
their indices contracted yields a polynomial in \(D\)'s. This scheme is
often used for performance reasons and is assumed to give the same
results as the BMHV scheme. However, this is not a rigorous statement
and so when in doubt it might be better to use BMHV instead.

\subsection{See also}

\hyperlink{toc}{Overview},
\hyperlink{fcgetdiracgammascheme}{FCGetDiracGammaScheme},
\hyperlink{diractrace}{DiracTrace}.

\subsection{Examples}

In NDR chiral traces remain unevaluated. You decide how to treat them.

\begin{Shaded}
\begin{Highlighting}[]
\NormalTok{FCSetDiracGammaScheme}\OperatorTok{[}\StringTok{"NDR"}\OperatorTok{]} 
 
\NormalTok{DiracTrace}\OperatorTok{[}\NormalTok{GAD}\OperatorTok{[}\SpecialCharTok{\textbackslash{}}\OperatorTok{[}\NormalTok{Mu}\OperatorTok{],} \SpecialCharTok{\textbackslash{}}\OperatorTok{[}\NormalTok{Nu}\OperatorTok{],} \SpecialCharTok{\textbackslash{}}\OperatorTok{[}\NormalTok{Rho}\OperatorTok{],} \SpecialCharTok{\textbackslash{}}\OperatorTok{[}\NormalTok{Sigma}\OperatorTok{],} \SpecialCharTok{\textbackslash{}}\OperatorTok{[}\NormalTok{Tau}\OperatorTok{],} \SpecialCharTok{\textbackslash{}}\OperatorTok{[}\NormalTok{Kappa}\OperatorTok{],} \DecValTok{5}\OperatorTok{]]} 
 
\NormalTok{DiracSimplify}\OperatorTok{[}\SpecialCharTok{\%}\OperatorTok{]}
\end{Highlighting}
\end{Shaded}

\begin{dmath*}\breakingcomma
\text{NDR}
\end{dmath*}

\begin{dmath*}\breakingcomma
\text{tr}\left(\gamma ^{\mu }.\gamma ^{\nu }.\gamma ^{\rho }.\gamma ^{\sigma }.\gamma ^{\tau }.\gamma ^{\kappa }.\bar{\gamma }^5\right)
\end{dmath*}

\begin{dmath*}\breakingcomma
\text{tr}\left(\gamma ^{\mu }.\gamma ^{\nu }.\gamma ^{\rho }.\gamma ^{\sigma }.\gamma ^{\tau }.\gamma ^{\kappa }.\bar{\gamma }^5\right)
\end{dmath*}

If you know that such traces do not contribute, use NDR-Discard scheme
to put them to zero

\begin{Shaded}
\begin{Highlighting}[]
\NormalTok{FCSetDiracGammaScheme}\OperatorTok{[}\StringTok{"NDR{-}Discard"}\OperatorTok{]} 
 
\NormalTok{DiracSimplify}\OperatorTok{[}\NormalTok{DiracTrace}\OperatorTok{[}\NormalTok{GAD}\OperatorTok{[}\SpecialCharTok{\textbackslash{}}\OperatorTok{[}\NormalTok{Mu}\OperatorTok{],} \SpecialCharTok{\textbackslash{}}\OperatorTok{[}\NormalTok{Nu}\OperatorTok{],} \SpecialCharTok{\textbackslash{}}\OperatorTok{[}\NormalTok{Rho}\OperatorTok{],} \SpecialCharTok{\textbackslash{}}\OperatorTok{[}\NormalTok{Sigma}\OperatorTok{],} \SpecialCharTok{\textbackslash{}}\OperatorTok{[}\NormalTok{Tau}\OperatorTok{],} \SpecialCharTok{\textbackslash{}}\OperatorTok{[}\NormalTok{Kappa}\OperatorTok{],} \DecValTok{5}\OperatorTok{]]]}
\end{Highlighting}
\end{Shaded}

\begin{dmath*}\breakingcomma
\text{NDR-Discard}
\end{dmath*}

\begin{dmath*}\breakingcomma
0
\end{dmath*}

In BMHV chiral traces are algebraically well-defined

\begin{Shaded}
\begin{Highlighting}[]
\NormalTok{FCSetDiracGammaScheme}\OperatorTok{[}\StringTok{"BMHV"}\OperatorTok{]} 
 
\NormalTok{res1 }\ExtensionTok{=}\NormalTok{ DiracSimplify}\OperatorTok{[}\NormalTok{DiracTrace}\OperatorTok{[}\NormalTok{GAD}\OperatorTok{[}\SpecialCharTok{\textbackslash{}}\OperatorTok{[}\NormalTok{Mu}\OperatorTok{],} \SpecialCharTok{\textbackslash{}}\OperatorTok{[}\NormalTok{Nu}\OperatorTok{],} \SpecialCharTok{\textbackslash{}}\OperatorTok{[}\NormalTok{Rho}\OperatorTok{],} \SpecialCharTok{\textbackslash{}}\OperatorTok{[}\NormalTok{Sigma}\OperatorTok{],} \SpecialCharTok{\textbackslash{}}\OperatorTok{[}\NormalTok{Tau}\OperatorTok{],} \SpecialCharTok{\textbackslash{}}\OperatorTok{[}\NormalTok{Kappa}\OperatorTok{],} \DecValTok{5}\OperatorTok{]]]}
\end{Highlighting}
\end{Shaded}

\begin{dmath*}\breakingcomma
\text{BMHV}
\end{dmath*}

\begin{dmath*}\breakingcomma
-4 i g^{\kappa \mu } \bar{\epsilon }^{\nu \rho \sigma \tau }+4 i g^{\kappa \nu } \bar{\epsilon }^{\mu \rho \sigma \tau }-4 i g^{\kappa \rho } \bar{\epsilon }^{\mu \nu \sigma \tau }+4 i g^{\kappa \sigma } \bar{\epsilon }^{\mu \nu \rho \tau }-4 i g^{\kappa \tau } \bar{\epsilon }^{\mu \nu \rho \sigma }+4 i g^{\mu \nu } \bar{\epsilon }^{\kappa \rho \sigma \tau }-4 i g^{\mu \rho } \bar{\epsilon }^{\kappa \nu \sigma \tau }+4 i g^{\mu \sigma } \bar{\epsilon }^{\kappa \nu \rho \tau }-4 i g^{\mu \tau } \bar{\epsilon }^{\kappa \nu \rho \sigma }+4 i g^{\nu \rho } \bar{\epsilon }^{\kappa \mu \sigma \tau }-4 i g^{\nu \sigma } \bar{\epsilon }^{\kappa \mu \rho \tau }+4 i g^{\nu \tau } \bar{\epsilon }^{\kappa \mu \rho \sigma }+4 i g^{\rho \sigma } \bar{\epsilon }^{\kappa \mu \nu \tau }-4 i g^{\rho \tau } \bar{\epsilon }^{\kappa \mu \nu \sigma }+4 i g^{\sigma \tau } \bar{\epsilon }^{\kappa \mu \nu \rho }
\end{dmath*}

Larin's scheme reproduces the results of the BMHV scheme, but this may
not be immediately obvious

\begin{Shaded}
\begin{Highlighting}[]
\NormalTok{FCSetDiracGammaScheme}\OperatorTok{[}\StringTok{"Larin"}\OperatorTok{]} 
 
\NormalTok{res2 }\ExtensionTok{=}\NormalTok{ DiracSimplify}\OperatorTok{[}\NormalTok{DiracTrace}\OperatorTok{[}\NormalTok{GAD}\OperatorTok{[}\SpecialCharTok{\textbackslash{}}\OperatorTok{[}\NormalTok{Mu}\OperatorTok{],} \SpecialCharTok{\textbackslash{}}\OperatorTok{[}\NormalTok{Nu}\OperatorTok{],} \SpecialCharTok{\textbackslash{}}\OperatorTok{[}\NormalTok{Rho}\OperatorTok{],} \SpecialCharTok{\textbackslash{}}\OperatorTok{[}\NormalTok{Sigma}\OperatorTok{],} \SpecialCharTok{\textbackslash{}}\OperatorTok{[}\NormalTok{Tau}\OperatorTok{],} \SpecialCharTok{\textbackslash{}}\OperatorTok{[}\NormalTok{Kappa}\OperatorTok{],} \DecValTok{5}\OperatorTok{]]]}
\end{Highlighting}
\end{Shaded}

\begin{dmath*}\breakingcomma
\text{Larin}
\end{dmath*}

\begin{dmath*}\breakingcomma
4 i g^{\mu \nu } \overset{\text{}}{\epsilon }^{\kappa \rho \sigma \tau }-4 i g^{\mu \rho } \overset{\text{}}{\epsilon }^{\kappa \nu \sigma \tau }+4 i g^{\mu \sigma } \overset{\text{}}{\epsilon }^{\kappa \nu \rho \tau }-4 i g^{\mu \tau } \overset{\text{}}{\epsilon }^{\kappa \nu \rho \sigma }+4 i g^{\nu \rho } \overset{\text{}}{\epsilon }^{\kappa \mu \sigma \tau }-4 i g^{\nu \sigma } \overset{\text{}}{\epsilon }^{\kappa \mu \rho \tau }+4 i g^{\nu \tau } \overset{\text{}}{\epsilon }^{\kappa \mu \rho \sigma }+4 i g^{\rho \sigma } \overset{\text{}}{\epsilon }^{\kappa \mu \nu \tau }-4 i g^{\rho \tau } \overset{\text{}}{\epsilon }^{\kappa \mu \nu \sigma }+4 i g^{\sigma \tau } \overset{\text{}}{\epsilon }^{\kappa \mu \nu \rho }
\end{dmath*}

Owing to Schouten identities, proving the equivalence of chiral traces
is not so simple, especially for many terms.
\texttt{FCSchoutenBruteForce} can be helpful here

\begin{Shaded}
\begin{Highlighting}[]
\NormalTok{diff }\ExtensionTok{=}\NormalTok{ ChangeDimension}\OperatorTok{[}\NormalTok{res1 }\SpecialCharTok{{-}}\NormalTok{ res2}\OperatorTok{,} \FunctionTok{D}\OperatorTok{]} 
 
\NormalTok{Contract}\OperatorTok{[}\NormalTok{FV}\OperatorTok{[}\NormalTok{p1}\OperatorTok{,} \SpecialCharTok{\textbackslash{}}\OperatorTok{[}\NormalTok{Mu}\OperatorTok{]]}\NormalTok{ FV}\OperatorTok{[}\NormalTok{p2}\OperatorTok{,} \SpecialCharTok{\textbackslash{}}\OperatorTok{[}\NormalTok{Nu}\OperatorTok{]]}\NormalTok{ FV}\OperatorTok{[}\NormalTok{p3}\OperatorTok{,} \SpecialCharTok{\textbackslash{}}\OperatorTok{[}\NormalTok{Rho}\OperatorTok{]]}\NormalTok{ FV}\OperatorTok{[}\NormalTok{p4}\OperatorTok{,} \SpecialCharTok{\textbackslash{}}\OperatorTok{[}\NormalTok{Sigma}\OperatorTok{]]}\NormalTok{ FV}\OperatorTok{[}\NormalTok{p5}\OperatorTok{,} \SpecialCharTok{\textbackslash{}}\OperatorTok{[}\NormalTok{Tau}\OperatorTok{]]}\NormalTok{ FV}\OperatorTok{[}\NormalTok{p6}\OperatorTok{,} \SpecialCharTok{\textbackslash{}}\OperatorTok{[}\NormalTok{Kappa}\OperatorTok{]]}\NormalTok{ diff}\OperatorTok{]} 
 
\NormalTok{FCSchoutenBruteForce}\OperatorTok{[}\SpecialCharTok{\%}\OperatorTok{,} \OperatorTok{\{\},} \OperatorTok{\{\}]}
\end{Highlighting}
\end{Shaded}

\begin{dmath*}\breakingcomma
-4 i g^{\kappa \mu } \overset{\text{}}{\epsilon }^{\nu \rho \sigma \tau }+4 i g^{\kappa \nu } \overset{\text{}}{\epsilon }^{\mu \rho \sigma \tau }-4 i g^{\kappa \rho } \overset{\text{}}{\epsilon }^{\mu \nu \sigma \tau }+4 i g^{\kappa \sigma } \overset{\text{}}{\epsilon }^{\mu \nu \rho \tau }-4 i g^{\kappa \tau } \overset{\text{}}{\epsilon }^{\mu \nu \rho \sigma }
\end{dmath*}

\begin{dmath*}\breakingcomma
-4 i \left(\overline{\text{p1}}\cdot \overline{\text{p6}}\right) \bar{\epsilon }^{\overline{\text{p2}}\;\overline{\text{p3}}\;\overline{\text{p4}}\;\overline{\text{p5}}}+4 i \left(\overline{\text{p2}}\cdot \overline{\text{p6}}\right) \bar{\epsilon }^{\overline{\text{p1}}\;\overline{\text{p3}}\;\overline{\text{p4}}\;\overline{\text{p5}}}-4 i \left(\overline{\text{p3}}\cdot \overline{\text{p6}}\right) \bar{\epsilon }^{\overline{\text{p1}}\;\overline{\text{p2}}\;\overline{\text{p4}}\;\overline{\text{p5}}}+4 i \left(\overline{\text{p4}}\cdot \overline{\text{p6}}\right) \bar{\epsilon }^{\overline{\text{p1}}\;\overline{\text{p2}}\;\overline{\text{p3}}\;\overline{\text{p5}}}-4 i \left(\overline{\text{p5}}\cdot \overline{\text{p6}}\right) \bar{\epsilon }^{\overline{\text{p1}}\;\overline{\text{p2}}\;\overline{\text{p3}}\;\overline{\text{p4}}}
\end{dmath*}

\begin{dmath*}\breakingcomma
\text{FCSchoutenBruteForce: The following rule was applied: }\bar{\epsilon }^{\overline{\text{p2}}\;\overline{\text{p3}}\;\overline{\text{p4}}\;\overline{\text{p5}}} \left(\overline{\text{p1}}\cdot \overline{\text{p6}}\right):\to \bar{\epsilon }^{\overline{\text{p1}}\;\overline{\text{p3}}\;\overline{\text{p4}}\;\overline{\text{p5}}} \left(\overline{\text{p2}}\cdot \overline{\text{p6}}\right)-\bar{\epsilon }^{\overline{\text{p1}}\;\overline{\text{p2}}\;\overline{\text{p4}}\;\overline{\text{p5}}} \left(\overline{\text{p3}}\cdot \overline{\text{p6}}\right)+\bar{\epsilon }^{\overline{\text{p1}}\;\overline{\text{p2}}\;\overline{\text{p3}}\;\overline{\text{p5}}} \left(\overline{\text{p4}}\cdot \overline{\text{p6}}\right)-\bar{\epsilon }^{\overline{\text{p1}}\;\overline{\text{p2}}\;\overline{\text{p3}}\;\overline{\text{p4}}} \left(\overline{\text{p5}}\cdot \overline{\text{p6}}\right)
\end{dmath*}

\begin{dmath*}\breakingcomma
\text{FCSchoutenBruteForce: The numbers of terms in the expression decreased by: }5
\end{dmath*}

\begin{dmath*}\breakingcomma
\text{FCSchoutenBruteForce: Current length of the expression: }0
\end{dmath*}

\begin{dmath*}\breakingcomma
0
\end{dmath*}

\begin{Shaded}
\begin{Highlighting}[]
\NormalTok{FCSetDiracGammaScheme}\OperatorTok{[}\StringTok{"NDR"}\OperatorTok{]}
\end{Highlighting}
\end{Shaded}

\begin{dmath*}\breakingcomma
\text{NDR}
\end{dmath*}
\end{document}
