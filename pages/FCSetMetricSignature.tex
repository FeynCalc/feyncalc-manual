% !TeX program = pdflatex
% !TeX root = FCSetMetricSignature.tex

\documentclass[../FeynCalcManual.tex]{subfiles}
\begin{document}
\hypertarget{fcsetmetricsignature}{%
\section{FCSetMetricSignature}\label{fcsetmetricsignature}}

\texttt{FCSetMetricSignature} sets the signature of the Minkowski metric
used when working with Cartesian objects, like \texttt{CartesianPair},
\texttt{CartesianIndex}, \texttt{CartesianMomentum} etc.

The default choice is \((1,-1,-1,-1)\) which corresponds to
\texttt{FCSetMetricSignature[\allowbreak{}\{\allowbreak{}1,\ \allowbreak{}-1\}]}.

\subsection{See also}

\hyperlink{toc}{Overview},
\hyperlink{fcgetmetricsignature}{FCGetMetricSignature}.

\subsection{Examples}

\begin{Shaded}
\begin{Highlighting}[]
\NormalTok{FCSetMetricSignature}\OperatorTok{[\{}\SpecialCharTok{{-}}\DecValTok{1}\OperatorTok{,} \DecValTok{1}\OperatorTok{\}]} 
 
\NormalTok{SPD}\OperatorTok{[}\FunctionTok{p}\OperatorTok{,} \FunctionTok{q}\OperatorTok{]} \SpecialCharTok{//}\NormalTok{ LorentzToCartesian}
\end{Highlighting}
\end{Shaded}

\begin{dmath*}\breakingcomma
p\cdot q-p^0 q^0
\end{dmath*}

\begin{Shaded}
\begin{Highlighting}[]
\NormalTok{FCSetMetricSignature}\OperatorTok{[\{}\DecValTok{1}\OperatorTok{,} \SpecialCharTok{{-}}\DecValTok{1}\OperatorTok{\}]} 
 
\NormalTok{SPD}\OperatorTok{[}\FunctionTok{p}\OperatorTok{,} \FunctionTok{q}\OperatorTok{]} \SpecialCharTok{//}\NormalTok{ LorentzToCartesian}
\end{Highlighting}
\end{Shaded}

\begin{dmath*}\breakingcomma
p^0 q^0-p\cdot q
\end{dmath*}
\end{document}
