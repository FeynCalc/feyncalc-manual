% !TeX program = pdflatex
% !TeX root = FCSetPauliSigmaScheme.tex

\documentclass[../FeynCalcManual.tex]{subfiles}
\begin{document}
\hypertarget{fcsetpaulisigmascheme}{%
\section{FCSetPauliSigmaScheme}\label{fcsetpaulisigmascheme}}

\texttt{FCSetPauliSigmaScheme[\allowbreak{}scheme]} allows you to
specify how Pauli matrices will be handled in \(D-1\) dimensions.

This is mainly related to the commutator of two Pauli matrices, which
involves a Levi-Civita tensor. The latter is not a well-defined quantity
in \(D-1\) dimensions. Following schemes are supported:

\begin{itemize}
\item
  \texttt{"None"} - This is the default value. The anticommutator
  relation is not applied to \(D-1\) dimensional Pauli matrices.
\item
  \texttt{"Naive"} - Naively apply the commutator relation in
  \(D-1\)-dimensions,
  i.e.~\(\{\sigma^i, \sigma^j \} = 2 i \varepsilon^{ijk} \sigma^k\). The
  Levi-Civita tensor lives in \(D-1\)-dimensions, so that a contraction
  of two such tensors which have all indices in common yields
  \((D-3) (D-2) (D-1)\).
\end{itemize}

\subsection{See also}

\hyperlink{toc}{Overview}, \hyperlink{paulisigma}{PauliSigma},
\hyperlink{fcgetpaulisigmascheme}{FCGetPauliSigmaScheme}.

\subsection{Examples}

\begin{Shaded}
\begin{Highlighting}[]
\NormalTok{FCGetPauliSigmaScheme}\OperatorTok{[]}
\end{Highlighting}
\end{Shaded}

\begin{dmath*}\breakingcomma
\text{None}
\end{dmath*}

\begin{Shaded}
\begin{Highlighting}[]
\NormalTok{CSID}\OperatorTok{[}\FunctionTok{i}\OperatorTok{,} \FunctionTok{j}\OperatorTok{,} \FunctionTok{k}\OperatorTok{]} 
 
\NormalTok{PauliSimplify}\OperatorTok{[}\SpecialCharTok{\%}\OperatorTok{,}\NormalTok{ PauliReduce }\OtherTok{{-}\textgreater{}} \ConstantTok{True}\OperatorTok{]}
\end{Highlighting}
\end{Shaded}

\begin{dmath*}\breakingcomma
\sigma ^i.\sigma ^j.\sigma ^k
\end{dmath*}

\begin{dmath*}\breakingcomma
\sigma ^i.\sigma ^j.\sigma ^k
\end{dmath*}

\begin{Shaded}
\begin{Highlighting}[]
\NormalTok{FCSetPauliSigmaScheme}\OperatorTok{[}\StringTok{"Naive"}\OperatorTok{]}\NormalTok{;}
\end{Highlighting}
\end{Shaded}

\begin{Shaded}
\begin{Highlighting}[]
\NormalTok{FCGetPauliSigmaScheme}\OperatorTok{[]}
\end{Highlighting}
\end{Shaded}

\begin{dmath*}\breakingcomma
\text{Naive}
\end{dmath*}

\begin{Shaded}
\begin{Highlighting}[]
\NormalTok{ex }\ExtensionTok{=}\NormalTok{ PauliSimplify}\OperatorTok{[}\NormalTok{CSID}\OperatorTok{[}\FunctionTok{i}\OperatorTok{,} \FunctionTok{j}\OperatorTok{,} \FunctionTok{k}\OperatorTok{],}\NormalTok{ PauliReduce }\OtherTok{{-}\textgreater{}} \ConstantTok{True}\OperatorTok{]}
\end{Highlighting}
\end{Shaded}

\begin{dmath*}\breakingcomma
i \overset{\text{}}{\epsilon }^{ijk}+D \sigma ^i \delta ^{jk}-D \sigma ^j \delta ^{ik}-3 \sigma ^i \delta ^{jk}+3 \sigma ^j \delta ^{ik}+\sigma ^k \delta ^{ij}
\end{dmath*}

\begin{Shaded}
\begin{Highlighting}[]
\NormalTok{ex }\SpecialCharTok{//}\NormalTok{ FCE }\SpecialCharTok{//} \FunctionTok{StandardForm}

\CommentTok{(*I CLCD[i, j, k] + CSID[k] KDD[i, j] + 3 CSID[j] KDD[i, k] {-} D CSID[j] KDD[i, k] {-} 3 CSID[i] KDD[j, k] + D CSID[i] KDD[j, k]*)}
\end{Highlighting}
\end{Shaded}

\begin{Shaded}
\begin{Highlighting}[]
\NormalTok{FCSetPauliSigmaScheme}\OperatorTok{[}\StringTok{"None"}\OperatorTok{]}\NormalTok{;}
\end{Highlighting}
\end{Shaded}

\end{document}
