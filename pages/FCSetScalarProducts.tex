% !TeX program = pdflatex
% !TeX root = FCSetScalarProducts.tex

\documentclass[../FeynCalcManual.tex]{subfiles}
\begin{document}
\hypertarget{fcsetscalarproducts}{%
\section{FCSetScalarProducts}\label{fcsetscalarproducts}}

\texttt{FCSetScalarProducts[\allowbreak{}]} assigns values in the second
list to scalar products (or other kinematic-related symbols such as
\texttt{Momentum}, \texttt{CartesianMomentum}, \texttt{TC} etc.) in the
first list.

The values can be also modified if the quantities in the first list are
entered by hand. To modify the definitions programmatically without
resorting to \texttt{With} and similar delayed evaluation tricks one can
use placeholders in conjunction with the \texttt{InitialSubstitutions}
option.

\subsection{See also}

\hyperlink{toc}{Overview}, \hyperlink{scalarproduct}{ScalarProduct}.

\subsection{Examples}

\begin{Shaded}
\begin{Highlighting}[]
\NormalTok{FCClearScalarProducts}\OperatorTok{[]}\NormalTok{; }
 
\NormalTok{FCSetScalarProducts}\OperatorTok{[\{}\NormalTok{SPD}\OperatorTok{[}\NormalTok{p1}\OperatorTok{],}\NormalTok{ SPD}\OperatorTok{[}\NormalTok{p2}\OperatorTok{],}\NormalTok{ SPD}\OperatorTok{[}\NormalTok{p3}\OperatorTok{,}\NormalTok{ p4}\OperatorTok{]\},} \OperatorTok{\{}\DecValTok{0}\OperatorTok{,}\NormalTok{ xx1}\OperatorTok{,}\NormalTok{ xx2}\OperatorTok{\}]}\NormalTok{;}
\end{Highlighting}
\end{Shaded}

\begin{Shaded}
\begin{Highlighting}[]
\OperatorTok{\{}\NormalTok{SPD}\OperatorTok{[}\NormalTok{p1}\OperatorTok{],}\NormalTok{ SPD}\OperatorTok{[}\NormalTok{p2}\OperatorTok{],}\NormalTok{ SPD}\OperatorTok{[}\NormalTok{p3}\OperatorTok{,}\NormalTok{ p4}\OperatorTok{]\}}
\end{Highlighting}
\end{Shaded}

\begin{dmath*}\breakingcomma
\{0,\text{xx1},\text{xx2}\}
\end{dmath*}

\begin{Shaded}
\begin{Highlighting}[]
\NormalTok{FCSetScalarProducts}\OperatorTok{[\{}\NormalTok{spd}\OperatorTok{[}\NormalTok{p1}\OperatorTok{]\},} \OperatorTok{\{}\NormalTok{val}\OperatorTok{\},}\NormalTok{ InitialSubstitutions }\OtherTok{{-}\textgreater{}} \OperatorTok{\{}\NormalTok{spd }\OtherTok{{-}\textgreater{}}\NormalTok{ SPD}\OperatorTok{\}]}
\end{Highlighting}
\end{Shaded}

\begin{dmath*}\breakingcomma
\{\text{val}\}
\end{dmath*}
\end{document}
