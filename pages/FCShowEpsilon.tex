% !TeX program = pdflatex
% !TeX root = FCShowEpsilon.tex

\documentclass[../FeynCalcManual.tex]{subfiles}
\begin{document}
\hypertarget{fcshowepsilon}{
\section{FCShowEpsilon}\label{fcshowepsilon}\index{FCShowEpsilon}}

\texttt{FCShowEpsilon[\allowbreak{}expr]} substitutes
\texttt{SMP[\allowbreak{}"Delta"]} with
\texttt{1/Epsilon - EulerGamma + Log[\allowbreak{}4 Pi]}.

\subsection{See also}

\hyperlink{toc}{Overview}, \hyperlink{fchideepsilon}{FCHideEpsilon}.

\subsection{Examples}

\begin{Shaded}
\begin{Highlighting}[]
\NormalTok{SMP}\OperatorTok{[}\StringTok{"Delta"}\OperatorTok{]} 
 
\NormalTok{FCShowEpsilon}\OperatorTok{[}\SpecialCharTok{\%}\OperatorTok{]}
\end{Highlighting}
\end{Shaded}

\begin{dmath*}\breakingcomma
\Delta
\end{dmath*}

\begin{dmath*}\breakingcomma
\frac{1}{\varepsilon }-\gamma +\log (4 \pi )
\end{dmath*}

\begin{Shaded}
\begin{Highlighting}[]
\NormalTok{SMP}\OperatorTok{[}\StringTok{"Delta\_UV"}\OperatorTok{]} 
 
\NormalTok{FCShowEpsilon}\OperatorTok{[}\SpecialCharTok{\%}\OperatorTok{]}
\end{Highlighting}
\end{Shaded}

\begin{dmath*}\breakingcomma
\Delta _{\text{UV}}
\end{dmath*}

\begin{dmath*}\breakingcomma
\frac{1}{\varepsilon _{\text{UV}}}-\gamma +\log (4 \pi )
\end{dmath*}

\begin{Shaded}
\begin{Highlighting}[]
\NormalTok{SMP}\OperatorTok{[}\StringTok{"Delta\_IR"}\OperatorTok{]} 
 
\NormalTok{FCShowEpsilon}\OperatorTok{[}\SpecialCharTok{\%}\OperatorTok{]}
\end{Highlighting}
\end{Shaded}

\begin{dmath*}\breakingcomma
\Delta _{\text{IR}}
\end{dmath*}

\begin{dmath*}\breakingcomma
\frac{1}{\varepsilon _{\text{IR}}}-\gamma +\log (4 \pi )
\end{dmath*}
\end{document}
