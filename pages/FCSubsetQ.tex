% !TeX program = pdflatex
% !TeX root = FCSubsetQ.tex

\documentclass[../FeynCalcManual.tex]{subfiles}
\begin{document}
\hypertarget{fcsubsetq}{
\section{FCSubsetQ}\label{fcsubsetq}\index{FCSubsetQ}}

\texttt{FCSubsetQ[\allowbreak{}list1,\ \allowbreak{}list2]} yields
\texttt{True} if \texttt{list2} is a subset of \texttt{list1} and
\texttt{False} otherwise. It returns the same results as the standard
\texttt{SubsetQ}. The only reason for introducing \texttt{FCSubsetQ} is
that \texttt{SubsetQ} is not available in Mathematica 8 and 9, which are
still supported by FeynCalc.

\subsection{See also}

\hyperlink{toc}{Overview},
\hyperlink{fcduplicatefreeq}{FCDuplicateFreeQ}.

\subsection{Examples}

\begin{Shaded}
\begin{Highlighting}[]
\NormalTok{FCSubsetQ}\OperatorTok{[\{}\FunctionTok{a}\OperatorTok{,} \FunctionTok{b}\OperatorTok{,} \FunctionTok{c}\OperatorTok{,} \FunctionTok{d}\OperatorTok{\},} \OperatorTok{\{}\FunctionTok{a}\OperatorTok{,} \FunctionTok{d}\OperatorTok{,} \FunctionTok{e}\OperatorTok{\}]}
\end{Highlighting}
\end{Shaded}

\begin{dmath*}\breakingcomma
\text{False}
\end{dmath*}

\begin{Shaded}
\begin{Highlighting}[]
\NormalTok{FCSubsetQ}\OperatorTok{[\{}\FunctionTok{a}\OperatorTok{,} \FunctionTok{b}\OperatorTok{,} \FunctionTok{c}\OperatorTok{,} \FunctionTok{d}\OperatorTok{\},} \OperatorTok{\{}\FunctionTok{a}\OperatorTok{,} \FunctionTok{d}\OperatorTok{\}]}
\end{Highlighting}
\end{Shaded}

\begin{dmath*}\breakingcomma
\text{True}
\end{dmath*}
\end{document}
