% !TeX program = pdflatex
% !TeX root = FCSymmetrize.tex

\documentclass[../FeynCalcManual.tex]{subfiles}
\begin{document}
\hypertarget{fcsymmetrize}{%
\section{FCSymmetrize}\label{fcsymmetrize}}

\texttt{FCSymmetrize[\allowbreak{}expr,\ \allowbreak{}\{\allowbreak{}a1,\ \allowbreak{}a2,\ \allowbreak{}...\}]}
symmetrizes expr with respect to the variables
\texttt{a1,\ \allowbreak{}a2,\ \allowbreak{}...}.

\subsection{See also}

\hyperlink{toc}{Overview},
\hyperlink{fcantisymmetrize}{FCAntiSymmetrize}.

\subsection{Examples}

\begin{Shaded}
\begin{Highlighting}[]
\NormalTok{FCSymmetrize}\OperatorTok{[}\FunctionTok{f}\OperatorTok{[}\FunctionTok{a}\OperatorTok{,} \FunctionTok{b}\OperatorTok{],} \OperatorTok{\{}\FunctionTok{a}\OperatorTok{,} \FunctionTok{b}\OperatorTok{\}]}
\end{Highlighting}
\end{Shaded}

\begin{dmath*}\breakingcomma
\frac{1}{2} (f(a,b)+f(b,a))
\end{dmath*}

\begin{Shaded}
\begin{Highlighting}[]
\NormalTok{FCSymmetrize}\OperatorTok{[}\FunctionTok{f}\OperatorTok{[}\FunctionTok{x}\OperatorTok{,} \FunctionTok{y}\OperatorTok{,} \FunctionTok{z}\OperatorTok{],} \OperatorTok{\{}\FunctionTok{x}\OperatorTok{,} \FunctionTok{y}\OperatorTok{,} \FunctionTok{z}\OperatorTok{\}]}
\end{Highlighting}
\end{Shaded}

\begin{dmath*}\breakingcomma
\frac{1}{6} (f(x,y,z)+f(x,z,y)+f(y,x,z)+f(y,z,x)+f(z,x,y)+f(z,y,x))
\end{dmath*}
\end{document}
