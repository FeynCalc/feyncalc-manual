% !TeX program = pdflatex
% !TeX root = FCTopology.tex

\documentclass[../FeynCalcManual.tex]{subfiles}
\begin{document}
\hypertarget{fctopology}{%
\section{FCTopology}\label{fctopology}}

\texttt{FCTopology[\allowbreak{}id,\ \allowbreak{}\{\allowbreak{}prop1,\ \allowbreak{}prop2,\ \allowbreak{}...\}]}
denotes a topology with the identifier id that is characterized by the
propagators
\texttt{\{\allowbreak{}prop1,\ \allowbreak{}prop2,\ \allowbreak{}...\}}.
The propagators in the list do not necessarily have to form a valid
basis, i.e.~the basis may also be incomplete or overdetermined.

\subsection{See also}

\hyperlink{toc}{Overview}.

\subsection{Examples}

A 2-loop topology with one external momentum \texttt{Q}

\begin{Shaded}
\begin{Highlighting}[]
\NormalTok{FCTopology}\OperatorTok{[}\NormalTok{topo1}\OperatorTok{,} \OperatorTok{\{}\NormalTok{SFAD}\OperatorTok{[}\NormalTok{p1}\OperatorTok{],}\NormalTok{ SFAD}\OperatorTok{[}\NormalTok{p2}\OperatorTok{],}\NormalTok{ SFAD}\OperatorTok{[}\FunctionTok{Q} \SpecialCharTok{{-}}\NormalTok{ p1 }\SpecialCharTok{{-}}\NormalTok{ p2}\OperatorTok{],}\NormalTok{ SFAD}\OperatorTok{[}\FunctionTok{Q} \SpecialCharTok{{-}}\NormalTok{ p2}\OperatorTok{],}\NormalTok{ SFAD}\OperatorTok{[}\FunctionTok{Q} \SpecialCharTok{{-}}\NormalTok{ p1}\OperatorTok{]\}]}
\end{Highlighting}
\end{Shaded}

\begin{dmath*}\breakingcomma
\text{FCTopology}\left(\text{topo1},\left\{\frac{1}{(\text{p1}^2+i \eta )},\frac{1}{(\text{p2}^2+i \eta )},\frac{1}{((-\text{p1}-\text{p2}+Q)^2+i \eta )},\frac{1}{((Q-\text{p2})^2+i \eta )},\frac{1}{((Q-\text{p1})^2+i \eta )}\right\}\right)
\end{dmath*}

A 3-loop topology with one external momentum \texttt{Q}

\begin{Shaded}
\begin{Highlighting}[]
\NormalTok{FCTopology}\OperatorTok{[}\NormalTok{topo2}\OperatorTok{,} \OperatorTok{\{}\NormalTok{SFAD}\OperatorTok{[}\NormalTok{p1}\OperatorTok{],}\NormalTok{ SFAD}\OperatorTok{[}\NormalTok{p2}\OperatorTok{],}\NormalTok{ SFAD}\OperatorTok{[}\NormalTok{p3}\OperatorTok{],}\NormalTok{ SFAD}\OperatorTok{[}\FunctionTok{Q} \SpecialCharTok{{-}}\NormalTok{ p1 }\SpecialCharTok{{-}}\NormalTok{ p2 }\SpecialCharTok{{-}}\NormalTok{ p3}\OperatorTok{],}\NormalTok{ SFAD}\OperatorTok{[}\FunctionTok{Q} \SpecialCharTok{{-}}\NormalTok{ p1 }\SpecialCharTok{{-}}\NormalTok{ p2}\OperatorTok{],} 
\NormalTok{   SFAD}\OperatorTok{[}\FunctionTok{Q} \SpecialCharTok{{-}}\NormalTok{ p1}\OperatorTok{],}\NormalTok{ SFAD}\OperatorTok{[}\FunctionTok{Q} \SpecialCharTok{{-}}\NormalTok{ p2}\OperatorTok{],}\NormalTok{ SFAD}\OperatorTok{[}\NormalTok{p1 }\SpecialCharTok{+}\NormalTok{ p3}\OperatorTok{],}\NormalTok{ SFAD}\OperatorTok{[}\NormalTok{p2 }\SpecialCharTok{+}\NormalTok{ p3}\OperatorTok{]\}]}
\end{Highlighting}
\end{Shaded}

\begin{dmath*}\breakingcomma
\text{FCTopology}\left(\text{topo2},\left\{\frac{1}{(\text{p1}^2+i \eta )},\frac{1}{(\text{p2}^2+i \eta )},\frac{1}{(\text{p3}^2+i \eta )},\frac{1}{((-\text{p1}-\text{p2}-\text{p3}+Q)^2+i \eta )},\frac{1}{((-\text{p1}-\text{p2}+Q)^2+i \eta )},\frac{1}{((Q-\text{p1})^2+i \eta )},\frac{1}{((Q-\text{p2})^2+i \eta )},\frac{1}{((\text{p1}+\text{p3})^2+i \eta )},\frac{1}{((\text{p2}+\text{p3})^2+i \eta )}\right\}\right)
\end{dmath*}
\end{document}
