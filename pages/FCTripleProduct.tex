% !TeX program = pdflatex
% !TeX root = FCTripleProduct.tex

\documentclass[../FeynCalcManual.tex]{subfiles}
\begin{document}
\hypertarget{fctripleproduct}{
\section{FCTripleProduct}\label{fctripleproduct}\index{FCTripleProduct}}

\texttt{FCTripleProduct[\allowbreak{}a,\ \allowbreak{}b,\ \allowbreak{}c]}
returns the triple product \(a \cdot (b \times c)\). By default
\texttt{a},\texttt{b} and \texttt{c} are assumed to be Cartesian
vectors. Wrapping the arguments with \texttt{CartesianIndex} will create
an expression with open indices.

If any of the arguments is noncommutative, \texttt{DOT} will be used
instead of \texttt{Times} and the function will introduce dummy indices.
To give those indices some specific names, use the option
\texttt{CartesianIndexNames}.

If the arguments already contain free CartesianIndices, the first such
index will be used for the contraction.

To obtain an explicit expression you need to set the option
\texttt{Explicit} to \texttt{True} or apply the function
\texttt{Explicit}

\subsection{See also}

\hyperlink{toc}{Overview}, \hyperlink{eps}{Eps}.

\subsection{Examples}

\begin{Shaded}
\begin{Highlighting}[]
\NormalTok{FCTP}\OperatorTok{[}\FunctionTok{a}\OperatorTok{,} \FunctionTok{b}\OperatorTok{,} \FunctionTok{c}\OperatorTok{]} 
 
\SpecialCharTok{\%} \SpecialCharTok{//} \FunctionTok{StandardForm}
\end{Highlighting}
\end{Shaded}

\begin{dmath*}\breakingcomma
\overline{a}\cdot \left(\overline{b}\times \overline{c}\right)
\end{dmath*}

\begin{Shaded}
\begin{Highlighting}[]
\CommentTok{(*FCTripleProduct[a, b, c]*)}
\end{Highlighting}
\end{Shaded}

\begin{Shaded}
\begin{Highlighting}[]
\NormalTok{FCTP}\OperatorTok{[}\FunctionTok{a}\OperatorTok{,} \FunctionTok{b}\OperatorTok{,} \FunctionTok{c}\OperatorTok{,}\NormalTok{ Explicit }\OtherTok{{-}\textgreater{}} \ConstantTok{True}\OperatorTok{]} 
 
\SpecialCharTok{\%} \SpecialCharTok{//} \FunctionTok{StandardForm}
\end{Highlighting}
\end{Shaded}

\begin{dmath*}\breakingcomma
\bar{\epsilon }^{\overline{a}\overline{b}\overline{c}}
\end{dmath*}

\begin{Shaded}
\begin{Highlighting}[]
\CommentTok{(*Eps[CartesianMomentum[a], CartesianMomentum[b], CartesianMomentum[c]]*)}
\end{Highlighting}
\end{Shaded}

\begin{Shaded}
\begin{Highlighting}[]
\NormalTok{FCTP}\OperatorTok{[}\NormalTok{QuantumField}\OperatorTok{[}\FunctionTok{A}\OperatorTok{,}\NormalTok{ CartesianIndex}\OperatorTok{[}\FunctionTok{i}\OperatorTok{]],}\NormalTok{ QuantumField}\OperatorTok{[}\FunctionTok{B}\OperatorTok{,}\NormalTok{ CartesianIndex}\OperatorTok{[}\FunctionTok{j}\OperatorTok{]],} 
\NormalTok{  QuantumField}\OperatorTok{[}\FunctionTok{C}\OperatorTok{,}\NormalTok{ CartesianIndex}\OperatorTok{[}\FunctionTok{k}\OperatorTok{]],}\NormalTok{ Explicit }\OtherTok{{-}\textgreater{}} \ConstantTok{True}\OperatorTok{]}
\end{Highlighting}
\end{Shaded}

\begin{dmath*}\breakingcomma
\bar{\epsilon }^{ijk} A^i.B^j.C^k
\end{dmath*}

\begin{Shaded}
\begin{Highlighting}[]
\NormalTok{FCTP}\OperatorTok{[}\FunctionTok{a}\OperatorTok{,} \FunctionTok{b}\OperatorTok{,} \FunctionTok{c}\OperatorTok{,}\NormalTok{ Explicit }\OtherTok{{-}\textgreater{}} \ConstantTok{True}\OperatorTok{,}\NormalTok{ NonCommutative }\OtherTok{{-}\textgreater{}} \ConstantTok{True}\OperatorTok{,}\NormalTok{ CartesianIndexNames }\OtherTok{{-}\textgreater{}} \OperatorTok{\{}\FunctionTok{i}\OperatorTok{,} \FunctionTok{j}\OperatorTok{,} \FunctionTok{k}\OperatorTok{\}]}
\end{Highlighting}
\end{Shaded}

\begin{dmath*}\breakingcomma
\bar{\epsilon }^{ijk} \overline{a}^i.\overline{b}^j.\overline{c}^k
\end{dmath*}
\end{document}
