% !TeX program = pdflatex
% !TeX root = FCUseCache.tex

\documentclass[../FeynCalcManual.tex]{subfiles}
\begin{document}
\hypertarget{fcusecache}{%
\section{FCUseCache}\label{fcusecache}}

\texttt{FCUseCache[\allowbreak{}func,\ \allowbreak{}\{\allowbreak{}arg1,\ \allowbreak{}...\},\ \allowbreak{}\{\allowbreak{}opt1...\}]}
evaluates
\texttt{func[\allowbreak{}arg1,\ \allowbreak{}...,\ \allowbreak{}opt1,\ \allowbreak{}...]}
and caches the result such that the next evaluation of same expressions
occurs almost immediately. This caching also takes into account
\texttt{DownValues} and global variables that enter into evaluation of
func.

For example, \texttt{ExpandScalarProduct} can't be naively cached,
because its result depends on the \texttt{DownValues} of \texttt{Pair}
and \texttt{ScalarProduct}, which may be changed multiple times during
the session by setting and erasing values of scalar products. With
\texttt{FCUseCache}, however, caching will work properly, as
\texttt{FCUseCache} knows the dependence on \texttt{ExpandScalarProduct}
on those DownValues. For all this to work, a function should be
explicitly white-listed in FCUseCache.

\subsection{See also}

\hyperlink{toc}{Overview}, \hyperlink{fcshowcache}{FCShowCache}.

\subsection{Examples}
\end{document}
