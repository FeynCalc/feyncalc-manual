% !TeX program = pdflatex
% !TeX root = FCVariable.tex

\documentclass[../FeynCalcManual.tex]{subfiles}
\begin{document}
\hypertarget{fcvariable}{
\section{FCVariable}\label{fcvariable}\index{FCVariable}}

\texttt{FCVariable} is a data type. E.g.
\texttt{DataType[\allowbreak{}z,\ \allowbreak{}FCVariable] = True}.

\subsection{See also}

\hyperlink{toc}{Overview},
\hyperlink{expandscalarproduct}{ExpandScalarProduct},
\hyperlink{datatype}{DataType}.

\subsection{Examples}

If we want to introduce constants \texttt{c1} and \texttt{c2}, the naive
way doesn't lead to the desired result

\begin{Shaded}
\begin{Highlighting}[]
\NormalTok{SPD}\OperatorTok{[}\NormalTok{c1 p1 }\SpecialCharTok{+}\NormalTok{ c2 p2}\OperatorTok{,} \FunctionTok{q}\OperatorTok{]} \SpecialCharTok{//}\NormalTok{ ExpandScalarProduct}
\end{Highlighting}
\end{Shaded}

\begin{dmath*}\breakingcomma
\text{c1} \;\text{p1}\cdot q+\text{c2} \;\text{p2}\cdot q
\end{dmath*}

The solution is to declare \texttt{c1} and \texttt{c2} as
\texttt{FCVariable} so that FeynCalc can distinguish them from the
4-momenta

\begin{Shaded}
\begin{Highlighting}[]
\NormalTok{DataType}\OperatorTok{[}\NormalTok{c1}\OperatorTok{,}\NormalTok{ FCVariable}\OperatorTok{]} \ExtensionTok{=} \ConstantTok{True}\NormalTok{; }
 
\NormalTok{DataType}\OperatorTok{[}\NormalTok{c2}\OperatorTok{,}\NormalTok{ FCVariable}\OperatorTok{]} \ExtensionTok{=} \ConstantTok{True}\NormalTok{;}
\end{Highlighting}
\end{Shaded}

\begin{Shaded}
\begin{Highlighting}[]
\NormalTok{SPD}\OperatorTok{[}\NormalTok{c1 p1 }\SpecialCharTok{+}\NormalTok{ c2 p2}\OperatorTok{,} \FunctionTok{q}\OperatorTok{]} \SpecialCharTok{//}\NormalTok{ ExpandScalarProduct}
\end{Highlighting}
\end{Shaded}

\begin{dmath*}\breakingcomma
\text{c1} (\text{p1}\cdot q)+\text{c2} (\text{p2}\cdot q)
\end{dmath*}

This works also for propagator denominators and matrices

\begin{Shaded}
\begin{Highlighting}[]
\NormalTok{FCI}\OperatorTok{[}\NormalTok{SFAD}\OperatorTok{[\{}\FunctionTok{q} \SpecialCharTok{+}\NormalTok{ c1 p1}\OperatorTok{,} \FunctionTok{m}\OperatorTok{\}]]}
\end{Highlighting}
\end{Shaded}

\begin{dmath*}\breakingcomma
\frac{1}{((\text{c1} \;\text{p1}+q)^2-m+i \eta )}
\end{dmath*}

\begin{Shaded}
\begin{Highlighting}[]
\NormalTok{FCI}\OperatorTok{[}\NormalTok{SFAD}\OperatorTok{[\{}\FunctionTok{q} \SpecialCharTok{+}\NormalTok{ c1 p1}\OperatorTok{,} \FunctionTok{m}\OperatorTok{\}]]} \SpecialCharTok{//} \FunctionTok{StandardForm}

\CommentTok{(*FeynAmpDenominator[StandardPropagatorDenominator[c1 Momentum[p1, D] + Momentum[q, D], 0, {-}m, \{1, 1\}]]*)}
\end{Highlighting}
\end{Shaded}

\begin{Shaded}
\begin{Highlighting}[]
\NormalTok{GAD}\OperatorTok{[}\SpecialCharTok{\textbackslash{}}\OperatorTok{[}\NormalTok{Mu}\OperatorTok{]]}\NormalTok{ . (GSD}\OperatorTok{[}\NormalTok{c1 }\FunctionTok{p}\OperatorTok{]} \SpecialCharTok{+} \FunctionTok{m}\NormalTok{) . GAD}\OperatorTok{[}\SpecialCharTok{\textbackslash{}}\OperatorTok{[}\NormalTok{Nu}\OperatorTok{]]} \SpecialCharTok{//}\NormalTok{ FCI}
\end{Highlighting}
\end{Shaded}

\begin{dmath*}\breakingcomma
\gamma ^{\mu }.(\text{c1} \gamma \cdot p+m).\gamma ^{\nu }
\end{dmath*}

\begin{Shaded}
\begin{Highlighting}[]
\NormalTok{GAD}\OperatorTok{[}\SpecialCharTok{\textbackslash{}}\OperatorTok{[}\NormalTok{Mu}\OperatorTok{]]}\NormalTok{ . (GSD}\OperatorTok{[}\NormalTok{c1 }\FunctionTok{p}\OperatorTok{]} \SpecialCharTok{+} \FunctionTok{m}\NormalTok{) . GAD}\OperatorTok{[}\SpecialCharTok{\textbackslash{}}\OperatorTok{[}\NormalTok{Nu}\OperatorTok{]]} \SpecialCharTok{//}\NormalTok{ FCI }\SpecialCharTok{//} \FunctionTok{StandardForm}

\CommentTok{(*DiracGamma[LorentzIndex[\textbackslash{}[Mu], D], D] . (m + c1 DiracGamma[Momentum[p, D], D]) . DiracGamma[LorentzIndex[\textbackslash{}[Nu], D], D]*)}
\end{Highlighting}
\end{Shaded}

\begin{Shaded}
\begin{Highlighting}[]
\NormalTok{CSI}\OperatorTok{[}\FunctionTok{i}\OperatorTok{]}\NormalTok{ . CSIS}\OperatorTok{[}\NormalTok{c1 }\FunctionTok{p}\OperatorTok{]}\NormalTok{ . CSI}\OperatorTok{[}\FunctionTok{j}\OperatorTok{]} \SpecialCharTok{//}\NormalTok{ FCI}
\end{Highlighting}
\end{Shaded}

\begin{dmath*}\breakingcomma
\overline{\sigma }^i.\left(\text{c1} \overline{\sigma }\cdot \overline{p}\right).\overline{\sigma }^j
\end{dmath*}

\begin{Shaded}
\begin{Highlighting}[]
\NormalTok{CSI}\OperatorTok{[}\FunctionTok{i}\OperatorTok{]}\NormalTok{ . CSIS}\OperatorTok{[}\NormalTok{c1 }\FunctionTok{p}\OperatorTok{]}\NormalTok{ . CSI}\OperatorTok{[}\FunctionTok{j}\OperatorTok{]} \SpecialCharTok{//}\NormalTok{ FCI }\SpecialCharTok{//} \FunctionTok{StandardForm}

\CommentTok{(*PauliSigma[CartesianIndex[i]] . (c1 PauliSigma[CartesianMomentum[p]]) . PauliSigma[CartesianIndex[j]]*)}
\end{Highlighting}
\end{Shaded}

To undo the declarations use

\begin{Shaded}
\begin{Highlighting}[]
\NormalTok{DataType}\OperatorTok{[}\NormalTok{c1}\OperatorTok{,}\NormalTok{ FCVariable}\OperatorTok{]} \ExtensionTok{=} \ConstantTok{False} 
 
\NormalTok{DataType}\OperatorTok{[}\NormalTok{c2}\OperatorTok{,}\NormalTok{ FCVariable}\OperatorTok{]} \ExtensionTok{=} \ConstantTok{False}
\end{Highlighting}
\end{Shaded}

\begin{dmath*}\breakingcomma
\text{False}
\end{dmath*}

\begin{dmath*}\breakingcomma
\text{False}
\end{dmath*}
\end{document}
