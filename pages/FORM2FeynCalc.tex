% !TeX program = pdflatex
% !TeX root = FORM2FeynCalc.tex

\documentclass[../FeynCalcManual.tex]{subfiles}
\begin{document}
\hypertarget{form2feyncalc}{
\section{FORM2FeynCalc}\label{form2feyncalc}\index{FORM2FeynCalc}}

\texttt{FORM2FeynCalc[\allowbreak{}exp]} translates the FORM expression
\texttt{exp} into FeynCalc notation.

\texttt{FORM2FeynCalc[\allowbreak{}file]} translates the FORM
expressions in \texttt{file} into FeynCalc notation.

\texttt{FORM2FeynCalc[\allowbreak{}file,\ \allowbreak{}x1,\ \allowbreak{}x2,\ \allowbreak{}...]}
reads in a file in FORM-format and translates the assignments for the
variables \(a, b, \ldots\) into FeynCalc syntax.

If the option \texttt{Set} is \texttt{True}, the variables \texttt{x1},
\texttt{x2} are assigned to the right hand sides defined in the
FORM-file.The capabilities of this function are very limited, so that
you should not expect it to easily handle large and complicated
expressions.

\subsection{See also}

\hyperlink{toc}{Overview}, \hyperlink{feyncalc2form}{FeynCalc2FORM}.

\subsection{Examples}

\begin{Shaded}
\begin{Highlighting}[]
\NormalTok{FORM2FeynCalc}\OperatorTok{[}\StringTok{"p.q + 2*x m\^{}2"}\OperatorTok{]}
\end{Highlighting}
\end{Shaded}

\begin{dmath*}\breakingcomma
\overline{p}\cdot \overline{q}+2 m^2.x
\end{dmath*}

\begin{Shaded}
\begin{Highlighting}[]
\NormalTok{FORM2FeynCalc}\OperatorTok{[}\StringTok{"p.q + 2*x m\^{}2"}\OperatorTok{]} \SpecialCharTok{//} \FunctionTok{StandardForm}

\CommentTok{(*2 x . m\^{}2 + SP[p, q]*)}
\end{Highlighting}
\end{Shaded}

Functions are automatically converted in a proper way, but bracketed
expressions need to be substituted explicitly.

\begin{Shaded}
\begin{Highlighting}[]
\NormalTok{FORM2FeynCalc}\OperatorTok{[}\StringTok{"x +f(z)+ log(x)\^{}2+[li2(1{-}x)]"}\OperatorTok{,} \FunctionTok{Replace} \OtherTok{{-}\textgreater{}} \OperatorTok{\{}\StringTok{"[li2(1{-}x)]"} \OtherTok{{-}\textgreater{}} \StringTok{"PolyLog[2,1{-}x]"}\OperatorTok{\}]}
\end{Highlighting}
\end{Shaded}

\begin{dmath*}\breakingcomma
f(z)+\text{Li}_2(1-x)+x+\log ^2(x)
\end{dmath*}

\begin{Shaded}
\begin{Highlighting}[]
\NormalTok{FORM2FeynCalc}\OperatorTok{[}\StringTok{"x +f(z)+ log(x)\^{}2+[li2(1{-}x)]"}\OperatorTok{,} \FunctionTok{Replace} \OtherTok{{-}\textgreater{}} \OperatorTok{\{}\StringTok{"[li2(1{-}x)]"} \OtherTok{{-}\textgreater{}} \StringTok{"PolyLog[2,1{-}x]"}\OperatorTok{\}]} \SpecialCharTok{//} \FunctionTok{StandardForm}

\CommentTok{(*x + f[z] + Log[x]\^{}2 + PolyLog[2, 1 {-} x]*)}
\end{Highlighting}
\end{Shaded}

\begin{Shaded}
\begin{Highlighting}[]
\NormalTok{FORM2FeynCalc}\OperatorTok{[}\StringTok{"x + [(1)]*y {-}[({-}1)\^{}m]"}\OperatorTok{]}
\end{Highlighting}
\end{Shaded}

\begin{dmath*}\breakingcomma
-\text{Hold}\left[(-1)^m\right]+\text{Hold}[1].y+x
\end{dmath*}

\begin{Shaded}
\begin{Highlighting}[]
\FunctionTok{ReleaseHold}\OperatorTok{[}\NormalTok{FORM2FeynCalc}\OperatorTok{[}\StringTok{"x + [(1)]*y {-}[({-}1)\^{}m]"}\OperatorTok{]]}
\end{Highlighting}
\end{Shaded}

\begin{dmath*}\breakingcomma
-(-1)^m+x+1.y
\end{dmath*}

\begin{Shaded}
\begin{Highlighting}[]
\NormalTok{FORM2FeynCalc}\OperatorTok{[}\StringTok{"p(mu)*q(nu)+d\_(mu,nu)"}\OperatorTok{]}
\end{Highlighting}
\end{Shaded}

\begin{dmath*}\breakingcomma
\bar{g}^{\text{mu}\;\text{nu}}+p(\text{mu}).q(\text{nu})
\end{dmath*}

\begin{Shaded}
\begin{Highlighting}[]
\NormalTok{FORM2FeynCalc}\OperatorTok{[}\StringTok{"p(mu)*q(nu)+d\_(mu,nu)"}\OperatorTok{]} \SpecialCharTok{//} \FunctionTok{StandardForm}

\CommentTok{(*p[mu] . q[nu] + MT[mu, nu]*)}
\end{Highlighting}
\end{Shaded}

\begin{Shaded}
\begin{Highlighting}[]
\NormalTok{FORM2FeynCalc}\OperatorTok{[}\StringTok{"p(mu)*q(nu)+d\_(mu,nu)"}\OperatorTok{,} \FunctionTok{Replace} \OtherTok{{-}\textgreater{}} \OperatorTok{\{}\NormalTok{mu }\OtherTok{{-}\textgreater{}} \SpecialCharTok{\textbackslash{}}\OperatorTok{[}\NormalTok{Mu}\OperatorTok{],}\NormalTok{ nu }\OtherTok{{-}\textgreater{}} \SpecialCharTok{\textbackslash{}}\OperatorTok{[}\NormalTok{Nu}\OperatorTok{]\}]}
\end{Highlighting}
\end{Shaded}

\begin{dmath*}\breakingcomma
\bar{g}^{\mu \nu }+p(\mu ).q(\nu )
\end{dmath*}

\begin{Shaded}
\begin{Highlighting}[]
\NormalTok{FORM2FeynCalc}\OperatorTok{[}\StringTok{"i\_*az*bz*aM\^{}2*D1*[(1)]*b\_G1 * ( 4*eperp(mu,nu)*avec.bvec*blam )"}\OperatorTok{]}
\end{Highlighting}
\end{Shaded}

\begin{dmath*}\breakingcomma
(4 i).\text{az}.\text{bz}.\text{aM}^2.\text{D1}.\text{Hold}[1].\text{b\$G1}.\text{eperp}(\text{mu},\text{nu}).\left(\overline{\text{avec}}\cdot \overline{\text{bvec}}\right).\text{blam}
\end{dmath*}
\end{document}
