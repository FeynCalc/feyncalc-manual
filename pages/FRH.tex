% !TeX program = pdflatex
% !TeX root = FRH.tex

\documentclass[../FeynCalcManual.tex]{subfiles}
\begin{document}
\hypertarget{frh}{
\section{FRH}\label{frh}\index{FRH}}

\texttt{FRH[\allowbreak{}exp_]} corresponds to
\texttt{FixedPoint[\allowbreak{}ReleaseHold,\ \allowbreak{}exp]},
i.e.~\texttt{FRH} removes all \texttt{HoldForm} and \texttt{Hold} in
\texttt{exp}.

\subsection{See also}

\hyperlink{toc}{Overview}, \hyperlink{isolate}{Isolate}.

\subsection{Examples}

\begin{Shaded}
\begin{Highlighting}[]
\FunctionTok{Hold}\OperatorTok{[}\DecValTok{1} \SpecialCharTok{{-}} \DecValTok{1} \SpecialCharTok{{-}} \FunctionTok{Hold}\OperatorTok{[}\DecValTok{2} \SpecialCharTok{{-}} \DecValTok{2}\OperatorTok{]]}
\end{Highlighting}
\end{Shaded}

\begin{dmath*}\breakingcomma
\text{Hold}[-\text{Hold}[2-2]+1-1]
\end{dmath*}

\begin{Shaded}
\begin{Highlighting}[]
\NormalTok{FRH}\OperatorTok{[}\SpecialCharTok{\%}\OperatorTok{]}
\end{Highlighting}
\end{Shaded}

\begin{dmath*}\breakingcomma
0
\end{dmath*}

\begin{Shaded}
\begin{Highlighting}[]
\NormalTok{Isolate}\OperatorTok{[}\FunctionTok{ToRadicals}\OperatorTok{[}\FunctionTok{Solve}\OperatorTok{[}\FunctionTok{x}\SpecialCharTok{\^{}}\DecValTok{3} \SpecialCharTok{{-}} \FunctionTok{x} \SpecialCharTok{{-}} \DecValTok{1} \ExtensionTok{==} \DecValTok{0}\OperatorTok{]],} \FunctionTok{x}\OperatorTok{,}\NormalTok{ IsolateNames }\OtherTok{{-}\textgreater{}}\NormalTok{ KK}\OperatorTok{]}
\end{Highlighting}
\end{Shaded}

\begin{dmath*}\breakingcomma
\{\{x\to \;\text{KK}(21)\},\{x\to \;\text{KK}(24)\},\{x\to \;\text{KK}(25)\}\}
\end{dmath*}

\begin{Shaded}
\begin{Highlighting}[]
\NormalTok{FRH}\OperatorTok{[}\SpecialCharTok{\%}\OperatorTok{]}
\end{Highlighting}
\end{Shaded}

\begin{dmath*}\breakingcomma
\left\{\left\{x\to \frac{1}{3} \sqrt[3]{\frac{27}{2}-\frac{3 \sqrt{69}}{2}}+\frac{\sqrt[3]{\frac{1}{2} \left(9+\sqrt{69}\right)}}{3^{2/3}}\right\},\left\{x\to -\frac{1}{6} \left(1-i \sqrt{3}\right) \sqrt[3]{\frac{27}{2}-\frac{3 \sqrt{69}}{2}}-\frac{\left(1+i \sqrt{3}\right) \sqrt[3]{\frac{1}{2} \left(9+\sqrt{69}\right)}}{2\ 3^{2/3}}\right\},\left\{x\to -\frac{1}{6} \left(1+i \sqrt{3}\right) \sqrt[3]{\frac{27}{2}-\frac{3 \sqrt{69}}{2}}-\frac{\left(1-i \sqrt{3}\right) \sqrt[3]{\frac{1}{2} \left(9+\sqrt{69}\right)}}{2\ 3^{2/3}}\right\}\right\}
\end{dmath*}
\end{document}
