% !TeX program = pdflatex
% !TeX root = FV.tex

\documentclass[../FeynCalcManual.tex]{subfiles}
\begin{document}
\hypertarget{fv}{
\section{FV}\label{fv}\index{FV}}

\texttt{FV[\allowbreak{}p,\ \allowbreak{}mu]} is the \(4\)-dimensional
vector \(p^{\mu }\).

\subsection{See also}

\hyperlink{toc}{Overview}, \hyperlink{fce}{FCE}, \hyperlink{fci}{FCI},
\hyperlink{fvd}{FVD}, \hyperlink{pair}{Pair}.

\subsection{Examples}

\begin{Shaded}
\begin{Highlighting}[]
\NormalTok{FV}\OperatorTok{[}\FunctionTok{p}\OperatorTok{,} \SpecialCharTok{\textbackslash{}}\OperatorTok{[}\NormalTok{Mu}\OperatorTok{]]}
\end{Highlighting}
\end{Shaded}

\begin{dmath*}\breakingcomma
\overline{p}^{\mu }
\end{dmath*}

\begin{Shaded}
\begin{Highlighting}[]
\NormalTok{FV}\OperatorTok{[}\FunctionTok{p} \SpecialCharTok{{-}} \FunctionTok{q}\OperatorTok{,} \SpecialCharTok{\textbackslash{}}\OperatorTok{[}\NormalTok{Mu}\OperatorTok{]]}
\end{Highlighting}
\end{Shaded}

\begin{dmath*}\breakingcomma
\left(\overline{p}-\overline{q}\right)^{\mu }
\end{dmath*}

\begin{Shaded}
\begin{Highlighting}[]
\NormalTok{FV}\OperatorTok{[}\FunctionTok{p}\OperatorTok{,} \SpecialCharTok{\textbackslash{}}\OperatorTok{[}\NormalTok{Mu}\OperatorTok{]]} \SpecialCharTok{//} \FunctionTok{StandardForm}

\CommentTok{(*FV[p, \textbackslash{}[Mu]]*)}
\end{Highlighting}
\end{Shaded}

\begin{Shaded}
\begin{Highlighting}[]
\NormalTok{FCI}\OperatorTok{[}\NormalTok{FV}\OperatorTok{[}\FunctionTok{p}\OperatorTok{,} \SpecialCharTok{\textbackslash{}}\OperatorTok{[}\NormalTok{Mu}\OperatorTok{]]]} \SpecialCharTok{//} \FunctionTok{StandardForm}

\CommentTok{(*Pair[LorentzIndex[\textbackslash{}[Mu]], Momentum[p]]*)}
\end{Highlighting}
\end{Shaded}

\texttt{ExpandScalarProduct} is used to expand momenta in \texttt{FV}

\begin{Shaded}
\begin{Highlighting}[]
\NormalTok{ExpandScalarProduct}\OperatorTok{[}\NormalTok{FV}\OperatorTok{[}\FunctionTok{p} \SpecialCharTok{{-}} \FunctionTok{q}\OperatorTok{,} \SpecialCharTok{\textbackslash{}}\OperatorTok{[}\NormalTok{Mu}\OperatorTok{]]]}
\end{Highlighting}
\end{Shaded}

\begin{dmath*}\breakingcomma
\overline{p}^{\mu }-\overline{q}^{\mu }
\end{dmath*}
\end{document}
