% !TeX program = pdflatex
% !TeX root = FVD.tex

\documentclass[../FeynCalcManual.tex]{subfiles}
\begin{document}
\hypertarget{fvd}{%
\section{FVD}\label{fvd}}

\texttt{FVD[\allowbreak{}p,\ \allowbreak{}mu]} is the \(D\)-dimensional
vector \(p\) with Lorentz index \texttt{mu}.

\subsection{See also}

\hyperlink{toc}{Overview}, \hyperlink{fce}{FCE}, \hyperlink{fci}{FCI},
\hyperlink{fv}{FV}, \hyperlink{pair}{Pair}.

\subsection{Examples}

\begin{Shaded}
\begin{Highlighting}[]
\NormalTok{FVD}\OperatorTok{[}\FunctionTok{p}\OperatorTok{,} \SpecialCharTok{\textbackslash{}}\OperatorTok{[}\NormalTok{Mu}\OperatorTok{]]}
\end{Highlighting}
\end{Shaded}

\begin{dmath*}\breakingcomma
p^{\mu }
\end{dmath*}

\begin{Shaded}
\begin{Highlighting}[]
\NormalTok{FVD}\OperatorTok{[}\FunctionTok{p} \SpecialCharTok{{-}} \FunctionTok{q}\OperatorTok{,} \SpecialCharTok{\textbackslash{}}\OperatorTok{[}\NormalTok{Mu}\OperatorTok{]]}
\end{Highlighting}
\end{Shaded}

\begin{dmath*}\breakingcomma
(p-q)^{\mu }
\end{dmath*}

\begin{Shaded}
\begin{Highlighting}[]
\NormalTok{FVD}\OperatorTok{[}\FunctionTok{p}\OperatorTok{,} \SpecialCharTok{\textbackslash{}}\OperatorTok{[}\NormalTok{Mu}\OperatorTok{]]} \SpecialCharTok{//} \FunctionTok{StandardForm}

\CommentTok{(*FVD[p, \textbackslash{}[Mu]]*)}
\end{Highlighting}
\end{Shaded}

\begin{Shaded}
\begin{Highlighting}[]
\NormalTok{FCI}\OperatorTok{[}\NormalTok{FVD}\OperatorTok{[}\FunctionTok{p}\OperatorTok{,} \SpecialCharTok{\textbackslash{}}\OperatorTok{[}\NormalTok{Mu}\OperatorTok{]]]} \SpecialCharTok{//} \FunctionTok{StandardForm}

\CommentTok{(*Pair[LorentzIndex[\textbackslash{}[Mu], D], Momentum[p, D]]*)}
\end{Highlighting}
\end{Shaded}

There is no special function to expand momenta in \texttt{FVD}.

\begin{Shaded}
\begin{Highlighting}[]
\NormalTok{ex }\ExtensionTok{=}\NormalTok{ ExpandScalarProduct}\OperatorTok{[}\NormalTok{FVD}\OperatorTok{[}\FunctionTok{p} \SpecialCharTok{{-}} \FunctionTok{q}\OperatorTok{,} \SpecialCharTok{\textbackslash{}}\OperatorTok{[}\NormalTok{Mu}\OperatorTok{]]]}
\end{Highlighting}
\end{Shaded}

\begin{dmath*}\breakingcomma
p^{\mu }-q^{\mu }
\end{dmath*}

\begin{Shaded}
\begin{Highlighting}[]
\NormalTok{ex }\SpecialCharTok{//} \FunctionTok{StandardForm}

\CommentTok{(*Pair[LorentzIndex[\textbackslash{}[Mu], D], Momentum[p, D]] {-} Pair[LorentzIndex[\textbackslash{}[Mu], D], Momentum[q, D]]*)}
\end{Highlighting}
\end{Shaded}

\end{document}
