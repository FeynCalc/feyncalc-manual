% !TeX program = pdflatex
% !TeX root = FVLRD.tex

\documentclass[../FeynCalcManual.tex]{subfiles}
\begin{document}
\hypertarget{fvlrd}{
\section{FVLRD}\label{fvlrd}\index{FVLRD}}

\texttt{FVLRD[\allowbreak{}p,\ \allowbreak{}mu,\ \allowbreak{}n,\ \allowbreak{}nb]}
denotes the perpendicular component in the lightcone decomposition of
the Lorentz vector \(p^{\mu }\) along the vectors \texttt{n} and
\texttt{nb} in \(D\) dimensions. It corresponds to \(p^{\mu }_{\perp}\).

If one omits \texttt{n} and \texttt{nb}, the program will use default
vectors specified via \texttt{\$FCDefaultLightconeVectorN} and
\texttt{\$FCDefaultLightconeVectorNB}.

\subsection{See also}

\hyperlink{toc}{Overview}, \hyperlink{pair}{Pair},
\hyperlink{fvlnd}{FVLND}, \hyperlink{fvlpd}{FVLPD},
\hyperlink{splpd}{SPLPD}, \hyperlink{splnd}{SPLND},
\hyperlink{splrd}{SPLRD}, \hyperlink{mtlpd}{MTLPD},
\hyperlink{mtlnd}{MTLND}, \hyperlink{mtlrd}{MTLRD}.

\subsection{Examples}

\begin{Shaded}
\begin{Highlighting}[]
\NormalTok{FVLRD}\OperatorTok{[}\FunctionTok{p}\OperatorTok{,} \SpecialCharTok{\textbackslash{}}\OperatorTok{[}\NormalTok{Mu}\OperatorTok{],} \FunctionTok{n}\OperatorTok{,}\NormalTok{ nb}\OperatorTok{]}
\end{Highlighting}
\end{Shaded}

\begin{dmath*}\breakingcomma
p^{\mu }{}_{\perp }
\end{dmath*}

\begin{Shaded}
\begin{Highlighting}[]
\NormalTok{FVLRD}\OperatorTok{[}\FunctionTok{p}\OperatorTok{,} \SpecialCharTok{\textbackslash{}}\OperatorTok{[}\NormalTok{Mu}\OperatorTok{],} \FunctionTok{n}\OperatorTok{,}\NormalTok{ nb}\OperatorTok{]} \SpecialCharTok{//}\NormalTok{ FCI }\SpecialCharTok{//} \FunctionTok{StandardForm}

\CommentTok{(*Pair[LightConePerpendicularComponent[LorentzIndex[\textbackslash{}[Mu], D], Momentum[n, D], Momentum[nb, D]], LightConePerpendicularComponent[Momentum[p, D], Momentum[n, D], Momentum[nb, D]]]*)}
\end{Highlighting}
\end{Shaded}

\begin{Shaded}
\begin{Highlighting}[]
\NormalTok{FVLRD}\OperatorTok{[}\FunctionTok{p}\OperatorTok{,} \SpecialCharTok{\textbackslash{}}\OperatorTok{[}\NormalTok{Mu}\OperatorTok{],} \FunctionTok{n}\OperatorTok{,}\NormalTok{ nb}\OperatorTok{]}\NormalTok{ FVLRD}\OperatorTok{[}\FunctionTok{q}\OperatorTok{,} \SpecialCharTok{\textbackslash{}}\OperatorTok{[}\NormalTok{Mu}\OperatorTok{],} \FunctionTok{n}\OperatorTok{,}\NormalTok{ nb}\OperatorTok{]} \SpecialCharTok{//}\NormalTok{ Contract}
\end{Highlighting}
\end{Shaded}

\begin{dmath*}\breakingcomma
p\cdot q_{\perp }
\end{dmath*}

\begin{Shaded}
\begin{Highlighting}[]
\NormalTok{FVLRD}\OperatorTok{[}\FunctionTok{p}\OperatorTok{,} \SpecialCharTok{\textbackslash{}}\OperatorTok{[}\NormalTok{Mu}\OperatorTok{],} \FunctionTok{n}\OperatorTok{,}\NormalTok{ nb}\OperatorTok{]}\NormalTok{ . FVLPD}\OperatorTok{[}\FunctionTok{q}\OperatorTok{,} \SpecialCharTok{\textbackslash{}}\OperatorTok{[}\NormalTok{Mu}\OperatorTok{],} \FunctionTok{n}\OperatorTok{,}\NormalTok{ nb}\OperatorTok{]} \SpecialCharTok{//}\NormalTok{ Contract}
\end{Highlighting}
\end{Shaded}

\begin{dmath*}\breakingcomma
0
\end{dmath*}
\end{document}
