% !TeX program = pdflatex
% !TeX root = Factor1.tex

\documentclass[../FeynCalcManual.tex]{subfiles}
\begin{document}
\hypertarget{factor1}{%
\section{Factor1}\label{factor1}}

\texttt{Factor1[\allowbreak{}poly]} factorizes common terms in the
summands of poly. It uses basically \texttt{PolynomialGCD}.

\subsection{See also}

\hyperlink{toc}{Overview}, \hyperlink{factor2}{Factor2}.

\subsection{Examples}

\begin{Shaded}
\begin{Highlighting}[]
\NormalTok{(}\FunctionTok{a} \SpecialCharTok{{-}} \FunctionTok{x}\NormalTok{) (}\FunctionTok{b} \SpecialCharTok{{-}} \FunctionTok{x}\NormalTok{) }
 
\OperatorTok{\{}\NormalTok{Factor1}\OperatorTok{[}\SpecialCharTok{\%}\OperatorTok{],} \FunctionTok{Factor}\OperatorTok{[}\SpecialCharTok{\%}\OperatorTok{]\}}
\end{Highlighting}
\end{Shaded}

\begin{dmath*}\breakingcomma
(a-x) (b-x)
\end{dmath*}

\begin{dmath*}\breakingcomma
\{(a-x) (b-x),-((a-x) (x-b))\}
\end{dmath*}

\begin{Shaded}
\begin{Highlighting}[]
\NormalTok{ex }\ExtensionTok{=} \FunctionTok{Expand}\OperatorTok{[}\NormalTok{(}\FunctionTok{a} \SpecialCharTok{{-}} \FunctionTok{b}\NormalTok{) (}\FunctionTok{a} \SpecialCharTok{+} \FunctionTok{b}\NormalTok{)}\OperatorTok{]}
\end{Highlighting}
\end{Shaded}

\begin{dmath*}\breakingcomma
a^2-b^2
\end{dmath*}

\begin{Shaded}
\begin{Highlighting}[]
\FunctionTok{Factor}\OperatorTok{[}\NormalTok{ex}\OperatorTok{]}
\end{Highlighting}
\end{Shaded}

\begin{dmath*}\breakingcomma
(a-b) (a+b)
\end{dmath*}

\begin{Shaded}
\begin{Highlighting}[]
\NormalTok{Factor1}\OperatorTok{[}\NormalTok{ex}\OperatorTok{]}
\end{Highlighting}
\end{Shaded}

\begin{dmath*}\breakingcomma
a^2-b^2
\end{dmath*}
\end{document}
