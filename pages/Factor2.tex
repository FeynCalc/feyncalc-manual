% !TeX program = pdflatex
% !TeX root = Factor2.tex

\documentclass[../FeynCalcManual.tex]{subfiles}
\begin{document}
\hypertarget{factor2}{
\section{Factor2}\label{factor2}\index{Factor2}}

\texttt{Factor2[\allowbreak{}poly]} factors a polynomial in a standard
way.

\texttt{Factor2} works sometimes better than Factor on polynomials
involving rationals with sums in the denominator.

\texttt{Factor2} uses \texttt{Factor} internally and is in general
slower than \texttt{Factor}. There are four possible settings of the
option \texttt{Method} (\texttt{0},\texttt{1},\texttt{2},\texttt{3}). In
general, \texttt{Factor} will work faster than \texttt{Factor2}.

\subsection{See also}

\hyperlink{toc}{Overview}, \hyperlink{collect2}{Collect2}.

\subsection{Examples}

\begin{Shaded}
\begin{Highlighting}[]
\NormalTok{(}\FunctionTok{a} \SpecialCharTok{{-}} \FunctionTok{x}\NormalTok{) (}\FunctionTok{b} \SpecialCharTok{{-}} \FunctionTok{x}\NormalTok{) }
 
\OperatorTok{\{}\NormalTok{Factor2}\OperatorTok{[}\SpecialCharTok{\%}\OperatorTok{],} \FunctionTok{Factor}\OperatorTok{[}\SpecialCharTok{\%}\OperatorTok{]\}}
\end{Highlighting}
\end{Shaded}

\begin{dmath*}\breakingcomma
(a-x) (b-x)
\end{dmath*}

\begin{dmath*}\breakingcomma
\{(a-x) (b-x),-((a-x) (x-b))\}
\end{dmath*}

\begin{Shaded}
\begin{Highlighting}[]
\NormalTok{ex }\ExtensionTok{=} \FunctionTok{Expand}\OperatorTok{[}\NormalTok{(}\FunctionTok{a} \SpecialCharTok{{-}} \FunctionTok{b}\NormalTok{) (}\FunctionTok{a} \SpecialCharTok{+} \FunctionTok{b}\NormalTok{)}\OperatorTok{]}
\end{Highlighting}
\end{Shaded}

\begin{dmath*}\breakingcomma
a^2-b^2
\end{dmath*}

\begin{Shaded}
\begin{Highlighting}[]
\FunctionTok{Factor}\OperatorTok{[}\NormalTok{ex}\OperatorTok{]}
\end{Highlighting}
\end{Shaded}

\begin{dmath*}\breakingcomma
(a-b) (a+b)
\end{dmath*}

\begin{Shaded}
\begin{Highlighting}[]
\NormalTok{Factor2}\OperatorTok{[}\NormalTok{ex}\OperatorTok{]}
\end{Highlighting}
\end{Shaded}

\begin{dmath*}\breakingcomma
a^2-b^2
\end{dmath*}

\begin{Shaded}
\begin{Highlighting}[]
\NormalTok{Factor2}\OperatorTok{[}\NormalTok{ex}\OperatorTok{,}\NormalTok{ FactorFull }\OtherTok{{-}\textgreater{}} \ConstantTok{True}\OperatorTok{]}
\end{Highlighting}
\end{Shaded}

\begin{dmath*}\breakingcomma
(a-b) (a+b)
\end{dmath*}
\end{document}
