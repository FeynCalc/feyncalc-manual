% !TeX program = pdflatex
% !TeX root = FactorList2.tex

\documentclass[../FeynCalcManual.tex]{subfiles}
\begin{document}
\hypertarget{factorlist2}{
\section{FactorList2}\label{factorlist2}\index{FactorList2}}

\texttt{FactorList2[\allowbreak{}exp]} is similar to \texttt{FactorList}
except that it correctly handles symbolic exponents.

\subsection{See also}

\hyperlink{toc}{Overview}, \hyperlink{factor2}{Factor2}.

\subsection{Examples}

\begin{Shaded}
\begin{Highlighting}[]
\NormalTok{FactorList2}\OperatorTok{[}\NormalTok{(}\FunctionTok{x}\OperatorTok{[}\DecValTok{1}\OperatorTok{]} \FunctionTok{x}\OperatorTok{[}\DecValTok{2}\OperatorTok{]} \SpecialCharTok{+} \FunctionTok{x}\OperatorTok{[}\DecValTok{1}\OperatorTok{]} \FunctionTok{x}\OperatorTok{[}\DecValTok{3}\OperatorTok{]} \SpecialCharTok{+} \FunctionTok{x}\OperatorTok{[}\DecValTok{2}\OperatorTok{]} \FunctionTok{x}\OperatorTok{[}\DecValTok{3}\OperatorTok{]}\NormalTok{)}\SpecialCharTok{\^{}}\NormalTok{(}\SpecialCharTok{{-}}\DecValTok{3} \SpecialCharTok{+} \DecValTok{3}\NormalTok{ ep)}\SpecialCharTok{/}\NormalTok{(}\FunctionTok{x}\OperatorTok{[}\DecValTok{1}\OperatorTok{]}\SpecialCharTok{\^{}}\DecValTok{2} \FunctionTok{x}\OperatorTok{[}\DecValTok{2}\OperatorTok{]} \SpecialCharTok{+} \FunctionTok{x}\OperatorTok{[}\DecValTok{1}\OperatorTok{]}\SpecialCharTok{\^{}}\DecValTok{2} \FunctionTok{x}\OperatorTok{[}\DecValTok{3}\OperatorTok{]}\NormalTok{)}\SpecialCharTok{\^{}}\NormalTok{(}\SpecialCharTok{{-}}\DecValTok{1} \SpecialCharTok{+} \DecValTok{2}\NormalTok{ ep)}\OperatorTok{]}
\end{Highlighting}
\end{Shaded}

\begin{dmath*}\breakingcomma
\left(
\begin{array}{cc}
 1 & 1 \\
 x(1)^2 (x(2)+x(3)) & -2 \;\text{ep} \\
 x(1) x(2)+x(3) x(2)+x(1) x(3) & 3 (\text{ep}-1) \\
 x(1) & 2 \\
 x(2)+x(3) & 1 \\
\end{array}
\right)
\end{dmath*}
\end{document}
