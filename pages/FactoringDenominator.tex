% !TeX program = pdflatex
% !TeX root = FactoringDenominator.tex

\documentclass[../FeynCalcManual.tex]{subfiles}
\begin{document}
\hypertarget{factoringdenominator}{%
\section{FactoringDenominator}\label{factoringdenominator}}

\texttt{FactoringDenominator} is an option for \texttt{Collect2}. It is
taken into account only when the option \texttt{Numerator} is set to
\texttt{True}. If \texttt{FactoringDenominator} is set to any function
\texttt{f}, this function will be applied to the denominator of the
fraction. The default value is \texttt{False}, i.e.~the denominator will
be left unchanged.

\subsection{See also}

\hyperlink{toc}{Overview}, \hyperlink{collect2}{Collect2}.

\subsection{Examples}

\begin{Shaded}
\begin{Highlighting}[]
\NormalTok{ex }\ExtensionTok{=}\NormalTok{ (x1 }\FunctionTok{a}\SpecialCharTok{\^{}}\DecValTok{2} \SpecialCharTok{+} \FunctionTok{y} \DecValTok{1} \FunctionTok{a}\SpecialCharTok{\^{}}\DecValTok{2} \SpecialCharTok{+} \DecValTok{2} \FunctionTok{a} \FunctionTok{b} \SpecialCharTok{+}\NormalTok{ x2 }\FunctionTok{b}\SpecialCharTok{\^{}}\DecValTok{2} \SpecialCharTok{+}\NormalTok{ y2 }\FunctionTok{b}\SpecialCharTok{\^{}}\DecValTok{2}\NormalTok{)}\SpecialCharTok{/}\NormalTok{(}\FunctionTok{a} \SpecialCharTok{+} \FunctionTok{b} \SpecialCharTok{+} \FunctionTok{c}\SpecialCharTok{\^{}}\DecValTok{2} \SpecialCharTok{+} 
        \DecValTok{2} \FunctionTok{c} \FunctionTok{d} \SpecialCharTok{+} \FunctionTok{d}\SpecialCharTok{\^{}}\DecValTok{2}\NormalTok{)}
\end{Highlighting}
\end{Shaded}

\begin{dmath*}\breakingcomma
\frac{a^2 \;\text{x1}+a^2 y+2 a b+b^2 \;\text{x2}+b^2 \;\text{y2}}{a+b+c^2+2 c d+d^2}
\end{dmath*}

\begin{Shaded}
\begin{Highlighting}[]
\NormalTok{Collect2}\OperatorTok{[}\NormalTok{ex}\OperatorTok{,} \FunctionTok{a}\OperatorTok{,} \FunctionTok{b}\OperatorTok{]}
\end{Highlighting}
\end{Shaded}

\begin{dmath*}\breakingcomma
\frac{a^2 (\text{x1}+y)}{a+b+c^2+2 c d+d^2}+\frac{b^2 (\text{x2}+\text{y2})}{a+b+c^2+2 c d+d^2}+\frac{2 a b}{a+b+c^2+2 c d+d^2}
\end{dmath*}

\begin{Shaded}
\begin{Highlighting}[]
\NormalTok{Collect2}\OperatorTok{[}\NormalTok{ex}\OperatorTok{,} \FunctionTok{a}\OperatorTok{,} \FunctionTok{b}\OperatorTok{,} \FunctionTok{Numerator} \OtherTok{{-}\textgreater{}} \ConstantTok{True}\OperatorTok{]}
\end{Highlighting}
\end{Shaded}

\begin{dmath*}\breakingcomma
\frac{a^2 (\text{x1}+y)+2 a b+b^2 (\text{x2}+\text{y2})}{a+b+c^2+2 c d+d^2}
\end{dmath*}

\begin{Shaded}
\begin{Highlighting}[]
\NormalTok{Collect2}\OperatorTok{[}\NormalTok{ex}\OperatorTok{,} \FunctionTok{a}\OperatorTok{,} \FunctionTok{b}\OperatorTok{,} \FunctionTok{Numerator} \OtherTok{{-}\textgreater{}} \ConstantTok{True}\OperatorTok{,}\NormalTok{ FactoringDenominator }\OtherTok{{-}\textgreater{}} \FunctionTok{Simplify}\OperatorTok{]}
\end{Highlighting}
\end{Shaded}

\begin{dmath*}\breakingcomma
\frac{a^2 (\text{x1}+y)+2 a b+b^2 (\text{x2}+\text{y2})}{a+b+(c+d)^2}
\end{dmath*}
\end{document}
