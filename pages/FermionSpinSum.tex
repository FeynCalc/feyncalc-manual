% !TeX program = pdflatex
% !TeX root = FermionSpinSum.tex

\documentclass[../FeynCalcManual.tex]{subfiles}
\begin{document}
\hypertarget{fermionspinsum}{%
\section{FermionSpinSum}\label{fermionspinsum}}

\texttt{FermionSpinSum[\allowbreak{}exp]} converts products of closed
spinor chains in \texttt{exp} into Dirac traces. Both Dirac and Majorana
particles are supported. It is understood, that \texttt{exp} represents
a squared amplitude.

\subsection{See also}

\hyperlink{toc}{Overview}, \hyperlink{spinor}{Spinor},
\hyperlink{complexconjugate}{ComplexConjugate},
\hyperlink{diractrace}{DiracTrace}.

\subsection{Examples}

FeynCalc uses the customary relativistic normalization of the spinors.

\begin{Shaded}
\begin{Highlighting}[]
\NormalTok{SpinorUBar}\OperatorTok{[}\NormalTok{Momentum}\OperatorTok{[}\FunctionTok{p}\OperatorTok{],} \FunctionTok{m}\OperatorTok{]}\NormalTok{ . SpinorU}\OperatorTok{[}\NormalTok{Momentum}\OperatorTok{[}\FunctionTok{p}\OperatorTok{],} \FunctionTok{m}\OperatorTok{]} 
 
\NormalTok{FermionSpinSum}\OperatorTok{[}\SpecialCharTok{\%}\OperatorTok{]} 
 
\NormalTok{DiracSimplify}\OperatorTok{[}\SpecialCharTok{\%}\OperatorTok{]}
\end{Highlighting}
\end{Shaded}

\begin{dmath*}\breakingcomma
\bar{u}\left(\overline{p},m\right).u\left(\overline{p},m\right)
\end{dmath*}

\begin{dmath*}\breakingcomma
\text{tr}\left(\bar{\gamma }\cdot \overline{p}+m\right)
\end{dmath*}

\begin{dmath*}\breakingcomma
4 m
\end{dmath*}

\begin{Shaded}
\begin{Highlighting}[]
\NormalTok{SpinorVBar}\OperatorTok{[}\NormalTok{Momentum}\OperatorTok{[}\FunctionTok{p}\OperatorTok{],} \FunctionTok{m}\OperatorTok{]}\NormalTok{ . SpinorV}\OperatorTok{[}\NormalTok{Momentum}\OperatorTok{[}\FunctionTok{p}\OperatorTok{],} \FunctionTok{m}\OperatorTok{]} 
 
\NormalTok{FermionSpinSum}\OperatorTok{[}\SpecialCharTok{\%}\OperatorTok{]} 
 
\NormalTok{DiracSimplify}\OperatorTok{[}\SpecialCharTok{\%}\OperatorTok{]}
\end{Highlighting}
\end{Shaded}

\begin{dmath*}\breakingcomma
\bar{v}\left(\overline{p},m\right).v\left(\overline{p},m\right)
\end{dmath*}

\begin{dmath*}\breakingcomma
\text{tr}\left(\bar{\gamma }\cdot \overline{p}-m\right)
\end{dmath*}

\begin{dmath*}\breakingcomma
-4 m
\end{dmath*}

\begin{Shaded}
\begin{Highlighting}[]
\NormalTok{amp }\ExtensionTok{=}\NormalTok{ SpinorUBar}\OperatorTok{[}\NormalTok{k1}\OperatorTok{,} \FunctionTok{m}\OperatorTok{]}\NormalTok{ . GS}\OperatorTok{[}\FunctionTok{p}\OperatorTok{]}\NormalTok{ . GA}\OperatorTok{[}\DecValTok{5}\OperatorTok{]}\NormalTok{ . SpinorU}\OperatorTok{[}\NormalTok{p1}\OperatorTok{,} \FunctionTok{m}\OperatorTok{]} 
 
\NormalTok{ampSq }\ExtensionTok{=}\NormalTok{ amp ComplexConjugate}\OperatorTok{[}\NormalTok{amp}\OperatorTok{]}
\end{Highlighting}
\end{Shaded}

\begin{dmath*}\breakingcomma
\bar{u}(\text{k1},m).\left(\bar{\gamma }\cdot \overline{p}\right).\bar{\gamma }^5.u(\text{p1},m)
\end{dmath*}

\begin{dmath*}\breakingcomma
\bar{u}(\text{k1},m).\left(\bar{\gamma }\cdot \overline{p}\right).\bar{\gamma }^5.u(\text{p1},m) \left(-\left(\varphi (\overline{\text{p1}},m)\right).\bar{\gamma }^5.\left(\bar{\gamma }\cdot \overline{p}\right).\left(\varphi (\overline{\text{k1}},m)\right)\right)
\end{dmath*}

\begin{Shaded}
\begin{Highlighting}[]
\NormalTok{FermionSpinSum}\OperatorTok{[}\NormalTok{ampSq}\OperatorTok{]} 
 
\NormalTok{DiracSimplify}\OperatorTok{[}\SpecialCharTok{\%}\OperatorTok{]}
\end{Highlighting}
\end{Shaded}

\begin{dmath*}\breakingcomma
-\text{tr}\left(\left(\bar{\gamma }\cdot \overline{\text{k1}}+m\right).\left(\bar{\gamma }\cdot \overline{p}\right).\bar{\gamma }^5.\left(\bar{\gamma }\cdot \overline{\text{p1}}+m\right).\bar{\gamma }^5.\left(\bar{\gamma }\cdot \overline{p}\right)\right)
\end{dmath*}

\begin{dmath*}\breakingcomma
-4 \overline{p}^2 \left(\overline{\text{k1}}\cdot \overline{\text{p1}}\right)+8 \left(\overline{\text{k1}}\cdot \overline{p}\right) \left(\overline{p}\cdot \overline{\text{p1}}\right)-4 m^2 \overline{p}^2
\end{dmath*}
\end{document}
