% !TeX program = pdflatex
% !TeX root = FeynAmp.tex

\documentclass[../FeynCalcManual.tex]{subfiles}
\begin{document}
\hypertarget{feynamp}{%
\section{FeynAmp}\label{feynamp}}

\texttt{FeynAmp[\allowbreak{}q,\ \allowbreak{}amp]} is the head of a
Feynman amplitude, where amp denotes the analytical expression for the
amplitude and q is the integration variable.
\texttt{FeynAmp[\allowbreak{}q1,\ \allowbreak{}q2,\ \allowbreak{}amp]}
denotes a two-loop amplitude. \texttt{FeynAmp} has no functional
properties and serves just as a head. There are however special
typesetting rules attached.

\subsection{See also}

\hyperlink{toc}{Overview}, \hyperlink{amplitude}{Amplitude}.

\subsection{Examples}

This is a 1-loop gluon self-energy amplitude (ignoring factors of
\(2 \pi\)).

\begin{Shaded}
\begin{Highlighting}[]
\NormalTok{FeynAmp}\OperatorTok{[}\FunctionTok{q}\OperatorTok{,}\NormalTok{ GV}\OperatorTok{[}\FunctionTok{p}\OperatorTok{,} \SpecialCharTok{\textbackslash{}}\OperatorTok{[}\NormalTok{Mu}\OperatorTok{],} \FunctionTok{a}\OperatorTok{,} \FunctionTok{q} \SpecialCharTok{{-}} \FunctionTok{p}\OperatorTok{,} \SpecialCharTok{\textbackslash{}}\OperatorTok{[}\NormalTok{Alpha}\OperatorTok{],} \FunctionTok{c}\OperatorTok{,} \SpecialCharTok{{-}}\FunctionTok{q}\OperatorTok{,} \SpecialCharTok{\textbackslash{}}\OperatorTok{[}\FunctionTok{Beta}\OperatorTok{],} \FunctionTok{e}\OperatorTok{]}\NormalTok{ GP}\OperatorTok{[}\FunctionTok{p} \SpecialCharTok{{-}} \FunctionTok{q}\OperatorTok{,} \SpecialCharTok{\textbackslash{}}\OperatorTok{[}\NormalTok{Alpha}\OperatorTok{],} \FunctionTok{c}\OperatorTok{,} \SpecialCharTok{\textbackslash{}}\OperatorTok{[}\NormalTok{Rho}\OperatorTok{],} \FunctionTok{d}\OperatorTok{]}\NormalTok{ GV}\OperatorTok{[}\SpecialCharTok{{-}}\FunctionTok{p}\OperatorTok{,} \SpecialCharTok{\textbackslash{}}\OperatorTok{[}\NormalTok{Nu}\OperatorTok{],} \FunctionTok{b}\OperatorTok{,} \FunctionTok{p} \SpecialCharTok{{-}} \FunctionTok{q}\OperatorTok{,} \SpecialCharTok{\textbackslash{}}\OperatorTok{[}\NormalTok{Rho}\OperatorTok{],} \FunctionTok{d}\OperatorTok{,} \FunctionTok{q}\OperatorTok{,} \SpecialCharTok{\textbackslash{}}\OperatorTok{[}\NormalTok{Sigma}\OperatorTok{],} \FunctionTok{f}\OperatorTok{]} \SpecialCharTok{*}
\NormalTok{   GP}\OperatorTok{[}\FunctionTok{q}\OperatorTok{,} \SpecialCharTok{\textbackslash{}}\OperatorTok{[}\FunctionTok{Beta}\OperatorTok{],} \FunctionTok{e}\OperatorTok{,} \SpecialCharTok{\textbackslash{}}\OperatorTok{[}\NormalTok{Sigma}\OperatorTok{],} \FunctionTok{f}\OperatorTok{]]}
\end{Highlighting}
\end{Shaded}

\begin{dmath*}\breakingcomma
\int d^Dq\left(f^{bdf} f^{ace} \Pi _{ef}^{\beta \sigma }(q) V^{\mu \alpha \beta }(p\text{, }q-p\text{, }-q) V^{\nu \rho \sigma }(-p\text{, }p-q\text{, }q) \Pi _{cd}^{\alpha \rho }(p-q)\right)
\end{dmath*}

This is a generic 2-loop amplitude.

\begin{Shaded}
\begin{Highlighting}[]
\NormalTok{FeynAmp}\OperatorTok{[}\FunctionTok{Subscript}\OperatorTok{[}\FunctionTok{q}\OperatorTok{,} \DecValTok{1}\OperatorTok{],} \FunctionTok{Subscript}\OperatorTok{[}\FunctionTok{q}\OperatorTok{,} \DecValTok{2}\OperatorTok{],}\NormalTok{ anyexpression}\OperatorTok{]}
\end{Highlighting}
\end{Shaded}

\begin{dmath*}\breakingcomma
\text{FeynAmp}\left(q_1,q_2,\text{anyexpression}\right)
\end{dmath*}
\end{document}
