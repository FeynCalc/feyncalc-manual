% !TeX program = pdflatex
% !TeX root = FeynAmpDenominator.tex

\documentclass[../FeynCalcManual.tex]{subfiles}
\begin{document}
\hypertarget{feynampdenominator}{
\section{FeynAmpDenominator}\label{feynampdenominator}\index{FeynAmpDenominator}}

\texttt{FeynAmpDenominator[\allowbreak{}...]} represents the inverse
denominators of the propagators,
i.e.~\texttt{FeynAmpDenominator[\allowbreak{}x]} is \(1/x\). Different
propagator denominators are represented using special heads such as
\texttt{PropagatorDenominator}, \texttt{StandardPropagatorDenominator},
\texttt{CartesianPropagatorDenominator} etc.

\subsection{See also}

\hyperlink{toc}{Overview}, \hyperlink{fad}{FAD}, \hyperlink{sfad}{SFAD},
\hyperlink{cfad}{CFAD}, \hyperlink{gfad}{GFAD},
\hyperlink{feynampdenominatorsimplify}{FeynAmpDenominatorSimplify}.

\subsection{Examples}

The legacy way to represent standard Lorentzian propagators is to use
\texttt{PropagatorDenominator}. Here the sign of the mass term is fixed
to be \(-1\) and no information on the \(i \eta\)- prescription is
available. Furthermore, this way it is not possible to enter eikonal
propagators

\begin{Shaded}
\begin{Highlighting}[]
\NormalTok{FeynAmpDenominator}\OperatorTok{[}\NormalTok{PropagatorDenominator}\OperatorTok{[}\NormalTok{Momentum}\OperatorTok{[}\FunctionTok{p}\OperatorTok{,} \FunctionTok{D}\OperatorTok{],} \FunctionTok{m}\OperatorTok{]]}
\end{Highlighting}
\end{Shaded}

\begin{dmath*}\breakingcomma
\frac{1}{p^2-m^2}
\end{dmath*}

\begin{Shaded}
\begin{Highlighting}[]
\NormalTok{FeynAmpDenominator}\OperatorTok{[}\NormalTok{PropagatorDenominator}\OperatorTok{[}\NormalTok{Momentum}\OperatorTok{[}\FunctionTok{p}\OperatorTok{,} \FunctionTok{D}\OperatorTok{],} \FunctionTok{m}\OperatorTok{],} 
\NormalTok{  PropagatorDenominator}\OperatorTok{[}\NormalTok{Momentum}\OperatorTok{[}\FunctionTok{p} \SpecialCharTok{{-}} \FunctionTok{q}\OperatorTok{,} \FunctionTok{D}\OperatorTok{],} \FunctionTok{m}\OperatorTok{]]}
\end{Highlighting}
\end{Shaded}

\begin{dmath*}\breakingcomma
\frac{1}{\left(p^2-m^2\right).\left((p-q)^2-m^2\right)}
\end{dmath*}

It is worth noting that the Euclidean mass dependence still can be
introduced via a trick where the mass symbol is multiplied by the
imaginary unit \(i\)

\begin{Shaded}
\begin{Highlighting}[]
\NormalTok{FeynAmpDenominator}\OperatorTok{[}\NormalTok{PropagatorDenominator}\OperatorTok{[}\NormalTok{Momentum}\OperatorTok{[}\FunctionTok{p}\OperatorTok{,} \FunctionTok{D}\OperatorTok{],} \FunctionTok{I} \FunctionTok{m}\OperatorTok{]]} 
 
\SpecialCharTok{\%} \SpecialCharTok{//}\NormalTok{ FeynAmpDenominatorExplicit}
\end{Highlighting}
\end{Shaded}

\begin{dmath*}\breakingcomma
\frac{1}{p^2--m^2}
\end{dmath*}

\begin{dmath*}\breakingcomma
\frac{1}{m^2+p^2}
\end{dmath*}

The shortcut to enter \texttt{FeynAmpDenominator}s with
\texttt{PropagatorDenominator}s is \texttt{FAD}

\begin{Shaded}
\begin{Highlighting}[]
\NormalTok{FAD}\OperatorTok{[}\FunctionTok{p}\OperatorTok{]}
\end{Highlighting}
\end{Shaded}

\begin{dmath*}\breakingcomma
\frac{1}{p^2}
\end{dmath*}

\begin{Shaded}
\begin{Highlighting}[]
\NormalTok{FAD}\OperatorTok{[\{}\FunctionTok{p}\OperatorTok{,} \FunctionTok{m}\OperatorTok{\}]}
\end{Highlighting}
\end{Shaded}

\begin{dmath*}\breakingcomma
\frac{1}{p^2-m^2}
\end{dmath*}

\begin{Shaded}
\begin{Highlighting}[]
\NormalTok{FAD}\OperatorTok{[\{}\FunctionTok{p}\OperatorTok{,} \FunctionTok{m}\OperatorTok{,} \DecValTok{3}\OperatorTok{\}]}
\end{Highlighting}
\end{Shaded}

\begin{dmath*}\breakingcomma
\frac{1}{\left(p^2-m^2\right)^3}
\end{dmath*}

\begin{Shaded}
\begin{Highlighting}[]
\NormalTok{FeynAmpDenominator}\OperatorTok{[}\NormalTok{PropagatorDenominator}\OperatorTok{[}\NormalTok{Momentum}\OperatorTok{[}\FunctionTok{p}\OperatorTok{,} \FunctionTok{D}\OperatorTok{],} \FunctionTok{m}\OperatorTok{]]} \SpecialCharTok{//}\NormalTok{ FCE }\SpecialCharTok{//} \FunctionTok{StandardForm}

\CommentTok{(*FAD[\{p, m\}]*)}
\end{Highlighting}
\end{Shaded}

Since version 9.3, a more flexible input is possible using
\texttt{StandardPropagatorDenominator}

\begin{Shaded}
\begin{Highlighting}[]
\NormalTok{FeynAmpDenominator}\OperatorTok{[}\NormalTok{StandardPropagatorDenominator}\OperatorTok{[}\NormalTok{Momentum}\OperatorTok{[}\FunctionTok{p}\OperatorTok{,} \FunctionTok{D}\OperatorTok{],} \DecValTok{0}\OperatorTok{,} \SpecialCharTok{{-}}\FunctionTok{m}\SpecialCharTok{\^{}}\DecValTok{2}\OperatorTok{,} \OperatorTok{\{}\DecValTok{1}\OperatorTok{,} \DecValTok{1}\OperatorTok{\}]]}
\end{Highlighting}
\end{Shaded}

\begin{dmath*}\breakingcomma
\frac{1}{(p^2-m^2+i \eta )}
\end{dmath*}

The mass term can be anything, as long as it does not depend on the loop
momenta

\begin{Shaded}
\begin{Highlighting}[]
\NormalTok{FeynAmpDenominator}\OperatorTok{[}\NormalTok{StandardPropagatorDenominator}\OperatorTok{[}\NormalTok{Momentum}\OperatorTok{[}\FunctionTok{p}\OperatorTok{,} \FunctionTok{D}\OperatorTok{],} \DecValTok{0}\OperatorTok{,} \FunctionTok{m}\SpecialCharTok{\^{}}\DecValTok{2}\OperatorTok{,} \OperatorTok{\{}\DecValTok{1}\OperatorTok{,} \DecValTok{1}\OperatorTok{\}]]}
\end{Highlighting}
\end{Shaded}

\begin{dmath*}\breakingcomma
\frac{1}{(p^2+m^2+i \eta )}
\end{dmath*}

\begin{Shaded}
\begin{Highlighting}[]
\NormalTok{FeynAmpDenominator}\OperatorTok{[}\NormalTok{StandardPropagatorDenominator}\OperatorTok{[}\NormalTok{Momentum}\OperatorTok{[}\FunctionTok{p}\OperatorTok{,} \FunctionTok{D}\OperatorTok{],} \DecValTok{0}\OperatorTok{,} \SpecialCharTok{{-}}\FunctionTok{m}\SpecialCharTok{\^{}}\DecValTok{2}\OperatorTok{,} \OperatorTok{\{}\DecValTok{1}\OperatorTok{,} \DecValTok{1}\OperatorTok{\}]]}
\end{Highlighting}
\end{Shaded}

\begin{dmath*}\breakingcomma
\frac{1}{(p^2-m^2+i \eta )}
\end{dmath*}

\begin{Shaded}
\begin{Highlighting}[]
\NormalTok{FeynAmpDenominator}\OperatorTok{[}\NormalTok{StandardPropagatorDenominator}\OperatorTok{[}\NormalTok{Momentum}\OperatorTok{[}\FunctionTok{p}\OperatorTok{,} \FunctionTok{D}\OperatorTok{],} \DecValTok{0}\OperatorTok{,}\NormalTok{ SPD}\OperatorTok{[}\FunctionTok{q}\OperatorTok{,} \FunctionTok{q}\OperatorTok{],} \OperatorTok{\{}\DecValTok{1}\OperatorTok{,} \DecValTok{1}\OperatorTok{\}]]}
\end{Highlighting}
\end{Shaded}

\begin{dmath*}\breakingcomma
\frac{1}{(p^2+q^2+i \eta )}
\end{dmath*}

One can also change the sign of \(i \eta\), although currently no
internal functions make use of it

\begin{Shaded}
\begin{Highlighting}[]
\NormalTok{FeynAmpDenominator}\OperatorTok{[}\NormalTok{StandardPropagatorDenominator}\OperatorTok{[}\NormalTok{Momentum}\OperatorTok{[}\FunctionTok{p}\OperatorTok{,} \FunctionTok{D}\OperatorTok{],} \DecValTok{0}\OperatorTok{,} \SpecialCharTok{{-}}\FunctionTok{m}\SpecialCharTok{\^{}}\DecValTok{2}\OperatorTok{,} \OperatorTok{\{}\DecValTok{1}\OperatorTok{,} \SpecialCharTok{{-}}\DecValTok{1}\OperatorTok{\}]]}
\end{Highlighting}
\end{Shaded}

\begin{dmath*}\breakingcomma
\frac{1}{(p^2-m^2-i \eta )}
\end{dmath*}

The propagator may also be raised to integer or symbolic powers

\begin{Shaded}
\begin{Highlighting}[]
\NormalTok{FeynAmpDenominator}\OperatorTok{[}\NormalTok{StandardPropagatorDenominator}\OperatorTok{[}\NormalTok{Momentum}\OperatorTok{[}\FunctionTok{p}\OperatorTok{,} \FunctionTok{D}\OperatorTok{],} \DecValTok{0}\OperatorTok{,} \FunctionTok{m}\SpecialCharTok{\^{}}\DecValTok{2}\OperatorTok{,} \OperatorTok{\{}\DecValTok{3}\OperatorTok{,} \DecValTok{1}\OperatorTok{\}]]}
\end{Highlighting}
\end{Shaded}

\begin{dmath*}\breakingcomma
\frac{1}{(p^2+m^2+i \eta )^3}
\end{dmath*}

\begin{Shaded}
\begin{Highlighting}[]
\NormalTok{FeynAmpDenominator}\OperatorTok{[}\NormalTok{StandardPropagatorDenominator}\OperatorTok{[}\NormalTok{Momentum}\OperatorTok{[}\FunctionTok{p}\OperatorTok{,} \FunctionTok{D}\OperatorTok{],} \DecValTok{0}\OperatorTok{,} \FunctionTok{m}\SpecialCharTok{\^{}}\DecValTok{2}\OperatorTok{,} \OperatorTok{\{}\SpecialCharTok{{-}}\DecValTok{2}\OperatorTok{,} \DecValTok{1}\OperatorTok{\}]]}
\end{Highlighting}
\end{Shaded}

\begin{dmath*}\breakingcomma
(p^2+m^2+i \eta )^2
\end{dmath*}

\begin{Shaded}
\begin{Highlighting}[]
\NormalTok{FeynAmpDenominator}\OperatorTok{[}\NormalTok{StandardPropagatorDenominator}\OperatorTok{[}\NormalTok{Momentum}\OperatorTok{[}\FunctionTok{p}\OperatorTok{,} \FunctionTok{D}\OperatorTok{],} \DecValTok{0}\OperatorTok{,} \FunctionTok{m}\SpecialCharTok{\^{}}\DecValTok{2}\OperatorTok{,} \OperatorTok{\{}\FunctionTok{n}\OperatorTok{,} \DecValTok{1}\OperatorTok{\}]]}
\end{Highlighting}
\end{Shaded}

\begin{dmath*}\breakingcomma
(p^2+m^2+i \eta )^{-n}
\end{dmath*}

Eikonal propagators are fully supported

\begin{Shaded}
\begin{Highlighting}[]
\NormalTok{FeynAmpDenominator}\OperatorTok{[}\NormalTok{StandardPropagatorDenominator}\OperatorTok{[}\DecValTok{0}\OperatorTok{,}\NormalTok{ Pair}\OperatorTok{[}\NormalTok{Momentum}\OperatorTok{[}\FunctionTok{p}\OperatorTok{,} \FunctionTok{D}\OperatorTok{],}\NormalTok{ Momentum}\OperatorTok{[}\FunctionTok{q}\OperatorTok{,} \FunctionTok{D}\OperatorTok{]],} 
   \SpecialCharTok{{-}}\FunctionTok{m}\SpecialCharTok{\^{}}\DecValTok{2}\OperatorTok{,} \OperatorTok{\{}\DecValTok{1}\OperatorTok{,} \DecValTok{1}\OperatorTok{\}]]}
\end{Highlighting}
\end{Shaded}

\begin{dmath*}\breakingcomma
\frac{1}{(p\cdot q-m^2+i \eta )}
\end{dmath*}

\begin{Shaded}
\begin{Highlighting}[]
\NormalTok{FeynAmpDenominator}\OperatorTok{[}\NormalTok{StandardPropagatorDenominator}\OperatorTok{[}\DecValTok{0}\OperatorTok{,}\NormalTok{ Pair}\OperatorTok{[}\NormalTok{Momentum}\OperatorTok{[}\FunctionTok{p}\OperatorTok{,} \FunctionTok{D}\OperatorTok{],}\NormalTok{ Momentum}\OperatorTok{[}\FunctionTok{q}\OperatorTok{,} \FunctionTok{D}\OperatorTok{]],} 
   \DecValTok{0}\OperatorTok{,} \OperatorTok{\{}\DecValTok{1}\OperatorTok{,} \DecValTok{1}\OperatorTok{\}]]}
\end{Highlighting}
\end{Shaded}

\begin{dmath*}\breakingcomma
\frac{1}{(p\cdot q+i \eta )}
\end{dmath*}

FeynCalc keeps trace of the signs of the scalar products in the eikonal
propagators. This is where the \(i \eta\)- prescription may come handy

\begin{Shaded}
\begin{Highlighting}[]
\NormalTok{FeynAmpDenominator}\OperatorTok{[}\NormalTok{StandardPropagatorDenominator}\OperatorTok{[}\DecValTok{0}\OperatorTok{,} \SpecialCharTok{{-}}\NormalTok{Pair}\OperatorTok{[}\NormalTok{Momentum}\OperatorTok{[}\FunctionTok{p}\OperatorTok{,} \FunctionTok{D}\OperatorTok{],}\NormalTok{ Momentum}\OperatorTok{[}\FunctionTok{q}\OperatorTok{,} \FunctionTok{D}\OperatorTok{]],} 
   \DecValTok{0}\OperatorTok{,} \OperatorTok{\{}\DecValTok{1}\OperatorTok{,} \DecValTok{1}\OperatorTok{\}]]}
\end{Highlighting}
\end{Shaded}

\begin{dmath*}\breakingcomma
\frac{1}{(-p\cdot q+i \eta )}
\end{dmath*}

\begin{Shaded}
\begin{Highlighting}[]
\NormalTok{FeynAmpDenominator}\OperatorTok{[}\NormalTok{StandardPropagatorDenominator}\OperatorTok{[}\DecValTok{0}\OperatorTok{,}\NormalTok{ Pair}\OperatorTok{[}\NormalTok{Momentum}\OperatorTok{[}\FunctionTok{p}\OperatorTok{,} \FunctionTok{D}\OperatorTok{],}\NormalTok{ Momentum}\OperatorTok{[}\FunctionTok{q}\OperatorTok{,} \FunctionTok{D}\OperatorTok{]],} 
   \DecValTok{0}\OperatorTok{,} \OperatorTok{\{}\DecValTok{1}\OperatorTok{,} \SpecialCharTok{{-}}\DecValTok{1}\OperatorTok{\}]]}
\end{Highlighting}
\end{Shaded}

\begin{dmath*}\breakingcomma
\frac{1}{(p\cdot q-i \eta )}
\end{dmath*}

The shortcut to enter \texttt{FeynAmpDenominators} with
\texttt{StandardPropagatorDenominators} is \texttt{SFAD}

\begin{Shaded}
\begin{Highlighting}[]
\NormalTok{FeynAmpDenominator}\OperatorTok{[}\NormalTok{StandardPropagatorDenominator}\OperatorTok{[}\NormalTok{Momentum}\OperatorTok{[}\FunctionTok{p}\OperatorTok{,} \FunctionTok{D}\OperatorTok{],} \DecValTok{0}\OperatorTok{,} \SpecialCharTok{{-}}\FunctionTok{m}\SpecialCharTok{\^{}}\DecValTok{2}\OperatorTok{,} \OperatorTok{\{}\DecValTok{1}\OperatorTok{,} \DecValTok{1}\OperatorTok{\}]]} \SpecialCharTok{//}\NormalTok{ FCE }\SpecialCharTok{//} 
  \FunctionTok{StandardForm}

\CommentTok{(*SFAD[\{\{p, 0\}, \{m\^{}2, 1\}, 1\}]*)}
\end{Highlighting}
\end{Shaded}

Eikonal propagators are entered using the \texttt{Dot} (\texttt{.}) as
in noncommutative products

\begin{Shaded}
\begin{Highlighting}[]
\NormalTok{FeynAmpDenominator}\OperatorTok{[}\NormalTok{StandardPropagatorDenominator}\OperatorTok{[}\DecValTok{0}\OperatorTok{,}\NormalTok{ Pair}\OperatorTok{[}\NormalTok{Momentum}\OperatorTok{[}\FunctionTok{p}\OperatorTok{,} \FunctionTok{D}\OperatorTok{],} 
\NormalTok{      Momentum}\OperatorTok{[}\FunctionTok{q}\OperatorTok{,} \FunctionTok{D}\OperatorTok{]],} \SpecialCharTok{{-}}\FunctionTok{m}\SpecialCharTok{\^{}}\DecValTok{2}\OperatorTok{,} \OperatorTok{\{}\DecValTok{1}\OperatorTok{,} \DecValTok{1}\OperatorTok{\}]]} \SpecialCharTok{//}\NormalTok{ FCE }\SpecialCharTok{//} \FunctionTok{StandardForm}

\CommentTok{(*SFAD[\{\{0, p . q\}, \{m\^{}2, 1\}, 1\}]*)}
\end{Highlighting}
\end{Shaded}

The Cartesian version of \texttt{StandardPropagatorDenominator} is
called \texttt{CartesianPropagatorDenominator}. The syntax is almost the
same as for \texttt{StandardPropagatorDenominator}, except that the
momenta and scalar products must be Cartesian.

\begin{Shaded}
\begin{Highlighting}[]
\NormalTok{FeynAmpDenominator}\OperatorTok{[}\NormalTok{CartesianPropagatorDenominator}\OperatorTok{[}\NormalTok{CartesianMomentum}\OperatorTok{[}\FunctionTok{p}\OperatorTok{,} \FunctionTok{D} \SpecialCharTok{{-}} \DecValTok{1}\OperatorTok{],} \DecValTok{0}\OperatorTok{,} \FunctionTok{m}\SpecialCharTok{\^{}}\DecValTok{2}\OperatorTok{,} 
   \OperatorTok{\{}\DecValTok{1}\OperatorTok{,} \SpecialCharTok{{-}}\DecValTok{1}\OperatorTok{\}]]}
\end{Highlighting}
\end{Shaded}

\begin{dmath*}\breakingcomma
\frac{1}{(p^2+m^2-i \eta )}
\end{dmath*}

\begin{Shaded}
\begin{Highlighting}[]
\NormalTok{FeynAmpDenominator}\OperatorTok{[}\NormalTok{CartesianPropagatorDenominator}\OperatorTok{[}\DecValTok{0}\OperatorTok{,}\NormalTok{ CartesianPair}\OperatorTok{[}\NormalTok{CartesianMomentum}\OperatorTok{[}\FunctionTok{p}\OperatorTok{,} 
     \FunctionTok{D} \SpecialCharTok{{-}} \DecValTok{1}\OperatorTok{],}\NormalTok{ CartesianMomentum}\OperatorTok{[}\FunctionTok{q}\OperatorTok{,} \FunctionTok{D} \SpecialCharTok{{-}} \DecValTok{1}\OperatorTok{]],} \FunctionTok{m}\SpecialCharTok{\^{}}\DecValTok{2}\OperatorTok{,} \OperatorTok{\{}\DecValTok{1}\OperatorTok{,} \SpecialCharTok{{-}}\DecValTok{1}\OperatorTok{\}]]}
\end{Highlighting}
\end{Shaded}

\begin{dmath*}\breakingcomma
\frac{1}{(p\cdot q+m^2-i \eta )}
\end{dmath*}

The shortcut to enter \texttt{FeynAmpDenominators} with
\texttt{CartesianPropagatorDenominators} is \texttt{CFAD}

\begin{Shaded}
\begin{Highlighting}[]
\NormalTok{FCE}\OperatorTok{[}\NormalTok{FeynAmpDenominator}\OperatorTok{[}\NormalTok{CartesianPropagatorDenominator}\OperatorTok{[}\NormalTok{CartesianMomentum}\OperatorTok{[}\FunctionTok{p}\OperatorTok{,} \FunctionTok{D} \SpecialCharTok{{-}} \DecValTok{1}\OperatorTok{],} \DecValTok{0}\OperatorTok{,} \FunctionTok{m}\SpecialCharTok{\^{}}\DecValTok{2}\OperatorTok{,}
     \OperatorTok{\{}\DecValTok{1}\OperatorTok{,} \SpecialCharTok{{-}}\DecValTok{1}\OperatorTok{\}]]]} \SpecialCharTok{//} \FunctionTok{StandardForm}

\CommentTok{(*CFAD[\{\{p, 0\}, \{m\^{}2, {-}1\}, 1\}]*)}
\end{Highlighting}
\end{Shaded}

To represent completely arbitrary propagators one can use
\texttt{GenericPropagatorDenominator}. However, one should keep in mind
that the number of useful manipulations and automatic simplifications
available for such propagators is very limited.

\begin{Shaded}
\begin{Highlighting}[]
\NormalTok{FeynAmpDenominator}\OperatorTok{[}\NormalTok{GenericPropagatorDenominator}\OperatorTok{[}\FunctionTok{x}\OperatorTok{,} \OperatorTok{\{}\DecValTok{1}\OperatorTok{,} \DecValTok{1}\OperatorTok{\}]]}
\end{Highlighting}
\end{Shaded}

\begin{dmath*}\breakingcomma
\frac{1}{(x+i \eta )}
\end{dmath*}

This is a nonlinear propagator that appears in the calculation of the
QCD Energy-Energy-Correlation function

\begin{Shaded}
\begin{Highlighting}[]
\NormalTok{FeynAmpDenominator}\OperatorTok{[}\NormalTok{GenericPropagatorDenominator}\OperatorTok{[}\DecValTok{2} \FunctionTok{z}\NormalTok{ Pair}\OperatorTok{[}\NormalTok{Momentum}\OperatorTok{[}\NormalTok{p1}\OperatorTok{,} \FunctionTok{D}\OperatorTok{],}\NormalTok{ Momentum}\OperatorTok{[}\FunctionTok{Q}\OperatorTok{,} 
       \FunctionTok{D}\OperatorTok{]]}\NormalTok{ Pair}\OperatorTok{[}\NormalTok{Momentum}\OperatorTok{[}\NormalTok{p2}\OperatorTok{,} \FunctionTok{D}\OperatorTok{],}\NormalTok{ Momentum}\OperatorTok{[}\FunctionTok{Q}\OperatorTok{,} \FunctionTok{D}\OperatorTok{]]} \SpecialCharTok{{-}}\NormalTok{ Pair}\OperatorTok{[}\NormalTok{Momentum}\OperatorTok{[}\NormalTok{p1}\OperatorTok{,} \FunctionTok{D}\OperatorTok{],}\NormalTok{ Momentum}\OperatorTok{[}\NormalTok{p2}\OperatorTok{,} \FunctionTok{D}\OperatorTok{]],} \OperatorTok{\{}\DecValTok{1}\OperatorTok{,} \DecValTok{1}\OperatorTok{\}]]}
\end{Highlighting}
\end{Shaded}

\begin{dmath*}\breakingcomma
\frac{1}{(2 z (\text{p1}\cdot Q) (\text{p2}\cdot Q)-\text{p1}\cdot \;\text{p2}+i \eta )}
\end{dmath*}

The shortcut to enter \texttt{FeynAmpDenominator}s with
\texttt{GenericPropagatorDenominator}s is \texttt{GFAD}

\begin{Shaded}
\begin{Highlighting}[]
\NormalTok{FeynAmpDenominator}\OperatorTok{[}\NormalTok{GenericPropagatorDenominator}\OperatorTok{[}\DecValTok{2} \FunctionTok{z}\NormalTok{ Pair}\OperatorTok{[}\NormalTok{Momentum}\OperatorTok{[}\NormalTok{p1}\OperatorTok{,} \FunctionTok{D}\OperatorTok{],}\NormalTok{ Momentum}\OperatorTok{[}\FunctionTok{Q}\OperatorTok{,} 
         \FunctionTok{D}\OperatorTok{]]}\NormalTok{ Pair}\OperatorTok{[}\NormalTok{Momentum}\OperatorTok{[}\NormalTok{p2}\OperatorTok{,} \FunctionTok{D}\OperatorTok{],}\NormalTok{ Momentum}\OperatorTok{[}\FunctionTok{Q}\OperatorTok{,} \FunctionTok{D}\OperatorTok{]]} \SpecialCharTok{{-}}\NormalTok{ Pair}\OperatorTok{[}\NormalTok{Momentum}\OperatorTok{[}\NormalTok{p1}\OperatorTok{,} \FunctionTok{D}\OperatorTok{],}\NormalTok{ Momentum}\OperatorTok{[}\NormalTok{p2}\OperatorTok{,} \FunctionTok{D}\OperatorTok{]],} \OperatorTok{\{}\DecValTok{1}\OperatorTok{,} \DecValTok{1}\OperatorTok{\}]]} \SpecialCharTok{//} 
\NormalTok{   FCE }\SpecialCharTok{//} \FunctionTok{StandardForm}

\CommentTok{(*GFAD[\{\{{-}SPD[p1, p2] + 2 z SPD[p1, Q] SPD[p2, Q], 1\}, 1\}]*)}
\end{Highlighting}
\end{Shaded}

\end{document}
