% !TeX program = pdflatex
% !TeX root = FeynAmpDenominatorCombine.tex

\documentclass[../FeynCalcManual.tex]{subfiles}
\begin{document}
\hypertarget{feynampdenominatorcombine}{%
\section{FeynAmpDenominatorCombine}\label{feynampdenominatorcombine}}

\texttt{FeynAmpDenominatorCombine[\allowbreak{}expr]} expands expr with
respect to \texttt{FeynAmpDenominator} and combines products of
\texttt{FeynAmpDenominator} in expr into one
\texttt{FeynAmpDenominator}.

\subsection{See also}

\hyperlink{toc}{Overview},
\hyperlink{feynampdenominatorsplit}{FeynAmpDenominatorSplit}.

\subsection{Examples}

\begin{Shaded}
\begin{Highlighting}[]
\NormalTok{FAD}\OperatorTok{[}\FunctionTok{q}\OperatorTok{]}\NormalTok{ FAD}\OperatorTok{[}\FunctionTok{q} \SpecialCharTok{{-}} \FunctionTok{p}\OperatorTok{]} 
 
\NormalTok{ex }\ExtensionTok{=}\NormalTok{ FeynAmpDenominatorCombine}\OperatorTok{[}\SpecialCharTok{\%}\OperatorTok{]}
\end{Highlighting}
\end{Shaded}

\begin{dmath*}\breakingcomma
\frac{1}{q^2 (q-p)^2}
\end{dmath*}

\begin{dmath*}\breakingcomma
\frac{1}{q^2.(q-p)^2}
\end{dmath*}

\begin{Shaded}
\begin{Highlighting}[]
\NormalTok{ex }\SpecialCharTok{//}\NormalTok{ FCE }\SpecialCharTok{//} \FunctionTok{StandardForm}

\CommentTok{(*FAD[q, {-}p + q]*)}
\end{Highlighting}
\end{Shaded}

\begin{Shaded}
\begin{Highlighting}[]
\NormalTok{ex2 }\ExtensionTok{=}\NormalTok{ FeynAmpDenominatorSplit}\OperatorTok{[}\NormalTok{ex}\OperatorTok{]}
\end{Highlighting}
\end{Shaded}

\begin{dmath*}\breakingcomma
\frac{1}{q^2 (q-p)^2}
\end{dmath*}

\begin{Shaded}
\begin{Highlighting}[]
\NormalTok{ex2 }\SpecialCharTok{//}\NormalTok{ FCE }\SpecialCharTok{//} \FunctionTok{StandardForm}

\CommentTok{(*FAD[q] FAD[{-}p + q]*)}
\end{Highlighting}
\end{Shaded}

\end{document}
