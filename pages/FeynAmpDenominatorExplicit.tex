% !TeX program = pdflatex
% !TeX root = FeynAmpDenominatorExplicit.tex

\documentclass[../FeynCalcManual.tex]{subfiles}
\begin{document}
\hypertarget{feynampdenominatorexplicit}{
\section{FeynAmpDenominatorExplicit}\label{feynampdenominatorexplicit}\index{FeynAmpDenominatorExplicit}}

\texttt{FeynAmpDenominatorExplicit[\allowbreak{}exp]} changes each
occurrence of
\texttt{PropagatorDenominator[\allowbreak{}a,\ \allowbreak{}b]} in exp
into \texttt{1/(SPD[\allowbreak{}a,\ \allowbreak{}a]-b^2)} and replaces
\texttt{FeynAmpDenominator} by \texttt{Identity}.

\subsection{See also}

\hyperlink{toc}{Overview},
\hyperlink{feynampdenominator}{FeynAmpDenominator},
\hyperlink{propagatordenominator}{PropagatorDenominator}.

\subsection{Examples}

\begin{Shaded}
\begin{Highlighting}[]
\NormalTok{FAD}\OperatorTok{[\{}\FunctionTok{q}\OperatorTok{,} \FunctionTok{m}\OperatorTok{\},} \OperatorTok{\{}\FunctionTok{q} \SpecialCharTok{{-}} \FunctionTok{p}\OperatorTok{,} \DecValTok{0}\OperatorTok{\}]} 
 
\NormalTok{FeynAmpDenominatorExplicit}\OperatorTok{[}\SpecialCharTok{\%}\OperatorTok{]} 
 
\SpecialCharTok{\%} \SpecialCharTok{//}\NormalTok{ FCE }\SpecialCharTok{//} \FunctionTok{StandardForm}
\end{Highlighting}
\end{Shaded}

\begin{dmath*}\breakingcomma
\frac{1}{\left(q^2-m^2\right).(q-p)^2}
\end{dmath*}

\begin{dmath*}\breakingcomma
\frac{1}{\left(q^2-m^2\right) \left(-2 (p\cdot q)+p^2+q^2\right)}
\end{dmath*}

\begin{dmath*}\breakingcomma
\frac{1}{\left(-m^2+\text{SPD}[q,q]\right) (\text{SPD}[p,p]-2 \;\text{SPD}[p,q]+\text{SPD}[q,q])}
\end{dmath*}

Notice that you should never apply \texttt{FeynAmpDenominatorExplicit}
to loop integrals. Denominators in a proper loop integral should be
written as \texttt{FeynAmpDenominator}s. Otherwise, the given integral
is assumed to have no denominators and consequently set to zero as being
scaleless.

\begin{Shaded}
\begin{Highlighting}[]
\NormalTok{TID}\OperatorTok{[}\NormalTok{FVD}\OperatorTok{[}\FunctionTok{q}\OperatorTok{,}\NormalTok{ mu}\OperatorTok{]}\NormalTok{ FAD}\OperatorTok{[\{}\FunctionTok{q}\OperatorTok{,} \FunctionTok{m}\OperatorTok{\},} \OperatorTok{\{}\FunctionTok{q} \SpecialCharTok{{-}} \FunctionTok{p}\OperatorTok{,} \DecValTok{0}\OperatorTok{\}],} \FunctionTok{q}\OperatorTok{]}
\end{Highlighting}
\end{Shaded}

\begin{dmath*}\breakingcomma
\frac{\left(m^2+p^2\right) p^{\text{mu}}}{2 p^2 q^2.\left((q-p)^2-m^2\right)}-\frac{p^{\text{mu}}}{2 p^2 \left(q^2-m^2\right)}
\end{dmath*}

\begin{Shaded}
\begin{Highlighting}[]
\NormalTok{TID}\OperatorTok{[}\NormalTok{FeynAmpDenominatorExplicit}\OperatorTok{[}\NormalTok{FVD}\OperatorTok{[}\FunctionTok{q}\OperatorTok{,}\NormalTok{ mu}\OperatorTok{]}\NormalTok{ FAD}\OperatorTok{[\{}\FunctionTok{q}\OperatorTok{,} \FunctionTok{m}\OperatorTok{\},} \OperatorTok{\{}\FunctionTok{q} \SpecialCharTok{{-}} \FunctionTok{p}\OperatorTok{,} \DecValTok{0}\OperatorTok{\}]],} \FunctionTok{q}\OperatorTok{]}
\end{Highlighting}
\end{Shaded}

\begin{dmath*}\breakingcomma
0
\end{dmath*}
\end{document}
