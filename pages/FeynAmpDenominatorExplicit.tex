% !TeX program = pdflatex
% !TeX root = FeynAmpDenominatorExplicit.tex

\documentclass[../FeynCalcManual.tex]{subfiles}
\begin{document}
\hypertarget{feynampdenominatorexplicit}{%
\section{FeynAmpDenominatorExplicit}\label{feynampdenominatorexplicit}}

\texttt{FeynAmpDenominatorExplicit[\allowbreak{}exp]} changes each
occurence of
\texttt{PropagatorDenominator[\allowbreak{}a,\ \allowbreak{}b]} in exp
into \texttt{1/(SPD[\allowbreak{}a,\ \allowbreak{}a]-b^2)} and replaces
\texttt{FeynAmpDenominator} by \texttt{Identity}.

\subsection{See also}

\hyperlink{toc}{Overview},
\hyperlink{feynampdenominator}{FeynAmpDenominator},
\hyperlink{propagatordenominator}{PropagatorDenominator}.

\subsection{Examples}

\begin{Shaded}
\begin{Highlighting}[]
\NormalTok{FAD}\OperatorTok{[\{}\FunctionTok{q}\OperatorTok{,} \FunctionTok{m}\OperatorTok{\},} \OperatorTok{\{}\FunctionTok{q} \SpecialCharTok{{-}} \FunctionTok{p}\OperatorTok{,} \DecValTok{0}\OperatorTok{\}]} 
 
\NormalTok{FeynAmpDenominatorExplicit}\OperatorTok{[}\SpecialCharTok{\%}\OperatorTok{]} 
 
\SpecialCharTok{\%} \SpecialCharTok{//}\NormalTok{ FCE }\SpecialCharTok{//} \FunctionTok{StandardForm}
\end{Highlighting}
\end{Shaded}

\begin{dmath*}\breakingcomma
\frac{1}{\left(q^2-m^2\right).(q-p)^2}
\end{dmath*}

\begin{dmath*}\breakingcomma
\frac{1}{\left(q^2-m^2\right) \left(-2 (p\cdot q)+p^2+q^2\right)}
\end{dmath*}

\begin{dmath*}\breakingcomma
\frac{1}{\left(-m^2+\text{SPD}[q,q]\right) (\text{SPD}[p,p]-2 \;\text{SPD}[p,q]+\text{SPD}[q,q])}
\end{dmath*}
\end{document}
