% !TeX program = pdflatex
% !TeX root = FeynAmpDenominatorSimplify.tex

\documentclass[../FeynCalcManual.tex]{subfiles}
\begin{document}
\hypertarget{feynampdenominatorsimplify}{%
\section{FeynAmpDenominatorSimplify}\label{feynampdenominatorsimplify}}

\texttt{FeynAmpDenominatorSimplify[\allowbreak{}exp]} tries to simplify
each \texttt{PropagatorDenominator} in a canonical way.
\texttt{FeynAmpDenominatorSimplify[\allowbreak{}exp,\ \allowbreak{}q1]}
simplifies all \texttt{FeynAmpDenominator}s in \texttt{exp} in a
canonical way, including momentum shifts. Scaleless integrals are
discarded.

\subsection{See also}

\hyperlink{toc}{Overview}, \hyperlink{tid}{TID}.

\subsection{Examples}

\begin{Shaded}
\begin{Highlighting}[]
\NormalTok{FDS}
\end{Highlighting}
\end{Shaded}

\begin{dmath*}\breakingcomma
\text{FeynAmpDenominatorSimplify}
\end{dmath*}

The cornerstone of dimensional regularization is that
\(\int d^n k f(k)/k^4 = 0\)

\begin{Shaded}
\begin{Highlighting}[]
\NormalTok{FeynAmpDenominatorSimplify}\OperatorTok{[}\FunctionTok{f}\OperatorTok{[}\FunctionTok{k}\OperatorTok{]}\NormalTok{ FAD}\OperatorTok{[}\FunctionTok{k}\OperatorTok{,} \FunctionTok{k}\OperatorTok{],} \FunctionTok{k}\OperatorTok{]}
\end{Highlighting}
\end{Shaded}

\begin{dmath*}\breakingcomma
0
\end{dmath*}

This brings some loop integrals into a standard form.

\begin{Shaded}
\begin{Highlighting}[]
\NormalTok{FeynAmpDenominatorSimplify}\OperatorTok{[}\NormalTok{FAD}\OperatorTok{[}\FunctionTok{k} \SpecialCharTok{{-}} \FunctionTok{Subscript}\OperatorTok{[}\FunctionTok{p}\OperatorTok{,} \DecValTok{1}\OperatorTok{],} \FunctionTok{k} \SpecialCharTok{{-}} \FunctionTok{Subscript}\OperatorTok{[}\FunctionTok{p}\OperatorTok{,} \DecValTok{2}\OperatorTok{]],} \FunctionTok{k}\OperatorTok{]}
\end{Highlighting}
\end{Shaded}

\begin{dmath*}\breakingcomma
\frac{1}{k^2.(k-p_1+p_2){}^2}
\end{dmath*}

\begin{Shaded}
\begin{Highlighting}[]
\NormalTok{FeynAmpDenominatorSimplify}\OperatorTok{[}\NormalTok{FAD}\OperatorTok{[}\FunctionTok{k}\OperatorTok{,} \FunctionTok{k}\OperatorTok{,} \FunctionTok{k} \SpecialCharTok{{-}} \FunctionTok{q}\OperatorTok{],} \FunctionTok{k}\OperatorTok{]}
\end{Highlighting}
\end{Shaded}

\begin{dmath*}\breakingcomma
\frac{1}{\left(k^2\right)^2.(k-q)^2}
\end{dmath*}

\begin{Shaded}
\begin{Highlighting}[]
\NormalTok{FeynAmpDenominatorSimplify}\OperatorTok{[}\FunctionTok{f}\OperatorTok{[}\FunctionTok{k}\OperatorTok{]}\NormalTok{ FAD}\OperatorTok{[}\FunctionTok{k}\OperatorTok{,} \FunctionTok{k} \SpecialCharTok{{-}} \FunctionTok{q}\OperatorTok{,} \FunctionTok{k} \SpecialCharTok{{-}} \FunctionTok{q}\OperatorTok{],} \FunctionTok{k}\OperatorTok{]}
\end{Highlighting}
\end{Shaded}

\begin{dmath*}\breakingcomma
\frac{f(q-k)}{\left(k^2\right)^2.(k-q)^2}
\end{dmath*}

\begin{Shaded}
\begin{Highlighting}[]
\NormalTok{FeynAmpDenominatorSimplify}\OperatorTok{[}\NormalTok{FAD}\OperatorTok{[}\FunctionTok{k} \SpecialCharTok{{-}} \FunctionTok{Subscript}\OperatorTok{[}\FunctionTok{p}\OperatorTok{,} \DecValTok{1}\OperatorTok{],} \FunctionTok{k} \SpecialCharTok{{-}} \FunctionTok{Subscript}\OperatorTok{[}\FunctionTok{p}\OperatorTok{,} \DecValTok{2}\OperatorTok{]]}\NormalTok{ SPD}\OperatorTok{[}\FunctionTok{k}\OperatorTok{,} \FunctionTok{k}\OperatorTok{],} \FunctionTok{k}\OperatorTok{]} 
 
\NormalTok{ApartFF}\OperatorTok{[}\SpecialCharTok{\%}\OperatorTok{,} \OperatorTok{\{}\FunctionTok{k}\OperatorTok{\}]} 
 
\NormalTok{TID}\OperatorTok{[}\SpecialCharTok{\%}\OperatorTok{,} \FunctionTok{k}\OperatorTok{]} \SpecialCharTok{//}\NormalTok{ Factor2}
\end{Highlighting}
\end{Shaded}

\begin{dmath*}\breakingcomma
\frac{2 \left(k\cdot p_2\right)+k^2+p_2{}^2}{k^2.(k-p_1+p_2){}^2}
\end{dmath*}

\begin{dmath*}\breakingcomma
\frac{2 \left(k\cdot p_2\right)+p_2{}^2}{k^2.(k-p_1+p_2){}^2}
\end{dmath*}

\begin{dmath*}\breakingcomma
\frac{p_1\cdot p_2}{k^2.(k-p_1+p_2){}^2}
\end{dmath*}

\begin{Shaded}
\begin{Highlighting}[]
\NormalTok{FDS}\OperatorTok{[}\NormalTok{FAD}\OperatorTok{[}\FunctionTok{k} \SpecialCharTok{{-}}\NormalTok{ p1}\OperatorTok{,} \FunctionTok{k} \SpecialCharTok{{-}}\NormalTok{ p2}\OperatorTok{]}\NormalTok{ SPD}\OperatorTok{[}\FunctionTok{k}\OperatorTok{,}\NormalTok{ OPEDelta}\OperatorTok{]}\SpecialCharTok{\^{}}\DecValTok{2}\OperatorTok{,} \FunctionTok{k}\OperatorTok{]}
\end{Highlighting}
\end{Shaded}

\begin{dmath*}\breakingcomma
\frac{(k\cdot \Delta +\Delta \cdot \;\text{p2})^2}{k^2.(k-\text{p1}+\text{p2})^2}
\end{dmath*}
\end{document}
