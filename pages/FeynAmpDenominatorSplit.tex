% !TeX program = pdflatex
% !TeX root = FeynAmpDenominatorSplit.tex

\documentclass[../FeynCalcManual.tex]{subfiles}
\begin{document}
\hypertarget{feynampdenominatorsplit}{%
\section{FeynAmpDenominatorSplit}\label{feynampdenominatorsplit}}

\texttt{FeynAmpDenominatorSplit[\allowbreak{}expr]} splits all
\texttt{FeynAmpDenominator[\allowbreak{}a,\ \allowbreak{}b,\ \allowbreak{}...]}
in \texttt{expr} into
\texttt{FeynAmpDenominator[\allowbreak{}a]*FeynAmpDenominator[\allowbreak{}b]*...}
.
\texttt{FeynAmpDenominatorSplit[\allowbreak{}expr,\ \allowbreak{} Momentum ->q1]}
splits all \texttt{FeynAmpDenominator} in expr into two products, one
containing \texttt{q1} and other momenta, the second being free of
\texttt{q1}.

\subsection{See also}

\hyperlink{toc}{Overview},
\hyperlink{feynampdenominatorcombine}{FeynAmpDenominatorCombine}.

\subsection{Examples}

\begin{Shaded}
\begin{Highlighting}[]
\NormalTok{FAD}\OperatorTok{[}\NormalTok{q1}\OperatorTok{,}\NormalTok{ q1 }\SpecialCharTok{{-}} \FunctionTok{p}\OperatorTok{,}\NormalTok{ q1 }\SpecialCharTok{{-}}\NormalTok{ q2}\OperatorTok{,}\NormalTok{ q2}\OperatorTok{,}\NormalTok{ q2 }\SpecialCharTok{{-}} \FunctionTok{p}\OperatorTok{]} 
 
\NormalTok{ex }\ExtensionTok{=}\NormalTok{ FeynAmpDenominatorSplit}\OperatorTok{[}\SpecialCharTok{\%}\OperatorTok{]}
\end{Highlighting}
\end{Shaded}

\begin{dmath*}\breakingcomma
\frac{1}{\text{q1}^2.(\text{q1}-p)^2.(\text{q1}-\text{q2})^2.\text{q2}^2.(\text{q2}-p)^2}
\end{dmath*}

\begin{dmath*}\breakingcomma
\frac{1}{\text{q1}^2 \;\text{q2}^2 (\text{q1}-p)^2 (\text{q2}-p)^2 (\text{q1}-\text{q2})^2}
\end{dmath*}

\begin{Shaded}
\begin{Highlighting}[]
\NormalTok{ex }\SpecialCharTok{//}\NormalTok{ FCE }\SpecialCharTok{//} \FunctionTok{StandardForm}

\CommentTok{(*FAD[q1] FAD[{-}p + q1] FAD[q1 {-} q2] FAD[q2] FAD[{-}p + q2]*)}
\end{Highlighting}
\end{Shaded}

\begin{Shaded}
\begin{Highlighting}[]
\NormalTok{ex }\ExtensionTok{=}\NormalTok{ FeynAmpDenominatorSplit}\OperatorTok{[}\NormalTok{FAD}\OperatorTok{[}\NormalTok{q1}\OperatorTok{,}\NormalTok{ q1 }\SpecialCharTok{{-}} \FunctionTok{p}\OperatorTok{,}\NormalTok{ q1 }\SpecialCharTok{{-}}\NormalTok{ q2}\OperatorTok{,}\NormalTok{ q2}\OperatorTok{,}\NormalTok{ q2 }\SpecialCharTok{{-}} \FunctionTok{p}\OperatorTok{],}\NormalTok{ Momentum }\OtherTok{{-}\textgreater{}} \OperatorTok{\{}\NormalTok{q1}\OperatorTok{\}]}
\end{Highlighting}
\end{Shaded}

\begin{dmath*}\breakingcomma
\frac{1}{\text{q2}^2.(\text{q2}-p)^2 \;\text{q1}^2.(\text{q1}-p)^2.(\text{q1}-\text{q2})^2}
\end{dmath*}

\begin{Shaded}
\begin{Highlighting}[]
\NormalTok{ex }\SpecialCharTok{//}\NormalTok{ FCE }\SpecialCharTok{//} \FunctionTok{StandardForm}

\CommentTok{(*FAD[q2, {-}p + q2] FAD[q1, {-}p + q1, q1 {-} q2]*)}
\end{Highlighting}
\end{Shaded}

\begin{Shaded}
\begin{Highlighting}[]
\NormalTok{FeynAmpDenominatorCombine}\OperatorTok{[}\NormalTok{ex}\OperatorTok{]} \SpecialCharTok{//}\NormalTok{ FCE }\SpecialCharTok{//} \FunctionTok{StandardForm}

\CommentTok{(*FAD[q1, q2, q1 {-} q2, {-}p + q1, {-}p + q2]*)}
\end{Highlighting}
\end{Shaded}

\end{document}
