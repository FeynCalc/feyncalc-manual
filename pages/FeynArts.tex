% !TeX program = pdflatex
% !TeX root = FeynArts.tex

\documentclass[../FeynCalcManual.tex]{subfiles}
\begin{document}
\hypertarget{feynarts}{
\section{FeynArts}\label{feynarts}\index{FeynArts}}

\subsection{See also}

\hyperlink{toc}{Overview}.

\subsection{What is FeynArts?}\label{what-is-feynarts}

FeynArts is a Mathematica package for generating Feynman diagrams and
the corresponding amplitudes. The original FeynArts was created by J.
Küblbeck, M. Böhm and A. Denner in 1990
(\href{http://inspirehep.net/record/27276}{INSPIRE}). Since 1998 it is
developed further by Thomas Hahn
(\href{http://arxiv.org/abs/hep-ph/0012260}{hep-ph/0012260}). For more
information about FeynArts please visit the
\href{http://www.feynarts.de/}{official site}. The manual is available
\href{http://www.feynarts.de/FA3Guide.pdf}{here}.

\subsection{Using FeynArts with
FeynCalc}\label{using-feynarts-with-feyncalc}

FeynArts is not a part of FeynCalc but its output can be used by
FeynCalc to evaluate the generated amplitudes. Unfortunately, many
FeynArts functions have the same name as the FeynCalc functions which
makes Mathematica produce lots of warnings when loading both packages in
the same session.

One possible workaround is to first generate the amplitudes with
FeynArts, then save them in a notebook, quit Mathematica, open the
notebook and only then load FeynCalc and evaluate the amplitudes.
However, this method is rather inconvenient if one wants to play with
different options and see how this affects the final result.

The preferred way of using FeynArts with FeynCalc is to patch FeynArts,
such that all corresponding FeynArts functions are renamed and no
shadowing occurs. In this case one can use FeynArts and FeynCalc in the
same Mathematica session without any unwanted interference effects.

\subsection{Patching FeynArts for
FeynCalc}\label{patching-feynarts-for-feyncalc}

If you install the stable or development version of FeynCalc using the
\href{https://github.com/FeynCalc/feyncalc/wiki/Installation}{automatic
installer}, you will be asked if the latest version of FeynArts should
be downloaded and patched. Therefore, no additional steps are necessary.

However, it may happen that you want to update your version of FeynArts
without resintalling FeynCalc. In this case follow these steps:

\begin{itemize}
\item
  Download the \href{http://www.feynarts.de/}{latest version} of
  FeynArts and extract the tarball into

\begin{Shaded}
\begin{Highlighting}[]
\NormalTok{\textless{}\textless{} FeynCalc\textasciigrave{}}
\FunctionTok{Print}\OperatorTok{[}\NormalTok{$FeynArtsDirectory}\OperatorTok{]}
\end{Highlighting}
\end{Shaded}
\item
  Start Mathematica and type

\begin{Shaded}
\begin{Highlighting}[]
\NormalTok{$LoadFeynArts }\ExtensionTok{=} \ConstantTok{True}\NormalTok{;}
\NormalTok{\textless{}\textless{}FeynCalc\textasciigrave{}; }
\end{Highlighting}
\end{Shaded}
\item
  A dialog asking if you want to patch FeynArts will appear. Hit OK.
  Wait until the patching process finishes.
\item
  Restart Mathematica kernel and try to evaluate some example codes
  (click on the examples link in the banner that appears when FeynCalc
  loads). Make sure that everything correctly evaluates without any
  warnings and errors.
\end{itemize}

\subsection{Using patched and unpatched FeynArts on the same
system}\label{using-patched-and-unpatched-feynarts-on-the-same-system}

If you patched FeynArts for FeynCalc but also want to use the unpatched
version (e.g.~for FormCalc), you can do so without any problems. When
you issue

\begin{Shaded}
\begin{Highlighting}[]
\NormalTok{    \textless{}\textless{} FeynArts\textasciigrave{}}
\end{Highlighting}
\end{Shaded}

Mathematica will load the original FeynArts version from your
\texttt{FileNameJoin[\allowbreak{}\{\allowbreak{}\$UserBaseDirectory,\ \allowbreak{}"Applications"\}]}
directory. The version patched for FeynCalc resides inside the
\texttt{\$FeynArtsDirectory} directory and is loaded only from FeynCalc.
Furthermore, you can also use Thomas Hahn's FeynInstall script to update
the original FeynArts, FormCalc and LoopTools packages without worrying
that it will damage the patched FeynArts version. Just make sure that
you never load patched FeynArts+FeynCalc and unpatched FeynArts+FormCalc
in the same Mathematica session.

\subsection{Technical details}\label{technical-details}

The patch is applied by the \texttt{FAPatch} function. Have a look at
the
\href{https://github.com/FeynCalc/feyncalc/blob/master/FeynCalc/Feynman/FAPatch.m}{source
code} if you want to know how it works.

\subsection{Evaluating the output of FeynArts in
FeynCalc}\label{evaluating-the-output-of-feynarts-in-feyncalc}

In general, the amplitudes produced by FeynArts cannot be directly
evaluated by FeynCalc. However, it is always possible to convert the
FeynArts output into the form required by FeynCalc.

In FeynCalc 9.x or later a new function \texttt{FCFAConvert}
significantly facilitates the conversion by doing the most common
replacements. The syntax is briefly described in the section 3.5 of
\href{https://arxiv.org/pdf/1601.01167.pdf}{arXiv:1601.01167}. Many the
example files that come with FeynCalc also use this function.

\subsection{The feynarts-mirror
repository}\label{the-feynarts-mirror-repository}

The website of FeynArts doesn't provide a public repository and uses a
special versioning system, where the version number is upped only for
new features but not for bugfixes. Since the FeynCalc developers are
highly interested in maintaining some level of compatibility between
FeynCalc and FeynArts, we created a Git repository
\href{https://github.com/FeynCalc/feynarts-mirror}{here} on GitHub to
monitor the changes between different FeynArts releases. Except for the
\emph{Readme.md} file, the content of that repository should be
identical to the content of the most recent FeynArts tarball.
\end{document}
