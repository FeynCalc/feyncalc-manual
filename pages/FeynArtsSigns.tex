% !TeX program = pdflatex
% !TeX root = FeynArtsSigns.tex

\documentclass[../FeynCalcManual.tex]{subfiles}
\begin{document}
\hypertarget{feynarts sign conventions}{
\section{FeynArts sign conventions}\label{feynarts sign conventions}\index{FeynArts sign conventions}}

\subsection{See also}

\hyperlink{toc}{Overview}.

The overall sign of an amplitude \(i \mathcal{M}\) is always a
convention. It does not have to agree between different textbooks and
papers and there is nothing wrong with that. Nevertheless, it should be
always possible to trace the origins of various signs appearing in
different pieces of the amplitude. Understanding how the overall sign
comes about allows you to \emph{adjust} it to the convention you prefer.

The amplitude generated by FeynArts' \texttt{CreateFeynAmp} function has
three sources of signs that contribute to the final overall sign. These
are the overall prefactor, the definition of the vertices and the
fermion sign.

\hypertarget{overall-prefactor}{%
\subsection{Overall prefactor}\label{overall-prefactor}}

\begin{itemize}
\tightlist
\item
  \texttt{CreateFeynAmp} generates \(i \mathcal{M}\) multiplied by the
  value of the option \texttt{PreFactor}.
\item
  The default setting of this option is
  \texttt{-I*(2Pi)^(-4 LoopNumber)}.
\item
  This means that for a tree-level diagram you get \(\mathcal{M}\) and
  for a 1-loop diagram \((2\pi)^{-4} \mathcal{M}\) is returned.
\item
  To obtain \(i \mathcal{M}\) change the value of \texttt{PreFactor} to
  \texttt{1} as in
  \texttt{CreateFeynAmp[\allowbreak{}diags,\ \allowbreak{}PreFactor -> 1]}.
\end{itemize}

\hypertarget{vertices}{%
\subsection{Vertices}\label{vertices}}

The signs in the vertices always depend on the model under
consideration. The built-in \texttt{SM} model is based on the
conventions used in
\href{https://arxiv.org/abs/0709.1075v1}{arXiv:0709.1075}. This means
that

\begin{itemize}
\tightlist
\item
  the lepton-gauge boson vertex (e.g.~\emph{QED electron-photon vertex})
  is proportional to \(i e \gamma^\mu\)
\item
  the quark-gauge boson vertex is proportional to
  \(- i Q g_c \gamma^\mu\), where \(g_c\) is the coupling constant and
  \(Q\) is thee electric charge of the quark
\item
  for charged gauge bosons (\(\gamma\), \(Z\), \(W^{\pm}\)) we have
  \(Q = 2/3\) for the up-type quarks (\(u\),\(c\),\(t\)) and \(-1/3\)
  for the down-type quarks (\(d\),\(s\),\(b\))
\item
  for gluons one sets \(Q=1\) and so the \emph{QCD quark-gluon vertex}
  corresponds to \(- i g_s \gamma^\mu\)
\end{itemize}

This FeynArts vertex convention agrees with the one used in
\href{https://doi.org/10.1007/978-3-322-80160-9}{Gauge Theories of the
Strong and Electroweak Interaction} by M. Bohm, A. Denner and H. Joos.
However, it disagrees with the convention in
\href{https://doi.org/10.1201/9780429503559}{An Introduction to Quantum
Field Theory} by M. Peskin and D. Schroeder. There the QED
electron-photon vertex is proportional to \(- i e \gamma^\mu\), the QCD
quark-gluon vertex amounts to \(i g_s \gamma^\mu\) and the quark-photon
vertex yields \(i Q g_s \gamma^\mu\).

Notice also that models generated with FeynRules will agree with
FeynArts on the QED vertex convention but disagree on the sign of the
QCD vertex.

\hypertarget{fermion-sign}{%
\subsection{Fermion sign}\label{fermion-sign}}

Diagrams with external fermions receive additional signs that stem from
the anticommuting properties of Grassmann fields when applying Wick's
theorem. One of the simplest processes that exhibits this effect is the
tree-level QED
\href{https://en.wikipedia.org/wiki/Bhabha_scattering}{Bhabha
scattering} \(e^+ e^- \to e^+ e^-\). There are no ambiguities regarding
the fact that both amplitudes have a relative minus sign. However, we
are free to choose which of the two amplitudes should be multiplied by
\texttt{+1} and which by \texttt{-1}.

The fermion sign algorithm implemented in FeynArts
(\href{https://inspirehep.net/literature/336411}{flip rules}) is
described in Section 6.6 of the
\href{http://www.feynarts.de/FA3Guide.pdf}{program manual}. In the case
of the process \(e^- e^+ \to e^- e^+\) the \(s\)-channel diagram is
multiplied by \(-1\), while the \(t\)-channel diagram receives a
prefactor of \(+1\). However, if one generates the physically equivalent
process \(e^- e^+ \to e^+ e^-\), the signs will flip due to the reversed
ordering of the final state particles.

Since March 2022 FeynArts features an option to make the fermion sign of
each diagram explicit. To this end you just need to evaluate
\texttt{FermionSign = fSign;} before generating the diagrams. Here
\texttt{fSign} is a head that will be wrapped around fermion signs. In
the case of the Bhabha scattering you can explicitly see that

\begin{Shaded}
\begin{Highlighting}[]
\NormalTok{diagsV1 }\ExtensionTok{=}\NormalTok{   InsertFields}\OperatorTok{[}\NormalTok{CreateTopologies}\OperatorTok{[}\DecValTok{0}\OperatorTok{,} \DecValTok{2} \OtherTok{{-}\textgreater{}} \DecValTok{2}\OperatorTok{],} \OperatorTok{\{}\FunctionTok{F}\OperatorTok{[}\DecValTok{2}\OperatorTok{,} \OperatorTok{\{}\DecValTok{1}\OperatorTok{\}],} \SpecialCharTok{{-}}\FunctionTok{F}\OperatorTok{[}\DecValTok{2}\OperatorTok{,} \OperatorTok{\{}\DecValTok{1}\OperatorTok{\}]\}} \OtherTok{{-}\textgreater{}} 
\OperatorTok{\{}\FunctionTok{F}\OperatorTok{[}\DecValTok{2}\OperatorTok{,} \OperatorTok{\{}\DecValTok{2}\OperatorTok{\}],} \SpecialCharTok{{-}}\FunctionTok{F}\OperatorTok{[}\DecValTok{2}\OperatorTok{,} \OperatorTok{\{}\DecValTok{2}\OperatorTok{\}]\},}\NormalTok{ InsertionLevel }\OtherTok{{-}\textgreater{}} \OperatorTok{\{}\NormalTok{Classes}\OperatorTok{\},}\NormalTok{ Restrictions }\OtherTok{{-}\textgreater{}}\NormalTok{ QEDOnly}\OperatorTok{]}\NormalTok{;}
\NormalTok{CreateFeynAmp}\OperatorTok{[}\NormalTok{diagsV1}\OperatorTok{,}\NormalTok{PreFactor}\OtherTok{{-}\textgreater{}}\DecValTok{1}\OperatorTok{]}
\end{Highlighting}
\end{Shaded}

contains \texttt{fSign[\allowbreak{}-1]}, while

\begin{Shaded}
\begin{Highlighting}[]
\NormalTok{diagsV2 }\ExtensionTok{=}\NormalTok{   InsertFields}\OperatorTok{[}\NormalTok{CreateTopologies}\OperatorTok{[}\DecValTok{0}\OperatorTok{,} \DecValTok{2} \OtherTok{{-}\textgreater{}} \DecValTok{2}\OperatorTok{],} \OperatorTok{\{}\FunctionTok{F}\OperatorTok{[}\DecValTok{2}\OperatorTok{,} \OperatorTok{\{}\DecValTok{1}\OperatorTok{\}],} \SpecialCharTok{{-}}\FunctionTok{F}\OperatorTok{[}\DecValTok{2}\OperatorTok{,} \OperatorTok{\{}\DecValTok{1}\OperatorTok{\}]\}} \OtherTok{{-}\textgreater{}} 
\OperatorTok{\{}\SpecialCharTok{{-}}\FunctionTok{F}\OperatorTok{[}\DecValTok{2}\OperatorTok{,} \OperatorTok{\{}\DecValTok{2}\OperatorTok{\}],} \FunctionTok{F}\OperatorTok{[}\DecValTok{2}\OperatorTok{,} \OperatorTok{\{}\DecValTok{2}\OperatorTok{\}]\},}\NormalTok{ InsertionLevel }\OtherTok{{-}\textgreater{}} \OperatorTok{\{}\NormalTok{Classes}\OperatorTok{\},}\NormalTok{ Restrictions }\OtherTok{{-}\textgreater{}}\NormalTok{ QEDOnly}\OperatorTok{]}\NormalTok{;}
\NormalTok{CreateFeynAmp}\OperatorTok{[}\NormalTok{diagsV2}\OperatorTok{,}\NormalTok{PreFactor}\OtherTok{{-}\textgreater{}}\DecValTok{1}\OperatorTok{]}
\end{Highlighting}
\end{Shaded}

has \texttt{fSign[\allowbreak{}1]}.

\hypertarget{general-advice}{%
\subsection{General advice}\label{general-advice}}

If you are doing a calculation where the overall sign of the amplitude
must agree with a particular convention, follow these steps

\begin{enumerate}
\def\labelenumi{\arabic{enumi}.}
\tightlist
\item
  Set the option \texttt{PreFactor} of \texttt{CreateFeynAmp} to
  \texttt{1} to generate \(i \mathcal{M}\).
\item
  Check compatibility of all vertex signs between your preferred
  convention and the employed model. If necessary, adjust the signs in
  the generated amplitudes.
\item
  Use \texttt{FermionSign = fSign;} to figure out the fermion sign of
  each diagram. If it doesn't agree with your convention, flips the
  signs for all diagrams in the given process via
  \texttt{amps/. fSign[\allowbreak{}x_] :> -x}. This will preserve the
  physical relative signs but change the conventional overall sign.
\end{enumerate}
\end{document}
