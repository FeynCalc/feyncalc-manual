% !TeX program = pdflatex
% !TeX root = FeynCalcExternal.tex

\documentclass[../FeynCalcManual.tex]{subfiles}
\begin{document}
\hypertarget{feyncalcexternal}{%
\section{FeynCalcExternal}\label{feyncalcexternal}}

\texttt{FeynCalcExternal[\allowbreak{}exp]} translates exp from the
internal FeynCalc representation to a shorthand form.

\subsection{See also}

\hyperlink{toc}{Overview},
\hyperlink{feyncalcinternal}{FeynCalcInternal}.

\subsection{Examples}

\begin{Shaded}
\begin{Highlighting}[]
\NormalTok{FeynCalcExternal}\OperatorTok{[}\NormalTok{DiracGamma}\OperatorTok{[}\DecValTok{5}\OperatorTok{]]}
\end{Highlighting}
\end{Shaded}

\begin{dmath*}\breakingcomma
\bar{\gamma }^5
\end{dmath*}

\begin{Shaded}
\begin{Highlighting}[]
\NormalTok{FeynCalcExternal}\OperatorTok{[}\NormalTok{DiracGamma}\OperatorTok{[}\DecValTok{5}\OperatorTok{]]} \SpecialCharTok{//} \FunctionTok{StandardForm}

\CommentTok{(*GA[5]*)}
\end{Highlighting}
\end{Shaded}

\begin{Shaded}
\begin{Highlighting}[]
\NormalTok{ex }\ExtensionTok{=} \OperatorTok{\{}\NormalTok{GA}\OperatorTok{[}\SpecialCharTok{\textbackslash{}}\OperatorTok{[}\NormalTok{Mu}\OperatorTok{]],}\NormalTok{ GAD}\OperatorTok{[}\SpecialCharTok{\textbackslash{}}\OperatorTok{[}\NormalTok{Rho}\OperatorTok{]],}\NormalTok{ GS}\OperatorTok{[}\FunctionTok{p}\OperatorTok{],}\NormalTok{ SP}\OperatorTok{[}\FunctionTok{p}\OperatorTok{,} \FunctionTok{q}\OperatorTok{],}\NormalTok{ MT}\OperatorTok{[}\SpecialCharTok{\textbackslash{}}\OperatorTok{[}\NormalTok{Alpha}\OperatorTok{],} \SpecialCharTok{\textbackslash{}}\OperatorTok{[}\FunctionTok{Beta}\OperatorTok{]],}\NormalTok{ FV}\OperatorTok{[}\FunctionTok{p}\OperatorTok{,} \SpecialCharTok{\textbackslash{}}\OperatorTok{[}\NormalTok{Mu}\OperatorTok{]]\}}
\end{Highlighting}
\end{Shaded}

\begin{dmath*}\breakingcomma
\left\{\bar{\gamma }^{\mu },\gamma ^{\rho },\bar{\gamma }\cdot \overline{p},\overline{p}\cdot \overline{q},\bar{g}^{\alpha \beta },\overline{p}^{\mu }\right\}
\end{dmath*}

\begin{Shaded}
\begin{Highlighting}[]
\NormalTok{ex }\SpecialCharTok{//} \FunctionTok{StandardForm}

\CommentTok{(*\{GA[\textbackslash{}[Mu]], GAD[\textbackslash{}[Rho]], GS[p], SP[p, q], MT[\textbackslash{}[Alpha], \textbackslash{}[Beta]], FV[p, \textbackslash{}[Mu]]\}*)}
\end{Highlighting}
\end{Shaded}

\begin{Shaded}
\begin{Highlighting}[]
\NormalTok{ex }\SpecialCharTok{//}\NormalTok{ FeynCalcInternal}
\end{Highlighting}
\end{Shaded}

\begin{dmath*}\breakingcomma
\left\{\bar{\gamma }^{\mu },\gamma ^{\rho },\bar{\gamma }\cdot \overline{p},\overline{p}\cdot \overline{q},\bar{g}^{\alpha \beta },\overline{p}^{\mu }\right\}
\end{dmath*}

\begin{Shaded}
\begin{Highlighting}[]
\NormalTok{ex }\SpecialCharTok{//}\NormalTok{ FeynCalcInternal }\SpecialCharTok{//} \FunctionTok{StandardForm}

\CommentTok{(*\{DiracGamma[LorentzIndex[\textbackslash{}[Mu]]], DiracGamma[LorentzIndex[\textbackslash{}[Rho], D], D], DiracGamma[Momentum[p]], Pair[Momentum[p], Momentum[q]], Pair[LorentzIndex[\textbackslash{}[Alpha]], LorentzIndex[\textbackslash{}[Beta]]], Pair[LorentzIndex[\textbackslash{}[Mu]], Momentum[p]]\}*)}
\end{Highlighting}
\end{Shaded}

\begin{Shaded}
\begin{Highlighting}[]
\NormalTok{ex }\SpecialCharTok{//}\NormalTok{ FeynCalcInternal }\SpecialCharTok{//}\NormalTok{ FeynCalcExternal }\SpecialCharTok{//} \FunctionTok{StandardForm}

\CommentTok{(*\{GA[\textbackslash{}[Mu]], GAD[\textbackslash{}[Rho]], GS[p], SP[p, q], MT[\textbackslash{}[Alpha], \textbackslash{}[Beta]], FV[p, \textbackslash{}[Mu]]\}*)}
\end{Highlighting}
\end{Shaded}

\end{document}
