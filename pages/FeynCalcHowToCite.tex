% !TeX program = pdflatex
% !TeX root = FeynCalcHowToCite.tex

\documentclass[../FeynCalcManual.tex]{subfiles}
\begin{document}
\hypertarget{feyncalchowtocite}{%
\section{FeynCalcHowToCite}\label{feyncalchowtocite}}

\texttt{FeynCalcHowToCite[\allowbreak{}]} lists publications that should
be cited when mentioning FeynCalc in scientific works.

\subsection{See also}

\hyperlink{toc}{Overview}

\subsection{Examples}

\begin{Shaded}
\begin{Highlighting}[]
\NormalTok{FeynCalcHowToCite}\OperatorTok{[]}
\end{Highlighting}
\end{Shaded}

\begin{dmath*}\breakingcomma
\text{ $\bullet $ V. Shtabovenko, R. Mertig and F. Orellana, Comput.Phys.Commun. 256 (2020) 107478, arXiv:2001.04407.}
\end{dmath*}

\begin{dmath*}\breakingcomma
\text{ $\bullet $ V. Shtabovenko, R. Mertig and F. Orellana, Comput.Phys.Commun. 207 (2016) 432-444, arXiv:1601.01167.}
\end{dmath*}

\begin{dmath*}\breakingcomma
\text{ $\bullet $ R. Mertig, M. B{\" o}hm, and A. Denner, Comput. Phys. Commun. 64 (1991) 345-359.}
\end{dmath*}
\end{document}
