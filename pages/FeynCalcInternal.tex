% !TeX program = pdflatex
% !TeX root = FeynCalcInternal.tex

\documentclass[../FeynCalcManual.tex]{subfiles}
\begin{document}
\hypertarget{feyncalcinternal}{%
\section{FeynCalcInternal}\label{feyncalcinternal}}

\texttt{FeynCalcInternal[\allowbreak{}exp]} translates \texttt{exp} into
the internal FeynCalc (abstract data-type) representation.

\subsection{See also}

\hyperlink{toc}{Overview},
\hyperlink{feyncalcexternal}{FeynCalcExternal}, \hyperlink{fci}{FCI},
\hyperlink{fce}{FCE}.

\subsection{Examples}

\begin{Shaded}
\begin{Highlighting}[]
\NormalTok{ex }\ExtensionTok{=} \OperatorTok{\{}\NormalTok{GA}\OperatorTok{[}\SpecialCharTok{\textbackslash{}}\OperatorTok{[}\NormalTok{Mu}\OperatorTok{]],}\NormalTok{ GAD}\OperatorTok{[}\SpecialCharTok{\textbackslash{}}\OperatorTok{[}\NormalTok{Rho}\OperatorTok{]],}\NormalTok{ GS}\OperatorTok{[}\FunctionTok{p}\OperatorTok{],}\NormalTok{ SP}\OperatorTok{[}\FunctionTok{p}\OperatorTok{,} \FunctionTok{q}\OperatorTok{],}\NormalTok{ MT}\OperatorTok{[}\SpecialCharTok{\textbackslash{}}\OperatorTok{[}\NormalTok{Alpha}\OperatorTok{],} \SpecialCharTok{\textbackslash{}}\OperatorTok{[}\FunctionTok{Beta}\OperatorTok{]],}\NormalTok{ FV}\OperatorTok{[}\FunctionTok{p}\OperatorTok{,} \SpecialCharTok{\textbackslash{}}\OperatorTok{[}\NormalTok{Mu}\OperatorTok{]]\}}
\end{Highlighting}
\end{Shaded}

\begin{dmath*}\breakingcomma
\left\{\bar{\gamma }^{\mu },\gamma ^{\rho },\bar{\gamma }\cdot \overline{p},\overline{p}\cdot \overline{q},\bar{g}^{\alpha \beta },\overline{p}^{\mu }\right\}
\end{dmath*}

\begin{Shaded}
\begin{Highlighting}[]
\NormalTok{ex }\SpecialCharTok{//} \FunctionTok{StandardForm}

\CommentTok{(*\{GA[\textbackslash{}[Mu]], GAD[\textbackslash{}[Rho]], GS[p], SP[p, q], MT[\textbackslash{}[Alpha], \textbackslash{}[Beta]], FV[p, \textbackslash{}[Mu]]\}*)}
\end{Highlighting}
\end{Shaded}

\begin{Shaded}
\begin{Highlighting}[]
\NormalTok{ex }\SpecialCharTok{//}\NormalTok{ FeynCalcInternal}
\end{Highlighting}
\end{Shaded}

\begin{dmath*}\breakingcomma
\left\{\bar{\gamma }^{\mu },\gamma ^{\rho },\bar{\gamma }\cdot \overline{p},\overline{p}\cdot \overline{q},\bar{g}^{\alpha \beta },\overline{p}^{\mu }\right\}
\end{dmath*}

\begin{Shaded}
\begin{Highlighting}[]
\NormalTok{ex }\SpecialCharTok{//} \FunctionTok{StandardForm}

\CommentTok{(*\{GA[\textbackslash{}[Mu]], GAD[\textbackslash{}[Rho]], GS[p], SP[p, q], MT[\textbackslash{}[Alpha], \textbackslash{}[Beta]], FV[p, \textbackslash{}[Mu]]\}*)}
\end{Highlighting}
\end{Shaded}

\begin{Shaded}
\begin{Highlighting}[]
\NormalTok{FeynCalcExternal}\OperatorTok{[}\NormalTok{ex}\OperatorTok{]} \SpecialCharTok{//} \FunctionTok{StandardForm}

\CommentTok{(*\{GA[\textbackslash{}[Mu]], GAD[\textbackslash{}[Rho]], GS[p], SP[p, q], MT[\textbackslash{}[Alpha], \textbackslash{}[Beta]], FV[p, \textbackslash{}[Mu]]\}*)}
\end{Highlighting}
\end{Shaded}

\begin{Shaded}
\begin{Highlighting}[]
\NormalTok{ex }\ExtensionTok{=}\NormalTok{ FCI}\OperatorTok{[\{}\NormalTok{SD}\OperatorTok{[}\FunctionTok{a}\OperatorTok{,} \FunctionTok{b}\OperatorTok{],}\NormalTok{ SUND}\OperatorTok{[}\FunctionTok{a}\OperatorTok{,} \FunctionTok{b}\OperatorTok{,} \FunctionTok{c}\OperatorTok{],}\NormalTok{ SUNF}\OperatorTok{[}\FunctionTok{a}\OperatorTok{,} \FunctionTok{b}\OperatorTok{,} \FunctionTok{c}\OperatorTok{],}\NormalTok{ FAD}\OperatorTok{[}\FunctionTok{q}\OperatorTok{],}\NormalTok{ LC}\OperatorTok{[}\SpecialCharTok{\textbackslash{}}\OperatorTok{[}\NormalTok{Mu}\OperatorTok{],} \SpecialCharTok{\textbackslash{}}\OperatorTok{[}\NormalTok{Nu}\OperatorTok{],} \SpecialCharTok{\textbackslash{}}\OperatorTok{[}\NormalTok{Rho}\OperatorTok{],} \SpecialCharTok{\textbackslash{}}\OperatorTok{[}\NormalTok{Sigma}\OperatorTok{]]\}]}
\end{Highlighting}
\end{Shaded}

\begin{dmath*}\breakingcomma
\left\{\delta ^{ab},d^{abc},f^{abc},\frac{1}{q^2},\bar{\epsilon }^{\mu \nu \rho \sigma }\right\}
\end{dmath*}

\begin{Shaded}
\begin{Highlighting}[]
\NormalTok{ex }\SpecialCharTok{//} \FunctionTok{StandardForm}

\CommentTok{(*\{SUNDelta[SUNIndex[a], SUNIndex[b]], SUND[SUNIndex[a], SUNIndex[b], SUNIndex[c]], SUNF[SUNIndex[a], SUNIndex[b], SUNIndex[c]], FeynAmpDenominator[PropagatorDenominator[Momentum[q, D], 0]], Eps[LorentzIndex[\textbackslash{}[Mu]], LorentzIndex[\textbackslash{}[Nu]], LorentzIndex[\textbackslash{}[Rho]], LorentzIndex[\textbackslash{}[Sigma]]]\}*)}
\end{Highlighting}
\end{Shaded}

\end{document}
