% !TeX program = pdflatex
% !TeX root = FeynmanIntegralPrefactor.tex

\documentclass[../FeynCalcManual.tex]{subfiles}
\begin{document}
\hypertarget{feynmanintegralprefactor}{
\section{FeynmanIntegralPrefactor}\label{feynmanintegralprefactor}\index{FeynmanIntegralPrefactor}}

\texttt{FeynmanIntegralPrefactor} is an option for
\texttt{FCFeynmanParametrize} and other functions. It denotes an
implicit prefactor that has to be understood in front of a loop integral
in the usual \texttt{FeynAmpDenominator}-notation. The prefactor is the
quantity that multiplies the loop integral measure
\(d^D q_1 \ldots d^D q_n\) and plays an important role e.g.~when
deriving the Feynman parameter representation of the given integral.
Apart from specifying an explicit value, the user may also choose from
the following predefined conventions:

\begin{itemize}
\tightlist
\item
  ``Unity'' - 1 for each loop
\item
  ``Textbook'' - \(\frac{1}{(2\pi)^D}\) for each loop.
\item
  ``Multiloop1'' - \(\frac{1}{i \pi^{D/2}}\) for each loop if the
  integral is Minkowskian, \(\frac{1}{i \pi^{D/2}}\) or
  \(\frac{1}{i \pi^{(D-1)/2}}\) for each loop if the integral is
  Euclidean or Cartesian respectively.
\item
  ``Multiloop2'' - like ``Multiloop1'' but with an extra
  \(e^{\frac{(4-D)}{2} \gamma_E}\) for each loop
\end{itemize}

The standard value is ``Multiloop1''.

\subsection{See also}

\hyperlink{toc}{Overview},
\hyperlink{fcfeynmanparametrize}{FCFeynmanParametrize}.

\subsection{Examples}

\begin{Shaded}
\begin{Highlighting}[]
\NormalTok{FCFeynmanParametrize}\OperatorTok{[}\NormalTok{FAD}\OperatorTok{[}\FunctionTok{p}\OperatorTok{,} \FunctionTok{p} \SpecialCharTok{{-}} \FunctionTok{q}\OperatorTok{],} \OperatorTok{\{}\FunctionTok{p}\OperatorTok{\},} \FunctionTok{Names} \OtherTok{{-}\textgreater{}} \FunctionTok{x}\OperatorTok{,}\NormalTok{ FCReplaceD }\OtherTok{{-}\textgreater{}} \OperatorTok{\{}\FunctionTok{D} \OtherTok{{-}\textgreater{}} \DecValTok{4} \SpecialCharTok{{-}} \DecValTok{2}\NormalTok{ Epsilon}\OperatorTok{\}]}
\end{Highlighting}
\end{Shaded}

\begin{dmath*}\breakingcomma
\left\{(x(1)+x(2))^{2 \varepsilon -2} \left(-q^2 x(1) x(2)\right)^{-\varepsilon },\Gamma (\varepsilon ),\{x(1),x(2)\}\right\}
\end{dmath*}

\begin{Shaded}
\begin{Highlighting}[]
\NormalTok{FCFeynmanParametrize}\OperatorTok{[}\NormalTok{FAD}\OperatorTok{[}\FunctionTok{p}\OperatorTok{,} \FunctionTok{p} \SpecialCharTok{{-}} \FunctionTok{q}\OperatorTok{],} \OperatorTok{\{}\FunctionTok{p}\OperatorTok{\},} \FunctionTok{Names} \OtherTok{{-}\textgreater{}} \FunctionTok{x}\OperatorTok{,}\NormalTok{ FCReplaceD }\OtherTok{{-}\textgreater{}} \OperatorTok{\{}\FunctionTok{D} \OtherTok{{-}\textgreater{}} \DecValTok{4} \SpecialCharTok{{-}} \DecValTok{2}\NormalTok{ Epsilon}\OperatorTok{\}]} 
 
\FunctionTok{Times}\NormalTok{ @@ }\FunctionTok{Most}\OperatorTok{[}\SpecialCharTok{\%}\OperatorTok{]}
\end{Highlighting}
\end{Shaded}

\begin{dmath*}\breakingcomma
\left\{(x(1)+x(2))^{2 \varepsilon -2} \left(-q^2 x(1) x(2)\right)^{-\varepsilon },\Gamma (\varepsilon ),\{x(1),x(2)\}\right\}
\end{dmath*}

\begin{dmath*}\breakingcomma
\Gamma (\varepsilon ) (x(1)+x(2))^{2 \varepsilon -2} \left(-q^2 x(1) x(2)\right)^{-\varepsilon }
\end{dmath*}

\begin{Shaded}
\begin{Highlighting}[]
\NormalTok{FCFeynmanParametrize}\OperatorTok{[}\NormalTok{FAD}\OperatorTok{[}\FunctionTok{p}\OperatorTok{,} \FunctionTok{p} \SpecialCharTok{{-}} \FunctionTok{q}\OperatorTok{],} \OperatorTok{\{}\FunctionTok{p}\OperatorTok{\},} \FunctionTok{Names} \OtherTok{{-}\textgreater{}} \FunctionTok{x}\OperatorTok{,}\NormalTok{ FCReplaceD }\OtherTok{{-}\textgreater{}} \OperatorTok{\{}\FunctionTok{D} \OtherTok{{-}\textgreater{}} \DecValTok{4} \SpecialCharTok{{-}} \DecValTok{2}\NormalTok{ Epsilon}\OperatorTok{\},} 
\NormalTok{   FeynmanIntegralPrefactor }\OtherTok{{-}\textgreater{}} \StringTok{"Multiloop1"}\OperatorTok{]} 
 
\FunctionTok{Times}\NormalTok{ @@ }\FunctionTok{Most}\OperatorTok{[}\SpecialCharTok{\%}\OperatorTok{]}
\end{Highlighting}
\end{Shaded}

\begin{dmath*}\breakingcomma
\left\{(x(1)+x(2))^{2 \varepsilon -2} \left(-q^2 x(1) x(2)\right)^{-\varepsilon },\Gamma (\varepsilon ),\{x(1),x(2)\}\right\}
\end{dmath*}

\begin{dmath*}\breakingcomma
\Gamma (\varepsilon ) (x(1)+x(2))^{2 \varepsilon -2} \left(-q^2 x(1) x(2)\right)^{-\varepsilon }
\end{dmath*}

\begin{Shaded}
\begin{Highlighting}[]
\NormalTok{FCFeynmanParametrize}\OperatorTok{[}\NormalTok{FAD}\OperatorTok{[}\FunctionTok{p}\OperatorTok{,} \FunctionTok{p} \SpecialCharTok{{-}} \FunctionTok{q}\OperatorTok{],} \OperatorTok{\{}\FunctionTok{p}\OperatorTok{\},} \FunctionTok{Names} \OtherTok{{-}\textgreater{}} \FunctionTok{x}\OperatorTok{,}\NormalTok{ FCReplaceD }\OtherTok{{-}\textgreater{}} \OperatorTok{\{}\FunctionTok{D} \OtherTok{{-}\textgreater{}} \DecValTok{4} \SpecialCharTok{{-}} \DecValTok{2}\NormalTok{ Epsilon}\OperatorTok{\},} 
\NormalTok{   FeynmanIntegralPrefactor }\OtherTok{{-}\textgreater{}} \StringTok{"Unity"}\OperatorTok{]} 
 
\FunctionTok{Times}\NormalTok{ @@ }\FunctionTok{Most}\OperatorTok{[}\SpecialCharTok{\%}\OperatorTok{]}
\end{Highlighting}
\end{Shaded}

\begin{dmath*}\breakingcomma
\left\{(x(1)+x(2))^{2 \varepsilon -2} \left(-q^2 x(1) x(2)\right)^{-\varepsilon },i \pi ^{2-\varepsilon } \Gamma (\varepsilon ),\{x(1),x(2)\}\right\}
\end{dmath*}

\begin{dmath*}\breakingcomma
i \pi ^{2-\varepsilon } \Gamma (\varepsilon ) (x(1)+x(2))^{2 \varepsilon -2} \left(-q^2 x(1) x(2)\right)^{-\varepsilon }
\end{dmath*}

\begin{Shaded}
\begin{Highlighting}[]
\NormalTok{FCFeynmanParametrize}\OperatorTok{[}\NormalTok{FAD}\OperatorTok{[}\FunctionTok{p}\OperatorTok{,} \FunctionTok{p} \SpecialCharTok{{-}} \FunctionTok{q}\OperatorTok{],} \OperatorTok{\{}\FunctionTok{p}\OperatorTok{\},} \FunctionTok{Names} \OtherTok{{-}\textgreater{}} \FunctionTok{x}\OperatorTok{,}\NormalTok{ FCReplaceD }\OtherTok{{-}\textgreater{}} \OperatorTok{\{}\FunctionTok{D} \OtherTok{{-}\textgreater{}} \DecValTok{4} \SpecialCharTok{{-}} \DecValTok{2}\NormalTok{ Epsilon}\OperatorTok{\},} 
\NormalTok{   FeynmanIntegralPrefactor }\OtherTok{{-}\textgreater{}} \StringTok{"Textbook"}\OperatorTok{]} 
 
\FunctionTok{Times}\NormalTok{ @@ }\FunctionTok{Most}\OperatorTok{[}\SpecialCharTok{\%}\OperatorTok{]}
\end{Highlighting}
\end{Shaded}

\begin{dmath*}\breakingcomma
\left\{(x(1)+x(2))^{2 \varepsilon -2} \left(-q^2 x(1) x(2)\right)^{-\varepsilon },i 2^{2 \varepsilon -4} \pi ^{\varepsilon -2} \Gamma (\varepsilon ),\{x(1),x(2)\}\right\}
\end{dmath*}

\begin{dmath*}\breakingcomma
i 2^{2 \varepsilon -4} \pi ^{\varepsilon -2} \Gamma (\varepsilon ) (x(1)+x(2))^{2 \varepsilon -2} \left(-q^2 x(1) x(2)\right)^{-\varepsilon }
\end{dmath*}

\begin{Shaded}
\begin{Highlighting}[]
\NormalTok{FCFeynmanParametrize}\OperatorTok{[}\NormalTok{FAD}\OperatorTok{[}\FunctionTok{p}\OperatorTok{,} \FunctionTok{p} \SpecialCharTok{{-}} \FunctionTok{q}\OperatorTok{],} \OperatorTok{\{}\FunctionTok{p}\OperatorTok{\},} \FunctionTok{Names} \OtherTok{{-}\textgreater{}} \FunctionTok{x}\OperatorTok{,}\NormalTok{ FCReplaceD }\OtherTok{{-}\textgreater{}} \OperatorTok{\{}\FunctionTok{D} \OtherTok{{-}\textgreater{}} \DecValTok{4} \SpecialCharTok{{-}} \DecValTok{2}\NormalTok{ Epsilon}\OperatorTok{\},} 
\NormalTok{   FeynmanIntegralPrefactor }\OtherTok{{-}\textgreater{}} \StringTok{"Multiloop2"}\OperatorTok{]} 
 
\FunctionTok{Times}\NormalTok{ @@ }\FunctionTok{Most}\OperatorTok{[}\SpecialCharTok{\%}\OperatorTok{]}
\end{Highlighting}
\end{Shaded}

\begin{dmath*}\breakingcomma
\left\{(x(1)+x(2))^{2 \varepsilon -2} \left(-q^2 x(1) x(2)\right)^{-\varepsilon },e^{\gamma  \varepsilon } \Gamma (\varepsilon ),\{x(1),x(2)\}\right\}
\end{dmath*}

\begin{dmath*}\breakingcomma
e^{\gamma  \varepsilon } \Gamma (\varepsilon ) (x(1)+x(2))^{2 \varepsilon -2} \left(-q^2 x(1) x(2)\right)^{-\varepsilon }
\end{dmath*}

\begin{Shaded}
\begin{Highlighting}[]
\NormalTok{FCFeynmanParametrize}\OperatorTok{[}\NormalTok{FAD}\OperatorTok{[\{}\FunctionTok{p}\OperatorTok{,} \FunctionTok{m}\OperatorTok{\}],} \OperatorTok{\{}\FunctionTok{p}\OperatorTok{\},} \FunctionTok{Names} \OtherTok{{-}\textgreater{}} \FunctionTok{x}\OperatorTok{,}\NormalTok{ FCReplaceD }\OtherTok{{-}\textgreater{}} \OperatorTok{\{}\FunctionTok{D} \OtherTok{{-}\textgreater{}} \DecValTok{4} \SpecialCharTok{{-}} \DecValTok{2}\NormalTok{ Epsilon}\OperatorTok{\},} 
\NormalTok{   FeynmanIntegralPrefactor }\OtherTok{{-}\textgreater{}} \StringTok{"Multiloop2"}\OperatorTok{]} 
 
\FunctionTok{Times}\NormalTok{ @@ }\FunctionTok{Most}\OperatorTok{[}\SpecialCharTok{\%}\OperatorTok{]} 
 
\FunctionTok{Series}\OperatorTok{[}\SpecialCharTok{\%}\OperatorTok{,} \OperatorTok{\{}\NormalTok{Epsilon}\OperatorTok{,} \DecValTok{0}\OperatorTok{,} \DecValTok{1}\OperatorTok{\}]} \SpecialCharTok{//} \FunctionTok{Normal} \SpecialCharTok{//} \FunctionTok{FunctionExpand}
\end{Highlighting}
\end{Shaded}

\begin{dmath*}\breakingcomma
\left\{1,-e^{\gamma  \varepsilon } \Gamma (\varepsilon -1) \left(m^2\right)^{1-\varepsilon },\{\}\right\}
\end{dmath*}

\begin{dmath*}\breakingcomma
-e^{\gamma  \varepsilon } \Gamma (\varepsilon -1) \left(m^2\right)^{1-\varepsilon }
\end{dmath*}

\begin{dmath*}\breakingcomma
\frac{m^2}{\varepsilon }+\frac{1}{12} \varepsilon  \left(\pi ^2 m^2+12 m^2+6 m^2 \log ^2\left(m^2\right)-12 m^2 \log \left(m^2\right)\right)+m^2+m^2 \left(-\log \left(m^2\right)\right)
\end{dmath*}
\end{document}
