% !TeX program = pdflatex
% !TeX root = FieldDerivative.tex

\documentclass[../FeynCalcManual.tex]{subfiles}
\begin{document}
\hypertarget{fieldderivative}{
\section{FieldDerivative}\label{fieldderivative}\index{FieldDerivative}}

\texttt{FieldDerivative[\allowbreak{}f[\allowbreak{}x],\ \allowbreak{}x,\ \allowbreak{}li1,\ \allowbreak{}li2,\ \allowbreak{}...]}
is the derivative of \texttt{f[\allowbreak{}x]} with respect to
space-time variables \texttt{x} and with Lorentz indices
\texttt{li1,\ \allowbreak{}li2,\ \allowbreak{} ...}, where
\texttt{li1,\ \allowbreak{}li2,\ \allowbreak{}...} have head
\texttt{LorentzIndex}.

\texttt{FieldDerivative[\allowbreak{}f[\allowbreak{}x],\ \allowbreak{}x,\ \allowbreak{}li1,\ \allowbreak{}li2,\ \allowbreak{}...]}
can be given as
\texttt{FieldDerivative[\allowbreak{}f[\allowbreak{}x],\ \allowbreak{}x,\ \allowbreak{}\{\allowbreak{}l1,\ \allowbreak{}l2,\ \allowbreak{}...\}]},
where \(l1\) is \(li1\) without the head.

\texttt{FieldDerivative} is defined only for objects with head
\texttt{QuantumField}. If the space-time derivative of other objects is
wanted, the corresponding rule must be specified.

\subsection{See also}

\hyperlink{toc}{Overview}, \hyperlink{fcpartiald}{FCPartialD},
\hyperlink{expandpartiald}{ExpandPartialD}.

\subsection{Examples}

\begin{Shaded}
\begin{Highlighting}[]
\NormalTok{QuantumField}\OperatorTok{[}\FunctionTok{A}\OperatorTok{,} \OperatorTok{\{}\SpecialCharTok{\textbackslash{}}\OperatorTok{[}\NormalTok{Mu}\OperatorTok{]\}][}\FunctionTok{x}\OperatorTok{]}\NormalTok{ . QuantumField}\OperatorTok{[}\FunctionTok{B}\OperatorTok{,} \OperatorTok{\{}\SpecialCharTok{\textbackslash{}}\OperatorTok{[}\NormalTok{Nu}\OperatorTok{]\}][}\FunctionTok{y}\OperatorTok{]}\NormalTok{ . QuantumField}\OperatorTok{[}\FunctionTok{C}\OperatorTok{,} \OperatorTok{\{}\SpecialCharTok{\textbackslash{}}\OperatorTok{[}\NormalTok{Rho}\OperatorTok{]\}][}\FunctionTok{x}\OperatorTok{]}\NormalTok{ . QuantumField}\OperatorTok{[}\FunctionTok{D}\OperatorTok{,} \OperatorTok{\{}\SpecialCharTok{\textbackslash{}}\OperatorTok{[}\NormalTok{Sigma}\OperatorTok{]\}][}\FunctionTok{y}\OperatorTok{]}
\end{Highlighting}
\end{Shaded}

\begin{dmath*}\breakingcomma
A_{\mu }(x).B_{\nu }(y).C_{\rho }(x).D_{\sigma }(y)
\end{dmath*}

\begin{Shaded}
\begin{Highlighting}[]
\NormalTok{FieldDerivative}\OperatorTok{[}\SpecialCharTok{\%}\OperatorTok{,} \FunctionTok{x}\OperatorTok{,} \OperatorTok{\{}\SpecialCharTok{\textbackslash{}}\OperatorTok{[}\NormalTok{Mu}\OperatorTok{]\}]} \SpecialCharTok{//}\NormalTok{ DotExpand}
\end{Highlighting}
\end{Shaded}

\begin{dmath*}\breakingcomma
A_{\mu }(x).B_{\nu }(y).\left(\left.(\partial _{\mu }C_{\rho }\right)\right)(x).D_{\sigma }(y)+\left(\left.(\partial _{\mu }A_{\mu }\right)\right)(x).B_{\nu }(y).C_{\rho }(x).D_{\sigma }(y)
\end{dmath*}

\begin{Shaded}
\begin{Highlighting}[]
\NormalTok{FieldDerivative}\OperatorTok{[}\SpecialCharTok{\%\%}\OperatorTok{,} \FunctionTok{y}\OperatorTok{,} \OperatorTok{\{}\SpecialCharTok{\textbackslash{}}\OperatorTok{[}\NormalTok{Nu}\OperatorTok{]\}]} \SpecialCharTok{//}\NormalTok{ DotExpand}
\end{Highlighting}
\end{Shaded}

\begin{dmath*}\breakingcomma
A_{\mu }(x).B_{\nu }(y).C_{\rho }(x).\left(\left.(\partial _{\nu }D_{\sigma }\right)\right)(y)+A_{\mu }(x).\left(\left.(\partial _{\nu }B_{\nu }\right)\right)(y).C_{\rho }(x).D_{\sigma }(y)
\end{dmath*}
\end{document}
