% !TeX program = pdflatex
% !TeX root = FieldStrength.tex

\documentclass[../FeynCalcManual.tex]{subfiles}
\begin{document}
\hypertarget{fieldstrength}{
\section{FieldStrength}\label{fieldstrength}\index{FieldStrength}}

\texttt{FieldStrength[\allowbreak{}mu,\ \allowbreak{}nu,\ \allowbreak{}a]}
is the field strength tensor
\(\partial _{\mu } A_{\nu }^a - \partial _{\nu } A_{\mu }^a + g_s A_{\mu }^b A_{\nu }^c f^{abc}\).

\texttt{FieldStrength[\allowbreak{}mu,\ \allowbreak{}nu]} is the field
strength tensor \((\partial _{\mu } A_{\nu}- \partial_{\nu } A_{\mu})\).

The name of the field (\(A\)) and the coupling constant (\(g\)) can be
set through the options or by additional arguments. The first two
indices are interpreted as type \texttt{LorentzIndex}, except
\texttt{OPEDelta}, which is converted to
\texttt{Momentum[\allowbreak{}OPEDelta]}.

\subsection{See also}

\hyperlink{toc}{Overview}

\subsection{Examples}

\begin{Shaded}
\begin{Highlighting}[]
\NormalTok{FieldStrength}\OperatorTok{[}\SpecialCharTok{\textbackslash{}}\OperatorTok{[}\NormalTok{Mu}\OperatorTok{],} \SpecialCharTok{\textbackslash{}}\OperatorTok{[}\NormalTok{Nu}\OperatorTok{]]}
\end{Highlighting}
\end{Shaded}

\begin{dmath*}\breakingcomma
F_{\mu \nu }^{}
\end{dmath*}

\begin{Shaded}
\begin{Highlighting}[]
\NormalTok{FieldStrength}\OperatorTok{[}\SpecialCharTok{\textbackslash{}}\OperatorTok{[}\NormalTok{Mu}\OperatorTok{],} \SpecialCharTok{\textbackslash{}}\OperatorTok{[}\NormalTok{Nu}\OperatorTok{],} \FunctionTok{a}\OperatorTok{]}
\end{Highlighting}
\end{Shaded}

\begin{dmath*}\breakingcomma
F_{\mu \nu }^a
\end{dmath*}

\begin{Shaded}
\begin{Highlighting}[]
\NormalTok{FieldStrength}\OperatorTok{[}\SpecialCharTok{\textbackslash{}}\OperatorTok{[}\NormalTok{Mu}\OperatorTok{],} \SpecialCharTok{\textbackslash{}}\OperatorTok{[}\NormalTok{Nu}\OperatorTok{],}\NormalTok{ Explicit }\OtherTok{{-}\textgreater{}} \ConstantTok{True}\OperatorTok{]}
\end{Highlighting}
\end{Shaded}

\begin{dmath*}\breakingcomma
\left.(\partial _{\mu }A_{\nu }\right)-\left.(\partial _{\nu }A_{\mu }\right)
\end{dmath*}

\begin{Shaded}
\begin{Highlighting}[]
\NormalTok{FieldStrength}\OperatorTok{[}\SpecialCharTok{\textbackslash{}}\OperatorTok{[}\NormalTok{Mu}\OperatorTok{],} \SpecialCharTok{\textbackslash{}}\OperatorTok{[}\NormalTok{Nu}\OperatorTok{],} \FunctionTok{a}\OperatorTok{,}\NormalTok{ Explicit }\OtherTok{{-}\textgreater{}} \ConstantTok{True}\OperatorTok{]}
\end{Highlighting}
\end{Shaded}

\begin{dmath*}\breakingcomma
g_s f^{a\text{b19}\;\text{c20}} A_{\mu }^{\text{b19}}.A_{\nu }^{\text{c20}}+\left.(\partial _{\mu }A_{\nu }^a\right)-\left.(\partial _{\nu }A_{\mu }^a\right)
\end{dmath*}

\begin{Shaded}
\begin{Highlighting}[]
\NormalTok{FieldStrength}\OperatorTok{[}\SpecialCharTok{\textbackslash{}}\OperatorTok{[}\NormalTok{Mu}\OperatorTok{],} \SpecialCharTok{\textbackslash{}}\OperatorTok{[}\NormalTok{Nu}\OperatorTok{],} \FunctionTok{a}\OperatorTok{,}\NormalTok{ CouplingConstant }\OtherTok{{-}\textgreater{}} \SpecialCharTok{{-}}\NormalTok{SMP}\OperatorTok{[}\StringTok{"g\_s"}\OperatorTok{],}\NormalTok{ Explicit }\OtherTok{{-}\textgreater{}} \ConstantTok{True}\OperatorTok{]}
\end{Highlighting}
\end{Shaded}

\begin{dmath*}\breakingcomma
-g_s f^{a\text{b21}\;\text{c22}} A_{\mu }^{\text{b21}}.A_{\nu }^{\text{c22}}+\left.(\partial _{\mu }A_{\nu }^a\right)-\left.(\partial _{\nu }A_{\mu }^a\right)
\end{dmath*}
\end{document}
