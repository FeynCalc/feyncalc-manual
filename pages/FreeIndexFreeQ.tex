% !TeX program = pdflatex
% !TeX root = FreeIndexFreeQ.tex

\documentclass[../FeynCalcManual.tex]{subfiles}
\begin{document}
\hypertarget{freeindexfreeq}{
\section{FreeIndexFreeQ}\label{freeindexfreeq}\index{FreeIndexFreeQ}}

\texttt{FreeIndexFreeQ[\allowbreak{}exp,\ \allowbreak{}\{\allowbreak{}head1,\ \allowbreak{}head2,\ \allowbreak{}...\}]}
returns \texttt{True} if the expression contains uncontracted indices
with heads \texttt{head1}, \texttt{head2}, \ldots{} and \texttt{False}
otherwise.

As always in FeynCalc, Einstein summation convention is implicitly
assumed. The function is optimized for large expressions, i.e.~it is not
so good as a criterion in e.g.~\texttt{Select}.

\subsection{See also}

\hyperlink{toc}{Overview},
\hyperlink{fcrenamedummyindices}{FCRenameDummyIndices},
\hyperlink{contract}{Contract},
\hyperlink{dummyindexfreeq}{DummyIndexFreeQ}.

\subsection{Examples}

\begin{Shaded}
\begin{Highlighting}[]
\NormalTok{FCI}\OperatorTok{[}\NormalTok{FV}\OperatorTok{[}\FunctionTok{p}\OperatorTok{,} \SpecialCharTok{\textbackslash{}}\OperatorTok{[}\NormalTok{Mu}\OperatorTok{]]}\NormalTok{ FV}\OperatorTok{[}\FunctionTok{q}\OperatorTok{,} \SpecialCharTok{\textbackslash{}}\OperatorTok{[}\NormalTok{Nu}\OperatorTok{]]]} 
 
\NormalTok{FreeIndexFreeQ}\OperatorTok{[}\SpecialCharTok{\%}\OperatorTok{,} \OperatorTok{\{}\NormalTok{LorentzIndex}\OperatorTok{\}]}
\end{Highlighting}
\end{Shaded}

\begin{dmath*}\breakingcomma
\overline{p}^{\mu } \overline{q}^{\nu }
\end{dmath*}

\begin{dmath*}\breakingcomma
\text{False}
\end{dmath*}

\begin{Shaded}
\begin{Highlighting}[]
\NormalTok{FCI}\OperatorTok{[}\NormalTok{FV}\OperatorTok{[}\FunctionTok{p}\OperatorTok{,} \SpecialCharTok{\textbackslash{}}\OperatorTok{[}\NormalTok{Mu}\OperatorTok{]]}\NormalTok{ FV}\OperatorTok{[}\FunctionTok{q}\OperatorTok{,} \SpecialCharTok{\textbackslash{}}\OperatorTok{[}\NormalTok{Mu}\OperatorTok{]]]} 
 
\NormalTok{FreeIndexFreeQ}\OperatorTok{[}\SpecialCharTok{\%}\OperatorTok{,} \OperatorTok{\{}\NormalTok{LorentzIndex}\OperatorTok{\}]}
\end{Highlighting}
\end{Shaded}

\begin{dmath*}\breakingcomma
\overline{p}^{\mu } \overline{q}^{\mu }
\end{dmath*}

\begin{dmath*}\breakingcomma
\text{True}
\end{dmath*}

\begin{Shaded}
\begin{Highlighting}[]
\NormalTok{FCI}\OperatorTok{[}\NormalTok{SUNT}\OperatorTok{[}\FunctionTok{a}\OperatorTok{,} \FunctionTok{b}\OperatorTok{]]} 
 
\NormalTok{FreeIndexFreeQ}\OperatorTok{[}\SpecialCharTok{\%}\OperatorTok{,} \OperatorTok{\{}\NormalTok{SUNIndex}\OperatorTok{\}]}
\end{Highlighting}
\end{Shaded}

\begin{dmath*}\breakingcomma
T^a.T^b
\end{dmath*}

\begin{dmath*}\breakingcomma
\text{False}
\end{dmath*}

\begin{Shaded}
\begin{Highlighting}[]
\NormalTok{FCI}\OperatorTok{[}\NormalTok{SUNT}\OperatorTok{[}\FunctionTok{a}\OperatorTok{,} \FunctionTok{a}\OperatorTok{]]} 
 
\NormalTok{FreeIndexFreeQ}\OperatorTok{[}\SpecialCharTok{\%}\OperatorTok{,} \OperatorTok{\{}\NormalTok{SUNIndex}\OperatorTok{\}]}
\end{Highlighting}
\end{Shaded}

\begin{dmath*}\breakingcomma
T^a.T^a
\end{dmath*}

\begin{dmath*}\breakingcomma
\text{True}
\end{dmath*}
\end{document}
