% !TeX program = pdflatex
% !TeX root = FromFCPartialFractionForm.tex

\documentclass[../FeynCalcManual.tex]{subfiles}
\begin{document}
\begin{Shaded}
\begin{Highlighting}[]
 
\end{Highlighting}
\end{Shaded}

\hypertarget{fromfcpartialfractionform}{
\section{FromFCPartialFractionForm}\label{fromfcpartialfractionform}\index{FromFCPartialFractionForm}}

\texttt{FromFCPartialFractionForm[\allowbreak{}exp]} converts all
\texttt{FCPartialFractionForm} symbols present in \texttt{exp} back into
the standard representation.

\subsection{See also}

\hyperlink{toc}{Overview},
\hyperlink{tofcpartialfractionform}{ToFCPartialFractionForm},
\hyperlink{fcpartialfractionform}{FCPartialFractionForm}.

\subsection{Examples}

\begin{Shaded}
\begin{Highlighting}[]
\NormalTok{FromFCPartialFractionForm}\OperatorTok{[}\NormalTok{FCPartialFractionForm}\OperatorTok{[}\FunctionTok{x}\OperatorTok{,} \OperatorTok{\{\},} \FunctionTok{x}\OperatorTok{]]}
\end{Highlighting}
\end{Shaded}

\begin{dmath*}\breakingcomma
x
\end{dmath*}

\begin{Shaded}
\begin{Highlighting}[]
\NormalTok{FromFCPartialFractionForm}\OperatorTok{[}\NormalTok{FCPartialFractionForm}\OperatorTok{[}\DecValTok{0}\OperatorTok{,} \OperatorTok{\{\{\{}\FunctionTok{x} \SpecialCharTok{{-}} \DecValTok{1}\OperatorTok{,} \SpecialCharTok{{-}}\DecValTok{2}\OperatorTok{\},} \DecValTok{1}\OperatorTok{\}\},} \FunctionTok{x}\OperatorTok{]]}
\end{Highlighting}
\end{Shaded}

\begin{dmath*}\breakingcomma
\frac{1}{(x-1)^2}
\end{dmath*}

\begin{Shaded}
\begin{Highlighting}[]
\NormalTok{FromFCPartialFractionForm}\OperatorTok{[}\NormalTok{FCPartialFractionForm}\OperatorTok{[}\DecValTok{0}\OperatorTok{,} \OperatorTok{\{\{\{}\FunctionTok{x} \SpecialCharTok{+} \DecValTok{1}\OperatorTok{,} \SpecialCharTok{{-}}\DecValTok{1}\OperatorTok{\},} \DecValTok{1}\OperatorTok{\},} \OperatorTok{\{\{}\FunctionTok{x} \SpecialCharTok{{-}} \FunctionTok{y}\OperatorTok{,} \SpecialCharTok{{-}}\DecValTok{2}\OperatorTok{\},} \FunctionTok{c}\OperatorTok{\}\},} \FunctionTok{x}\OperatorTok{]]}
\end{Highlighting}
\end{Shaded}

\begin{dmath*}\breakingcomma
\frac{c}{(x-y)^2}+\frac{1}{x+1}
\end{dmath*}

\begin{Shaded}
\begin{Highlighting}[]
\NormalTok{FromFCPartialFractionForm}\OperatorTok{[}\NormalTok{FCPartialFractionForm}\OperatorTok{[}\DecValTok{0}\OperatorTok{,} \OperatorTok{\{\{\{}\FunctionTok{x} \SpecialCharTok{+} \DecValTok{1}\OperatorTok{,} \SpecialCharTok{{-}}\DecValTok{1}\OperatorTok{\},} \DecValTok{1}\OperatorTok{\},} \OperatorTok{\{\{}\FunctionTok{x} \SpecialCharTok{{-}} \FunctionTok{y}\OperatorTok{,} \SpecialCharTok{{-}}\DecValTok{2}\OperatorTok{\},} \FunctionTok{c}\OperatorTok{\}\},} \FunctionTok{x}\OperatorTok{],}\NormalTok{ Factoring }\OtherTok{{-}\textgreater{}} \FunctionTok{Together}\OperatorTok{]}
\end{Highlighting}
\end{Shaded}

\begin{dmath*}\breakingcomma
\frac{c x+c+x^2-2 x y+y^2}{(x+1) (x-y)^2}
\end{dmath*}
\end{document}
