% !TeX program = pdflatex
% !TeX root = FromGFAD.tex

\documentclass[../FeynCalcManual.tex]{subfiles}
\begin{document}
\hypertarget{fromgfad}{
\section{FromGFAD}\label{fromgfad}\index{FromGFAD}}

\texttt{FromGFAD[\allowbreak{}exp]} converts all suitable generic
propagator denominators into standard and Cartesian propagator
denominators.

The options \texttt{InitialSubstitutions} and
\texttt{IntermediateSubstitutions} can be used to help the function
handle nontrivial propagators. In particular,
\texttt{InitialSubstitutions} can define rules for completing the square
in the loop momenta of the propagator, while
\texttt{IntermediateSubstitutions} contains relations for scalar
products appearing in those rules.

Another useful option is \texttt{LoopMomenta} which is particularly
helpful when converting mixed quadratic-eikonal propagators to quadratic
ones.

For propagators containing symbolic variables it might be necessary to
tell the function that those are larger than zero (if applicable), so
that expressions such as \(\sqrt{\lambda^2}\) can be simplified
accordingly. To that aim one should use the option \texttt{PowerExpand}.

\subsection{See also}

\hyperlink{toc}{Overview}, \hyperlink{gfad}{GFAD},
\hyperlink{sfad}{SFAD}, \hyperlink{cfad}{CFAD},
\hyperlink{feynampdenominatorexplicit}{FeynAmpDenominatorExplicit}.

\subsection{Examples}

\begin{Shaded}
\begin{Highlighting}[]
\NormalTok{GFAD}\OperatorTok{[}\NormalTok{SPD}\OperatorTok{[}\NormalTok{p1}\OperatorTok{]]} 
 
\NormalTok{ex }\ExtensionTok{=}\NormalTok{ FromGFAD}\OperatorTok{[}\SpecialCharTok{\%}\OperatorTok{]}
\end{Highlighting}
\end{Shaded}

\begin{dmath*}\breakingcomma
\text{GFAD}(\text{SPD}(\text{p1}))
\end{dmath*}

\begin{dmath*}\breakingcomma
\text{FromGFAD}(\text{GFAD}(\text{SPD}(\text{p1})))
\end{dmath*}

\begin{Shaded}
\begin{Highlighting}[]
\NormalTok{ex }\SpecialCharTok{//} \FunctionTok{StandardForm}

\CommentTok{(*FromGFAD[GFAD[SPD[p1]]]*)}
\end{Highlighting}
\end{Shaded}

\begin{Shaded}
\begin{Highlighting}[]
\NormalTok{ex }\ExtensionTok{=}\NormalTok{ GFAD}\OperatorTok{[}\NormalTok{SPD}\OperatorTok{[}\NormalTok{p1}\OperatorTok{]} \SpecialCharTok{+} \DecValTok{2}\NormalTok{ SPD}\OperatorTok{[}\NormalTok{p1}\OperatorTok{,}\NormalTok{ p2}\OperatorTok{]]}
\end{Highlighting}
\end{Shaded}

\begin{dmath*}\breakingcomma
\text{GFAD}(2 \;\text{SPD}(\text{p1},\text{p2})+\text{SPD}(\text{p1}))
\end{dmath*}

\begin{Shaded}
\begin{Highlighting}[]
\NormalTok{FromGFAD}\OperatorTok{[}\NormalTok{ex}\OperatorTok{]}
\end{Highlighting}
\end{Shaded}

\begin{dmath*}\breakingcomma
\text{FromGFAD}(\text{GFAD}(2 \;\text{SPD}(\text{p1},\text{p2})+\text{SPD}(\text{p1})))
\end{dmath*}

We can get a proper conversion into a quadratic propagator using the
option \texttt{LoopMomenta}. Notice that here \texttt{p2.p2} is being
put into the mass slot

\begin{Shaded}
\begin{Highlighting}[]
\NormalTok{FromGFAD}\OperatorTok{[}\NormalTok{ex}\OperatorTok{,}\NormalTok{ LoopMomenta }\OtherTok{{-}\textgreater{}} \OperatorTok{\{}\NormalTok{p1}\OperatorTok{\}]}
\end{Highlighting}
\end{Shaded}

\begin{dmath*}\breakingcomma
\text{FromGFAD}(\text{GFAD}(2 \;\text{SPD}(\text{p1},\text{p2})+\text{SPD}(\text{p1})),\text{LoopMomenta}\to \{\text{p1}\})
\end{dmath*}

\begin{Shaded}
\begin{Highlighting}[]
\NormalTok{ex }\SpecialCharTok{//} \FunctionTok{StandardForm}

\CommentTok{(*GFAD[SPD[p1] + 2 SPD[p1, p2]]*)}
\end{Highlighting}
\end{Shaded}

\begin{Shaded}
\begin{Highlighting}[]
\NormalTok{GFAD}\OperatorTok{[\{\{}\NormalTok{CSPD}\OperatorTok{[}\NormalTok{p1}\OperatorTok{]} \SpecialCharTok{+} \DecValTok{2}\NormalTok{ CSPD}\OperatorTok{[}\NormalTok{p1}\OperatorTok{,}\NormalTok{ p2}\OperatorTok{]} \SpecialCharTok{+} \FunctionTok{m}\SpecialCharTok{\^{}}\DecValTok{2}\OperatorTok{,} \SpecialCharTok{{-}}\DecValTok{1}\OperatorTok{\},} \DecValTok{2}\OperatorTok{\}]} 
 
\NormalTok{ex }\ExtensionTok{=}\NormalTok{ FromGFAD}\OperatorTok{[}\SpecialCharTok{\%}\OperatorTok{]}
\end{Highlighting}
\end{Shaded}

\begin{dmath*}\breakingcomma
\text{GFAD}\left(\left\{\left\{2 \;\text{CSPD}(\text{p1},\text{p2})+\text{CSPD}(\text{p1})+m^2,-1\right\},2\right\}\right)
\end{dmath*}

\begin{dmath*}\breakingcomma
\text{FromGFAD}\left(\text{GFAD}\left(\left\{\left\{2 \;\text{CSPD}(\text{p1},\text{p2})+\text{CSPD}(\text{p1})+m^2,-1\right\},2\right\}\right)\right)
\end{dmath*}

\begin{Shaded}
\begin{Highlighting}[]
\NormalTok{ex }\SpecialCharTok{//} \FunctionTok{StandardForm}

\CommentTok{(*FromGFAD[GFAD[\{\{m\^{}2 + CSPD[p1] + 2 CSPD[p1, p2], {-}1\}, 2\}]]*)}
\end{Highlighting}
\end{Shaded}

\begin{Shaded}
\begin{Highlighting}[]
\NormalTok{DataType}\OperatorTok{[}\NormalTok{la}\OperatorTok{,}\NormalTok{ FCVariable}\OperatorTok{]} \ExtensionTok{=} \ConstantTok{True}\NormalTok{;}
\NormalTok{prop }\ExtensionTok{=}\NormalTok{ FeynAmpDenominator}\OperatorTok{[}\NormalTok{GenericPropagatorDenominator}\OperatorTok{[}\SpecialCharTok{{-}}\NormalTok{la Pair}\OperatorTok{[}\NormalTok{Momentum}\OperatorTok{[}\NormalTok{p1}\OperatorTok{,} \FunctionTok{D}\OperatorTok{],} 
\NormalTok{       Momentum}\OperatorTok{[}\NormalTok{p1}\OperatorTok{,} \FunctionTok{D}\OperatorTok{]]} \SpecialCharTok{+} \DecValTok{2}\NormalTok{ Pair}\OperatorTok{[}\NormalTok{Momentum}\OperatorTok{[}\NormalTok{p1}\OperatorTok{,} \FunctionTok{D}\OperatorTok{],}\NormalTok{ Momentum}\OperatorTok{[}\FunctionTok{q}\OperatorTok{,} \FunctionTok{D}\OperatorTok{]],} \OperatorTok{\{}\DecValTok{1}\OperatorTok{,}\DecValTok{1}\OperatorTok{\}]]}
\end{Highlighting}
\end{Shaded}

\begin{dmath*}\breakingcomma
\text{FeynAmpDenominator}(\text{GenericPropagatorDenominator}(2 \;\text{Pair}(\text{Momentum}(\text{p1},D),\text{Momentum}(q,D))-\text{la} \;\text{Pair}(\text{Momentum}(\text{p1},D),\text{Momentum}(\text{p1},D)),\{1,1\}))
\end{dmath*}

\begin{Shaded}
\begin{Highlighting}[]
\NormalTok{ex }\ExtensionTok{=}\NormalTok{ FromGFAD}\OperatorTok{[}\NormalTok{prop}\OperatorTok{]}
\end{Highlighting}
\end{Shaded}

\begin{dmath*}\breakingcomma
\text{FromGFAD}(\text{FeynAmpDenominator}(\text{GenericPropagatorDenominator}(2 \;\text{Pair}(\text{Momentum}(\text{p1},D),\text{Momentum}(q,D))-\text{la} \;\text{Pair}(\text{Momentum}(\text{p1},D),\text{Momentum}(\text{p1},D)),\{1,1\})))
\end{dmath*}

\begin{Shaded}
\begin{Highlighting}[]
\NormalTok{ex }\ExtensionTok{=}\NormalTok{ FromGFAD}\OperatorTok{[}\NormalTok{prop}\OperatorTok{,}\NormalTok{ LoopMomenta }\OtherTok{{-}\textgreater{}} \OperatorTok{\{}\NormalTok{p1}\OperatorTok{\}]}
\end{Highlighting}
\end{Shaded}

\begin{dmath*}\breakingcomma
\text{FromGFAD}(\text{FeynAmpDenominator}(\text{GenericPropagatorDenominator}(2 \;\text{Pair}(\text{Momentum}(\text{p1},D),\text{Momentum}(q,D))-\text{la} \;\text{Pair}(\text{Momentum}(\text{p1},D),\text{Momentum}(\text{p1},D)),\{1,1\})),\text{LoopMomenta}\to \{\text{p1}\})
\end{dmath*}

\begin{Shaded}
\begin{Highlighting}[]
\NormalTok{ex }\ExtensionTok{=}\NormalTok{ GFAD}\OperatorTok{[\{\{}\SpecialCharTok{{-}}\NormalTok{SPD}\OperatorTok{[}\NormalTok{p1}\OperatorTok{,}\NormalTok{ p1}\OperatorTok{],} \DecValTok{1}\OperatorTok{\},} \DecValTok{1}\OperatorTok{\}]}\SpecialCharTok{*}\NormalTok{GFAD}\OperatorTok{[\{\{}\NormalTok{SPD}\OperatorTok{[}\NormalTok{p1}\OperatorTok{,} \SpecialCharTok{{-}}\NormalTok{p1 }\SpecialCharTok{+} \DecValTok{2}\SpecialCharTok{*}\NormalTok{p3}\OperatorTok{]} \SpecialCharTok{{-}}\NormalTok{ SPD}\OperatorTok{[}\NormalTok{p3}\OperatorTok{,}\NormalTok{ p3}\OperatorTok{],} \DecValTok{1}\OperatorTok{\},} \DecValTok{1}\OperatorTok{\}]}\SpecialCharTok{*}
\NormalTok{    GFAD}\OperatorTok{[\{\{}\SpecialCharTok{{-}}\NormalTok{SPD}\OperatorTok{[}\NormalTok{p3}\OperatorTok{,}\NormalTok{ p3}\OperatorTok{],} \DecValTok{1}\OperatorTok{\},} \DecValTok{1}\OperatorTok{\}]}\SpecialCharTok{*}\NormalTok{SFAD}\OperatorTok{[\{\{}\FunctionTok{I}\SpecialCharTok{*}\NormalTok{(p1 }\SpecialCharTok{+} \FunctionTok{q}\NormalTok{)}\OperatorTok{,} \DecValTok{0}\OperatorTok{\},} \OperatorTok{\{}\SpecialCharTok{{-}}\NormalTok{mb}\SpecialCharTok{\^{}}\DecValTok{2}\OperatorTok{,} \DecValTok{1}\OperatorTok{\},} \DecValTok{1}\OperatorTok{\}]}\SpecialCharTok{*}
\NormalTok{    SFAD}\OperatorTok{[\{\{}\FunctionTok{I}\SpecialCharTok{*}\NormalTok{(p3 }\SpecialCharTok{+} \FunctionTok{q}\NormalTok{)}\OperatorTok{,} \DecValTok{0}\OperatorTok{\},} \OperatorTok{\{}\SpecialCharTok{{-}}\NormalTok{mb}\SpecialCharTok{\^{}}\DecValTok{2}\OperatorTok{,} \DecValTok{1}\OperatorTok{\},} \DecValTok{1}\OperatorTok{\}]} \SpecialCharTok{+}\NormalTok{  (}\SpecialCharTok{{-}}\DecValTok{2}\SpecialCharTok{*}\NormalTok{mg}\SpecialCharTok{\^{}}\DecValTok{2}\SpecialCharTok{*}\NormalTok{GFAD}\OperatorTok{[\{\{}\SpecialCharTok{{-}}\NormalTok{SPD}\OperatorTok{[}\NormalTok{p1}\OperatorTok{,}\NormalTok{ p1}\OperatorTok{],} \DecValTok{1}\OperatorTok{\},} \DecValTok{2}\OperatorTok{\}]}\SpecialCharTok{*}
\NormalTok{       GFAD}\OperatorTok{[\{\{}\NormalTok{SPD}\OperatorTok{[}\NormalTok{p1}\OperatorTok{,} \SpecialCharTok{{-}}\NormalTok{p1 }\SpecialCharTok{+} \DecValTok{2}\SpecialCharTok{*}\NormalTok{p3}\OperatorTok{]} \SpecialCharTok{{-}}\NormalTok{ SPD}\OperatorTok{[}\NormalTok{p3}\OperatorTok{,}\NormalTok{ p3}\OperatorTok{],} \DecValTok{1}\OperatorTok{\},} \DecValTok{1}\OperatorTok{\}]}\SpecialCharTok{*}\NormalTok{GFAD}\OperatorTok{[\{\{}\SpecialCharTok{{-}}\NormalTok{SPD}\OperatorTok{[}\NormalTok{p3}\OperatorTok{,}\NormalTok{ p3}\OperatorTok{],} \DecValTok{1}\OperatorTok{\},} \DecValTok{1}\OperatorTok{\}]}\SpecialCharTok{*}
\NormalTok{       SFAD}\OperatorTok{[\{\{}\FunctionTok{I}\SpecialCharTok{*}\NormalTok{(p1 }\SpecialCharTok{+} \FunctionTok{q}\NormalTok{)}\OperatorTok{,} \DecValTok{0}\OperatorTok{\},} \OperatorTok{\{}\SpecialCharTok{{-}}\NormalTok{mb}\SpecialCharTok{\^{}}\DecValTok{2}\OperatorTok{,} \DecValTok{1}\OperatorTok{\},} \DecValTok{1}\OperatorTok{\}]}\SpecialCharTok{*}\NormalTok{SFAD}\OperatorTok{[\{\{}\FunctionTok{I}\SpecialCharTok{*}\NormalTok{(p3 }\SpecialCharTok{+} \FunctionTok{q}\NormalTok{)}\OperatorTok{,} \DecValTok{0}\OperatorTok{\},} \OperatorTok{\{}\SpecialCharTok{{-}}\NormalTok{mb}\SpecialCharTok{\^{}}\DecValTok{2}\OperatorTok{,} \DecValTok{1}\OperatorTok{\},} \DecValTok{1}\OperatorTok{\}]} \SpecialCharTok{{-}} 
        \DecValTok{2}\SpecialCharTok{*}\NormalTok{mg}\SpecialCharTok{\^{}}\DecValTok{2}\SpecialCharTok{*}\NormalTok{GFAD}\OperatorTok{[\{\{}\SpecialCharTok{{-}}\NormalTok{SPD}\OperatorTok{[}\NormalTok{p1}\OperatorTok{,}\NormalTok{ p1}\OperatorTok{],} \DecValTok{1}\OperatorTok{\},} \DecValTok{1}\OperatorTok{\}]}\SpecialCharTok{*}\NormalTok{GFAD}\OperatorTok{[\{\{}\NormalTok{SPD}\OperatorTok{[}\NormalTok{p1}\OperatorTok{,} \SpecialCharTok{{-}}\NormalTok{p1 }\SpecialCharTok{+} \DecValTok{2}\SpecialCharTok{*}\NormalTok{p3}\OperatorTok{]} \SpecialCharTok{{-}}\NormalTok{ SPD}\OperatorTok{[}\NormalTok{p3}\OperatorTok{,}\NormalTok{ p3}\OperatorTok{],} 
            \DecValTok{1}\OperatorTok{\},} \DecValTok{2}\OperatorTok{\}]}\SpecialCharTok{*}\NormalTok{GFAD}\OperatorTok{[\{\{}\SpecialCharTok{{-}}\NormalTok{SPD}\OperatorTok{[}\NormalTok{p3}\OperatorTok{,}\NormalTok{ p3}\OperatorTok{],} \DecValTok{1}\OperatorTok{\},} \DecValTok{1}\OperatorTok{\}]}\SpecialCharTok{*}\NormalTok{SFAD}\OperatorTok{[\{\{}\FunctionTok{I}\SpecialCharTok{*}\NormalTok{(p1 }\SpecialCharTok{+} \FunctionTok{q}\NormalTok{)}\OperatorTok{,} \DecValTok{0}\OperatorTok{\},} \OperatorTok{\{}\SpecialCharTok{{-}}\NormalTok{mb}\SpecialCharTok{\^{}}\DecValTok{2}\OperatorTok{,} \DecValTok{1}\OperatorTok{\},} \DecValTok{1}\OperatorTok{\}]}\SpecialCharTok{*}
\NormalTok{         SFAD}\OperatorTok{[\{\{}\FunctionTok{I}\SpecialCharTok{*}\NormalTok{(p3 }\SpecialCharTok{+} \FunctionTok{q}\NormalTok{)}\OperatorTok{,} \DecValTok{0}\OperatorTok{\},} \OperatorTok{\{}\SpecialCharTok{{-}}\NormalTok{mb}\SpecialCharTok{\^{}}\DecValTok{2}\OperatorTok{,} \DecValTok{1}\OperatorTok{\},} \DecValTok{1}\OperatorTok{\}]} \SpecialCharTok{{-}}    \DecValTok{2}\SpecialCharTok{*}\NormalTok{mg}\SpecialCharTok{\^{}}\DecValTok{2}\SpecialCharTok{*}\NormalTok{GFAD}\OperatorTok{[\{\{}\SpecialCharTok{{-}}\NormalTok{SPD}\OperatorTok{[}\NormalTok{p1}\OperatorTok{,}\NormalTok{ p1}\OperatorTok{],} \DecValTok{1}\OperatorTok{\},} 
           \DecValTok{1}\OperatorTok{\}]}\SpecialCharTok{*}\NormalTok{GFAD}\OperatorTok{[\{\{}\NormalTok{SPD}\OperatorTok{[}\NormalTok{p1}\OperatorTok{,} \SpecialCharTok{{-}}\NormalTok{p1 }\SpecialCharTok{+} \DecValTok{2}\SpecialCharTok{*}\NormalTok{p3}\OperatorTok{]} \SpecialCharTok{{-}}\NormalTok{ SPD}\OperatorTok{[}\NormalTok{p3}\OperatorTok{,}\NormalTok{ p3}\OperatorTok{],} \DecValTok{1}\OperatorTok{\},} \DecValTok{1}\OperatorTok{\}]}\SpecialCharTok{*}\NormalTok{GFAD}\OperatorTok{[\{\{}\SpecialCharTok{{-}}\NormalTok{SPD}\OperatorTok{[}\NormalTok{p3}\OperatorTok{,}\NormalTok{ p3}\OperatorTok{],} 
            \DecValTok{1}\OperatorTok{\},} \DecValTok{2}\OperatorTok{\}]}\SpecialCharTok{*}\NormalTok{SFAD}\OperatorTok{[\{\{}\FunctionTok{I}\SpecialCharTok{*}\NormalTok{(p1 }\SpecialCharTok{+} \FunctionTok{q}\NormalTok{)}\OperatorTok{,} \DecValTok{0}\OperatorTok{\},} \OperatorTok{\{}\SpecialCharTok{{-}}\NormalTok{mb}\SpecialCharTok{\^{}}\DecValTok{2}\OperatorTok{,} \DecValTok{1}\OperatorTok{\},} \DecValTok{1}\OperatorTok{\}]}\SpecialCharTok{*}\NormalTok{SFAD}\OperatorTok{[\{\{}\FunctionTok{I}\SpecialCharTok{*}\NormalTok{(p3 }\SpecialCharTok{+} \FunctionTok{q}\NormalTok{)}\OperatorTok{,} \DecValTok{0}\OperatorTok{\},} 
           \OperatorTok{\{}\SpecialCharTok{{-}}\NormalTok{mb}\SpecialCharTok{\^{}}\DecValTok{2}\OperatorTok{,} \DecValTok{1}\OperatorTok{\},} \DecValTok{1}\OperatorTok{\}]}\NormalTok{)}\SpecialCharTok{/}\DecValTok{2}
\end{Highlighting}
\end{Shaded}

\begin{dmath*}\breakingcomma
\frac{1}{2} \left(-2 \;\text{mg}^2 \;\text{GFAD}(\{\{-\text{SPD}(\text{p1},\text{p1}),1\},2\}) \;\text{GFAD}(\{\{-\text{SPD}(\text{p3},\text{p3}),1\},1\}) \;\text{GFAD}(\{\{\text{SPD}(\text{p1},2 \;\text{p3}-\text{p1})-\text{SPD}(\text{p3},\text{p3}),1\},1\}) \;\text{SFAD}\left(\left\{\{i (\text{p1}+q),0\},\left\{-\text{mb}^2,1\right\},1\right\}\right) \;\text{SFAD}\left(\left\{\{i (\text{p3}+q),0\},\left\{-\text{mb}^2,1\right\},1\right\}\right)-2 \;\text{mg}^2 \;\text{GFAD}(\{\{-\text{SPD}(\text{p1},\text{p1}),1\},1\}) \;\text{GFAD}(\{\{-\text{SPD}(\text{p3},\text{p3}),1\},1\}) \;\text{GFAD}(\{\{\text{SPD}(\text{p1},2 \;\text{p3}-\text{p1})-\text{SPD}(\text{p3},\text{p3}),1\},2\}) \;\text{SFAD}\left(\left\{\{i (\text{p1}+q),0\},\left\{-\text{mb}^2,1\right\},1\right\}\right) \;\text{SFAD}\left(\left\{\{i (\text{p3}+q),0\},\left\{-\text{mb}^2,1\right\},1\right\}\right)-2 \;\text{mg}^2 \;\text{GFAD}(\{\{-\text{SPD}(\text{p1},\text{p1}),1\},1\}) \;\text{GFAD}(\{\{-\text{SPD}(\text{p3},\text{p3}),1\},2\}) \;\text{GFAD}(\{\{\text{SPD}(\text{p1},2 \;\text{p3}-\text{p1})-\text{SPD}(\text{p3},\text{p3}),1\},1\}) \;\text{SFAD}\left(\left\{\{i (\text{p1}+q),0\},\left\{-\text{mb}^2,1\right\},1\right\}\right) \;\text{SFAD}\left(\left\{\{i (\text{p3}+q),0\},\left\{-\text{mb}^2,1\right\},1\right\}\right)\right)+\text{GFAD}(\{\{-\text{SPD}(\text{p1},\text{p1}),1\},1\}) \;\text{GFAD}(\{\{-\text{SPD}(\text{p3},\text{p3}),1\},1\}) \;\text{GFAD}(\{\{\text{SPD}(\text{p1},2 \;\text{p3}-\text{p1})-\text{SPD}(\text{p3},\text{p3}),1\},1\}) \;\text{SFAD}\left(\left\{\{i (\text{p1}+q),0\},\left\{-\text{mb}^2,1\right\},1\right\}\right) \;\text{SFAD}\left(\left\{\{i (\text{p3}+q),0\},\left\{-\text{mb}^2,1\right\},1\right\}\right)
\end{dmath*}

Notice that \texttt{FromGFAD} does not expand scalar products in the
propagators before trying to convert them to \texttt{SFAD}s or
\texttt{CFAD}s. If this is needed, the user should better apply
\texttt{ExpandScalarProduct} to the expression by hand.

\begin{Shaded}
\begin{Highlighting}[]
\NormalTok{FromGFAD}\OperatorTok{[}\NormalTok{ex}\OperatorTok{]}
\end{Highlighting}
\end{Shaded}

\begin{dmath*}\breakingcomma
\text{FromGFAD}\left(\frac{1}{2} \left(-2 \;\text{mg}^2 \;\text{GFAD}(\{\{-\text{SPD}(\text{p1},\text{p1}),1\},2\}) \;\text{GFAD}(\{\{-\text{SPD}(\text{p3},\text{p3}),1\},1\}) \;\text{GFAD}(\{\{\text{SPD}(\text{p1},2 \;\text{p3}-\text{p1})-\text{SPD}(\text{p3},\text{p3}),1\},1\}) \;\text{SFAD}\left(\left\{\{i (\text{p1}+q),0\},\left\{-\text{mb}^2,1\right\},1\right\}\right) \;\text{SFAD}\left(\left\{\{i (\text{p3}+q),0\},\left\{-\text{mb}^2,1\right\},1\right\}\right)-2 \;\text{mg}^2 \;\text{GFAD}(\{\{-\text{SPD}(\text{p1},\text{p1}),1\},1\}) \;\text{GFAD}(\{\{-\text{SPD}(\text{p3},\text{p3}),1\},1\}) \;\text{GFAD}(\{\{\text{SPD}(\text{p1},2 \;\text{p3}-\text{p1})-\text{SPD}(\text{p3},\text{p3}),1\},2\}) \;\text{SFAD}\left(\left\{\{i (\text{p1}+q),0\},\left\{-\text{mb}^2,1\right\},1\right\}\right) \;\text{SFAD}\left(\left\{\{i (\text{p3}+q),0\},\left\{-\text{mb}^2,1\right\},1\right\}\right)-2 \;\text{mg}^2 \;\text{GFAD}(\{\{-\text{SPD}(\text{p1},\text{p1}),1\},1\}) \;\text{GFAD}(\{\{-\text{SPD}(\text{p3},\text{p3}),1\},2\}) \;\text{GFAD}(\{\{\text{SPD}(\text{p1},2 \;\text{p3}-\text{p1})-\text{SPD}(\text{p3},\text{p3}),1\},1\}) \;\text{SFAD}\left(\left\{\{i (\text{p1}+q),0\},\left\{-\text{mb}^2,1\right\},1\right\}\right) \;\text{SFAD}\left(\left\{\{i (\text{p3}+q),0\},\left\{-\text{mb}^2,1\right\},1\right\}\right)\right)+\text{GFAD}(\{\{-\text{SPD}(\text{p1},\text{p1}),1\},1\}) \;\text{GFAD}(\{\{-\text{SPD}(\text{p3},\text{p3}),1\},1\}) \;\text{GFAD}(\{\{\text{SPD}(\text{p1},2 \;\text{p3}-\text{p1})-\text{SPD}(\text{p3},\text{p3}),1\},1\}) \;\text{SFAD}\left(\left\{\{i (\text{p1}+q),0\},\left\{-\text{mb}^2,1\right\},1\right\}\right) \;\text{SFAD}\left(\left\{\{i (\text{p3}+q),0\},\left\{-\text{mb}^2,1\right\},1\right\}\right)\right)
\end{dmath*}

\begin{Shaded}
\begin{Highlighting}[]
\NormalTok{FromGFAD}\OperatorTok{[}\NormalTok{ExpandScalarProduct}\OperatorTok{[}\NormalTok{ex}\OperatorTok{]]}
\end{Highlighting}
\end{Shaded}

\begin{dmath*}\breakingcomma
\text{FromGFAD}\left(\text{ExpandScalarProduct}\left(\frac{1}{2} \left(-2 \;\text{mg}^2 \;\text{GFAD}(\{\{-\text{SPD}(\text{p1},\text{p1}),1\},2\}) \;\text{GFAD}(\{\{-\text{SPD}(\text{p3},\text{p3}),1\},1\}) \;\text{GFAD}(\{\{\text{SPD}(\text{p1},2 \;\text{p3}-\text{p1})-\text{SPD}(\text{p3},\text{p3}),1\},1\}) \;\text{SFAD}\left(\left\{\{i (\text{p1}+q),0\},\left\{-\text{mb}^2,1\right\},1\right\}\right) \;\text{SFAD}\left(\left\{\{i (\text{p3}+q),0\},\left\{-\text{mb}^2,1\right\},1\right\}\right)-2 \;\text{mg}^2 \;\text{GFAD}(\{\{-\text{SPD}(\text{p1},\text{p1}),1\},1\}) \;\text{GFAD}(\{\{-\text{SPD}(\text{p3},\text{p3}),1\},1\}) \;\text{GFAD}(\{\{\text{SPD}(\text{p1},2 \;\text{p3}-\text{p1})-\text{SPD}(\text{p3},\text{p3}),1\},2\}) \;\text{SFAD}\left(\left\{\{i (\text{p1}+q),0\},\left\{-\text{mb}^2,1\right\},1\right\}\right) \;\text{SFAD}\left(\left\{\{i (\text{p3}+q),0\},\left\{-\text{mb}^2,1\right\},1\right\}\right)-2 \;\text{mg}^2 \;\text{GFAD}(\{\{-\text{SPD}(\text{p1},\text{p1}),1\},1\}) \;\text{GFAD}(\{\{-\text{SPD}(\text{p3},\text{p3}),1\},2\}) \;\text{GFAD}(\{\{\text{SPD}(\text{p1},2 \;\text{p3}-\text{p1})-\text{SPD}(\text{p3},\text{p3}),1\},1\}) \;\text{SFAD}\left(\left\{\{i (\text{p1}+q),0\},\left\{-\text{mb}^2,1\right\},1\right\}\right) \;\text{SFAD}\left(\left\{\{i (\text{p3}+q),0\},\left\{-\text{mb}^2,1\right\},1\right\}\right)\right)+\text{GFAD}(\{\{-\text{SPD}(\text{p1},\text{p1}),1\},1\}) \;\text{GFAD}(\{\{-\text{SPD}(\text{p3},\text{p3}),1\},1\}) \;\text{GFAD}(\{\{\text{SPD}(\text{p1},2 \;\text{p3}-\text{p1})-\text{SPD}(\text{p3},\text{p3}),1\},1\}) \;\text{SFAD}\left(\left\{\{i (\text{p1}+q),0\},\left\{-\text{mb}^2,1\right\},1\right\}\right) \;\text{SFAD}\left(\left\{\{i (\text{p3}+q),0\},\left\{-\text{mb}^2,1\right\},1\right\}\right)\right)\right)
\end{dmath*}

Using the option \texttt{InitialSubstitutions} one can perform certain
replacement that might not be found automatically. The values of scalar
products can be set using \texttt{IntermediateSubstitutions}

\begin{Shaded}
\begin{Highlighting}[]
\NormalTok{ex }\ExtensionTok{=}\NormalTok{ GFAD}\OperatorTok{[\{\{}\NormalTok{SPD}\OperatorTok{[}\NormalTok{k1}\OperatorTok{,}\NormalTok{ k1}\OperatorTok{]} \SpecialCharTok{{-}} \DecValTok{2}\SpecialCharTok{*}\NormalTok{gkin}\SpecialCharTok{*}\NormalTok{meta}\SpecialCharTok{*}\NormalTok{u0b}\SpecialCharTok{*}\NormalTok{SPD}\OperatorTok{[}\NormalTok{k1}\OperatorTok{,} \FunctionTok{n}\OperatorTok{],} \DecValTok{1}\OperatorTok{\},} \DecValTok{1}\OperatorTok{\}]}\NormalTok{;}
\end{Highlighting}
\end{Shaded}

Notice that we need to declare the appearing variables as
\texttt{FCVariable}s

\begin{Shaded}
\begin{Highlighting}[]
\NormalTok{(DataType}\OperatorTok{[}\NormalTok{\#}\OperatorTok{,}\NormalTok{ FCVariable}\OperatorTok{]} \ExtensionTok{=} \ConstantTok{True}\NormalTok{) \& }\SpecialCharTok{/}\NormalTok{@ }\OperatorTok{\{}\NormalTok{gkin}\OperatorTok{,}\NormalTok{ meta}\OperatorTok{,}\NormalTok{ u0b}\OperatorTok{\}}\NormalTok{;}
\end{Highlighting}
\end{Shaded}

Without these options we get a mixed quadratic-eikonal propagator that
will cause us troubles when doing topology minimizations.

\begin{Shaded}
\begin{Highlighting}[]
\NormalTok{FromGFAD}\OperatorTok{[}\NormalTok{ex}\OperatorTok{,}\NormalTok{ FCE }\OtherTok{{-}\textgreater{}} \ConstantTok{True}\OperatorTok{]}
\SpecialCharTok{\%} \SpecialCharTok{//} \FunctionTok{InputForm}
\end{Highlighting}
\end{Shaded}

\begin{dmath*}\breakingcomma
\text{FromGFAD}(\text{GFAD}(\{\{\text{SPD}(\text{k1},\text{k1})-2 \;\text{gkin} \;\text{meta} \;\text{u0b} \;\text{SPD}(\text{k1},n),1\},1\}),\text{FCE}\to \;\text{True})
\end{dmath*}

\begin{Shaded}
\begin{Highlighting}[]
\NormalTok{FromGFAD}\OperatorTok{[}\NormalTok{GFAD}\OperatorTok{[\{\{}\NormalTok{SPD}\OperatorTok{[}\NormalTok{k1}\OperatorTok{,}\NormalTok{ k1}\OperatorTok{]} \SpecialCharTok{{-}} \DecValTok{2}\SpecialCharTok{*}\NormalTok{gkin}\SpecialCharTok{*}\NormalTok{meta}\SpecialCharTok{*}\NormalTok{u0b}\SpecialCharTok{*}\NormalTok{SPD}\OperatorTok{[}\NormalTok{k1}\OperatorTok{,} \FunctionTok{n}\OperatorTok{],} \DecValTok{1}\OperatorTok{\},} \DecValTok{1}\OperatorTok{\}],} 
\NormalTok{ FCE }\OtherTok{{-}\textgreater{}} \ConstantTok{True}\OperatorTok{]}
\end{Highlighting}
\end{Shaded}

But when doing everything right we end up with a purely quadratic
propagator

\begin{Shaded}
\begin{Highlighting}[]
\NormalTok{FromGFAD}\OperatorTok{[}\NormalTok{ex}\OperatorTok{,}\NormalTok{ InitialSubstitutions }\OtherTok{{-}\textgreater{}} \OperatorTok{\{}\NormalTok{ExpandScalarProduct}\OperatorTok{[}\NormalTok{SPD}\OperatorTok{[}\NormalTok{k1 }\SpecialCharTok{{-}}\NormalTok{ gkin meta u0b }\FunctionTok{n}\OperatorTok{]]} \OtherTok{{-}\textgreater{}}\NormalTok{ SPD}\OperatorTok{[}\NormalTok{k1 }\SpecialCharTok{{-}}\NormalTok{ gkin meta u0b }\FunctionTok{n}\OperatorTok{]\},} 
\NormalTok{  IntermediateSubstitutions }\OtherTok{{-}\textgreater{}} \OperatorTok{\{}\NormalTok{SPD}\OperatorTok{[}\FunctionTok{n}\OperatorTok{]} \OtherTok{{-}\textgreater{}} \DecValTok{0}\OperatorTok{,}\NormalTok{ SPD}\OperatorTok{[}\NormalTok{nb}\OperatorTok{]} \OtherTok{{-}\textgreater{}} \DecValTok{0}\OperatorTok{,}\NormalTok{ SPD}\OperatorTok{[}\FunctionTok{n}\OperatorTok{,}\NormalTok{ nb}\OperatorTok{]} \OtherTok{{-}\textgreater{}} \DecValTok{2}\OperatorTok{\}]}
\end{Highlighting}
\end{Shaded}

\begin{dmath*}\breakingcomma
\text{FromGFAD}(\text{GFAD}(\{\{\text{SPD}(\text{k1},\text{k1})-2 \;\text{gkin} \;\text{meta} \;\text{u0b} \;\text{SPD}(\text{k1},n),1\},1\}),\text{InitialSubstitutions}\to \{\text{ExpandScalarProduct}(\text{SPD}(\text{k1}-\text{gkin} \;\text{meta} n \;\text{u0b}))\to \;\text{SPD}(\text{k1}-\text{gkin} \;\text{meta} n \;\text{u0b})\},\text{IntermediateSubstitutions}\to \{\text{SPD}(n)\to 0,\text{SPD}(\text{nb})\to 0,\text{SPD}(n,\text{nb})\to 2\})
\end{dmath*}

However, in this case the function can also figure out the necessary
square completion on its own if we tell it that \texttt{k1} is a
momentum w.r.t which the square should be completed. In this case the
option \texttt{IntermediateSubstitutions} is not really needed

\begin{Shaded}
\begin{Highlighting}[]
\NormalTok{FromGFAD}\OperatorTok{[}\NormalTok{ex}\OperatorTok{,}\NormalTok{ LoopMomenta }\OtherTok{{-}\textgreater{}} \OperatorTok{\{}\NormalTok{k1}\OperatorTok{\}]}
\end{Highlighting}
\end{Shaded}

\begin{dmath*}\breakingcomma
\text{FromGFAD}(\text{GFAD}(\{\{\text{SPD}(\text{k1},\text{k1})-2 \;\text{gkin} \;\text{meta} \;\text{u0b} \;\text{SPD}(\text{k1},n),1\},1\}),\text{LoopMomenta}\to \{\text{k1}\})
\end{dmath*}

It is still helpful, though

\begin{Shaded}
\begin{Highlighting}[]
\NormalTok{FromGFAD}\OperatorTok{[}\NormalTok{ex}\OperatorTok{,}\NormalTok{ LoopMomenta }\OtherTok{{-}\textgreater{}} \OperatorTok{\{}\NormalTok{k1}\OperatorTok{\},}\NormalTok{ IntermediateSubstitutions }\OtherTok{{-}\textgreater{}} \OperatorTok{\{}\NormalTok{SPD}\OperatorTok{[}\FunctionTok{n}\OperatorTok{]} \OtherTok{{-}\textgreater{}} \DecValTok{0}\OperatorTok{,}\NormalTok{ SPD}\OperatorTok{[}\NormalTok{nb}\OperatorTok{]} \OtherTok{{-}\textgreater{}} \DecValTok{0}\OperatorTok{,}\NormalTok{ SPD}\OperatorTok{[}\FunctionTok{n}\OperatorTok{,}\NormalTok{ nb}\OperatorTok{]} \OtherTok{{-}\textgreater{}} \DecValTok{2}\OperatorTok{\}]}
\end{Highlighting}
\end{Shaded}

\begin{dmath*}\breakingcomma
\text{FromGFAD}(\text{GFAD}(\{\{\text{SPD}(\text{k1},\text{k1})-2 \;\text{gkin} \;\text{meta} \;\text{u0b} \;\text{SPD}(\text{k1},n),1\},1\}),\text{LoopMomenta}\to \{\text{k1}\},\text{IntermediateSubstitutions}\to \{\text{SPD}(n)\to 0,\text{SPD}(\text{nb})\to 0,\text{SPD}(n,\text{nb})\to 2\})
\end{dmath*}

If we have multiple loop momenta, we need to first complete the square
with respect to them before handling the full expression

\begin{Shaded}
\begin{Highlighting}[]
\NormalTok{ex }\ExtensionTok{=}\NormalTok{ GFAD}\OperatorTok{[\{\{}\NormalTok{SPD}\OperatorTok{[}\NormalTok{k1}\OperatorTok{,}\NormalTok{ k1}\OperatorTok{]} \SpecialCharTok{+} \DecValTok{2}\NormalTok{ SPD}\OperatorTok{[}\NormalTok{k1}\OperatorTok{,}\NormalTok{ k2}\OperatorTok{]} \SpecialCharTok{+}\NormalTok{ SPD}\OperatorTok{[}\NormalTok{k2}\OperatorTok{,}\NormalTok{ k2}\OperatorTok{]} \SpecialCharTok{+} \DecValTok{2}\NormalTok{ gkin meta (SPD}\OperatorTok{[}\NormalTok{k1}\OperatorTok{,} \FunctionTok{n}\OperatorTok{]} \SpecialCharTok{+}\NormalTok{ SPD}\OperatorTok{[}\NormalTok{k2}\OperatorTok{,} \FunctionTok{n}\OperatorTok{]}\NormalTok{)}\OperatorTok{,} \DecValTok{1}\OperatorTok{\},} \DecValTok{1}\OperatorTok{\}]}
\end{Highlighting}
\end{Shaded}

\begin{dmath*}\breakingcomma
\text{GFAD}(\{\{2 \;\text{gkin} \;\text{meta} (\text{SPD}(\text{k1},n)+\text{SPD}(\text{k2},n))+2 \;\text{SPD}(\text{k1},\text{k2})+\text{SPD}(\text{k1},\text{k1})+\text{SPD}(\text{k2},\text{k2}),1\},1\})
\end{dmath*}

\begin{Shaded}
\begin{Highlighting}[]
\NormalTok{FromGFAD}\OperatorTok{[}\NormalTok{ex}\OperatorTok{,}\NormalTok{ LoopMomenta }\OtherTok{{-}\textgreater{}} \OperatorTok{\{}\NormalTok{k1}\OperatorTok{,}\NormalTok{ k2}\OperatorTok{\}]}
\end{Highlighting}
\end{Shaded}

\begin{dmath*}\breakingcomma
\text{FromGFAD}(\text{GFAD}(\{\{2 \;\text{gkin} \;\text{meta} (\text{SPD}(\text{k1},n)+\text{SPD}(\text{k2},n))+2 \;\text{SPD}(\text{k1},\text{k2})+\text{SPD}(\text{k1},\text{k1})+\text{SPD}(\text{k2},\text{k2}),1\},1\}),\text{LoopMomenta}\to \{\text{k1},\text{k2}\})
\end{dmath*}

\begin{Shaded}
\begin{Highlighting}[]
\NormalTok{FromGFAD}\OperatorTok{[}\NormalTok{ex}\OperatorTok{,}\NormalTok{ LoopMomenta }\OtherTok{{-}\textgreater{}} \OperatorTok{\{}\NormalTok{k1}\OperatorTok{,}\NormalTok{ k2}\OperatorTok{\},} 
\NormalTok{  InitialSubstitutions }\OtherTok{{-}\textgreater{}} \OperatorTok{\{}\NormalTok{ExpandScalarProduct}\OperatorTok{[}\NormalTok{SPD}\OperatorTok{[}\NormalTok{k1 }\SpecialCharTok{+}\NormalTok{ k2}\OperatorTok{]]} \OtherTok{{-}\textgreater{}}\NormalTok{ SPD}\OperatorTok{[}\NormalTok{k1 }\SpecialCharTok{+}\NormalTok{ k2}\OperatorTok{]\},} 
\NormalTok{  IntermediateSubstitutions }\OtherTok{{-}\textgreater{}} \OperatorTok{\{}\NormalTok{SPD}\OperatorTok{[}\FunctionTok{n}\OperatorTok{]} \OtherTok{{-}\textgreater{}} \DecValTok{0}\OperatorTok{\}]}
\end{Highlighting}
\end{Shaded}

\begin{dmath*}\breakingcomma
\text{FromGFAD}(\text{GFAD}(\{\{2 \;\text{gkin} \;\text{meta} (\text{SPD}(\text{k1},n)+\text{SPD}(\text{k2},n))+2 \;\text{SPD}(\text{k1},\text{k2})+\text{SPD}(\text{k1},\text{k1})+\text{SPD}(\text{k2},\text{k2}),1\},1\}),\text{LoopMomenta}\to \{\text{k1},\text{k2}\},\text{InitialSubstitutions}\to \{\text{ExpandScalarProduct}(\text{SPD}(\text{k1}+\text{k2}))\to \;\text{SPD}(\text{k1}+\text{k2})\},\text{IntermediateSubstitutions}\to \{\text{SPD}(n)\to 0\})
\end{dmath*}
\end{document}
