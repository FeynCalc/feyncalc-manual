% !TeX program = pdflatex
% !TeX root = GALND.tex

\documentclass[../FeynCalcManual.tex]{subfiles}
\begin{document}
\hypertarget{galnd}{
\section{GALND}\label{galnd}\index{GALND}}

\texttt{GALND[\allowbreak{}mu,\ \allowbreak{}n,\ \allowbreak{}nb]}
denotes the negative component in the lightcone decomposition of the
Dirac matrix \(\gamma^{\mu }\) along the vectors \texttt{n} and
\texttt{nb} in \(D\)-dimensions. It corresponds to
\(\frac{1}{2} n^{\mu} (\gamma \cdot \bar{n})\).

If one omits \texttt{n} and \texttt{nb}, the program will use default
vectors specified via \texttt{\$FCDefaultLightconeVectorN} and
\texttt{\$FCDefaultLightconeVectorNB}.

\subsection{See also}

\hyperlink{toc}{Overview}, \hyperlink{diracgamma}{DiracGamma},
\hyperlink{galpd}{GALPD}, \hyperlink{galrd}{GALRD},
\hyperlink{gslpd}{GSLPD}, \hyperlink{gslnd}{GSLND},
\hyperlink{gslrd}{GSLRD}.

\subsection{Examples}

\begin{Shaded}
\begin{Highlighting}[]
\NormalTok{GALND}\OperatorTok{[}\SpecialCharTok{\textbackslash{}}\OperatorTok{[}\NormalTok{Mu}\OperatorTok{],} \FunctionTok{n}\OperatorTok{,}\NormalTok{ nb}\OperatorTok{]}
\end{Highlighting}
\end{Shaded}

\begin{dmath*}\breakingcomma
\frac{1}{2} n^{\mu } \gamma \cdot \;\text{nb}
\end{dmath*}

\begin{Shaded}
\begin{Highlighting}[]
\FunctionTok{StandardForm}\OperatorTok{[}\NormalTok{GALND}\OperatorTok{[}\SpecialCharTok{\textbackslash{}}\OperatorTok{[}\NormalTok{Mu}\OperatorTok{],} \FunctionTok{n}\OperatorTok{,}\NormalTok{ nb}\OperatorTok{]} \SpecialCharTok{//}\NormalTok{ FCI}\OperatorTok{]}
\end{Highlighting}
\end{Shaded}

\begin{dmath*}\breakingcomma
\frac{1}{2} \;\text{DiracGamma}[\text{Momentum}[\text{nb},D],D] \;\text{Pair}[\text{LorentzIndex}[\mu ,D],\text{Momentum}[n,D]]
\end{dmath*}

Notice that the properties of \texttt{n} and \texttt{nb} vectors have to
be set by hand before doing the actual computation

\begin{Shaded}
\begin{Highlighting}[]
\NormalTok{GALND}\OperatorTok{[}\SpecialCharTok{\textbackslash{}}\OperatorTok{[}\NormalTok{Mu}\OperatorTok{],} \FunctionTok{n}\OperatorTok{,}\NormalTok{ nb}\OperatorTok{]}\NormalTok{ . GALND}\OperatorTok{[}\SpecialCharTok{\textbackslash{}}\OperatorTok{[}\NormalTok{Nu}\OperatorTok{],} \FunctionTok{n}\OperatorTok{,}\NormalTok{ nb}\OperatorTok{]} \SpecialCharTok{//}\NormalTok{ DiracSimplify}
\end{Highlighting}
\end{Shaded}

\begin{dmath*}\breakingcomma
\frac{1}{4} \;\text{nb}^2 n^{\mu } n^{\nu }
\end{dmath*}

\begin{Shaded}
\begin{Highlighting}[]
\NormalTok{FCClearScalarProducts}\OperatorTok{[]}
\NormalTok{SPD}\OperatorTok{[}\FunctionTok{n}\OperatorTok{]} \ExtensionTok{=} \DecValTok{0}\NormalTok{;}
\NormalTok{SPD}\OperatorTok{[}\NormalTok{nb}\OperatorTok{]} \ExtensionTok{=} \DecValTok{0}\NormalTok{;}
\NormalTok{SPD}\OperatorTok{[}\FunctionTok{n}\OperatorTok{,}\NormalTok{ nb}\OperatorTok{]} \ExtensionTok{=} \DecValTok{2}\NormalTok{;}
\end{Highlighting}
\end{Shaded}

\begin{Shaded}
\begin{Highlighting}[]
\NormalTok{GALND}\OperatorTok{[}\SpecialCharTok{\textbackslash{}}\OperatorTok{[}\NormalTok{Mu}\OperatorTok{],} \FunctionTok{n}\OperatorTok{,}\NormalTok{ nb}\OperatorTok{]}\NormalTok{ . GALND}\OperatorTok{[}\SpecialCharTok{\textbackslash{}}\OperatorTok{[}\NormalTok{Nu}\OperatorTok{],} \FunctionTok{n}\OperatorTok{,}\NormalTok{ nb}\OperatorTok{]} \SpecialCharTok{//}\NormalTok{ DiracSimplify}
\end{Highlighting}
\end{Shaded}

\begin{dmath*}\breakingcomma
0
\end{dmath*}

\begin{Shaded}
\begin{Highlighting}[]
\NormalTok{FCClearScalarProducts}\OperatorTok{[]}
\end{Highlighting}
\end{Shaded}

\end{document}
