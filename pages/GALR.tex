% !TeX program = pdflatex
% !TeX root = GALR.tex

\documentclass[../FeynCalcManual.tex]{subfiles}
\begin{document}
\hypertarget{galr}{
\section{GALR}\label{galr}\index{GALR}}

\texttt{GALR[\allowbreak{}mu,\ \allowbreak{}n,\ \allowbreak{}nb]}
denotes the perpendicular component in the lightcone decomposition of
the Dirac matrix \(\gamma^{\mu }\) along the vectors \texttt{n} and
\texttt{nb}. It corresponds to \(\gamma^{\mu}_{\perp}\).

If one omits \texttt{n} and \texttt{nb}, the program will use default
vectors specified via \texttt{\$FCDefaultLightconeVectorN} and
\texttt{\$FCDefaultLightconeVectorNB}.

\subsection{See also}

\hyperlink{toc}{Overview}, \hyperlink{diracgamma}{DiracGamma},
\hyperlink{galn}{GALN}, \hyperlink{galp}{GALP}, \hyperlink{gslp}{GSLP},
\hyperlink{gsln}{GSLN}, \hyperlink{gslr}{GSLR}.

\subsection{Examples}

\begin{Shaded}
\begin{Highlighting}[]
\NormalTok{GALR}\OperatorTok{[}\SpecialCharTok{\textbackslash{}}\OperatorTok{[}\NormalTok{Mu}\OperatorTok{],} \FunctionTok{n}\OperatorTok{,}\NormalTok{ nb}\OperatorTok{]}
\end{Highlighting}
\end{Shaded}

\begin{dmath*}\breakingcomma
\bar{\gamma }^{\mu }{}_{\perp }
\end{dmath*}

\begin{Shaded}
\begin{Highlighting}[]
\FunctionTok{StandardForm}\OperatorTok{[}\NormalTok{GALR}\OperatorTok{[}\SpecialCharTok{\textbackslash{}}\OperatorTok{[}\NormalTok{Mu}\OperatorTok{],} \FunctionTok{n}\OperatorTok{,}\NormalTok{ nb}\OperatorTok{]} \SpecialCharTok{//}\NormalTok{ FCI}\OperatorTok{]}

\CommentTok{(*DiracGamma[LightConePerpendicularComponent[LorentzIndex[\textbackslash{}[Mu]], Momentum[n], Momentum[nb]]]*)}
\end{Highlighting}
\end{Shaded}

Notice that the properties of \texttt{n} and \texttt{nb} vectors have to
be set by hand before doing the actual computation

\begin{Shaded}
\begin{Highlighting}[]
\NormalTok{GALR}\OperatorTok{[}\SpecialCharTok{\textbackslash{}}\OperatorTok{[}\NormalTok{Mu}\OperatorTok{],} \FunctionTok{n}\OperatorTok{,}\NormalTok{ nb}\OperatorTok{]}\NormalTok{ . GALP}\OperatorTok{[}\SpecialCharTok{\textbackslash{}}\OperatorTok{[}\NormalTok{Nu}\OperatorTok{],} \FunctionTok{n}\OperatorTok{,}\NormalTok{ nb}\OperatorTok{]} \SpecialCharTok{//}\NormalTok{ iracSimplify}
\end{Highlighting}
\end{Shaded}

\begin{dmath*}\breakingcomma
\text{iracSimplify}\left(\bar{\gamma }^{\mu }{}_{\perp }.\left(\frac{1}{2} \overline{\text{nb}}^{\nu } \bar{\gamma }\cdot \overline{n}\right)\right)
\end{dmath*}

\begin{Shaded}
\begin{Highlighting}[]
\NormalTok{FCClearScalarProducts}\OperatorTok{[]}
\NormalTok{SP}\OperatorTok{[}\FunctionTok{n}\OperatorTok{]} \ExtensionTok{=} \DecValTok{0}\NormalTok{;}
\NormalTok{SP}\OperatorTok{[}\NormalTok{nb}\OperatorTok{]} \ExtensionTok{=} \DecValTok{0}\NormalTok{;}
\NormalTok{SP}\OperatorTok{[}\FunctionTok{n}\OperatorTok{,}\NormalTok{ nb}\OperatorTok{]} \ExtensionTok{=} \DecValTok{2}\NormalTok{;}
\end{Highlighting}
\end{Shaded}

\begin{Shaded}
\begin{Highlighting}[]
\NormalTok{GALR}\OperatorTok{[}\SpecialCharTok{\textbackslash{}}\OperatorTok{[}\NormalTok{Mu}\OperatorTok{],} \FunctionTok{n}\OperatorTok{,}\NormalTok{ nb}\OperatorTok{]}\NormalTok{ . GALP}\OperatorTok{[}\SpecialCharTok{\textbackslash{}}\OperatorTok{[}\NormalTok{Nu}\OperatorTok{],} \FunctionTok{n}\OperatorTok{,}\NormalTok{ nb}\OperatorTok{]} \SpecialCharTok{//}\NormalTok{ DiracSimplify}
\end{Highlighting}
\end{Shaded}

\begin{dmath*}\breakingcomma
-\frac{1}{2} \overline{\text{nb}}^{\nu } \left(\bar{\gamma }\cdot \overline{n}\right).\bar{\gamma }^{\mu }{}_{\perp }
\end{dmath*}

\begin{Shaded}
\begin{Highlighting}[]
\NormalTok{FCClearScalarProducts}\OperatorTok{[]}
\end{Highlighting}
\end{Shaded}

\end{document}
