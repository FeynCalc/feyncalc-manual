% !TeX program = pdflatex
% !TeX root = GFAD.tex

\documentclass[../FeynCalcManual.tex]{subfiles}
\begin{document}
\hypertarget{gfad}{
\section{GFAD}\label{gfad}\index{GFAD}}

\texttt{GFAD[\allowbreak{}\{\allowbreak{}\{\allowbreak{}\{\allowbreak{}x,\ \allowbreak{}s\},\ \allowbreak{}n\},\ \allowbreak{}...]}
denotes a generic propagator given by \(\frac{1}{[x + s i \eta]^n}\),
where \texttt{x} can be an arbitrary expression. For brevity one can
also use shorter forms such as
\texttt{GFAD[\allowbreak{}\{\allowbreak{}x,\ \allowbreak{}n\},\ \allowbreak{}...]},
\texttt{GFAD[\allowbreak{}\{\allowbreak{}x\},\ \allowbreak{}...]} or
\texttt{GFAD[\allowbreak{}x,\ \allowbreak{}...]}.

If s is not explicitly specified, then its value is determined by the
option \texttt{EtaSign}, which has the default value \texttt{+1}.

If \texttt{n} is not explicitly specified, then the default value
\texttt{1} is assumed. Translation into FeynCalc internal form is
performed by \texttt{FeynCalcInternal}, where a \texttt{GFAD} is encoded
using the special head \texttt{GenericPropagatorDenominator}.

\subsection{See also}

\hyperlink{toc}{Overview}, \hyperlink{fad}{FAD}, \hyperlink{sfad}{SFAD},
\hyperlink{cfad}{CFAD}.

\subsection{Examples}

\begin{Shaded}
\begin{Highlighting}[]
\NormalTok{GFAD}\OperatorTok{[}\DecValTok{2} \FunctionTok{z}\NormalTok{ SPD}\OperatorTok{[}\NormalTok{p1}\OperatorTok{,} \FunctionTok{q}\OperatorTok{]}\NormalTok{ SPD}\OperatorTok{[}\NormalTok{p2}\OperatorTok{,} \FunctionTok{q}\OperatorTok{]} \SpecialCharTok{+} \FunctionTok{x}\NormalTok{ SPD}\OperatorTok{[}\NormalTok{p1}\OperatorTok{,}\NormalTok{ p2}\OperatorTok{]]} 
 
\NormalTok{FeynAmpDenominatorExplicit}\OperatorTok{[}\SpecialCharTok{\%}\OperatorTok{]} 
 
\SpecialCharTok{\%} \SpecialCharTok{//}\NormalTok{ FCE }\SpecialCharTok{//} \FunctionTok{StandardForm}
\end{Highlighting}
\end{Shaded}

\begin{dmath*}\breakingcomma
\frac{1}{(x (\text{p1}\cdot \;\text{p2})+2 z (\text{p1}\cdot q) (\text{p2}\cdot q)+i \eta )}
\end{dmath*}

\begin{dmath*}\breakingcomma
\frac{1}{2 z (\text{p1}\cdot q) (\text{p2}\cdot q)+x (\text{p1}\cdot \;\text{p2})}
\end{dmath*}

\begin{dmath*}\breakingcomma
\frac{1}{x \;\text{SPD}[\text{p1},\text{p2}]+2 z \;\text{SPD}[\text{p1},q] \;\text{SPD}[\text{p2},q]}
\end{dmath*}
\end{document}
