% !TeX program = pdflatex
% !TeX root = GHP.tex

\documentclass[../FeynCalcManual.tex]{subfiles}
\begin{document}
\hypertarget{ghp}{
\section{GHP}\label{ghp}\index{GHP}}

\texttt{GHP[\allowbreak{}p,\ \allowbreak{}a,\ \allowbreak{}b]} gives the
ghost propagator where \texttt{a} and \texttt{b} are the color indices.

\texttt{GHP[\allowbreak{}p]} omits the \(\delta _{ab}\).

\subsection{See also}

\hyperlink{toc}{Overview}, \hyperlink{ghostpropagator}{GhostPropagator},
\hyperlink{gluonpropagator}{GluonPropagator},
\hyperlink{gluonghostvertex}{GluonGhostVertex}.

\subsection{Examples}

\begin{Shaded}
\begin{Highlighting}[]
\NormalTok{GHP}\OperatorTok{[}\FunctionTok{p}\OperatorTok{,} \FunctionTok{a}\OperatorTok{,} \FunctionTok{b}\OperatorTok{]}
\end{Highlighting}
\end{Shaded}

\begin{dmath*}\breakingcomma
\Pi _{ab}(p)
\end{dmath*}

\begin{Shaded}
\begin{Highlighting}[]
\NormalTok{GHP}\OperatorTok{[}\FunctionTok{p}\OperatorTok{]} \SpecialCharTok{//}\NormalTok{ Explicit}
\end{Highlighting}
\end{Shaded}

\begin{dmath*}\breakingcomma
\frac{i}{p^2}
\end{dmath*}

\begin{Shaded}
\begin{Highlighting}[]
\NormalTok{GHP}\OperatorTok{[}\FunctionTok{p}\OperatorTok{,}\NormalTok{ c1}\OperatorTok{,}\NormalTok{ c2}\OperatorTok{]}
\end{Highlighting}
\end{Shaded}

\begin{dmath*}\breakingcomma
\Pi _{\text{c1}\;\text{c2}}(p)
\end{dmath*}

\begin{Shaded}
\begin{Highlighting}[]
\FunctionTok{StandardForm}\OperatorTok{[}\NormalTok{FCE}\OperatorTok{[}\NormalTok{GHP}\OperatorTok{[}\SpecialCharTok{{-}}\FunctionTok{k}\OperatorTok{,}\NormalTok{ c3}\OperatorTok{,}\NormalTok{ c4}\OperatorTok{]} \SpecialCharTok{//}\NormalTok{ Explicit}\OperatorTok{]]}

\CommentTok{(*I FAD[k] SD[c3, c4]*)}
\end{Highlighting}
\end{Shaded}

\end{document}
