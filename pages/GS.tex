% !TeX program = pdflatex
% !TeX root = GS.tex

\documentclass[../FeynCalcManual.tex]{subfiles}
\begin{document}
\hypertarget{gs}{
\section{GS}\label{gs}\index{GS}}

\texttt{GS[\allowbreak{}p]} can be used as input for a 4-dimensional
\(p^\mu \gamma_\mu\) and is transformed into
\texttt{DiracGamma[\allowbreak{}Momentum[\allowbreak{}p]]} by
\texttt{FeynCalcInternal} (=\texttt{FCI}).

\texttt{GS[\allowbreak{}p,\ \allowbreak{}q,\ \allowbreak{}...]} is a
short form for \texttt{GS[\allowbreak{}p].GS[\allowbreak{}q]}.

\subsection{See also}

\hyperlink{toc}{Overview}, \hyperlink{diracgamma}{DiracGamma},
\hyperlink{ga}{GA}, \hyperlink{gad}{GAD}.

\subsection{Examples}

\begin{Shaded}
\begin{Highlighting}[]
\NormalTok{GS}\OperatorTok{[}\FunctionTok{p}\OperatorTok{]}
\end{Highlighting}
\end{Shaded}

\begin{dmath*}\breakingcomma
\bar{\gamma }\cdot \overline{p}
\end{dmath*}

\begin{Shaded}
\begin{Highlighting}[]
\NormalTok{GS}\OperatorTok{[}\FunctionTok{p}\OperatorTok{]} \SpecialCharTok{//}\NormalTok{ FCI }\SpecialCharTok{//} \FunctionTok{StandardForm}

\CommentTok{(*DiracGamma[Momentum[p]]*)}
\end{Highlighting}
\end{Shaded}

\begin{Shaded}
\begin{Highlighting}[]
\NormalTok{GS}\OperatorTok{[}\FunctionTok{p}\OperatorTok{,} \FunctionTok{q}\OperatorTok{,} \FunctionTok{r}\OperatorTok{,} \FunctionTok{s}\OperatorTok{]}
\end{Highlighting}
\end{Shaded}

\begin{dmath*}\breakingcomma
\left(\bar{\gamma }\cdot \overline{p}\right).\left(\bar{\gamma }\cdot \overline{q}\right).\left(\bar{\gamma }\cdot \overline{r}\right).\left(\bar{\gamma }\cdot \overline{s}\right)
\end{dmath*}

\begin{Shaded}
\begin{Highlighting}[]
\NormalTok{GS}\OperatorTok{[}\FunctionTok{p}\OperatorTok{,} \FunctionTok{q}\OperatorTok{,} \FunctionTok{r}\OperatorTok{,} \FunctionTok{s}\OperatorTok{]} \SpecialCharTok{//} \FunctionTok{StandardForm}

\CommentTok{(*GS[p] . GS[q] . GS[r] . GS[s]*)}
\end{Highlighting}
\end{Shaded}

\begin{Shaded}
\begin{Highlighting}[]
\NormalTok{GS}\OperatorTok{[}\FunctionTok{q}\OperatorTok{]}\NormalTok{ . (GS}\OperatorTok{[}\FunctionTok{p}\OperatorTok{]} \SpecialCharTok{+} \FunctionTok{m}\NormalTok{) . GS}\OperatorTok{[}\FunctionTok{q}\OperatorTok{]}
\end{Highlighting}
\end{Shaded}

\begin{dmath*}\breakingcomma
\left(\bar{\gamma }\cdot \overline{q}\right).\left(\bar{\gamma }\cdot \overline{p}+m\right).\left(\bar{\gamma }\cdot \overline{q}\right)
\end{dmath*}
\end{document}
