% !TeX program = pdflatex
% !TeX root = GSD.tex

\documentclass[../FeynCalcManual.tex]{subfiles}
\begin{document}
\hypertarget{gsd}{%
\section{GSD}\label{gsd}}

GSD{[}p{]} can be used as input for a \(D\)-dimensional
\(p^\mu \gamma_\mu\) and is transformed into
\texttt{DiracGamma[\allowbreak{}Momentum[\allowbreak{}p,\ \allowbreak{}D],\ \allowbreak{}D]}
by \texttt{FeynCalcInternal} (=\texttt{FCI}).

\texttt{GSD[\allowbreak{}p,\ \allowbreak{}q,\ \allowbreak{}...]} is a
short form for \texttt{GSD[\allowbreak{}p].GSD[\allowbreak{}q]}.

\subsection{See also}

\hyperlink{toc}{Overview}, \hyperlink{diracgamma}{DiracGamma},
\hyperlink{ga}{GA}, \hyperlink{gad}{GAD}.

\subsection{Examples}

\begin{Shaded}
\begin{Highlighting}[]
\NormalTok{GSD}\OperatorTok{[}\FunctionTok{p}\OperatorTok{]}
\end{Highlighting}
\end{Shaded}

\begin{dmath*}\breakingcomma
\gamma \cdot p
\end{dmath*}

\begin{Shaded}
\begin{Highlighting}[]
\NormalTok{GSD}\OperatorTok{[}\FunctionTok{p}\OperatorTok{]} \SpecialCharTok{//}\NormalTok{ FCI }\SpecialCharTok{//} \FunctionTok{StandardForm}

\CommentTok{(*DiracGamma[Momentum[p, D], D]*)}
\end{Highlighting}
\end{Shaded}

\begin{Shaded}
\begin{Highlighting}[]
\NormalTok{GSD}\OperatorTok{[}\FunctionTok{p}\OperatorTok{,} \FunctionTok{q}\OperatorTok{,} \FunctionTok{r}\OperatorTok{,} \FunctionTok{s}\OperatorTok{]}
\end{Highlighting}
\end{Shaded}

\begin{dmath*}\breakingcomma
(\gamma \cdot p).(\gamma \cdot q).(\gamma \cdot r).(\gamma \cdot s)
\end{dmath*}

\begin{Shaded}
\begin{Highlighting}[]
\NormalTok{GSD}\OperatorTok{[}\FunctionTok{p}\OperatorTok{,} \FunctionTok{q}\OperatorTok{,} \FunctionTok{r}\OperatorTok{,} \FunctionTok{s}\OperatorTok{]} \SpecialCharTok{//} \FunctionTok{StandardForm}

\CommentTok{(*GSD[p] . GSD[q] . GSD[r] . GSD[s]*)}
\end{Highlighting}
\end{Shaded}

\begin{Shaded}
\begin{Highlighting}[]
\NormalTok{GSD}\OperatorTok{[}\FunctionTok{q}\OperatorTok{]}\NormalTok{ . (GSD}\OperatorTok{[}\FunctionTok{p}\OperatorTok{]} \SpecialCharTok{+} \FunctionTok{m}\NormalTok{) . GSD}\OperatorTok{[}\FunctionTok{q}\OperatorTok{]}
\end{Highlighting}
\end{Shaded}

\begin{dmath*}\breakingcomma
(\gamma \cdot q).(m+\gamma \cdot p).(\gamma \cdot q)
\end{dmath*}
\end{document}
