% !TeX program = pdflatex
% !TeX root = GSLND.tex

\documentclass[../FeynCalcManual.tex]{subfiles}
\begin{document}
\begin{Shaded}
\begin{Highlighting}[]
 
\end{Highlighting}
\end{Shaded}

\hypertarget{gslnd}{
\section{GSLND}\label{gslnd}\index{GSLND}}

\texttt{GSLND[\allowbreak{}p,\ \allowbreak{}n,\ \allowbreak{}nb]}
denotes the negative component in the lightcone decomposition of the
slashed Dirac matrix \((\gamma \cdot p)\) along the vectors \texttt{n}
and \texttt{nb} in \(D\) dimensions. It corresponds to
\(\frac{1}{2} (n \cdot p) (\gamma \cdot \bar{n})\).

If one omits \texttt{n} and \texttt{nb}, the program will use default
vectors specified via \texttt{\$FCDefaultLightconeVectorN} and
\texttt{\$FCDefaultLightconeVectorNB}.

\subsection{See also}

\hyperlink{toc}{Overview}, \hyperlink{diracgamma}{DiracGamma},
\hyperlink{galpd}{GALPD}, \hyperlink{galnd}{GALND},
\hyperlink{galrd}{GALRD}, \hyperlink{gslpd}{GSLPD},
\hyperlink{gslrd}{GSLRD}.

\subsection{Examples}

\begin{Shaded}
\begin{Highlighting}[]
\NormalTok{GSLND}\OperatorTok{[}\FunctionTok{p}\OperatorTok{,} \FunctionTok{n}\OperatorTok{,}\NormalTok{ nb}\OperatorTok{]}
\end{Highlighting}
\end{Shaded}

\begin{dmath*}\breakingcomma
\frac{1}{2} (n\cdot p) \gamma \cdot \;\text{nb}
\end{dmath*}

\begin{Shaded}
\begin{Highlighting}[]
\FunctionTok{StandardForm}\OperatorTok{[}\NormalTok{GSLND}\OperatorTok{[}\FunctionTok{p}\OperatorTok{,} \FunctionTok{n}\OperatorTok{,}\NormalTok{ nb}\OperatorTok{]} \SpecialCharTok{//}\NormalTok{ FCI}\OperatorTok{]}
\end{Highlighting}
\end{Shaded}

\begin{dmath*}\breakingcomma
\frac{1}{2} \;\text{DiracGamma}[\text{Momentum}[\text{nb},D],D] \;\text{Pair}[\text{Momentum}[n,D],\text{Momentum}[p,D]]
\end{dmath*}

Notice that the properties of \texttt{n} and \texttt{nb} vectors have to
be set by hand before doing the actual computation

\begin{Shaded}
\begin{Highlighting}[]
\NormalTok{GSLND}\OperatorTok{[}\FunctionTok{p}\OperatorTok{,} \FunctionTok{n}\OperatorTok{,}\NormalTok{ nb}\OperatorTok{]}\NormalTok{ . GSLND}\OperatorTok{[}\FunctionTok{q}\OperatorTok{,} \FunctionTok{n}\OperatorTok{,}\NormalTok{ nb}\OperatorTok{]} \SpecialCharTok{//}\NormalTok{ DiracSimplify}
\end{Highlighting}
\end{Shaded}

\begin{dmath*}\breakingcomma
\frac{1}{4} \;\text{nb}^2 (n\cdot p) (n\cdot q)
\end{dmath*}

\begin{Shaded}
\begin{Highlighting}[]
\NormalTok{FCClearScalarProducts}\OperatorTok{[]}
\NormalTok{SPD}\OperatorTok{[}\FunctionTok{n}\OperatorTok{]} \ExtensionTok{=} \DecValTok{0}\NormalTok{;}
\NormalTok{SPD}\OperatorTok{[}\NormalTok{nb}\OperatorTok{]} \ExtensionTok{=} \DecValTok{0}\NormalTok{;}
\NormalTok{SPD}\OperatorTok{[}\FunctionTok{n}\OperatorTok{,}\NormalTok{ nb}\OperatorTok{]} \ExtensionTok{=} \DecValTok{2}\NormalTok{;}
\end{Highlighting}
\end{Shaded}

\begin{Shaded}
\begin{Highlighting}[]
\NormalTok{GSLND}\OperatorTok{[}\FunctionTok{p}\OperatorTok{,} \FunctionTok{n}\OperatorTok{,}\NormalTok{ nb}\OperatorTok{]}\NormalTok{ . GSLND}\OperatorTok{[}\FunctionTok{q}\OperatorTok{,} \FunctionTok{n}\OperatorTok{,}\NormalTok{ nb}\OperatorTok{]} \SpecialCharTok{//}\NormalTok{ DiracSimplify}
\end{Highlighting}
\end{Shaded}

\begin{dmath*}\breakingcomma
0
\end{dmath*}

\begin{Shaded}
\begin{Highlighting}[]
\NormalTok{FCClearScalarProducts}\OperatorTok{[]}
\end{Highlighting}
\end{Shaded}

\end{document}
