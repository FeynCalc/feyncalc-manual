% !TeX program = pdflatex
% !TeX root = GSLR.tex

\documentclass[../FeynCalcManual.tex]{subfiles}
\begin{document}
\hypertarget{gslr}{
\section{GSLR}\label{gslr}\index{GSLR}}

\texttt{GSLR[\allowbreak{}p,\ \allowbreak{}n,\ \allowbreak{}nb]} denotes
the perpendicular component in the lightcone decomposition of the
slashed Dirac matrix \((\gamma \cdot p)\) along the vectors \texttt{n}
and \texttt{nb}. It corresponds to \((\gamma \cdot p)_{\perp}\).

If one omits \texttt{n} and \texttt{nb}, the program will use default
vectors specified via \texttt{\$FCDefaultLightconeVectorN} and
\texttt{\$FCDefaultLightconeVectorNB}.

\subsection{See also}

\hyperlink{toc}{Overview}, \hyperlink{diracgamma}{DiracGamma},
\hyperlink{galp}{GALP}, \hyperlink{galn}{GALN}, \hyperlink{galr}{GALR},
\hyperlink{gslp}{GSLP}, \hyperlink{gsln}{GSLN}.

\subsection{Examples}

\begin{Shaded}
\begin{Highlighting}[]
\NormalTok{GSLR}\OperatorTok{[}\FunctionTok{p}\OperatorTok{,} \FunctionTok{n}\OperatorTok{,}\NormalTok{ nb}\OperatorTok{]}
\end{Highlighting}
\end{Shaded}

\begin{dmath*}\breakingcomma
\bar{\gamma }\cdot \overline{p}_{\perp }
\end{dmath*}

\begin{Shaded}
\begin{Highlighting}[]
\FunctionTok{StandardForm}\OperatorTok{[}\NormalTok{GSLR}\OperatorTok{[}\FunctionTok{p}\OperatorTok{,} \FunctionTok{n}\OperatorTok{,}\NormalTok{ nb}\OperatorTok{]} \SpecialCharTok{//}\NormalTok{ FCI}\OperatorTok{]}

\CommentTok{(*DiracGamma[LightConePerpendicularComponent[Momentum[p], Momentum[n], Momentum[nb]]]*)}
\end{Highlighting}
\end{Shaded}

Notice that the properties of \texttt{n} and \texttt{nb} vectors have to
be set by hand before doing the actual computation

\begin{Shaded}
\begin{Highlighting}[]
\NormalTok{GSLR}\OperatorTok{[}\FunctionTok{p}\OperatorTok{,} \FunctionTok{n}\OperatorTok{,}\NormalTok{ nb}\OperatorTok{]}\NormalTok{ . GSLP}\OperatorTok{[}\FunctionTok{q}\OperatorTok{,} \FunctionTok{n}\OperatorTok{,}\NormalTok{ nb}\OperatorTok{]} \SpecialCharTok{//}\NormalTok{ DiracSimplify}
\end{Highlighting}
\end{Shaded}

\begin{dmath*}\breakingcomma
-\frac{1}{4} \overline{n}^2 \left(\overline{\text{nb}}\cdot \overline{q}\right) \left(\bar{\gamma }\cdot \overline{\text{nb}}\right).\left(\bar{\gamma }\cdot \overline{p}_{\perp }\right)-\frac{1}{4} \left(\overline{n}\cdot \overline{\text{nb}}\right) \left(\overline{\text{nb}}\cdot \overline{q}\right) \left(\bar{\gamma }\cdot \overline{n}\right).\left(\bar{\gamma }\cdot \overline{p}_{\perp }\right)
\end{dmath*}

\begin{Shaded}
\begin{Highlighting}[]
\NormalTok{FCClearScalarProducts}\OperatorTok{[]}
\NormalTok{SP}\OperatorTok{[}\FunctionTok{n}\OperatorTok{]} \ExtensionTok{=} \DecValTok{0}\NormalTok{;}
\NormalTok{SP}\OperatorTok{[}\NormalTok{nb}\OperatorTok{]} \ExtensionTok{=} \DecValTok{0}\NormalTok{;}
\NormalTok{SP}\OperatorTok{[}\FunctionTok{n}\OperatorTok{,}\NormalTok{ nb}\OperatorTok{]} \ExtensionTok{=} \DecValTok{2}\NormalTok{;}
\end{Highlighting}
\end{Shaded}

\begin{Shaded}
\begin{Highlighting}[]
\NormalTok{GSLR}\OperatorTok{[}\FunctionTok{p}\OperatorTok{,} \FunctionTok{n}\OperatorTok{,}\NormalTok{ nb}\OperatorTok{]}\NormalTok{ . GSLP}\OperatorTok{[}\FunctionTok{q}\OperatorTok{,} \FunctionTok{n}\OperatorTok{,}\NormalTok{ nb}\OperatorTok{]} \SpecialCharTok{//}\NormalTok{ DiracSimplify}
\end{Highlighting}
\end{Shaded}

\begin{dmath*}\breakingcomma
-\frac{1}{2} \left(\overline{\text{nb}}\cdot \overline{q}\right) \left(\bar{\gamma }\cdot \overline{n}\right).\left(\bar{\gamma }\cdot \overline{p}_{\perp }\right)
\end{dmath*}

\begin{Shaded}
\begin{Highlighting}[]
\NormalTok{FCClearScalarProducts}\OperatorTok{[]}
\end{Highlighting}
\end{Shaded}

\end{document}
