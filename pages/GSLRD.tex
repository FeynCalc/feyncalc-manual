% !TeX program = pdflatex
% !TeX root = GSLRD.tex

\documentclass[../FeynCalcManual.tex]{subfiles}
\begin{document}
\hypertarget{gslrd}{
\section{GSLRD}\label{gslrd}\index{GSLRD}}

\texttt{GSLRD[\allowbreak{}p,\ \allowbreak{}n,\ \allowbreak{}nb]}
denotes the perpendicular component in the lightcone decomposition of
the slashed Dirac matrix \((\gamma \cdot p)\) along the vectors
\texttt{n} and \texttt{nb} in \(D\) dimensions. It corresponds to
\((\gamma \cdot p)_{\perp}\).

If one omits \texttt{n} and \texttt{nb}, the program will use default
vectors specified via \texttt{\$FCDefaultLightconeVectorN} and
\texttt{\$FCDefaultLightconeVectorNB}.

\subsection{See also}

\hyperlink{toc}{Overview}, \hyperlink{diracgamma}{DiracGamma},
\hyperlink{galpd}{GALPD}, \hyperlink{galnd}{GALND},
\hyperlink{galrd}{GALRD}, \hyperlink{gslpd}{GSLPD},
\hyperlink{gslnd}{GSLND}.

\subsection{Examples}

\begin{Shaded}
\begin{Highlighting}[]
\NormalTok{GSLRD}\OperatorTok{[}\FunctionTok{p}\OperatorTok{,} \FunctionTok{n}\OperatorTok{,}\NormalTok{ nb}\OperatorTok{]}
\end{Highlighting}
\end{Shaded}

\begin{dmath*}\breakingcomma
\gamma \cdot p_{\perp }
\end{dmath*}

\begin{Shaded}
\begin{Highlighting}[]
\FunctionTok{StandardForm}\OperatorTok{[}\NormalTok{GSLRD}\OperatorTok{[}\FunctionTok{p}\OperatorTok{,} \FunctionTok{n}\OperatorTok{,}\NormalTok{ nb}\OperatorTok{]} \SpecialCharTok{//}\NormalTok{ FCI}\OperatorTok{]}

\CommentTok{(*DiracGamma[LightConePerpendicularComponent[Momentum[p, D], Momentum[n, D], Momentum[nb, D]], D]*)}
\end{Highlighting}
\end{Shaded}

Notice that the properties of \texttt{n} and \texttt{nb} vectors have to
be set by hand before doing the actual computation

\begin{Shaded}
\begin{Highlighting}[]
\NormalTok{GSLRD}\OperatorTok{[}\FunctionTok{p}\OperatorTok{,} \FunctionTok{n}\OperatorTok{,}\NormalTok{ nb}\OperatorTok{]}\NormalTok{ . GSLPD}\OperatorTok{[}\FunctionTok{q}\OperatorTok{,} \FunctionTok{n}\OperatorTok{,}\NormalTok{ nb}\OperatorTok{]} \SpecialCharTok{//}\NormalTok{ DiracSimplify}
\end{Highlighting}
\end{Shaded}

\begin{dmath*}\breakingcomma
-\frac{1}{4} n^2 (\text{nb}\cdot q) (\gamma \cdot \;\text{nb}).\left(\gamma \cdot p_{\perp }\right)-\frac{1}{4} (n\cdot \;\text{nb}) (\text{nb}\cdot q) (\gamma \cdot n).\left(\gamma \cdot p_{\perp }\right)
\end{dmath*}

\begin{Shaded}
\begin{Highlighting}[]
\NormalTok{FCClearScalarProducts}\OperatorTok{[]}
\NormalTok{SPD}\OperatorTok{[}\FunctionTok{n}\OperatorTok{]} \ExtensionTok{=} \DecValTok{0}\NormalTok{;}
\NormalTok{SPD}\OperatorTok{[}\NormalTok{nb}\OperatorTok{]} \ExtensionTok{=} \DecValTok{0}\NormalTok{;}
\NormalTok{SPD}\OperatorTok{[}\FunctionTok{n}\OperatorTok{,}\NormalTok{ nb}\OperatorTok{]} \ExtensionTok{=} \DecValTok{2}\NormalTok{;}
\end{Highlighting}
\end{Shaded}

\begin{Shaded}
\begin{Highlighting}[]
\NormalTok{GSLRD}\OperatorTok{[}\FunctionTok{p}\OperatorTok{,} \FunctionTok{n}\OperatorTok{,}\NormalTok{ nb}\OperatorTok{]}\NormalTok{ . GSLPD}\OperatorTok{[}\FunctionTok{q}\OperatorTok{,} \FunctionTok{n}\OperatorTok{,}\NormalTok{ nb}\OperatorTok{]} \SpecialCharTok{//}\NormalTok{ DiracSimplify}
\end{Highlighting}
\end{Shaded}

\begin{dmath*}\breakingcomma
-\frac{1}{2} (\text{nb}\cdot q) (\gamma \cdot n).\left(\gamma \cdot p_{\perp }\right)
\end{dmath*}

\begin{Shaded}
\begin{Highlighting}[]
\NormalTok{FCClearScalarProducts}\OperatorTok{[]}
\end{Highlighting}
\end{Shaded}

\end{document}
