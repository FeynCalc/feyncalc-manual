% !TeX program = pdflatex
% !TeX root = Gamma5.tex

\documentclass[../FeynCalcManual.tex]{subfiles}
\begin{document}
\hypertarget{treatment of gamma5 in d dimensions}{
\section{Treatment of gamma5 in D dimensions}\label{treatment of gamma5 in d dimensions}\index{Treatment of gamma5 in D dimensions}}

\subsection{See also}

\hyperlink{toc}{Overview}.

\hypertarget{nature-of-the-problem}{%
\subsection{Nature of the problem}\label{nature-of-the-problem}}

It is a well-known fact (cf.~eg.
\href{https://arxiv.org/pdf/hep-th/0005255}{Jegerlehner:2000dz}) that
the definition of \(\gamma^5\) in 4 dimensions cannot be consistently
extended to \(D\) dimensions without giving up either the
anticommutativity property

\begin{equation}
\{\gamma^5, \gamma^\mu\} = 0
\end{equation}

or the cyclicity of the Dirac trace, e.g.~that

\begin{equation}
\mathrm{Tr}( \gamma^{\mu_1} \ldots \gamma^{\mu_{2n}} \gamma^5 ) = \mathrm{Tr}( \gamma^{\mu_2} \ldots \gamma^{\mu_{2n}} \gamma^5 \gamma^{\mu_1} ) = \mathrm{Tr}( \gamma^{\mu_3} \ldots \gamma^{\mu_{2n}} \gamma^5 \gamma^{\mu_1} \gamma^{\mu_2} ) = \ldots
\end{equation}

This explains the existence of multiple prescriptions (called
\(\gamma^5\)-schemes) that aim at avoiding these issues and obtaining
physical results in the \emph{given calculation}.

Indeed, as of now there is no simple solution or cookbook recipe that
can be readily applied to any theory at any loop order in a fully
automatic fashion.

The reason for this is that calculations involving \(\gamma^5\) are not
limited to the algebraic manipulations of Dirac matrices. In general,
once \(\gamma^5\) shows up in \(D\)-dimensional amplitudes, there is a
high chance that the final result will violate some of the essential
symmetries, such as generalized Ward identities or Bose symmetry.

Once this happens, symmetries violated due to the chosen \(\gamma^5\)
scheme must be restored by hand, e.g.~by introducing special finite
counterterms. Unfortunately, an explicit determination of such
counterterms for a given model is a nontrivial task, especially beyond
1-loop. This explains why people usually try to avoid this situation and
would rather opt for figuring out special tricks that work only for this
particular calculation but manage to preserve the symmetries.

Further discussions on this topic can be found e.g.~in

\begin{itemize}
\tightlist
\item
  chapter D of \href{https://arxiv.org/pdf/1809.01830}{Blondel:2018mad}
\item
  \href{https://arxiv.org/pdf/hep-ph/9504315.pdf}{Trueman:1995ca}
\item
  \href{https://arxiv.org/pdf/1912.06823.pdf}{Denner:2019vbn}
\item
  \href{https://arxiv.org/abs/1705.01827}{Gnendiger:2017pys}
\item
  \href{https://arxiv.org/abs/2312.11291}{Stockinger:2023ndm}
\end{itemize}

\hypertarget{feyncalc-implementation}{%
\subsection{FeynCalc implementation}\label{feyncalc-implementation}}

FeynCalc has built-in support for several \(\gamma^5\)-schemes in the
sense that it can manipulate \(D\)-dimensional algebraic expressions
involving \(\gamma^5\) in accordance with the rules provided by the
scheme authors.

The nonalgebraic part of a typical \(\gamma^5\)-calculation,
e.g.~checking for violated symmetries and restoring them is \textbf{not
handled} by FeynCalc. This is also not something easy to automatize (due
to the reasons explained above) so that here we expect the user to
employ their understanding of physics and common sense.

The responsibility of FeynCalc is to ensure that algebraic manipulations
of Dirac matrices (including \(\gamma^5\)) are consistent within the
chosen scheme. For the purpose of dealing with \(\gamma^5\) in \(D\)
dimensions FeynCalc implements three different schemes.

\hypertarget{ndr}{%
\subsubsection{NDR}\label{ndr}}

The Naive or Conventional Dimensional Regularization (NDR or CDR
respectively)
\href{https://doi.org/10.1016/0550-3213(79)90333-X}{Chanowitz:1979zu}
simply \emph{assumes} that one can define a \(D\)-dimensional
\(\gamma^5\) that anticommutes with any other Dirac matrix and does not
break the cyclicity of the trace. For FeynCalc this means that in every
string of Dirac matrices all \(\gamma^5\) can be safely anticommuted to
the right end of the string. In the course of this operation FeynCalc
can always apply \((\gamma^5)^2 = 1\).

Consequently, all Dirac traces with an even number of \(\gamma^5\) can
be rewritten as traces that involve only the first four
\(\gamma\)-matrices and evaluated directly, e.g.

\begin{equation}
\mathrm{Tr}( \gamma^{\mu_1} \gamma^{\mu_2} \gamma^5 \gamma^{\mu_3} \ldots \gamma^{\mu_{2n}} \gamma^5 ) = 
\mathrm{Tr}( \gamma^{\mu_1} \gamma^{\mu_2} \ldots \gamma^{\mu_{2n}}  )
\end{equation}

The problematic cases are \(\gamma^5\)-odd traces with an even number of
other Dirac matrices, where the \(\mathcal{O}(D-4)\) pieces of the
result depend on the initial position of \(\gamma^5\) in the string.
Using the anticommutativity property they can be always rewritten as
traces of a string of other Dirac matrices and one \(\gamma^5\). If the
number of the other Dirac matrices is odd, such a trace is put to zero
i.e. \begin{equation}
\mathrm{Tr}(\gamma^{\mu_1} \ldots \gamma^{\mu_{2n-1}} \gamma^5) = 0, \quad n \in \mathbb{N}
\end{equation} If the number is even, the trace \begin{equation}
\mathrm{Tr}(\gamma^{\mu_1} \ldots \gamma^{\mu_{2n}} \gamma^5)
\end{equation} is returned unevaluated, since FeynCalc does not know how
to calculate it in a consistent way. A user who knows how these
ambiguous objects should be treated in the particular calculation can
still take care of the remaining traces by hand. This ensures that the
output produced by FeynCalc is algebraically consistent to the maximal
extent possible in the NDR scheme without extra assumptions.

In FeynCalc, this scheme the default choice. It can also be explicitly
activated via

\begin{Shaded}
\begin{Highlighting}[]
\NormalTok{FCSetDiracGammaScheme}\OperatorTok{[}\StringTok{"NDR"}\OperatorTok{]}
\end{Highlighting}
\end{Shaded}

Sometimes \(\gamma^5\) may show up in the calculation as an artifact of
using a particular set of operators or projectors even though the
results itself is not supposed to be affected by the
\(\gamma^5\)-problem. For such cases FeynCalc offers a variety of the
NDR scheme, where all traces of the form \begin{equation}
\mathrm{Tr}(\gamma^{\mu_1} \ldots \gamma^{\mu_{2n}} \gamma^5)
\end{equation} are simply put to zero. It can be used to e.g.~examine
the effects of the chosen scheme on the final result and can be
activated via

\begin{Shaded}
\begin{Highlighting}[]
\NormalTok{FCSetDiracGammaScheme}\OperatorTok{[}\StringTok{"NDR{-}Discard"}\OperatorTok{]}
\end{Highlighting}
\end{Shaded}

\hypertarget{bmhv}{%
\subsection{BMHV}\label{bmhv}}

FeynCalc also supports the Breitenlohner-Maison implementation
\href{https://doi.org/10.1007/BF01609069}{Breitenlohner:1977hr} of the
t'Hooft-Veltman
\href{https://doi.org/10.1016/0550-3213(72)90279-9}{tHooft:1972tcz}
prescription, often abbreviated as BMHV, HVBM, HV or BM scheme. In this
approach \(\gamma^5\) is treated as a purely 4-dimensional object, while
\(D\)-dimensional Dirac matrices and 4-vectors are decomposed into
\(4\)- and \(D-4\)-dimensional components. Following
\href{https://doi.org/10.1016/0550-3213(90)90223-Z}{Buras:1989xd}
FeynCalc typesets the former with a bar and the latter with a hat e.g.

\begin{equation}
\gamma^\mu = \bar{\gamma}^\mu + \hat{\gamma}^\mu, \quad p^\mu = \bar{p}^\mu + \hat{p}^\mu
\end{equation}

The main advantage of the BMHV scheme is that the Dirac algebra
(including traces) can be evaluated without any algebraic ambiguities.
However, calculations involving tensors from three different spaces
(\(D\), \(4\) and \(D-4\)) often turn out to be rather cumbersome, even
when using computer codes. Moreover, this prescription is known to
artificially violate Ward identities in chiral theories, which is
something that can be often avoided when using NDR. Within BMHV FeynCalc
can simplify arbitrary strings of Dirac matrices and calculate arbitrary
traces out-of-the-box. The evaluation of \(\gamma^5\)-odd Dirac traces
is performed using the West-formula from
\href{https://doi.org/10.1016/0010-4655(93)90011-Z}{West:1991xv}. It is
worth noting that \(D-4\)-dimensional components of external momenta are
not set to zero by default, as it is conventionally done in the
literature. If this is required, the user should evaluate
\texttt{Momentum[\allowbreak{}pi,\ \allowbreak{}D-4]=0} for each
relevant momentum \(p_i\). To remove such assignments one should use
\texttt{FCClearScalarProducts[\allowbreak{}]}.

This scheme is activated by evaluating

\begin{Shaded}
\begin{Highlighting}[]
\NormalTok{FCSetDiracGammaScheme}\OperatorTok{[}\StringTok{"BMHV"}\OperatorTok{]}
\end{Highlighting}
\end{Shaded}

\hypertarget{larins-scheme}{%
\subsection{Larin's scheme}\label{larins-scheme}}

Larin's scheme
\href{https://arxiv.org/pdf/hep-ph/9302240.pdf}{Larin:1993tq} is a
variety of the BMHV scheme that has been extensively used in QCD
calculations involving axial vector currents. The main idea is to
replace the products of \(\gamma^\mu\) and \(\gamma^5\) in a chiral
trace as in

\begin{equation}
\gamma^\mu \gamma^5 \to \frac{1}{6} i \varepsilon^{\mu \nu \rho \sigma} \gamma_\nu \gamma_\rho \gamma_\sigma
\end{equation}

and then calculate the resulting trace. Then, all
\(\varepsilon^{\mu \nu \rho \sigma}\)-tensors occurring in the amplitude
should be evaluated in \(D\) dimensions. Together with the correct
counterterm, this prescription is known to give the same result as when
using the full BMHV scheme.

FeynCalc implement the so-called Moch-Vermaseren-Vogt MVV formula from
\href{https://arxiv.org/pdf/1506.04517.pdf}{Moch:2015usa} for
calculating \(\gamma^5\)-traces in this scheme. The scheme itself is
activated by setting

\begin{Shaded}
\begin{Highlighting}[]
\NormalTok{FCSetDiracGammaScheme}\OperatorTok{[}\StringTok{"Larin"}\OperatorTok{]}
\end{Highlighting}
\end{Shaded}

\end{document}
