% !TeX program = pdflatex
% !TeX root = GammaEpsilon.tex

\documentclass[../FeynCalcManual.tex]{subfiles}
\begin{document}
\hypertarget{gammaepsilon}{
\section{GammaEpsilon}\label{gammaepsilon}\index{GammaEpsilon}}

\texttt{GammaEpsilon[\allowbreak{}exp]} gives a series expansion of
\texttt{Gamma[\allowbreak{}exp]} in \texttt{Epsilon} up to order
\texttt{6} (where \texttt{EulerGamma} is neglected).

\subsection{See also}

\hyperlink{toc}{Overview}, \hyperlink{gammaexpand}{GammaExpand},
\hyperlink{series2}{Series2}.

\subsection{Examples}

If the argument is of the form \texttt{(1+a Epsilon)} the result is not
calculated but tabulated.

\begin{Shaded}
\begin{Highlighting}[]
\NormalTok{GammaEpsilon}\OperatorTok{[}\DecValTok{1} \SpecialCharTok{+} \FunctionTok{a}\NormalTok{ Epsilon}\OperatorTok{]}
\end{Highlighting}
\end{Shaded}

\begin{dmath*}\breakingcomma
\varepsilon ^5 \left(-\frac{a^5 \zeta (5)}{5}-\frac{1}{36} \pi ^2 a^5 \zeta (3)\right)+\frac{1}{160} \pi ^4 a^4 \varepsilon ^4-\frac{1}{3} a^3 \varepsilon ^3 \zeta (3)+\frac{1}{12} \pi ^2 a^2 \varepsilon ^2+\text{C\$16471} \varepsilon ^6+1
\end{dmath*}

\begin{Shaded}
\begin{Highlighting}[]
\NormalTok{GammaEpsilon}\OperatorTok{[}\DecValTok{1} \SpecialCharTok{{-}}\NormalTok{ Epsilon}\SpecialCharTok{/}\DecValTok{2}\OperatorTok{]}
\end{Highlighting}
\end{Shaded}

\begin{dmath*}\breakingcomma
\text{C\$16508} \varepsilon ^6+\frac{\pi ^4 \varepsilon ^4}{2560}+\frac{\pi ^2 \varepsilon ^2}{48}+\varepsilon ^5 \left(\frac{\pi ^2 \zeta (3)}{1152}+\frac{\zeta (5)}{160}\right)+\frac{\varepsilon ^3 \zeta (3)}{24}+1
\end{dmath*}

For other arguments the expansion is calculated.

\begin{Shaded}
\begin{Highlighting}[]
\NormalTok{GammaEpsilon}\OperatorTok{[}\NormalTok{Epsilon}\OperatorTok{]}
\end{Highlighting}
\end{Shaded}

\begin{dmath*}\breakingcomma
\text{C\$17651} \varepsilon ^6+\frac{\pi ^4 \varepsilon ^3}{160}+\frac{\pi ^2 \varepsilon }{12}+\frac{1}{\varepsilon }+\frac{1}{720} \varepsilon ^5 \left(\frac{61 \pi ^6}{168}+10 \psi ^{(2)}(1)^2\right)+\frac{\varepsilon ^2 \psi ^{(2)}(1)}{6}+\frac{\varepsilon ^6 \left(-84 \pi ^2 \zeta (5)+\frac{21 \pi ^4 \psi ^{(2)}(1)}{4}+\psi ^{(6)}(1)\right)}{5040}+\frac{1}{120} \varepsilon ^4 \left(\frac{5 \pi ^2 \psi ^{(2)}(1)}{3}-24 \zeta (5)\right)
\end{dmath*}

\begin{Shaded}
\begin{Highlighting}[]
\NormalTok{GammaEpsilon}\OperatorTok{[}\FunctionTok{x}\OperatorTok{]}
\end{Highlighting}
\end{Shaded}

\begin{dmath*}\breakingcomma
\Gamma (x)
\end{dmath*}
\end{document}
