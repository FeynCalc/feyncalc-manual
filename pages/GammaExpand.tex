% !TeX program = pdflatex
% !TeX root = GammaExpand.tex

\documentclass[../FeynCalcManual.tex]{subfiles}
\begin{document}
\hypertarget{gammaexpand}{
\section{GammaExpand}\label{gammaexpand}\index{GammaExpand}}

\texttt{GammaExpand[\allowbreak{}exp]} rewrites
\texttt{Gamma[\allowbreak{}n + m]} in \texttt{exp} (where \texttt{n} has
\texttt{Head} \texttt{Integer}).

\subsection{See also}

\hyperlink{toc}{Overview}, \hyperlink{gammaepsilon}{GammaEpsilon}.

\subsection{Examples}

\begin{Shaded}
\begin{Highlighting}[]
\NormalTok{GammaExpand}\OperatorTok{[}\FunctionTok{Gamma}\OperatorTok{[}\DecValTok{2} \SpecialCharTok{+}\NormalTok{ Epsilon}\OperatorTok{]]}
\end{Highlighting}
\end{Shaded}

\begin{dmath*}\breakingcomma
(\varepsilon +1) \Gamma (\varepsilon +1)
\end{dmath*}

\begin{Shaded}
\begin{Highlighting}[]
\NormalTok{GammaExpand}\OperatorTok{[}\FunctionTok{Gamma}\OperatorTok{[}\SpecialCharTok{{-}}\DecValTok{3} \SpecialCharTok{+}\NormalTok{ Epsilon}\OperatorTok{]]}
\end{Highlighting}
\end{Shaded}

\begin{dmath*}\breakingcomma
\frac{\Gamma (\varepsilon +1)}{(\varepsilon -3) (\varepsilon -2) (\varepsilon -1) \varepsilon }
\end{dmath*}

\begin{Shaded}
\begin{Highlighting}[]
\NormalTok{GammaExpand}\OperatorTok{[}\FunctionTok{Gamma}\OperatorTok{[}\DecValTok{1} \SpecialCharTok{+}\NormalTok{ Epsilon}\OperatorTok{]]}
\end{Highlighting}
\end{Shaded}

\begin{dmath*}\breakingcomma
\Gamma (\varepsilon +1)
\end{dmath*}
\end{document}
