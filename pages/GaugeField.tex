% !TeX program = pdflatex
% !TeX root = GaugeField.tex

\documentclass[../FeynCalcManual.tex]{subfiles}
\begin{document}
\hypertarget{gaugefield}{%
\section{GaugeField}\label{gaugefield}}

\texttt{GaugeField} is just a name. No functional properties are
associated with it. \texttt{GaugeField} is used as default setting for
the option \texttt{QuantumField} of \texttt{FieldStrength}.

\subsection{See also}

\hyperlink{toc}{Overview}, \hyperlink{fieldstrength}{FieldStrength},
\hyperlink{quantumfield}{QuantumField}.

\subsection{Examples}

\begin{Shaded}
\begin{Highlighting}[]
\NormalTok{GaugeField}
\end{Highlighting}
\end{Shaded}

\begin{dmath*}\breakingcomma
A
\end{dmath*}

\begin{Shaded}
\begin{Highlighting}[]
\NormalTok{QuantumField}\OperatorTok{[}\NormalTok{GaugeField}\OperatorTok{,}\NormalTok{ LorentzIndex}\OperatorTok{[}\SpecialCharTok{\textbackslash{}}\OperatorTok{[}\NormalTok{Mu}\OperatorTok{]],}\NormalTok{ SUNIndex}\OperatorTok{[}\FunctionTok{a}\OperatorTok{]]}
\end{Highlighting}
\end{Shaded}

\begin{dmath*}\breakingcomma
A_{\mu }^a
\end{dmath*}
\end{document}
