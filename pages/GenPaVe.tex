% !TeX program = pdflatex
% !TeX root = GenPaVe.tex

\documentclass[../FeynCalcManual.tex]{subfiles}
\begin{document}
\hypertarget{genpave}{%
\section{GenPaVe}\label{genpave}}

\texttt{GenPaVe[\allowbreak{}i,\ \allowbreak{}j,\ \allowbreak{}...,\ \allowbreak{}\{\allowbreak{}\{\allowbreak{}0,\ \allowbreak{}m0\},\ \allowbreak{}\{\allowbreak{}Momentum[\allowbreak{}p1],\ \allowbreak{}m1\},\ \allowbreak{}\{\allowbreak{}Momentum[\allowbreak{}p2],\ \allowbreak{}m2\},\ \allowbreak{}...]}
denotes the invariant (and scalar) Passarino-Veltman integrals, i.e.~the
coefficient functions of the tensor integral decomposition. In contrast
to \texttt{PaVe} which uses the LoopTools convention, masses and
external momenta in \texttt{GenPaVe} are written in the same order as
they appear in the original tensor integral,
i.e.~\texttt{FAD[\allowbreak{}\{\allowbreak{}q,\ \allowbreak{}m0\},\ \allowbreak{}\{\allowbreak{}q-p1,\ \allowbreak{}m1\},\ \allowbreak{}\{\allowbreak{}q-p2,\ \allowbreak{}m2\},\ \allowbreak{}...]}.

\subsection{See also}

\hyperlink{toc}{Overview}, \hyperlink{pave}{PaVe}.

\subsection{Examples}

\begin{Shaded}
\begin{Highlighting}[]
\NormalTok{FVD}\OperatorTok{[}\FunctionTok{q}\OperatorTok{,} \SpecialCharTok{\textbackslash{}}\OperatorTok{[}\NormalTok{Mu}\OperatorTok{]]}\NormalTok{ FVD}\OperatorTok{[}\FunctionTok{q}\OperatorTok{,} \SpecialCharTok{\textbackslash{}}\OperatorTok{[}\NormalTok{Nu}\OperatorTok{]]}\NormalTok{ FAD}\OperatorTok{[\{}\FunctionTok{q}\OperatorTok{,}\NormalTok{ m0}\OperatorTok{\},} \OperatorTok{\{}\FunctionTok{q} \SpecialCharTok{+}\NormalTok{ p1}\OperatorTok{,}\NormalTok{ m1}\OperatorTok{\},} \OperatorTok{\{}\FunctionTok{q} \SpecialCharTok{+}\NormalTok{ p2}\OperatorTok{,}\NormalTok{ m2}\OperatorTok{\}]}\SpecialCharTok{/}\NormalTok{(}\FunctionTok{I}\SpecialCharTok{*}\FunctionTok{Pi}\SpecialCharTok{\^{}}\DecValTok{2}\NormalTok{) }
 
\NormalTok{TID}\OperatorTok{[}\SpecialCharTok{\%}\OperatorTok{,} \FunctionTok{q}\OperatorTok{,}\NormalTok{ UsePaVeBasis }\OtherTok{{-}\textgreater{}} \ConstantTok{True}\OperatorTok{]} 
 
\NormalTok{TID}\OperatorTok{[}\SpecialCharTok{\%\%}\OperatorTok{,} \FunctionTok{q}\OperatorTok{,}\NormalTok{ UsePaVeBasis }\OtherTok{{-}\textgreater{}} \ConstantTok{True}\OperatorTok{,}\NormalTok{ GenPaVe }\OtherTok{{-}\textgreater{}} \ConstantTok{True}\OperatorTok{]}
\end{Highlighting}
\end{Shaded}

\begin{dmath*}\breakingcomma
-\frac{i q^{\mu } q^{\nu }}{\pi ^2 \left(q^2-\text{m0}^2\right).\left((\text{p1}+q)^2-\text{m1}^2\right).\left((\text{p2}+q)^2-\text{m2}^2\right)}
\end{dmath*}

\begin{dmath*}\breakingcomma
g^{\mu \nu } \;\text{C}_{00}\left(\text{p1}^2,\text{p2}^2,-2 (\text{p1}\cdot \;\text{p2})+\text{p1}^2+\text{p2}^2,\text{m1}^2,\text{m0}^2,\text{m2}^2\right)+\text{p1}^{\mu } \;\text{p1}^{\nu } \;\text{C}_{11}\left(\text{p1}^2,-2 (\text{p1}\cdot \;\text{p2})+\text{p1}^2+\text{p2}^2,\text{p2}^2,\text{m0}^2,\text{m1}^2,\text{m2}^2\right)+\text{p2}^{\mu } \;\text{p2}^{\nu } \;\text{C}_{11}\left(\text{p2}^2,-2 (\text{p1}\cdot \;\text{p2})+\text{p1}^2+\text{p2}^2,\text{p1}^2,\text{m0}^2,\text{m2}^2,\text{m1}^2\right)+\left(\text{p1}^{\nu } \;\text{p2}^{\mu }+\text{p1}^{\mu } \;\text{p2}^{\nu }\right) \;\text{C}_{12}\left(\text{p1}^2,-2 (\text{p1}\cdot \;\text{p2})+\text{p1}^2+\text{p2}^2,\text{p2}^2,\text{m0}^2,\text{m1}^2,\text{m2}^2\right)
\end{dmath*}

\begin{dmath*}\breakingcomma
g^{\mu \nu } \;\text{GenPaVe}\left(\{0,0\},\left(
\begin{array}{cc}
 0 & \;\text{m0} \\
 \;\text{p1} & \;\text{m1} \\
 \;\text{p2} & \;\text{m2} \\
\end{array}
\right)\right)+\text{p1}^{\mu } \;\text{p1}^{\nu } \;\text{GenPaVe}\left(\{1,1\},\left(
\begin{array}{cc}
 0 & \;\text{m0} \\
 \;\text{p1} & \;\text{m1} \\
 \;\text{p2} & \;\text{m2} \\
\end{array}
\right)\right)+\text{p2}^{\mu } \;\text{p2}^{\nu } \;\text{GenPaVe}\left(\{2,2\},\left(
\begin{array}{cc}
 0 & \;\text{m0} \\
 \;\text{p1} & \;\text{m1} \\
 \;\text{p2} & \;\text{m2} \\
\end{array}
\right)\right)+\left(\text{p1}^{\nu } \;\text{p2}^{\mu }+\text{p1}^{\mu } \;\text{p2}^{\nu }\right) \;\text{GenPaVe}\left(\{1,2\},\left(
\begin{array}{cc}
 0 & \;\text{m0} \\
 \;\text{p1} & \;\text{m1} \\
 \;\text{p2} & \;\text{m2} \\
\end{array}
\right)\right)
\end{dmath*}
\end{document}
