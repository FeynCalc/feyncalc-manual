% !TeX program = pdflatex
% !TeX root = GenericPropagatorDenominator.tex

\documentclass[../FeynCalcManual.tex]{subfiles}
\begin{document}
\hypertarget{genericpropagatordenominator}{
\section{GenericPropagatorDenominator}\label{genericpropagatordenominator}\index{GenericPropagatorDenominator}}

\texttt{GenericPropagatorDenominator[\allowbreak{}expr,\ \allowbreak{}\{\allowbreak{}n,\ \allowbreak{}s\}]}
is a generic factor of the denominator of a propagator. Unlike
\texttt{PropagatorDenominator} that is supposed to mean \(1/(q^2-m^2)\),
\texttt{expr} in \texttt{GenericPropagatorDenominator} can be an
arbitrary combination of \texttt{Pair}, \texttt{CartesianPair} and
\texttt{TemporalPair} objects.

\subsection{See also}

\hyperlink{toc}{Overview},
\hyperlink{propagatordenominator}{PropagatorDenominator},
\hyperlink{standardpropagatordenominator}{StandardPropagatorDenominator},
\hyperlink{cartesianpropagatordenominator}{CartesianPropagatorDenominator}.

Using \texttt{n} one can specify the power of the propagator, while
\texttt{s} (\texttt{+1} or \texttt{-1}) fixes the sign of
\texttt{I*eta}. \texttt{GenericPropagatorDenominator} is an internal
object. To enter such propagators in FeynCalc you should use
\texttt{GFAD}.

\subsection{Examples}

\begin{Shaded}
\begin{Highlighting}[]
\NormalTok{FeynAmpDenominator}\OperatorTok{[}\NormalTok{GenericPropagatorDenominator}\OperatorTok{[}\FunctionTok{x}\OperatorTok{,} \OperatorTok{\{}\DecValTok{1}\OperatorTok{,} \DecValTok{1}\OperatorTok{\}]]}
\end{Highlighting}
\end{Shaded}

\begin{dmath*}\breakingcomma
\frac{1}{(x+i \eta )}
\end{dmath*}

\begin{Shaded}
\begin{Highlighting}[]
\NormalTok{FeynAmpDenominator}\OperatorTok{[}\NormalTok{GenericPropagatorDenominator}\OperatorTok{[}\DecValTok{2} \FunctionTok{z}\NormalTok{ Pair}\OperatorTok{[}\NormalTok{Momentum}\OperatorTok{[}\NormalTok{p1}\OperatorTok{,} \FunctionTok{D}\OperatorTok{],} 
\NormalTok{      Momentum}\OperatorTok{[}\FunctionTok{Q}\OperatorTok{,} \FunctionTok{D}\OperatorTok{]]}\NormalTok{ Pair}\OperatorTok{[}\NormalTok{Momentum}\OperatorTok{[}\NormalTok{p2}\OperatorTok{,} \FunctionTok{D}\OperatorTok{],}\NormalTok{ Momentum}\OperatorTok{[}\FunctionTok{Q}\OperatorTok{,} \FunctionTok{D}\OperatorTok{]]} \SpecialCharTok{{-}}\NormalTok{ Pair}\OperatorTok{[}\NormalTok{Momentum}\OperatorTok{[}\NormalTok{p1}\OperatorTok{,} \FunctionTok{D}\OperatorTok{],} 
\NormalTok{     Momentum}\OperatorTok{[}\NormalTok{p2}\OperatorTok{,} \FunctionTok{D}\OperatorTok{]],} \OperatorTok{\{}\DecValTok{1}\OperatorTok{,} \DecValTok{1}\OperatorTok{\}]]}
\end{Highlighting}
\end{Shaded}

\begin{dmath*}\breakingcomma
\frac{1}{(2 z (\text{p1}\cdot Q) (\text{p2}\cdot Q)-\text{p1}\cdot \;\text{p2}+i \eta )}
\end{dmath*}
\end{document}
