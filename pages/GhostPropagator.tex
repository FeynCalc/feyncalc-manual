% !TeX program = pdflatex
% !TeX root = GhostPropagator.tex

\documentclass[../FeynCalcManual.tex]{subfiles}
\begin{document}
\hypertarget{ghostpropagator}{
\section{GhostPropagator}\label{ghostpropagator}\index{GhostPropagator}}

\texttt{GhostPropagator[\allowbreak{}p,\ \allowbreak{}a,\ \allowbreak{}b]}
gives the ghost propagator where \texttt{a} and \texttt{b} are the color
indices.

\texttt{GhostPropagator[\allowbreak{}p]} omits the \(\delta _{ab}\).

\texttt{GHP} can be used as an abbreviation of \texttt{GhostPropagator}.

\subsection{See also}

\hyperlink{toc}{Overview}, \hyperlink{gluonpropagator}{GluonPropagator},
\hyperlink{qcdfeynmanruleconvention}{QCDFeynmanRuleConvention},
\hyperlink{gluonghostvertex}{GluonGhostVertex}.

\subsection{Examples}

\begin{Shaded}
\begin{Highlighting}[]
\NormalTok{GhostPropagator}\OperatorTok{[}\FunctionTok{p}\OperatorTok{,} \FunctionTok{a}\OperatorTok{,} \FunctionTok{b}\OperatorTok{]} 
 
\NormalTok{Explicit}\OperatorTok{[}\SpecialCharTok{\%}\OperatorTok{]}
\end{Highlighting}
\end{Shaded}

\begin{dmath*}\breakingcomma
\Pi _{ab}(p)
\end{dmath*}

\begin{dmath*}\breakingcomma
\frac{i \delta ^{ab}}{p^2}
\end{dmath*}

\begin{Shaded}
\begin{Highlighting}[]
\NormalTok{GHP}\OperatorTok{[}\FunctionTok{p}\OperatorTok{]} 
 
\NormalTok{Explicit}\OperatorTok{[}\SpecialCharTok{\%}\OperatorTok{]}
\end{Highlighting}
\end{Shaded}

\begin{dmath*}\breakingcomma
\Pi _u(p)
\end{dmath*}

\begin{dmath*}\breakingcomma
\frac{i}{p^2}
\end{dmath*}
\end{document}
