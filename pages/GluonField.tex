% !TeX program = pdflatex
% !TeX root = GluonField.tex

\documentclass[../FeynCalcManual.tex]{subfiles}
\begin{document}
\hypertarget{gluonfield}{%
\section{GluonField}\label{gluonfield}}

\texttt{GluonField} is a name of a gauge field.

\subsection{See also}

\hyperlink{toc}{Overview}, \hyperlink{gaugefield}{GaugeField}.

\subsection{Examples}

\begin{Shaded}
\begin{Highlighting}[]
\NormalTok{GluonField}
\end{Highlighting}
\end{Shaded}

\begin{dmath*}\breakingcomma
A
\end{dmath*}

\begin{Shaded}
\begin{Highlighting}[]
\NormalTok{QuantumField}\OperatorTok{[}\NormalTok{GluonField}\OperatorTok{,}\NormalTok{ LorentzIndex}\OperatorTok{[}\SpecialCharTok{\textbackslash{}}\OperatorTok{[}\NormalTok{Mu}\OperatorTok{]],}\NormalTok{ SUNIndex}\OperatorTok{[}\FunctionTok{a}\OperatorTok{]]}
\end{Highlighting}
\end{Shaded}

\begin{dmath*}\breakingcomma
A_{\mu }^a
\end{dmath*}
\end{document}
