% !TeX program = pdflatex
% !TeX root = GluonGhostVertex.tex

\documentclass[../FeynCalcManual.tex]{subfiles}
\begin{document}
\hypertarget{gluonghostvertex}{
\section{GluonGhostVertex}\label{gluonghostvertex}\index{GluonGhostVertex}}

\texttt{GluonGhostVertex[\allowbreak{}\{\allowbreak{}p,\ \allowbreak{}mu,\ \allowbreak{}a\},\ \allowbreak{}\{\allowbreak{}q,\ \allowbreak{}nu,\ \allowbreak{}b\},\ \allowbreak{}\{\allowbreak{}k,\ \allowbreak{}rho,\ \allowbreak{}c\}]}
or
\texttt{GluonGhostVertex[\allowbreak{} p,\ \allowbreak{}mu,\ \allowbreak{}a ,\ \allowbreak{}q,\ \allowbreak{}nu,\ \allowbreak{}b ,\ \allowbreak{}k,\ \allowbreak{}rho,\ \allowbreak{}c]}
yields the Gluon-Ghost vertex. The first argument represents the gluon
and the third argument the outgoing ghost field (but incoming
4-momentum).

\texttt{GGV} can be used as an abbreviation of
\texttt{GluonGhostVertex}.The dimension and the name of the coupling
constant are determined by the options \texttt{Dimension} and
\texttt{CouplingConstant}.

\subsection{See also}

\hyperlink{toc}{Overview}, \hyperlink{gluonpropagator}{GluonPropagator},
\hyperlink{gluonselfenergy}{GluonSelfEnergy},
\hyperlink{gluonvertex}{GluonVertex},
\hyperlink{qcdfeynmanruleconvention}{QCDFeynmanRuleConvention},
\hyperlink{ghostpropagator}{GhostPropagator}.

\subsection{Examples}

\begin{Shaded}
\begin{Highlighting}[]
\NormalTok{GluonGhostVertex}\OperatorTok{[\{}\FunctionTok{p}\OperatorTok{,} \SpecialCharTok{\textbackslash{}}\OperatorTok{[}\NormalTok{Mu}\OperatorTok{],} \FunctionTok{a}\OperatorTok{\},} \OperatorTok{\{}\FunctionTok{q}\OperatorTok{,} \SpecialCharTok{\textbackslash{}}\OperatorTok{[}\NormalTok{Nu}\OperatorTok{],} \FunctionTok{b}\OperatorTok{\},} \OperatorTok{\{}\FunctionTok{k}\OperatorTok{,} \SpecialCharTok{\textbackslash{}}\OperatorTok{[}\NormalTok{Rho}\OperatorTok{],} \FunctionTok{c}\OperatorTok{\}]} 
 
\NormalTok{Explicit}\OperatorTok{[}\SpecialCharTok{\%}\OperatorTok{]}
\end{Highlighting}
\end{Shaded}

\begin{dmath*}\breakingcomma
\tilde{\Lambda }^{\mu }(k) f^{abc}
\end{dmath*}

\begin{dmath*}\breakingcomma
g_s \left(-k^{\mu }\right) f^{abc}
\end{dmath*}
\end{document}
