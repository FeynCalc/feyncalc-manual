% !TeX program = pdflatex
% !TeX root = GluonSelfEnergy.tex

\documentclass[../FeynCalcManual.tex]{subfiles}
\begin{document}
\hypertarget{gluonselfenergy}{%
\section{GluonSelfEnergy}\label{gluonselfenergy}}

\texttt{GluonSelfEnergy[\allowbreak{}\{\allowbreak{}mu,\ \allowbreak{}a\},\ \allowbreak{}\{\allowbreak{}nu,\ \allowbreak{}b\}]}
yields the 1-loop gluon self-energy.

\subsection{See also}

\hyperlink{toc}{Overview}, \hyperlink{gluonpropagator}{GluonPropagator},
\hyperlink{gluonghostvertex}{GluonGhostVertex},
\hyperlink{gluonvertex}{GluonVertex},
\hyperlink{ghostpropagator}{GhostPropagator}.

\subsection{Examples}

\begin{Shaded}
\begin{Highlighting}[]
\NormalTok{GluonSelfEnergy}\OperatorTok{[\{}\SpecialCharTok{\textbackslash{}}\OperatorTok{[}\NormalTok{Mu}\OperatorTok{],} \FunctionTok{a}\OperatorTok{\},} \OperatorTok{\{}\SpecialCharTok{\textbackslash{}}\OperatorTok{[}\NormalTok{Nu}\OperatorTok{],} \FunctionTok{b}\OperatorTok{\},}\NormalTok{ Momentum }\OtherTok{{-}\textgreater{}} \FunctionTok{p}\OperatorTok{]}
\end{Highlighting}
\end{Shaded}

\begin{dmath*}\breakingcomma
\frac{1}{2} i \left(\frac{20}{3 \varepsilon }-\frac{62}{9}\right) C_A g_s^2 \delta ^{ab} \left(p^{\mu } p^{\nu }-p^2 g^{\mu \nu }\right)+i \left(\frac{20}{9}-\frac{8}{3 \varepsilon }\right) T_f g_s^2 \delta ^{ab} \left(p^{\mu } p^{\nu }-p^2 g^{\mu \nu }\right)
\end{dmath*}

\begin{Shaded}
\begin{Highlighting}[]
\NormalTok{GluonSelfEnergy}\OperatorTok{[\{}\SpecialCharTok{\textbackslash{}}\OperatorTok{[}\NormalTok{Mu}\OperatorTok{],} \FunctionTok{a}\OperatorTok{\},} \OperatorTok{\{}\SpecialCharTok{\textbackslash{}}\OperatorTok{[}\NormalTok{Nu}\OperatorTok{],} \FunctionTok{b}\OperatorTok{\},}\NormalTok{ Gauge }\OtherTok{{-}\textgreater{}} \SpecialCharTok{\textbackslash{}}\OperatorTok{[}\NormalTok{Xi}\OperatorTok{],}\NormalTok{ Momentum }\OtherTok{{-}\textgreater{}} \FunctionTok{q}\OperatorTok{]}
\end{Highlighting}
\end{Shaded}

\begin{dmath*}\breakingcomma
\frac{1}{2} i C_A \left(\frac{2 \left(\frac{13}{3}-\xi \right)}{\varepsilon }-\frac{1}{2} (1-\xi )^2+2 (1-\xi )-\frac{62}{9}\right) g_s^2 \delta ^{ab} \left(q^{\mu } q^{\nu }-q^2 g^{\mu \nu }\right)+i \left(\frac{20}{9}-\frac{8}{3 \varepsilon }\right) T_f g_s^2 \delta ^{ab} \left(q^{\mu } q^{\nu }-q^2 g^{\mu \nu }\right)
\end{dmath*}
\end{document}
