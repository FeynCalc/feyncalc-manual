% !TeX program = pdflatex
% !TeX root = GluonVertex.tex

\documentclass[../FeynCalcManual.tex]{subfiles}
\begin{document}
\hypertarget{gluonvertex}{%
\section{GluonVertex}\label{gluonvertex}}

\texttt{GluonVertex[\allowbreak{}\{\allowbreak{}p,\ \allowbreak{}mu,\ \allowbreak{}a\},\ \allowbreak{}\{\allowbreak{}q,\ \allowbreak{}nu,\ \allowbreak{}b\},\ \allowbreak{}\{\allowbreak{}k,\ \allowbreak{}la,\ \allowbreak{}c\}]}
or
\texttt{GluonVertex[\allowbreak{}p,\ \allowbreak{}mu,\ \allowbreak{}a,\ \allowbreak{}q,\ \allowbreak{}nu,\ \allowbreak{}b,\ \allowbreak{}k,\ \allowbreak{}la,\ \allowbreak{}c]}
yields the 3-gluon vertex.

\texttt{GluonVertex[\allowbreak{}\{\allowbreak{}p,\ \allowbreak{}mu\},\ \allowbreak{}\{\allowbreak{}q,\ \allowbreak{}nu\},\ \allowbreak{}\{\allowbreak{}k,\ \allowbreak{}la\}]}
yields the 3-gluon vertex without color structure and the coupling
constant.

\texttt{GluonVertex[\allowbreak{}\{\allowbreak{}p,\ \allowbreak{}mu,\ \allowbreak{}a\},\ \allowbreak{}\{\allowbreak{}q,\ \allowbreak{}nu,\ \allowbreak{}b\},\ \allowbreak{}\{\allowbreak{}k,\ \allowbreak{}la,\ \allowbreak{}c\},\ \allowbreak{}\{\allowbreak{}s,\ \allowbreak{}si,\ \allowbreak{}d\}]}
or
\texttt{GluonVertex[\allowbreak{}\{\allowbreak{}mu,\ \allowbreak{}a\},\ \allowbreak{}\{\allowbreak{}nu,\ \allowbreak{}b\},\ \allowbreak{}\{\allowbreak{}la,\ \allowbreak{}c\},\ \allowbreak{}\{\allowbreak{}si,\ \allowbreak{}d\}]}
or
\texttt{GluonVertex[\allowbreak{}p,\ \allowbreak{}mu,\ \allowbreak{}a,\ \allowbreak{}q,\ \allowbreak{}nu,\ \allowbreak{}b,\ \allowbreak{}k,\ \allowbreak{}la,\ \allowbreak{}c ,\ \allowbreak{}s,\ \allowbreak{}si,\ \allowbreak{}d]}
or
\texttt{GluonVertex[\allowbreak{}mu,\ \allowbreak{}a,\ \allowbreak{}nu,\ \allowbreak{}b,\ \allowbreak{}la,\ \allowbreak{}c,\ \allowbreak{}si,\ \allowbreak{}d]}
yields the 4-gluon vertex.

\texttt{GV} can be used as an abbreviation of \texttt{GluonVertex}.

The dimension and the name of the coupling constant are determined by
the options \texttt{Dimension} and \texttt{CouplingConstant}. All
momenta are flowing into the vertex.

\subsection{See also}

\hyperlink{toc}{Overview}, \hyperlink{gluonpropagator}{GluonPropagator},
\hyperlink{gluonghostvertex}{GluonGhostVertex}.

\subsection{Examples}

\begin{Shaded}
\begin{Highlighting}[]
\NormalTok{GluonVertex}\OperatorTok{[\{}\FunctionTok{p}\OperatorTok{,} \SpecialCharTok{\textbackslash{}}\OperatorTok{[}\NormalTok{Mu}\OperatorTok{],} \FunctionTok{a}\OperatorTok{\},} \OperatorTok{\{}\FunctionTok{q}\OperatorTok{,} \SpecialCharTok{\textbackslash{}}\OperatorTok{[}\NormalTok{Nu}\OperatorTok{],} \FunctionTok{b}\OperatorTok{\},} \OperatorTok{\{}\FunctionTok{r}\OperatorTok{,} \SpecialCharTok{\textbackslash{}}\OperatorTok{[}\NormalTok{Rho}\OperatorTok{],} \FunctionTok{c}\OperatorTok{\}]} 
 
\NormalTok{Explicit}\OperatorTok{[}\SpecialCharTok{\%}\OperatorTok{]}
\end{Highlighting}
\end{Shaded}

\begin{dmath*}\breakingcomma
f^{abc} V^{\mu \nu \rho }(p\text{, }q\text{, }r)
\end{dmath*}

\begin{dmath*}\breakingcomma
g_s f^{abc} \left(g^{\mu \nu } \left(p^{\rho }-q^{\rho }\right)+g^{\mu \rho } \left(r^{\nu }-p^{\nu }\right)+g^{\nu \rho } \left(q^{\mu }-r^{\mu }\right)\right)
\end{dmath*}

\begin{Shaded}
\begin{Highlighting}[]
\NormalTok{GV}\OperatorTok{[\{}\FunctionTok{p}\OperatorTok{,} \SpecialCharTok{\textbackslash{}}\OperatorTok{[}\NormalTok{Mu}\OperatorTok{]\},} \OperatorTok{\{}\FunctionTok{q}\OperatorTok{,} \SpecialCharTok{\textbackslash{}}\OperatorTok{[}\NormalTok{Nu}\OperatorTok{]\},} \OperatorTok{\{}\FunctionTok{r}\OperatorTok{,} \SpecialCharTok{\textbackslash{}}\OperatorTok{[}\NormalTok{Rho}\OperatorTok{]\}]} 
 
\NormalTok{Explicit}\OperatorTok{[}\SpecialCharTok{\%}\OperatorTok{]}
\end{Highlighting}
\end{Shaded}

\begin{dmath*}\breakingcomma
V^{\mu \nu \rho }(p\text{, }q\text{, }r)
\end{dmath*}

\begin{dmath*}\breakingcomma
g_s \left(g^{\mu \nu } \left(p^{\rho }-q^{\rho }\right)+g^{\mu \rho } \left(r^{\nu }-p^{\nu }\right)+g^{\nu \rho } \left(q^{\mu }-r^{\mu }\right)\right)
\end{dmath*}

\begin{Shaded}
\begin{Highlighting}[]
\NormalTok{GluonVertex}\OperatorTok{[\{}\FunctionTok{p}\OperatorTok{,} \SpecialCharTok{\textbackslash{}}\OperatorTok{[}\NormalTok{Mu}\OperatorTok{],} \FunctionTok{a}\OperatorTok{\},} \OperatorTok{\{}\FunctionTok{q}\OperatorTok{,} \SpecialCharTok{\textbackslash{}}\OperatorTok{[}\NormalTok{Nu}\OperatorTok{],} \FunctionTok{b}\OperatorTok{\},} \OperatorTok{\{}\FunctionTok{r}\OperatorTok{,} \SpecialCharTok{\textbackslash{}}\OperatorTok{[}\NormalTok{Rho}\OperatorTok{],} \FunctionTok{c}\OperatorTok{\},} \OperatorTok{\{}\FunctionTok{s}\OperatorTok{,} \SpecialCharTok{\textbackslash{}}\OperatorTok{[}\NormalTok{Sigma}\OperatorTok{],} \FunctionTok{d}\OperatorTok{\}]} 
 
\NormalTok{Explicit}\OperatorTok{[}\SpecialCharTok{\%}\OperatorTok{]}
\end{Highlighting}
\end{Shaded}

\begin{dmath*}\breakingcomma
V_{abcd}^{\mu \nu \rho \sigma }(p\text{, }q\text{, }r\text{, }s)
\end{dmath*}

\begin{dmath*}\breakingcomma
-i g_s^2 \left(f^{ad\text{FCGV}(\text{u19})} f^{bc\text{FCGV}(\text{u19})} \left(g^{\mu \nu } g^{\rho \sigma }-g^{\mu \rho } g^{\nu \sigma }\right)+f^{ac\text{FCGV}(\text{u19})} f^{bd\text{FCGV}(\text{u19})} \left(g^{\mu \nu } g^{\rho \sigma }-g^{\mu \sigma } g^{\nu \rho }\right)+f^{ab\text{FCGV}(\text{u19})} f^{cd\text{FCGV}(\text{u19})} \left(g^{\mu \rho } g^{\nu \sigma }-g^{\mu \sigma } g^{\nu \rho }\right)\right)
\end{dmath*}

\begin{Shaded}
\begin{Highlighting}[]
\NormalTok{GV}\OperatorTok{[\{}\SpecialCharTok{\textbackslash{}}\OperatorTok{[}\NormalTok{Mu}\OperatorTok{],} \FunctionTok{a}\OperatorTok{\},} \OperatorTok{\{}\SpecialCharTok{\textbackslash{}}\OperatorTok{[}\NormalTok{Nu}\OperatorTok{],} \FunctionTok{b}\OperatorTok{\},} \OperatorTok{\{}\SpecialCharTok{\textbackslash{}}\OperatorTok{[}\NormalTok{Rho}\OperatorTok{],} \FunctionTok{c}\OperatorTok{\},} \OperatorTok{\{}\SpecialCharTok{\textbackslash{}}\OperatorTok{[}\NormalTok{Sigma}\OperatorTok{],} \FunctionTok{d}\OperatorTok{\}]} 
 
\NormalTok{Explicit}\OperatorTok{[}\SpecialCharTok{\%}\OperatorTok{]}
\end{Highlighting}
\end{Shaded}

\begin{dmath*}\breakingcomma
V^{abcd}
\end{dmath*}

\begin{dmath*}\breakingcomma
-i g_s^2 \left(f^{ad\text{FCGV}(\text{u20})} f^{bc\text{FCGV}(\text{u20})} \left(g^{\mu \nu } g^{\rho \sigma }-g^{\mu \rho } g^{\nu \sigma }\right)+f^{ac\text{FCGV}(\text{u20})} f^{bd\text{FCGV}(\text{u20})} \left(g^{\mu \nu } g^{\rho \sigma }-g^{\mu \sigma } g^{\nu \rho }\right)+f^{ab\text{FCGV}(\text{u20})} f^{cd\text{FCGV}(\text{u20})} \left(g^{\mu \rho } g^{\nu \sigma }-g^{\mu \sigma } g^{\nu \rho }\right)\right)
\end{dmath*}
\end{document}
