% !TeX program = pdflatex
% !TeX root = GordonSimplify.tex

\documentclass[../FeynCalcManual.tex]{subfiles}
\begin{document}
\hypertarget{gordonsimplify}{
\section{GordonSimplify}\label{gordonsimplify}\index{GordonSimplify}}

\texttt{GordonSimplify[\allowbreak{}exp]} rewrites spinor chains
describing a vector or an axial-vector current using Gordon identities.

\subsection{See also}

\hyperlink{toc}{Overview}, \hyperlink{diracgamma}{DiracGamma},
\hyperlink{spinor}{Spinor},
\hyperlink{spinorchaintrick}{SpinorChainTrick}.

\subsection{Examples}

\begin{Shaded}
\begin{Highlighting}[]
\NormalTok{SpinorUBar}\OperatorTok{[}\NormalTok{p1}\OperatorTok{,}\NormalTok{ m1}\OperatorTok{]}\NormalTok{ . GA}\OperatorTok{[}\SpecialCharTok{\textbackslash{}}\OperatorTok{[}\NormalTok{Mu}\OperatorTok{]]}\NormalTok{ . SpinorU}\OperatorTok{[}\NormalTok{p2}\OperatorTok{,}\NormalTok{ m2}\OperatorTok{]} 
 
\NormalTok{GordonSimplify}\OperatorTok{[}\SpecialCharTok{\%}\OperatorTok{]}
\end{Highlighting}
\end{Shaded}

\begin{dmath*}\breakingcomma
\bar{u}(\text{p1},\text{m1}).\bar{\gamma }^{\mu }.u(\text{p2},\text{m2})
\end{dmath*}

\begin{dmath*}\breakingcomma
\frac{\left(\overline{\text{p1}}+\overline{\text{p2}}\right)^{\mu } \left(\varphi (\overline{\text{p1}},\text{m1})\right).\left(\varphi (\overline{\text{p2}},\text{m2})\right)}{\text{m1}+\text{m2}}+\frac{i \left(\varphi (\overline{\text{p1}},\text{m1})\right).\sigma ^{\mu \overline{\text{p1}}-\overline{\text{p2}}}.\left(\varphi (\overline{\text{p2}},\text{m2})\right)}{\text{m1}+\text{m2}}
\end{dmath*}

\begin{Shaded}
\begin{Highlighting}[]
\NormalTok{SpinorUBar}\OperatorTok{[}\NormalTok{p1}\OperatorTok{,}\NormalTok{ m1}\OperatorTok{]}\NormalTok{ . GA}\OperatorTok{[}\SpecialCharTok{\textbackslash{}}\OperatorTok{[}\NormalTok{Mu}\OperatorTok{],} \DecValTok{5}\OperatorTok{]}\NormalTok{ . SpinorV}\OperatorTok{[}\NormalTok{p2}\OperatorTok{,}\NormalTok{ m2}\OperatorTok{]} 
 
\NormalTok{GordonSimplify}\OperatorTok{[}\SpecialCharTok{\%}\OperatorTok{]}
\end{Highlighting}
\end{Shaded}

\begin{dmath*}\breakingcomma
\bar{u}(\text{p1},\text{m1}).\bar{\gamma }^{\mu }.\bar{\gamma }^5.v(\text{p2},\text{m2})
\end{dmath*}

\begin{dmath*}\breakingcomma
\frac{\left(\overline{\text{p1}}+\overline{\text{p2}}\right)^{\mu } \left(\varphi (\overline{\text{p1}},\text{m1})\right).\bar{\gamma }^5.\left(\varphi (-\overline{\text{p2}},\text{m2})\right)}{\text{m1}+\text{m2}}+\frac{i \left(\varphi (\overline{\text{p1}},\text{m1})\right).\sigma ^{\mu \overline{\text{p1}}-\overline{\text{p2}}}.\bar{\gamma }^5.\left(\varphi (-\overline{\text{p2}},\text{m2})\right)}{\text{m1}+\text{m2}}
\end{dmath*}

Relations involving projectors can be used to trade the right projector
for a left one

\begin{Shaded}
\begin{Highlighting}[]
\NormalTok{SpinorVBar}\OperatorTok{[}\NormalTok{p1}\OperatorTok{,}\NormalTok{ m1}\OperatorTok{]}\NormalTok{ . GA}\OperatorTok{[}\SpecialCharTok{\textbackslash{}}\OperatorTok{[}\NormalTok{Mu}\OperatorTok{],} \DecValTok{6}\OperatorTok{]}\NormalTok{ . SpinorV}\OperatorTok{[}\NormalTok{p2}\OperatorTok{,}\NormalTok{ m2}\OperatorTok{]} 
 
\NormalTok{GordonSimplify}\OperatorTok{[}\SpecialCharTok{\%}\OperatorTok{]}
\end{Highlighting}
\end{Shaded}

\begin{dmath*}\breakingcomma
\bar{v}(\text{p1},\text{m1}).\bar{\gamma }^{\mu }.\bar{\gamma }^6.v(\text{p2},\text{m2})
\end{dmath*}

\begin{dmath*}\breakingcomma
-\frac{i \left(\varphi (-\overline{\text{p1}},\text{m1})\right).\sigma ^{\mu \overline{\text{p1}}-\overline{\text{p2}}}.\bar{\gamma }^6.\left(\varphi (-\overline{\text{p2}},\text{m2})\right)}{\text{m1}}-\frac{\text{m2} \left(\varphi (-\overline{\text{p1}},\text{m1})\right).\bar{\gamma }^{\mu }.\bar{\gamma }^7.\left(\varphi (-\overline{\text{p2}},\text{m2})\right)}{\text{m1}}-\frac{\left(\overline{\text{p1}}+\overline{\text{p2}}\right)^{\mu } \left(\varphi (-\overline{\text{p1}},\text{m1})\right).\bar{\gamma }^6.\left(\varphi (-\overline{\text{p2}},\text{m2})\right)}{\text{m1}}
\end{dmath*}

Use the \texttt{Select} option to achieve the opposite

\begin{Shaded}
\begin{Highlighting}[]
\NormalTok{ex }\ExtensionTok{=}\NormalTok{ SpinorVBar}\OperatorTok{[}\NormalTok{p1}\OperatorTok{,}\NormalTok{ m1}\OperatorTok{]}\NormalTok{ . GA}\OperatorTok{[}\SpecialCharTok{\textbackslash{}}\OperatorTok{[}\NormalTok{Mu}\OperatorTok{],} \DecValTok{7}\OperatorTok{]}\NormalTok{ . SpinorV}\OperatorTok{[}\NormalTok{p2}\OperatorTok{,}\NormalTok{ m2}\OperatorTok{]} 
 
\NormalTok{GordonSimplify}\OperatorTok{[}\NormalTok{ex}\OperatorTok{]}
\end{Highlighting}
\end{Shaded}

\begin{dmath*}\breakingcomma
\bar{v}(\text{p1},\text{m1}).\bar{\gamma }^{\mu }.\bar{\gamma }^7.v(\text{p2},\text{m2})
\end{dmath*}

\begin{dmath*}\breakingcomma
\left(\varphi (-\overline{\text{p1}},\text{m1})\right).\bar{\gamma }^{\mu }.\bar{\gamma }^7.\left(\varphi (-\overline{\text{p2}},\text{m2})\right)
\end{dmath*}

\begin{Shaded}
\begin{Highlighting}[]
\NormalTok{GordonSimplify}\OperatorTok{[}\NormalTok{ex}\OperatorTok{,} \FunctionTok{Select} \OtherTok{{-}\textgreater{}} \OperatorTok{\{\{}\NormalTok{Spinor}\OperatorTok{[}\AttributeTok{\_\_}\OperatorTok{],}\NormalTok{ DiracGamma}\OperatorTok{[}\AttributeTok{\_\_}\OperatorTok{],}\NormalTok{ GA}\OperatorTok{[}\DecValTok{7}\OperatorTok{],}\NormalTok{ Spinor}\OperatorTok{[}\AttributeTok{\_\_}\OperatorTok{]\}\}]}
\end{Highlighting}
\end{Shaded}

\begin{dmath*}\breakingcomma
-\frac{i \left(\varphi (-\overline{\text{p1}},\text{m1})\right).\sigma ^{\mu \overline{\text{p1}}-\overline{\text{p2}}}.\bar{\gamma }^7.\left(\varphi (-\overline{\text{p2}},\text{m2})\right)}{\text{m1}}-\frac{\text{m2} \left(\varphi (-\overline{\text{p1}},\text{m1})\right).\bar{\gamma }^{\mu }.\bar{\gamma }^6.\left(\varphi (-\overline{\text{p2}},\text{m2})\right)}{\text{m1}}-\frac{\left(\overline{\text{p1}}+\overline{\text{p2}}\right)^{\mu } \left(\varphi (-\overline{\text{p1}},\text{m1})\right).\bar{\gamma }^7.\left(\varphi (-\overline{\text{p2}},\text{m2})\right)}{\text{m1}}
\end{dmath*}

We can choose between having expressions proportional to \(1/m_1\) (mass
of the first spinor) or \(1/m_2\) (mass of the second spinor)

\begin{Shaded}
\begin{Highlighting}[]
\NormalTok{GordonSimplify}\OperatorTok{[}\NormalTok{SpinorVBar}\OperatorTok{[}\NormalTok{p1}\OperatorTok{,}\NormalTok{ m1}\OperatorTok{]}\NormalTok{ . GA}\OperatorTok{[}\SpecialCharTok{\textbackslash{}}\OperatorTok{[}\NormalTok{Mu}\OperatorTok{],} \DecValTok{6}\OperatorTok{]}\NormalTok{ . SpinorV}\OperatorTok{[}\NormalTok{p2}\OperatorTok{,}\NormalTok{ m2}\OperatorTok{],} \FunctionTok{Inverse} \OtherTok{{-}\textgreater{}} \FunctionTok{First}\OperatorTok{]}
\end{Highlighting}
\end{Shaded}

\begin{dmath*}\breakingcomma
-\frac{i \left(\varphi (-\overline{\text{p1}},\text{m1})\right).\sigma ^{\mu \overline{\text{p1}}-\overline{\text{p2}}}.\bar{\gamma }^6.\left(\varphi (-\overline{\text{p2}},\text{m2})\right)}{\text{m1}}-\frac{\text{m2} \left(\varphi (-\overline{\text{p1}},\text{m1})\right).\bar{\gamma }^{\mu }.\bar{\gamma }^7.\left(\varphi (-\overline{\text{p2}},\text{m2})\right)}{\text{m1}}-\frac{\left(\overline{\text{p1}}+\overline{\text{p2}}\right)^{\mu } \left(\varphi (-\overline{\text{p1}},\text{m1})\right).\bar{\gamma }^6.\left(\varphi (-\overline{\text{p2}},\text{m2})\right)}{\text{m1}}
\end{dmath*}

\begin{Shaded}
\begin{Highlighting}[]
\NormalTok{GordonSimplify}\OperatorTok{[}\NormalTok{SpinorVBar}\OperatorTok{[}\NormalTok{p1}\OperatorTok{,}\NormalTok{ m1}\OperatorTok{]}\NormalTok{ . GA}\OperatorTok{[}\SpecialCharTok{\textbackslash{}}\OperatorTok{[}\NormalTok{Mu}\OperatorTok{],} \DecValTok{6}\OperatorTok{]}\NormalTok{ . SpinorV}\OperatorTok{[}\NormalTok{p2}\OperatorTok{,}\NormalTok{ m2}\OperatorTok{],} \FunctionTok{Inverse} \OtherTok{{-}\textgreater{}} \FunctionTok{Last}\OperatorTok{]}
\end{Highlighting}
\end{Shaded}

\begin{dmath*}\breakingcomma
-\frac{i \left(\varphi (-\overline{\text{p1}},\text{m1})\right).\sigma ^{\mu \overline{\text{p1}}-\overline{\text{p2}}}.\bar{\gamma }^7.\left(\varphi (-\overline{\text{p2}},\text{m2})\right)}{\text{m2}}-\frac{\text{m1} \left(\varphi (-\overline{\text{p1}},\text{m1})\right).\bar{\gamma }^{\mu }.\bar{\gamma }^7.\left(\varphi (-\overline{\text{p2}},\text{m2})\right)}{\text{m2}}-\frac{\left(\overline{\text{p1}}+\overline{\text{p2}}\right)^{\mu } \left(\varphi (-\overline{\text{p1}},\text{m1})\right).\bar{\gamma }^7.\left(\varphi (-\overline{\text{p2}},\text{m2})\right)}{\text{m2}}
\end{dmath*}

In \(D\)-dimensions chiral Gordon identities are scheme dependent!

\begin{Shaded}
\begin{Highlighting}[]
\NormalTok{ex }\ExtensionTok{=}\NormalTok{ SpinorVBarD}\OperatorTok{[}\NormalTok{p1}\OperatorTok{,}\NormalTok{ m1}\OperatorTok{]}\NormalTok{ . GAD}\OperatorTok{[}\SpecialCharTok{\textbackslash{}}\OperatorTok{[}\NormalTok{Mu}\OperatorTok{],} \DecValTok{5}\OperatorTok{]}\NormalTok{ . SpinorVD}\OperatorTok{[}\NormalTok{p2}\OperatorTok{,}\NormalTok{ m2}\OperatorTok{]}
\end{Highlighting}
\end{Shaded}

\begin{dmath*}\breakingcomma
\bar{v}(\text{p1},\text{m1}).\gamma ^{\mu }.\bar{\gamma }^5.v(\text{p2},\text{m2})
\end{dmath*}

\begin{Shaded}
\begin{Highlighting}[]
\NormalTok{FCGetDiracGammaScheme}\OperatorTok{[]} 
 
\NormalTok{GordonSimplify}\OperatorTok{[}\NormalTok{ex}\OperatorTok{]}
\end{Highlighting}
\end{Shaded}

\begin{dmath*}\breakingcomma
\text{NDR}
\end{dmath*}

\begin{dmath*}\breakingcomma
-\frac{(\text{p1}+\text{p2})^{\mu } (\varphi (-\text{p1},\text{m1})).\bar{\gamma }^5.(\varphi (-\text{p2},\text{m2}))}{\text{m1}-\text{m2}}-\frac{i (\varphi (-\text{p1},\text{m1})).\sigma ^{\mu \;\text{p1}-\text{p2}}.\bar{\gamma }^5.(\varphi (-\text{p2},\text{m2}))}{\text{m1}-\text{m2}}
\end{dmath*}

\begin{Shaded}
\begin{Highlighting}[]
\NormalTok{FCSetDiracGammaScheme}\OperatorTok{[}\StringTok{"BMHV"}\OperatorTok{]} 
 
\NormalTok{GordonSimplify}\OperatorTok{[}\NormalTok{ex}\OperatorTok{]}
\end{Highlighting}
\end{Shaded}

\begin{dmath*}\breakingcomma
\text{BMHV}
\end{dmath*}

\begin{dmath*}\breakingcomma
-\frac{i (\varphi (-\text{p1},\text{m1})).\sigma ^{\mu \;\text{p1}-\text{p2}}.\bar{\gamma }^5.(\varphi (-\text{p2},\text{m2}))}{\text{m1}-\text{m2}}-\frac{(\text{p1}+\text{p2})^{\mu } (\varphi (-\text{p1},\text{m1})).\bar{\gamma }^5.(\varphi (-\text{p2},\text{m2}))}{\text{m1}-\text{m2}}+\frac{2 (\varphi (-\text{p1},\text{m1})).\gamma ^{\mu }.\left(\hat{\gamma }\cdot \hat{\text{p2}}\right).\bar{\gamma }^5.(\varphi (-\text{p2},\text{m2}))}{\text{m1}-\text{m2}}
\end{dmath*}

\begin{Shaded}
\begin{Highlighting}[]
\NormalTok{FCSetDiracGammaScheme}\OperatorTok{[}\StringTok{"NDR"}\OperatorTok{]}
\end{Highlighting}
\end{Shaded}

\begin{dmath*}\breakingcomma
\text{NDR}
\end{dmath*}
\end{document}
