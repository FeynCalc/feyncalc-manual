% !TeX program = pdflatex
% !TeX root = HypExplicit.tex

\documentclass[../FeynCalcManual.tex]{subfiles}
\begin{document}
\hypertarget{hypexplicit}{%
\section{HypExplicit}\label{hypexplicit}}

\texttt{HypExplicit[\allowbreak{}exp,\ \allowbreak{}nu]} expresses
Hypergeometric functions in exp by their definition in terms of a sum
(the \texttt{Sum} is omitted and \texttt{nu} is the summation index).

\subsection{See also}

\hyperlink{toc}{Overview},
\hyperlink{hypergeometricir}{HypergeometricIR}.

\subsection{Examples}

\begin{Shaded}
\begin{Highlighting}[]
\FunctionTok{Hypergeometric2F1}\OperatorTok{[}\FunctionTok{a}\OperatorTok{,} \FunctionTok{b}\OperatorTok{,} \FunctionTok{c}\OperatorTok{,} \FunctionTok{z}\OperatorTok{]} 
 
\NormalTok{HypExplicit}\OperatorTok{[}\SpecialCharTok{\%}\OperatorTok{,} \SpecialCharTok{\textbackslash{}}\OperatorTok{[}\NormalTok{Nu}\OperatorTok{]]}
\end{Highlighting}
\end{Shaded}

\begin{dmath*}\breakingcomma
\, _2F_1(a,b;c;z)
\end{dmath*}

\begin{dmath*}\breakingcomma
\frac{\Gamma (c) z^{\nu } \Gamma (a+\nu ) \Gamma (b+\nu )}{\Gamma (a) \Gamma (b) \Gamma (\nu +1) \Gamma (c+\nu )}
\end{dmath*}

\begin{Shaded}
\begin{Highlighting}[]
\FunctionTok{HypergeometricPFQ}\OperatorTok{[\{}\FunctionTok{a}\OperatorTok{,} \FunctionTok{b}\OperatorTok{,} \FunctionTok{c}\OperatorTok{\},} \OperatorTok{\{}\FunctionTok{d}\OperatorTok{,} \FunctionTok{e}\OperatorTok{\},} \FunctionTok{z}\OperatorTok{]} 
 
\NormalTok{HypExplicit}\OperatorTok{[}\SpecialCharTok{\%}\OperatorTok{,} \SpecialCharTok{\textbackslash{}}\OperatorTok{[}\NormalTok{Nu}\OperatorTok{]]}
\end{Highlighting}
\end{Shaded}

\begin{dmath*}\breakingcomma
\, _3F_2(a,b,c;d,e;z)
\end{dmath*}

\begin{dmath*}\breakingcomma
\frac{\Gamma (d) \Gamma (e) z^{\nu } \Gamma (a+\nu ) \Gamma (b+\nu ) \Gamma (c+\nu )}{\Gamma (a) \Gamma (b) \Gamma (c) \Gamma (\nu +1) \Gamma (d+\nu ) \Gamma (e+\nu )}
\end{dmath*}
\end{document}
