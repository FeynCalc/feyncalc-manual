% !TeX program = pdflatex
% !TeX root = HypergeometricAC.tex

\documentclass[../FeynCalcManual.tex]{subfiles}
\begin{document}
\hypertarget{hypergeometricac}{%
\section{HypergeometricAC}\label{hypergeometricac}}

\texttt{HypergeometricAC[\allowbreak{}n][\allowbreak{}exp]} analytically
continues \texttt{Hypergeometric2F1} functions in \texttt{exp}. The
second argument \texttt{n} refers to the equation number (\(n\)) in
chapter 2.10 of ``Higher Transcendental Functions'' by Erdelyi, Magnus,
Oberhettinger, Tricomi. In case of eq. (6) (p.109) the last line is
returned for
\texttt{HypergeometricAC[\allowbreak{}6][\allowbreak{}exp]}, while the
first equality is given by
\texttt{HypergeometricAC[\allowbreak{}61][\allowbreak{}exp]}.

(2.10.1) is identical to eq. (9.5.7) of ``Special Functions \& their
Applications'' by N.N.Lebedev.

\subsection{See also}

\hyperlink{toc}{Overview}, \hyperlink{hypexplicit}{HypExplicit},
\hyperlink{hypergeometricir}{HypergeometricIR},
\hyperlink{hypergeometricse}{HypergeometricSE},
\hyperlink{tohypergeometric}{ToHypergeometric}.

\subsection{Examples}

These are all transformation rules currently built in.

\begin{Shaded}
\begin{Highlighting}[]
\NormalTok{HypergeometricAC}\OperatorTok{[}\DecValTok{1}\OperatorTok{][}\FunctionTok{Hypergeometric2F1}\OperatorTok{[}\SpecialCharTok{\textbackslash{}}\OperatorTok{[}\NormalTok{Alpha}\OperatorTok{],} \SpecialCharTok{\textbackslash{}}\OperatorTok{[}\FunctionTok{Beta}\OperatorTok{],} \SpecialCharTok{\textbackslash{}}\OperatorTok{[}\FunctionTok{Gamma}\OperatorTok{],} \FunctionTok{z}\OperatorTok{]]}
\end{Highlighting}
\end{Shaded}

\begin{dmath*}\breakingcomma
\frac{\Gamma (\gamma ) \Gamma (\alpha +\beta -\gamma ) (1-z)^{-\alpha -\beta +\gamma } \, _2F_1(\gamma -\alpha ,\gamma -\beta ;-\alpha -\beta +\gamma +1;1-z)}{\Gamma (\alpha ) \Gamma (\beta )}+\frac{\Gamma (\gamma ) \Gamma (-\alpha -\beta +\gamma ) \, _2F_1(\alpha ,\beta ;\alpha +\beta -\gamma +1;1-z)}{\Gamma (\gamma -\alpha ) \Gamma (\gamma -\beta )}
\end{dmath*}

\begin{Shaded}
\begin{Highlighting}[]
\NormalTok{HypergeometricAC}\OperatorTok{[}\DecValTok{2}\OperatorTok{][}\FunctionTok{Hypergeometric2F1}\OperatorTok{[}\SpecialCharTok{\textbackslash{}}\OperatorTok{[}\NormalTok{Alpha}\OperatorTok{],} \SpecialCharTok{\textbackslash{}}\OperatorTok{[}\FunctionTok{Beta}\OperatorTok{],} \SpecialCharTok{\textbackslash{}}\OperatorTok{[}\FunctionTok{Gamma}\OperatorTok{],} \FunctionTok{z}\OperatorTok{]]}
\end{Highlighting}
\end{Shaded}

\begin{dmath*}\breakingcomma
\frac{\Gamma (\gamma ) (-z)^{-\alpha } \Gamma (\beta -\alpha ) \, _2F_1\left(\alpha ,\alpha -\gamma +1;\alpha -\beta +1;\frac{1}{z}\right)}{\Gamma (\beta ) \Gamma (\gamma -\alpha )}+\frac{\Gamma (\gamma ) (-z)^{-\beta } \Gamma (\alpha -\beta ) \, _2F_1\left(\beta ,\beta -\gamma +1;-\alpha +\beta +1;\frac{1}{z}\right)}{\Gamma (\alpha ) \Gamma (\gamma -\beta )}
\end{dmath*}

\begin{Shaded}
\begin{Highlighting}[]
\NormalTok{HypergeometricAC}\OperatorTok{[}\DecValTok{3}\OperatorTok{][}\FunctionTok{Hypergeometric2F1}\OperatorTok{[}\SpecialCharTok{\textbackslash{}}\OperatorTok{[}\NormalTok{Alpha}\OperatorTok{],} \SpecialCharTok{\textbackslash{}}\OperatorTok{[}\FunctionTok{Beta}\OperatorTok{],} \SpecialCharTok{\textbackslash{}}\OperatorTok{[}\FunctionTok{Gamma}\OperatorTok{],} \FunctionTok{z}\OperatorTok{]]}
\end{Highlighting}
\end{Shaded}

\begin{dmath*}\breakingcomma
\frac{\Gamma (\gamma ) (1-z)^{-\alpha } \Gamma (\beta -\alpha ) \, _2F_1\left(\alpha ,\gamma -\beta ;\alpha -\beta +1;\frac{1}{1-z}\right)}{\Gamma (\beta ) \Gamma (\gamma -\alpha )}+\frac{\Gamma (\gamma ) (1-z)^{-\beta } \Gamma (\alpha -\beta ) \, _2F_1\left(\beta ,\gamma -\alpha ;-\alpha +\beta +1;\frac{1}{1-z}\right)}{\Gamma (\alpha ) \Gamma (\gamma -\beta )}
\end{dmath*}

\begin{Shaded}
\begin{Highlighting}[]
\NormalTok{HypergeometricAC}\OperatorTok{[}\DecValTok{4}\OperatorTok{][}\FunctionTok{Hypergeometric2F1}\OperatorTok{[}\SpecialCharTok{\textbackslash{}}\OperatorTok{[}\NormalTok{Alpha}\OperatorTok{],} \SpecialCharTok{\textbackslash{}}\OperatorTok{[}\FunctionTok{Beta}\OperatorTok{],} \SpecialCharTok{\textbackslash{}}\OperatorTok{[}\FunctionTok{Gamma}\OperatorTok{],} \FunctionTok{z}\OperatorTok{]]}
\end{Highlighting}
\end{Shaded}

\begin{dmath*}\breakingcomma
\frac{\Gamma (\gamma ) z^{-\alpha } \Gamma (-\alpha -\beta +\gamma ) \, _2F_1\left(\alpha ,\alpha -\gamma +1;\alpha +\beta -\gamma +1;-\frac{1-z}{z}\right)}{\Gamma (\gamma -\alpha ) \Gamma (\gamma -\beta )}+\frac{\Gamma (\gamma ) z^{\alpha -\gamma } \Gamma (\alpha +\beta -\gamma ) (1-z)^{-\alpha -\beta +\gamma } \, _2F_1\left(1-\alpha ,\gamma -\alpha ;-\alpha -\beta +\gamma +1;-\frac{1-z}{z}\right)}{\Gamma (\alpha ) \Gamma (\beta )}
\end{dmath*}

\begin{Shaded}
\begin{Highlighting}[]
\NormalTok{HypergeometricAC}\OperatorTok{[}\DecValTok{6}\OperatorTok{][}\FunctionTok{Hypergeometric2F1}\OperatorTok{[}\SpecialCharTok{\textbackslash{}}\OperatorTok{[}\NormalTok{Alpha}\OperatorTok{],} \SpecialCharTok{\textbackslash{}}\OperatorTok{[}\FunctionTok{Beta}\OperatorTok{],} \SpecialCharTok{\textbackslash{}}\OperatorTok{[}\FunctionTok{Gamma}\OperatorTok{],} \FunctionTok{z}\OperatorTok{]]}
\end{Highlighting}
\end{Shaded}

\begin{dmath*}\breakingcomma
(1-z)^{-\beta } \, _2F_1\left(\beta ,\gamma -\alpha ;\gamma ;-\frac{z}{1-z}\right)
\end{dmath*}

\begin{Shaded}
\begin{Highlighting}[]
\NormalTok{HypergeometricAC}\OperatorTok{[}\DecValTok{61}\OperatorTok{][}\FunctionTok{Hypergeometric2F1}\OperatorTok{[}\SpecialCharTok{\textbackslash{}}\OperatorTok{[}\NormalTok{Alpha}\OperatorTok{],} \SpecialCharTok{\textbackslash{}}\OperatorTok{[}\FunctionTok{Beta}\OperatorTok{],} \SpecialCharTok{\textbackslash{}}\OperatorTok{[}\FunctionTok{Gamma}\OperatorTok{],} \FunctionTok{z}\OperatorTok{]]}
\end{Highlighting}
\end{Shaded}

\begin{dmath*}\breakingcomma
(1-z)^{-\alpha } \, _2F_1\left(\alpha ,\gamma -\beta ;\gamma ;-\frac{z}{1-z}\right)
\end{dmath*}
\end{document}
