% !TeX program = pdflatex
% !TeX root = HypergeometricIR.tex

\documentclass[../FeynCalcManual.tex]{subfiles}
\begin{document}
\hypertarget{hypergeometricir}{
\section{HypergeometricIR}\label{hypergeometricir}\index{HypergeometricIR}}

\texttt{HypergeometricIR[\allowbreak{}exp,\ \allowbreak{}t]} substitutes
for all
\texttt{Hypergeometric2F1[\allowbreak{}a,\ \allowbreak{}b,\ \allowbreak{}c,\ \allowbreak{}z]}
in \texttt{exp} by its Euler integral representation. The factor
\texttt{Integratedx[\allowbreak{}t,\ \allowbreak{}0,\ \allowbreak{}1]}
can be omitted by setting the option \texttt{Integratedx -> False}.

\subsection{See also}

\hyperlink{toc}{Overview},
\hyperlink{hypergeometricac}{HypergeometricAC},
\hyperlink{hypergeometricse}{HypergeometricSE},
\hyperlink{tohypergeometric}{ToHypergeometric}.

\subsection{Examples}

\begin{Shaded}
\begin{Highlighting}[]
\NormalTok{HypergeometricIR}\OperatorTok{[}\FunctionTok{Hypergeometric2F1}\OperatorTok{[}\FunctionTok{a}\OperatorTok{,} \FunctionTok{b}\OperatorTok{,} \FunctionTok{c}\OperatorTok{,} \FunctionTok{z}\OperatorTok{],} \FunctionTok{t}\OperatorTok{]}
\end{Highlighting}
\end{Shaded}

\begin{dmath*}\breakingcomma
\frac{t^{b-1} \Gamma (c) (1-t z)^{-a} (1-t)^{-b+c-1}}{\Gamma (b) \Gamma (c-b)}
\end{dmath*}

\begin{Shaded}
\begin{Highlighting}[]
\NormalTok{ToHypergeometric}\OperatorTok{[}\FunctionTok{t}\SpecialCharTok{\^{}}\FunctionTok{b}\NormalTok{ (}\DecValTok{1} \SpecialCharTok{{-}} \FunctionTok{t}\NormalTok{)}\SpecialCharTok{\^{}}\FunctionTok{c}\NormalTok{ (}\DecValTok{1} \SpecialCharTok{+} \FunctionTok{t} \FunctionTok{z}\NormalTok{)}\SpecialCharTok{\^{}}\FunctionTok{a}\OperatorTok{,} \FunctionTok{t}\OperatorTok{]} 
 
\NormalTok{HypergeometricIR}\OperatorTok{[}\SpecialCharTok{\%}\OperatorTok{,} \FunctionTok{t}\OperatorTok{]}
\end{Highlighting}
\end{Shaded}

\begin{dmath*}\breakingcomma
\frac{\Gamma (b+1) \Gamma (c+1) \, _2F_1(-a,b+1;b+c+2;-z)}{\Gamma (b+c+2)}
\end{dmath*}

\begin{dmath*}\breakingcomma
t^b (1-t)^c (t z+1)^a
\end{dmath*}
\end{document}
