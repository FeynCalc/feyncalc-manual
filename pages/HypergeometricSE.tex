% !TeX program = pdflatex
% !TeX root = HypergeometricSE.tex

\documentclass[../FeynCalcManual.tex]{subfiles}
\begin{document}
\hypertarget{hypergeometricse}{
\section{HypergeometricSE}\label{hypergeometricse}\index{HypergeometricSE}}

\texttt{HypergeometricSE[\allowbreak{}exp,\ \allowbreak{}nu]} expresses
Hypergeometric functions by their series expansion in terms of a sum
(the \texttt{Sum} is omitted and \texttt{nu}, running from \(0\) to
\(\infty\), is the summation index).

\subsection{See also}

\hyperlink{toc}{Overview},
\hyperlink{hypergeometricir}{HypergeometricIR}.

\subsection{Examples}

\begin{Shaded}
\begin{Highlighting}[]
\NormalTok{HypergeometricSE}\OperatorTok{[}\FunctionTok{Hypergeometric2F1}\OperatorTok{[}\FunctionTok{a}\OperatorTok{,} \FunctionTok{b}\OperatorTok{,} \FunctionTok{c}\OperatorTok{,} \FunctionTok{z}\OperatorTok{],} \SpecialCharTok{\textbackslash{}}\OperatorTok{[}\NormalTok{Nu}\OperatorTok{]]}
\end{Highlighting}
\end{Shaded}

\begin{dmath*}\breakingcomma
\frac{\Gamma (c) z^{\nu } \Gamma (a+\nu ) \Gamma (b+\nu )}{\Gamma (a) \Gamma (b) \Gamma (\nu +1) \Gamma (c+\nu )}
\end{dmath*}

\begin{Shaded}
\begin{Highlighting}[]
\NormalTok{HypergeometricSE}\OperatorTok{[}\FunctionTok{HypergeometricPFQ}\OperatorTok{[\{}\FunctionTok{a}\OperatorTok{,} \FunctionTok{b}\OperatorTok{,} \FunctionTok{c}\OperatorTok{\},} \OperatorTok{\{}\FunctionTok{d}\OperatorTok{,} \FunctionTok{e}\OperatorTok{\},} \FunctionTok{z}\OperatorTok{],} \SpecialCharTok{\textbackslash{}}\OperatorTok{[}\NormalTok{Nu}\OperatorTok{]]}
\end{Highlighting}
\end{Shaded}

\begin{dmath*}\breakingcomma
\frac{\Gamma (d) \Gamma (e) z^{\nu } \Gamma (a+\nu ) \Gamma (b+\nu ) \Gamma (c+\nu )}{\Gamma (a) \Gamma (b) \Gamma (c) \Gamma (\nu +1) \Gamma (d+\nu ) \Gamma (e+\nu )}
\end{dmath*}
\end{document}
