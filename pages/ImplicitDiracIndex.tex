% !TeX program = pdflatex
% !TeX root = ImplicitDiracIndex.tex

\documentclass[../FeynCalcManual.tex]{subfiles}
\begin{document}
\hypertarget{implicitdiracindex}{
\section{ImplicitDiracIndex}\label{implicitdiracindex}\index{ImplicitDiracIndex}}

\texttt{ImplicitDiracIndex} is a data type. It mainly applies to names
of quantum fields specifying that the corresponding field carries an
implicit Dirac index.

This information can be supplied e.g.~via
\texttt{DataType[\allowbreak{}QuarkField,\ \allowbreak{}ImplicitDiracIndex] = True},
where \texttt{QuarkField} is a possible name of the relevant field.

The \texttt{ImplicitDiracIndex} property becomes relevant when
simplifying noncommutative products involving \texttt{QuantumField}s via
\texttt{ExpandPartialD}, \texttt{DotSimplify}.

\subsection{See also}

\hyperlink{toc}{Overview}, \hyperlink{datatype}{DataType},
\hyperlink{implicitsunfindex}{ImplicitSUNFIndex},
\hyperlink{implicitpauliindex}{ImplicitPauliIndex}

\subsection{Examples}

Default (possibly unwanted) behavior

\begin{Shaded}
\begin{Highlighting}[]
\NormalTok{ex }\ExtensionTok{=}\NormalTok{ QuantumField}\OperatorTok{[}\NormalTok{AntiQuarkField}\OperatorTok{]}\NormalTok{ . GA}\OperatorTok{[}\SpecialCharTok{\textbackslash{}}\OperatorTok{[}\NormalTok{Mu}\OperatorTok{]]}\NormalTok{ . QuantumField}\OperatorTok{[}\NormalTok{QuarkField}\OperatorTok{]}
\end{Highlighting}
\end{Shaded}

\begin{dmath*}\breakingcomma
\bar{\psi }.\bar{\gamma }^{\mu }.\psi
\end{dmath*}

\begin{Shaded}
\begin{Highlighting}[]
\NormalTok{ExpandPartialD}\OperatorTok{[}\NormalTok{ex}\OperatorTok{]}
\end{Highlighting}
\end{Shaded}

\begin{dmath*}\breakingcomma
\bar{\gamma }^{\mu }.\bar{\psi }.\psi
\end{dmath*}

Now we let FeynCalc know that \texttt{AntiQuarkField} and
\texttt{QuarkField} carry an implicit Dirac index that connects them to
the Dirac matrix.

\begin{Shaded}
\begin{Highlighting}[]
\NormalTok{DataType}\OperatorTok{[}\NormalTok{QuarkField}\OperatorTok{,}\NormalTok{ ImplicitDiracIndex}\OperatorTok{]} \ExtensionTok{=} \ConstantTok{True}\NormalTok{;}
\NormalTok{DataType}\OperatorTok{[}\NormalTok{AntiQuarkField}\OperatorTok{,}\NormalTok{ ImplicitDiracIndex}\OperatorTok{]} \ExtensionTok{=} \ConstantTok{True}\NormalTok{;}
\end{Highlighting}
\end{Shaded}

\begin{Shaded}
\begin{Highlighting}[]
\NormalTok{ExpandPartialD}\OperatorTok{[}\NormalTok{ex}\OperatorTok{]}
\end{Highlighting}
\end{Shaded}

\begin{dmath*}\breakingcomma
\bar{\psi }.\bar{\gamma }^{\mu }.\psi
\end{dmath*}
\end{document}
