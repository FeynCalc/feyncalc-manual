% !TeX program = pdflatex
% !TeX root = ImplicitPauliIndex.tex

\documentclass[../FeynCalcManual.tex]{subfiles}
\begin{document}
\hypertarget{implicitpauliindex}{
\section{ImplicitPauliIndex}\label{implicitpauliindex}\index{ImplicitPauliIndex}}

\texttt{ImplicitPauliIndex} is a data type. It mainly applies to names
of quantum fields specifying that the corresponding field carries an
implicit Pauli index.

This information can be supplied e.g.~via
\texttt{DataType[\allowbreak{}QuarkFieldChi,\ \allowbreak{}ImplicitPauliIndex] = True},
where \texttt{QuarkFieldChi} is a possible name of the relevant field.

The \texttt{ImplicitDiracIndex} property becomes relevant when
simplifying noncommutative products involving \texttt{QuantumField}s via
\texttt{ExpandPartialD}, \texttt{DotSimplify}.

\subsection{See also}

\hyperlink{toc}{Overview}, \hyperlink{datatype}{DataType},
\hyperlink{implicitsunfindex}{ImplicitSUNFIndex},
\hyperlink{implicitdiracindex}{ImplicitDiracIndex}

\subsection{Examples}

Default (possibly unwanted) behavior

\begin{Shaded}
\begin{Highlighting}[]
\NormalTok{ex }\ExtensionTok{=}\NormalTok{ QuantumField}\OperatorTok{[}\NormalTok{QuarkFieldChiDagger}\OperatorTok{]}\NormalTok{ . CSI}\OperatorTok{[}\FunctionTok{i}\OperatorTok{]}\NormalTok{ . QuantumField}\OperatorTok{[}\NormalTok{QuarkFieldChi}\OperatorTok{]}
\end{Highlighting}
\end{Shaded}

\begin{dmath*}\breakingcomma
\chi ^{\dagger }.\overline{\sigma }^i.\chi
\end{dmath*}

\begin{Shaded}
\begin{Highlighting}[]
\NormalTok{ExpandPartialD}\OperatorTok{[}\NormalTok{ex}\OperatorTok{]}
\end{Highlighting}
\end{Shaded}

\begin{dmath*}\breakingcomma
\overline{\sigma }^i.\chi ^{\dagger }.\chi
\end{dmath*}

Now we let FeynCalc know that \texttt{QuarkFieldChiDagger} and
\texttt{QuarkFieldChi} carry an implicit Pauli index that connects them
to the Pauli matrix.

\begin{Shaded}
\begin{Highlighting}[]
\NormalTok{DataType}\OperatorTok{[}\NormalTok{QuarkFieldChi}\OperatorTok{,}\NormalTok{ ImplicitPauliIndex}\OperatorTok{]} \ExtensionTok{=} \ConstantTok{True}\NormalTok{;}
\NormalTok{DataType}\OperatorTok{[}\NormalTok{QuarkFieldChiDagger}\OperatorTok{,}\NormalTok{ ImplicitPauliIndex}\OperatorTok{]} \ExtensionTok{=} \ConstantTok{True}\NormalTok{;}
\end{Highlighting}
\end{Shaded}

\begin{Shaded}
\begin{Highlighting}[]
\NormalTok{ExpandPartialD}\OperatorTok{[}\NormalTok{ex}\OperatorTok{]}
\end{Highlighting}
\end{Shaded}

\begin{dmath*}\breakingcomma
\chi ^{\dagger }.\overline{\sigma }^i.\chi
\end{dmath*}
\end{document}
