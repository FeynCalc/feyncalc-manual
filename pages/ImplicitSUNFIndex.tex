% !TeX program = pdflatex
% !TeX root = ImplicitSUNFIndex.tex

\documentclass[../FeynCalcManual.tex]{subfiles}
\begin{document}
\hypertarget{implicitsunfindex}{
\section{ImplicitSUNFIndex}\label{implicitsunfindex}\index{ImplicitSUNFIndex}}

\texttt{ImplicitSUNFIndex} is a data type. It mainly applies to names of
quantum fields specifying that the corresponding field carries an
implicit \(SU(N)\) index in the fundamental representation.

This information can be supplied e.g.~via
\texttt{DataType[\allowbreak{}QuarkField,\ \allowbreak{}ImplicitSUNFIndex] = True},
where \texttt{QuarkField} is a possible name of the relevant field.

The \texttt{ImplicitSUNFIndex} property becomes relevant when
simplifying noncommutative products involving \texttt{QuantumField}s via
\texttt{ExpandPartialD}, \texttt{DotSimplify}.

\subsection{See also}

\hyperlink{toc}{Overview}, \hyperlink{datatype}{DataType},
\hyperlink{implicitdiracindex}{ImplicitDiracIndex},
\hyperlink{implicitpauliindex}{ImplicitPauliIndex}

\subsection{Examples}

Default (possibly unwanted) behavior

\begin{Shaded}
\begin{Highlighting}[]
\NormalTok{ex }\ExtensionTok{=}\NormalTok{ QuantumField}\OperatorTok{[}\NormalTok{AntiQuarkField}\OperatorTok{]}\NormalTok{ . SUNT}\OperatorTok{[}\FunctionTok{a}\OperatorTok{]}\NormalTok{ . QuantumField}\OperatorTok{[}\NormalTok{QuarkField}\OperatorTok{]}
\end{Highlighting}
\end{Shaded}

\begin{dmath*}\breakingcomma
\bar{\psi }.T^a.\psi
\end{dmath*}

\begin{Shaded}
\begin{Highlighting}[]
\NormalTok{DotSimplify}\OperatorTok{[}\NormalTok{ex}\OperatorTok{]}
\end{Highlighting}
\end{Shaded}

\begin{dmath*}\breakingcomma
\bar{\psi }.\psi  T^a
\end{dmath*}

\begin{Shaded}
\begin{Highlighting}[]
\NormalTok{ExpandPartialD}\OperatorTok{[}\NormalTok{ex}\OperatorTok{]}
\end{Highlighting}
\end{Shaded}

\begin{dmath*}\breakingcomma
\bar{\psi }.\psi  T^a
\end{dmath*}

Now we let FeynCalc know that \texttt{AntiQuarkField} and
\texttt{QuarkField} carry an implicit color index that connects them to
the color matrix.

\begin{Shaded}
\begin{Highlighting}[]
\NormalTok{DataType}\OperatorTok{[}\NormalTok{QuarkField}\OperatorTok{,}\NormalTok{ ImplicitSUNFIndex}\OperatorTok{]} \ExtensionTok{=} \ConstantTok{True}\NormalTok{;}
\NormalTok{DataType}\OperatorTok{[}\NormalTok{AntiQuarkField}\OperatorTok{,}\NormalTok{ ImplicitSUNFIndex}\OperatorTok{]} \ExtensionTok{=} \ConstantTok{True}\NormalTok{;}
\end{Highlighting}
\end{Shaded}

\begin{Shaded}
\begin{Highlighting}[]
\NormalTok{DotSimplify}\OperatorTok{[}\NormalTok{ex}\OperatorTok{]}
\end{Highlighting}
\end{Shaded}

\begin{dmath*}\breakingcomma
\bar{\psi }.T^a.\psi
\end{dmath*}

\begin{Shaded}
\begin{Highlighting}[]
\NormalTok{ExpandPartialD}\OperatorTok{[}\NormalTok{ex}\OperatorTok{]}
\end{Highlighting}
\end{Shaded}

\begin{dmath*}\breakingcomma
\bar{\psi }.T^a.\psi
\end{dmath*}
\end{document}
