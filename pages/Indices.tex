% !TeX program = pdflatex
% !TeX root = Indices.tex

\documentclass[../FeynCalcManual.tex]{subfiles}
\begin{document}
\hypertarget{handling indices}{
\section{Handling indices}\label{handling indices}\index{Handling indices}}

\subsection{See also}

\hyperlink{toc}{Overview}.

\subsection{Manipulations of tensorial
quantities}\label{manipulations-of-tensorial-quantities}

When you square an expression with dummy indices, you must rename them
first. People often do this by hand, e.g.~as in

\begin{Shaded}
\begin{Highlighting}[]
\NormalTok{ex1 }\ExtensionTok{=}\NormalTok{ (FV}\OperatorTok{[}\FunctionTok{p}\OperatorTok{,} \SpecialCharTok{\textbackslash{}}\OperatorTok{[}\NormalTok{Mu}\OperatorTok{]]} \SpecialCharTok{+}\NormalTok{ FV}\OperatorTok{[}\FunctionTok{q}\OperatorTok{,} \SpecialCharTok{\textbackslash{}}\OperatorTok{[}\NormalTok{Mu}\OperatorTok{]]}\NormalTok{) FV}\OperatorTok{[}\FunctionTok{r}\OperatorTok{,} \SpecialCharTok{\textbackslash{}}\OperatorTok{[}\NormalTok{Mu}\OperatorTok{]]}\NormalTok{ FV}\OperatorTok{[}\FunctionTok{r}\OperatorTok{,} \SpecialCharTok{\textbackslash{}}\OperatorTok{[}\NormalTok{Nu}\OperatorTok{]]}
\end{Highlighting}
\end{Shaded}

\begin{dmath*}\breakingcomma
\overline{r}^{\mu } \overline{r}^{\nu } \left(\overline{p}^{\mu }+\overline{q}^{\mu }\right)
\end{dmath*}

\begin{Shaded}
\begin{Highlighting}[]
\NormalTok{ex1 (ex1 }\OtherTok{/.} \SpecialCharTok{\textbackslash{}}\OperatorTok{[}\NormalTok{Mu}\OperatorTok{]} \OtherTok{{-}\textgreater{}} \SpecialCharTok{\textbackslash{}}\OperatorTok{[}\NormalTok{Rho}\OperatorTok{]}\NormalTok{)}
\NormalTok{Contract}\OperatorTok{[}\SpecialCharTok{\%}\OperatorTok{]}
\end{Highlighting}
\end{Shaded}

\begin{dmath*}\breakingcomma
\overline{r}^{\mu } \left(\overline{r}^{\nu }\right)^2 \overline{r}^{\rho } \left(\overline{p}^{\mu }+\overline{q}^{\mu }\right) \left(\overline{p}^{\rho }+\overline{q}^{\rho }\right)
\end{dmath*}

\begin{dmath*}\breakingcomma
\overline{r}^2 \left(\overline{p}\cdot \overline{r}+\overline{q}\cdot \overline{r}\right)^2
\end{dmath*}

However, FeynCalc offers a function for that

\begin{Shaded}
\begin{Highlighting}[]
\NormalTok{FCRenameDummyIndices}\OperatorTok{[}\NormalTok{ex1}\OperatorTok{]}
\end{Highlighting}
\end{Shaded}

\begin{dmath*}\breakingcomma
\overline{r}^{\nu } \overline{r}^{\text{\$AL}(\text{\$19})} \left(\overline{p}^{\text{\$AL}(\text{\$19})}+\overline{q}^{\text{\$AL}(\text{\$19})}\right)
\end{dmath*}

\begin{Shaded}
\begin{Highlighting}[]
\NormalTok{ex1 FCRenameDummyIndices}\OperatorTok{[}\NormalTok{ex1}\OperatorTok{]}
\NormalTok{Contract}\OperatorTok{[}\SpecialCharTok{\%}\OperatorTok{]}
\end{Highlighting}
\end{Shaded}

\begin{dmath*}\breakingcomma
\overline{r}^{\mu } \overline{r}^{\nu } \overline{r}^{\nu } \left(\overline{p}^{\mu }+\overline{q}^{\mu }\right) \overline{r}^{\text{\$AL}(\text{\$20})} \left(\overline{p}^{\text{\$AL}(\text{\$20})}+\overline{q}^{\text{\$AL}(\text{\$20})}\right)
\end{dmath*}

\begin{dmath*}\breakingcomma
\overline{r}^2 \left(\overline{p}\cdot \overline{r}+\overline{q}\cdot \overline{r}\right)^2
\end{dmath*}

Notice that \texttt{FCRenameDummyIndices} does not canonicalize the
indices

\begin{Shaded}
\begin{Highlighting}[]
\NormalTok{FV}\OperatorTok{[}\FunctionTok{p}\OperatorTok{,} \SpecialCharTok{\textbackslash{}}\OperatorTok{[}\NormalTok{Nu}\OperatorTok{]]}\NormalTok{ FV}\OperatorTok{[}\FunctionTok{q}\OperatorTok{,} \SpecialCharTok{\textbackslash{}}\OperatorTok{[}\NormalTok{Nu}\OperatorTok{]]} \SpecialCharTok{{-}}\NormalTok{ FV}\OperatorTok{[}\FunctionTok{p}\OperatorTok{,} \SpecialCharTok{\textbackslash{}}\OperatorTok{[}\NormalTok{Mu}\OperatorTok{]]}\NormalTok{ FV}\OperatorTok{[}\FunctionTok{q}\OperatorTok{,} \SpecialCharTok{\textbackslash{}}\OperatorTok{[}\NormalTok{Mu}\OperatorTok{]]}
\NormalTok{FCRenameDummyIndices}\OperatorTok{[}\SpecialCharTok{\%}\OperatorTok{]}
\end{Highlighting}
\end{Shaded}

\begin{dmath*}\breakingcomma
\overline{p}^{\nu } \overline{q}^{\nu }-\overline{p}^{\mu } \overline{q}^{\mu }
\end{dmath*}

\begin{dmath*}\breakingcomma
\overline{p}^{\text{\$AL}(\text{\$22})} \overline{q}^{\text{\$AL}(\text{\$22})}-\overline{p}^{\text{\$AL}(\text{\$21})} \overline{q}^{\text{\$AL}(\text{\$21})}
\end{dmath*}

There is a function for that too

\begin{Shaded}
\begin{Highlighting}[]
\NormalTok{FV}\OperatorTok{[}\FunctionTok{p}\OperatorTok{,} \SpecialCharTok{\textbackslash{}}\OperatorTok{[}\NormalTok{Nu}\OperatorTok{]]}\NormalTok{ FV}\OperatorTok{[}\FunctionTok{q}\OperatorTok{,} \SpecialCharTok{\textbackslash{}}\OperatorTok{[}\NormalTok{Nu}\OperatorTok{]]} \SpecialCharTok{{-}}\NormalTok{ FV}\OperatorTok{[}\FunctionTok{p}\OperatorTok{,} \SpecialCharTok{\textbackslash{}}\OperatorTok{[}\NormalTok{Mu}\OperatorTok{]]}\NormalTok{ FV}\OperatorTok{[}\FunctionTok{q}\OperatorTok{,} \SpecialCharTok{\textbackslash{}}\OperatorTok{[}\NormalTok{Mu}\OperatorTok{]]}
\NormalTok{FCCanonicalizeDummyIndices}\OperatorTok{[}\SpecialCharTok{\%}\OperatorTok{]}
\end{Highlighting}
\end{Shaded}

\begin{dmath*}\breakingcomma
\overline{p}^{\nu } \overline{q}^{\nu }-\overline{p}^{\mu } \overline{q}^{\mu }
\end{dmath*}

\begin{dmath*}\breakingcomma
0
\end{dmath*}

Often we also need to uncontract already contracted indices. This is
done by \texttt{Uncontract}. By default, it handles only contractions
with Dirac matrices and Levi-Civita tensors

\begin{Shaded}
\begin{Highlighting}[]
\NormalTok{LC}\OperatorTok{[][}\FunctionTok{p}\OperatorTok{,} \FunctionTok{q}\OperatorTok{,} \FunctionTok{r}\OperatorTok{,} \FunctionTok{s}\OperatorTok{]}
\NormalTok{Uncontract}\OperatorTok{[}\SpecialCharTok{\%}\OperatorTok{,} \FunctionTok{p}\OperatorTok{]}
\NormalTok{Uncontract}\OperatorTok{[}\SpecialCharTok{\%\%}\OperatorTok{,} \FunctionTok{p}\OperatorTok{,} \FunctionTok{q}\OperatorTok{]}
\end{Highlighting}
\end{Shaded}

\begin{dmath*}\breakingcomma
\bar{\epsilon }^{\overline{p}\overline{q}\overline{r}\overline{s}}
\end{dmath*}

\begin{dmath*}\breakingcomma
\overline{p}^{\text{\$AL}(\text{\$31})} \bar{\epsilon }^{\text{\$AL}(\text{\$31})\overline{q}\overline{r}\overline{s}}
\end{dmath*}

\begin{dmath*}\breakingcomma
\overline{p}^{\text{\$AL}(\text{\$33})} \overline{q}^{\text{\$AL}(\text{\$32})} \left(-\bar{\epsilon }^{\text{\$AL}(\text{\$32})\text{\$AL}(\text{\$33})\overline{r}\overline{s}}\right)
\end{dmath*}

\begin{Shaded}
\begin{Highlighting}[]
\NormalTok{SP}\OperatorTok{[}\FunctionTok{p}\OperatorTok{,} \FunctionTok{q}\OperatorTok{]}
\NormalTok{Uncontract}\OperatorTok{[}\SpecialCharTok{\%}\OperatorTok{,} \FunctionTok{p}\OperatorTok{]}
\end{Highlighting}
\end{Shaded}

\begin{dmath*}\breakingcomma
\overline{p}\cdot \overline{q}
\end{dmath*}

\begin{dmath*}\breakingcomma
\overline{p}\cdot \overline{q}
\end{dmath*}

To uncontract scalar products as well, use the option \texttt{Pair->All}

\begin{Shaded}
\begin{Highlighting}[]
\NormalTok{Uncontract}\OperatorTok{[}\SpecialCharTok{\%}\OperatorTok{,} \FunctionTok{p}\OperatorTok{,}\NormalTok{ Pair }\OtherTok{{-}\textgreater{}} \ConstantTok{All}\OperatorTok{]}
\end{Highlighting}
\end{Shaded}

\begin{dmath*}\breakingcomma
\overline{p}^{\text{\$AL}(\text{\$34})} \overline{q}^{\text{\$AL}(\text{\$34})}
\end{dmath*}

Sometimes one might want to define custom symbolic tensors that are not
specified in terms of the 4-vectors, metric tensors and Levi-Civitas.
This is possible in FeynCalc, but the handling of such objects is not as
good as that of the built-in quantities

\begin{Shaded}
\begin{Highlighting}[]
\NormalTok{DeclareFCTensor}\OperatorTok{[}\NormalTok{myTensor}\OperatorTok{]}\NormalTok{;}
\end{Highlighting}
\end{Shaded}

\begin{Shaded}
\begin{Highlighting}[]
\NormalTok{myTensor}\OperatorTok{[}\NormalTok{LorentzIndex}\OperatorTok{[}\SpecialCharTok{\textbackslash{}}\OperatorTok{[}\NormalTok{Mu}\OperatorTok{]],}\NormalTok{ LorentzIndex}\OperatorTok{[}\SpecialCharTok{\textbackslash{}}\OperatorTok{[}\NormalTok{Nu}\OperatorTok{]]]}\NormalTok{ FV}\OperatorTok{[}\FunctionTok{p}\OperatorTok{,} \SpecialCharTok{\textbackslash{}}\OperatorTok{[}\NormalTok{Nu}\OperatorTok{]]}\NormalTok{ FV}\OperatorTok{[}\FunctionTok{q}\OperatorTok{,} \SpecialCharTok{\textbackslash{}}\OperatorTok{[}\NormalTok{Mu}\OperatorTok{]]}
\NormalTok{ex }\ExtensionTok{=}\NormalTok{ Contract}\OperatorTok{[}\SpecialCharTok{\%}\OperatorTok{]}
\end{Highlighting}
\end{Shaded}

\begin{dmath*}\breakingcomma
\overline{p}^{\nu } \overline{q}^{\mu } \;\text{myTensor}(\mu ,\nu )
\end{dmath*}

\begin{dmath*}\breakingcomma
\text{myTensor}\left(\overline{q},\overline{p}\right)
\end{dmath*}

\begin{Shaded}
\begin{Highlighting}[]
\NormalTok{Uncontract}\OperatorTok{[}\NormalTok{ex}\OperatorTok{,} \FunctionTok{p}\OperatorTok{,} \FunctionTok{q}\OperatorTok{,}\NormalTok{ Pair }\OtherTok{{-}\textgreater{}} \ConstantTok{All}\OperatorTok{]}
\end{Highlighting}
\end{Shaded}

\begin{dmath*}\breakingcomma
\overline{p}^{\text{\$AL}(\text{\$36})} \overline{q}^{\text{\$AL}(\text{\$35})} \;\text{myTensor}(\text{\$AL}(\text{\$35}),\text{\$AL}(\text{\$36}))
\end{dmath*}

\begin{Shaded}
\begin{Highlighting}[]
\NormalTok{(myTensor}\OperatorTok{[}\NormalTok{LorentzIndex}\OperatorTok{[}\SpecialCharTok{\textbackslash{}}\OperatorTok{[}\NormalTok{Mu}\OperatorTok{]],}\NormalTok{ LorentzIndex}\OperatorTok{[}\SpecialCharTok{\textbackslash{}}\OperatorTok{[}\NormalTok{Nu}\OperatorTok{]]]}\NormalTok{ MT}\OperatorTok{[}\NormalTok{LorentzIndex}\OperatorTok{[}\SpecialCharTok{\textbackslash{}}\OperatorTok{[}\NormalTok{Mu}\OperatorTok{]],}\NormalTok{ LorentzIndex}\OperatorTok{[}\SpecialCharTok{\textbackslash{}}\OperatorTok{[}\NormalTok{Nu}\OperatorTok{]]]} \SpecialCharTok{+} 
\NormalTok{   myTensor}\OperatorTok{[}\NormalTok{LorentzIndex}\OperatorTok{[}\SpecialCharTok{\textbackslash{}}\OperatorTok{[}\NormalTok{Alpha}\OperatorTok{]],}\NormalTok{ LorentzIndex}\OperatorTok{[}\SpecialCharTok{\textbackslash{}}\OperatorTok{[}\FunctionTok{Beta}\OperatorTok{]]]}\NormalTok{ MT}\OperatorTok{[}\NormalTok{LorentzIndex}\OperatorTok{[}\SpecialCharTok{\textbackslash{}}\OperatorTok{[}\NormalTok{Alpha}\OperatorTok{]],}\NormalTok{ LorentzIndex}\OperatorTok{[}\SpecialCharTok{\textbackslash{}}\OperatorTok{[}\FunctionTok{Beta}\OperatorTok{]]]}\NormalTok{)}
\NormalTok{FCCanonicalizeDummyIndices}\OperatorTok{[}\SpecialCharTok{\%}\OperatorTok{,}\NormalTok{ LorentzIndexNames }\OtherTok{{-}\textgreater{}} \OperatorTok{\{}\NormalTok{i1}\OperatorTok{,}\NormalTok{ i2}\OperatorTok{\}]}
\end{Highlighting}
\end{Shaded}

\begin{dmath*}\breakingcomma
\bar{g}^{\alpha \beta } \;\text{myTensor}(\alpha ,\beta )+\bar{g}^{\mu \nu } \;\text{myTensor}(\mu ,\nu )
\end{dmath*}

\begin{dmath*}\breakingcomma
2 \bar{g}^{\text{i1}\;\text{i2}} \;\text{myTensor}(\text{i1},\text{i2})
\end{dmath*}

To extract the list of free or dummy indices present in the expression,
one can use \texttt{FCGetFreeIndices} and \texttt{FCGetDummyIndices}
respectively

\begin{Shaded}
\begin{Highlighting}[]
\NormalTok{FCI}\OperatorTok{[}\NormalTok{FV}\OperatorTok{[}\FunctionTok{p}\OperatorTok{,} \SpecialCharTok{\textbackslash{}}\OperatorTok{[}\NormalTok{Mu}\OperatorTok{]]}\NormalTok{ FV}\OperatorTok{[}\FunctionTok{q}\OperatorTok{,} \SpecialCharTok{\textbackslash{}}\OperatorTok{[}\NormalTok{Nu}\OperatorTok{]]]} 
\NormalTok{FCGetFreeIndices}\OperatorTok{[}\SpecialCharTok{\%}\OperatorTok{,} \OperatorTok{\{}\NormalTok{LorentzIndex}\OperatorTok{\}]}
\end{Highlighting}
\end{Shaded}

\begin{dmath*}\breakingcomma
\overline{p}^{\mu } \overline{q}^{\nu }
\end{dmath*}

\begin{dmath*}\breakingcomma
\{\mu ,\nu \}
\end{dmath*}

```mathematica FCI{[}FV{[}p, {[}Mu{]}{]} FV{[}q, {[}Mu{]}{]}{]}
FCGetDummyIndices{[}\%, \{LorentzIndex\}{]}

```mathematica

\begin{dmath*}\breakingcomma
\overline{p}^{\mu } \overline{q}^{\mu }
\end{dmath*}

\begin{dmath*}\breakingcomma
\{\mu \}
\end{dmath*}
\end{document}
