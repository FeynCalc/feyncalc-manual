% !TeX program = pdflatex
% !TeX root = Indices.tex

\documentclass[../FeynCalcManual.tex]{subfiles}
\begin{document}
\hypertarget{upper-and-lower-indices}{%
\section{Upper and lower indices}\label{upper-and-lower-indices}}

\subsection{See also}

\hyperlink{toc}{Overview}.

Here we list a set of rules that allows to reconstruct the positions of
indices (upstairs or downstairs) appearing in FeynCalc expressions

\begin{itemize}
\item
  Every expression must satisfy Einsteins's summation convention, both
  for Lorentz and Cartesian indices. Single terms containing more than
  two identical Lorentz or Cartesian indices are illegal and will lead
  to inconsistent results.
\item
  In a contraction of two Lorentz indices it is understood that one of
  them is upstairs and the other is downstairs.
\item
  In a contraction of two Cartesian indices, both indices are understood
  to be upper indices.
\item
  A free Lorentz or Cartesian index is always understood to be an upper
  index
\end{itemize}
\end{document}
