% !TeX program = pdflatex
% !TeX root = InsideDiracTrace.tex

\documentclass[../FeynCalcManual.tex]{subfiles}
\begin{document}
\hypertarget{insidediractrace}{
\section{InsideDiracTrace}\label{insidediractrace}\index{InsideDiracTrace}}

\texttt{InsideDiracTrace} is an option of \texttt{DiracSimplify} and
some other functions dealing with Dirac algebra. If set to
\texttt{True}, the function assumes to operate inside a Dirac trace,
i.e., products of an odd number of Dirac matrices are discarded. For
more details, see the documentation for \texttt{DiracSimplify}.

\subsection{See also}

\hyperlink{toc}{Overview}

\subsection{Examples}

\begin{Shaded}
\begin{Highlighting}[]
\NormalTok{DiracSimplify}\OperatorTok{[}\NormalTok{GA}\OperatorTok{[}\SpecialCharTok{\textbackslash{}}\OperatorTok{[}\NormalTok{Mu}\OperatorTok{],} \SpecialCharTok{\textbackslash{}}\OperatorTok{[}\NormalTok{Nu}\OperatorTok{],} \SpecialCharTok{\textbackslash{}}\OperatorTok{[}\NormalTok{Rho}\OperatorTok{]]]}
\end{Highlighting}
\end{Shaded}

\begin{dmath*}\breakingcomma
\bar{\gamma }^{\mu }.\bar{\gamma }^{\nu }.\bar{\gamma }^{\rho }
\end{dmath*}

A trace of an 3 Dirac matrices vanishes:

\begin{Shaded}
\begin{Highlighting}[]
\NormalTok{DiracSimplify}\OperatorTok{[}\NormalTok{GA}\OperatorTok{[}\SpecialCharTok{\textbackslash{}}\OperatorTok{[}\NormalTok{Mu}\OperatorTok{],} \SpecialCharTok{\textbackslash{}}\OperatorTok{[}\NormalTok{Nu}\OperatorTok{],} \SpecialCharTok{\textbackslash{}}\OperatorTok{[}\NormalTok{Rho}\OperatorTok{]],}\NormalTok{ InsideDiracTrace }\OtherTok{{-}\textgreater{}} \ConstantTok{True}\OperatorTok{]}
\end{Highlighting}
\end{Shaded}

\begin{dmath*}\breakingcomma
0
\end{dmath*}
\end{document}
