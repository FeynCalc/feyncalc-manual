% !TeX program = pdflatex
% !TeX root = Install.tex

\documentclass[../FeynCalcManual.tex]{subfiles}
\begin{document}
\hypertarget{installation}{
\section{Installation}\label{installation}\index{Installation}}

\subsection{See also}

\hyperlink{toc}{Overview}.

The installation of FeynCalc can be done either automatically using the
provided Mathematica script or manually by copying the code to specific
locations on your computer.

\subsection{Automatic installation}\label{automatic-installation}

\subsubsection{Stable version}\label{stable-version}

The stable version is the latest official release of FeynCalc. We don't
have fixed development cycles so that the stable version is released
just when it's ready. The code of the stable version is located in the
\href{https://github.com/FeynCalc/feyncalc/tree/hotfix-stable}{hotfix-stable}
branch of our main repository. Bugs that are discovered in the latest
stable version will be fixed in that branch. When you install FeynCalc
using the automatic installer you will automatically receive all the
current fixes. Note that the stable branch will not contain any new
features until the next stable release.

To install the stable version run the following instruction in a Kernel
or Notebook session of Mathematica.

\begin{Shaded}
\begin{Highlighting}[]
\FunctionTok{Import}\OperatorTok{[}\StringTok{"https://raw.githubusercontent.com/FeynCalc/feyncalc/master/install.m"}\OperatorTok{]}
\NormalTok{InstallFeynCalc}\OperatorTok{[]}
\end{Highlighting}
\end{Shaded}

\subsubsection{Development version}\label{development-version}

The development version is the content of the master branch in the
\href{https://github.com/FeynCalc/feyncalc}{Git repository} of FeynCalc.
It contains all the fixes and new features that were implemented since
the last stable release. In the development process we use a growing
number of
\href{https://github.com/FeynCalc/feyncalc/tree/master/Tests}{unit
tests} to ensure that the changes in the code do not break existing
behavior or introduce new bugs. \textbf{However, despite thorough
testing the development version may still contain bugs and incomplete or
untested features. You can greatly help FeynCalc developers by testing
the development version and reporting issues with it!}

To install the development version run the following instruction in a
Kernel or Notebook session of Mathematica

\begin{Shaded}
\begin{Highlighting}[]
\FunctionTok{Import}\OperatorTok{[}\StringTok{"https://raw.githubusercontent.com/FeynCalc/feyncalc/master/install.m"}\OperatorTok{]}
\NormalTok{InstallFeynCalc}\OperatorTok{[}\NormalTok{InstallFeynCalcDevelopmentVersion }\OtherTok{{-}\textgreater{}} \ConstantTok{True}\OperatorTok{]}
\end{Highlighting}
\end{Shaded}

\subsubsection{Troubleshooting}\label{troubleshooting}

On Linux (possibly also Windows and macOS) the above code might fail
when run with Mathematica 10 or 11. The error messages will look like
\texttt{URLSave::invhttp: SSL connect error} or
\texttt{URLSave::invhttp}. This is most likely caused by some library
incompatibilities. A workaround for versions 10 and 11 is available
\href{https://mathematica.stackexchange.com/questions/212453/urlsave-in-mathematica-10-and-11-on-linux}{here}.
If nothing helps, you can still download the necessary files by yourself
and run the automatic installer offline.

\subsection{Manual installation}\label{manual-installation}

Manual installation is also possible, but is slightly less convenient as
compared to using the automatic installer.

\begin{itemize}
\item
  Download
  \href{https://github.com/FeynCalc/feyncalc/archive/hotfix-stable.zip}{this}
  (for the stable version) or
  \href{https://github.com/FeynCalc/feyncalc/archive/master.zip}{this}
  (for the development version) zip file.
\item
  Copy the \emph{FeynCalc} directory from the extracted archive to the
  \emph{Applications} directory inside \texttt{\$UserBaseDirectory}
  (evaluate
  \texttt{FileNameJoin[\allowbreak{}\{\allowbreak{}\$UserBaseDirectory,\ \allowbreak{}"Applications"\}]}
  in Mathematica).
\item
  If you want to allow \texttt{FeynCalc} to activate
  \texttt{TraditionalForm} typesetting when it is loaded, create
  ``FCConfig.m'' inside \emph{FeynCalc} directory and add there the
  following line

\begin{Shaded}
\begin{Highlighting}[]
\NormalTok{$FCTraditionalFormOutput}\ExtensionTok{=}\ConstantTok{True}\NormalTok{;}
\end{Highlighting}
\end{Shaded}

  this change will affect only the current \texttt{FeynCalc} session and
  will not modify the default behavior of Mathematica.
\end{itemize}

\subsection{Offline automatic
installation}\label{offline-automatic-installation}

\begin{itemize}
\item
  Download the following 3 files:

  \begin{itemize}
  \tightlist
  \item
    \href{https://github.com/FeynCalc/feyncalc/archive/master.zip}{master.zip}
    (FeynCalc)
  \item
    \href{https://github.com/FeynCalc/feynarts-mirror/archive/master.zip}{master.zip}
    (FeynArts)
  \item
    \href{https://github.com/FeynCalc/feyncalc/raw/master/install.m}{install.m}
  \end{itemize}
\item
  Put these files to the same folder. In the following I assume that it
  is \texttt{"/home/vs/Downloads"} which will be of course different on
  your system. Then open Mathematica and run
\end{itemize}

\begin{Shaded}
\begin{Highlighting}[]
\CommentTok{(*Change myPath accordingly! *)}
\NormalTok{myPath }\ExtensionTok{=} \StringTok{"/home/vs/Downloads"}\NormalTok{;}
\FunctionTok{Get}\OperatorTok{[}\FunctionTok{FileNameJoin}\OperatorTok{[\{}\NormalTok{myPath}\OperatorTok{,}\StringTok{"install.m"}\OperatorTok{\}]]}
\NormalTok{$PathToFCArc }\ExtensionTok{=} \FunctionTok{FileNameJoin}\OperatorTok{[\{}\NormalTok{myPath}\OperatorTok{,}\StringTok{"feyncalc{-}master.zip"}\OperatorTok{\}]}\NormalTok{;}
\NormalTok{$PathToFAArc }\ExtensionTok{=} \FunctionTok{FileNameJoin}\OperatorTok{[\{}\NormalTok{myPath}\OperatorTok{,}\StringTok{"feynarts{-}mirror{-}master.zip"}\OperatorTok{\}]}\NormalTok{;}
\NormalTok{InstallFeynCalc}\OperatorTok{[}\NormalTok{InstallFeynCalcDevelopmentVersion }\OtherTok{{-}\textgreater{}} \ConstantTok{True}\OperatorTok{]}
\end{Highlighting}
\end{Shaded}

\end{document}
