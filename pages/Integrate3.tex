% !TeX program = pdflatex
% !TeX root = Integrate3.tex

\documentclass[../FeynCalcManual.tex]{subfiles}
\begin{document}
\hypertarget{integrate3}{
\section{Integrate3}\label{integrate3}\index{Integrate3}}

\texttt{Integrate3} contains the integral table used by
\texttt{Integrate2}. Integration is performed in a distributional sense.
\texttt{Integrate3} works more effectively on a sum of expressions if
they are expanded or collected with respect to the integration variable.
See the examples in \texttt{Integrate2}.

\subsection{See also}

\hyperlink{toc}{Overview}, \hyperlink{integrate2}{Integrate2}.

\subsection{Examples}

\begin{Shaded}
\begin{Highlighting}[]
\NormalTok{Integrate3}\OperatorTok{[}\FunctionTok{x}\SpecialCharTok{\^{}}\NormalTok{OPEm }\FunctionTok{Log}\OperatorTok{[}\FunctionTok{x}\OperatorTok{],} \OperatorTok{\{}\FunctionTok{x}\OperatorTok{,} \DecValTok{0}\OperatorTok{,} \DecValTok{1}\OperatorTok{\}]}
\end{Highlighting}
\end{Shaded}

\begin{dmath*}\breakingcomma
-\frac{1}{(m+1)^2}
\end{dmath*}

\begin{Shaded}
\begin{Highlighting}[]
\NormalTok{Integrate3}\OperatorTok{[}\NormalTok{(}\FunctionTok{x}\SpecialCharTok{\^{}}\NormalTok{OPEm }\FunctionTok{Log}\OperatorTok{[}\FunctionTok{x}\OperatorTok{]} \FunctionTok{Log}\OperatorTok{[}\DecValTok{1} \SpecialCharTok{{-}} \FunctionTok{x}\OperatorTok{]}\NormalTok{)}\SpecialCharTok{/}\NormalTok{(}\DecValTok{1} \SpecialCharTok{{-}} \FunctionTok{x}\NormalTok{)}\OperatorTok{,} \OperatorTok{\{}\FunctionTok{x}\OperatorTok{,} \DecValTok{0}\OperatorTok{,} \DecValTok{1}\OperatorTok{\}]}
\end{Highlighting}
\end{Shaded}

\begin{dmath*}\breakingcomma
\zeta (2) S_1(m)-S_{12}(m)-S_{21}(m)+\zeta (3)
\end{dmath*}

\begin{Shaded}
\begin{Highlighting}[]
\NormalTok{Integrate3}\OperatorTok{[}\FunctionTok{a}\NormalTok{ (}\FunctionTok{x}\SpecialCharTok{\^{}}\NormalTok{OPEm }\FunctionTok{Log}\OperatorTok{[}\FunctionTok{x}\OperatorTok{]} \FunctionTok{Log}\OperatorTok{[}\DecValTok{1} \SpecialCharTok{{-}} \FunctionTok{x}\OperatorTok{]}\NormalTok{)}\SpecialCharTok{/}\NormalTok{(}\DecValTok{1} \SpecialCharTok{{-}} \FunctionTok{x}\NormalTok{) }\SpecialCharTok{+} \FunctionTok{b}\NormalTok{ (}\FunctionTok{x}\SpecialCharTok{\^{}}\NormalTok{OPEm }\FunctionTok{PolyLog}\OperatorTok{[}\DecValTok{3}\OperatorTok{,} \SpecialCharTok{{-}}\FunctionTok{x}\OperatorTok{]}\NormalTok{)}\SpecialCharTok{/}\NormalTok{(}\DecValTok{1} \SpecialCharTok{+} \FunctionTok{x}\NormalTok{)}\OperatorTok{,} \OperatorTok{\{}\FunctionTok{x}\OperatorTok{,} \DecValTok{0}\OperatorTok{,} \DecValTok{1}\OperatorTok{\}]}
\end{Highlighting}
\end{Shaded}

\begin{dmath*}\breakingcomma
a \left(\zeta (2) S_1(m)-S_{12}(m)-S_{21}(m)+\zeta (3)\right)+b (-1)^m \left(\frac{\zeta (2)^2}{8}+\frac{1}{2} \zeta (2) S_{-2}(m)-\frac{3}{4} \zeta (3) S_{-1}(m)+S_{3-1}(m)+\log (2) \left(S_3(m)-S_{-3}(m)\right)-\frac{3}{4} \zeta (3) \log (2)\right)
\end{dmath*}

\begin{Shaded}
\begin{Highlighting}[]
\NormalTok{Integrate3}\OperatorTok{[}\NormalTok{DeltaFunctionPrime}\OperatorTok{[}\DecValTok{1} \SpecialCharTok{{-}} \FunctionTok{x}\OperatorTok{],} \OperatorTok{\{}\FunctionTok{x}\OperatorTok{,} \DecValTok{0}\OperatorTok{,} \DecValTok{1}\OperatorTok{\}]}
\end{Highlighting}
\end{Shaded}

\begin{dmath*}\breakingcomma
0
\end{dmath*}

\begin{Shaded}
\begin{Highlighting}[]
\NormalTok{Integrate3}\OperatorTok{[}\FunctionTok{f}\OperatorTok{[}\FunctionTok{x}\OperatorTok{]}\NormalTok{ DeltaFunctionPrime}\OperatorTok{[}\DecValTok{1} \SpecialCharTok{{-}} \FunctionTok{x}\OperatorTok{],} \OperatorTok{\{}\FunctionTok{x}\OperatorTok{,} \DecValTok{0}\OperatorTok{,} \DecValTok{1}\OperatorTok{\}]}
\end{Highlighting}
\end{Shaded}

\begin{dmath*}\breakingcomma
f'(1)
\end{dmath*}

\begin{Shaded}
\begin{Highlighting}[]
\NormalTok{Integrate3}\OperatorTok{[}\DecValTok{1}\SpecialCharTok{/}\NormalTok{(}\DecValTok{1} \SpecialCharTok{{-}} \FunctionTok{x}\NormalTok{)}\OperatorTok{,} \OperatorTok{\{}\FunctionTok{x}\OperatorTok{,} \DecValTok{0}\OperatorTok{,} \DecValTok{1}\OperatorTok{\}]}
\end{Highlighting}
\end{Shaded}

\begin{dmath*}\breakingcomma
0
\end{dmath*}
\end{document}
