% !TeX program = pdflatex
% !TeX root = InternalExternal.tex

\documentclass[../FeynCalcManual.tex]{subfiles}
\begin{document}
\hypertarget{internal vs. external representations}{
\section{Internal vs. External Representations}\label{internal vs. external representations}\index{Internal vs. External Representations}}

\subsection{See also}

\hyperlink{toc}{Overview}.

\hypertarget{internal-representation}{%
\subsection{Internal representation}\label{internal-representation}}

The internal representation (\texttt{FeynCalcIntenral} or \texttt{FCI})
is how FeynCalc internally ``sees'' the objects. For example, a
\(4\)-dimensional \(4\)-vector is represented by

\begin{Shaded}
\begin{Highlighting}[]
\NormalTok{Pair}\OperatorTok{[}\NormalTok{LorentzIndex}\OperatorTok{[}\SpecialCharTok{\textbackslash{}}\OperatorTok{[}\NormalTok{Mu}\OperatorTok{]],}\NormalTok{ Momentum}\OperatorTok{[}\FunctionTok{p}\OperatorTok{]]}
\end{Highlighting}
\end{Shaded}

\begin{dmath*}\breakingcomma
\overline{p}^{\mu }
\end{dmath*}

Pair is one of the most basic FeynCalc objects. Depending on its
arguments, it can represent a \(4\)-vector, a metric tensor

\begin{Shaded}
\begin{Highlighting}[]
\NormalTok{Pair}\OperatorTok{[}\NormalTok{LorentzIndex}\OperatorTok{[}\SpecialCharTok{\textbackslash{}}\OperatorTok{[}\NormalTok{Mu}\OperatorTok{]],}\NormalTok{ LorentzIndex}\OperatorTok{[}\SpecialCharTok{\textbackslash{}}\OperatorTok{[}\NormalTok{Nu}\OperatorTok{]]]}
\end{Highlighting}
\end{Shaded}

\begin{dmath*}\breakingcomma
\bar{g}^{\mu \nu }
\end{dmath*}

or a scalar product of two 4-vectors

\begin{Shaded}
\begin{Highlighting}[]
\NormalTok{Pair}\OperatorTok{[}\NormalTok{Momentum}\OperatorTok{[}\FunctionTok{p}\OperatorTok{],}\NormalTok{ Momentum}\OperatorTok{[}\FunctionTok{q}\OperatorTok{]]}
\end{Highlighting}
\end{Shaded}

\begin{dmath*}\breakingcomma
\overline{p}\cdot \overline{q}
\end{dmath*}

Another essential object is \texttt{DiracGamma} that is used to
represent Dirac matrices. An uncontracted Dirac matrix is

\begin{Shaded}
\begin{Highlighting}[]
\NormalTok{DiracGamma}\OperatorTok{[}\NormalTok{LorentzIndex}\OperatorTok{[}\SpecialCharTok{\textbackslash{}}\OperatorTok{[}\NormalTok{Mu}\OperatorTok{]]]}
\end{Highlighting}
\end{Shaded}

\begin{dmath*}\breakingcomma
\bar{\gamma }^{\mu }
\end{dmath*}

and for a Feynman slash we use

\begin{Shaded}
\begin{Highlighting}[]
\NormalTok{DiracGamma}\OperatorTok{[}\NormalTok{Momentum}\OperatorTok{[}\FunctionTok{p}\OperatorTok{]]}
\end{Highlighting}
\end{Shaded}

\begin{dmath*}\breakingcomma
\bar{\gamma }\cdot \overline{p}
\end{dmath*}

The Levi-Civita-Tensor is

\begin{Shaded}
\begin{Highlighting}[]
\NormalTok{Eps}\OperatorTok{[}\NormalTok{LorentzIndex}\OperatorTok{[}\SpecialCharTok{\textbackslash{}}\OperatorTok{[}\NormalTok{Mu}\OperatorTok{]],}\NormalTok{ LorentzIndex}\OperatorTok{[}\SpecialCharTok{\textbackslash{}}\OperatorTok{[}\NormalTok{Nu}\OperatorTok{]],}\NormalTok{ LorentzIndex}\OperatorTok{[}\SpecialCharTok{\textbackslash{}}\OperatorTok{[}\NormalTok{Rho}\OperatorTok{]],}\NormalTok{ LorentzIndex}\OperatorTok{[}\SpecialCharTok{\textbackslash{}}\OperatorTok{[}\NormalTok{Sigma}\OperatorTok{]]]}
\end{Highlighting}
\end{Shaded}

\begin{dmath*}\breakingcomma
\bar{\epsilon }^{\mu \nu \rho \sigma }
\end{dmath*}

or, when contracted with \(4\)-momenta

\begin{Shaded}
\begin{Highlighting}[]
\NormalTok{Eps}\OperatorTok{[}\NormalTok{Momentum}\OperatorTok{[}\NormalTok{p1}\OperatorTok{],}\NormalTok{ Momentum}\OperatorTok{[}\NormalTok{p2}\OperatorTok{],}\NormalTok{ Momentum}\OperatorTok{[}\NormalTok{q1}\OperatorTok{],}\NormalTok{ Momentum}\OperatorTok{[}\NormalTok{q2}\OperatorTok{]]}
\end{Highlighting}
\end{Shaded}

\begin{dmath*}\breakingcomma
\bar{\epsilon }^{\overline{\text{p1}}\;\overline{\text{p2}}\;\overline{\text{q1}}\;\overline{\text{q2}}}
\end{dmath*}

This notation (momenta in the index slots) is also used in many other
tools (e.g.~FORM). The advantage is, that we do not need to canonicalize
the indices of the Levi-Civita-Tensor, e.g.~to ensure that

\begin{Shaded}
\begin{Highlighting}[]
\NormalTok{diff }\ExtensionTok{=}\NormalTok{ Eps}\OperatorTok{[}\NormalTok{LorentzIndex}\OperatorTok{[}\SpecialCharTok{\textbackslash{}}\OperatorTok{[}\NormalTok{Mu}\OperatorTok{]],}\NormalTok{ Momentum}\OperatorTok{[}\NormalTok{p2}\OperatorTok{],}\NormalTok{ Momentum}\OperatorTok{[}\NormalTok{q1}\OperatorTok{],}\NormalTok{ Momentum}\OperatorTok{[}\NormalTok{q2}\OperatorTok{]]}\NormalTok{ Pair}\OperatorTok{[}\NormalTok{LorentzIndex}\OperatorTok{[}\SpecialCharTok{\textbackslash{}}\OperatorTok{[}\NormalTok{Mu}\OperatorTok{]],}\NormalTok{ Momentum}\OperatorTok{[}\NormalTok{p1}\OperatorTok{]]} \SpecialCharTok{{-}} 
\NormalTok{   Eps}\OperatorTok{[}\NormalTok{LorentzIndex}\OperatorTok{[}\SpecialCharTok{\textbackslash{}}\OperatorTok{[}\NormalTok{Nu}\OperatorTok{]],}\NormalTok{ Momentum}\OperatorTok{[}\NormalTok{p2}\OperatorTok{],}\NormalTok{ Momentum}\OperatorTok{[}\NormalTok{q1}\OperatorTok{],}\NormalTok{ Momentum}\OperatorTok{[}\NormalTok{q2}\OperatorTok{]]}\NormalTok{ Pair}\OperatorTok{[}\NormalTok{LorentzIndex}\OperatorTok{[}\SpecialCharTok{\textbackslash{}}\OperatorTok{[}\NormalTok{Nu}\OperatorTok{]],}\NormalTok{ Momentum}\OperatorTok{[}\NormalTok{p1}\OperatorTok{]]}
\end{Highlighting}
\end{Shaded}

\begin{dmath*}\breakingcomma
\overline{\text{p1}}^{\mu } \bar{\epsilon }^{\mu \overline{\text{p2}}\;\overline{\text{q1}}\;\overline{\text{q2}}}-\overline{\text{p1}}^{\nu } \bar{\epsilon }^{\nu \overline{\text{p2}}\;\overline{\text{q1}}\;\overline{\text{q2}}}
\end{dmath*}

\begin{Shaded}
\begin{Highlighting}[]
\NormalTok{diff }\SpecialCharTok{//}\NormalTok{ Contract}
\end{Highlighting}
\end{Shaded}

\begin{dmath*}\breakingcomma
0
\end{dmath*}

is zero.

\hypertarget{external-representation}{%
\subsection{External representation}\label{external-representation}}

The internal representation is useful for the internal programming
FeynCalc, but obviously too cumbersome for the user input. This is why
FeynCalc also has an external representation (\texttt{FeynCalcExternal}
or \texttt{FCE}), that is concise and convenient.

Let us start with the \(4\)-vector. In the FCE-notation it is just
\texttt{FV} (``FourVector'')

\begin{Shaded}
\begin{Highlighting}[]
\NormalTok{FV}\OperatorTok{[}\FunctionTok{p}\OperatorTok{,} \SpecialCharTok{\textbackslash{}}\OperatorTok{[}\NormalTok{Mu}\OperatorTok{]]}
\end{Highlighting}
\end{Shaded}

\begin{dmath*}\breakingcomma
\overline{p}^{\mu }
\end{dmath*}

It is not hard to guess that the scalar product is \texttt{SP}

\begin{Shaded}
\begin{Highlighting}[]
\NormalTok{SP}\OperatorTok{[}\FunctionTok{p}\OperatorTok{,} \FunctionTok{q}\OperatorTok{]}
\end{Highlighting}
\end{Shaded}

\begin{dmath*}\breakingcomma
\overline{p}\cdot \overline{q}
\end{dmath*}

while for the metric tensor we write \texttt{MT}

\begin{Shaded}
\begin{Highlighting}[]
\NormalTok{MT}\OperatorTok{[}\SpecialCharTok{\textbackslash{}}\OperatorTok{[}\NormalTok{Mu}\OperatorTok{],} \SpecialCharTok{\textbackslash{}}\OperatorTok{[}\NormalTok{Nu}\OperatorTok{]]}
\end{Highlighting}
\end{Shaded}

\begin{dmath*}\breakingcomma
\bar{g}^{\mu \nu }
\end{dmath*}

To input a Dirac matrix or a Feynman slash, use \texttt{GA} or
\texttt{GS} respectively

\begin{Shaded}
\begin{Highlighting}[]
\NormalTok{GA}\OperatorTok{[}\SpecialCharTok{\textbackslash{}}\OperatorTok{[}\NormalTok{Mu}\OperatorTok{]]}
\end{Highlighting}
\end{Shaded}

\begin{dmath*}\breakingcomma
\bar{\gamma }^{\mu }
\end{dmath*}

\begin{Shaded}
\begin{Highlighting}[]
\NormalTok{GS}\OperatorTok{[}\FunctionTok{p}\OperatorTok{]}
\end{Highlighting}
\end{Shaded}

\begin{dmath*}\breakingcomma
\bar{\gamma }\cdot \overline{p}
\end{dmath*}

The Levi-Civita tensor is \texttt{LC}

\begin{Shaded}
\begin{Highlighting}[]
\NormalTok{LC}\OperatorTok{[}\SpecialCharTok{\textbackslash{}}\OperatorTok{[}\NormalTok{Mu}\OperatorTok{],} \SpecialCharTok{\textbackslash{}}\OperatorTok{[}\NormalTok{Nu}\OperatorTok{],} \SpecialCharTok{\textbackslash{}}\OperatorTok{[}\NormalTok{Rho}\OperatorTok{],} \SpecialCharTok{\textbackslash{}}\OperatorTok{[}\NormalTok{Sigma}\OperatorTok{]]}
\end{Highlighting}
\end{Shaded}

\begin{dmath*}\breakingcomma
\bar{\epsilon }^{\mu \nu \rho \sigma }
\end{dmath*}

The fully contracted form is entered via

\begin{Shaded}
\begin{Highlighting}[]
\NormalTok{LC}\OperatorTok{[][}\NormalTok{p1}\OperatorTok{,}\NormalTok{ p2}\OperatorTok{,}\NormalTok{ q1}\OperatorTok{,}\NormalTok{ q2}\OperatorTok{]}
\end{Highlighting}
\end{Shaded}

\begin{dmath*}\breakingcomma
\bar{\epsilon }^{\overline{\text{p1}}\;\overline{\text{p2}}\;\overline{\text{q1}}\;\overline{\text{q2}}}
\end{dmath*}

It is also possible to enter a mixed form

\begin{Shaded}
\begin{Highlighting}[]
\NormalTok{LC}\OperatorTok{[}\SpecialCharTok{\textbackslash{}}\OperatorTok{[}\NormalTok{Mu}\OperatorTok{]][}\NormalTok{p1}\OperatorTok{,}\NormalTok{ p2}\OperatorTok{,} \FunctionTok{q}\OperatorTok{]}
\end{Highlighting}
\end{Shaded}

\begin{dmath*}\breakingcomma
\bar{\epsilon }^{\mu \overline{\text{p1}}\;\overline{\text{p2}}\overline{q}}
\end{dmath*}

\begin{Shaded}
\begin{Highlighting}[]
\NormalTok{LC}\OperatorTok{[}\SpecialCharTok{\textbackslash{}}\OperatorTok{[}\NormalTok{Mu}\OperatorTok{],} \SpecialCharTok{\textbackslash{}}\OperatorTok{[}\NormalTok{Nu}\OperatorTok{]][}\NormalTok{p1}\OperatorTok{,}\NormalTok{ p2}\OperatorTok{]}
\end{Highlighting}
\end{Shaded}

\begin{dmath*}\breakingcomma
\bar{\epsilon }^{\mu \nu \overline{\text{p1}}\;\overline{\text{p2}}}
\end{dmath*}

\hypertarget{switching-between-the-representations}{%
\subsection{Switching between the
representations}\label{switching-between-the-representations}}

To convert between the two representations we use the functions
\texttt{FCI} and \texttt{FCE}, which are shortcuts for
\texttt{FeynCalcInternal} and \texttt{FeynCalcExternal}. One cannot
distinguish between the notations using the typesetting, i.e.~when we
see a typeset object in the \texttt{TraditionalForm}, we cannot really
tell if it is in the \texttt{FCI} or \texttt{FCE} notation.

\begin{Shaded}
\begin{Highlighting}[]
\NormalTok{ex1 }\ExtensionTok{=}\NormalTok{ FV}\OperatorTok{[}\FunctionTok{p}\OperatorTok{,} \SpecialCharTok{\textbackslash{}}\OperatorTok{[}\NormalTok{Mu}\OperatorTok{]]}
\NormalTok{ex2 }\ExtensionTok{=}\NormalTok{ Pair}\OperatorTok{[}\NormalTok{Momentum}\OperatorTok{[}\FunctionTok{p}\OperatorTok{],}\NormalTok{ LorentzIndex}\OperatorTok{[}\SpecialCharTok{\textbackslash{}}\OperatorTok{[}\NormalTok{Mu}\OperatorTok{]]]}
\end{Highlighting}
\end{Shaded}

\begin{dmath*}\breakingcomma
\overline{p}^{\mu }
\end{dmath*}

\begin{dmath*}\breakingcomma
\overline{p}^{\mu }
\end{dmath*}

However, we can always use StandardForm to see the difference

\begin{Shaded}
\begin{Highlighting}[]
\NormalTok{ex1 }\SpecialCharTok{//} \FunctionTok{StandardForm}
\NormalTok{ex2 }\SpecialCharTok{//} \FunctionTok{StandardForm}

\CommentTok{(*FV[p, \textbackslash{}[Mu]]*)}

\CommentTok{(*Pair[LorentzIndex[\textbackslash{}[Mu]], Momentum[p]]*)}
\end{Highlighting}
\end{Shaded}

\hypertarget{why-it-matters}{%
\subsection{Why it matters}\label{why-it-matters}}

All FeynCalc functions that are meant for users will automatically
convert the user input in the \texttt{FCE} notation into the
\texttt{FCI} notation. You do not have to do it by yourself.

On the other hand, virtually all FeynCalc functions produce their output
in the \texttt{FCI} form. So when you have an expression that was
obtained from FeynCalc and want to apply some replacement rules to it,
we have to use the \texttt{FCI} form in the rule

\begin{Shaded}
\begin{Highlighting}[]
\NormalTok{ex }\ExtensionTok{=}\NormalTok{ Pair}\OperatorTok{[}\NormalTok{Momentum}\OperatorTok{[}\FunctionTok{p}\OperatorTok{],}\NormalTok{ Momentum}\OperatorTok{[}\FunctionTok{q}\OperatorTok{]]}
\end{Highlighting}
\end{Shaded}

\begin{dmath*}\breakingcomma
\overline{p}\cdot \overline{q}
\end{dmath*}

No surprise that following does not work

\begin{Shaded}
\begin{Highlighting}[]
\NormalTok{ex }\OtherTok{/.}\NormalTok{ SP}\OperatorTok{[}\FunctionTok{p}\OperatorTok{,} \FunctionTok{q}\OperatorTok{]} \OtherTok{{-}\textgreater{}} \DecValTok{1}
\end{Highlighting}
\end{Shaded}

\begin{dmath*}\breakingcomma
\overline{p}\cdot \overline{q}
\end{dmath*}

But if we wrap the r.h.s of the rule with \texttt{FCI}, then everything
is fine

\begin{Shaded}
\begin{Highlighting}[]
\NormalTok{ex }\OtherTok{/.}\NormalTok{ FCI}\OperatorTok{[}\NormalTok{SP}\OperatorTok{[}\FunctionTok{p}\OperatorTok{,} \FunctionTok{q}\OperatorTok{]]} \OtherTok{{-}\textgreater{}} \DecValTok{1}
\end{Highlighting}
\end{Shaded}

\begin{dmath*}\breakingcomma
1
\end{dmath*}
\end{document}
