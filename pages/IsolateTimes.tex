% !TeX program = pdflatex
% !TeX root = IsolateTimes.tex

\documentclass[../FeynCalcManual.tex]{subfiles}
\begin{document}
\hypertarget{isolatetimes}{%
\section{IsolateTimes}\label{isolatetimes}}

\texttt{IsolateTimes} is an option for \texttt{Isolate} and other
functions using \texttt{Isolate}. If it is set to \texttt{True}, Isolate
will be applied also to pure products.

\subsection{See also}

\hyperlink{toc}{Overview}, \hyperlink{isolate}{Isolate},
\hyperlink{collect2}{Collect2}.

\subsection{Examples}

By default, this expression does not become abbreviated

\begin{Shaded}
\begin{Highlighting}[]
\NormalTok{Isolate}\OperatorTok{[}\FunctionTok{a}\SpecialCharTok{*}\FunctionTok{b}\SpecialCharTok{*}\FunctionTok{c}\SpecialCharTok{*}\FunctionTok{d}\OperatorTok{,} \FunctionTok{a}\OperatorTok{]}
\end{Highlighting}
\end{Shaded}

\begin{dmath*}\breakingcomma
a b c d
\end{dmath*}

Now an abbreviation is introduced

\begin{Shaded}
\begin{Highlighting}[]
\NormalTok{ Isolate}\OperatorTok{[}\FunctionTok{a}\SpecialCharTok{*}\FunctionTok{b}\SpecialCharTok{*}\FunctionTok{c}\SpecialCharTok{*}\FunctionTok{d}\OperatorTok{,} \FunctionTok{a}\OperatorTok{,}\NormalTok{ IsolateTimes }\OtherTok{{-}\textgreater{}} \ConstantTok{True}\OperatorTok{]}
\end{Highlighting}
\end{Shaded}

\begin{dmath*}\breakingcomma
a \;\text{KK}(19)
\end{dmath*}
\end{document}
