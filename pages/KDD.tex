% !TeX program = pdflatex
% !TeX root = KDD.tex

\documentclass[../FeynCalcManual.tex]{subfiles}
\begin{document}
\hypertarget{kdd}{
\section{KDD}\label{kdd}\index{KDD}}

\texttt{KDD[\allowbreak{}i,\ \allowbreak{}j]} is the Kronecker delta in
\(D-1\) dimensions.

\subsection{See also}

\hyperlink{toc}{Overview}, \hyperlink{cartesianpair}{CartesianPair},
\hyperlink{kd}{KD}.

\subsection{Examples}

\begin{Shaded}
\begin{Highlighting}[]
\NormalTok{KDD}\OperatorTok{[}\FunctionTok{i}\OperatorTok{,} \FunctionTok{j}\OperatorTok{]}
\end{Highlighting}
\end{Shaded}

\begin{dmath*}\breakingcomma
\delta ^{ij}
\end{dmath*}

\begin{Shaded}
\begin{Highlighting}[]
\NormalTok{Contract}\OperatorTok{[}\NormalTok{KDD}\OperatorTok{[}\FunctionTok{i}\OperatorTok{,} \FunctionTok{j}\OperatorTok{]}\NormalTok{ KDD}\OperatorTok{[}\FunctionTok{i}\OperatorTok{,} \FunctionTok{j}\OperatorTok{]]}
\end{Highlighting}
\end{Shaded}

\begin{dmath*}\breakingcomma
D-1
\end{dmath*}

\begin{Shaded}
\begin{Highlighting}[]
\NormalTok{KDD}\OperatorTok{[}\FunctionTok{a}\OperatorTok{,} \FunctionTok{b}\OperatorTok{]} \SpecialCharTok{//} \FunctionTok{StandardForm}

\CommentTok{(*KDD[a, b]*)}
\end{Highlighting}
\end{Shaded}

\begin{Shaded}
\begin{Highlighting}[]
\NormalTok{FCI}\OperatorTok{[}\NormalTok{KDD}\OperatorTok{[}\FunctionTok{a}\OperatorTok{,} \FunctionTok{b}\OperatorTok{]]} \SpecialCharTok{//} \FunctionTok{StandardForm}

\CommentTok{(*CartesianPair[CartesianIndex[a, {-}1 + D], CartesianIndex[b, {-}1 + D]]*)}
\end{Highlighting}
\end{Shaded}

\begin{Shaded}
\begin{Highlighting}[]
\NormalTok{FCE}\OperatorTok{[}\NormalTok{FCI}\OperatorTok{[}\NormalTok{KDD}\OperatorTok{[}\FunctionTok{a}\OperatorTok{,} \FunctionTok{b}\OperatorTok{]]]} \SpecialCharTok{//} \FunctionTok{StandardForm}

\CommentTok{(*KDD[a, b]*)}
\end{Highlighting}
\end{Shaded}

\end{document}
