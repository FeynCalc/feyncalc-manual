% !TeX program = pdflatex
% !TeX root = Kinematics.tex

\documentclass[../FeynCalcManual.tex]{subfiles}
\begin{document}
\hypertarget{kinematics}{
\section{Kinematics}\label{kinematics}\index{Kinematics}}

\subsection{See also}

\hyperlink{toc}{Overview}.

\subsection{Manipulations of scalar
products}\label{manipulations-of-scalar-products}

FeynCalc allows you to specify the values of scalar products before
doing the calculation.

\begin{Shaded}
\begin{Highlighting}[]
\NormalTok{SP}\OperatorTok{[}\FunctionTok{p}\OperatorTok{,} \FunctionTok{q}\OperatorTok{]} \ExtensionTok{=} \FunctionTok{s}\NormalTok{;}
\end{Highlighting}
\end{Shaded}

\begin{Shaded}
\begin{Highlighting}[]
\NormalTok{SP}\OperatorTok{[}\FunctionTok{p}\OperatorTok{,} \FunctionTok{q}\OperatorTok{]}
\end{Highlighting}
\end{Shaded}

\begin{dmath*}\breakingcomma
s
\end{dmath*}

To clear the previously set values, use

\begin{Shaded}
\begin{Highlighting}[]
\NormalTok{FCClearScalarProducts}\OperatorTok{[]}
\end{Highlighting}
\end{Shaded}

\begin{Shaded}
\begin{Highlighting}[]
\NormalTok{SP}\OperatorTok{[}\FunctionTok{p}\OperatorTok{,} \FunctionTok{q}\OperatorTok{]}
\end{Highlighting}
\end{Shaded}

\begin{dmath*}\breakingcomma
\overline{p}\cdot \overline{q}
\end{dmath*}

A good habit is to always apply
\texttt{FCClearScalarProducts[\allowbreak{}]} before setting the values,
like in

\begin{Shaded}
\begin{Highlighting}[]
\NormalTok{FCClearScalarProducts}\OperatorTok{[]}\NormalTok{;}
\NormalTok{SP}\OperatorTok{[}\NormalTok{p1}\OperatorTok{,}\NormalTok{ p1}\OperatorTok{]} \ExtensionTok{=}\NormalTok{ m1}\SpecialCharTok{\^{}}\DecValTok{2}\NormalTok{;}
\NormalTok{SP}\OperatorTok{[}\NormalTok{p2}\OperatorTok{,}\NormalTok{ p2}\OperatorTok{]} \ExtensionTok{=}\NormalTok{ m2}\SpecialCharTok{\^{}}\DecValTok{2}\NormalTok{;}
\end{Highlighting}
\end{Shaded}

Setting up the kinematics in advance improves performance of FeynCalc
and leads to more compact results. The results with the fully arbitrary
kinematics are the most complicated and the longest ones.
\end{document}
