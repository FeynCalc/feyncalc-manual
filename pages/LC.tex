% !TeX program = pdflatex
% !TeX root = LC.tex

\documentclass[../FeynCalcManual.tex]{subfiles}
\begin{document}
\hypertarget{lc}{
\section{LC}\label{lc}\index{LC}}

\texttt{LC[\allowbreak{}m,\ \allowbreak{}n,\ \allowbreak{}r,\ \allowbreak{}s]}
evaluates to 4-dimensional \(\varepsilon^{m n r s}\) by virtue of
applying \texttt{FeynCalcInternal}.

\texttt{LC[\allowbreak{}m,\ \allowbreak{}...][\allowbreak{}p,\ \allowbreak{}...]}
evaluates to 4-dimensional
\(\epsilon ^{m \ldots \mu \ldots}p_{\mu \ldots}\) applying
\texttt{FeynCalcInternal}.

When some indices of a Levi-Civita-tensor are contracted with 4-vectors,
FeynCalc suppresses explicit dummy indices by putting those vectors into
the corresponding index slots. For example,
\(\varepsilon^{p_1 p_2 p_3 p_4}\) (accessible via
\texttt{LC[\allowbreak{}][\allowbreak{}p1,\ \allowbreak{}p2,\ \allowbreak{}p3,\ \allowbreak{}p4]})
correspond to
\(\varepsilon_{\mu \nu \rho \sigma} p_1^\mu p_2^\nu p_3^\rho p_4^\sigma\).

\subsection{See also}

\hyperlink{toc}{Overview}, \hyperlink{eps}{Eps}, \hyperlink{lcd}{LCD}.

\subsection{Examples}

\begin{Shaded}
\begin{Highlighting}[]
\NormalTok{LC}\OperatorTok{[}\SpecialCharTok{\textbackslash{}}\OperatorTok{[}\NormalTok{Mu}\OperatorTok{],} \SpecialCharTok{\textbackslash{}}\OperatorTok{[}\NormalTok{Nu}\OperatorTok{],} \SpecialCharTok{\textbackslash{}}\OperatorTok{[}\NormalTok{Rho}\OperatorTok{],} \SpecialCharTok{\textbackslash{}}\OperatorTok{[}\NormalTok{Sigma}\OperatorTok{]]}
\end{Highlighting}
\end{Shaded}

\begin{dmath*}\breakingcomma
\bar{\epsilon }^{\mu \nu \rho \sigma }
\end{dmath*}

\begin{Shaded}
\begin{Highlighting}[]
\NormalTok{LC}\OperatorTok{[}\SpecialCharTok{\textbackslash{}}\OperatorTok{[}\NormalTok{Mu}\OperatorTok{],} \SpecialCharTok{\textbackslash{}}\OperatorTok{[}\NormalTok{Nu}\OperatorTok{],} \SpecialCharTok{\textbackslash{}}\OperatorTok{[}\NormalTok{Rho}\OperatorTok{],} \SpecialCharTok{\textbackslash{}}\OperatorTok{[}\NormalTok{Sigma}\OperatorTok{]]} \SpecialCharTok{//}\NormalTok{ FCI }\SpecialCharTok{//} \FunctionTok{StandardForm}

\CommentTok{(*Eps[LorentzIndex[\textbackslash{}[Mu]], LorentzIndex[\textbackslash{}[Nu]], LorentzIndex[\textbackslash{}[Rho]], LorentzIndex[\textbackslash{}[Sigma]]]*)}
\end{Highlighting}
\end{Shaded}

\begin{Shaded}
\begin{Highlighting}[]
\NormalTok{LC}\OperatorTok{[}\SpecialCharTok{\textbackslash{}}\OperatorTok{[}\NormalTok{Mu}\OperatorTok{],} \SpecialCharTok{\textbackslash{}}\OperatorTok{[}\NormalTok{Nu}\OperatorTok{]][}\FunctionTok{p}\OperatorTok{,} \FunctionTok{q}\OperatorTok{]}
\end{Highlighting}
\end{Shaded}

\begin{dmath*}\breakingcomma
\bar{\epsilon }^{\mu \nu \overline{p}\overline{q}}
\end{dmath*}

\begin{Shaded}
\begin{Highlighting}[]
\NormalTok{LC}\OperatorTok{[}\SpecialCharTok{\textbackslash{}}\OperatorTok{[}\NormalTok{Mu}\OperatorTok{],} \SpecialCharTok{\textbackslash{}}\OperatorTok{[}\NormalTok{Nu}\OperatorTok{]][}\FunctionTok{p}\OperatorTok{,} \FunctionTok{q}\OperatorTok{]} \SpecialCharTok{//}\NormalTok{ FCI }\SpecialCharTok{//} \FunctionTok{StandardForm}

\CommentTok{(*Eps[LorentzIndex[\textbackslash{}[Mu]], LorentzIndex[\textbackslash{}[Nu]], Momentum[p], Momentum[q]]*)}
\end{Highlighting}
\end{Shaded}

\begin{Shaded}
\begin{Highlighting}[]
\NormalTok{Contract}\OperatorTok{[}\NormalTok{LC}\OperatorTok{[}\SpecialCharTok{\textbackslash{}}\OperatorTok{[}\NormalTok{Mu}\OperatorTok{],} \SpecialCharTok{\textbackslash{}}\OperatorTok{[}\NormalTok{Nu}\OperatorTok{],} \SpecialCharTok{\textbackslash{}}\OperatorTok{[}\NormalTok{Rho}\OperatorTok{]][}\FunctionTok{p}\OperatorTok{]}\NormalTok{ LC}\OperatorTok{[}\SpecialCharTok{\textbackslash{}}\OperatorTok{[}\NormalTok{Mu}\OperatorTok{],} \SpecialCharTok{\textbackslash{}}\OperatorTok{[}\NormalTok{Nu}\OperatorTok{],} \SpecialCharTok{\textbackslash{}}\OperatorTok{[}\NormalTok{Rho}\OperatorTok{]][}\FunctionTok{q}\OperatorTok{]]} 
\end{Highlighting}
\end{Shaded}

\begin{dmath*}\breakingcomma
-6 \left(\overline{p}\cdot \overline{q}\right)
\end{dmath*}

\begin{Shaded}
\begin{Highlighting}[]
\NormalTok{LC}\OperatorTok{[}\SpecialCharTok{\textbackslash{}}\OperatorTok{[}\NormalTok{Mu}\OperatorTok{],} \SpecialCharTok{\textbackslash{}}\OperatorTok{[}\NormalTok{Nu}\OperatorTok{],} \SpecialCharTok{\textbackslash{}}\OperatorTok{[}\NormalTok{Rho}\OperatorTok{],} \SpecialCharTok{\textbackslash{}}\OperatorTok{[}\NormalTok{Sigma}\OperatorTok{]]}\NormalTok{ FV}\OperatorTok{[}\FunctionTok{Subscript}\OperatorTok{[}\FunctionTok{p}\OperatorTok{,} \DecValTok{1}\OperatorTok{],} \SpecialCharTok{\textbackslash{}}\OperatorTok{[}\NormalTok{Mu}\OperatorTok{]]}\NormalTok{ FV}\OperatorTok{[}\FunctionTok{Subscript}\OperatorTok{[}\FunctionTok{p}\OperatorTok{,} \DecValTok{2}\OperatorTok{],} \SpecialCharTok{\textbackslash{}}\OperatorTok{[}\NormalTok{Nu}\OperatorTok{]]}\NormalTok{ FV}\OperatorTok{[}\FunctionTok{Subscript}\OperatorTok{[}\FunctionTok{p}\OperatorTok{,} \DecValTok{3}\OperatorTok{],} \SpecialCharTok{\textbackslash{}}\OperatorTok{[}\NormalTok{Rho}\OperatorTok{]]}\NormalTok{ FV}\OperatorTok{[}\FunctionTok{Subscript}\OperatorTok{[}\FunctionTok{p}\OperatorTok{,} \DecValTok{4}\OperatorTok{],} \SpecialCharTok{\textbackslash{}}\OperatorTok{[}\NormalTok{Sigma}\OperatorTok{]]} 
 
\NormalTok{Contract}\OperatorTok{[}\SpecialCharTok{\%}\OperatorTok{]} 
  
 
\end{Highlighting}
\end{Shaded}

\begin{dmath*}\breakingcomma
\overline{p}_1{}^{\mu } \overline{p}_2{}^{\nu } \overline{p}_3{}^{\rho } \overline{p}_4{}^{\sigma } \bar{\epsilon }^{\mu \nu \rho \sigma }
\end{dmath*}

\begin{dmath*}\breakingcomma
\bar{\epsilon }^{\overline{p}_1\overline{p}_2\overline{p}_3\overline{p}_4}
\end{dmath*}
\end{document}
