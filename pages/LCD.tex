% !TeX program = pdflatex
% !TeX root = LCD.tex

\documentclass[../FeynCalcManual.tex]{subfiles}
\begin{document}
\begin{Shaded}
\begin{Highlighting}[]
 
\end{Highlighting}
\end{Shaded}

\hypertarget{lcd}{
\section{LCD}\label{lcd}\index{LCD}}

\texttt{LCD[\allowbreak{}m,\ \allowbreak{}n,\ \allowbreak{}r,\ \allowbreak{}s]}
evaluates to \(D\)-dimensional \(\varepsilon^{m n r s}\) by virtue of
applying \texttt{FeynCalcInternal}.

\texttt{LCD[\allowbreak{}m,\ \allowbreak{}...][\allowbreak{}p,\ \allowbreak{}...]}
evaluates to \(D\)-dimensional
\(\epsilon ^{m \ldots \mu \ldots}p_{\mu \ldots}\) applying
\texttt{FeynCalcInternal}.

When some indices of a Levi-Civita-tensor are contracted with 4-vectors,
FeynCalc suppresses explicit dummy indices by putting those vectors into
the corresponding index slots. For example,
\(\varepsilon^{p_1 p_2 p_3 p_4}\) (accessible via
\texttt{LCD[\allowbreak{}][\allowbreak{}p1,\ \allowbreak{}p2,\ \allowbreak{}p3,\ \allowbreak{}p4]})
correspond to
\(\varepsilon_{\mu \nu \rho \sigma} p_1^\mu p_2^\nu p_3^\rho p_4^\sigma\).

\subsection{See also}

\hyperlink{toc}{Overview}, \hyperlink{eps}{Eps}, \hyperlink{lc}{LC}.

\subsection{Examples}

\begin{Shaded}
\begin{Highlighting}[]
\NormalTok{LCD}\OperatorTok{[}\SpecialCharTok{\textbackslash{}}\OperatorTok{[}\NormalTok{Mu}\OperatorTok{],} \SpecialCharTok{\textbackslash{}}\OperatorTok{[}\NormalTok{Nu}\OperatorTok{],} \SpecialCharTok{\textbackslash{}}\OperatorTok{[}\NormalTok{Rho}\OperatorTok{],} \SpecialCharTok{\textbackslash{}}\OperatorTok{[}\NormalTok{Sigma}\OperatorTok{]]}
\end{Highlighting}
\end{Shaded}

\begin{dmath*}\breakingcomma
\overset{\text{}}{\epsilon }^{\mu \nu \rho \sigma }
\end{dmath*}

\begin{Shaded}
\begin{Highlighting}[]
\NormalTok{LCD}\OperatorTok{[}\SpecialCharTok{\textbackslash{}}\OperatorTok{[}\NormalTok{Mu}\OperatorTok{],} \SpecialCharTok{\textbackslash{}}\OperatorTok{[}\NormalTok{Nu}\OperatorTok{],} \SpecialCharTok{\textbackslash{}}\OperatorTok{[}\NormalTok{Rho}\OperatorTok{],} \SpecialCharTok{\textbackslash{}}\OperatorTok{[}\NormalTok{Sigma}\OperatorTok{]]} \SpecialCharTok{//}\NormalTok{ FCI }\SpecialCharTok{//} \FunctionTok{StandardForm}

\CommentTok{(*Eps[LorentzIndex[\textbackslash{}[Mu], D], LorentzIndex[\textbackslash{}[Nu], D], LorentzIndex[\textbackslash{}[Rho], D], LorentzIndex[\textbackslash{}[Sigma], D]]*)}
\end{Highlighting}
\end{Shaded}

\begin{Shaded}
\begin{Highlighting}[]
\NormalTok{LCD}\OperatorTok{[}\SpecialCharTok{\textbackslash{}}\OperatorTok{[}\NormalTok{Mu}\OperatorTok{],} \SpecialCharTok{\textbackslash{}}\OperatorTok{[}\NormalTok{Nu}\OperatorTok{]][}\FunctionTok{p}\OperatorTok{,} \FunctionTok{q}\OperatorTok{]}
\end{Highlighting}
\end{Shaded}

\begin{dmath*}\breakingcomma
\overset{\text{}}{\epsilon }^{\mu \nu pq}
\end{dmath*}

\begin{Shaded}
\begin{Highlighting}[]
\NormalTok{LCD}\OperatorTok{[}\SpecialCharTok{\textbackslash{}}\OperatorTok{[}\NormalTok{Mu}\OperatorTok{],} \SpecialCharTok{\textbackslash{}}\OperatorTok{[}\NormalTok{Nu}\OperatorTok{]][}\FunctionTok{p}\OperatorTok{,} \FunctionTok{q}\OperatorTok{]} \SpecialCharTok{//}\NormalTok{ FCI }\SpecialCharTok{//} \FunctionTok{StandardForm}

\CommentTok{(*Eps[LorentzIndex[\textbackslash{}[Mu], D], LorentzIndex[\textbackslash{}[Nu], D], Momentum[p, D], Momentum[q, D]]*)}
\end{Highlighting}
\end{Shaded}

\begin{Shaded}
\begin{Highlighting}[]
\NormalTok{Factor2}\OperatorTok{[}\NormalTok{Contract}\OperatorTok{[}\NormalTok{LCD}\OperatorTok{[}\SpecialCharTok{\textbackslash{}}\OperatorTok{[}\NormalTok{Mu}\OperatorTok{],} \SpecialCharTok{\textbackslash{}}\OperatorTok{[}\NormalTok{Nu}\OperatorTok{],} \SpecialCharTok{\textbackslash{}}\OperatorTok{[}\NormalTok{Rho}\OperatorTok{]][}\FunctionTok{p}\OperatorTok{]}\NormalTok{ LCD}\OperatorTok{[}\SpecialCharTok{\textbackslash{}}\OperatorTok{[}\NormalTok{Mu}\OperatorTok{],} \SpecialCharTok{\textbackslash{}}\OperatorTok{[}\NormalTok{Nu}\OperatorTok{],} \SpecialCharTok{\textbackslash{}}\OperatorTok{[}\NormalTok{Rho}\OperatorTok{]][}\FunctionTok{q}\OperatorTok{]]]}
\end{Highlighting}
\end{Shaded}

\begin{dmath*}\breakingcomma
(1-D) (2-D) (3-D) (p\cdot q)
\end{dmath*}

\begin{Shaded}
\begin{Highlighting}[]
\NormalTok{LCD}\OperatorTok{[}\SpecialCharTok{\textbackslash{}}\OperatorTok{[}\NormalTok{Mu}\OperatorTok{],} \SpecialCharTok{\textbackslash{}}\OperatorTok{[}\NormalTok{Nu}\OperatorTok{],} \SpecialCharTok{\textbackslash{}}\OperatorTok{[}\NormalTok{Rho}\OperatorTok{],} \SpecialCharTok{\textbackslash{}}\OperatorTok{[}\NormalTok{Sigma}\OperatorTok{]]}\NormalTok{ FVD}\OperatorTok{[}\FunctionTok{Subscript}\OperatorTok{[}\FunctionTok{p}\OperatorTok{,} \DecValTok{1}\OperatorTok{],} \SpecialCharTok{\textbackslash{}}\OperatorTok{[}\NormalTok{Mu}\OperatorTok{]]}\NormalTok{ FVD}\OperatorTok{[}\FunctionTok{Subscript}\OperatorTok{[}\FunctionTok{p}\OperatorTok{,} \DecValTok{2}\OperatorTok{],} \SpecialCharTok{\textbackslash{}}\OperatorTok{[}\NormalTok{Nu}\OperatorTok{]]}\NormalTok{ FVD}\OperatorTok{[}\FunctionTok{Subscript}\OperatorTok{[}\FunctionTok{p}\OperatorTok{,} \DecValTok{3}\OperatorTok{],} \SpecialCharTok{\textbackslash{}}\OperatorTok{[}\NormalTok{Rho}\OperatorTok{]]}\NormalTok{ FVD}\OperatorTok{[}\FunctionTok{Subscript}\OperatorTok{[}\FunctionTok{p}\OperatorTok{,} \DecValTok{4}\OperatorTok{],} \SpecialCharTok{\textbackslash{}}\OperatorTok{[}\NormalTok{Sigma}\OperatorTok{]]} 
 
\NormalTok{Contract}\OperatorTok{[}\SpecialCharTok{\%}\OperatorTok{]}
\end{Highlighting}
\end{Shaded}

\begin{dmath*}\breakingcomma
p_1{}^{\mu } p_2{}^{\nu } p_3{}^{\rho } p_4{}^{\sigma } \overset{\text{}}{\epsilon }^{\mu \nu \rho \sigma }
\end{dmath*}

\begin{dmath*}\breakingcomma
\overset{\text{}}{\epsilon }^{p_1p_2p_3p_4}
\end{dmath*}
\end{document}
