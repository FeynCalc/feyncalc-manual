% !TeX program = pdflatex
% !TeX root = Lagrangian.tex

\documentclass[../FeynCalcManual.tex]{subfiles}
\begin{document}
\hypertarget{lagrangian}{%
\section{Lagrangian}\label{lagrangian}}

\texttt{Lagrangian[\allowbreak{}"oqu"]} gives the unpolarized OPE quark
operator.

\texttt{Lagrangian[\allowbreak{}"oqp"]} gives the polarized quark OPE
operator.

\texttt{Lagrangian[\allowbreak{}"ogu"]} gives the unpolarized gluon OPE
operator.

\texttt{Lagrangian[\allowbreak{}"ogp"]} gives the polarized gluon OPE
operator.

\texttt{Lagrangian[\allowbreak{}"ogd"]} gives the sigma-term part of the
QCD Lagrangian.

\texttt{Lagrangian[\allowbreak{}"QCD"]} gives the gluon self interaction
part of the QCD Lagrangian.

\subsection{See also}

\hyperlink{toc}{Overview}, \hyperlink{feynrule}{FeynRule}.

\subsection{Examples}

\begin{Shaded}
\begin{Highlighting}[]
\NormalTok{Lagrangian}\OperatorTok{[}\StringTok{"QCD"}\OperatorTok{]}
\end{Highlighting}
\end{Shaded}

\begin{dmath*}\breakingcomma
-\frac{1}{4} F_{\text{FCGV}(\alpha )\text{FCGV}(\beta )}^{\text{FCGV}(\text{a})}.F_{\text{FCGV}(\alpha )\text{FCGV}(\beta )}^{\text{FCGV}(\text{a})}
\end{dmath*}

Twist-2 operator product expansion operators

\begin{Shaded}
\begin{Highlighting}[]
\NormalTok{Lagrangian}\OperatorTok{[}\StringTok{"ogu"}\OperatorTok{]}
\end{Highlighting}
\end{Shaded}

\begin{dmath*}\breakingcomma
\frac{1}{2} i^{m-1} F_{\text{FCGV}(\alpha )\Delta }^{\text{FCGV}(\text{a})}.\left(D_{\Delta }^{\text{FCGV}(\text{a})\text{FCGV}(\text{b})}\right){}^{m-2}.F_{\text{FCGV}(\alpha )\Delta }^{\text{FCGV}(\text{b})}
\end{dmath*}

\begin{Shaded}
\begin{Highlighting}[]
\NormalTok{Lagrangian}\OperatorTok{[}\StringTok{"ogp"}\OperatorTok{]}
\end{Highlighting}
\end{Shaded}

\begin{dmath*}\breakingcomma
\frac{1}{2} i^m \bar{\epsilon }^{\text{FCGV}(\alpha )\text{FCGV}(\beta )\text{FCGV}(\gamma )\Delta }.F_{\text{FCGV}(\beta )\text{FCGV}(\gamma )}^{\text{FCGV}(\text{a})}.\left(D_{\Delta }^{\text{FCGV}(\text{a})\text{FCGV}(\text{b})}\right){}^{m-2}.F_{\text{FCGV}(\alpha )\Delta }^{\text{FCGV}(\text{b})}
\end{dmath*}

\begin{Shaded}
\begin{Highlighting}[]
\NormalTok{Lagrangian}\OperatorTok{[}\StringTok{"oqu"}\OperatorTok{]}
\end{Highlighting}
\end{Shaded}

\begin{dmath*}\breakingcomma
i^m \bar{\psi }.\left(\bar{\gamma }\cdot \Delta \right).D_{\Delta }{}^{m-1}.\psi
\end{dmath*}

\begin{Shaded}
\begin{Highlighting}[]
\NormalTok{Lagrangian}\OperatorTok{[}\StringTok{"oqp"}\OperatorTok{]}
\end{Highlighting}
\end{Shaded}

\begin{dmath*}\breakingcomma
i^m \bar{\psi }.\bar{\gamma }^5.\left(\bar{\gamma }\cdot \Delta \right).D_{\Delta }{}^{m-1}.\psi
\end{dmath*}
\end{document}
