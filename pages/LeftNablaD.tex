% !TeX program = pdflatex
% !TeX root = LeftNablaD.tex

\documentclass[../FeynCalcManual.tex]{subfiles}
\begin{document}
\hypertarget{leftnablad}{
\section{LeftNablaD}\label{leftnablad}\index{LeftNablaD}}

\texttt{LeftNablaD[\allowbreak{}i]} denotes
\(\overleftarrow{\nabla}_{i}\) acting to the left.

\subsection{See also}

\hyperlink{toc}{Overview}, \hyperlink{expandpartiald}{ExpandPartialD},
\hyperlink{fcpartiald}{FCPartialD},
\hyperlink{leftrightnablad}{LeftRightNablaD},
\hyperlink{rightnablad}{RightNablaD}.

\subsection{Examples}

\begin{Shaded}
\begin{Highlighting}[]
\NormalTok{QuantumField}\OperatorTok{[}\FunctionTok{A}\OperatorTok{,}\NormalTok{ LorentzIndex}\OperatorTok{[}\SpecialCharTok{\textbackslash{}}\OperatorTok{[}\NormalTok{Mu}\OperatorTok{]]]}\NormalTok{ . LeftNablaD}\OperatorTok{[}\FunctionTok{i}\OperatorTok{]} 
 
\NormalTok{ex }\ExtensionTok{=}\NormalTok{ ExpandPartialD}\OperatorTok{[}\SpecialCharTok{\%}\OperatorTok{]}
\end{Highlighting}
\end{Shaded}

\begin{dmath*}\breakingcomma
A_{\mu }.\overleftarrow{\nabla }^i
\end{dmath*}

\begin{dmath*}\breakingcomma
-\left(\partial _iA_{\mu }\right)
\end{dmath*}

\begin{Shaded}
\begin{Highlighting}[]
\NormalTok{ex }\SpecialCharTok{//} \FunctionTok{StandardForm}

\CommentTok{(*{-}QuantumField[FCPartialD[CartesianIndex[i]], A, LorentzIndex[\textbackslash{}[Mu]]]*)}
\end{Highlighting}
\end{Shaded}

\begin{Shaded}
\begin{Highlighting}[]
\FunctionTok{StandardForm}\OperatorTok{[}\NormalTok{LeftNablaD}\OperatorTok{[}\FunctionTok{i}\OperatorTok{]]}

\CommentTok{(*LeftNablaD[CartesianIndex[i]]*)}
\end{Highlighting}
\end{Shaded}

\begin{Shaded}
\begin{Highlighting}[]
\NormalTok{QuantumField}\OperatorTok{[}\FunctionTok{A}\OperatorTok{,}\NormalTok{ LorentzIndex}\OperatorTok{[}\SpecialCharTok{\textbackslash{}}\OperatorTok{[}\NormalTok{Mu}\OperatorTok{]]]}\NormalTok{ . QuantumField}\OperatorTok{[}\FunctionTok{A}\OperatorTok{,}\NormalTok{ LorentzIndex}\OperatorTok{[}\SpecialCharTok{\textbackslash{}}\OperatorTok{[}\NormalTok{Nu}\OperatorTok{]]]}\NormalTok{ . LeftNablaD}\OperatorTok{[}\FunctionTok{i}\OperatorTok{]} 
 
\NormalTok{ex }\ExtensionTok{=}\NormalTok{ ExpandPartialD}\OperatorTok{[}\SpecialCharTok{\%}\OperatorTok{]}
\end{Highlighting}
\end{Shaded}

\begin{dmath*}\breakingcomma
A_{\mu }.A_{\nu }.\overleftarrow{\nabla }^i
\end{dmath*}

\begin{dmath*}\breakingcomma
-A_{\mu }.\left(\partial _iA_{\nu }\right)-\left(\partial _iA_{\mu }\right).A_{\nu }
\end{dmath*}

\begin{Shaded}
\begin{Highlighting}[]
\NormalTok{ex }\SpecialCharTok{//} \FunctionTok{StandardForm}

\CommentTok{(*{-}QuantumField[A, LorentzIndex[\textbackslash{}[Mu]]] . QuantumField[FCPartialD[CartesianIndex[i]], A, LorentzIndex[\textbackslash{}[Nu]]] {-} QuantumField[FCPartialD[CartesianIndex[i]], A, LorentzIndex[\textbackslash{}[Mu]]] . QuantumField[A, LorentzIndex[\textbackslash{}[Nu]]]*)}
\end{Highlighting}
\end{Shaded}

\end{document}
