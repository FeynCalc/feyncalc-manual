% !TeX program = pdflatex
% !TeX root = LeftPartialD.tex

\documentclass[../FeynCalcManual.tex]{subfiles}
\begin{document}
\hypertarget{leftpartiald}{
\section{LeftPartialD}\label{leftpartiald}\index{LeftPartialD}}

\texttt{LeftPartialD[\allowbreak{}mu]} denotes
\(\overleftarrow{\partial }_{\mu }\) acting to the left.

\subsection{See also}

\hyperlink{toc}{Overview}, \hyperlink{expandpartiald}{ExpandPartialD},
\hyperlink{fcpartiald}{FCPartialD},
\hyperlink{leftrightpartiald}{LeftRightPartialD},
\hyperlink{rightpartiald}{RightPartialD}.

\subsection{Examples}

\begin{Shaded}
\begin{Highlighting}[]
\NormalTok{QuantumField}\OperatorTok{[}\FunctionTok{A}\OperatorTok{,}\NormalTok{ LorentzIndex}\OperatorTok{[}\SpecialCharTok{\textbackslash{}}\OperatorTok{[}\NormalTok{Mu}\OperatorTok{]]]}\NormalTok{ . LeftPartialD}\OperatorTok{[}\SpecialCharTok{\textbackslash{}}\OperatorTok{[}\NormalTok{Nu}\OperatorTok{]]} 
 
\NormalTok{ex }\ExtensionTok{=}\NormalTok{ ExpandPartialD}\OperatorTok{[}\SpecialCharTok{\%}\OperatorTok{]}
\end{Highlighting}
\end{Shaded}

\begin{dmath*}\breakingcomma
A_{\mu }.\overleftarrow{\partial }_{\nu }
\end{dmath*}

\begin{dmath*}\breakingcomma
\left.(\partial _{\nu }A_{\mu }\right)
\end{dmath*}

\begin{Shaded}
\begin{Highlighting}[]
\NormalTok{ex }\SpecialCharTok{//} \FunctionTok{StandardForm}

\CommentTok{(*QuantumField[FCPartialD[LorentzIndex[\textbackslash{}[Nu]]], A, LorentzIndex[\textbackslash{}[Mu]]]*)}
\end{Highlighting}
\end{Shaded}

\begin{Shaded}
\begin{Highlighting}[]
\FunctionTok{StandardForm}\OperatorTok{[}\NormalTok{LeftPartialD}\OperatorTok{[}\SpecialCharTok{\textbackslash{}}\OperatorTok{[}\NormalTok{Mu}\OperatorTok{]]]}

\CommentTok{(*LeftPartialD[LorentzIndex[\textbackslash{}[Mu]]]*)}
\end{Highlighting}
\end{Shaded}

\begin{Shaded}
\begin{Highlighting}[]
\NormalTok{QuantumField}\OperatorTok{[}\FunctionTok{A}\OperatorTok{,}\NormalTok{ LorentzIndex}\OperatorTok{[}\SpecialCharTok{\textbackslash{}}\OperatorTok{[}\NormalTok{Mu}\OperatorTok{]]]}\NormalTok{ . QuantumField}\OperatorTok{[}\FunctionTok{A}\OperatorTok{,}\NormalTok{ LorentzIndex}\OperatorTok{[}\SpecialCharTok{\textbackslash{}}\OperatorTok{[}\NormalTok{Nu}\OperatorTok{]]]}\NormalTok{ . LeftPartialD}\OperatorTok{[}\SpecialCharTok{\textbackslash{}}\OperatorTok{[}\NormalTok{Rho}\OperatorTok{]]} 
 
\NormalTok{ex }\ExtensionTok{=}\NormalTok{ ExpandPartialD}\OperatorTok{[}\SpecialCharTok{\%}\OperatorTok{]}
\end{Highlighting}
\end{Shaded}

\begin{dmath*}\breakingcomma
A_{\mu }.A_{\nu }.\overleftarrow{\partial }_{\rho }
\end{dmath*}

\begin{dmath*}\breakingcomma
A_{\mu }.\left(\left.(\partial _{\rho }A_{\nu }\right)\right)+\left(\left.(\partial _{\rho }A_{\mu }\right)\right).A_{\nu }
\end{dmath*}

\begin{Shaded}
\begin{Highlighting}[]
\NormalTok{ex }\SpecialCharTok{//} \FunctionTok{StandardForm}

\CommentTok{(*QuantumField[A, LorentzIndex[\textbackslash{}[Mu]]] . QuantumField[FCPartialD[LorentzIndex[\textbackslash{}[Rho]]], A, LorentzIndex[\textbackslash{}[Nu]]] + QuantumField[FCPartialD[LorentzIndex[\textbackslash{}[Rho]]], A, LorentzIndex[\textbackslash{}[Mu]]] . QuantumField[A, LorentzIndex[\textbackslash{}[Nu]]]*)}
\end{Highlighting}
\end{Shaded}

\end{document}
