% !TeX program = pdflatex
% !TeX root = LeftRightPartialD.tex

\documentclass[../FeynCalcManual.tex]{subfiles}
\begin{document}
\hypertarget{leftrightpartiald}{%
\section{LeftRightPartialD}\label{leftrightpartiald}}

\texttt{LeftRightPartialD[\allowbreak{}mu]} denotes
\(\overleftrightarrow {\partial }_{\mu }\), acting to the left and
right.

\texttt{ExplicitPartialD[\allowbreak{}LeftRightPartialD[\allowbreak{}mu]]}
gives
\texttt{1/2 (RightPartialD[\allowbreak{}mu] - LeftPartialD[\allowbreak{}mu])}.

\subsection{See also}

\hyperlink{toc}{Overview},
\hyperlink{explicitpartiald}{ExplicitPartialD},
\hyperlink{expandpartiald}{ExpandPartialD},
\hyperlink{fcpartiald}{FCPartialD},
\hyperlink{leftpartiald}{LeftPartialD},
\hyperlink{leftrightpartiald2}{LeftRightPartialD2},
\hyperlink{rightpartiald}{RightPartialD}.

\subsection{Examples}

\begin{Shaded}
\begin{Highlighting}[]
\NormalTok{LeftRightPartialD}\OperatorTok{[}\SpecialCharTok{\textbackslash{}}\OperatorTok{[}\NormalTok{Mu}\OperatorTok{]]} 
 
\NormalTok{ExplicitPartialD}\OperatorTok{[}\SpecialCharTok{\%}\OperatorTok{]}
\end{Highlighting}
\end{Shaded}

\begin{dmath*}\breakingcomma
\overleftrightarrow{\partial }_{\mu }
\end{dmath*}

\begin{dmath*}\breakingcomma
\frac{1}{2} \left(\vec{\partial }_{\mu }-\overleftarrow{\partial }_{\mu }\right)
\end{dmath*}

\begin{Shaded}
\begin{Highlighting}[]
\NormalTok{LeftRightPartialD}\OperatorTok{[}\SpecialCharTok{\textbackslash{}}\OperatorTok{[}\NormalTok{Mu}\OperatorTok{]]}\NormalTok{ . QuantumField}\OperatorTok{[}\FunctionTok{A}\OperatorTok{,}\NormalTok{ LorentzIndex}\OperatorTok{[}\SpecialCharTok{\textbackslash{}}\OperatorTok{[}\NormalTok{Nu}\OperatorTok{]]]} 
 
\NormalTok{ExpandPartialD}\OperatorTok{[}\SpecialCharTok{\%}\OperatorTok{]}
\end{Highlighting}
\end{Shaded}

\begin{dmath*}\breakingcomma
\overleftrightarrow{\partial }_{\mu }.A_{\nu }
\end{dmath*}

\begin{dmath*}\breakingcomma
\frac{1}{2} \left(\left.(\partial _{\mu }A_{\nu }\right)-\overleftarrow{\partial }_{\mu }.A_{\nu }\right)
\end{dmath*}

\begin{Shaded}
\begin{Highlighting}[]
\NormalTok{QuantumField}\OperatorTok{[}\FunctionTok{A}\OperatorTok{,}\NormalTok{ LorentzIndex}\OperatorTok{[}\SpecialCharTok{\textbackslash{}}\OperatorTok{[}\NormalTok{Mu}\OperatorTok{]]]}\NormalTok{ . LeftRightPartialD}\OperatorTok{[}\SpecialCharTok{\textbackslash{}}\OperatorTok{[}\NormalTok{Nu}\OperatorTok{]]}\NormalTok{ . QuantumField}\OperatorTok{[}\FunctionTok{A}\OperatorTok{,}\NormalTok{ LorentzIndex}\OperatorTok{[}\SpecialCharTok{\textbackslash{}}\OperatorTok{[}\NormalTok{Rho}\OperatorTok{]]]} 
 
\NormalTok{ExpandPartialD}\OperatorTok{[}\SpecialCharTok{\%}\OperatorTok{]}
\end{Highlighting}
\end{Shaded}

\begin{dmath*}\breakingcomma
A_{\mu }.\overleftrightarrow{\partial }_{\nu }.A_{\rho }
\end{dmath*}

\begin{dmath*}\breakingcomma
\frac{1}{2} \left(A_{\mu }.\left(\left.(\partial _{\nu }A_{\rho }\right)\right)-\left(\left.(\partial _{\nu }A_{\mu }\right)\right).A_{\rho }\right)
\end{dmath*}
\end{document}
