% !TeX program = pdflatex
% !TeX root = LeviCivita.tex

\documentclass[../FeynCalcManual.tex]{subfiles}
\begin{document}
\hypertarget{levicivita}{
\section{LeviCivita}\label{levicivita}\index{LeviCivita}}

\texttt{LeviCivita[\allowbreak{}mu,\ \allowbreak{}nu,\ \allowbreak{}rho,\ \allowbreak{}si]}
is an input function for the totally antisymmetric Levi-Civita tensor.
It evaluates automatically to the internal representation
\texttt{Eps[\allowbreak{}LorentzIndex[\allowbreak{}mu],\ \allowbreak{}LorentzIndex[\allowbreak{}nu],\ \allowbreak{}LorentzIndex[\allowbreak{}rho],\ \allowbreak{}LorentzIndex[\allowbreak{}si]]}
(or with a second argument in \texttt{LorentzIndex} for the
\texttt{Dimension}, if the option \texttt{Dimension} of
\texttt{LeviCivita} is changed).

\texttt{LeviCivita[\allowbreak{}mu ,\ \allowbreak{}nu,\ \allowbreak{}...][\allowbreak{}p,\ \allowbreak{}...]}
evaluates to
\texttt{Eps[\allowbreak{}LorentzIndex[\allowbreak{}mu],\ \allowbreak{}LorentzIndex[\allowbreak{}nu],\ \allowbreak{}...,\ \allowbreak{}Momentum[\allowbreak{}p],\ \allowbreak{}...]}.

The shortcut \texttt{LeviCivita} is deprecated, please use \texttt{LC}
instead!

\subsection{See also}

\hyperlink{toc}{Overview}, \hyperlink{lc}{LC}, \hyperlink{fci}{FCI}.

\subsection{Examples}

\begin{Shaded}
\begin{Highlighting}[]
\NormalTok{LeviCivita}\OperatorTok{[}\SpecialCharTok{\textbackslash{}}\OperatorTok{[}\NormalTok{Alpha}\OperatorTok{],} \SpecialCharTok{\textbackslash{}}\OperatorTok{[}\FunctionTok{Beta}\OperatorTok{],} \SpecialCharTok{\textbackslash{}}\OperatorTok{[}\FunctionTok{Gamma}\OperatorTok{],} \SpecialCharTok{\textbackslash{}}\OperatorTok{[}\NormalTok{Delta}\OperatorTok{]]}
\end{Highlighting}
\end{Shaded}

\begin{dmath*}\breakingcomma
\bar{\epsilon }^{\alpha \beta \gamma \delta }
\end{dmath*}

\begin{Shaded}
\begin{Highlighting}[]
\NormalTok{LeviCivita}\OperatorTok{[][}\FunctionTok{p}\OperatorTok{,} \FunctionTok{q}\OperatorTok{,} \FunctionTok{r}\OperatorTok{,} \FunctionTok{s}\OperatorTok{]}
\end{Highlighting}
\end{Shaded}

\begin{dmath*}\breakingcomma
\bar{\epsilon }^{\overline{p}\overline{q}\overline{r}\overline{s}}
\end{dmath*}

\begin{Shaded}
\begin{Highlighting}[]
\NormalTok{LeviCivita}\OperatorTok{[}\SpecialCharTok{\textbackslash{}}\OperatorTok{[}\NormalTok{Alpha}\OperatorTok{],} \SpecialCharTok{\textbackslash{}}\OperatorTok{[}\FunctionTok{Beta}\OperatorTok{]][}\FunctionTok{p}\OperatorTok{,} \FunctionTok{q}\OperatorTok{]}
\end{Highlighting}
\end{Shaded}

\begin{dmath*}\breakingcomma
\bar{\epsilon }^{\alpha \beta \overline{p}\overline{q}}
\end{dmath*}

\begin{Shaded}
\begin{Highlighting}[]
\NormalTok{LeviCivita}\OperatorTok{[}\SpecialCharTok{\textbackslash{}}\OperatorTok{[}\NormalTok{Alpha}\OperatorTok{],} \SpecialCharTok{\textbackslash{}}\OperatorTok{[}\FunctionTok{Beta}\OperatorTok{]][}\FunctionTok{p}\OperatorTok{,} \FunctionTok{q}\OperatorTok{]} \SpecialCharTok{//} \FunctionTok{StandardForm}

\CommentTok{(*Eps[LorentzIndex[\textbackslash{}[Alpha]], LorentzIndex[\textbackslash{}[Beta]], Momentum[p], Momentum[q]]*)}
\end{Highlighting}
\end{Shaded}

\texttt{LeviCivita} is scheduled for removal in the future versions of
FeynCalc. The safe alternative is to use \texttt{LC}.

\begin{Shaded}
\begin{Highlighting}[]
\NormalTok{LC}\OperatorTok{[}\SpecialCharTok{\textbackslash{}}\OperatorTok{[}\NormalTok{Alpha}\OperatorTok{],} \SpecialCharTok{\textbackslash{}}\OperatorTok{[}\FunctionTok{Beta}\OperatorTok{],} \SpecialCharTok{\textbackslash{}}\OperatorTok{[}\FunctionTok{Gamma}\OperatorTok{],} \SpecialCharTok{\textbackslash{}}\OperatorTok{[}\NormalTok{Delta}\OperatorTok{]]}
\end{Highlighting}
\end{Shaded}

\begin{dmath*}\breakingcomma
\bar{\epsilon }^{\alpha \beta \gamma \delta }
\end{dmath*}

\begin{Shaded}
\begin{Highlighting}[]
\NormalTok{LC}\OperatorTok{[][}\FunctionTok{p}\OperatorTok{,} \FunctionTok{q}\OperatorTok{,} \FunctionTok{r}\OperatorTok{,} \FunctionTok{s}\OperatorTok{]}
\end{Highlighting}
\end{Shaded}

\begin{dmath*}\breakingcomma
\bar{\epsilon }^{\overline{p}\overline{q}\overline{r}\overline{s}}
\end{dmath*}

\begin{Shaded}
\begin{Highlighting}[]
\NormalTok{LC}\OperatorTok{[}\SpecialCharTok{\textbackslash{}}\OperatorTok{[}\NormalTok{Alpha}\OperatorTok{],} \SpecialCharTok{\textbackslash{}}\OperatorTok{[}\FunctionTok{Beta}\OperatorTok{]][}\FunctionTok{p}\OperatorTok{,} \FunctionTok{q}\OperatorTok{]}
\end{Highlighting}
\end{Shaded}

\begin{dmath*}\breakingcomma
\bar{\epsilon }^{\alpha \beta \overline{p}\overline{q}}
\end{dmath*}

\begin{Shaded}
\begin{Highlighting}[]
\NormalTok{LCD}\OperatorTok{[}\SpecialCharTok{\textbackslash{}}\OperatorTok{[}\NormalTok{Alpha}\OperatorTok{],} \SpecialCharTok{\textbackslash{}}\OperatorTok{[}\FunctionTok{Beta}\OperatorTok{],} \SpecialCharTok{\textbackslash{}}\OperatorTok{[}\FunctionTok{Gamma}\OperatorTok{],} \SpecialCharTok{\textbackslash{}}\OperatorTok{[}\NormalTok{Delta}\OperatorTok{]]}
\end{Highlighting}
\end{Shaded}

\begin{dmath*}\breakingcomma
\overset{\text{}}{\epsilon }^{\alpha \beta \gamma \delta }
\end{dmath*}

\begin{Shaded}
\begin{Highlighting}[]
\NormalTok{LCD}\OperatorTok{[][}\FunctionTok{p}\OperatorTok{,} \FunctionTok{q}\OperatorTok{,} \FunctionTok{r}\OperatorTok{,} \FunctionTok{s}\OperatorTok{]}
\end{Highlighting}
\end{Shaded}

\begin{dmath*}\breakingcomma
\overset{\text{}}{\epsilon }^{pqrs}
\end{dmath*}

\begin{Shaded}
\begin{Highlighting}[]
\NormalTok{LCD}\OperatorTok{[}\SpecialCharTok{\textbackslash{}}\OperatorTok{[}\NormalTok{Alpha}\OperatorTok{],} \SpecialCharTok{\textbackslash{}}\OperatorTok{[}\FunctionTok{Beta}\OperatorTok{]][}\FunctionTok{p}\OperatorTok{,} \FunctionTok{q}\OperatorTok{]}
\end{Highlighting}
\end{Shaded}

\begin{dmath*}\breakingcomma
\overset{\text{}}{\epsilon }^{\alpha \beta pq}
\end{dmath*}

\begin{Shaded}
\begin{Highlighting}[]
\NormalTok{FCI}\OperatorTok{[}\NormalTok{LC}\OperatorTok{[}\SpecialCharTok{\textbackslash{}}\OperatorTok{[}\NormalTok{Alpha}\OperatorTok{],} \SpecialCharTok{\textbackslash{}}\OperatorTok{[}\FunctionTok{Beta}\OperatorTok{],} \SpecialCharTok{\textbackslash{}}\OperatorTok{[}\FunctionTok{Gamma}\OperatorTok{],} \SpecialCharTok{\textbackslash{}}\OperatorTok{[}\NormalTok{Delta}\OperatorTok{]]]} \ExtensionTok{===}\NormalTok{ LeviCivita}\OperatorTok{[}\SpecialCharTok{\textbackslash{}}\OperatorTok{[}\NormalTok{Alpha}\OperatorTok{],} \SpecialCharTok{\textbackslash{}}\OperatorTok{[}\FunctionTok{Beta}\OperatorTok{],} \SpecialCharTok{\textbackslash{}}\OperatorTok{[}\FunctionTok{Gamma}\OperatorTok{],} \SpecialCharTok{\textbackslash{}}\OperatorTok{[}\NormalTok{Delta}\OperatorTok{]]}
\end{Highlighting}
\end{Shaded}

\begin{dmath*}\breakingcomma
\text{True}
\end{dmath*}

\begin{Shaded}
\begin{Highlighting}[]
\NormalTok{FCI}\OperatorTok{[}\NormalTok{LCD}\OperatorTok{[}\SpecialCharTok{\textbackslash{}}\OperatorTok{[}\NormalTok{Alpha}\OperatorTok{],} \SpecialCharTok{\textbackslash{}}\OperatorTok{[}\FunctionTok{Beta}\OperatorTok{],} \SpecialCharTok{\textbackslash{}}\OperatorTok{[}\FunctionTok{Gamma}\OperatorTok{],} \SpecialCharTok{\textbackslash{}}\OperatorTok{[}\NormalTok{Delta}\OperatorTok{]]]} \ExtensionTok{===}\NormalTok{ LeviCivita}\OperatorTok{[}\SpecialCharTok{\textbackslash{}}\OperatorTok{[}\NormalTok{Alpha}\OperatorTok{],} \SpecialCharTok{\textbackslash{}}\OperatorTok{[}\FunctionTok{Beta}\OperatorTok{],} \SpecialCharTok{\textbackslash{}}\OperatorTok{[}\FunctionTok{Gamma}\OperatorTok{],} \SpecialCharTok{\textbackslash{}}\OperatorTok{[}\NormalTok{Delta}\OperatorTok{],}\NormalTok{ Dimension }\OtherTok{{-}\textgreater{}} \FunctionTok{D}\OperatorTok{]}
\end{Highlighting}
\end{Shaded}

\begin{dmath*}\breakingcomma
\text{True}
\end{dmath*}
\end{document}
