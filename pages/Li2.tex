% !TeX program = pdflatex
% !TeX root = Li2.tex

\documentclass[../FeynCalcManual.tex]{subfiles}
\begin{document}
\hypertarget{li2}{
\section{Li2}\label{li2}\index{Li2}}

\texttt{Li2} is an abbreviation for the dilogarithm function,
i.e.~\texttt{Li2 = PolyLog[\allowbreak{}2,\ \allowbreak{}\#{}\allowbreak{}]\&{}\allowbreak{}}.

\subsection{See also}

\hyperlink{toc}{Overview}, \hyperlink{li3}{Li3}, \hyperlink{li4}{Li4}.

\subsection{Examples}

\begin{Shaded}
\begin{Highlighting}[]
\NormalTok{Li2}\OperatorTok{[}\FunctionTok{x}\OperatorTok{]}
\end{Highlighting}
\end{Shaded}

\begin{dmath*}\breakingcomma
\text{Li}_2(x)
\end{dmath*}

\begin{Shaded}
\begin{Highlighting}[]
\NormalTok{Li2 }\SpecialCharTok{//} \FunctionTok{StandardForm}

\CommentTok{(*PolyLog[2, \#1] \&*)}
\end{Highlighting}
\end{Shaded}

\begin{Shaded}
\begin{Highlighting}[]
\FunctionTok{Integrate}\OperatorTok{[}\SpecialCharTok{{-}}\FunctionTok{Log}\OperatorTok{[}\DecValTok{1} \SpecialCharTok{{-}} \FunctionTok{x}\OperatorTok{]}\SpecialCharTok{/}\FunctionTok{x}\OperatorTok{,} \FunctionTok{x}\OperatorTok{]}
\end{Highlighting}
\end{Shaded}

\begin{dmath*}\breakingcomma
\text{Li}_2(x)
\end{dmath*}
\end{document}
