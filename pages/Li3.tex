% !TeX program = pdflatex
% !TeX root = Li3.tex

\documentclass[../FeynCalcManual.tex]{subfiles}
\begin{document}
\hypertarget{li3}{%
\section{Li3}\label{li3}}

\texttt{Li3} is an abbreviation for the trilogarithm function,
i.e.~\texttt{Li3 = PolyLog[\allowbreak{}3,\ \allowbreak{}\#{}\allowbreak{}]\&{}\allowbreak{}}.

\subsection{See also}

\hyperlink{toc}{Overview}, \hyperlink{li2}{Li2}.

\subsection{Examples}

\begin{Shaded}
\begin{Highlighting}[]
\NormalTok{Li3}\OperatorTok{[}\FunctionTok{x}\OperatorTok{]}
\end{Highlighting}
\end{Shaded}

\begin{dmath*}\breakingcomma
\text{Li}_3(x)
\end{dmath*}

\begin{Shaded}
\begin{Highlighting}[]
\NormalTok{Li3 }\SpecialCharTok{//} \FunctionTok{StandardForm}

\CommentTok{(*PolyLog[3, \#1] \&*)}
\end{Highlighting}
\end{Shaded}

\begin{Shaded}
\begin{Highlighting}[]
\FunctionTok{D}\OperatorTok{[}\NormalTok{Li3}\OperatorTok{[}\FunctionTok{x}\OperatorTok{],} \FunctionTok{x}\OperatorTok{]}
\end{Highlighting}
\end{Shaded}

\begin{dmath*}\breakingcomma
\frac{\text{Li}_2(x)}{x}
\end{dmath*}

\begin{Shaded}
\begin{Highlighting}[]
\FunctionTok{Integrate}\OperatorTok{[}\NormalTok{Li3}\OperatorTok{[}\FunctionTok{x}\OperatorTok{]}\SpecialCharTok{/}\FunctionTok{x}\OperatorTok{,} \FunctionTok{x}\OperatorTok{]}
\end{Highlighting}
\end{Shaded}

\begin{dmath*}\breakingcomma
\text{Li}_4(x)
\end{dmath*}
\end{document}
