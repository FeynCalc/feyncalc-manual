% !TeX program = pdflatex
% !TeX root = LightConePerpendicularComponent.tex

\documentclass[../FeynCalcManual.tex]{subfiles}
\begin{document}
\begin{Shaded}
\begin{Highlighting}[]
 
\end{Highlighting}
\end{Shaded}

\hypertarget{lightconeperpendicularcomponent}{
\section{LightConePerpendicularComponent}\label{lightconeperpendicularcomponent}\index{LightConePerpendicularComponent}}

\texttt{LightConePerpendicularComponent[\allowbreak{}LorentzIndex[\allowbreak{}mu],\ \allowbreak{}Momentum[\allowbreak{}n],\ \allowbreak{}Momentum[\allowbreak{}nb]]}
denotes the perpendicular component of the Lorentz index \texttt{mu}
with respect to the lightcone momenta \texttt{n} and \texttt{nb}.

\texttt{LightConePerpendicularComponent[\allowbreak{}Momentum[\allowbreak{}p],\ \allowbreak{}Momentum[\allowbreak{}n],\ \allowbreak{}Momentum[\allowbreak{}nb]]}
denotes the perpendicular component of the 4-momentum \texttt{p} with
respect to the lightcone momenta \texttt{n} and \texttt{nb}.

\subsection{See also}

\hyperlink{toc}{Overview}, \hyperlink{lorentzindex}{LorentzIndex},
\hyperlink{momentum}{Momentum}.

\subsection{Examples}

\(4\)-dimensional Lorentz vector

\begin{Shaded}
\begin{Highlighting}[]
\NormalTok{Pair}\OperatorTok{[}\NormalTok{LightConePerpendicularComponent}\OperatorTok{[}\NormalTok{LorentzIndex}\OperatorTok{[}\SpecialCharTok{\textbackslash{}}\OperatorTok{[}\NormalTok{Mu}\OperatorTok{]],}\NormalTok{ Momentum}\OperatorTok{[}\FunctionTok{n}\OperatorTok{],}\NormalTok{Momentum}\OperatorTok{[}\NormalTok{nb}\OperatorTok{]],} 
\NormalTok{  LightConePerpendicularComponent}\OperatorTok{[}\NormalTok{Momentum}\OperatorTok{[}\FunctionTok{p}\OperatorTok{],}\NormalTok{ Momentum}\OperatorTok{[}\FunctionTok{n}\OperatorTok{],}\NormalTok{ Momentum}\OperatorTok{[}\NormalTok{nb}\OperatorTok{]]]}
\end{Highlighting}
\end{Shaded}

\begin{dmath*}\breakingcomma
\overline{p}^{\mu }{}_{\perp }
\end{dmath*}

Metric tensor

\begin{Shaded}
\begin{Highlighting}[]
\NormalTok{Pair}\OperatorTok{[}\NormalTok{LightConePerpendicularComponent}\OperatorTok{[}\NormalTok{LorentzIndex}\OperatorTok{[}\SpecialCharTok{\textbackslash{}}\OperatorTok{[}\NormalTok{Mu}\OperatorTok{]],}\NormalTok{ Momentum}\OperatorTok{[}\FunctionTok{n}\OperatorTok{],}\NormalTok{Momentum}\OperatorTok{[}\NormalTok{nb}\OperatorTok{]],} 
\NormalTok{  LightConePerpendicularComponent}\OperatorTok{[}\NormalTok{LorentzIndex}\OperatorTok{[}\SpecialCharTok{\textbackslash{}}\OperatorTok{[}\NormalTok{Nu}\OperatorTok{]],}\NormalTok{ Momentum}\OperatorTok{[}\FunctionTok{n}\OperatorTok{],}\NormalTok{ Momentum}\OperatorTok{[}\NormalTok{nb}\OperatorTok{]]]}
\end{Highlighting}
\end{Shaded}

\begin{dmath*}\breakingcomma
\bar{g}^{\mu \nu }{}_{\perp }
\end{dmath*}

Dirac matrix

\begin{Shaded}
\begin{Highlighting}[]
\NormalTok{DiracGamma}\OperatorTok{[}\NormalTok{LightConePerpendicularComponent}\OperatorTok{[}\NormalTok{LorentzIndex}\OperatorTok{[}\SpecialCharTok{\textbackslash{}}\OperatorTok{[}\NormalTok{Mu}\OperatorTok{]],}\NormalTok{ Momentum}\OperatorTok{[}\FunctionTok{n}\OperatorTok{],}\NormalTok{ Momentum}\OperatorTok{[}\NormalTok{nb}\OperatorTok{]]]}
\end{Highlighting}
\end{Shaded}

\begin{dmath*}\breakingcomma
\bar{\gamma }^{\mu }{}_{\perp }
\end{dmath*}

Contractions

\begin{Shaded}
\begin{Highlighting}[]
\NormalTok{DiracGamma}\OperatorTok{[}\NormalTok{LightConePerpendicularComponent}\OperatorTok{[}\NormalTok{LorentzIndex}\OperatorTok{[}\SpecialCharTok{\textbackslash{}}\OperatorTok{[}\NormalTok{Mu}\OperatorTok{]],} 
\NormalTok{      Momentum}\OperatorTok{[}\FunctionTok{n}\OperatorTok{],}\NormalTok{ Momentum}\OperatorTok{[}\NormalTok{nb}\OperatorTok{]]]}\NormalTok{ FV}\OperatorTok{[}\FunctionTok{p}\OperatorTok{,} \SpecialCharTok{\textbackslash{}}\OperatorTok{[}\NormalTok{Mu}\OperatorTok{]]} \SpecialCharTok{//}\NormalTok{ Contract }
 
\SpecialCharTok{\%} \SpecialCharTok{//} \FunctionTok{StandardForm}
\end{Highlighting}
\end{Shaded}

\begin{dmath*}\breakingcomma
\bar{\gamma }\cdot \overline{p}_{\perp }
\end{dmath*}

\begin{verbatim}
(*DiracGamma[LightConePerpendicularComponent[Momentum[p], Momentum[n], Momentum[nb]]]*)
\end{verbatim}
\end{document}
