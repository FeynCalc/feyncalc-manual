% !TeX program = pdflatex
% !TeX root = Load.tex

\documentclass[../FeynCalcManual.tex]{subfiles}
\begin{document}
\hypertarget{loading the package}{
\section{Loading the package}\label{loading the package}\index{Loading the package}}

\subsection{See also}

\hyperlink{toc}{Overview}.

\subsection{Getting started}\label{getting-started}

To load FeynCalc 9 or newer run

\begin{Shaded}
\begin{Highlighting}[]
\NormalTok{\textless{}\textless{}FeynCalc\textasciigrave{}}
\end{Highlighting}
\end{Shaded}

in your Mathematica session. Notice that once FeynCalc has been loaded,
it cannot be ``reloaded'' again. You need to restart the kernel first.

\subsection{Add-ons}\label{add-ons}

Extensions such FeynHelpers, FeynOnium, FeynArts loader, PHI etc. are
loaded by adding the corresponding text strings to the global variable
\texttt{\$LoadAddOns}. This variable must be set before loading
FeynCalc. For example

\begin{Shaded}
\begin{Highlighting}[]
\NormalTok{$LoadAddOns}\ExtensionTok{=}\OperatorTok{\{}\StringTok{"FeynArts"}\OperatorTok{,}\StringTok{"FeynHelpers"}\OperatorTok{\}}\NormalTok{;}
\NormalTok{\textless{}\textless{}FeynCalc\textasciigrave{}}
\end{Highlighting}
\end{Shaded}

\subsection{Startup messages}\label{startup-messages}

To suppress the package startup messages you can set the global variable
\texttt{\$FeynCalcStartupMessages} to \texttt{False} before loading
FeynCalc. For example

\begin{Shaded}
\begin{Highlighting}[]
\NormalTok{$FeynCalcStartupMessages}\ExtensionTok{=}\ConstantTok{False}\NormalTok{;}
\NormalTok{\textless{}\textless{}FeynCalc\textasciigrave{}}
\end{Highlighting}
\end{Shaded}

\end{document}
