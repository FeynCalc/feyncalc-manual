% !TeX program = pdflatex
% !TeX root = LorentzIndex.tex

\documentclass[../FeynCalcManual.tex]{subfiles}
\begin{document}
\hypertarget{lorentzindex}{
\section{LorentzIndex}\label{lorentzindex}\index{LorentzIndex}}

\texttt{LorentzIndex[\allowbreak{}mu]} denotes a \(4\)-dimensional
Lorentz index.

For other than \(4\) dimensions:
\texttt{LorentzIndex[\allowbreak{}mu,\ \allowbreak{}D]} or
\texttt{LorentzIndex[\allowbreak{}mu]} etc.

\texttt{LorentzIndex[\allowbreak{}mu,\ \allowbreak{}4]} simplifies to
\texttt{LorentzIndex[\allowbreak{}mu]}.

\subsection{See also}

\hyperlink{toc}{Overview}, \hyperlink{changedimension}{ChangeDimension},
\hyperlink{momentum}{Momentum}.

\subsection{Examples}

This denotes a \(4\)-dimensional Lorentz index.

\begin{Shaded}
\begin{Highlighting}[]
\NormalTok{LorentzIndex}\OperatorTok{[}\SpecialCharTok{\textbackslash{}}\OperatorTok{[}\NormalTok{Alpha}\OperatorTok{]]}
\end{Highlighting}
\end{Shaded}

\begin{dmath*}\breakingcomma
\alpha
\end{dmath*}

An optional second argument can be given for a dimension different from
\(4\).

\begin{Shaded}
\begin{Highlighting}[]
\NormalTok{LorentzIndex}\OperatorTok{[}\SpecialCharTok{\textbackslash{}}\OperatorTok{[}\NormalTok{Alpha}\OperatorTok{],} \FunctionTok{n}\OperatorTok{]}
\end{Highlighting}
\end{Shaded}

\begin{dmath*}\breakingcomma
\alpha
\end{dmath*}
\end{document}
