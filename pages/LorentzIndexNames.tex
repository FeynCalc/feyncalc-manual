% !TeX program = pdflatex
% !TeX root = LorentzIndexNames.tex

\documentclass[../FeynCalcManual.tex]{subfiles}
\begin{document}
\hypertarget{lorentzindexnames}{%
\section{LorentzIndexNames}\label{lorentzindexnames}}

\texttt{LorentzIndexNames} is an option for \texttt{FCFAConvert},
\texttt{FCCanonicalizeDummyIndices} and other functions. It renames the
generic dummy Lorentz indices to the indices in the supplied list.

\subsection{See also}

\hyperlink{toc}{Overview}, \hyperlink{fcfaconvert}{FCFAConvert},
\hyperlink{fccanonicalizedummyindices}{FCCanonicalizeDummyIndices},
\hyperlink{cartesianindexnames}{CartesianIndexNames}.

\subsection{Examples}

\begin{Shaded}
\begin{Highlighting}[]
\NormalTok{LC}\OperatorTok{[}\NormalTok{i1}\OperatorTok{,}\NormalTok{ i2}\OperatorTok{,}\NormalTok{ i3}\OperatorTok{,}\NormalTok{ i4}\OperatorTok{]}\NormalTok{ GA}\OperatorTok{[}\NormalTok{i1}\OperatorTok{,}\NormalTok{ i2}\OperatorTok{,}\NormalTok{ i3}\OperatorTok{,}\NormalTok{ i4}\OperatorTok{]} 
 
\NormalTok{FCCanonicalizeDummyIndices}\OperatorTok{[}\SpecialCharTok{\%}\OperatorTok{]}
\end{Highlighting}
\end{Shaded}

\begin{dmath*}\breakingcomma
\bar{\gamma }^{\text{i1}}.\bar{\gamma }^{\text{i2}}.\bar{\gamma }^{\text{i3}}.\bar{\gamma }^{\text{i4}} \bar{\epsilon }^{\text{i1}\;\text{i2}\;\text{i3}\;\text{i4}}
\end{dmath*}

\begin{dmath*}\breakingcomma
\bar{\gamma }^{\text{FCGV}(\text{li191})}.\bar{\gamma }^{\text{FCGV}(\text{li192})}.\bar{\gamma }^{\text{FCGV}(\text{li193})}.\bar{\gamma }^{\text{FCGV}(\text{li194})} \bar{\epsilon }^{\text{FCGV}(\text{li191})\text{FCGV}(\text{li192})\text{FCGV}(\text{li193})\text{FCGV}(\text{li194})}
\end{dmath*}

\begin{Shaded}
\begin{Highlighting}[]
\NormalTok{LC}\OperatorTok{[}\NormalTok{i1}\OperatorTok{,}\NormalTok{ i2}\OperatorTok{,}\NormalTok{ i3}\OperatorTok{,}\NormalTok{ i4}\OperatorTok{]}\NormalTok{ GA}\OperatorTok{[}\NormalTok{i1}\OperatorTok{,}\NormalTok{ i2}\OperatorTok{,}\NormalTok{ i3}\OperatorTok{,}\NormalTok{ i4}\OperatorTok{]} 
 
\NormalTok{FCCanonicalizeDummyIndices}\OperatorTok{[}\SpecialCharTok{\%}\OperatorTok{,}\NormalTok{ LorentzIndexNames }\OtherTok{{-}\textgreater{}} \OperatorTok{\{}\NormalTok{mu}\OperatorTok{,}\NormalTok{ nu}\OperatorTok{,}\NormalTok{ rho}\OperatorTok{,}\NormalTok{ si}\OperatorTok{\}]}
\end{Highlighting}
\end{Shaded}

\begin{dmath*}\breakingcomma
\bar{\gamma }^{\text{i1}}.\bar{\gamma }^{\text{i2}}.\bar{\gamma }^{\text{i3}}.\bar{\gamma }^{\text{i4}} \bar{\epsilon }^{\text{i1}\;\text{i2}\;\text{i3}\;\text{i4}}
\end{dmath*}

\begin{dmath*}\breakingcomma
\bar{\gamma }^{\text{mu}}.\bar{\gamma }^{\text{nu}}.\bar{\gamma }^{\text{rho}}.\bar{\gamma }^{\text{si}} \bar{\epsilon }^{\text{mu}\;\text{nu}\;\text{rho}\;\text{si}}
\end{dmath*}
\end{document}
