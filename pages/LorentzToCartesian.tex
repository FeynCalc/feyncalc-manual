% !TeX program = pdflatex
% !TeX root = LorentzToCartesian.tex

\documentclass[../FeynCalcManual.tex]{subfiles}
\begin{document}
\hypertarget{lorentztocartesian}{
\section{LorentzToCartesian}\label{lorentztocartesian}\index{LorentzToCartesian}}

\texttt{LorentzToCartesian[\allowbreak{}exp]} rewrites Lorentz tensors
in form of Cartesian tensors (when possible). Using options one can
specify which types of tensors should be converted.

\subsection{See also}

\hyperlink{toc}{Overview},
\hyperlink{cartesiantolorentz}{CartesianToLorentz}.

\subsection{Examples}

\begin{Shaded}
\begin{Highlighting}[]
\NormalTok{SPD}\OperatorTok{[}\FunctionTok{p}\OperatorTok{,} \FunctionTok{q}\OperatorTok{]} 
 
\SpecialCharTok{\%} \SpecialCharTok{//}\NormalTok{ LorentzToCartesian}
\end{Highlighting}
\end{Shaded}

\begin{dmath*}\breakingcomma
p\cdot q
\end{dmath*}

\begin{dmath*}\breakingcomma
p^0 q^0-p\cdot q
\end{dmath*}

\begin{Shaded}
\begin{Highlighting}[]
\NormalTok{LC}\OperatorTok{[}\SpecialCharTok{\textbackslash{}}\OperatorTok{[}\NormalTok{Mu}\OperatorTok{],} \SpecialCharTok{\textbackslash{}}\OperatorTok{[}\NormalTok{Nu}\OperatorTok{]][}\FunctionTok{p}\OperatorTok{,} \FunctionTok{q}\OperatorTok{]} 
 
\SpecialCharTok{\%} \SpecialCharTok{//}\NormalTok{ LorentzToCartesian}
\end{Highlighting}
\end{Shaded}

\begin{dmath*}\breakingcomma
\bar{\epsilon }^{\mu \nu \overline{p}\overline{q}}
\end{dmath*}

\begin{dmath*}\breakingcomma
\bar{g}^{0\mu } \bar{g}^{\text{\$MU}(\text{\$20})\nu } \left(-\bar{\epsilon }^{\text{\$MU}(\text{\$20})\overline{p}\overline{q}}\right)-\bar{g}^{\text{\$MU}(\text{\$20})\mu } \left(\bar{g}^{0\nu } \left(-\bar{\epsilon }^{\text{\$MU}(\text{\$20})\overline{p}\overline{q}}\right)-\bar{g}^{\text{\$MU}(\text{\$21})\nu } \left(q^0 \bar{\epsilon }^{\text{\$MU}(\text{\$20})\text{\$MU}(\text{\$21})\overline{p}}-p^0 \bar{\epsilon }^{\text{\$MU}(\text{\$20})\text{\$MU}(\text{\$21})\overline{q}}\right)\right)
\end{dmath*}

\begin{Shaded}
\begin{Highlighting}[]
\NormalTok{GAD}\OperatorTok{[}\SpecialCharTok{\textbackslash{}}\OperatorTok{[}\NormalTok{Mu}\OperatorTok{]]} 
 
\SpecialCharTok{\%} \SpecialCharTok{//}\NormalTok{ LorentzToCartesian}
\end{Highlighting}
\end{Shaded}

\begin{dmath*}\breakingcomma
\gamma ^{\mu }
\end{dmath*}

\begin{dmath*}\breakingcomma
\bar{\gamma }^0 \bar{g}^{0\mu }-\gamma ^{\text{\$MU}(\text{\$22})} g^{\text{\$MU}(\text{\$22})\mu }
\end{dmath*}
\end{document}
