% !TeX program = pdflatex
% !TeX root = MTLND.tex

\documentclass[../FeynCalcManual.tex]{subfiles}
\begin{document}
\hypertarget{mtlnd}{
\section{MTLND}\label{mtlnd}\index{MTLND}}

\texttt{MTLND[\allowbreak{}mu,\ \allowbreak{}nu,\ \allowbreak{}n,\ \allowbreak{}nb]}
denotes the positive component in the lightcone decomposition of the
metric tensor \(g^{\mu \nu}\) along the vectors \texttt{n} and
\texttt{nb}in \(D\) dimensions. It corresponds to
\(\frac{1}{2} n^{\mu}  \bar{n}^\nu\).

If one omits \texttt{n} and \texttt{nb}, the program will use default
vectors specified via \texttt{\$FCDefaultLightconeVectorN} and
\texttt{\$FCDefaultLightconeVectorNB}.

\subsection{See also}

\hyperlink{toc}{Overview}, \hyperlink{pair}{Pair},
\hyperlink{fvlpd}{FVLPD}, \hyperlink{fvlnd}{FVLND},
\hyperlink{fvlrd}{FVLRD}, \hyperlink{splpd}{SPLPD},
\hyperlink{splnd}{SPLND}, \hyperlink{splrd}{SPLRD},
\hyperlink{mtlpd}{MTLPD}, \hyperlink{mtlrd}{MTLRD}.

\subsection{Examples}

\begin{Shaded}
\begin{Highlighting}[]
\NormalTok{MTLND}\OperatorTok{[}\SpecialCharTok{\textbackslash{}}\OperatorTok{[}\NormalTok{Mu}\OperatorTok{],} \SpecialCharTok{\textbackslash{}}\OperatorTok{[}\NormalTok{Nu}\OperatorTok{],} \FunctionTok{n}\OperatorTok{,}\NormalTok{ nb}\OperatorTok{]}
\end{Highlighting}
\end{Shaded}

\begin{dmath*}\breakingcomma
\frac{n^{\mu } \;\text{nb}^{\nu }}{2}
\end{dmath*}

\begin{Shaded}
\begin{Highlighting}[]
\FunctionTok{StandardForm}\OperatorTok{[}\NormalTok{MTLND}\OperatorTok{[}\SpecialCharTok{\textbackslash{}}\OperatorTok{[}\NormalTok{Mu}\OperatorTok{],} \SpecialCharTok{\textbackslash{}}\OperatorTok{[}\NormalTok{Nu}\OperatorTok{],} \FunctionTok{n}\OperatorTok{,}\NormalTok{ nb}\OperatorTok{]} \SpecialCharTok{//}\NormalTok{ FCI}\OperatorTok{]}
\end{Highlighting}
\end{Shaded}

\begin{dmath*}\breakingcomma
\frac{1}{2} \;\text{Pair}[\text{LorentzIndex}[\mu ,D],\text{Momentum}[n,D]] \;\text{Pair}[\text{LorentzIndex}[\nu ,D],\text{Momentum}[\text{nb},D]]
\end{dmath*}

Notice that the properties of \texttt{n} and \texttt{nb} vectors have to
be set by hand before doing the actual computation

\begin{Shaded}
\begin{Highlighting}[]
\NormalTok{MTLND}\OperatorTok{[}\SpecialCharTok{\textbackslash{}}\OperatorTok{[}\NormalTok{Mu}\OperatorTok{],} \SpecialCharTok{\textbackslash{}}\OperatorTok{[}\NormalTok{Nu}\OperatorTok{],} \FunctionTok{n}\OperatorTok{,}\NormalTok{ nb}\OperatorTok{]}\NormalTok{ FVD}\OperatorTok{[}\FunctionTok{p}\OperatorTok{,} \SpecialCharTok{\textbackslash{}}\OperatorTok{[}\NormalTok{Mu}\OperatorTok{]]} \SpecialCharTok{//}\NormalTok{ Contract}
\end{Highlighting}
\end{Shaded}

\begin{dmath*}\breakingcomma
\frac{1}{2} \;\text{nb}^{\nu } (n\cdot p)
\end{dmath*}

\begin{Shaded}
\begin{Highlighting}[]
\NormalTok{MTLND}\OperatorTok{[}\SpecialCharTok{\textbackslash{}}\OperatorTok{[}\NormalTok{Mu}\OperatorTok{],} \SpecialCharTok{\textbackslash{}}\OperatorTok{[}\NormalTok{Nu}\OperatorTok{],} \FunctionTok{n}\OperatorTok{,}\NormalTok{ nb}\OperatorTok{]}\NormalTok{ FVD}\OperatorTok{[}\FunctionTok{p}\OperatorTok{,} \SpecialCharTok{\textbackslash{}}\OperatorTok{[}\NormalTok{Nu}\OperatorTok{]]} \SpecialCharTok{//}\NormalTok{ Contract}
\end{Highlighting}
\end{Shaded}

\begin{dmath*}\breakingcomma
\frac{1}{2} n^{\mu } (\text{nb}\cdot p)
\end{dmath*}

\begin{Shaded}
\begin{Highlighting}[]
\NormalTok{MTLND}\OperatorTok{[}\SpecialCharTok{\textbackslash{}}\OperatorTok{[}\NormalTok{Mu}\OperatorTok{],} \SpecialCharTok{\textbackslash{}}\OperatorTok{[}\NormalTok{Nu}\OperatorTok{],} \FunctionTok{n}\OperatorTok{,}\NormalTok{ nb}\OperatorTok{]}\NormalTok{ FVD}\OperatorTok{[}\FunctionTok{n}\OperatorTok{,} \SpecialCharTok{\textbackslash{}}\OperatorTok{[}\NormalTok{Nu}\OperatorTok{]]} \SpecialCharTok{//}\NormalTok{ Contract}
\end{Highlighting}
\end{Shaded}

\begin{dmath*}\breakingcomma
\frac{1}{2} n^{\mu } (n\cdot \;\text{nb})
\end{dmath*}

\begin{Shaded}
\begin{Highlighting}[]
\NormalTok{FCClearScalarProducts}\OperatorTok{[]}
\NormalTok{SPD}\OperatorTok{[}\FunctionTok{n}\OperatorTok{]} \ExtensionTok{=} \DecValTok{0}\NormalTok{;}
\NormalTok{SPD}\OperatorTok{[}\NormalTok{nb}\OperatorTok{]} \ExtensionTok{=} \DecValTok{0}\NormalTok{;}
\NormalTok{SPD}\OperatorTok{[}\FunctionTok{n}\OperatorTok{,}\NormalTok{ nb}\OperatorTok{]} \ExtensionTok{=} \DecValTok{2}\NormalTok{;}
\end{Highlighting}
\end{Shaded}

\begin{Shaded}
\begin{Highlighting}[]
\NormalTok{MTLND}\OperatorTok{[}\SpecialCharTok{\textbackslash{}}\OperatorTok{[}\NormalTok{Mu}\OperatorTok{],} \SpecialCharTok{\textbackslash{}}\OperatorTok{[}\NormalTok{Nu}\OperatorTok{],} \FunctionTok{n}\OperatorTok{,}\NormalTok{ nb}\OperatorTok{]}\NormalTok{ FVD}\OperatorTok{[}\FunctionTok{n}\OperatorTok{,} \SpecialCharTok{\textbackslash{}}\OperatorTok{[}\NormalTok{Nu}\OperatorTok{]]} \SpecialCharTok{//}\NormalTok{ Contract}
\end{Highlighting}
\end{Shaded}

\begin{dmath*}\breakingcomma
n^{\mu }
\end{dmath*}

\begin{Shaded}
\begin{Highlighting}[]
\NormalTok{FCClearScalarProducts}\OperatorTok{[]}
\end{Highlighting}
\end{Shaded}

\end{document}
