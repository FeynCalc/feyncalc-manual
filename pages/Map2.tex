% !TeX program = pdflatex
% !TeX root = Map2.tex

\documentclass[../FeynCalcManual.tex]{subfiles}
\begin{document}
\hypertarget{map2}{%
\section{Map2}\label{map2}}

\texttt{Map2[\allowbreak{}f,\ \allowbreak{}exp]} is equivalent to
\texttt{Map} if \texttt{Nterms[\allowbreak{}exp] > 0}, otherwise
\texttt{Map2[\allowbreak{}f,\ \allowbreak{}exp]} gives
\texttt{f[\allowbreak{}exp]}.

\subsection{See also}

\hyperlink{toc}{Overview}, \hyperlink{nterms}{NTerms}.

\subsection{Examples}

\begin{Shaded}
\begin{Highlighting}[]
\NormalTok{Map2}\OperatorTok{[}\FunctionTok{f}\OperatorTok{,} \FunctionTok{a} \SpecialCharTok{{-}} \FunctionTok{b}\OperatorTok{]}
\end{Highlighting}
\end{Shaded}

\begin{dmath*}\breakingcomma
f(a)+f(-b)
\end{dmath*}

\begin{Shaded}
\begin{Highlighting}[]
\NormalTok{Map2}\OperatorTok{[}\FunctionTok{f}\OperatorTok{,} \FunctionTok{x}\OperatorTok{]}
\end{Highlighting}
\end{Shaded}

\begin{dmath*}\breakingcomma
f(x)
\end{dmath*}

\begin{Shaded}
\begin{Highlighting}[]
\NormalTok{Map2}\OperatorTok{[}\FunctionTok{f}\OperatorTok{,} \OperatorTok{\{}\FunctionTok{a}\OperatorTok{,} \FunctionTok{b}\OperatorTok{,} \FunctionTok{c}\OperatorTok{\}]}
\end{Highlighting}
\end{Shaded}

\begin{dmath*}\breakingcomma
f(\{a,b,c\})
\end{dmath*}

\begin{Shaded}
\begin{Highlighting}[]
\NormalTok{Map2}\OperatorTok{[}\FunctionTok{f}\OperatorTok{,} \DecValTok{1}\OperatorTok{]}
\end{Highlighting}
\end{Shaded}

\begin{dmath*}\breakingcomma
f(1)
\end{dmath*}
\end{document}
