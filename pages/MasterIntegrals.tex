% !TeX program = pdflatex
% !TeX root = MasterIntegrals.tex

\documentclass[../FeynCalcManual.tex]{subfiles}
\begin{document}
\hypertarget{master integrals}{
\section{Master integrals}\label{master integrals}\index{Master integrals}}

\subsection{See also}

\hyperlink{toc}{Overview}.

For the sake of the users that try to employ FeynCalc in calculations
beyond 1-loop but do not belong to the multiloop community, let us
summarize some basic facts about master integrals in multiloop
calculations

\hypertarget{analytic-results}{%
\subsection{Analytic results}\label{analytic-results}}

\begin{itemize}
\item
  Even at 2-loop there is no library containing analytic results for all
  master integrals with arbitrary mass distributions. That is, there is
  simply nothing similar to Package-X beyond 1-loop
\item
  The available libraries usually focus on very specific integral
  families,
  e.g.~\href{https://www.nikhef.nl/~form/maindir/packages/mincer/mincer.html}{Mincer}
  (3-loop massless 2-point functions),
  \href{https://github.com/benruijl/forcer}{Forcer} (4-loop massless
  2-point functions),
  \href{https://www.ttp.kit.edu/~ms/software.html}{MATAD},
  \href{https://github.com/apik/matad-ng}{MATAD-ng} (massive 3-loop
  tadpoles), \href{https://github.com/apik/fmft}{FMFT} (massive 4-loop
  tadpoles),
  \href{http://theor.jinr.ru/~kalmykov/onshell2/onshell2.html}{ON-SHELL2}
  (on-shell 2-point functions with one mass scale)
\item
  The main source for analytic results are scientific publications. When
  people calculate new master integrals needed for their research, they
  often provide explicit analytic results in the paper itself or put
  them into ancillary files accompanying the preprint. Unfortunately,
  there is no compendium of all calculated integrals that would tell you
  where to find the corresponding expression. The
  \href{https://arxiv.org/abs/1709.01266}{Loopedia} project is an
  attempt to create something like a search engine for loop integrals,
  but its database is still far from being comprehensive.
\item
  Some relevant publications for 2-loop 2-point functions include (this
  list is far from being complete)
  \href{https://arxiv.org/abs/hep-ph/9907431}{arXiv:hep-ph/9907431},
  \href{https://arxiv.org/abs/hep-ph/0202123v2}{arXiv:hep-ph/0202123},
  \href{https://arxiv.org/abs/hep-ph/0307101v1}{hep-ph/0307101}
\end{itemize}

\hypertarget{numerical-results}{%
\subsection{Numerical results}\label{numerical-results}}

\begin{itemize}
\tightlist
\item
  Numerical results are much simpler to obtain and universal libraries
  that can calculate almost any integral (given enough time and
  computing resources) are publicly available. Two prominent examples
  are \href{https://secdec.readthedocs.io/en/stable/}{pySecDec} and
  \href{https://bitbucket.org/feynmanIntegrals/fiesta/src/master/}{FIESTA}
\item
  Apart from that, there are also libraries that cover specific integral
  families and may offer better numerical stability due to the
  corresponding optimizations. Two very useful tools are
  \href{https://sites.pitt.edu/~afreitas/}{TVID2} for the evaluation of
  3-loop 2-point functions with arbitrary masses and
  \href{https://www.niu.edu/spmartin/3VIL/}{3VIL} for the calculation of
  3-loop tadpoles with arbitrary masses.
\end{itemize}
\end{document}
