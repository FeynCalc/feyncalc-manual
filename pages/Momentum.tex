% !TeX program = pdflatex
% !TeX root = Momentum.tex

\documentclass[../FeynCalcManual.tex]{subfiles}
\begin{document}
\hypertarget{momentum}{
\section{Momentum}\label{momentum}\index{Momentum}}

\texttt{Momentum[\allowbreak{}p]} is the head of a four momentum
\texttt{p}.

The internal representation of a \(4\)-dimensional \(p\) is
\texttt{Momentum[\allowbreak{}p]}.

For other than \(4\) dimensions:
\texttt{Momentum[\allowbreak{}p,\ \allowbreak{}dim]}.

\texttt{Momentum[\allowbreak{}p,\ \allowbreak{}4]} simplifies to
\texttt{Momentum[\allowbreak{}p]}.

\subsection{See also}

\hyperlink{toc}{Overview}, \hyperlink{diracgamma}{DiracGamma},
\hyperlink{eps}{Eps}, \hyperlink{lorentzindex}{LorentzIndex},
\hyperlink{momentumexpand}{MomentumExpand}.

\subsection{Examples}

This is a \(4\)-dimensional momentum.

\begin{Shaded}
\begin{Highlighting}[]
\NormalTok{Momentum}\OperatorTok{[}\FunctionTok{p}\OperatorTok{]}
\end{Highlighting}
\end{Shaded}

\begin{dmath*}\breakingcomma
\overline{p}
\end{dmath*}

As an optional second argument the dimension must be specified if it is
different from \(4\).

\begin{Shaded}
\begin{Highlighting}[]
\NormalTok{Momentum}\OperatorTok{[}\FunctionTok{p}\OperatorTok{,} \FunctionTok{D}\OperatorTok{]}
\end{Highlighting}
\end{Shaded}

\begin{dmath*}\breakingcomma
p
\end{dmath*}

The dimension index is suppressed in the output.

\begin{Shaded}
\begin{Highlighting}[]
\NormalTok{Momentum}\OperatorTok{[}\FunctionTok{p}\OperatorTok{,} \FunctionTok{d}\OperatorTok{]}
\end{Highlighting}
\end{Shaded}

\begin{dmath*}\breakingcomma
p
\end{dmath*}

\begin{Shaded}
\begin{Highlighting}[]
\NormalTok{Momentum}\OperatorTok{[}\SpecialCharTok{{-}}\FunctionTok{q}\OperatorTok{]}
\end{Highlighting}
\end{Shaded}

\begin{dmath*}\breakingcomma
-\overline{q}
\end{dmath*}

\begin{Shaded}
\begin{Highlighting}[]
\NormalTok{Momentum}\OperatorTok{[}\SpecialCharTok{{-}}\FunctionTok{q}\OperatorTok{]} \SpecialCharTok{//} \FunctionTok{StandardForm}

\CommentTok{(*{-}Momentum[q]*)}
\end{Highlighting}
\end{Shaded}

\begin{Shaded}
\begin{Highlighting}[]
\NormalTok{ex }\ExtensionTok{=}\NormalTok{ Momentum}\OperatorTok{[}\FunctionTok{p} \SpecialCharTok{{-}} \FunctionTok{q}\OperatorTok{]} \SpecialCharTok{+}\NormalTok{ Momentum}\OperatorTok{[}\DecValTok{2} \FunctionTok{q}\OperatorTok{]}
\end{Highlighting}
\end{Shaded}

\begin{dmath*}\breakingcomma
\left(\overline{p}-\overline{q}\right)+2 \overline{q}
\end{dmath*}

\begin{Shaded}
\begin{Highlighting}[]
\NormalTok{ex }\SpecialCharTok{//} \FunctionTok{StandardForm}

\CommentTok{(*Momentum[p {-} q] + 2 Momentum[q]*)}
\end{Highlighting}
\end{Shaded}

\begin{Shaded}
\begin{Highlighting}[]
\NormalTok{ex }\SpecialCharTok{//}\NormalTok{ MomentumExpand }\SpecialCharTok{//} \FunctionTok{StandardForm}

\CommentTok{(*Momentum[p] + Momentum[q]*)}
\end{Highlighting}
\end{Shaded}

\begin{Shaded}
\begin{Highlighting}[]
\NormalTok{ex }\SpecialCharTok{//}\NormalTok{ MomentumCombine }\SpecialCharTok{//} \FunctionTok{StandardForm}

\CommentTok{(*Momentum[p + q]*)}
\end{Highlighting}
\end{Shaded}

\begin{Shaded}
\begin{Highlighting}[]
\NormalTok{ChangeDimension}\OperatorTok{[}\NormalTok{Momentum}\OperatorTok{[}\FunctionTok{p}\OperatorTok{],} \FunctionTok{d}\OperatorTok{]} \SpecialCharTok{//} \FunctionTok{StandardForm}

\CommentTok{(*Momentum[p, d]*)}
\end{Highlighting}
\end{Shaded}

\end{document}
