% !TeX program = pdflatex
% !TeX root = MomentumCombine.tex

\documentclass[../FeynCalcManual.tex]{subfiles}
\begin{document}
\begin{Shaded}
\begin{Highlighting}[]
 
\end{Highlighting}
\end{Shaded}

\hypertarget{momentumcombine}{
\section{MomentumCombine}\label{momentumcombine}\index{MomentumCombine}}

\texttt{MomentumCombine[\allowbreak{}expr]} is the inverse operation to
\texttt{MomentumExpand} and \texttt{ExpandScalarProduct}.
\texttt{MomentumCombine} combines also \texttt{Pair}s.

\subsection{See also}

\hyperlink{toc}{Overview},
\hyperlink{expandscalarproduct}{ExpandScalarProduct},
\hyperlink{momentum}{Momentum},
\hyperlink{momentumexpand}{MomentumExpand}.

\subsection{Examples}

\begin{Shaded}
\begin{Highlighting}[]
\NormalTok{Momentum}\OperatorTok{[}\FunctionTok{p}\OperatorTok{]} \SpecialCharTok{{-}} \DecValTok{2}\NormalTok{ Momentum}\OperatorTok{[}\FunctionTok{q}\OperatorTok{]} \SpecialCharTok{//}\NormalTok{ MomentumCombine }\SpecialCharTok{//} \FunctionTok{StandardForm}

\CommentTok{(*Momentum[p {-} 2 q]*)}
\end{Highlighting}
\end{Shaded}

\begin{Shaded}
\begin{Highlighting}[]
\NormalTok{FV}\OperatorTok{[}\FunctionTok{p}\OperatorTok{,} \SpecialCharTok{\textbackslash{}}\OperatorTok{[}\NormalTok{Mu}\OperatorTok{]]} \SpecialCharTok{+} \DecValTok{2}\NormalTok{ FV}\OperatorTok{[}\FunctionTok{q}\OperatorTok{,} \SpecialCharTok{\textbackslash{}}\OperatorTok{[}\NormalTok{Mu}\OperatorTok{]]} 
 
\NormalTok{ex }\ExtensionTok{=}\NormalTok{ MomentumCombine}\OperatorTok{[}\SpecialCharTok{\%}\OperatorTok{]}
\end{Highlighting}
\end{Shaded}

\begin{dmath*}\breakingcomma
\overline{p}^{\mu }+2 \overline{q}^{\mu }
\end{dmath*}

\begin{dmath*}\breakingcomma
\left(\overline{p}+2 \overline{q}\right)^{\mu }
\end{dmath*}

\begin{Shaded}
\begin{Highlighting}[]
\NormalTok{ex }\SpecialCharTok{//} \FunctionTok{StandardForm}

\CommentTok{(*Pair[LorentzIndex[\textbackslash{}[Mu]], Momentum[p + 2 q]]*)}
\end{Highlighting}
\end{Shaded}

\begin{Shaded}
\begin{Highlighting}[]
\NormalTok{ex }\SpecialCharTok{//}\NormalTok{ ExpandScalarProduct}
\end{Highlighting}
\end{Shaded}

\begin{dmath*}\breakingcomma
\overline{p}^{\mu }+2 \overline{q}^{\mu }
\end{dmath*}

\begin{Shaded}
\begin{Highlighting}[]
\DecValTok{3}\NormalTok{ Pair}\OperatorTok{[}\NormalTok{LorentzIndex}\OperatorTok{[}\SpecialCharTok{\textbackslash{}}\OperatorTok{[}\NormalTok{Mu}\OperatorTok{]],}\NormalTok{ Momentum}\OperatorTok{[}\FunctionTok{p}\OperatorTok{]]} \SpecialCharTok{+} \DecValTok{2}\NormalTok{ Pair}\OperatorTok{[}\NormalTok{LorentzIndex}\OperatorTok{[}\SpecialCharTok{\textbackslash{}}\OperatorTok{[}\NormalTok{Mu}\OperatorTok{]],}\NormalTok{ Momentum}\OperatorTok{[}\FunctionTok{q}\OperatorTok{]]} 
 
\NormalTok{ex }\ExtensionTok{=}\NormalTok{ MomentumCombine}\OperatorTok{[}\SpecialCharTok{\%}\OperatorTok{]}
\end{Highlighting}
\end{Shaded}

\begin{dmath*}\breakingcomma
3 \overline{p}^{\mu }+2 \overline{q}^{\mu }
\end{dmath*}

\begin{dmath*}\breakingcomma
\left(3 \overline{p}+2 \overline{q}\right)^{\mu }
\end{dmath*}

\begin{Shaded}
\begin{Highlighting}[]
\NormalTok{ex }\SpecialCharTok{//} \FunctionTok{StandardForm}

\CommentTok{(*Pair[LorentzIndex[\textbackslash{}[Mu]], Momentum[3 p + 2 q]]*)}
\end{Highlighting}
\end{Shaded}

In some cases one might need a better control over the types of
expressions getting combined. For example, the following expression will
not be combined by default, since the coefficients of scalar products
are not numbers

\begin{Shaded}
\begin{Highlighting}[]
\NormalTok{DataType}\OperatorTok{[}\NormalTok{a1}\OperatorTok{,}\NormalTok{ FCVariable}\OperatorTok{]} \ExtensionTok{=} \ConstantTok{True}\NormalTok{;}
\NormalTok{DataType}\OperatorTok{[}\NormalTok{a2}\OperatorTok{,}\NormalTok{ FCVariable}\OperatorTok{]} \ExtensionTok{=} \ConstantTok{True}\NormalTok{;}
\end{Highlighting}
\end{Shaded}

\begin{Shaded}
\begin{Highlighting}[]
\NormalTok{ex }\ExtensionTok{=}\NormalTok{ SPD}\OperatorTok{[}\NormalTok{a1 }\FunctionTok{p}\OperatorTok{,} \FunctionTok{n}\OperatorTok{]} \SpecialCharTok{+}\NormalTok{ SPD}\OperatorTok{[}\NormalTok{a2 }\FunctionTok{p}\OperatorTok{,}\NormalTok{ nb}\OperatorTok{]}
\end{Highlighting}
\end{Shaded}

\begin{dmath*}\breakingcomma
\text{a1} (n\cdot p)+\text{a2} (\text{nb}\cdot p)
\end{dmath*}

\begin{Shaded}
\begin{Highlighting}[]
\NormalTok{MomentumCombine}\OperatorTok{[}\NormalTok{ex}\OperatorTok{]}
\end{Highlighting}
\end{Shaded}

\begin{dmath*}\breakingcomma
\text{a1} (n\cdot p)+\text{a2} (\text{nb}\cdot p)
\end{dmath*}

Setting the option \texttt{NumberQ} to \texttt{False} we can still
achieve the desired form

\begin{Shaded}
\begin{Highlighting}[]
\NormalTok{MomentumCombine}\OperatorTok{[}\NormalTok{ex}\OperatorTok{,} \FunctionTok{NumberQ} \OtherTok{{-}\textgreater{}} \ConstantTok{False}\OperatorTok{]}
\end{Highlighting}
\end{Shaded}

\begin{dmath*}\breakingcomma
(\text{a1} n+\text{a2} \;\text{nb})\cdot p
\end{dmath*}

However, in the following case combing \(p^2\) with the other two scalar
products is not useful

\begin{Shaded}
\begin{Highlighting}[]
\NormalTok{ex }\ExtensionTok{=}\NormalTok{ SPD}\OperatorTok{[}\FunctionTok{p}\OperatorTok{]} \SpecialCharTok{+}\NormalTok{ SPD}\OperatorTok{[}\NormalTok{a1 }\FunctionTok{p}\OperatorTok{,} \FunctionTok{n}\OperatorTok{]} \SpecialCharTok{+}\NormalTok{ SPD}\OperatorTok{[}\NormalTok{a2 }\FunctionTok{p}\OperatorTok{,}\NormalTok{ nb}\OperatorTok{]}
\end{Highlighting}
\end{Shaded}

\begin{dmath*}\breakingcomma
\text{a1} (n\cdot p)+\text{a2} (\text{nb}\cdot p)+p^2
\end{dmath*}

\begin{Shaded}
\begin{Highlighting}[]
\NormalTok{MomentumCombine}\OperatorTok{[}\NormalTok{ex}\OperatorTok{,} \FunctionTok{NumberQ} \OtherTok{{-}\textgreater{}} \ConstantTok{False}\OperatorTok{]}
\end{Highlighting}
\end{Shaded}

\begin{dmath*}\breakingcomma
p\cdot (\text{a1} n+\text{a2} \;\text{nb}+p)
\end{dmath*}

To prevent this from happening there is a somewhat hidden option
\texttt{"Quadratic"} that can be set to \texttt{False}

\begin{Shaded}
\begin{Highlighting}[]
\NormalTok{MomentumCombine}\OperatorTok{[}\NormalTok{ex}\OperatorTok{,} \FunctionTok{NumberQ} \OtherTok{{-}\textgreater{}} \ConstantTok{False}\OperatorTok{,} \StringTok{"Quadratic"} \OtherTok{{-}\textgreater{}} \ConstantTok{False}\OperatorTok{]}
\end{Highlighting}
\end{Shaded}

\begin{dmath*}\breakingcomma
(\text{a1} n+\text{a2} \;\text{nb})\cdot p+p^2
\end{dmath*}

\begin{Shaded}
\begin{Highlighting}[]
\NormalTok{ex }\ExtensionTok{=}\NormalTok{ SPD}\OperatorTok{[}\FunctionTok{p}\OperatorTok{]} \SpecialCharTok{+}\NormalTok{ SPD}\OperatorTok{[}\NormalTok{a1 }\FunctionTok{p}\OperatorTok{,} \FunctionTok{n}\OperatorTok{]} \SpecialCharTok{+}\NormalTok{ SPD}\OperatorTok{[}\NormalTok{a2 }\FunctionTok{p}\OperatorTok{,}\NormalTok{ nb}\OperatorTok{]} \SpecialCharTok{+}\NormalTok{ SPD}\OperatorTok{[}\FunctionTok{p}\OperatorTok{,} \FunctionTok{l}\OperatorTok{]} \SpecialCharTok{+}\NormalTok{ SPD}\OperatorTok{[}\FunctionTok{p}\OperatorTok{,} \FunctionTok{k}\OperatorTok{]}
\end{Highlighting}
\end{Shaded}

\begin{dmath*}\breakingcomma
\text{a1} (n\cdot p)+\text{a2} (\text{nb}\cdot p)+k\cdot p+l\cdot p+p^2
\end{dmath*}

In this case we we would like to prevent the scalar products involving
\texttt{l} and \texttt{k} from being combined with the rest. To that end
we need to use the option \texttt{Except}

\begin{Shaded}
\begin{Highlighting}[]
\NormalTok{MomentumCombine}\OperatorTok{[}\NormalTok{ex}\OperatorTok{,} \FunctionTok{NumberQ} \OtherTok{{-}\textgreater{}} \ConstantTok{False}\OperatorTok{,} \StringTok{"Quadratic"} \OtherTok{{-}\textgreater{}} \ConstantTok{False}\OperatorTok{,} \FunctionTok{Except} \OtherTok{{-}\textgreater{}} \OperatorTok{\{}\FunctionTok{k}\OperatorTok{,} \FunctionTok{l}\OperatorTok{\}]}
\end{Highlighting}
\end{Shaded}

\begin{dmath*}\breakingcomma
(\text{a1} n+\text{a2} \;\text{nb})\cdot p+k\cdot p+l\cdot p+p^2
\end{dmath*}
\end{document}
