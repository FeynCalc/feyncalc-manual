% !TeX program = pdflatex
% !TeX root = MomentumCombine.tex

\documentclass[../FeynCalcManual.tex]{subfiles}
\begin{document}
\hypertarget{momentumcombine}{
\section{MomentumCombine}\label{momentumcombine}\index{MomentumCombine}}

\texttt{MomentumCombine[\allowbreak{}expr]} is the inverse operation to
\texttt{MomentumExpand} and \texttt{ExpandScalarProduct}.
\texttt{MomentumCombine} combines also \texttt{Pair}s.

\subsection{See also}

\hyperlink{toc}{Overview},
\hyperlink{expandscalarproduct}{ExpandScalarProduct},
\hyperlink{momentum}{Momentum},
\hyperlink{momentumexpand}{MomentumExpand}.

\subsection{Examples}

\begin{Shaded}
\begin{Highlighting}[]
\NormalTok{Momentum}\OperatorTok{[}\FunctionTok{p}\OperatorTok{]} \SpecialCharTok{{-}} \DecValTok{2}\NormalTok{ Momentum}\OperatorTok{[}\FunctionTok{q}\OperatorTok{]} \SpecialCharTok{//}\NormalTok{ MomentumCombine }\SpecialCharTok{//} \FunctionTok{StandardForm}

\CommentTok{(*Momentum[p {-} 2 q]*)}
\end{Highlighting}
\end{Shaded}

\begin{Shaded}
\begin{Highlighting}[]
\NormalTok{FV}\OperatorTok{[}\FunctionTok{p}\OperatorTok{,} \SpecialCharTok{\textbackslash{}}\OperatorTok{[}\NormalTok{Mu}\OperatorTok{]]} \SpecialCharTok{+} \DecValTok{2}\NormalTok{ FV}\OperatorTok{[}\FunctionTok{q}\OperatorTok{,} \SpecialCharTok{\textbackslash{}}\OperatorTok{[}\NormalTok{Mu}\OperatorTok{]]} 
 
\NormalTok{ex }\ExtensionTok{=}\NormalTok{ MomentumCombine}\OperatorTok{[}\SpecialCharTok{\%}\OperatorTok{]}
\end{Highlighting}
\end{Shaded}

\begin{dmath*}\breakingcomma
\overline{p}^{\mu }+2 \overline{q}^{\mu }
\end{dmath*}

\begin{dmath*}\breakingcomma
\left(\overline{p}+2 \overline{q}\right)^{\mu }
\end{dmath*}

\begin{Shaded}
\begin{Highlighting}[]
\NormalTok{ex }\SpecialCharTok{//} \FunctionTok{StandardForm}

\CommentTok{(*Pair[LorentzIndex[\textbackslash{}[Mu]], Momentum[p + 2 q]]*)}
\end{Highlighting}
\end{Shaded}

\begin{Shaded}
\begin{Highlighting}[]
\NormalTok{ex }\SpecialCharTok{//}\NormalTok{ ExpandScalarProduct}
\end{Highlighting}
\end{Shaded}

\begin{dmath*}\breakingcomma
\overline{p}^{\mu }+2 \overline{q}^{\mu }
\end{dmath*}

\begin{Shaded}
\begin{Highlighting}[]
\DecValTok{3}\NormalTok{ Pair}\OperatorTok{[}\NormalTok{LorentzIndex}\OperatorTok{[}\SpecialCharTok{\textbackslash{}}\OperatorTok{[}\NormalTok{Mu}\OperatorTok{]],}\NormalTok{ Momentum}\OperatorTok{[}\FunctionTok{p}\OperatorTok{]]} \SpecialCharTok{+} \DecValTok{2}\NormalTok{ Pair}\OperatorTok{[}\NormalTok{LorentzIndex}\OperatorTok{[}\SpecialCharTok{\textbackslash{}}\OperatorTok{[}\NormalTok{Mu}\OperatorTok{]],}\NormalTok{ Momentum}\OperatorTok{[}\FunctionTok{q}\OperatorTok{]]} 
 
\NormalTok{ex }\ExtensionTok{=}\NormalTok{ MomentumCombine}\OperatorTok{[}\SpecialCharTok{\%}\OperatorTok{]}
\end{Highlighting}
\end{Shaded}

\begin{dmath*}\breakingcomma
3 \overline{p}^{\mu }+2 \overline{q}^{\mu }
\end{dmath*}

\begin{dmath*}\breakingcomma
\left(3 \overline{p}+2 \overline{q}\right)^{\mu }
\end{dmath*}

\begin{Shaded}
\begin{Highlighting}[]
\NormalTok{ex }\SpecialCharTok{//} \FunctionTok{StandardForm}

\CommentTok{(*Pair[LorentzIndex[\textbackslash{}[Mu]], Momentum[3 p + 2 q]]*)}
\end{Highlighting}
\end{Shaded}

\end{document}
