% !TeX program = pdflatex
% !TeX root = MomentumCombine.tex

\documentclass[../FeynCalcManual.tex]{subfiles}
\begin{document}
\hypertarget{momentumcombine}{
\section{MomentumCombine}\label{momentumcombine}\index{MomentumCombine}}

\texttt{MomentumCombine[\allowbreak{}expr]} is the inverse operation to
\texttt{MomentumExpand} and \texttt{ExpandScalarProduct}.
\texttt{MomentumCombine} combines also \texttt{Pair}s. Notice, that
\texttt{MomentumCombine} cannot complete squares. It can, however, bring
expressions containing scalar products to a suitable form that allows
for a square completion using other means.

This function offers multiple options.

The option \texttt{NumberQ} (default is \texttt{True}) specifies whether
one should only merge quantities with numerical prefactors or not.
Setting it to \texttt{False} allows for symbolic prefactors.

Setting the option \texttt{"Quadratic"} to \texttt{False} (default is
\texttt{True}) effectively means that momenta squared will not be
combined with anything else.

With the option \texttt{"ExcludeScalarProducts"} we can ensure that
scalar products containing any of the momenta listed are not merged with
anything else. So \texttt{a.x + a.y} can be merged either if \texttt{a}
contains no such momenta, or if both \texttt{x} and \texttt{y} are free
of them.

The option \texttt{Except} forbids merging the listed momenta with
anything else. It is much more restrictive than
\texttt{"ExcludeScalarProducts"} that allows for merging terms linear in
the listed momenta.

The option \texttt{Select} allows for gathering all terms linear in the
given momenta before applying any other combining rules.

\subsection{See also}

\hyperlink{toc}{Overview},
\hyperlink{expandscalarproduct}{ExpandScalarProduct},
\hyperlink{momentum}{Momentum},
\hyperlink{momentumexpand}{MomentumExpand}.

\subsection{Examples}

\begin{Shaded}
\begin{Highlighting}[]
\NormalTok{Momentum}\OperatorTok{[}\FunctionTok{p}\OperatorTok{]} \SpecialCharTok{{-}} \DecValTok{2}\NormalTok{ Momentum}\OperatorTok{[}\FunctionTok{q}\OperatorTok{]} \SpecialCharTok{//}\NormalTok{ MomentumCombine }\SpecialCharTok{//} \FunctionTok{StandardForm}

\CommentTok{(*Momentum[p {-} 2 q]*)}
\end{Highlighting}
\end{Shaded}

\begin{Shaded}
\begin{Highlighting}[]
\NormalTok{FV}\OperatorTok{[}\FunctionTok{p}\OperatorTok{,} \SpecialCharTok{\textbackslash{}}\OperatorTok{[}\NormalTok{Mu}\OperatorTok{]]} \SpecialCharTok{+} \DecValTok{2}\NormalTok{ FV}\OperatorTok{[}\FunctionTok{q}\OperatorTok{,} \SpecialCharTok{\textbackslash{}}\OperatorTok{[}\NormalTok{Mu}\OperatorTok{]]} 
 
\NormalTok{ex }\ExtensionTok{=}\NormalTok{ MomentumCombine}\OperatorTok{[}\SpecialCharTok{\%}\OperatorTok{]}
\end{Highlighting}
\end{Shaded}

\begin{dmath*}\breakingcomma
\overline{p}^{\mu }+2 \overline{q}^{\mu }
\end{dmath*}

\begin{dmath*}\breakingcomma
\left(\overline{p}+2 \overline{q}\right)^{\mu }
\end{dmath*}

\begin{Shaded}
\begin{Highlighting}[]
\NormalTok{ex }\SpecialCharTok{//} \FunctionTok{StandardForm}

\CommentTok{(*Pair[LorentzIndex[\textbackslash{}[Mu]], Momentum[p + 2 q]]*)}
\end{Highlighting}
\end{Shaded}

\begin{Shaded}
\begin{Highlighting}[]
\NormalTok{ex }\SpecialCharTok{//}\NormalTok{ ExpandScalarProduct}
\end{Highlighting}
\end{Shaded}

\begin{dmath*}\breakingcomma
\overline{p}^{\mu }+2 \overline{q}^{\mu }
\end{dmath*}

\begin{Shaded}
\begin{Highlighting}[]
\DecValTok{3}\NormalTok{ Pair}\OperatorTok{[}\NormalTok{LorentzIndex}\OperatorTok{[}\SpecialCharTok{\textbackslash{}}\OperatorTok{[}\NormalTok{Mu}\OperatorTok{]],}\NormalTok{ Momentum}\OperatorTok{[}\FunctionTok{p}\OperatorTok{]]} \SpecialCharTok{+} \DecValTok{2}\NormalTok{ Pair}\OperatorTok{[}\NormalTok{LorentzIndex}\OperatorTok{[}\SpecialCharTok{\textbackslash{}}\OperatorTok{[}\NormalTok{Mu}\OperatorTok{]],}\NormalTok{ Momentum}\OperatorTok{[}\FunctionTok{q}\OperatorTok{]]} 
 
\NormalTok{ex }\ExtensionTok{=}\NormalTok{ MomentumCombine}\OperatorTok{[}\SpecialCharTok{\%}\OperatorTok{]}
\end{Highlighting}
\end{Shaded}

\begin{dmath*}\breakingcomma
3 \overline{p}^{\mu }+2 \overline{q}^{\mu }
\end{dmath*}

\begin{dmath*}\breakingcomma
\left(3 \overline{p}+2 \overline{q}\right)^{\mu }
\end{dmath*}

\begin{Shaded}
\begin{Highlighting}[]
\NormalTok{ex }\SpecialCharTok{//} \FunctionTok{StandardForm}

\CommentTok{(*Pair[LorentzIndex[\textbackslash{}[Mu]], Momentum[3 p + 2 q]]*)}
\end{Highlighting}
\end{Shaded}

In some cases one might need a better control over the types of
expressions getting combined. For example, the following expression will
not be combined by default, since the coefficients of scalar products
are not numbers

\begin{Shaded}
\begin{Highlighting}[]
\NormalTok{DataType}\OperatorTok{[}\NormalTok{a1}\OperatorTok{,}\NormalTok{ FCVariable}\OperatorTok{]} \ExtensionTok{=} \ConstantTok{True}\NormalTok{;}
\NormalTok{DataType}\OperatorTok{[}\NormalTok{a2}\OperatorTok{,}\NormalTok{ FCVariable}\OperatorTok{]} \ExtensionTok{=} \ConstantTok{True}\NormalTok{;}
\end{Highlighting}
\end{Shaded}

\begin{Shaded}
\begin{Highlighting}[]
\NormalTok{ex }\ExtensionTok{=}\NormalTok{ SPD}\OperatorTok{[}\NormalTok{a1 }\FunctionTok{p}\OperatorTok{,} \FunctionTok{n}\OperatorTok{]} \SpecialCharTok{+}\NormalTok{ SPD}\OperatorTok{[}\NormalTok{a2 }\FunctionTok{p}\OperatorTok{,}\NormalTok{ nb}\OperatorTok{]}
\end{Highlighting}
\end{Shaded}

\begin{dmath*}\breakingcomma
\text{a1} (n\cdot p)+\text{a2} (\text{nb}\cdot p)
\end{dmath*}

\begin{Shaded}
\begin{Highlighting}[]
\NormalTok{MomentumCombine}\OperatorTok{[}\NormalTok{ex}\OperatorTok{]}
\end{Highlighting}
\end{Shaded}

\begin{dmath*}\breakingcomma
\text{a1} (n\cdot p)+\text{a2} (\text{nb}\cdot p)
\end{dmath*}

Setting the option \texttt{NumberQ} to \texttt{False} we can still
achieve the desired form

\begin{Shaded}
\begin{Highlighting}[]
\NormalTok{MomentumCombine}\OperatorTok{[}\NormalTok{ex}\OperatorTok{,} \FunctionTok{NumberQ} \OtherTok{{-}\textgreater{}} \ConstantTok{False}\OperatorTok{]}
\end{Highlighting}
\end{Shaded}

\begin{dmath*}\breakingcomma
(\text{a1} n+\text{a2} \;\text{nb})\cdot p
\end{dmath*}

However, in the following case combing \(p^2\) with the other two scalar
products is not useful

\begin{Shaded}
\begin{Highlighting}[]
\NormalTok{ex }\ExtensionTok{=}\NormalTok{ SPD}\OperatorTok{[}\FunctionTok{p}\OperatorTok{]} \SpecialCharTok{+}\NormalTok{ SPD}\OperatorTok{[}\NormalTok{a1 }\FunctionTok{p}\OperatorTok{,} \FunctionTok{n}\OperatorTok{]} \SpecialCharTok{+}\NormalTok{ SPD}\OperatorTok{[}\NormalTok{a2 }\FunctionTok{p}\OperatorTok{,}\NormalTok{ nb}\OperatorTok{]}
\end{Highlighting}
\end{Shaded}

\begin{dmath*}\breakingcomma
\text{a1} (n\cdot p)+\text{a2} (\text{nb}\cdot p)+p^2
\end{dmath*}

\begin{Shaded}
\begin{Highlighting}[]
\NormalTok{MomentumCombine}\OperatorTok{[}\NormalTok{ex}\OperatorTok{,} \FunctionTok{NumberQ} \OtherTok{{-}\textgreater{}} \ConstantTok{False}\OperatorTok{]}
\end{Highlighting}
\end{Shaded}

\begin{dmath*}\breakingcomma
p\cdot (\text{a1} n+\text{a2} \;\text{nb}+p)
\end{dmath*}

To prevent this from happening there is a somewhat hidden option
\texttt{"Quadratic"} that can be set to \texttt{False}

\begin{Shaded}
\begin{Highlighting}[]
\NormalTok{MomentumCombine}\OperatorTok{[}\NormalTok{ex}\OperatorTok{,} \FunctionTok{NumberQ} \OtherTok{{-}\textgreater{}} \ConstantTok{False}\OperatorTok{,} \StringTok{"Quadratic"} \OtherTok{{-}\textgreater{}} \ConstantTok{False}\OperatorTok{]}
\end{Highlighting}
\end{Shaded}

\begin{dmath*}\breakingcomma
(\text{a1} n+\text{a2} \;\text{nb})\cdot p+p^2
\end{dmath*}

\begin{Shaded}
\begin{Highlighting}[]
\NormalTok{ex }\ExtensionTok{=}\NormalTok{ SPD}\OperatorTok{[}\FunctionTok{p}\OperatorTok{]} \SpecialCharTok{+}\NormalTok{ SPD}\OperatorTok{[}\NormalTok{a1 }\FunctionTok{p}\OperatorTok{,} \FunctionTok{n}\OperatorTok{]} \SpecialCharTok{+}\NormalTok{ SPD}\OperatorTok{[}\NormalTok{a2 }\FunctionTok{p}\OperatorTok{,}\NormalTok{ nb}\OperatorTok{]} \SpecialCharTok{+}\NormalTok{ SPD}\OperatorTok{[}\FunctionTok{p}\OperatorTok{,} \FunctionTok{l}\OperatorTok{]} \SpecialCharTok{+}\NormalTok{ SPD}\OperatorTok{[}\FunctionTok{p}\OperatorTok{,} \FunctionTok{k}\OperatorTok{]}
\end{Highlighting}
\end{Shaded}

\begin{dmath*}\breakingcomma
\text{a1} (n\cdot p)+\text{a2} (\text{nb}\cdot p)+k\cdot p+l\cdot p+p^2
\end{dmath*}

In this case we we would like to prevent the scalar products involving
\texttt{l} and \texttt{k} from being combined with the rest. To that end
we need to use the option \texttt{Except}

\begin{Shaded}
\begin{Highlighting}[]
\NormalTok{MomentumCombine}\OperatorTok{[}\NormalTok{ex}\OperatorTok{,} \FunctionTok{NumberQ} \OtherTok{{-}\textgreater{}} \ConstantTok{False}\OperatorTok{,} \StringTok{"Quadratic"} \OtherTok{{-}\textgreater{}} \ConstantTok{False}\OperatorTok{,} \FunctionTok{Except} \OtherTok{{-}\textgreater{}} \OperatorTok{\{}\FunctionTok{k}\OperatorTok{,} \FunctionTok{l}\OperatorTok{\}]}
\end{Highlighting}
\end{Shaded}

\begin{dmath*}\breakingcomma
(\text{a1} n+\text{a2} \;\text{nb})\cdot p+k\cdot p+l\cdot p+p^2
\end{dmath*}

Suppose that we have an expression that can be written as a square. To
achieve the desired combination of momenta we need to

\begin{Shaded}
\begin{Highlighting}[]
\NormalTok{(DataType}\OperatorTok{[}\NormalTok{\#}\OperatorTok{,}\NormalTok{ FCVariable}\OperatorTok{]} \ExtensionTok{=} \ConstantTok{True}\NormalTok{) \& }\SpecialCharTok{/}\NormalTok{@ }\OperatorTok{\{}\NormalTok{gkin}\OperatorTok{,}\NormalTok{ meta}\OperatorTok{,}\NormalTok{ u0b}\OperatorTok{\}}\NormalTok{;}
\end{Highlighting}
\end{Shaded}

\begin{Shaded}
\begin{Highlighting}[]
\NormalTok{ex }\ExtensionTok{=}\NormalTok{ SPD}\OperatorTok{[}\NormalTok{k1}\OperatorTok{,}\NormalTok{ k1}\OperatorTok{]} \SpecialCharTok{{-}} \DecValTok{2}\NormalTok{ SPD}\OperatorTok{[}\NormalTok{k1}\OperatorTok{,}\NormalTok{ k2}\OperatorTok{]} \SpecialCharTok{+} \DecValTok{2}\NormalTok{ gkin meta SPD}\OperatorTok{[}\NormalTok{k1}\OperatorTok{,} \FunctionTok{n}\OperatorTok{]} \SpecialCharTok{{-}} \DecValTok{2}\NormalTok{ gkin meta u0b SPD}\OperatorTok{[}\NormalTok{k1}\OperatorTok{,} \FunctionTok{n}\OperatorTok{]} \SpecialCharTok{{-}}\NormalTok{ meta u0b SPD}\OperatorTok{[}\NormalTok{k1}\OperatorTok{,}\NormalTok{ nb}\OperatorTok{]} \SpecialCharTok{+} 
\NormalTok{   SPD}\OperatorTok{[}\NormalTok{k2}\OperatorTok{,}\NormalTok{ k2}\OperatorTok{]} \SpecialCharTok{{-}} \DecValTok{2}\NormalTok{ gkin meta SPD}\OperatorTok{[}\NormalTok{k2}\OperatorTok{,} \FunctionTok{n}\OperatorTok{]} \SpecialCharTok{+} \DecValTok{2}\NormalTok{ gkin meta u0b SPD}\OperatorTok{[}\NormalTok{k2}\OperatorTok{,} \FunctionTok{n}\OperatorTok{]} \SpecialCharTok{+}\NormalTok{meta u0b SPD}\OperatorTok{[}\NormalTok{k2}\OperatorTok{,}\NormalTok{ nb}\OperatorTok{]}
\end{Highlighting}
\end{Shaded}

\begin{dmath*}\breakingcomma
-2 \;\text{gkin} \;\text{meta} \;\text{u0b} (\text{k1}\cdot n)+2 \;\text{gkin} \;\text{meta} (\text{k1}\cdot n)+2 \;\text{gkin} \;\text{meta} \;\text{u0b} (\text{k2}\cdot n)-2 \;\text{gkin} \;\text{meta} (\text{k2}\cdot n)-2 (\text{k1}\cdot \;\text{k2})-\text{meta} \;\text{u0b} (\text{k1}\cdot \;\text{nb})+\text{k1}^2+\text{meta} \;\text{u0b} (\text{k2}\cdot \;\text{nb})+\text{k2}^2
\end{dmath*}

The naive application of \texttt{MomentumCombine} doesn't return
anything useful

\begin{Shaded}
\begin{Highlighting}[]
\NormalTok{MomentumCombine}\OperatorTok{[}\NormalTok{ex}\OperatorTok{]}
\end{Highlighting}
\end{Shaded}

\begin{dmath*}\breakingcomma
\text{meta} \;\text{u0b} ((\text{k2}-\text{k1})\cdot \;\text{nb})+\text{k1}\cdot (\text{k1}-2 \;\text{k2})-2 \;\text{gkin} \;\text{meta} \;\text{u0b} (\text{k1}\cdot n)+2 \;\text{gkin} \;\text{meta} (\text{k1}\cdot n)+2 \;\text{gkin} \;\text{meta} \;\text{u0b} (\text{k2}\cdot n)-2 \;\text{gkin} \;\text{meta} (\text{k2}\cdot n)+\text{k2}^2
\end{dmath*}

Here we actually want to gather terms linear in \texttt{k1} and
\texttt{k2}first before trying to combine them together. To that aim we
can use the option \texttt{Select}. Employing the options
\texttt{"Quadratic"} and \texttt{"ExcludeScalarProducts"} we can prevent
\texttt{k1} and \texttt{k2} from getting combined with anything
containing those momenta. Furthermore, we enable symbolical prefactor by
setting \texttt{NumberQ} to false

\begin{Shaded}
\begin{Highlighting}[]
\NormalTok{MomentumCombine}\OperatorTok{[}\NormalTok{ex}\OperatorTok{,} \FunctionTok{Select} \OtherTok{{-}\textgreater{}} \OperatorTok{\{}\NormalTok{k1}\OperatorTok{,}\NormalTok{ k2}\OperatorTok{\},} \StringTok{"Quadratic"} \OtherTok{{-}\textgreater{}} \ConstantTok{False}\OperatorTok{,} \StringTok{"ExcludeScalarProducts"} \OtherTok{{-}\textgreater{}} \OperatorTok{\{}\NormalTok{k1}\OperatorTok{,}\NormalTok{ k2}\OperatorTok{\},} \FunctionTok{NumberQ} \OtherTok{{-}\textgreater{}} \ConstantTok{False}\OperatorTok{]}
\end{Highlighting}
\end{Shaded}

\begin{dmath*}\breakingcomma
\text{k1}\cdot (-2 \;\text{gkin} \;\text{meta} n \;\text{u0b}+2 \;\text{gkin} \;\text{meta} n-\text{meta} \;\text{nb} \;\text{u0b})+\text{k2}\cdot (2 \;\text{gkin} \;\text{meta} n \;\text{u0b}-2 \;\text{gkin} \;\text{meta} n+\text{meta} \;\text{nb} \;\text{u0b})-2 (\text{k1}\cdot \;\text{k2})+\text{k1}^2+\text{k2}^2
\end{dmath*}

This result looks very good, but \texttt{k1} and \texttt{k2} were not
combined because they are contracted to long linear combinations of
4-momenta that were not properly factorized. The option
\texttt{Factoring} solves this issue

\begin{Shaded}
\begin{Highlighting}[]
\NormalTok{res }\ExtensionTok{=}\NormalTok{ MomentumCombine}\OperatorTok{[}\NormalTok{ex}\OperatorTok{,} \FunctionTok{Select} \OtherTok{{-}\textgreater{}} \OperatorTok{\{}\NormalTok{k1}\OperatorTok{,}\NormalTok{ k2}\OperatorTok{\},} \StringTok{"Quadratic"} \OtherTok{{-}\textgreater{}} \ConstantTok{False}\OperatorTok{,} \StringTok{"ExcludeScalarProducts"} \OtherTok{{-}\textgreater{}} \OperatorTok{\{}\NormalTok{k1}\OperatorTok{,}\NormalTok{ k2}\OperatorTok{\},} \FunctionTok{NumberQ} \OtherTok{{-}\textgreater{}} \ConstantTok{False}\OperatorTok{,}\NormalTok{ Factoring }\OtherTok{{-}\textgreater{}}\NormalTok{ Factor2}\OperatorTok{]}
\end{Highlighting}
\end{Shaded}

\begin{dmath*}\breakingcomma
\text{meta} ((\text{k1}-\text{k2})\cdot (-2 \;\text{gkin} n \;\text{u0b}+2 \;\text{gkin} n-\text{nb} \;\text{u0b}))-2 (\text{k1}\cdot \;\text{k2})+\text{k1}^2+\text{k2}^2
\end{dmath*}
\end{document}
