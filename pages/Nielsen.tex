% !TeX program = pdflatex
% !TeX root = Nielsen.tex

\documentclass[../FeynCalcManual.tex]{subfiles}
\begin{document}
\hypertarget{nielsen}{
\section{Nielsen}\label{nielsen}\index{Nielsen}}

\texttt{Nielsen[\allowbreak{}i,\ \allowbreak{}j,\ \allowbreak{}x]}
denotes Nielsen's polylogarithm.

\subsection{See also}

\hyperlink{toc}{Overview}, \hyperlink{simplifypolylog}{SimplifyPolyLog}.

\subsection{Examples}

\begin{Shaded}
\begin{Highlighting}[]
\NormalTok{Nielsen}\OperatorTok{[}\DecValTok{1}\OperatorTok{,} \DecValTok{2}\OperatorTok{,} \FunctionTok{x}\OperatorTok{]}
\end{Highlighting}
\end{Shaded}

\begin{dmath*}\breakingcomma
S_{12}(x)
\end{dmath*}

Numerical evaluation is done via
\texttt{N[\allowbreak{}Nielsen[\allowbreak{}n_,\ \allowbreak{}p_,\ \allowbreak{}x_]] := (-1)^(n+p-1)/(n-1)!/p! NIntegrate[\allowbreak{}Log[\allowbreak{}1-x t]^p Log[\allowbreak{}t]^(n-1)/t,\ \allowbreak{}\{\allowbreak{}t,\ \allowbreak{}0,\ \allowbreak{}1\}]}

\begin{Shaded}
\begin{Highlighting}[]
\FunctionTok{N}\OperatorTok{[}\NormalTok{Nielsen}\OperatorTok{[}\DecValTok{1}\OperatorTok{,} \DecValTok{2}\OperatorTok{,}\NormalTok{ .}\DecValTok{45}\OperatorTok{]]}
\end{Highlighting}
\end{Shaded}

\begin{dmath*}\breakingcomma
0.0728716
\end{dmath*}

Some special values are built in.

\begin{Shaded}
\begin{Highlighting}[]
\OperatorTok{\{}\NormalTok{Nielsen}\OperatorTok{[}\DecValTok{1}\OperatorTok{,} \DecValTok{2}\OperatorTok{,} \DecValTok{0}\OperatorTok{],}\NormalTok{ Nielsen}\OperatorTok{[}\DecValTok{1}\OperatorTok{,} \DecValTok{2}\OperatorTok{,} \SpecialCharTok{{-}}\DecValTok{1}\OperatorTok{],}\NormalTok{ Nielsen}\OperatorTok{[}\DecValTok{1}\OperatorTok{,} \DecValTok{2}\OperatorTok{,} \DecValTok{1}\SpecialCharTok{/}\DecValTok{2}\OperatorTok{],}\NormalTok{ Nielsen}\OperatorTok{[}\DecValTok{1}\OperatorTok{,} \DecValTok{2}\OperatorTok{,} \DecValTok{1}\OperatorTok{]\}}
\end{Highlighting}
\end{Shaded}

\begin{dmath*}\breakingcomma
\left\{0,\frac{\zeta (3)}{8},\frac{\zeta (3)}{8},\zeta (3)\right\}
\end{dmath*}

\begin{Shaded}
\begin{Highlighting}[]
\NormalTok{Nielsen}\OperatorTok{[}\DecValTok{1}\OperatorTok{,} \DecValTok{2}\OperatorTok{,} \FunctionTok{x}\OperatorTok{,} \FunctionTok{PolyLog} \OtherTok{{-}\textgreater{}} \ConstantTok{True}\OperatorTok{]}
\end{Highlighting}
\end{Shaded}

\begin{dmath*}\breakingcomma
-\text{Li}_3(1-x)+\text{Li}_2(1-x) \log (1-x)+\frac{1}{2} \log (x) \log ^2(1-x)+\zeta (3)
\end{dmath*}

\begin{Shaded}
\begin{Highlighting}[]
\NormalTok{Nielsen}\OperatorTok{[}\DecValTok{1}\OperatorTok{,} \DecValTok{3}\OperatorTok{,} \FunctionTok{x}\OperatorTok{,} \FunctionTok{PolyLog} \OtherTok{{-}\textgreater{}} \ConstantTok{True}\OperatorTok{]}
\end{Highlighting}
\end{Shaded}

\begin{dmath*}\breakingcomma
-\text{Li}_4(1-x)-\frac{1}{2} \;\text{Li}_2(1-x) \log ^2(1-x)+\text{Li}_3(1-x) \log (1-x)-\frac{1}{6} \log (x) \log ^3(1-x)+\frac{\pi ^4}{90}
\end{dmath*}

\begin{Shaded}
\begin{Highlighting}[]
\NormalTok{Nielsen}\OperatorTok{[}\DecValTok{3}\OperatorTok{,} \DecValTok{1}\OperatorTok{,} \FunctionTok{x}\OperatorTok{,} \FunctionTok{PolyLog} \OtherTok{{-}\textgreater{}} \ConstantTok{True}\OperatorTok{]}
\end{Highlighting}
\end{Shaded}

\begin{dmath*}\breakingcomma
\text{Li}_4(x)
\end{dmath*}
\end{document}
