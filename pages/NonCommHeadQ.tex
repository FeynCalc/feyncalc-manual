% !TeX program = pdflatex
% !TeX root = NonCommHeadQ.tex

\documentclass[../FeynCalcManual.tex]{subfiles}
\begin{document}
\hypertarget{noncommheadq}{%
\section{NonCommHeadQ}\label{noncommheadq}}

\texttt{NonCommHeadQ[\allowbreak{}exp]} yields \texttt{True} if the head
of exp is a non-commutative object or \texttt{Dot}.

\subsection{See also}

\hyperlink{toc}{Overview}, \hyperlink{datatype}{DataType},
\hyperlink{declarenoncommutative}{DeclareNonCommutative},
\hyperlink{undeclarenoncommutative}{UnDeclareNonCommutative},
\hyperlink{noncommfreeq}{NonCommFreeQ}, \hyperlink{noncommq}{NonCommQ}

\subsection{Examples}

\begin{Shaded}
\begin{Highlighting}[]
\NormalTok{NonCommHeadQ}\OperatorTok{[}\NormalTok{GA}\OperatorTok{[}\NormalTok{mu}\OperatorTok{]]}
\end{Highlighting}
\end{Shaded}

\begin{dmath*}\breakingcomma
\text{True}
\end{dmath*}

\begin{Shaded}
\begin{Highlighting}[]
\NormalTok{NonCommHeadQ}\OperatorTok{[}\NormalTok{GA}\OperatorTok{[}\NormalTok{mu}\OperatorTok{,}\NormalTok{ nu}\OperatorTok{,}\NormalTok{ mu}\OperatorTok{]]}
\end{Highlighting}
\end{Shaded}

\begin{dmath*}\breakingcomma
\text{True}
\end{dmath*}

\begin{Shaded}
\begin{Highlighting}[]
\NormalTok{NonCommHeadQ}\OperatorTok{[}\NormalTok{FV}\OperatorTok{[}\FunctionTok{p}\OperatorTok{,}\NormalTok{ mu}\OperatorTok{]]}
\end{Highlighting}
\end{Shaded}

\begin{dmath*}\breakingcomma
\text{False}
\end{dmath*}

\begin{Shaded}
\begin{Highlighting}[]
\NormalTok{NonCommHeadQ}\OperatorTok{[}\NormalTok{FCI}\OperatorTok{[}\NormalTok{SUNT}\OperatorTok{[}\FunctionTok{a}\OperatorTok{]]]}
\end{Highlighting}
\end{Shaded}

\begin{dmath*}\breakingcomma
\text{True}
\end{dmath*}

\begin{Shaded}
\begin{Highlighting}[]
\NormalTok{NonCommHeadQ}\OperatorTok{[}\NormalTok{FCI}\OperatorTok{[}\NormalTok{SUNTF}\OperatorTok{[}\FunctionTok{a}\OperatorTok{,} \FunctionTok{i}\OperatorTok{,} \FunctionTok{j}\OperatorTok{]]]}
\end{Highlighting}
\end{Shaded}

\begin{dmath*}\breakingcomma
\text{False}
\end{dmath*}
\end{document}
