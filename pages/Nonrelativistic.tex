% !TeX program = pdflatex
% !TeX root = Nonrelativistic.tex

\documentclass[../FeynCalcManual.tex]{subfiles}
\begin{document}
\begin{Shaded}
\begin{Highlighting}[]
 
\end{Highlighting}
\end{Shaded}

\hypertarget{nonrelativistic calculations}{
\section{Nonrelativistic calculations}\label{nonrelativistic calculations}\index{Nonrelativistic calculations}}

Since version 9.3 FeynCalc can also deal with manifestly noncovariant
expressions, such as 3-vectors, Kronecker deltas and Pauli matrices

\begin{Shaded}
\begin{Highlighting}[]
\NormalTok{CV}\OperatorTok{[}\FunctionTok{p}\OperatorTok{,} \FunctionTok{i}\OperatorTok{]}
\end{Highlighting}
\end{Shaded}

\begin{dmath*}\breakingcomma
\overline{p}^i
\end{dmath*}

\begin{Shaded}
\begin{Highlighting}[]
\NormalTok{CV}\OperatorTok{[}\FunctionTok{p}\OperatorTok{,} \FunctionTok{i}\OperatorTok{]}\NormalTok{ CV}\OperatorTok{[}\FunctionTok{q}\OperatorTok{,} \FunctionTok{i}\OperatorTok{]}
\SpecialCharTok{\%} \SpecialCharTok{//}\NormalTok{ Contract}
\end{Highlighting}
\end{Shaded}

\begin{dmath*}\breakingcomma
\overline{p}^i \overline{q}^i
\end{dmath*}

\begin{dmath*}\breakingcomma
\overline{p}\cdot \overline{q}
\end{dmath*}

\begin{Shaded}
\begin{Highlighting}[]
\NormalTok{CLC}\OperatorTok{[}\FunctionTok{i}\OperatorTok{,} \FunctionTok{j}\OperatorTok{,} \FunctionTok{k}\OperatorTok{]}\NormalTok{ CLC}\OperatorTok{[}\FunctionTok{i}\OperatorTok{,} \FunctionTok{j}\OperatorTok{,} \FunctionTok{l}\OperatorTok{]}
\SpecialCharTok{\%} \SpecialCharTok{//}\NormalTok{ Contract}
\end{Highlighting}
\end{Shaded}

\begin{dmath*}\breakingcomma
\bar{\epsilon }^{ijk} \bar{\epsilon }^{ijl}
\end{dmath*}

\begin{dmath*}\breakingcomma
2 \bar{\delta }^{kl}
\end{dmath*}

\begin{Shaded}
\begin{Highlighting}[]
\NormalTok{CSI}\OperatorTok{[}\FunctionTok{i}\OperatorTok{,} \FunctionTok{j}\OperatorTok{,} \FunctionTok{i}\OperatorTok{]}
\SpecialCharTok{\%} \SpecialCharTok{//}\NormalTok{ PauliSimplify}
\end{Highlighting}
\end{Shaded}

\begin{dmath*}\breakingcomma
\overline{\sigma }^i.\overline{\sigma }^j.\overline{\sigma }^i
\end{dmath*}

\begin{dmath*}\breakingcomma
-\overline{\sigma }^j
\end{dmath*}

\begin{Shaded}
\begin{Highlighting}[]
\NormalTok{PauliTrace}\OperatorTok{[}\NormalTok{CSI}\OperatorTok{[}\FunctionTok{i}\OperatorTok{,} \FunctionTok{j}\OperatorTok{,} \FunctionTok{i}\OperatorTok{,} \FunctionTok{j}\OperatorTok{]]}
\SpecialCharTok{\%} \SpecialCharTok{//}\NormalTok{ PauliSimplify}
\end{Highlighting}
\end{Shaded}

\begin{dmath*}\breakingcomma
\text{tr}\left(\overline{\sigma }^i.\overline{\sigma }^j.\overline{\sigma }^i.\overline{\sigma }^j\right)
\end{dmath*}

\begin{dmath*}\breakingcomma
-6
\end{dmath*}

The function \texttt{LorentzToCartesian} is used to break the manifest
Lorentz covariance when doing nonrelativistic expansions

\begin{Shaded}
\begin{Highlighting}[]
\NormalTok{SP}\OperatorTok{[}\FunctionTok{p}\OperatorTok{,} \FunctionTok{q}\OperatorTok{]}
\SpecialCharTok{\%} \SpecialCharTok{//}\NormalTok{ LorentzToCartesian}
\end{Highlighting}
\end{Shaded}

\begin{dmath*}\breakingcomma
\overline{p}\cdot \overline{q}
\end{dmath*}

\begin{dmath*}\breakingcomma
p^0 q^0-\overline{p}\cdot \overline{q}
\end{dmath*}
\end{document}
