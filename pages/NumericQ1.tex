% !TeX program = pdflatex
% !TeX root = NumericQ1.tex

\documentclass[../FeynCalcManual.tex]{subfiles}
\begin{document}
\hypertarget{numericq1}{
\section{NumericQ1}\label{numericq1}\index{NumericQ1}}

\texttt{NumericQ1[\allowbreak{}x,\ \allowbreak{}\{\allowbreak{}a,\ \allowbreak{}b,\ \allowbreak{}..\}]}
is like \texttt{NumericQ}, but assumes that
\texttt{\{\allowbreak{}a,\ \allowbreak{}b,\ \allowbreak{}..\}} are
numeric quantities.

\subsection{See also}

\hyperlink{toc}{Overview}

\subsection{Examples}

\begin{Shaded}
\begin{Highlighting}[]
\FunctionTok{NumericQ}\OperatorTok{[}\DecValTok{3} \FunctionTok{a} \SpecialCharTok{+} \FunctionTok{Log}\OperatorTok{[}\FunctionTok{b}\OperatorTok{]} \SpecialCharTok{+} \FunctionTok{c}\SpecialCharTok{\^{}}\DecValTok{2}\OperatorTok{]}
\end{Highlighting}
\end{Shaded}

\begin{dmath*}\breakingcomma
\text{False}
\end{dmath*}

\begin{Shaded}
\begin{Highlighting}[]
\NormalTok{NumericQ1}\OperatorTok{[}\DecValTok{3} \FunctionTok{a} \SpecialCharTok{+} \FunctionTok{Log}\OperatorTok{[}\FunctionTok{b}\OperatorTok{]} \SpecialCharTok{+} \FunctionTok{c}\SpecialCharTok{\^{}}\DecValTok{2}\OperatorTok{,} \OperatorTok{\{\}]}
\end{Highlighting}
\end{Shaded}

\begin{dmath*}\breakingcomma
\text{False}
\end{dmath*}

\begin{Shaded}
\begin{Highlighting}[]
\NormalTok{NumericQ1}\OperatorTok{[}\DecValTok{3} \FunctionTok{a} \SpecialCharTok{+} \FunctionTok{Log}\OperatorTok{[}\FunctionTok{b}\OperatorTok{]} \SpecialCharTok{+} \FunctionTok{c}\SpecialCharTok{\^{}}\DecValTok{2}\OperatorTok{,} \OperatorTok{\{}\FunctionTok{a}\OperatorTok{,} \FunctionTok{b}\OperatorTok{,} \FunctionTok{c}\OperatorTok{\}]}
\end{Highlighting}
\end{Shaded}

\begin{dmath*}\breakingcomma
\text{True}
\end{dmath*}
\end{document}
