% !TeX program = pdflatex
% !TeX root = OPEIntegrateDelta.tex

\documentclass[../FeynCalcManual.tex]{subfiles}
\begin{document}
\hypertarget{opeintegratedelta}{
\section{OPEIntegrateDelta}\label{opeintegratedelta}\index{OPEIntegrateDelta}}

\texttt{OPEIntegrateDelta[\allowbreak{}expr,\ \allowbreak{}x,\ \allowbreak{}m]}
introduces the \(\delta(1-x)\)
(\texttt{DeltaFunction[\allowbreak{}1-x]}).

The Mathematica \texttt{Integrate} function is called and each
integration (from \(0\) to \(1\)) is recorded for reference (and
bug-checking) in the list \texttt{\$MIntegrate}.

Notice that the dimension specified by the option should also be the
dimension used in \texttt{expr}. It is replaced in OPEIntegrateDelta by
(\texttt{4+Epsilon}).

\subsection{See also}

\hyperlink{toc}{Overview}.

\subsection{Examples}
\end{document}
