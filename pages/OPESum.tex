% !TeX program = pdflatex
% !TeX root = OPESum.tex

\documentclass[../FeynCalcManual.tex]{subfiles}
\begin{document}
\hypertarget{opesum}{%
\section{OPESum}\label{opesum}}

\texttt{OPESum[\allowbreak{}exp,\ \allowbreak{}\{\allowbreak{}i,\ \allowbreak{}0,\ \allowbreak{}m\}]}
denotes a symbolic sum. The syntax is the same as for \texttt{Sum}.

\subsection{See also}

\hyperlink{toc}{Overview}, \hyperlink{opesumexplicit}{OPESumExplicit},
\hyperlink{opesumsimplify}{OPESumSimplify}.

\subsection{Examples}

\begin{Shaded}
\begin{Highlighting}[]
\NormalTok{OPESum}\OperatorTok{[}\NormalTok{SO}\OperatorTok{[}\FunctionTok{p}\OperatorTok{]}\SpecialCharTok{\^{}}\NormalTok{OPEiSO}\OperatorTok{[}\FunctionTok{k}\OperatorTok{]}\SpecialCharTok{\^{}}\NormalTok{(OPEm }\SpecialCharTok{{-}}\NormalTok{ OPEi }\SpecialCharTok{{-}} \DecValTok{3}\NormalTok{)}\OperatorTok{,} \OperatorTok{\{}\NormalTok{OPEi}\OperatorTok{,} \DecValTok{0}\OperatorTok{,}\NormalTok{ OPEm }\SpecialCharTok{{-}} \DecValTok{3}\OperatorTok{\}]} 
 
\NormalTok{OPESumExplicit}\OperatorTok{[}\SpecialCharTok{\%}\OperatorTok{]}
\end{Highlighting}
\end{Shaded}

\begin{dmath*}\breakingcomma
\sum _{i=0}^{-3+m} (\Delta \cdot p)^{\text{OPEiSO}(k)^{-3-i+m}}
\end{dmath*}

\begin{dmath*}\breakingcomma
\sum _{i=0}^{-3+m} (\Delta \cdot p)^{\text{OPEiSO}(k)^{-3-i+m}}
\end{dmath*}

\begin{Shaded}
\begin{Highlighting}[]
\NormalTok{OPESum}\OperatorTok{[}\FunctionTok{a}\SpecialCharTok{\^{}}\NormalTok{ib}\SpecialCharTok{\^{}}\NormalTok{(}\FunctionTok{j} \SpecialCharTok{{-}} \FunctionTok{i}\NormalTok{) }\FunctionTok{c}\SpecialCharTok{\^{}}\NormalTok{(}\FunctionTok{m} \SpecialCharTok{{-}} \FunctionTok{j} \SpecialCharTok{{-}} \DecValTok{4}\NormalTok{)}\OperatorTok{,} \OperatorTok{\{}\FunctionTok{i}\OperatorTok{,} \DecValTok{0}\OperatorTok{,} \FunctionTok{j}\OperatorTok{\},} \OperatorTok{\{}\FunctionTok{j}\OperatorTok{,} \DecValTok{0}\OperatorTok{,} \FunctionTok{m} \SpecialCharTok{{-}} \DecValTok{4}\OperatorTok{\}]} 
 
\NormalTok{OPESumExplicit}\OperatorTok{[}\SpecialCharTok{\%}\OperatorTok{]}
\end{Highlighting}
\end{Shaded}

\begin{dmath*}\breakingcomma
\sum _{j=0}^{-4+m} \;\text{}\;\text{} (j+1)c^{-j+m-4} a^{\text{ib}^{j-i}}
\end{dmath*}

\begin{dmath*}\breakingcomma
\sum _{j=0}^{-4+m} \;\text{}\;\text{} (j+1)c^{-j+m-4} a^{\text{ib}^{j-i}}
\end{dmath*}
\end{document}
