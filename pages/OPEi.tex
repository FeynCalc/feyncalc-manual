% !TeX program = pdflatex
% !TeX root = OPEi.tex

\documentclass[../FeynCalcManual.tex]{subfiles}
\begin{document}
\hypertarget{opei}{%
\section{OPEi}\label{opei}}

\texttt{OPEi} etc. are variables with \texttt{DataType}
\texttt{PositiveInteger} which are used in functions relating to the
operator product expansion.

\subsection{See also}

\hyperlink{toc}{Overview}, \hyperlink{opej}{OPEj},
\hyperlink{opek}{OPEk}, \hyperlink{opel}{OPEl}, \hyperlink{open}{OPEn},
\hyperlink{opeo}{OPEo}.

\subsection{Examples}

\begin{Shaded}
\begin{Highlighting}[]
\NormalTok{OPEi}
\end{Highlighting}
\end{Shaded}

\begin{dmath*}\breakingcomma
i
\end{dmath*}

\begin{Shaded}
\begin{Highlighting}[]
\NormalTok{DataType}\OperatorTok{[}\NormalTok{OPEi}\OperatorTok{,}\NormalTok{ OPEj}\OperatorTok{,}\NormalTok{ OPEk}\OperatorTok{,}\NormalTok{ OPEl}\OperatorTok{,}\NormalTok{ OPEm}\OperatorTok{,}\NormalTok{ OPEn}\OperatorTok{,}\NormalTok{ OPEo}\OperatorTok{,}\NormalTok{ PositiveInteger}\OperatorTok{]}
\end{Highlighting}
\end{Shaded}

\begin{dmath*}\breakingcomma
\{\text{True},\text{True},\text{True},\text{True},\text{True},\text{True},\text{True}\}
\end{dmath*}

\begin{Shaded}
\begin{Highlighting}[]
\NormalTok{PowerSimplify}\OperatorTok{[\{}\NormalTok{(}\SpecialCharTok{{-}}\DecValTok{1}\NormalTok{)}\SpecialCharTok{\^{}}\NormalTok{(}\DecValTok{2}\NormalTok{ OPEi)}\OperatorTok{,}\NormalTok{ (}\SpecialCharTok{{-}}\DecValTok{1}\NormalTok{)}\SpecialCharTok{\^{}}\NormalTok{(}\DecValTok{2}\NormalTok{ OPEj)}\OperatorTok{,}\NormalTok{ (}\SpecialCharTok{{-}}\DecValTok{1}\NormalTok{)}\SpecialCharTok{\^{}}\NormalTok{(}\DecValTok{2}\NormalTok{ OPEk)}\OperatorTok{,}\NormalTok{ (}\SpecialCharTok{{-}}\DecValTok{1}\NormalTok{)}\SpecialCharTok{\^{}}\NormalTok{(}\DecValTok{2}\NormalTok{ OPEl)}\OperatorTok{,}\NormalTok{ (}\SpecialCharTok{{-}}\DecValTok{1}\NormalTok{)}\SpecialCharTok{\^{}}\NormalTok{(}\DecValTok{2}\NormalTok{ OPEm)}\OperatorTok{,}\NormalTok{ (}\SpecialCharTok{{-}}\DecValTok{1}\NormalTok{)}\SpecialCharTok{\^{}}\NormalTok{(}\DecValTok{2}\NormalTok{ OPEn)}\OperatorTok{,}\NormalTok{ (}\SpecialCharTok{{-}}\DecValTok{1}\NormalTok{)}\SpecialCharTok{\^{}}\NormalTok{(}\DecValTok{2}\NormalTok{ OPEo)}\OperatorTok{\}]}
\end{Highlighting}
\end{Shaded}

\begin{dmath*}\breakingcomma
\{1,1,1,1,1,1,1\}
\end{dmath*}

Re has been changed:

\begin{Shaded}
\begin{Highlighting}[]
\OperatorTok{\{}\FunctionTok{Re}\OperatorTok{[}\NormalTok{OPEi}\OperatorTok{]}\NormalTok{ \textgreater{} }\SpecialCharTok{{-}}\DecValTok{3}\OperatorTok{,} \FunctionTok{Re}\OperatorTok{[}\NormalTok{OPEi}\OperatorTok{]}\NormalTok{ \textgreater{} }\SpecialCharTok{{-}}\DecValTok{2}\OperatorTok{,} \FunctionTok{Re}\OperatorTok{[}\NormalTok{OPEi}\OperatorTok{]}\NormalTok{ \textgreater{} }\SpecialCharTok{{-}}\DecValTok{1}\OperatorTok{,}   \FunctionTok{Re}\OperatorTok{[}\NormalTok{OPEi}\OperatorTok{]}\NormalTok{ \textgreater{} }\DecValTok{0}\OperatorTok{,} \FunctionTok{Re}\OperatorTok{[}\NormalTok{OPEi}\OperatorTok{]}\NormalTok{ \textgreater{} }\DecValTok{1}\OperatorTok{\}}
\end{Highlighting}
\end{Shaded}

\begin{dmath*}\breakingcomma
\{\Re(i)>-3,\Re(i)>-2,\Re(i)>-1,\Re(i)>0,\Re(i)>1\}
\end{dmath*}

\begin{Shaded}
\begin{Highlighting}[]
\OperatorTok{\{}\FunctionTok{Re}\OperatorTok{[}\SpecialCharTok{{-}}\NormalTok{OPEi }\SpecialCharTok{+}\NormalTok{ OPEm}\OperatorTok{]}\NormalTok{ \textgreater{} }\DecValTok{0}\OperatorTok{,} \FunctionTok{Re}\OperatorTok{[}\SpecialCharTok{{-}}\NormalTok{OPEi }\SpecialCharTok{+}\NormalTok{ OPEm}\OperatorTok{]}\NormalTok{ \textgreater{} }\DecValTok{1}\OperatorTok{,} \FunctionTok{Re}\OperatorTok{[}\SpecialCharTok{{-}}\NormalTok{OPEi }\SpecialCharTok{+}\NormalTok{ OPEm}\OperatorTok{]}\NormalTok{ \textgreater{} }\DecValTok{2}\OperatorTok{\}}
\end{Highlighting}
\end{Shaded}

\begin{dmath*}\breakingcomma
\{\Re(m-i)>0,\Re(m-i)>1,\Re(m-i)>2\}
\end{dmath*}

\begin{Shaded}
\begin{Highlighting}[]
\OperatorTok{\{}\FunctionTok{Re}\OperatorTok{[}\NormalTok{OPEm}\OperatorTok{]}\NormalTok{ \textgreater{} }\SpecialCharTok{{-}}\DecValTok{3}\OperatorTok{,} \FunctionTok{Re}\OperatorTok{[}\NormalTok{OPEm}\OperatorTok{]}\NormalTok{ \textgreater{} }\SpecialCharTok{{-}}\DecValTok{2}\OperatorTok{,} \FunctionTok{Re}\OperatorTok{[}\NormalTok{OPEm}\OperatorTok{]}\NormalTok{ \textgreater{} }\SpecialCharTok{{-}}\DecValTok{1}\OperatorTok{,}   \FunctionTok{Re}\OperatorTok{[}\NormalTok{OPEm}\OperatorTok{]}\NormalTok{ \textgreater{} }\DecValTok{0}\OperatorTok{,} \FunctionTok{Re}\OperatorTok{[}\NormalTok{OPEm}\OperatorTok{]}\NormalTok{ \textgreater{} }\DecValTok{1}\OperatorTok{\}}
\end{Highlighting}
\end{Shaded}

\begin{dmath*}\breakingcomma
\{\Re(m)>-3,\Re(m)>-2,\Re(m)>-1,\Re(m)>0,\Re(m)>1\}
\end{dmath*}
\end{document}
