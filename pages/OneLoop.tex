% !TeX program = pdflatex
% !TeX root = OneLoop.tex

\documentclass[../FeynCalcManual.tex]{subfiles}
\begin{document}
\hypertarget{oneloop}{
\section{OneLoop}\label{oneloop}\index{OneLoop}}

\texttt{OneLoop[\allowbreak{}q,\ \allowbreak{}amplitude]} calculates the
1-loop Feynman diagram amplitude. The argument \texttt{q} denotes the
integration variable, i.e., the loop momentum.
\texttt{OneLoop[\allowbreak{}name,\ \allowbreak{}q,\ \allowbreak{}amplitude]}
has as first argument a name of the amplitude. If the second argument
has head \texttt{FeynAmp} then
\texttt{OneLoop[\allowbreak{}q,\ \allowbreak{}FeynAmp[\allowbreak{}name,\ \allowbreak{}k,\ \allowbreak{}expr]]}
and
\texttt{OneLoop[\allowbreak{}FeynAmp[\allowbreak{}name,\ \allowbreak{}k,\ \allowbreak{}expr]]}
tranform to
\texttt{OneLoop[\allowbreak{}name,\ \allowbreak{}k,\ \allowbreak{}expr]}.
\texttt{OneLoop} is deprecated, please use \texttt{TID} instead!

\subsection{See also}

\hyperlink{toc}{Overview}, \hyperlink{topave}{ToPaVe},
\hyperlink{topave2}{ToPaVe2}, \hyperlink{a0}{A0}, \hyperlink{a00}{A00},
\hyperlink{b0}{B0}, \hyperlink{b1}{B1}, \hyperlink{b00}{B00},
\hyperlink{b11}{B11}, \hyperlink{c0}{C0}, \hyperlink{d0}{D0}.

\subsection{Examples}

\begin{Shaded}
\begin{Highlighting}[]
\SpecialCharTok{{-}}\FunctionTok{I}\SpecialCharTok{/}\FunctionTok{Pi}\SpecialCharTok{\^{}}\DecValTok{2}\NormalTok{ FAD}\OperatorTok{[\{}\FunctionTok{q}\OperatorTok{,} \FunctionTok{m}\OperatorTok{\}]} 
 
\NormalTok{OneLoop}\OperatorTok{[}\FunctionTok{q}\OperatorTok{,} \SpecialCharTok{\%}\OperatorTok{]}
\end{Highlighting}
\end{Shaded}

\begin{dmath*}\breakingcomma
-\frac{i}{\pi ^2 \left(q^2-m^2\right)}
\end{dmath*}

\begin{dmath*}\breakingcomma
\text{A}_0\left(m^2\right)
\end{dmath*}

\begin{Shaded}
\begin{Highlighting}[]
\FunctionTok{I}\NormalTok{ ((el}\SpecialCharTok{\^{}}\DecValTok{2}\NormalTok{)}\SpecialCharTok{/}\NormalTok{(}\DecValTok{16} \FunctionTok{Pi}\SpecialCharTok{\^{}}\DecValTok{4}\NormalTok{ (}\DecValTok{1} \SpecialCharTok{{-}} \FunctionTok{D}\NormalTok{))) FAD}\OperatorTok{[\{}\FunctionTok{q}\OperatorTok{,}\NormalTok{ mf}\OperatorTok{\},} \OperatorTok{\{}\FunctionTok{q} \SpecialCharTok{{-}} \FunctionTok{k}\OperatorTok{,}\NormalTok{ mf}\OperatorTok{\}]}\NormalTok{ DiracTrace}\OperatorTok{[}\NormalTok{(mf }\SpecialCharTok{+}\NormalTok{ GSD}\OperatorTok{[}\FunctionTok{q} \SpecialCharTok{{-}} \FunctionTok{k}\OperatorTok{]}\NormalTok{) . GAD}\OperatorTok{[}\SpecialCharTok{\textbackslash{}}\OperatorTok{[}\NormalTok{Mu}\OperatorTok{]]}\NormalTok{ . (mf }\SpecialCharTok{+}\NormalTok{ GSD}\OperatorTok{[}\FunctionTok{q}\OperatorTok{]}\NormalTok{) . GAD}\OperatorTok{[}\SpecialCharTok{\textbackslash{}}\OperatorTok{[}\NormalTok{Mu}\OperatorTok{]]]} 
 
\NormalTok{OneLoop}\OperatorTok{[}\FunctionTok{q}\OperatorTok{,} \SpecialCharTok{\%}\OperatorTok{]}
\end{Highlighting}
\end{Shaded}

\begin{dmath*}\breakingcomma
\frac{i \;\text{el}^2 \;\text{tr}\left((\gamma \cdot (q-k)+\text{mf}).\gamma ^{\mu }.(\text{mf}+\gamma \cdot q).\gamma ^{\mu }\right)}{16 \pi ^4 (1-D) \left(q^2-\text{mf}^2\right).\left((q-k)^2-\text{mf}^2\right)}
\end{dmath*}

\begin{dmath*}\breakingcomma
\frac{\text{el}^2 \left(-\frac{8 \;\text{mf}^2 \;\text{B}_0\left(\overline{k}^2,\text{mf}^2,\text{mf}^2\right)}{1-D}+\frac{2 (2-D) \overline{k}^2 \;\text{B}_0\left(\overline{k}^2,\text{mf}^2,\text{mf}^2\right)}{1-D}+\frac{4 D \;\text{A}_0\left(\text{mf}^2\right)}{1-D}-\frac{8 \;\text{A}_0\left(\text{mf}^2\right)}{1-D}\right)}{16 \pi ^2}
\end{dmath*}
\end{document}
