% !TeX program = pdflatex
% !TeX root = PIDelta.tex

\documentclass[../FeynCalcManual.tex]{subfiles}
\begin{document}
\hypertarget{pidelta}{%
\section{PIDelta}\label{pidelta}}

\texttt{PIDelta[\allowbreak{}i,\ \allowbreak{}j]} is the Kronecker-delta
in the Pauli space. \texttt{PIDelta[\allowbreak{}i,\ \allowbreak{}j]} is
transformed into
\texttt{PauliIndexDelta[\allowbreak{}PauliIndex[\allowbreak{}i],\ \allowbreak{}PauliIndex[\allowbreak{}j]]}
by FeynCalcInternal.

\subsection{See also}

\hyperlink{toc}{Overview}, \hyperlink{paulichain}{PauliChain},
\hyperlink{pchn}{PCHN}, \hyperlink{pauliindex}{PauliIndex},
\hyperlink{pauliindexdelta}{PauliIndexDelta},
\hyperlink{paulichainjoin}{PauliChainJoin},
\hyperlink{paulichainexpand}{PauliChainExpand},
\hyperlink{paulichainfactor}{PauliChainFactor}.

\subsection{Examples}

\begin{Shaded}
\begin{Highlighting}[]
\NormalTok{PIDelta}\OperatorTok{[}\FunctionTok{i}\OperatorTok{,} \FunctionTok{j}\OperatorTok{]}
\end{Highlighting}
\end{Shaded}

\begin{dmath*}\breakingcomma
\delta _{ij}
\end{dmath*}

\begin{Shaded}
\begin{Highlighting}[]
\NormalTok{PIDelta}\OperatorTok{[}\FunctionTok{i}\OperatorTok{,} \FunctionTok{i}\OperatorTok{]} 
 
\NormalTok{PauliChainJoin}\OperatorTok{[}\SpecialCharTok{\%}\OperatorTok{]}
\end{Highlighting}
\end{Shaded}

\begin{dmath*}\breakingcomma
\delta _{ii}
\end{dmath*}

\begin{dmath*}\breakingcomma
4
\end{dmath*}

\begin{Shaded}
\begin{Highlighting}[]
\NormalTok{PIDelta}\OperatorTok{[}\FunctionTok{i}\OperatorTok{,} \FunctionTok{j}\OperatorTok{]}\SpecialCharTok{\^{}}\DecValTok{2} 
 
\NormalTok{PauliChainJoin}\OperatorTok{[}\SpecialCharTok{\%}\OperatorTok{]}
\end{Highlighting}
\end{Shaded}

\begin{dmath*}\breakingcomma
\delta _{ij}^2
\end{dmath*}

\begin{dmath*}\breakingcomma
4
\end{dmath*}

\begin{Shaded}
\begin{Highlighting}[]
\NormalTok{PIDelta}\OperatorTok{[}\FunctionTok{i}\OperatorTok{,} \FunctionTok{j}\OperatorTok{]}\NormalTok{ PIDelta}\OperatorTok{[}\FunctionTok{j}\OperatorTok{,} \FunctionTok{k}\OperatorTok{]} 
 
\NormalTok{PauliChainJoin}\OperatorTok{[}\SpecialCharTok{\%}\OperatorTok{]}
\end{Highlighting}
\end{Shaded}

\begin{dmath*}\breakingcomma
\delta _{ij} \delta _{jk}
\end{dmath*}

\begin{dmath*}\breakingcomma
\delta _{ik}
\end{dmath*}

\begin{Shaded}
\begin{Highlighting}[]
\NormalTok{ex }\ExtensionTok{=}\NormalTok{ PIDelta}\OperatorTok{[}\NormalTok{i2}\OperatorTok{,}\NormalTok{ i3}\OperatorTok{]}\NormalTok{ PIDelta}\OperatorTok{[}\NormalTok{i4}\OperatorTok{,}\NormalTok{ i5}\OperatorTok{]}\NormalTok{  PCHN}\OperatorTok{[}\NormalTok{i7}\OperatorTok{,}\NormalTok{ PauliXi}\OperatorTok{[}\FunctionTok{I}\OperatorTok{]]}\NormalTok{ PauliChain}\OperatorTok{[}\NormalTok{PauliEta}\OperatorTok{[}\SpecialCharTok{{-}}\FunctionTok{I}\OperatorTok{],}\NormalTok{ PauliIndex}\OperatorTok{[}\NormalTok{i0}\OperatorTok{]]}\NormalTok{ PauliChain}\OperatorTok{[}\NormalTok{PauliSigma}\OperatorTok{[}\NormalTok{CartesianIndex}\OperatorTok{[}\FunctionTok{a}\OperatorTok{]],}\NormalTok{ PauliIndex}\OperatorTok{[}\NormalTok{i1}\OperatorTok{],}\NormalTok{ PauliIndex}\OperatorTok{[}\NormalTok{i2}\OperatorTok{]]}\NormalTok{ PauliChain}\OperatorTok{[}\NormalTok{PauliSigma}\OperatorTok{[}\NormalTok{CartesianIndex}\OperatorTok{[}\FunctionTok{b}\OperatorTok{]],}\NormalTok{ PauliIndex}\OperatorTok{[}\NormalTok{i5}\OperatorTok{],}\NormalTok{ PauliIndex}\OperatorTok{[}\NormalTok{i6}\OperatorTok{]]}\NormalTok{ PauliChain}\OperatorTok{[}\FunctionTok{m} \SpecialCharTok{+}\NormalTok{ PauliSigma}\OperatorTok{[}\NormalTok{CartesianMomentum}\OperatorTok{[}\FunctionTok{p}\OperatorTok{]],}\NormalTok{ PauliIndex}\OperatorTok{[}\NormalTok{i3}\OperatorTok{],}\NormalTok{ PauliIndex}\OperatorTok{[}\NormalTok{i4}\OperatorTok{]]}
\end{Highlighting}
\end{Shaded}

\begin{dmath*}\breakingcomma
(\xi )_{\text{i7}} \left(\eta ^{\dagger }\right){}_{\text{i0}} \delta _{\text{i2}\;\text{i3}} \delta _{\text{i4}\;\text{i5}} \left(\overline{\sigma }^a\right){}_{\text{i1}\;\text{i2}} \left(\overline{\sigma }^b\right){}_{\text{i5}\;\text{i6}} \left(\overline{\sigma }\cdot \overline{p}+m\right)_{\text{i3}\;\text{i4}}
\end{dmath*}

\begin{Shaded}
\begin{Highlighting}[]
\NormalTok{PauliChainJoin}\OperatorTok{[}\NormalTok{ex}\OperatorTok{]}
\end{Highlighting}
\end{Shaded}

\begin{dmath*}\breakingcomma
(\xi )_{\text{i7}} \left(\eta ^{\dagger }\right){}_{\text{i0}} \left(\overline{\sigma }^a.\left(\overline{\sigma }\cdot \overline{p}+m\right).\overline{\sigma }^b\right){}_{\text{i1}\;\text{i6}}
\end{dmath*}

\begin{Shaded}
\begin{Highlighting}[]
\NormalTok{PauliChainJoin}\OperatorTok{[}\NormalTok{ex PIDelta}\OperatorTok{[}\NormalTok{i0}\OperatorTok{,}\NormalTok{ i1}\OperatorTok{]]}
\end{Highlighting}
\end{Shaded}

\begin{dmath*}\breakingcomma
(\xi )_{\text{i7}} \left(\eta ^{\dagger }.\overline{\sigma }^a.\left(\overline{\sigma }\cdot \overline{p}+m\right).\overline{\sigma }^b\right){}_{\text{i6}}
\end{dmath*}

\begin{Shaded}
\begin{Highlighting}[]
\NormalTok{PauliChainJoin}\OperatorTok{[}\SpecialCharTok{\%}\NormalTok{ PIDelta}\OperatorTok{[}\NormalTok{i7}\OperatorTok{,}\NormalTok{ i6}\OperatorTok{]]}
\end{Highlighting}
\end{Shaded}

\begin{dmath*}\breakingcomma
\eta ^{\dagger }.\overline{\sigma }^a.\left(\overline{\sigma }\cdot \overline{p}+m\right).\overline{\sigma }^b.\xi
\end{dmath*}
\end{document}
