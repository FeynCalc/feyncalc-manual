% !TeX program = pdflatex
% !TeX root = PaVe.tex

\documentclass[../FeynCalcManual.tex]{subfiles}
\begin{document}
\hypertarget{pave}{
\section{PaVe}\label{pave}\index{PaVe}}

\texttt{PaVe[\allowbreak{}i,\ \allowbreak{}j,\ \allowbreak{}...,\ \allowbreak{}\{\allowbreak{}p10,\ \allowbreak{}p12,\ \allowbreak{}...\},\ \allowbreak{}\{\allowbreak{}m1^2,\ \allowbreak{}mw^2,\ \allowbreak{}...\}]}
denotes the invariant (and scalar) Passarino-Veltman integrals, i.e.~the
coefficient functions of the tensor integral decomposition. Joining
\texttt{plist} and \texttt{mlist} gives the same conventions as for
\texttt{A0}, \texttt{B0}, \texttt{C0}, \texttt{D0}. Automatic
simplifications are performed for the coefficient functions of two-point
integrals and for the scalar integrals.

\subsection{See also}

\hyperlink{toc}{Overview}, \hyperlink{pavereduce}{PaVeReduce}.

\subsection{Examples}

Some of the PaVe's reduce to special cases with
\texttt{PaVeAutoReduce}to \texttt{True}

\begin{Shaded}
\begin{Highlighting}[]
\NormalTok{PaVe}\OperatorTok{[}\DecValTok{0}\OperatorTok{,} \DecValTok{0}\OperatorTok{,} \OperatorTok{\{}\NormalTok{pp}\OperatorTok{\},} \OperatorTok{\{}\FunctionTok{m}\SpecialCharTok{\^{}}\DecValTok{2}\OperatorTok{,} \FunctionTok{M}\SpecialCharTok{\^{}}\DecValTok{2}\OperatorTok{\},}\NormalTok{ PaVeAutoReduce }\OtherTok{{-}\textgreater{}} \ConstantTok{True}\OperatorTok{]}
\end{Highlighting}
\end{Shaded}

\begin{dmath*}\breakingcomma
\frac{\left(m^2-2 m M+M^2-\text{pp}\right) \left(m^2+2 m M+M^2-\text{pp}\right) \;\text{B}_0\left(\text{pp},m^2,M^2\right)}{4 (1-D) \;\text{pp}}-\frac{\text{A}_0\left(m^2\right) \left(m^2-M^2+\text{pp}\right)}{4 (1-D) \;\text{pp}}+\frac{\text{A}_0\left(M^2\right) \left(m^2-M^2-\text{pp}\right)}{4 (1-D) \;\text{pp}}
\end{dmath*}
\end{document}
