% !TeX program = pdflatex
% !TeX root = PaVeUVPart.tex

\documentclass[../FeynCalcManual.tex]{subfiles}
\begin{document}
\hypertarget{paveuvpart}{%
\section{PaVeUVPart}\label{paveuvpart}}

\texttt{PaVeUVPart[\allowbreak{}expr]} replaces all occurring
Passarino-Veltman functions by their explicit values, where only the UV
divergent part is preserved, while possible IR divergences and the
finite part are discarded. The function uses the algorithm from
\href{https://arxiv.org/abs/hep-ph/0609282}{arXiv:hep-ph/0609282} by G.
Sulyok. This allows to treat Passarino-Veltman of arbitrary rank and
multiplicity

\subsection{See also}

\hyperlink{toc}{Overview}, \hyperlink{pave}{PaVe},
\hyperlink{pavereduce}{PaVeReduce}.

\subsection{Examples}

\begin{Shaded}
\begin{Highlighting}[]
\NormalTok{PaVeUVPart}\OperatorTok{[}\NormalTok{A0}\OperatorTok{[}\FunctionTok{m}\SpecialCharTok{\^{}}\DecValTok{2}\OperatorTok{]]}
\end{Highlighting}
\end{Shaded}

\begin{dmath*}\breakingcomma
-\frac{2 m^2}{D-4}
\end{dmath*}

\begin{Shaded}
\begin{Highlighting}[]
\NormalTok{PaVeUVPart}\OperatorTok{[}\FunctionTok{x} \SpecialCharTok{+} \FunctionTok{y}\NormalTok{ B0}\OperatorTok{[}\NormalTok{SPD}\OperatorTok{[}\FunctionTok{p}\OperatorTok{,} \FunctionTok{p}\OperatorTok{],} \DecValTok{0}\OperatorTok{,} \FunctionTok{M}\SpecialCharTok{\^{}}\DecValTok{2}\OperatorTok{]]}
\end{Highlighting}
\end{Shaded}

\begin{dmath*}\breakingcomma
\frac{D x-4 x-2 y}{D-4}
\end{dmath*}

\begin{Shaded}
\begin{Highlighting}[]
\NormalTok{PaVe}\OperatorTok{[}\DecValTok{0}\OperatorTok{,} \DecValTok{0}\OperatorTok{,} \OperatorTok{\{}\NormalTok{p10}\OperatorTok{,}\NormalTok{ p12}\OperatorTok{,}\NormalTok{ p20}\OperatorTok{\},} \OperatorTok{\{}\NormalTok{m1}\SpecialCharTok{\^{}}\DecValTok{2}\OperatorTok{,}\NormalTok{ m2}\SpecialCharTok{\^{}}\DecValTok{2}\OperatorTok{,}\NormalTok{ m3}\SpecialCharTok{\^{}}\DecValTok{2}\OperatorTok{\}]} 
 
\NormalTok{PaVeUVPart}\OperatorTok{[}\SpecialCharTok{\%}\OperatorTok{]}
\end{Highlighting}
\end{Shaded}

\begin{dmath*}\breakingcomma
\text{C}_{00}\left(\text{p10},\text{p12},\text{p20},\text{m1}^2,\text{m2}^2,\text{m3}^2\right)
\end{dmath*}

\begin{dmath*}\breakingcomma
-\frac{1}{2 (D-4)}
\end{dmath*}

\begin{Shaded}
\begin{Highlighting}[]
\NormalTok{PaVe}\OperatorTok{[}\DecValTok{0}\OperatorTok{,} \DecValTok{0}\OperatorTok{,} \DecValTok{0}\OperatorTok{,} \DecValTok{0}\OperatorTok{,} \DecValTok{0}\OperatorTok{,} \DecValTok{0}\OperatorTok{,} \OperatorTok{\{}\NormalTok{p10}\OperatorTok{,}\NormalTok{ p12}\OperatorTok{,}\NormalTok{ p23}\OperatorTok{,} \DecValTok{0}\OperatorTok{,}\NormalTok{ p20}\OperatorTok{,}\NormalTok{ p13}\OperatorTok{\},} \OperatorTok{\{}\NormalTok{m1}\SpecialCharTok{\^{}}\DecValTok{2}\OperatorTok{,}\NormalTok{ m2}\SpecialCharTok{\^{}}\DecValTok{2}\OperatorTok{,}\NormalTok{ m3}\SpecialCharTok{\^{}}\DecValTok{2}\OperatorTok{,}\NormalTok{ m4}\SpecialCharTok{\^{}}\DecValTok{2}\OperatorTok{\}]} 
 
\NormalTok{PaVeUVPart}\OperatorTok{[}\SpecialCharTok{\%}\OperatorTok{]}
\end{Highlighting}
\end{Shaded}

\begin{dmath*}\breakingcomma
\text{D}_{000000}\left(0,\text{p10},\text{p12},\text{p23},\text{p13},\text{p20},\text{m4}^2,\text{m1}^2,\text{m2}^2,\text{m3}^2\right)
\end{dmath*}

\begin{dmath*}\breakingcomma
\frac{-5 \;\text{m1}^2-5 \;\text{m2}^2-5 \;\text{m3}^2-5 \;\text{m4}^2+\text{p10}+\text{p12}+\text{p13}+\text{p20}+\text{p23}}{480 (D-4)}
\end{dmath*}

\begin{Shaded}
\begin{Highlighting}[]
\NormalTok{int }\ExtensionTok{=}\NormalTok{ FVD}\OperatorTok{[}\FunctionTok{k} \SpecialCharTok{+} \FunctionTok{p}\OperatorTok{,}\NormalTok{ rho}\OperatorTok{]}\NormalTok{ FVD}\OperatorTok{[}\FunctionTok{k} \SpecialCharTok{+} \FunctionTok{p}\OperatorTok{,}\NormalTok{ si}\OperatorTok{]}\NormalTok{ FAD}\OperatorTok{[}\FunctionTok{k}\OperatorTok{,} \OperatorTok{\{}\FunctionTok{k} \SpecialCharTok{+} \FunctionTok{p}\OperatorTok{,} \DecValTok{0}\OperatorTok{,} \DecValTok{2}\OperatorTok{\}]} 
 
\NormalTok{TID}\OperatorTok{[}\NormalTok{int}\OperatorTok{,} \FunctionTok{k}\OperatorTok{,}\NormalTok{ UsePaVeBasis }\OtherTok{{-}\textgreater{}} \ConstantTok{True}\OperatorTok{]} 
 
\SpecialCharTok{\%} \SpecialCharTok{//}\NormalTok{ PaVeUVPart}\OperatorTok{[}\NormalTok{\#}\OperatorTok{,}\NormalTok{ FCE }\OtherTok{{-}\textgreater{}} \ConstantTok{True}\OperatorTok{]}\NormalTok{ \&}
\end{Highlighting}
\end{Shaded}

\begin{dmath*}\breakingcomma
\frac{(k+p)^{\text{rho}} (k+p)^{\text{si}}}{k^2.(k+p)^4}
\end{dmath*}

\begin{dmath*}\breakingcomma
i \pi ^2 g^{\text{rho}\;\text{si}} \;\text{C}_{00}\left(0,p^2,p^2,0,0,0\right)+i \pi ^2 p^{\text{rho}} p^{\text{si}} \;\text{C}_{11}\left(p^2,p^2,0,0,0,0\right)
\end{dmath*}

\begin{dmath*}\breakingcomma
-\frac{i \pi ^2 g^{\text{rho}\;\text{si}}}{2 (D-4)}
\end{dmath*}
\end{document}
