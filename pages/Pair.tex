% !TeX program = pdflatex
% !TeX root = Pair.tex

\documentclass[../FeynCalcManual.tex]{subfiles}
\begin{document}
\hypertarget{pair}{
\section{Pair}\label{pair}\index{Pair}}

\texttt{Pair[\allowbreak{}x,\ \allowbreak{}y]} is the head of a special
pairing used in the internal representation: \texttt{x} and \texttt{y}
may have heads \texttt{LorentzIndex} or \texttt{Momentum}.

If both \texttt{x} and \texttt{y} have head \texttt{LorentzIndex}, the
metric tensor (e.g.~\(g^{\mu \nu}\)) is understood.

If \texttt{x} and \texttt{y} have head \texttt{Momentum}, a scalar
product (e.g.~\(p \cdot q\)) is meant.

If one of \texttt{x} and \texttt{y} has head \texttt{LorentzIndex} and
the other \texttt{Momentum}, a Lorentz vector (e.g.~\(p^{\mu }\)) is
implied.

\subsection{See also}

\hyperlink{toc}{Overview}, \hyperlink{fv}{FV}, \hyperlink{fvd}{FVD},
\hyperlink{mt}{MT}, \hyperlink{mtd}{MTD},
\hyperlink{scalarproduct}{ScalarProduct}, \hyperlink{sp}{SP},
\hyperlink{spd}{SPD}.

\subsection{Examples}

This represents a \(4\)-dimensional metric tensor

\begin{Shaded}
\begin{Highlighting}[]
\NormalTok{Pair}\OperatorTok{[}\NormalTok{LorentzIndex}\OperatorTok{[}\SpecialCharTok{\textbackslash{}}\OperatorTok{[}\NormalTok{Alpha}\OperatorTok{]],}\NormalTok{ LorentzIndex}\OperatorTok{[}\SpecialCharTok{\textbackslash{}}\OperatorTok{[}\FunctionTok{Beta}\OperatorTok{]]]}
\end{Highlighting}
\end{Shaded}

\begin{dmath*}\breakingcomma
\bar{g}^{\alpha \beta }
\end{dmath*}

This is a D-dimensional metric tensor

\begin{Shaded}
\begin{Highlighting}[]
\NormalTok{Pair}\OperatorTok{[}\NormalTok{LorentzIndex}\OperatorTok{[}\SpecialCharTok{\textbackslash{}}\OperatorTok{[}\NormalTok{Alpha}\OperatorTok{],} \FunctionTok{D}\OperatorTok{],}\NormalTok{ LorentzIndex}\OperatorTok{[}\SpecialCharTok{\textbackslash{}}\OperatorTok{[}\FunctionTok{Beta}\OperatorTok{],} \FunctionTok{D}\OperatorTok{]]}
\end{Highlighting}
\end{Shaded}

\begin{dmath*}\breakingcomma
g^{\alpha \beta }
\end{dmath*}

If the Lorentz indices live in different dimensions, this gets resolved
according to the t'Hooft-Veltman-Breitenlohner-Maison prescription

\begin{Shaded}
\begin{Highlighting}[]
\NormalTok{Pair}\OperatorTok{[}\NormalTok{LorentzIndex}\OperatorTok{[}\SpecialCharTok{\textbackslash{}}\OperatorTok{[}\NormalTok{Alpha}\OperatorTok{],} \FunctionTok{n} \SpecialCharTok{{-}} \DecValTok{4}\OperatorTok{],}\NormalTok{ LorentzIndex}\OperatorTok{[}\SpecialCharTok{\textbackslash{}}\OperatorTok{[}\FunctionTok{Beta}\OperatorTok{]]]}
\end{Highlighting}
\end{Shaded}

\begin{dmath*}\breakingcomma
0
\end{dmath*}

A \(4\)-dimensional Lorentz vector

\begin{Shaded}
\begin{Highlighting}[]
\NormalTok{Pair}\OperatorTok{[}\NormalTok{LorentzIndex}\OperatorTok{[}\SpecialCharTok{\textbackslash{}}\OperatorTok{[}\NormalTok{Alpha}\OperatorTok{]],}\NormalTok{ Momentum}\OperatorTok{[}\FunctionTok{p}\OperatorTok{]]}
\end{Highlighting}
\end{Shaded}

\begin{dmath*}\breakingcomma
\overline{p}^{\alpha }
\end{dmath*}

A \(D\)-dimensional Lorentz vector

\begin{Shaded}
\begin{Highlighting}[]
\NormalTok{Pair}\OperatorTok{[}\NormalTok{LorentzIndex}\OperatorTok{[}\SpecialCharTok{\textbackslash{}}\OperatorTok{[}\NormalTok{Alpha}\OperatorTok{],} \FunctionTok{D}\OperatorTok{],}\NormalTok{ Momentum}\OperatorTok{[}\FunctionTok{p}\OperatorTok{,} \FunctionTok{D}\OperatorTok{]]}
\end{Highlighting}
\end{Shaded}

\begin{dmath*}\breakingcomma
p^{\alpha }
\end{dmath*}

\(4\)-dimensional scalar products of Lorentz vectors

\begin{Shaded}
\begin{Highlighting}[]
\NormalTok{Pair}\OperatorTok{[}\NormalTok{Momentum}\OperatorTok{[}\FunctionTok{q}\OperatorTok{],}\NormalTok{ Momentum}\OperatorTok{[}\FunctionTok{p}\OperatorTok{]]}
\end{Highlighting}
\end{Shaded}

\begin{dmath*}\breakingcomma
\overline{p}\cdot \overline{q}
\end{dmath*}

\begin{Shaded}
\begin{Highlighting}[]
\NormalTok{Pair}\OperatorTok{[}\NormalTok{Momentum}\OperatorTok{[}\FunctionTok{p}\OperatorTok{],}\NormalTok{ Momentum}\OperatorTok{[}\FunctionTok{p}\OperatorTok{]]}
\end{Highlighting}
\end{Shaded}

\begin{dmath*}\breakingcomma
\overline{p}^2
\end{dmath*}

\begin{Shaded}
\begin{Highlighting}[]
\NormalTok{Pair}\OperatorTok{[}\NormalTok{Momentum}\OperatorTok{[}\FunctionTok{p} \SpecialCharTok{{-}} \FunctionTok{q}\OperatorTok{],}\NormalTok{ Momentum}\OperatorTok{[}\FunctionTok{p}\OperatorTok{]]}
\end{Highlighting}
\end{Shaded}

\begin{dmath*}\breakingcomma
\overline{p}\cdot (\overline{p}-\overline{q})
\end{dmath*}

\begin{Shaded}
\begin{Highlighting}[]
\NormalTok{Pair}\OperatorTok{[}\NormalTok{Momentum}\OperatorTok{[}\FunctionTok{p}\OperatorTok{],}\NormalTok{ Momentum}\OperatorTok{[}\FunctionTok{p}\OperatorTok{]]}\SpecialCharTok{\^{}}\DecValTok{2}
\end{Highlighting}
\end{Shaded}

\begin{dmath*}\breakingcomma
\overline{p}^4
\end{dmath*}

\begin{Shaded}
\begin{Highlighting}[]
\NormalTok{Pair}\OperatorTok{[}\NormalTok{Momentum}\OperatorTok{[}\FunctionTok{p}\OperatorTok{],}\NormalTok{ Momentum}\OperatorTok{[}\FunctionTok{p}\OperatorTok{]]}\SpecialCharTok{\^{}}\DecValTok{3}
\end{Highlighting}
\end{Shaded}

\begin{dmath*}\breakingcomma
\overline{p}^6
\end{dmath*}

\begin{Shaded}
\begin{Highlighting}[]
\NormalTok{ExpandScalarProduct}\OperatorTok{[}\NormalTok{Pair}\OperatorTok{[}\NormalTok{Momentum}\OperatorTok{[}\FunctionTok{p} \SpecialCharTok{{-}} \FunctionTok{q}\OperatorTok{],}\NormalTok{ Momentum}\OperatorTok{[}\FunctionTok{p}\OperatorTok{]]]}
\end{Highlighting}
\end{Shaded}

\begin{dmath*}\breakingcomma
\overline{p}^2-\overline{p}\cdot \overline{q}
\end{dmath*}

\begin{Shaded}
\begin{Highlighting}[]
\NormalTok{Pair}\OperatorTok{[}\NormalTok{Momentum}\OperatorTok{[}\SpecialCharTok{{-}}\FunctionTok{q}\OperatorTok{],}\NormalTok{ Momentum}\OperatorTok{[}\FunctionTok{p}\OperatorTok{]]} \SpecialCharTok{+}\NormalTok{ Pair}\OperatorTok{[}\NormalTok{Momentum}\OperatorTok{[}\FunctionTok{q}\OperatorTok{],}\NormalTok{ Momentum}\OperatorTok{[}\FunctionTok{p}\OperatorTok{]]}
\end{Highlighting}
\end{Shaded}

\begin{dmath*}\breakingcomma
0
\end{dmath*}
\end{document}
