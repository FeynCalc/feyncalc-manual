% !TeX program = pdflatex
% !TeX root = PairContract.tex

\documentclass[../FeynCalcManual.tex]{subfiles}
\begin{document}
\hypertarget{paircontract}{
\section{PairContract}\label{paircontract}\index{PairContract}}

\texttt{PairContract} is like \texttt{Pair}, but with (local)
contraction properties.

\subsection{See also}

\hyperlink{toc}{Overview}, \hyperlink{pair}{Pair},
\hyperlink{contract}{Contract}.

\subsection{Examples}

\begin{Shaded}
\begin{Highlighting}[]
\NormalTok{Pair}\OperatorTok{[}\NormalTok{LorentzIndex}\OperatorTok{[}\SpecialCharTok{\textbackslash{}}\OperatorTok{[}\NormalTok{Mu}\OperatorTok{]],}\NormalTok{ Momentum}\OperatorTok{[}\FunctionTok{p}\OperatorTok{]]}\NormalTok{ Pair}\OperatorTok{[}\NormalTok{LorentzIndex}\OperatorTok{[}\SpecialCharTok{\textbackslash{}}\OperatorTok{[}\NormalTok{Mu}\OperatorTok{]],}\NormalTok{ Momentum}\OperatorTok{[}\FunctionTok{q}\OperatorTok{]]} 
 
\SpecialCharTok{\%} \OtherTok{/.}\NormalTok{ Pair }\OtherTok{{-}\textgreater{}}\NormalTok{ PairContract}
\end{Highlighting}
\end{Shaded}

\begin{dmath*}\breakingcomma
\overline{p}^{\mu } \overline{q}^{\mu }
\end{dmath*}

\begin{dmath*}\breakingcomma
\overline{p}\cdot \overline{q}
\end{dmath*}

\begin{Shaded}
\begin{Highlighting}[]
\NormalTok{Pair}\OperatorTok{[}\NormalTok{LorentzIndex}\OperatorTok{[}\SpecialCharTok{\textbackslash{}}\OperatorTok{[}\NormalTok{Mu}\OperatorTok{]],}\NormalTok{ Momentum}\OperatorTok{[}\FunctionTok{p}\OperatorTok{]]}\NormalTok{ Pair}\OperatorTok{[}\NormalTok{LorentzIndex}\OperatorTok{[}\SpecialCharTok{\textbackslash{}}\OperatorTok{[}\NormalTok{Nu}\OperatorTok{]],}\NormalTok{ Momentum}\OperatorTok{[}\FunctionTok{q}\OperatorTok{]]}\NormalTok{ Pair}\OperatorTok{[}\NormalTok{LorentzIndex}\OperatorTok{[}\SpecialCharTok{\textbackslash{}}\OperatorTok{[}\NormalTok{Mu}\OperatorTok{]],}\NormalTok{ LorentzIndex}\OperatorTok{[}\SpecialCharTok{\textbackslash{}}\OperatorTok{[}\NormalTok{Nu}\OperatorTok{]]]} 
 
\SpecialCharTok{\%} \OtherTok{/.}\NormalTok{ Pair }\OtherTok{{-}\textgreater{}}\NormalTok{ PairContract}
\end{Highlighting}
\end{Shaded}

\begin{dmath*}\breakingcomma
\overline{p}^{\mu } \overline{q}^{\nu } \bar{g}^{\mu \nu }
\end{dmath*}

\begin{dmath*}\breakingcomma
\overline{p}\cdot \overline{q}
\end{dmath*}
\end{document}
