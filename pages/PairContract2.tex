% !TeX program = pdflatex
% !TeX root = PairContract2.tex

\documentclass[../FeynCalcManual.tex]{subfiles}
\begin{document}
\hypertarget{paircontract2}{
\section{PairContract2}\label{paircontract2}\index{PairContract2}}

\texttt{PairContract2} is like \texttt{Pair}, but with local contraction
properties. It works best with products of \texttt{Pair}s that are
expected to evaluate to a product of scalar products.

\begin{itemize}
\tightlist
\item
  Suitable contractions between products of \texttt{PairContract2}
  symbols are evaluated immediately.
\item
  Momenta are never expanded and every \texttt{PairContract2} symbol
  containing \texttt{Momentum} in both slots is immediately converted to
  a \texttt{Pair}.
\item
  BMHV algebra is not supported, every tensor must be purely \texttt{4}
  or \texttt{D}-dimensional
\end{itemize}

\texttt{PairContract2} is an auxiliary function used in higher level
FeynCalc functions that require fast contractions between multiple
expressions, where \texttt{Contract} would be too slow.

\subsection{See also}

\hyperlink{toc}{Overview}, \hyperlink{contract}{Contract},
\hyperlink{paircontract}{PairContract},
\hyperlink{paircontract3}{PairContract3}.

\subsection{Examples}

\begin{Shaded}
\begin{Highlighting}[]
\NormalTok{Pair}\OperatorTok{[}\NormalTok{LorentzIndex}\OperatorTok{[}\SpecialCharTok{\textbackslash{}}\OperatorTok{[}\NormalTok{Mu}\OperatorTok{]],}\NormalTok{ Momentum}\OperatorTok{[}\FunctionTok{p}\OperatorTok{]]}\NormalTok{ Pair}\OperatorTok{[}\NormalTok{LorentzIndex}\OperatorTok{[}\SpecialCharTok{\textbackslash{}}\OperatorTok{[}\NormalTok{Mu}\OperatorTok{]],}\NormalTok{ Momentum}\OperatorTok{[}\FunctionTok{q}\OperatorTok{]]} 
 
\SpecialCharTok{\%} \OtherTok{/.}\NormalTok{ Pair }\OtherTok{{-}\textgreater{}}\NormalTok{ PairContract2}
\end{Highlighting}
\end{Shaded}

\begin{dmath*}\breakingcomma
\overline{p}^{\mu } \overline{q}^{\mu }
\end{dmath*}

\begin{dmath*}\breakingcomma
\overline{p}\cdot \overline{q}
\end{dmath*}

\begin{Shaded}
\begin{Highlighting}[]
\NormalTok{Pair}\OperatorTok{[}\NormalTok{LorentzIndex}\OperatorTok{[}\SpecialCharTok{\textbackslash{}}\OperatorTok{[}\NormalTok{Mu}\OperatorTok{]],}\NormalTok{ Momentum}\OperatorTok{[}\FunctionTok{p} \SpecialCharTok{+} \FunctionTok{q}\OperatorTok{]]}\NormalTok{ Pair}\OperatorTok{[}\NormalTok{LorentzIndex}\OperatorTok{[}\SpecialCharTok{\textbackslash{}}\OperatorTok{[}\NormalTok{Mu}\OperatorTok{]],}\NormalTok{ Momentum}\OperatorTok{[}\FunctionTok{r} \SpecialCharTok{+} \FunctionTok{s}\OperatorTok{]]} 
 
\SpecialCharTok{\%} \OtherTok{/.}\NormalTok{ Pair }\OtherTok{{-}\textgreater{}}\NormalTok{ PairContract2}
\end{Highlighting}
\end{Shaded}

\begin{dmath*}\breakingcomma
\left(\overline{p}+\overline{q}\right)^{\mu } \left(\overline{r}+\overline{s}\right)^{\mu }
\end{dmath*}

\begin{dmath*}\breakingcomma
(\overline{p}+\overline{q})\cdot (\overline{r}+\overline{s})
\end{dmath*}
\end{document}
