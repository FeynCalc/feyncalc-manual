% !TeX program = pdflatex
% !TeX root = Parallelization.tex

\documentclass[../FeynCalcManual.tex]{subfiles}
\begin{document}
\hypertarget{parallelization}{
\section{Parallelization}\label{parallelization}\index{Parallelization}}

\subsection{See also}

\hyperlink{toc}{Overview}.

Some FeynCalc routines can be parallelized meaning that the code will
try to distribute chunks of the calculation to multiple Mathematica
kernels.

To this aim the number of subkernels should roughly correspond to the
number of CPU cores

\subsection{Enabling parallelization}\label{enabling-parallelization}

To enable the parallelization you need to actively launch some parallel
kernels and then set the variable \texttt{\$ParallelizeFeynCalc} to
\texttt{True}. In this case a copy of FeynCalc will be loaded on each of
the parallel kernels and used to parallelize some selected operations.
For example,

\begin{Shaded}
\begin{Highlighting}[]
\FunctionTok{LaunchKernels}\OperatorTok{[}\DecValTok{8}\OperatorTok{]}
\NormalTok{$ParallelizeFeynCalc }\ExtensionTok{=} \ConstantTok{True}
\end{Highlighting}
\end{Shaded}

Notice that as of now only a small subset of FeynCalc routines supports
parallelization. Every such function has an option
\texttt{FCParallelize} that is set to \texttt{False} by default. To
enable the parallel mode you explicitly need to call the given function
with the option \texttt{FCParallelize->True}.

\subsection{Synchronizing definitions between multiple
kernels}\label{synchronizing-definitions-between-multiple-kernels}

All definitions made via
\texttt{DataType[\allowbreak{}x,\ \allowbreak{}y]=True},
\texttt{ScalarProduct[\allowbreak{}a,\ \allowbreak{}b] = c},
\texttt{SPD[\allowbreak{}a,\ \allowbreak{}b] = c} as well as
\texttt{Commutator} or \texttt{AntiCommutator} \emph{before} activating
the parallel mode must be repeated. This is necessary to ensure that
they are synchronized between the master kernel and the subkernels. To
this aim it is recommended to remove all definitions via

\begin{verbatim}
FCClearScalarProducts[]
FCClearDataTypes[]
UnDeclareAllCommutators[]
UnDeclareAllAntiCommutators[]
\end{verbatim}

and then introduce them again.

Alternatively, you can \emph{first} activate the parallel mode without
making any definitions and \emph{then} define everything as you like.
\texttt{DataType}, \texttt{ScalarProduct}, \texttt{Commutator} and
\texttt{AntiCommutator} will automatically distribute the definitions
among all subkernels if they detect the parallel mode.

\subsection{Functions that support automatic execution on parallel
kernels}\label{functions-that-support-automatic-execution-on-parallel-kernels}

\begin{itemize}
\tightlist
\item
  \texttt{FCLoopFromGLI}
\item
  \texttt{FCFeynmanPrepare}
\item
  \texttt{FCLoopToPakForm}
\item
  \texttt{FCLoopFindIntegralMappings}
\item
  \texttt{FCLoopFindTopologyMappings}
\item
  \texttt{FCLoopFindSubtopologies}
\item
  \texttt{FCLoopFindTopologies}
\item
  \texttt{FCLoopCreatePartialFractioningRules}
\item
  \texttt{FCLoopApplyTopologyMappings}
\item
  \texttt{FCLoopGetKinematicInvariants}
\item
  \texttt{FCLoopTensorReduce}
\item
  \texttt{Tdec}
\item
  \texttt{FCLoopIsolate}
\item
  \texttt{MomentumCombine}
\item
  \texttt{Collect2} (only when isolation is disabled)
\item
  \texttt{FCClearScalarProducts} (parallelization is enabled by default)
\item
  \texttt{ScalarProduct} (parallelization is enabled by default)
\item
  \texttt{DataType} (parallelization is enabled by default)
\item
  \texttt{FCClearDataTypes} (parallelization is enabled by default)
\end{itemize}
\end{document}
