% !TeX program = pdflatex
% !TeX root = Parallelization.tex

\documentclass[../FeynCalcManual.tex]{subfiles}
\begin{document}
\hypertarget{parallelization}{
\section{Parallelization}\label{parallelization}\index{Parallelization}}

\subsection{See also}

\hyperlink{toc}{Overview}.

Some FeynCalc routines can be parallelized meaning that the code will
try to distribute chunks of the calculation to multiple Mathematica
kernels.

To this aim the number of subkernels should roughly correspond to the
number of CPU cores

\subsection{Enabling parallelization}\label{enabling-parallelization}

To enable the parallelization you need to actively launch some parallel
kernels and then set the variable \texttt{\$ParallelizeFeynCalc} to
\texttt{True}. In this case a copy of FeynCalc will be loaded on each of
the parallel kernels and used to parallelize some selected operations.
For example,

\begin{Shaded}
\begin{Highlighting}[]
\FunctionTok{LaunchKernels}\OperatorTok{[}\DecValTok{8}\OperatorTok{]}
\NormalTok{$ParallelizeFeynCalc }\ExtensionTok{=} \ConstantTok{True}
\end{Highlighting}
\end{Shaded}

\subsection{Functions that support automatic execution on parallel
kernels}\label{functions-that-support-automatic-execution-on-parallel-kernels}

\begin{itemize}
\tightlist
\item
  \texttt{FCLoopFromGLI}
\item
  \texttt{FCFeynmanPrepare}
\item
  \texttt{FCLoopToPakForm}
\item
  \texttt{FCLoopFindIntegralMappings}
\item
  \texttt{FCLoopFindTopologyMappings}
\end{itemize}
\end{document}
