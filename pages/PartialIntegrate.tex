% !TeX program = pdflatex
% !TeX root = PartialIntegrate.tex

\documentclass[../FeynCalcManual.tex]{subfiles}
\begin{document}
\hypertarget{partialintegrate}{
\section{PartialIntegrate}\label{partialintegrate}\index{PartialIntegrate}}

\texttt{PartialIntegrate[\allowbreak{}exp,\ \allowbreak{}ap,\ \allowbreak{}t]}
does a partial integration of the definite integral
\texttt{Integrate[\allowbreak{}exp,\ \allowbreak{}\{\allowbreak{}t,\ \allowbreak{}0,\ \allowbreak{}1\}]},
with \texttt{ap} the factor that is to be integrated and \texttt{exp/ap}
the factor that is to be differentiated.

\subsection{See also}

\hyperlink{toc}{Overview},
\hyperlink{integratebyparts}{IntegrateByParts},
\hyperlink{integrate2}{Integrate2}.

\subsection{Examples}

\begin{Shaded}
\begin{Highlighting}[]
\NormalTok{PartialIntegrate}\OperatorTok{[}\FunctionTok{f}\OperatorTok{[}\FunctionTok{x}\OperatorTok{]} \FunctionTok{g}\OperatorTok{[}\FunctionTok{x}\OperatorTok{],} \FunctionTok{g}\OperatorTok{[}\FunctionTok{x}\OperatorTok{],} \OperatorTok{\{}\FunctionTok{x}\OperatorTok{,} \DecValTok{0}\OperatorTok{,} \DecValTok{1}\OperatorTok{\}]}
\end{Highlighting}
\end{Shaded}

\begin{dmath*}\breakingcomma
-(f(x) \int g(x) \, dx\text{/.}\, x\to 0)+(f(x) \int g(x) \, dx\text{/.}\, x\to 1)-\int_0^1 f'(x) (\int g(x) \, dx) \, dx
\end{dmath*}

\begin{Shaded}
\begin{Highlighting}[]
\FunctionTok{f}\OperatorTok{[}\AttributeTok{x\_}\OperatorTok{]} \ExtensionTok{=} \FunctionTok{Integrate}\OperatorTok{[}\FunctionTok{Log}\OperatorTok{[}\DecValTok{3} \FunctionTok{x} \SpecialCharTok{+} \DecValTok{2}\OperatorTok{],} \FunctionTok{x}\OperatorTok{]} 
 
\FunctionTok{g}\OperatorTok{[}\AttributeTok{x\_}\OperatorTok{]} \ExtensionTok{=} \FunctionTok{D}\OperatorTok{[}\DecValTok{1}\SpecialCharTok{/}\FunctionTok{Log}\OperatorTok{[}\DecValTok{3} \FunctionTok{x} \SpecialCharTok{+} \DecValTok{2}\OperatorTok{],} \FunctionTok{x}\OperatorTok{]}
\end{Highlighting}
\end{Shaded}

\begin{dmath*}\breakingcomma
\left(x+\frac{2}{3}\right) \log (3 x+2)-x
\end{dmath*}

\begin{dmath*}\breakingcomma
-\frac{3}{(3 x+2) \log ^2(3 x+2)}
\end{dmath*}

\begin{Shaded}
\begin{Highlighting}[]
\FunctionTok{Integrate}\OperatorTok{[}\NormalTok{PartialIntegrate}\OperatorTok{[}\FunctionTok{f}\OperatorTok{[}\FunctionTok{x}\OperatorTok{]} \FunctionTok{g}\OperatorTok{[}\FunctionTok{x}\OperatorTok{],} \FunctionTok{f}\OperatorTok{[}\FunctionTok{x}\OperatorTok{],} \FunctionTok{x}\OperatorTok{],} \OperatorTok{\{}\FunctionTok{x}\OperatorTok{,} \DecValTok{0}\OperatorTok{,} \DecValTok{1}\OperatorTok{\}]} \SpecialCharTok{//} \FunctionTok{FullSimplify}
\end{Highlighting}
\end{Shaded}

\begin{dmath*}\breakingcomma
-\frac{1}{\log (5)}
\end{dmath*}

\begin{Shaded}
\begin{Highlighting}[]
\FunctionTok{Integrate}\OperatorTok{[}\FunctionTok{f}\OperatorTok{[}\FunctionTok{x}\OperatorTok{]} \FunctionTok{g}\OperatorTok{[}\FunctionTok{x}\OperatorTok{],} \OperatorTok{\{}\FunctionTok{x}\OperatorTok{,} \DecValTok{0}\OperatorTok{,} \DecValTok{1}\OperatorTok{\}]} \SpecialCharTok{//} \FunctionTok{Simplify}
\end{Highlighting}
\end{Shaded}

\begin{dmath*}\breakingcomma
-\frac{1}{\log (5)}
\end{dmath*}

\begin{Shaded}
\begin{Highlighting}[]
\FunctionTok{Clear}\OperatorTok{[}\FunctionTok{f}\OperatorTok{,} \FunctionTok{g}\OperatorTok{]}
\end{Highlighting}
\end{Shaded}

\end{document}
