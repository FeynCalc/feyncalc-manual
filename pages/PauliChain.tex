% !TeX program = pdflatex
% !TeX root = PauliChain.tex

\documentclass[../FeynCalcManual.tex]{subfiles}
\begin{document}
\hypertarget{paulichain}{%
\section{PauliChain}\label{paulichain}}

\texttt{PauliChain[\allowbreak{}x,\ \allowbreak{}i,\ \allowbreak{}j]}
denotes a chain of Pauli matrices \texttt{x}, where the Pauli indices
\texttt{i} and \texttt{j} are explicit.

\subsection{See also}

\hyperlink{toc}{Overview}, \hyperlink{pchn}{PCHN},
\hyperlink{pauliindex}{PauliIndex},
\hyperlink{pauliindexdelta}{PauliIndexDelta},
\hyperlink{paulichainjoin}{PauliChainJoin},
\hyperlink{paulichainexpand}{PauliChainExpand},
\hyperlink{paulichainfactor}{PauliChainFactor}.

\subsection{Examples}

A standalone Pauli matrix \(\sigma^i_{jk}\)

\begin{Shaded}
\begin{Highlighting}[]
\NormalTok{PauliChain}\OperatorTok{[}\NormalTok{PauliSigma}\OperatorTok{[}\NormalTok{CartesianIndex}\OperatorTok{[}\FunctionTok{a}\OperatorTok{]],}\NormalTok{ PauliIndex}\OperatorTok{[}\FunctionTok{i}\OperatorTok{],}\NormalTok{ PauliIndex}\OperatorTok{[}\FunctionTok{j}\OperatorTok{]]}
\end{Highlighting}
\end{Shaded}

\begin{dmath*}\breakingcomma
\left(\overline{\sigma }^a\right){}_{ij}
\end{dmath*}

A chain of Pauli matrices with open indices

\begin{Shaded}
\begin{Highlighting}[]
\NormalTok{PauliChain}\OperatorTok{[}\NormalTok{PauliSigma}\OperatorTok{[}\NormalTok{CartesianIndex}\OperatorTok{[}\FunctionTok{a}\OperatorTok{,} \FunctionTok{D} \SpecialCharTok{{-}} \DecValTok{1}\OperatorTok{],} \FunctionTok{D} \SpecialCharTok{{-}} \DecValTok{1}\OperatorTok{]}\NormalTok{ . PauliSigma}\OperatorTok{[}\NormalTok{CartesianIndex}\OperatorTok{[}\FunctionTok{b}\OperatorTok{,} \FunctionTok{D} \SpecialCharTok{{-}} \DecValTok{1}\OperatorTok{],} \FunctionTok{D} \SpecialCharTok{{-}} \DecValTok{1}\OperatorTok{],}\NormalTok{ PauliIndex}\OperatorTok{[}\FunctionTok{i}\OperatorTok{],}\NormalTok{ PauliIndex}\OperatorTok{[}\FunctionTok{j}\OperatorTok{]]}
\end{Highlighting}
\end{Shaded}

\begin{dmath*}\breakingcomma
\left(\sigma ^a.\sigma ^b\right){}_{ij}
\end{dmath*}

A \texttt{PauliChain} with only two arguments denotes a spinor component

\begin{Shaded}
\begin{Highlighting}[]
\NormalTok{PauliChain}\OperatorTok{[}\NormalTok{PauliXi}\OperatorTok{[}\SpecialCharTok{{-}}\FunctionTok{I}\OperatorTok{],}\NormalTok{ PauliIndex}\OperatorTok{[}\FunctionTok{i}\OperatorTok{]]}
\end{Highlighting}
\end{Shaded}

\begin{dmath*}\breakingcomma
\left(\xi ^{\dagger }\right){}_i
\end{dmath*}

\begin{Shaded}
\begin{Highlighting}[]
\NormalTok{PauliChain}\OperatorTok{[}\NormalTok{PauliEta}\OperatorTok{[}\SpecialCharTok{{-}}\FunctionTok{I}\OperatorTok{],}\NormalTok{ PauliIndex}\OperatorTok{[}\FunctionTok{i}\OperatorTok{]]}
\end{Highlighting}
\end{Shaded}

\begin{dmath*}\breakingcomma
\left(\eta ^{\dagger }\right){}_i
\end{dmath*}

\begin{Shaded}
\begin{Highlighting}[]
\NormalTok{PauliChain}\OperatorTok{[}\NormalTok{PauliIndex}\OperatorTok{[}\FunctionTok{i}\OperatorTok{],}\NormalTok{ PauliXi}\OperatorTok{[}\FunctionTok{I}\OperatorTok{]]}
\end{Highlighting}
\end{Shaded}

\begin{dmath*}\breakingcomma
(\xi )_i
\end{dmath*}

\begin{Shaded}
\begin{Highlighting}[]
\NormalTok{PauliChain}\OperatorTok{[}\NormalTok{PauliIndex}\OperatorTok{[}\FunctionTok{i}\OperatorTok{],}\NormalTok{ PauliEta}\OperatorTok{[}\FunctionTok{I}\OperatorTok{]]}
\end{Highlighting}
\end{Shaded}

\begin{dmath*}\breakingcomma
(\eta )_i
\end{dmath*}

The chain may also be partially open or closed

\begin{Shaded}
\begin{Highlighting}[]
\NormalTok{PauliChain}\OperatorTok{[}\NormalTok{PauliSigma}\OperatorTok{[}\NormalTok{CartesianIndex}\OperatorTok{[}\FunctionTok{a}\OperatorTok{]]}\NormalTok{ . (}\FunctionTok{m} \SpecialCharTok{+}\NormalTok{ PauliSigma}\OperatorTok{[}\NormalTok{CartesianMomentum}\OperatorTok{[}\FunctionTok{p}\OperatorTok{]]}\NormalTok{) . PauliSigma}\OperatorTok{[}\NormalTok{CartesianIndex}\OperatorTok{[}\FunctionTok{b}\OperatorTok{]],}\NormalTok{ PauliXi}\OperatorTok{[}\SpecialCharTok{{-}}\FunctionTok{I}\OperatorTok{],}\NormalTok{ PauliIndex}\OperatorTok{[}\FunctionTok{j}\OperatorTok{]]}
\end{Highlighting}
\end{Shaded}

\begin{dmath*}\breakingcomma
\left(\xi ^{\dagger }.\overline{\sigma }^a.\left(\overline{\sigma }\cdot \overline{p}+m\right).\overline{\sigma }^b\right){}_j
\end{dmath*}

\begin{Shaded}
\begin{Highlighting}[]
\NormalTok{PauliChain}\OperatorTok{[}\NormalTok{PauliSigma}\OperatorTok{[}\NormalTok{CartesianIndex}\OperatorTok{[}\FunctionTok{a}\OperatorTok{]]}\NormalTok{ . (}\FunctionTok{m} \SpecialCharTok{+}\NormalTok{ PauliSigma}\OperatorTok{[}\NormalTok{CartesianMomentum}\OperatorTok{[}\FunctionTok{p}\OperatorTok{]]}\NormalTok{) . PauliSigma}\OperatorTok{[}\NormalTok{CartesianIndex}\OperatorTok{[}\FunctionTok{b}\OperatorTok{]],}\NormalTok{ PauliIndex}\OperatorTok{[}\FunctionTok{i}\OperatorTok{],}\NormalTok{ PauliXi}\OperatorTok{[}\FunctionTok{I}\OperatorTok{]]}
\end{Highlighting}
\end{Shaded}

\begin{dmath*}\breakingcomma
\left(\overline{\sigma }^a.\left(\overline{\sigma }\cdot \overline{p}+m\right).\overline{\sigma }^b.\xi \right){}_i
\end{dmath*}

\begin{Shaded}
\begin{Highlighting}[]
\NormalTok{PauliChain}\OperatorTok{[}\NormalTok{PauliSigma}\OperatorTok{[}\NormalTok{CartesianIndex}\OperatorTok{[}\FunctionTok{a}\OperatorTok{]]}\NormalTok{ . (}\FunctionTok{m} \SpecialCharTok{+}\NormalTok{ PauliSigma}\OperatorTok{[}\NormalTok{CartesianMomentum}\OperatorTok{[}\FunctionTok{p}\OperatorTok{]]}\NormalTok{) . PauliSigma}\OperatorTok{[}\NormalTok{CartesianIndex}\OperatorTok{[}\FunctionTok{b}\OperatorTok{]],}\NormalTok{ PauliXi}\OperatorTok{[}\SpecialCharTok{{-}}\FunctionTok{I}\OperatorTok{],}\NormalTok{ PauliEta}\OperatorTok{[}\FunctionTok{I}\OperatorTok{]]}
\end{Highlighting}
\end{Shaded}

\begin{dmath*}\breakingcomma
\left(\xi ^{\dagger }.\overline{\sigma }^a.\left(\overline{\sigma }\cdot \overline{p}+m\right).\overline{\sigma }^b.\eta \right)
\end{dmath*}

\begin{Shaded}
\begin{Highlighting}[]
\NormalTok{PauliChain}\OperatorTok{[}\DecValTok{1}\OperatorTok{,}\NormalTok{ PauliXi}\OperatorTok{[}\SpecialCharTok{{-}}\FunctionTok{I}\OperatorTok{],}\NormalTok{ PauliEta}\OperatorTok{[}\FunctionTok{I}\OperatorTok{]]}
\end{Highlighting}
\end{Shaded}

\begin{dmath*}\breakingcomma
\left(\xi ^{\dagger }.\eta \right)
\end{dmath*}
\end{document}
