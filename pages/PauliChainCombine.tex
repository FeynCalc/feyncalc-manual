% !TeX program = pdflatex
% !TeX root = PauliChainCombine.tex

\documentclass[../FeynCalcManual.tex]{subfiles}
\begin{document}
\hypertarget{paulichaincombine}{
\section{PauliChainCombine}\label{paulichaincombine}\index{PauliChainCombine}}

\texttt{PauliChainCombine[\allowbreak{}exp]} is (nearly) the inverse
operation to \texttt{PauliChainExpand}.

\subsection{See also}

\hyperlink{toc}{Overview}, \hyperlink{paulichain}{PauliChain},
\hyperlink{pchn}{PCHN}, \hyperlink{pauliindex}{PauliIndex},
\hyperlink{pauliindexdelta}{PauliIndexDelta},
\hyperlink{didelta}{DIDelta},
\hyperlink{paulichainjoin}{PauliChainJoin},
\hyperlink{paulichainexpand}{PauliChainExpand},
\hyperlink{paulichainfactor}{PauliChainFactor}.

\subsection{Examples}

\begin{Shaded}
\begin{Highlighting}[]
\NormalTok{(PCHN}\OperatorTok{[}\NormalTok{CSISD}\OperatorTok{[}\FunctionTok{q}\OperatorTok{],}\NormalTok{ Dir3}\OperatorTok{,}\NormalTok{ Dir4}\OperatorTok{]}\NormalTok{ FAD}\OperatorTok{[\{}\FunctionTok{k}\OperatorTok{,}\NormalTok{ me}\OperatorTok{\}]}\NormalTok{)}\SpecialCharTok{/}\NormalTok{(}\DecValTok{2}\NormalTok{ CSPD}\OperatorTok{[}\FunctionTok{q}\OperatorTok{,} \FunctionTok{q}\OperatorTok{]}\NormalTok{) }\SpecialCharTok{+} \DecValTok{1}\SpecialCharTok{/}\NormalTok{(}\DecValTok{2}\NormalTok{ CSPD}\OperatorTok{[}\FunctionTok{q}\OperatorTok{,} \FunctionTok{q}\OperatorTok{]}\NormalTok{)}\SpecialCharTok{*} 
\NormalTok{    FAD}\OperatorTok{[}\FunctionTok{k}\OperatorTok{,} \OperatorTok{\{}\FunctionTok{k} \SpecialCharTok{{-}} \FunctionTok{q}\OperatorTok{,}\NormalTok{ me}\OperatorTok{\}]}\NormalTok{ (}\SpecialCharTok{{-}}\DecValTok{2}\NormalTok{ DCHN}\OperatorTok{[}\NormalTok{CSISD}\OperatorTok{[}\FunctionTok{q}\OperatorTok{],}\NormalTok{ Dir3}\OperatorTok{,}\NormalTok{ Dir4}\OperatorTok{]}\NormalTok{ CSPD}\OperatorTok{[}\FunctionTok{q}\OperatorTok{,} \FunctionTok{q}\OperatorTok{]} \SpecialCharTok{+} \DecValTok{2}\NormalTok{ DCHN}\OperatorTok{[}\DecValTok{1}\OperatorTok{,}\NormalTok{ Dir3}\OperatorTok{,}\NormalTok{ Dir4}\OperatorTok{]}\SpecialCharTok{*}
\NormalTok{       me CSPD}\OperatorTok{[}\FunctionTok{q}\OperatorTok{,} \FunctionTok{q}\OperatorTok{]} \SpecialCharTok{+}\NormalTok{ DCHN}\OperatorTok{[}\NormalTok{CSISD}\OperatorTok{[}\FunctionTok{q}\OperatorTok{],}\NormalTok{ Dir3}\OperatorTok{,}\NormalTok{ Dir4}\OperatorTok{]}\NormalTok{ (}\SpecialCharTok{{-}}\NormalTok{me}\SpecialCharTok{\^{}}\DecValTok{2} \SpecialCharTok{+}\NormalTok{ CSPD}\OperatorTok{[}\FunctionTok{q}\OperatorTok{,} \FunctionTok{q}\OperatorTok{]}\NormalTok{)) }
 
\NormalTok{PauliChainCombine}\OperatorTok{[}\SpecialCharTok{\%}\OperatorTok{]}
\end{Highlighting}
\end{Shaded}

\begin{dmath*}\breakingcomma
\frac{\left(q^2-\text{me}^2\right) \left((1)_{\text{Dir3}\;\text{Dir4}} \sigma \cdot q\right)+2 \;\text{me} q^2 (1)_{\text{Dir3}\;\text{Dir4}}-2 q^2 \left((1)_{\text{Dir3}\;\text{Dir4}} \sigma \cdot q\right)}{2 q^2 k^2.\left((k-q)^2-\text{me}^2\right)}+\frac{(\sigma \cdot q)_{\text{Dir3}\;\text{Dir4}}}{2 q^2 \left(k^2-\text{me}^2\right)}
\end{dmath*}

\begin{dmath*}\breakingcomma
\frac{(1)_{\text{Dir3}\;\text{Dir4}} \left(q^2-\text{me}^2\right) \sigma \cdot q+2 \;\text{me} q^2 (1)_{\text{Dir3}\;\text{Dir4}}-2 q^2 (1)_{\text{Dir3}\;\text{Dir4}} \sigma \cdot q}{2 q^2 k^2.\left((k-q)^2-\text{me}^2\right)}+\frac{(\sigma \cdot q)_{\text{Dir3}\;\text{Dir4}}}{2 q^2 \left(k^2-\text{me}^2\right)}
\end{dmath*}
\end{document}
