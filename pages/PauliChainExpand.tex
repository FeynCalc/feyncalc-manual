% !TeX program = pdflatex
% !TeX root = PauliChainExpand.tex

\documentclass[../FeynCalcManual.tex]{subfiles}
\begin{document}
\hypertarget{paulichainexpand}{%
\section{PauliChainExpand}\label{paulichainexpand}}

\texttt{PauliChainExpand[\allowbreak{}exp]} expands all Pauli chains
with explicit indices using linearity,
e.g.~\texttt{PCHN[\allowbreak{}CSIS[\allowbreak{}p1]+CSIS[\allowbreak{}p2]+m,\ \allowbreak{}i,\ \allowbreak{}j]}
becomes
\texttt{PCHN[\allowbreak{}CSIS[\allowbreak{}p1],\ \allowbreak{}i,\ \allowbreak{}j]+PCHN[\allowbreak{}CSIS[\allowbreak{}p2],\ \allowbreak{}i,\ \allowbreak{}j]+m*PCHN[\allowbreak{}1,\ \allowbreak{}i,\ \allowbreak{}j]}.

\subsection{See also}

\hyperlink{toc}{Overview}, \hyperlink{paulichain}{PauliChain},
\hyperlink{pchn}{PCHN}, \hyperlink{pauliindex}{PauliIndex},
\hyperlink{pauliindexdelta}{PauliIndexDelta},
\hyperlink{didelta}{DIDelta},
\hyperlink{paulichainjoin}{PauliChainJoin},
\hyperlink{paulichaincombine}{PauliChainCombine},
\hyperlink{paulichainfactor}{PauliChainFactor}.

\subsection{Examples}

\begin{Shaded}
\begin{Highlighting}[]
\NormalTok{PCHN}\OperatorTok{[}\NormalTok{(CSIS}\OperatorTok{[}\FunctionTok{p}\OperatorTok{]} \SpecialCharTok{+} \FunctionTok{m}\NormalTok{) . CSI}\OperatorTok{[}\FunctionTok{a}\OperatorTok{],} \FunctionTok{i}\OperatorTok{,} \FunctionTok{j}\OperatorTok{]} 
 
\NormalTok{PauliChainExpand}\OperatorTok{[}\SpecialCharTok{\%}\OperatorTok{]}
\end{Highlighting}
\end{Shaded}

\begin{dmath*}\breakingcomma
\left(\left(\overline{\sigma }\cdot \overline{p}+m\right).\overline{\sigma }^a\right){}_{ij}
\end{dmath*}

\begin{dmath*}\breakingcomma
m \left(\overline{\sigma }^a\right){}_{ij}+\left(\left(\overline{\sigma }\cdot \overline{p}\right).\overline{\sigma }^a\right){}_{ij}
\end{dmath*}
\end{document}
