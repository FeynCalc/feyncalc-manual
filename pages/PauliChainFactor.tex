% !TeX program = pdflatex
% !TeX root = PauliChainFactor.tex

\documentclass[../FeynCalcManual.tex]{subfiles}
\begin{document}
\hypertarget{paulichainfactor}{
\section{PauliChainFactor}\label{paulichainfactor}\index{PauliChainFactor}}

\texttt{PauliChainFactor[\allowbreak{}exp]} factors out all expressions
inside a \texttt{PauliChain} to which the chain doesn't apply. For
example, all objects that are not Pauli matrices can be safely factored
out from every Pauli chain.

\subsection{See also}

\hyperlink{toc}{Overview}, \hyperlink{paulichain}{PauliChain},
\hyperlink{pchn}{PCHN}, \hyperlink{pauliindex}{PauliIndex},
\hyperlink{pauliindexdelta}{PauliIndexDelta},
\hyperlink{didelta}{DIDelta},
\hyperlink{paulichainjoin}{PauliChainJoin},
\hyperlink{paulichaincombine}{PauliChainCombine},
\hyperlink{paulichainexpand}{PauliChainExpand}.

\subsection{Examples}

\begin{Shaded}
\begin{Highlighting}[]
\NormalTok{PCHN}\OperatorTok{[}\NormalTok{CV}\OperatorTok{[}\FunctionTok{p}\OperatorTok{,} \SpecialCharTok{\textbackslash{}}\OperatorTok{[}\NormalTok{Nu}\OperatorTok{]]}\NormalTok{ CSI}\OperatorTok{[}\FunctionTok{a}\OperatorTok{]}\NormalTok{ . CSI}\OperatorTok{[}\FunctionTok{b}\OperatorTok{]}\NormalTok{ . CSI}\OperatorTok{[}\FunctionTok{a}\OperatorTok{],} \FunctionTok{i}\OperatorTok{,} \FunctionTok{j}\OperatorTok{]} 
 
\NormalTok{PauliChainFactor}\OperatorTok{[}\SpecialCharTok{\%}\OperatorTok{]}
\end{Highlighting}
\end{Shaded}

\begin{dmath*}\breakingcomma
\left(\overline{p}^{\nu } \overline{\sigma }^a.\overline{\sigma }^b.\overline{\sigma }^a\right){}_{ij}
\end{dmath*}

\begin{dmath*}\breakingcomma
\overline{p}^{\nu } \left(\overline{\sigma }^a.\overline{\sigma }^b.\overline{\sigma }^a\right){}_{ij}
\end{dmath*}
\end{document}
