% !TeX program = pdflatex
% !TeX root = PauliChainJoin.tex

\documentclass[../FeynCalcManual.tex]{subfiles}
\begin{document}
\hypertarget{paulichainjoin}{
\section{PauliChainJoin}\label{paulichainjoin}\index{PauliChainJoin}}

\texttt{PauliChainJoin[\allowbreak{}exp]} joins chains of Pauli matrices
with explicit Pauli indices wrapped with a head \texttt{PauliChain}.

\subsection{See also}

\hyperlink{toc}{Overview}, \hyperlink{paulichain}{PauliChain},
\hyperlink{pchn}{PCHN}, \hyperlink{pauliindex}{PauliIndex},
\hyperlink{pauliindexdelta}{PauliIndexDelta},
\hyperlink{didelta}{DIDelta},
\hyperlink{paulichaincombine}{PauliChainCombine},
\hyperlink{paulichainexpand}{PauliChainExpand},
\hyperlink{paulichainfactor}{PauliChainFactor}.

\subsection{Examples}

\begin{Shaded}
\begin{Highlighting}[]
\NormalTok{PCHN}\OperatorTok{[}\NormalTok{PauliXi}\OperatorTok{[}\SpecialCharTok{{-}}\FunctionTok{I}\OperatorTok{],} \FunctionTok{i}\OperatorTok{]}\NormalTok{ PCHN}\OperatorTok{[}\NormalTok{CSID}\OperatorTok{[}\FunctionTok{a}\OperatorTok{]}\NormalTok{ . CSID}\OperatorTok{[}\FunctionTok{b}\OperatorTok{],} \FunctionTok{i}\OperatorTok{,} \FunctionTok{j}\OperatorTok{]}\NormalTok{ PCHN}\OperatorTok{[}\FunctionTok{j}\OperatorTok{,}\NormalTok{ PauliEta}\OperatorTok{[}\FunctionTok{I}\OperatorTok{]]} 
 
\NormalTok{PauliChainJoin}\OperatorTok{[}\SpecialCharTok{\%}\OperatorTok{]}
\end{Highlighting}
\end{Shaded}

\begin{dmath*}\breakingcomma
(\eta )_j \left(\xi ^{\dagger }\right){}_i \left(\sigma ^a.\sigma ^b\right){}_{ij}
\end{dmath*}

\begin{dmath*}\breakingcomma
\xi ^{\dagger }.\sigma ^a.\sigma ^b.\eta
\end{dmath*}
\end{document}
