% !TeX program = pdflatex
% !TeX root = PauliIndex.tex

\documentclass[../FeynCalcManual.tex]{subfiles}
\begin{document}
\hypertarget{pauliindex}{%
\section{PauliIndex}\label{pauliindex}}

\texttt{PauliIndex} is the head of Pauli indices. The internal
representation of a two-dimensional spinorial index \texttt{i} is
\texttt{PauliIndex[\allowbreak{}i]}.

If the first argument is an integer, \texttt{PauliIndex[\allowbreak{}i]}
turns into \texttt{ExplicitPauliIndex[\allowbreak{}i]}.

Pauli indices are the indices that denote the components of Pauli
matrices or spinors. They should not be confused with the Cartesian
indices attached to the Pauli matrices. For example in the case of
\(\sigma_{ij}^{k}\), \(k\) is a Lorentz index, while \(i\) and \(j\) are
Pauli (spinorial) indices.

\subsection{See also}

\hyperlink{toc}{Overview}, \hyperlink{paulichain}{PauliChain},
\hyperlink{pchn}{PCHN},
\hyperlink{explicitpauliindex}{ExplicitPauliIndex},
\hyperlink{pauliindexdelta}{PauliIndexDelta},
\hyperlink{didelta}{DIDelta},
\hyperlink{paulichainjoin}{PauliChainJoin},
\hyperlink{paulichaincombine}{PauliChainCombine},
\hyperlink{paulichainexpand}{PauliChainExpand},
\hyperlink{paulichainfactor}{PauliChainFactor}.

\subsection{Examples}

\begin{Shaded}
\begin{Highlighting}[]
\NormalTok{PauliIndex}\OperatorTok{[}\FunctionTok{i}\OperatorTok{]}
\end{Highlighting}
\end{Shaded}

\begin{dmath*}\breakingcomma
i
\end{dmath*}

\begin{Shaded}
\begin{Highlighting}[]
\NormalTok{PauliIndex}\OperatorTok{[}\FunctionTok{i}\OperatorTok{]} \SpecialCharTok{//} \FunctionTok{StandardForm}

\CommentTok{(*PauliIndex[i]*)}
\end{Highlighting}
\end{Shaded}

\begin{Shaded}
\begin{Highlighting}[]
\NormalTok{PauliIndex}\OperatorTok{[}\DecValTok{2}\OperatorTok{]}
\end{Highlighting}
\end{Shaded}

\begin{dmath*}\breakingcomma
2
\end{dmath*}

\begin{Shaded}
\begin{Highlighting}[]
\NormalTok{PauliIndex}\OperatorTok{[}\DecValTok{2}\OperatorTok{]} \SpecialCharTok{//} \FunctionTok{StandardForm}

\CommentTok{(*ExplicitPauliIndex[2]*)}
\end{Highlighting}
\end{Shaded}

\begin{Shaded}
\begin{Highlighting}[]
\NormalTok{PIDelta}\OperatorTok{[}\FunctionTok{i}\OperatorTok{,} \FunctionTok{j}\OperatorTok{]} \SpecialCharTok{//}\NormalTok{ FCI }\SpecialCharTok{//} \FunctionTok{StandardForm}

\CommentTok{(*PauliIndexDelta[PauliIndex[i], PauliIndex[j]]*)}
\end{Highlighting}
\end{Shaded}

\end{document}
