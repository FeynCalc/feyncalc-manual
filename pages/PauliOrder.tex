% !TeX program = pdflatex
% !TeX root = PauliOrder.tex

\documentclass[../FeynCalcManual.tex]{subfiles}
\begin{document}
\hypertarget{pauliorder}{%
\section{PauliOrder}\label{pauliorder}}

\texttt{PauliOrder[\allowbreak{}exp]} orders the Pauli matrices in
\texttt{expr} alphabetically.

\texttt{PauliOrder[\allowbreak{}exp,\ \allowbreak{}orderlist]} orders
the Pauli matrices in expr according to \texttt{orderlist}.

\subsection{See also}

\hyperlink{toc}{Overview}

\subsection{Examples}

\begin{Shaded}
\begin{Highlighting}[]
\NormalTok{CSI}\OperatorTok{[}\FunctionTok{k}\OperatorTok{,} \FunctionTok{j}\OperatorTok{,} \FunctionTok{i}\OperatorTok{]} 
 
\NormalTok{PauliOrder}\OperatorTok{[}\SpecialCharTok{\%}\OperatorTok{]}
\end{Highlighting}
\end{Shaded}

\begin{dmath*}\breakingcomma
\overline{\sigma }^k.\overline{\sigma }^j.\overline{\sigma }^i
\end{dmath*}

\begin{dmath*}\breakingcomma
2 \overline{\sigma }^i \bar{\delta }^{jk}-2 \overline{\sigma }^j \bar{\delta }^{ik}+2 \overline{\sigma }^k \bar{\delta }^{ij}-\overline{\sigma }^i.\overline{\sigma }^j.\overline{\sigma }^k
\end{dmath*}

\begin{Shaded}
\begin{Highlighting}[]
\NormalTok{CSID}\OperatorTok{[}\FunctionTok{i}\OperatorTok{,} \FunctionTok{j}\OperatorTok{,} \FunctionTok{k}\OperatorTok{]} 
 
\NormalTok{PauliOrder}\OperatorTok{[}\SpecialCharTok{\%}\OperatorTok{]}
\end{Highlighting}
\end{Shaded}

\begin{dmath*}\breakingcomma
\sigma ^i.\sigma ^j.\sigma ^k
\end{dmath*}

\begin{dmath*}\breakingcomma
\sigma ^i.\sigma ^j.\sigma ^k
\end{dmath*}

\begin{Shaded}
\begin{Highlighting}[]
\NormalTok{PauliOrder}\OperatorTok{[}\SpecialCharTok{\%\%}\OperatorTok{,} \OperatorTok{\{}\FunctionTok{j}\OperatorTok{,} \FunctionTok{i}\OperatorTok{,} \FunctionTok{k}\OperatorTok{\}]}
\end{Highlighting}
\end{Shaded}

\begin{dmath*}\breakingcomma
2 \sigma ^k \delta ^{ij}-\sigma ^j.\sigma ^i.\sigma ^k
\end{dmath*}
\end{document}
