% !TeX program = pdflatex
% !TeX root = PauliReduce.tex

\documentclass[../FeynCalcManual.tex]{subfiles}
\begin{document}
\hypertarget{paulireduce}{%
\section{PauliReduce}\label{paulireduce}}

\texttt{PauliReduce} is an option for \texttt{PauliTrick} and other
functions. It specifies whether a chain of Pauli matrices should be
reduced to at most one matrix by rewriting every pair of matrices in
terms of commutator and anticommutator.

\subsection{See also}

\hyperlink{toc}{Overview}, \hyperlink{paulitrick}{PauliTrick},
\hyperlink{paulisimplify}{PauliSimplify}.

\subsection{Examples}

\begin{Shaded}
\begin{Highlighting}[]
\NormalTok{CSI}\OperatorTok{[}\FunctionTok{i}\OperatorTok{,} \FunctionTok{j}\OperatorTok{,} \FunctionTok{k}\OperatorTok{]} 
 
\NormalTok{PauliSimplify}\OperatorTok{[}\SpecialCharTok{\%}\OperatorTok{]}
\end{Highlighting}
\end{Shaded}

\begin{dmath*}\breakingcomma
\overline{\sigma }^i.\overline{\sigma }^j.\overline{\sigma }^k
\end{dmath*}

\begin{dmath*}\breakingcomma
\overline{\sigma }^i.\overline{\sigma }^j.\overline{\sigma }^k
\end{dmath*}

\begin{Shaded}
\begin{Highlighting}[]
\NormalTok{CSI}\OperatorTok{[}\FunctionTok{i}\OperatorTok{,} \FunctionTok{j}\OperatorTok{,} \FunctionTok{k}\OperatorTok{]} 
 
\NormalTok{PauliSimplify}\OperatorTok{[}\SpecialCharTok{\%}\OperatorTok{,}\NormalTok{ PauliReduce }\OtherTok{{-}\textgreater{}} \ConstantTok{True}\OperatorTok{]}
\end{Highlighting}
\end{Shaded}

\begin{dmath*}\breakingcomma
\overline{\sigma }^i.\overline{\sigma }^j.\overline{\sigma }^k
\end{dmath*}

\begin{dmath*}\breakingcomma
\overline{\sigma }^i \bar{\delta }^{jk}-\overline{\sigma }^j \bar{\delta }^{ik}+\overline{\sigma }^k \bar{\delta }^{ij}+i \bar{\epsilon }^{ijk}
\end{dmath*}
\end{document}
