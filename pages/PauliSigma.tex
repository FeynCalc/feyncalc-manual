% !TeX program = pdflatex
% !TeX root = PauliSigma.tex

\documentclass[../FeynCalcManual.tex]{subfiles}
\begin{document}
\hypertarget{paulisigma}{%
\section{PauliSigma}\label{paulisigma}}

\texttt{PauliSigma[\allowbreak{}x,\ \allowbreak{}dim]} is the internal
representation of a Pauli matrix with a Lorentz or Cartesian index or a
contraction of a Pauli matrix and a Lorentz or Cartesian vector.

\texttt{PauliSigma[\allowbreak{}x,\ \allowbreak{}3]} simplifies to
\texttt{PauliSigma[\allowbreak{}x]}.

\subsection{See also}

\hyperlink{toc}{Overview}, \hyperlink{si}{SI}, \hyperlink{csi}{CSI}.

\subsection{Examples}

\begin{Shaded}
\begin{Highlighting}[]
\NormalTok{PauliSigma}\OperatorTok{[}\NormalTok{LorentzIndex}\OperatorTok{[}\SpecialCharTok{\textbackslash{}}\OperatorTok{[}\NormalTok{Alpha}\OperatorTok{]]]}
\end{Highlighting}
\end{Shaded}

\begin{dmath*}\breakingcomma
\bar{\sigma }^{\alpha }
\end{dmath*}

\begin{Shaded}
\begin{Highlighting}[]
\NormalTok{PauliSigma}\OperatorTok{[}\NormalTok{CartesianIndex}\OperatorTok{[}\FunctionTok{i}\OperatorTok{]]}
\end{Highlighting}
\end{Shaded}

\begin{dmath*}\breakingcomma
\overline{\sigma }^i
\end{dmath*}

A Pauli matrix contracted with a Lorentz or Cartesian vector is
displayed as \(\sigma \cdot p\)

\begin{Shaded}
\begin{Highlighting}[]
\NormalTok{PauliSigma}\OperatorTok{[}\NormalTok{Momentum}\OperatorTok{[}\FunctionTok{p}\OperatorTok{]]}
\end{Highlighting}
\end{Shaded}

\begin{dmath*}\breakingcomma
\bar{\sigma }\cdot \overline{p}
\end{dmath*}

\begin{Shaded}
\begin{Highlighting}[]
\NormalTok{PauliSigma}\OperatorTok{[}\NormalTok{CartesianMomentum}\OperatorTok{[}\FunctionTok{p}\OperatorTok{]]}
\end{Highlighting}
\end{Shaded}

\begin{dmath*}\breakingcomma
\overline{\sigma }\cdot \overline{p}
\end{dmath*}

\begin{Shaded}
\begin{Highlighting}[]
\NormalTok{PauliSigma}\OperatorTok{[}\NormalTok{Momentum}\OperatorTok{[}\FunctionTok{q}\OperatorTok{]]}\NormalTok{ . PauliSigma}\OperatorTok{[}\NormalTok{Momentum}\OperatorTok{[}\FunctionTok{p} \SpecialCharTok{{-}} \FunctionTok{q}\OperatorTok{]]} 
 
\SpecialCharTok{\%} \SpecialCharTok{//}\NormalTok{ PauliSigmaExpand}
\end{Highlighting}
\end{Shaded}

\begin{dmath*}\breakingcomma
\left(\bar{\sigma }\cdot \overline{q}\right).\left(\bar{\sigma }\cdot \left(\overline{p}-\overline{q}\right)\right)
\end{dmath*}

\begin{dmath*}\breakingcomma
\left(\bar{\sigma }\cdot \overline{q}\right).\left(\bar{\sigma }\cdot \overline{p}-\bar{\sigma }\cdot \overline{q}\right)
\end{dmath*}

\begin{Shaded}
\begin{Highlighting}[]
\NormalTok{PauliSigma}\OperatorTok{[}\NormalTok{CartesianMomentum}\OperatorTok{[}\FunctionTok{q}\OperatorTok{]]}\NormalTok{ . PauliSigma}\OperatorTok{[}\NormalTok{CartesianMomentum}\OperatorTok{[}\FunctionTok{p} \SpecialCharTok{{-}} \FunctionTok{q}\OperatorTok{]]} 
 
\SpecialCharTok{\%} \SpecialCharTok{//}\NormalTok{ PauliSigmaExpand }
  
 
\end{Highlighting}
\end{Shaded}

\begin{dmath*}\breakingcomma
\left(\overline{\sigma }\cdot \overline{q}\right).\left(\overline{\sigma }\cdot \left(\overline{p}-\overline{q}\right)\right)
\end{dmath*}

\begin{dmath*}\breakingcomma
\left(\overline{\sigma }\cdot \overline{q}\right).\left(\overline{\sigma }\cdot \overline{p}-\overline{\sigma }\cdot \overline{q}\right)
\end{dmath*}
\end{document}
