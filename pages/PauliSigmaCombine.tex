% !TeX program = pdflatex
% !TeX root = PauliSigmaCombine.tex

\documentclass[../FeynCalcManual.tex]{subfiles}
\begin{document}
\hypertarget{paulisigmacombine}{%
\section{PauliSigmaCombine}\label{paulisigmacombine}}

\texttt{PauliSigmaCombine[\allowbreak{}exp]} is (nearly) the inverse
operation to PauliSigmaExpand.

\subsection{See also}

\hyperlink{toc}{Overview},
\hyperlink{paulisigmaexpand}{PauliSigmaExpand}.

\subsection{Examples}

\begin{Shaded}
\begin{Highlighting}[]
\NormalTok{SIS}\OperatorTok{[}\FunctionTok{p}\OperatorTok{]} \SpecialCharTok{+}\NormalTok{ SIS}\OperatorTok{[}\FunctionTok{q}\OperatorTok{]} 
 
\NormalTok{PauliSigmaCombine}\OperatorTok{[}\SpecialCharTok{\%}\OperatorTok{]}
\end{Highlighting}
\end{Shaded}

\begin{dmath*}\breakingcomma
\bar{\sigma }\cdot \overline{p}+\bar{\sigma }\cdot \overline{q}
\end{dmath*}

\begin{dmath*}\breakingcomma
\bar{\sigma }\cdot \left(\overline{p}+\overline{q}\right)
\end{dmath*}

\begin{Shaded}
\begin{Highlighting}[]
\NormalTok{PauliXi}\OperatorTok{[}\SpecialCharTok{{-}}\FunctionTok{I}\OperatorTok{]}\NormalTok{ . (SIS}\OperatorTok{[}\NormalTok{p1 }\SpecialCharTok{+}\NormalTok{ p2}\OperatorTok{]} \SpecialCharTok{+}\NormalTok{ SIS}\OperatorTok{[}\FunctionTok{q}\OperatorTok{]}\NormalTok{) . PauliEta}\OperatorTok{[}\FunctionTok{I}\OperatorTok{]} 
 
\NormalTok{PauliSigmaCombine}\OperatorTok{[}\SpecialCharTok{\%}\OperatorTok{]}
\end{Highlighting}
\end{Shaded}

\begin{dmath*}\breakingcomma
\xi ^{\dagger }.\left(\bar{\sigma }\cdot \left(\overline{\text{p1}}+\overline{\text{p2}}\right)+\bar{\sigma }\cdot \overline{q}\right).\eta
\end{dmath*}

\begin{dmath*}\breakingcomma
\xi ^{\dagger }.\left(\bar{\sigma }\cdot \left(\overline{\text{p1}}+\overline{\text{p2}}+\overline{q}\right)\right).\eta
\end{dmath*}
\end{document}
