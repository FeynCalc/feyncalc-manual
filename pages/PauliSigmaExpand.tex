% !TeX program = pdflatex
% !TeX root = PauliSigmaExpand.tex

\documentclass[../FeynCalcManual.tex]{subfiles}
\begin{document}
\hypertarget{paulisigmaexpand}{%
\section{PauliSigmaExpand}\label{paulisigmaexpand}}

\texttt{PauliSigmaExpand[\allowbreak{}exp]} expands all
\texttt{PauliSigma[\allowbreak{}Momentum[\allowbreak{}a+b+..]]} in
\texttt{exp} into
\texttt{(PauliSigma[\allowbreak{}Momentum[\allowbreak{}a]] + PauliSigma[\allowbreak{}Momentum[\allowbreak{}b]] + ...)}.

\subsection{See also}

\hyperlink{toc}{Overview},
\hyperlink{paulisigmacombine}{PauliSigmaCombine}.

\subsection{Examples}

\begin{Shaded}
\begin{Highlighting}[]
\NormalTok{SIS}\OperatorTok{[}\FunctionTok{q}\OperatorTok{]}\NormalTok{ . SIS}\OperatorTok{[}\FunctionTok{p} \SpecialCharTok{{-}} \FunctionTok{q}\OperatorTok{]} 
 
\NormalTok{PauliSigmaExpand}\OperatorTok{[}\SpecialCharTok{\%}\OperatorTok{]}
\end{Highlighting}
\end{Shaded}

\begin{dmath*}\breakingcomma
\left(\bar{\sigma }\cdot \overline{q}\right).\left(\bar{\sigma }\cdot \left(\overline{p}-\overline{q}\right)\right)
\end{dmath*}

\begin{dmath*}\breakingcomma
\left(\bar{\sigma }\cdot \overline{q}\right).\left(\bar{\sigma }\cdot \overline{p}-\bar{\sigma }\cdot \overline{q}\right)
\end{dmath*}

\begin{Shaded}
\begin{Highlighting}[]
\NormalTok{SIS}\OperatorTok{[}\FunctionTok{a} \SpecialCharTok{+} \FunctionTok{b}\OperatorTok{]}\NormalTok{ . SIS}\OperatorTok{[}\FunctionTok{c} \SpecialCharTok{+} \FunctionTok{d}\OperatorTok{]} 
 
\NormalTok{PauliSigmaExpand}\OperatorTok{[}\SpecialCharTok{\%}\OperatorTok{,}\NormalTok{ Momentum }\OtherTok{{-}\textgreater{}} \OperatorTok{\{}\FunctionTok{a}\OperatorTok{\}]} 
 
\NormalTok{PauliSigmaExpand}\OperatorTok{[}\SpecialCharTok{\%\%}\OperatorTok{,}\NormalTok{ Momentum }\OtherTok{{-}\textgreater{}} \ConstantTok{All}\OperatorTok{]}
\end{Highlighting}
\end{Shaded}

\begin{dmath*}\breakingcomma
\left(\bar{\sigma }\cdot \left(\overline{a}+\overline{b}\right)\right).\left(\bar{\sigma }\cdot \left(\overline{c}+\overline{d}\right)\right)
\end{dmath*}

\begin{dmath*}\breakingcomma
\left(\bar{\sigma }\cdot \overline{a}+\bar{\sigma }\cdot \overline{b}\right).\left(\bar{\sigma }\cdot \left(\overline{c}+\overline{d}\right)\right)
\end{dmath*}

\begin{dmath*}\breakingcomma
\left(\bar{\sigma }\cdot \overline{a}+\bar{\sigma }\cdot \overline{b}\right).\left(\bar{\sigma }\cdot \overline{c}+\bar{\sigma }\cdot \overline{d}\right)
\end{dmath*}
\end{document}
