% !TeX program = pdflatex
% !TeX root = PauliSimplify.tex

\documentclass[../FeynCalcManual.tex]{subfiles}
\begin{document}
\hypertarget{paulisimplify}{
\section{PauliSimplify}\label{paulisimplify}\index{PauliSimplify}}

\texttt{PauliSimplify[\allowbreak{}exp]} simplifies products of Pauli
matrices and expands non-commutative products. Double indices and
vectors are contracted. The order of the Pauli matrices is not changed.

\subsection{See also}

\hyperlink{toc}{Overview}, \hyperlink{paulisigma}{PauliSigma},
\hyperlink{paulitrick}{PauliTrick}.

\subsection{Examples}

\begin{Shaded}
\begin{Highlighting}[]
\NormalTok{CSIS}\OperatorTok{[}\NormalTok{p1}\OperatorTok{]}\NormalTok{ . CSI}\OperatorTok{[}\FunctionTok{i}\OperatorTok{]}\NormalTok{ . CSIS}\OperatorTok{[}\NormalTok{p2}\OperatorTok{]} 
 
\NormalTok{PauliSimplify}\OperatorTok{[}\SpecialCharTok{\%}\OperatorTok{]}
\end{Highlighting}
\end{Shaded}

\begin{dmath*}\breakingcomma
\left(\overline{\sigma }\cdot \overline{\text{p1}}\right).\overline{\sigma }^i.\left(\overline{\sigma }\cdot \overline{\text{p2}}\right)
\end{dmath*}

\begin{dmath*}\breakingcomma
\left(\overline{\sigma }\cdot \overline{\text{p1}}\right).\overline{\sigma }^i.\left(\overline{\sigma }\cdot \overline{\text{p2}}\right)
\end{dmath*}

\begin{Shaded}
\begin{Highlighting}[]
\NormalTok{CSIS}\OperatorTok{[}\FunctionTok{p}\OperatorTok{]}\NormalTok{ . CSI}\OperatorTok{[}\FunctionTok{i}\OperatorTok{,} \FunctionTok{j}\OperatorTok{,} \FunctionTok{k}\OperatorTok{]}\NormalTok{ . CSIS}\OperatorTok{[}\FunctionTok{p}\OperatorTok{]} 
 
\NormalTok{PauliSimplify}\OperatorTok{[}\SpecialCharTok{\%}\OperatorTok{]}
\end{Highlighting}
\end{Shaded}

\begin{dmath*}\breakingcomma
\left(\overline{\sigma }\cdot \overline{p}\right).\overline{\sigma }^i.\overline{\sigma }^j.\overline{\sigma }^k.\left(\overline{\sigma }\cdot \overline{p}\right)
\end{dmath*}

\begin{dmath*}\breakingcomma
-\overline{p}^2 \overline{\sigma }^i.\overline{\sigma }^j.\overline{\sigma }^k+2 \overline{p}^k \overline{\sigma }^i.\overline{\sigma }^j.\left(\overline{\sigma }\cdot \overline{p}\right)-2 \overline{p}^j \overline{\sigma }^i.\overline{\sigma }^k.\left(\overline{\sigma }\cdot \overline{p}\right)+2 \overline{p}^i \overline{\sigma }^j.\overline{\sigma }^k.\left(\overline{\sigma }\cdot \overline{p}\right)
\end{dmath*}

\begin{Shaded}
\begin{Highlighting}[]
\NormalTok{PauliSimplify}\OperatorTok{[}\NormalTok{CSIS}\OperatorTok{[}\FunctionTok{p}\OperatorTok{]}\NormalTok{ . CSI}\OperatorTok{[}\FunctionTok{i}\OperatorTok{,} \FunctionTok{j}\OperatorTok{,} \FunctionTok{k}\OperatorTok{]}\NormalTok{ . CSIS}\OperatorTok{[}\FunctionTok{p}\OperatorTok{],}\NormalTok{ PauliReduce }\OtherTok{{-}\textgreater{}} \ConstantTok{False}\OperatorTok{]}
\end{Highlighting}
\end{Shaded}

\begin{dmath*}\breakingcomma
-\overline{p}^2 \overline{\sigma }^i.\overline{\sigma }^j.\overline{\sigma }^k+2 \overline{p}^k \overline{\sigma }^i.\overline{\sigma }^j.\left(\overline{\sigma }\cdot \overline{p}\right)-2 \overline{p}^j \overline{\sigma }^i.\overline{\sigma }^k.\left(\overline{\sigma }\cdot \overline{p}\right)+2 \overline{p}^i \overline{\sigma }^j.\overline{\sigma }^k.\left(\overline{\sigma }\cdot \overline{p}\right)
\end{dmath*}

\begin{Shaded}
\begin{Highlighting}[]
\NormalTok{CSID}\OperatorTok{[}\FunctionTok{i}\OperatorTok{,} \FunctionTok{j}\OperatorTok{,} \FunctionTok{i}\OperatorTok{]} 
 
\NormalTok{PauliSimplify}\OperatorTok{[}\SpecialCharTok{\%}\OperatorTok{]}
\end{Highlighting}
\end{Shaded}

\begin{dmath*}\breakingcomma
\sigma ^i.\sigma ^j.\sigma ^i
\end{dmath*}

\begin{dmath*}\breakingcomma
3 \sigma ^j-D \sigma ^j
\end{dmath*}

\begin{Shaded}
\begin{Highlighting}[]
\NormalTok{CSID}\OperatorTok{[}\FunctionTok{i}\OperatorTok{,} \FunctionTok{j}\OperatorTok{,} \FunctionTok{k}\OperatorTok{,} \FunctionTok{l}\OperatorTok{,} \FunctionTok{m}\OperatorTok{,} \FunctionTok{i}\OperatorTok{]} 
 
\NormalTok{PauliSimplify}\OperatorTok{[}\SpecialCharTok{\%}\OperatorTok{]}
\end{Highlighting}
\end{Shaded}

\begin{dmath*}\breakingcomma
\sigma ^i.\sigma ^j.\sigma ^k.\sigma ^l.\sigma ^m.\sigma ^i
\end{dmath*}

\begin{dmath*}\breakingcomma
D \sigma ^j.\sigma ^k.\sigma ^l.\sigma ^m-3 \sigma ^j.\sigma ^k.\sigma ^l.\sigma ^m+2 \sigma ^j.\sigma ^k.\sigma ^m.\sigma ^l-2 \sigma ^j.\sigma ^l.\sigma ^m.\sigma ^k+2 \sigma ^k.\sigma ^l.\sigma ^m.\sigma ^j
\end{dmath*}
\end{document}
