% !TeX program = pdflatex
% !TeX root = PauliTrace.tex

\documentclass[../FeynCalcManual.tex]{subfiles}
\begin{document}
\hypertarget{paulitrace}{
\section{PauliTrace}\label{paulitrace}\index{PauliTrace}}

\texttt{PauliTrace[\allowbreak{}exp]} is the head of Pauli traces. By
default the trace is not evaluated. The evaluation occurs only when the
option \texttt{PauliTraceEvaluate} is set to \texttt{True}. It is
recommended to use \texttt{PauliSimplify}, which will automatically
evaluate all Pauli traces in the input expression.

\subsection{See also}

\hyperlink{toc}{Overview}, \hyperlink{paulisimplify}{PauliSimplify}.

\subsection{Examples}

\begin{Shaded}
\begin{Highlighting}[]
\NormalTok{PauliTrace}\OperatorTok{[}\NormalTok{CSI}\OperatorTok{[}\FunctionTok{i}\OperatorTok{,} \FunctionTok{j}\OperatorTok{,} \FunctionTok{k}\OperatorTok{,} \FunctionTok{l}\OperatorTok{]]}
\end{Highlighting}
\end{Shaded}

\begin{dmath*}\breakingcomma
\text{tr}\left(\overline{\sigma }^i.\overline{\sigma }^j.\overline{\sigma }^k.\overline{\sigma }^l\right)
\end{dmath*}

\begin{Shaded}
\begin{Highlighting}[]
\NormalTok{PauliTrace}\OperatorTok{[}\NormalTok{CSI}\OperatorTok{[}\FunctionTok{i}\OperatorTok{,} \FunctionTok{j}\OperatorTok{,} \FunctionTok{k}\OperatorTok{,} \FunctionTok{l}\OperatorTok{],}\NormalTok{ PauliTraceEvaluate }\OtherTok{{-}\textgreater{}} \ConstantTok{True}\OperatorTok{]}
\end{Highlighting}
\end{Shaded}

\begin{dmath*}\breakingcomma
2 \left(\bar{\delta }^{il} \bar{\delta }^{jk}-\bar{\delta }^{ik} \bar{\delta }^{jl}+\bar{\delta }^{ij} \bar{\delta }^{kl}\right)
\end{dmath*}

\begin{Shaded}
\begin{Highlighting}[]
\NormalTok{PauliTrace}\OperatorTok{[}\NormalTok{CSI}\OperatorTok{[}\FunctionTok{i}\OperatorTok{,} \FunctionTok{j}\OperatorTok{,} \FunctionTok{k}\OperatorTok{,} \FunctionTok{l}\OperatorTok{]]} 
 
\SpecialCharTok{\%} \SpecialCharTok{//}\NormalTok{ PauliSimplify }
  
 
\end{Highlighting}
\end{Shaded}

\begin{dmath*}\breakingcomma
\text{tr}\left(\overline{\sigma }^i.\overline{\sigma }^j.\overline{\sigma }^k.\overline{\sigma }^l\right)
\end{dmath*}

\begin{dmath*}\breakingcomma
2 \bar{\delta }^{il} \bar{\delta }^{jk}-2 \bar{\delta }^{ik} \bar{\delta }^{jl}+2 \bar{\delta }^{ij} \bar{\delta }^{kl}
\end{dmath*}
\end{document}
