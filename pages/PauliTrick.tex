% !TeX program = pdflatex
% !TeX root = PauliTrick.tex

\documentclass[../FeynCalcManual.tex]{subfiles}
\begin{document}
\hypertarget{paulitrick}{%
\section{PauliTrick}\label{paulitrick}}

\texttt{PauliTrick[\allowbreak{}exp]} contracts \(\sigma\) matrices with
each other and performs several simplifications (no expansion, use
\texttt{PauliSimplify} for this).

\subsection{See also}

\hyperlink{toc}{Overview}, \hyperlink{paulisigma}{PauliSigma},
\hyperlink{paulisimplify}{PauliSimplify}.

\subsection{Examples}

\begin{Shaded}
\begin{Highlighting}[]
\NormalTok{CSIS}\OperatorTok{[}\NormalTok{p1}\OperatorTok{]}\NormalTok{ . CSI}\OperatorTok{[}\FunctionTok{i}\OperatorTok{]}\NormalTok{ . CSIS}\OperatorTok{[}\NormalTok{p2}\OperatorTok{]} 
 
\NormalTok{PauliTrick}\OperatorTok{[}\SpecialCharTok{\%}\OperatorTok{]} \SpecialCharTok{//}\NormalTok{ Contract}
\end{Highlighting}
\end{Shaded}

\begin{dmath*}\breakingcomma
\left(\overline{\sigma }\cdot \overline{\text{p1}}\right).\overline{\sigma }^i.\left(\overline{\sigma }\cdot \overline{\text{p2}}\right)
\end{dmath*}

\begin{dmath*}\breakingcomma
\left(\overline{\sigma }\cdot \overline{\text{p1}}\right).\overline{\sigma }^i.\left(\overline{\sigma }\cdot \overline{\text{p2}}\right)
\end{dmath*}

\begin{Shaded}
\begin{Highlighting}[]
\NormalTok{CSID}\OperatorTok{[}\FunctionTok{i}\OperatorTok{,} \FunctionTok{j}\OperatorTok{,} \FunctionTok{i}\OperatorTok{]} 
 
\NormalTok{PauliTrick}\OperatorTok{[}\SpecialCharTok{\%}\OperatorTok{]} \SpecialCharTok{//}\NormalTok{ Contract}
\end{Highlighting}
\end{Shaded}

\begin{dmath*}\breakingcomma
\sigma ^i.\sigma ^j.\sigma ^i
\end{dmath*}

\begin{dmath*}\breakingcomma
-\left((D-3) \sigma ^j\right)
\end{dmath*}

\begin{Shaded}
\begin{Highlighting}[]
\NormalTok{CSIS}\OperatorTok{[}\FunctionTok{p}\OperatorTok{]}\NormalTok{ . CSI}\OperatorTok{[}\FunctionTok{j}\OperatorTok{]}\NormalTok{ . CSIS}\OperatorTok{[}\FunctionTok{p}\OperatorTok{]}\NormalTok{ . CSIS}\OperatorTok{[}\FunctionTok{i}\OperatorTok{]} 
 
\NormalTok{PauliTrick}\OperatorTok{[}\SpecialCharTok{\%}\OperatorTok{]} \SpecialCharTok{//}\NormalTok{ Contract }\SpecialCharTok{//}\NormalTok{ EpsEvaluate }\SpecialCharTok{//}\NormalTok{ FCCanonicalizeDummyIndices }
 
\NormalTok{PauliTrick}\OperatorTok{[}\SpecialCharTok{\%\%}\OperatorTok{,}\NormalTok{ PauliReduce }\OtherTok{{-}\textgreater{}} \ConstantTok{False}\OperatorTok{]}
\end{Highlighting}
\end{Shaded}

\begin{dmath*}\breakingcomma
\left(\overline{\sigma }\cdot \overline{p}\right).\overline{\sigma }^j.\left(\overline{\sigma }\cdot \overline{p}\right).\left(\overline{\sigma }\cdot \overline{i}\right)
\end{dmath*}

\begin{dmath*}\breakingcomma
2 \overline{p}^j \left(\overline{\sigma }\cdot \overline{p}\right).\left(\overline{\sigma }\cdot \overline{i}\right)-\overline{p}^2 \overline{\sigma }^j.\left(\overline{\sigma }\cdot \overline{i}\right)
\end{dmath*}

\begin{dmath*}\breakingcomma
2 \overline{p}^j \left(\overline{\sigma }\cdot \overline{p}\right).\left(\overline{\sigma }\cdot \overline{i}\right)-\overline{p}^2 \overline{\sigma }^j.\left(\overline{\sigma }\cdot \overline{i}\right)
\end{dmath*}
\end{document}
