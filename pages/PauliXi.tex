% !TeX program = pdflatex
% !TeX root = PauliXi.tex

\documentclass[../FeynCalcManual.tex]{subfiles}
\begin{document}
\hypertarget{paulixi}{
\section{PauliXi}\label{paulixi}\index{PauliXi}}

\texttt{PauliXi[\allowbreak{}I]} represents a two-component Pauli spinor
\(\xi\), while \texttt{PauliXi[\allowbreak{}-I]} stands for
\(\xi^{\dagger }\).

\subsection{See also}

\hyperlink{toc}{Overview}, \hyperlink{paulieta}{PauliEta}.

\subsection{Examples}

\begin{Shaded}
\begin{Highlighting}[]
\NormalTok{PauliXi}\OperatorTok{[}\FunctionTok{I}\OperatorTok{]}
\end{Highlighting}
\end{Shaded}

\begin{dmath*}\breakingcomma
\xi
\end{dmath*}

\begin{Shaded}
\begin{Highlighting}[]
\NormalTok{PauliXi}\OperatorTok{[}\SpecialCharTok{{-}}\FunctionTok{I}\OperatorTok{]}
\end{Highlighting}
\end{Shaded}

\begin{dmath*}\breakingcomma
\xi ^{\dagger }
\end{dmath*}

\begin{Shaded}
\begin{Highlighting}[]
\NormalTok{PauliXi}\OperatorTok{[}\SpecialCharTok{{-}}\FunctionTok{I}\OperatorTok{]}\NormalTok{ . SIS}\OperatorTok{[}\FunctionTok{p}\OperatorTok{]}\NormalTok{ . PauliEta}\OperatorTok{[}\FunctionTok{I}\OperatorTok{]} 
 
\SpecialCharTok{\%} \SpecialCharTok{//}\NormalTok{ ComplexConjugate}
\end{Highlighting}
\end{Shaded}

\begin{dmath*}\breakingcomma
\xi ^{\dagger }.\left(\bar{\sigma }\cdot \overline{p}\right).\eta
\end{dmath*}

\begin{dmath*}\breakingcomma
\eta ^{\dagger }.\left(\bar{\sigma }\cdot \overline{p}\right).\xi
\end{dmath*}
\end{document}
