% !TeX program = pdflatex
% !TeX root = PlusDistribution.tex

\documentclass[../FeynCalcManual.tex]{subfiles}
\begin{document}
\hypertarget{plusdistribution}{
\section{PlusDistribution}\label{plusdistribution}\index{PlusDistribution}}

\texttt{PlusDistribution[\allowbreak{}1/(1 - x)]} denotes a distribution
(in the sense of the ``+'' prescription).

\subsection{See also}

\hyperlink{toc}{Overview}, \hyperlink{integrate2}{Integrate2}.

\subsection{Examples}

\begin{Shaded}
\begin{Highlighting}[]
\NormalTok{PlusDistribution}\OperatorTok{[}\DecValTok{1}\SpecialCharTok{/}\NormalTok{(}\DecValTok{1} \SpecialCharTok{{-}} \FunctionTok{x}\NormalTok{)}\OperatorTok{]}
\end{Highlighting}
\end{Shaded}

\begin{dmath*}\breakingcomma
\left(\frac{1}{1-x}\right)_+
\end{dmath*}

\begin{Shaded}
\begin{Highlighting}[]
\NormalTok{PlusDistribution}\OperatorTok{[}\FunctionTok{Log}\OperatorTok{[}\DecValTok{1} \SpecialCharTok{{-}} \FunctionTok{x}\OperatorTok{]}\SpecialCharTok{/}\NormalTok{(}\DecValTok{1} \SpecialCharTok{{-}} \FunctionTok{x}\NormalTok{)}\OperatorTok{]}
\end{Highlighting}
\end{Shaded}

\begin{dmath*}\breakingcomma
\left(\frac{\log (1-x)}{1-x}\right)_+
\end{dmath*}

\begin{Shaded}
\begin{Highlighting}[]
\NormalTok{Integrate2}\OperatorTok{[}\NormalTok{PlusDistribution}\OperatorTok{[}\DecValTok{1}\SpecialCharTok{/}\NormalTok{(}\DecValTok{1} \SpecialCharTok{{-}} \FunctionTok{x}\NormalTok{)}\OperatorTok{],} \OperatorTok{\{}\FunctionTok{x}\OperatorTok{,} \DecValTok{0}\OperatorTok{,} \DecValTok{1}\OperatorTok{\}]}
\end{Highlighting}
\end{Shaded}

\begin{dmath*}\breakingcomma
0
\end{dmath*}

\begin{Shaded}
\begin{Highlighting}[]
\NormalTok{Integrate2}\OperatorTok{[}\NormalTok{PlusDistribution}\OperatorTok{[}\FunctionTok{Log}\OperatorTok{[}\DecValTok{1} \SpecialCharTok{{-}} \FunctionTok{x}\OperatorTok{]}\SpecialCharTok{/}\NormalTok{(}\DecValTok{1} \SpecialCharTok{{-}} \FunctionTok{x}\NormalTok{)}\OperatorTok{],} \OperatorTok{\{}\FunctionTok{x}\OperatorTok{,} \DecValTok{0}\OperatorTok{,} \DecValTok{1}\OperatorTok{\}]}
\end{Highlighting}
\end{Shaded}

\begin{dmath*}\breakingcomma
0
\end{dmath*}

\begin{Shaded}
\begin{Highlighting}[]
\NormalTok{Integrate2}\OperatorTok{[}\NormalTok{PlusDistribution}\OperatorTok{[}\FunctionTok{Log}\OperatorTok{[}\DecValTok{1} \SpecialCharTok{{-}} \FunctionTok{x}\OperatorTok{]}\SpecialCharTok{\^{}}\DecValTok{2}\SpecialCharTok{/}\NormalTok{(}\DecValTok{1} \SpecialCharTok{{-}} \FunctionTok{x}\NormalTok{)}\OperatorTok{],} \OperatorTok{\{}\FunctionTok{x}\OperatorTok{,} \DecValTok{0}\OperatorTok{,} \DecValTok{1}\OperatorTok{\}]}
\end{Highlighting}
\end{Shaded}

\begin{dmath*}\breakingcomma
0
\end{dmath*}

\begin{Shaded}
\begin{Highlighting}[]
\NormalTok{PlusDistribution}\OperatorTok{[}\FunctionTok{Log}\OperatorTok{[}\FunctionTok{x}\NormalTok{ (}\DecValTok{1} \SpecialCharTok{{-}} \FunctionTok{x}\NormalTok{)}\OperatorTok{]}\SpecialCharTok{/}\NormalTok{(}\DecValTok{1} \SpecialCharTok{{-}} \FunctionTok{x}\NormalTok{)}\OperatorTok{]}
\end{Highlighting}
\end{Shaded}

\begin{dmath*}\breakingcomma
\frac{\log (x)}{1-x}+\left(\frac{\log (1-x)}{1-x}\right)_+
\end{dmath*}
\end{document}
