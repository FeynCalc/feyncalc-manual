% !TeX program = pdflatex
% !TeX root = Polarization.tex

\documentclass[../FeynCalcManual.tex]{subfiles}
\begin{document}
\hypertarget{polarization}{%
\section{Polarization}\label{polarization}}

\texttt{Polarization[\allowbreak{}k]} is the head of a polarization
momentum with momentum \texttt{k}.

A slashed polarization vector (\(\varepsilon_{\mu}(k) \gamma^\mu)\) has
to be entered as \texttt{GS[\allowbreak{}Polarization[\allowbreak{}k]]}.

Unless the option \texttt{Transversality} is set to \texttt{True}, all
polarization vectors are not transverse by default.

The internal representation for a polarization vector corresponding to a
boson with four momentum \(k\) is:
\texttt{Momentum[\allowbreak{}Polarization[\allowbreak{}k,\ \allowbreak{}I ]]}.

\texttt{Polarization[\allowbreak{}k,\ \allowbreak{}-I]} denotes the
complex conjugate polarization.

Polarization is also an option of various functions related to the
operator product expansion. The setting \texttt{0} denotes the
unpolarized and \texttt{1} the polarized case.

\texttt{Polarization} may appear only inside \texttt{Momentum}. Outside
of \texttt{Momentum} it is meaningless in FeynCalc.

The imaginary unit in the second argument of \texttt{Polarization} is
used to distinguish between incoming and outgoing polarization vectors.

\begin{itemize}
\item
  \texttt{Pair[\allowbreak{}Momentum[\allowbreak{}k],\ \allowbreak{}Momentum[\allowbreak{}Polarization[\allowbreak{}k,\ \allowbreak{}I]]]}
  corresponds to \(\varepsilon^{\mu}(k)\), i.e.~an ingoing polarization
  vector
\item
  \texttt{Pair[\allowbreak{}Momentum[\allowbreak{}k],\ \allowbreak{}Momentum[\allowbreak{}Polarization[\allowbreak{}k,\ \allowbreak{}-I]]]}
  corresponds to \(\varepsilon^{\ast \mu}(k)\), i.e.~an outgoing
  polarization vector
\end{itemize}

\subsection{See also}

\hyperlink{toc}{Overview},
\hyperlink{polarizationvector}{PolarizationVector},
\hyperlink{polarizationsum}{PolarizationSum},
\hyperlink{dopolarizationsums}{DoPolarizationSums}.

\subsection{Examples}

\begin{Shaded}
\begin{Highlighting}[]
\NormalTok{Polarization}\OperatorTok{[}\FunctionTok{k}\OperatorTok{]}
\end{Highlighting}
\end{Shaded}

\begin{dmath*}\breakingcomma
\text{Polarization}(k)
\end{dmath*}

\begin{Shaded}
\begin{Highlighting}[]
\NormalTok{Polarization}\OperatorTok{[}\FunctionTok{k}\OperatorTok{]} \SpecialCharTok{//}\NormalTok{ ComplexConjugate}
\end{Highlighting}
\end{Shaded}

\begin{dmath*}\breakingcomma
\text{Polarization}(k)
\end{dmath*}

\begin{Shaded}
\begin{Highlighting}[]
\NormalTok{GS}\OperatorTok{[}\NormalTok{Polarization}\OperatorTok{[}\FunctionTok{k}\OperatorTok{]]}
\end{Highlighting}
\end{Shaded}

\begin{dmath*}\breakingcomma
\bar{\gamma }\cdot \overline{\text{Polarization}(k)}
\end{dmath*}

\begin{Shaded}
\begin{Highlighting}[]
\NormalTok{GS}\OperatorTok{[}\NormalTok{Polarization}\OperatorTok{[}\FunctionTok{k}\OperatorTok{]]} \SpecialCharTok{//} \FunctionTok{StandardForm}

\CommentTok{(*GS[Polarization[k]]*)}
\end{Highlighting}
\end{Shaded}

\begin{Shaded}
\begin{Highlighting}[]
\NormalTok{Pair}\OperatorTok{[}\NormalTok{Momentum}\OperatorTok{[}\FunctionTok{k}\OperatorTok{],}\NormalTok{ Momentum}\OperatorTok{[}\NormalTok{Polarization}\OperatorTok{[}\FunctionTok{k}\OperatorTok{,} \FunctionTok{I}\OperatorTok{]]]}
\end{Highlighting}
\end{Shaded}

\begin{dmath*}\breakingcomma
\overline{k}\cdot \bar{\varepsilon }(k)
\end{dmath*}
\end{document}
