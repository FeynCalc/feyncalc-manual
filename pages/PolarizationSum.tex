% !TeX program = pdflatex
% !TeX root = PolarizationSum.tex

\documentclass[../FeynCalcManual.tex]{subfiles}
\begin{document}
\hypertarget{polarizationsum}{
\section{PolarizationSum}\label{polarizationsum}\index{PolarizationSum}}

\texttt{PolarizationSum[\allowbreak{}mu,\ \allowbreak{}nu,\ \allowbreak{}... ]}
represents the sum over a polarization vector and its complex conjugate
with two free indices. Depending on its arguments the function returns
different polarization sums for massive or massless vector bosons.

\begin{itemize}
\tightlist
\item
  \texttt{PolarizationSum[\allowbreak{}nu,\ \allowbreak{}nu,\ \allowbreak{}k]}
  returns \(-g^{\mu \nu}+\frac{k^{\mu} k^{\nu}}{k^2}\), i.e.~the sum
  over the 3 physical polarizations of a massive on-shell vector boson
  with \(m = k^2\).
\item
  \texttt{PolarizationSum[\allowbreak{}mu,\ \allowbreak{}nu]} or
  \texttt{PolarizationSum[\allowbreak{}mu,\ \allowbreak{}nu,\ \allowbreak{}k,\ \allowbreak{}0]}
  gives \(-g^{\mu \nu }\). This corresponds to the summation over all
  \(4\) polarizations of a massless vector boson, \(2\) of which are
  unphysical if the particle is on-shell.
\item
  \texttt{PolarizationSum[\allowbreak{}mu,\ \allowbreak{}nu,\ \allowbreak{}k,\ \allowbreak{}n]}
  yields
  \(-g^{\mu \nu}+\frac{k^{\mu }n^{\nu}+k^{\nu }n^{\mu }}{k \cdot n} - \frac{n^2 k^{\mu}k^{\nu}}{(k \cdot n)^2}\)
  which is the so-called axial-gauge polarization sum that picks up only
  the two physical polarizations of a massless vector boson. Here \(n\)
  is an auxiliary vector that must satisfy \(n \cdot k \neq 0\). The
  physical results will not depend on \(n\), yet in practice it is often
  convenient to identify \(n\) with one of the 4-vectors already present
  in the calculation. For example, in a final state with multiple gluons
  denoted by their momenta \(k_i\), the vector \(n\) for the \(i\)-th
  gluon could be a \(k_j\) with \(j \neq i\). Notice that when using
  this polarization sum in a QCD calculation, one doesn't have to
  consider diagrams with ghosts in the final states.
\end{itemize}

To obtain a \(D\)-dimensional polarization sum use the option
\texttt{Dimension}.

If you need to calculate a polarization sum depending on a 4-momentum
that is not on-shell, use the option \texttt{VirtualBoson}.

\subsection{See also}

\hyperlink{toc}{Overview}, \hyperlink{polarization}{Polarization},
\hyperlink{dopolarizationsums}{DoPolarizationSums},
\hyperlink{uncontract}{Uncontract}.

\subsection{Examples}

\begin{Shaded}
\begin{Highlighting}[]
\NormalTok{PolarizationSum}\OperatorTok{[}\SpecialCharTok{\textbackslash{}}\OperatorTok{[}\NormalTok{Mu}\OperatorTok{],} \SpecialCharTok{\textbackslash{}}\OperatorTok{[}\NormalTok{Nu}\OperatorTok{]]}
\end{Highlighting}
\end{Shaded}

\begin{dmath*}\breakingcomma
-\bar{g}^{\mu \nu }
\end{dmath*}

\begin{Shaded}
\begin{Highlighting}[]
\NormalTok{PolarizationSum}\OperatorTok{[}\SpecialCharTok{\textbackslash{}}\OperatorTok{[}\NormalTok{Mu}\OperatorTok{],} \SpecialCharTok{\textbackslash{}}\OperatorTok{[}\NormalTok{Nu}\OperatorTok{],} \FunctionTok{k}\OperatorTok{]}
\end{Highlighting}
\end{Shaded}

\begin{dmath*}\breakingcomma
\frac{\overline{k}^{\mu } \overline{k}^{\nu }}{\overline{k}^2}-\bar{g}^{\mu \nu }
\end{dmath*}

\begin{Shaded}
\begin{Highlighting}[]
\NormalTok{PolarizationSum}\OperatorTok{[}\SpecialCharTok{\textbackslash{}}\OperatorTok{[}\NormalTok{Mu}\OperatorTok{],} \SpecialCharTok{\textbackslash{}}\OperatorTok{[}\NormalTok{Nu}\OperatorTok{],} \FunctionTok{k}\OperatorTok{,}\NormalTok{ Dimension }\OtherTok{{-}\textgreater{}} \FunctionTok{D}\OperatorTok{]}
\end{Highlighting}
\end{Shaded}

\begin{dmath*}\breakingcomma
\frac{k^{\mu } k^{\nu }}{k^2}-g^{\mu \nu }
\end{dmath*}

\begin{Shaded}
\begin{Highlighting}[]
\NormalTok{FCClearScalarProducts}\OperatorTok{[]}\NormalTok{; SP}\OperatorTok{[}\FunctionTok{k}\OperatorTok{]} \ExtensionTok{=} \DecValTok{0}\NormalTok{; }
 
\NormalTok{PolarizationSum}\OperatorTok{[}\SpecialCharTok{\textbackslash{}}\OperatorTok{[}\NormalTok{Mu}\OperatorTok{],} \SpecialCharTok{\textbackslash{}}\OperatorTok{[}\NormalTok{Nu}\OperatorTok{],} \FunctionTok{k}\OperatorTok{,} \FunctionTok{n}\OperatorTok{]}
\end{Highlighting}
\end{Shaded}

\begin{dmath*}\breakingcomma
-\frac{\overline{n}^2 \overline{k}^{\mu } \overline{k}^{\nu }}{(\overline{k}\cdot \overline{n})^2}-\bar{g}^{\mu \nu }+\frac{\overline{k}^{\nu } \overline{n}^{\mu }}{\overline{k}\cdot \overline{n}}+\frac{\overline{k}^{\mu } \overline{n}^{\nu }}{\overline{k}\cdot \overline{n}}
\end{dmath*}

\begin{Shaded}
\begin{Highlighting}[]
\NormalTok{FCClearScalarProducts}\OperatorTok{[]} 
 
\NormalTok{PolarizationSum}\OperatorTok{[}\SpecialCharTok{\textbackslash{}}\OperatorTok{[}\NormalTok{Mu}\OperatorTok{],} \SpecialCharTok{\textbackslash{}}\OperatorTok{[}\NormalTok{Nu}\OperatorTok{],} \FunctionTok{k}\OperatorTok{,} \DecValTok{0}\OperatorTok{,}\NormalTok{ Dimension }\OtherTok{{-}\textgreater{}} \FunctionTok{D}\OperatorTok{]}
\end{Highlighting}
\end{Shaded}

\FloatBarrier
\begin{figure}[!ht]
\centering
\includegraphics[width=0.6\linewidth]{img/04px53fl0bwhw.pdf}
\end{figure}
\FloatBarrier

\begin{dmath*}\breakingcomma
-g^{\mu \nu }
\end{dmath*}

\begin{Shaded}
\begin{Highlighting}[]
\NormalTok{FCClearScalarProducts}\OperatorTok{[]} 
 
\NormalTok{PolarizationSum}\OperatorTok{[}\SpecialCharTok{\textbackslash{}}\OperatorTok{[}\NormalTok{Mu}\OperatorTok{],} \SpecialCharTok{\textbackslash{}}\OperatorTok{[}\NormalTok{Nu}\OperatorTok{],} \FunctionTok{k}\OperatorTok{,} \DecValTok{0}\OperatorTok{,}\NormalTok{ Dimension }\OtherTok{{-}\textgreater{}} \FunctionTok{D}\OperatorTok{,}\NormalTok{ VirtualBoson }\OtherTok{{-}\textgreater{}} \ConstantTok{True}\OperatorTok{]}
\end{Highlighting}
\end{Shaded}

\begin{dmath*}\breakingcomma
-g^{\mu \nu }
\end{dmath*}
\end{document}
