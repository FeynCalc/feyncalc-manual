% !TeX program = pdflatex
% !TeX root = PolarizationVector.tex

\documentclass[../FeynCalcManual.tex]{subfiles}
\begin{document}
\hypertarget{polarizationvector}{%
\section{PolarizationVector}\label{polarizationvector}}

\texttt{PolarizationVector[\allowbreak{}p,\ \allowbreak{}mu]} denotes a
4-dimensional ingoing polarization vector \(\varepsilon^\mu(p)\).

For the outgoing polarization vector \(\varepsilon^{\ast \mu}(p)\) use
\texttt{ComplexConjugate[\allowbreak{}PolarizationVector[\allowbreak{}p,\ \allowbreak{}mu]]}

To obtain a \(D\)-dimensional polarization vector, just use
\texttt{ChangeDimension[\allowbreak{}vec,\ \allowbreak{}D]}

In the internal representation following conventions are used

\begin{itemize}
\item
  \texttt{Pair[\allowbreak{}Momentum[\allowbreak{}k],\ \allowbreak{}Momentum[\allowbreak{}Polarization[\allowbreak{}k,\ \allowbreak{}I]]]}
  corresponds to \(\varepsilon^{\mu}(k)\), i.e.~an ingoing polarization
  vector
\item
  \texttt{Pair[\allowbreak{}Momentum[\allowbreak{}k],\ \allowbreak{}Momentum[\allowbreak{}Polarization[\allowbreak{}k,\ \allowbreak{}-I]]]}
  corresponds to \(\varepsilon^{\ast \mu}(k)\), i.e.~an outgoing
  polarization vector
\end{itemize}

Warning: The first argument of \texttt{PolarizationVector} should always
be a standalone symbol denoting a momentum (e.g.~\texttt{p},
\texttt{k1}, \texttt{q2} etc.). Never use symbols multiplied by \(-1\)
or other numbers as well as products of symbols (e.g.~\texttt{-p},
\texttt{2*k}, \texttt{x*p} etc.). Doing so will inevitably lead to wrong
results.

\subsection{See also}

\hyperlink{toc}{Overview}, \hyperlink{fv}{FV}, \hyperlink{pair}{Pair},
\hyperlink{polarization}{Polarization},
\hyperlink{polarizationsum}{PolarizationSum},
\hyperlink{dopolarizationsums}{DoPolarizationSums}.

\subsection{Examples}

A polarization vector \(\varepsilon^{\mu }(k)\) is a special
\(4\)-vector.

\begin{Shaded}
\begin{Highlighting}[]
\NormalTok{PolarizationVector}\OperatorTok{[}\FunctionTok{k}\OperatorTok{,} \SpecialCharTok{\textbackslash{}}\OperatorTok{[}\NormalTok{Mu}\OperatorTok{]]}
\end{Highlighting}
\end{Shaded}

\begin{dmath*}\breakingcomma
\bar{\varepsilon }^{\mu }(k)
\end{dmath*}

\begin{Shaded}
\begin{Highlighting}[]
\NormalTok{PolarizationVector}\OperatorTok{[}\FunctionTok{k}\OperatorTok{,} \SpecialCharTok{\textbackslash{}}\OperatorTok{[}\NormalTok{Mu}\OperatorTok{]]} \SpecialCharTok{//} \FunctionTok{StandardForm}

\CommentTok{(*Pair[LorentzIndex[\textbackslash{}[Mu]], Momentum[Polarization[k, I]]]*)}
\end{Highlighting}
\end{Shaded}

\begin{Shaded}
\begin{Highlighting}[]
\FunctionTok{Conjugate}\OperatorTok{[}\NormalTok{PolarizationVector}\OperatorTok{[}\FunctionTok{k}\OperatorTok{,} \SpecialCharTok{\textbackslash{}}\OperatorTok{[}\NormalTok{Mu}\OperatorTok{]]]}
\end{Highlighting}
\end{Shaded}

\begin{dmath*}\breakingcomma
\bar{\varepsilon }^{*\mu }(k)
\end{dmath*}

\begin{Shaded}
\begin{Highlighting}[]
\FunctionTok{Conjugate}\OperatorTok{[}\NormalTok{PolarizationVector}\OperatorTok{[}\FunctionTok{k}\OperatorTok{,} \SpecialCharTok{\textbackslash{}}\OperatorTok{[}\NormalTok{Mu}\OperatorTok{]]]} \SpecialCharTok{//} \FunctionTok{StandardForm}

\CommentTok{(*Pair[LorentzIndex[\textbackslash{}[Mu]], Momentum[Polarization[k, {-}I]]]*)}
\end{Highlighting}
\end{Shaded}

The transversality property is not automatic and must be explicitly
activated using the option \texttt{Transversality}

\begin{Shaded}
\begin{Highlighting}[]
\NormalTok{PolarizationVector}\OperatorTok{[}\FunctionTok{k}\OperatorTok{,} \SpecialCharTok{\textbackslash{}}\OperatorTok{[}\NormalTok{Mu}\OperatorTok{]]}\NormalTok{ FV}\OperatorTok{[}\FunctionTok{k}\OperatorTok{,} \SpecialCharTok{\textbackslash{}}\OperatorTok{[}\NormalTok{Mu}\OperatorTok{]]} 
 
\NormalTok{Contract}\OperatorTok{[}\SpecialCharTok{\%}\OperatorTok{]}
\end{Highlighting}
\end{Shaded}

\begin{dmath*}\breakingcomma
\overline{k}^{\mu } \bar{\varepsilon }^{\mu }(k)
\end{dmath*}

\begin{dmath*}\breakingcomma
\overline{k}\cdot \bar{\varepsilon }(k)
\end{dmath*}

\begin{Shaded}
\begin{Highlighting}[]
\NormalTok{PolarizationVector}\OperatorTok{[}\FunctionTok{k}\OperatorTok{,} \SpecialCharTok{\textbackslash{}}\OperatorTok{[}\NormalTok{Mu}\OperatorTok{],}\NormalTok{ Transversality }\OtherTok{{-}\textgreater{}} \ConstantTok{True}\OperatorTok{]}\NormalTok{ FV}\OperatorTok{[}\FunctionTok{k}\OperatorTok{,} \SpecialCharTok{\textbackslash{}}\OperatorTok{[}\NormalTok{Mu}\OperatorTok{]]} 
 
\NormalTok{Contract}\OperatorTok{[}\SpecialCharTok{\%}\OperatorTok{]}
\end{Highlighting}
\end{Shaded}

\begin{dmath*}\breakingcomma
\overline{k}^{\mu } \bar{\varepsilon }^{\mu }(k)
\end{dmath*}

\begin{dmath*}\breakingcomma
0
\end{dmath*}

Suppose that you are using unphysical polarization vectors for massless
gauge bosons and intend to remove the unphysical degrees of freedom at a
later stage using ghosts. In this case you must not use
\texttt{Transversality->True}, since your polarization vectors are not
transverse. Otherwise the result will be inconsistent.

Here everything is correct, we can use the gauge trick with unphysical
polarization vectors.

\begin{Shaded}
\begin{Highlighting}[]
\NormalTok{FCClearScalarProducts}\OperatorTok{[]}\NormalTok{; }
 
\NormalTok{SP}\OperatorTok{[}\NormalTok{k1}\OperatorTok{]} \ExtensionTok{=} \DecValTok{0}\NormalTok{; }
 
\NormalTok{SP}\OperatorTok{[}\NormalTok{k2}\OperatorTok{]} \ExtensionTok{=} \DecValTok{0}\NormalTok{;}
\end{Highlighting}
\end{Shaded}

\begin{Shaded}
\begin{Highlighting}[]
\NormalTok{ex1 }\ExtensionTok{=}\NormalTok{ SP}\OperatorTok{[}\NormalTok{k1}\OperatorTok{,}\NormalTok{ Polarization}\OperatorTok{[}\NormalTok{k1}\OperatorTok{,} \FunctionTok{I}\OperatorTok{]]}\NormalTok{ SP}\OperatorTok{[}\NormalTok{k2}\OperatorTok{,}\NormalTok{ Polarization}\OperatorTok{[}\NormalTok{k1}\OperatorTok{,} \SpecialCharTok{{-}}\FunctionTok{I}\OperatorTok{]]}\NormalTok{ SP}\OperatorTok{[}\NormalTok{k1}\OperatorTok{,}\NormalTok{ Polarization}\OperatorTok{[}\NormalTok{k2}\OperatorTok{,} \FunctionTok{I}\OperatorTok{]]}\SpecialCharTok{*}
\NormalTok{   SP}\OperatorTok{[}\NormalTok{k2}\OperatorTok{,}\NormalTok{ Polarization}\OperatorTok{[}\NormalTok{k2}\OperatorTok{,} \SpecialCharTok{{-}}\FunctionTok{I}\OperatorTok{]]}
\end{Highlighting}
\end{Shaded}

\begin{dmath*}\breakingcomma
\left(\overline{\text{k1}}\cdot \bar{\varepsilon }(\text{k1})\right) \left(\overline{\text{k2}}\cdot \bar{\varepsilon }^*(\text{k2})\right) \left(\overline{\text{k1}}\cdot \bar{\varepsilon }(\text{k2})\right) \left(\overline{\text{k2}}\cdot \bar{\varepsilon }^*(\text{k1})\right)
\end{dmath*}

\begin{Shaded}
\begin{Highlighting}[]
\NormalTok{ex1 }\SpecialCharTok{//}\NormalTok{ DoPolarizationSums}\OperatorTok{[}\NormalTok{\#}\OperatorTok{,}\NormalTok{ k1}\OperatorTok{,} \DecValTok{0}\OperatorTok{]}\NormalTok{ \& }\SpecialCharTok{//}\NormalTok{ DoPolarizationSums}\OperatorTok{[}\NormalTok{\#}\OperatorTok{,}\NormalTok{ k2}\OperatorTok{,} \DecValTok{0}\OperatorTok{]}\NormalTok{ \&}
\end{Highlighting}
\end{Shaded}

\begin{dmath*}\breakingcomma
(\overline{\text{k1}}\cdot \overline{\text{k2}})^2
\end{dmath*}

Here we erroneously set \texttt{Transversality->True} and consequently
obtain a wrong result. In pure QED the full result (physical amplitude
squared) would still come out right owing to the Ward identities, but
e.g.~in QCD this would not be the case.

\begin{Shaded}
\begin{Highlighting}[]
\NormalTok{ex2 }\ExtensionTok{=}\NormalTok{ SP}\OperatorTok{[}\NormalTok{k1}\OperatorTok{,}\NormalTok{ Polarization}\OperatorTok{[}\NormalTok{k1}\OperatorTok{,} \FunctionTok{I}\OperatorTok{,}\NormalTok{ Transversality }\OtherTok{{-}\textgreater{}} \ConstantTok{True}\OperatorTok{]]}\NormalTok{ SP}\OperatorTok{[}\NormalTok{k2}\OperatorTok{,}\NormalTok{ Polarization}\OperatorTok{[}\NormalTok{k1}\OperatorTok{,} \SpecialCharTok{{-}}\FunctionTok{I}\OperatorTok{,} 
\NormalTok{      Transversality }\OtherTok{{-}\textgreater{}} \ConstantTok{True}\OperatorTok{]]}\NormalTok{ SP}\OperatorTok{[}\NormalTok{k1}\OperatorTok{,}\NormalTok{ Polarization}\OperatorTok{[}\NormalTok{k2}\OperatorTok{,} \FunctionTok{I}\OperatorTok{,}\NormalTok{ Transversality }\OtherTok{{-}\textgreater{}} \ConstantTok{True}\OperatorTok{]]}\NormalTok{ SP}\OperatorTok{[}\NormalTok{k2}\OperatorTok{,} 
\NormalTok{     Polarization}\OperatorTok{[}\NormalTok{k2}\OperatorTok{,} \SpecialCharTok{{-}}\FunctionTok{I}\OperatorTok{,}\NormalTok{ Transversality }\OtherTok{{-}\textgreater{}} \ConstantTok{True}\OperatorTok{]]} \SpecialCharTok{//}\NormalTok{ FCI}
\end{Highlighting}
\end{Shaded}

\begin{dmath*}\breakingcomma
0
\end{dmath*}

\begin{Shaded}
\begin{Highlighting}[]
\NormalTok{ex2 }\SpecialCharTok{//}\NormalTok{ DoPolarizationSums}\OperatorTok{[}\NormalTok{\#}\OperatorTok{,}\NormalTok{ k1}\OperatorTok{,} \DecValTok{0}\OperatorTok{]}\NormalTok{ \& }\SpecialCharTok{//}\NormalTok{ DoPolarizationSums}\OperatorTok{[}\NormalTok{\#}\OperatorTok{,}\NormalTok{ k2}\OperatorTok{,} \DecValTok{0}\OperatorTok{]}\NormalTok{ \&}
\end{Highlighting}
\end{Shaded}

\begin{dmath*}\breakingcomma
0
\end{dmath*}

\begin{Shaded}
\begin{Highlighting}[]
\NormalTok{FCClearScalarProducts}\OperatorTok{[]}\NormalTok{;}
\end{Highlighting}
\end{Shaded}

\texttt{PolarizationVector} is a shortcut for \(4\)-dimensional
polarization vectors. Although \(D\)-dimensional polarization vectors
are fully supported by FeynCalc, as of now there is no shortcut for
entering such quantities. You can either use \texttt{ChangeDimension}

\begin{Shaded}
\begin{Highlighting}[]
\NormalTok{ChangeDimension}\OperatorTok{[}\NormalTok{PolarizationVector}\OperatorTok{[}\FunctionTok{q}\OperatorTok{,} \SpecialCharTok{\textbackslash{}}\OperatorTok{[}\NormalTok{Mu}\OperatorTok{]],} \FunctionTok{D}\OperatorTok{]}
\end{Highlighting}
\end{Shaded}

\begin{dmath*}\breakingcomma
\varepsilon ^{\mu }(q)
\end{dmath*}

or enter such quantities directly using the
\texttt{FeynCalcInternal}-notation

\begin{Shaded}
\begin{Highlighting}[]
\NormalTok{Pair}\OperatorTok{[}\NormalTok{Momentum}\OperatorTok{[}\NormalTok{Polarization}\OperatorTok{[}\FunctionTok{q}\OperatorTok{,} \FunctionTok{I}\OperatorTok{],} \FunctionTok{D}\OperatorTok{],}\NormalTok{ LorentzIndex}\OperatorTok{[}\SpecialCharTok{\textbackslash{}}\OperatorTok{[}\NormalTok{Mu}\OperatorTok{],} \FunctionTok{D}\OperatorTok{]]}
\end{Highlighting}
\end{Shaded}

\begin{dmath*}\breakingcomma
\varepsilon ^{\mu }(q)
\end{dmath*}
\end{document}
