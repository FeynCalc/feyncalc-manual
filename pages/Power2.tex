% !TeX program = pdflatex
% !TeX root = Power2.tex

\documentclass[../FeynCalcManual.tex]{subfiles}
\begin{document}
\hypertarget{power2}{%
\section{Power2}\label{power2}}

\texttt{Power2[\allowbreak{}x,\ \allowbreak{}y]} represents
\texttt{x^y}. Sometimes \texttt{Power2} is more useful than the
Mathematica \texttt{Power}.
\texttt{Power2[\allowbreak{}-a,\ \allowbreak{}b]} simplifies to
\texttt{(-1)^b Power2[\allowbreak{}a,\ \allowbreak{}b]} (if no
\texttt{Epsilon} is in \texttt{b} \ldots).

\subsection{See also}

\hyperlink{toc}{Overview}, \hyperlink{powerfactor}{PowerFactor}.

\subsection{Examples}

\begin{Shaded}
\begin{Highlighting}[]
\FunctionTok{Power}\OperatorTok{[}\SpecialCharTok{{-}}\FunctionTok{a}\OperatorTok{,} \FunctionTok{b}\OperatorTok{]}
\end{Highlighting}
\end{Shaded}

\begin{dmath*}\breakingcomma
(-a)^b
\end{dmath*}

\begin{Shaded}
\begin{Highlighting}[]
\NormalTok{Power2}\OperatorTok{[}\SpecialCharTok{{-}}\FunctionTok{a}\OperatorTok{,} \FunctionTok{b}\OperatorTok{]}
\end{Highlighting}
\end{Shaded}

\begin{dmath*}\breakingcomma
(-1)^b a^b
\end{dmath*}
\end{document}
