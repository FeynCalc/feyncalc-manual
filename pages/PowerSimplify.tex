% !TeX program = pdflatex
% !TeX root = PowerSimplify.tex

\documentclass[../FeynCalcManual.tex]{subfiles}
\begin{document}
\hypertarget{powersimplify}{
\section{PowerSimplify}\label{powersimplify}\index{PowerSimplify}}

\texttt{PowerSimplify[\allowbreak{}exp]} simplifies \texttt{(-x)^a} to
\texttt{(-1)^a x^a} and \texttt{(y-x)^n} to \texttt{(-1)^n (x-y)^n} thus
assuming that the exponent is an integer (even if it is symbolic).

Furthermore, \texttt{(-1)^(a+n)} and \texttt{I^(a+n)} are expanded and
\texttt{(I)^(2 m) -> (-1)^m and (-1)^(n_Integer?EvenQ m) -> 1} and
\texttt{(-1)^(n_Integer?OddQ m) -> (-1)^m} for \texttt{n} even and odd
respectively and (-1)\^{}(-n) -\textgreater{} (-1)\^{}n and Exp{[}I m
Pi{]} -\textgreater{} (-1)\^{}m.

\subsection{See also}

\hyperlink{toc}{Overview}, \hyperlink{datatype}{DataType},
\hyperlink{opem}{OPEm}.

\subsection{Examples}

\begin{Shaded}
\begin{Highlighting}[]
\NormalTok{PowerSimplify}\OperatorTok{[}\NormalTok{(}\SpecialCharTok{{-}}\DecValTok{1}\NormalTok{)}\SpecialCharTok{\^{}}\NormalTok{(}\DecValTok{2}\NormalTok{ OPEm)}\OperatorTok{]}
\end{Highlighting}
\end{Shaded}

\begin{dmath*}\breakingcomma
1
\end{dmath*}

\begin{Shaded}
\begin{Highlighting}[]
\NormalTok{PowerSimplify}\OperatorTok{[}\NormalTok{(}\SpecialCharTok{{-}}\DecValTok{1}\NormalTok{)}\SpecialCharTok{\^{}}\NormalTok{(OPEm }\SpecialCharTok{+} \DecValTok{2}\NormalTok{)}\OperatorTok{]}
\end{Highlighting}
\end{Shaded}

\begin{dmath*}\breakingcomma
(-1)^m
\end{dmath*}

\begin{Shaded}
\begin{Highlighting}[]
\NormalTok{PowerSimplify}\OperatorTok{[}\NormalTok{(}\SpecialCharTok{{-}}\DecValTok{1}\NormalTok{)}\SpecialCharTok{\^{}}\NormalTok{(OPEm }\SpecialCharTok{{-}} \DecValTok{2}\NormalTok{)}\OperatorTok{]}
\end{Highlighting}
\end{Shaded}

\begin{dmath*}\breakingcomma
(-1)^m
\end{dmath*}

\begin{Shaded}
\begin{Highlighting}[]
\NormalTok{PowerSimplify}\OperatorTok{[}\FunctionTok{I}\SpecialCharTok{\^{}}\NormalTok{(}\DecValTok{2}\NormalTok{ OPEm)}\OperatorTok{]}
\end{Highlighting}
\end{Shaded}

\begin{dmath*}\breakingcomma
(-1)^m
\end{dmath*}
\end{document}
