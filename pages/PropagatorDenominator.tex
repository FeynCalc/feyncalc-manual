% !TeX program = pdflatex
% !TeX root = PropagatorDenominator.tex

\documentclass[../FeynCalcManual.tex]{subfiles}
\begin{document}
\hypertarget{propagatordenominator}{
\section{PropagatorDenominator}\label{propagatordenominator}\index{PropagatorDenominator}}

\texttt{PropagatorDenominator[\allowbreak{}Momentum[\allowbreak{}q],\ \allowbreak{}m]}
is a factor of the denominator of a propagator. If \texttt{q} is
supposed to be \(D\)-dimensional, use
\texttt{PropagatorDenominator[\allowbreak{}Momentum[\allowbreak{}q,\ \allowbreak{}D],\ \allowbreak{}m]}.
What is meant is \(1/(q^2-m^2)\).

\texttt{PropagatorDenominator} must appear inside
\texttt{FeynAmpDenominator}, it is not a standalone object.

\subsection{See also}

\hyperlink{toc}{Overview},
\hyperlink{feynampdenominator}{FeynAmpDenominator},
\hyperlink{feynampdenominatorexplicit}{FeynAmpDenominatorExplicit}.

\subsection{Examples}

\begin{Shaded}
\begin{Highlighting}[]
\NormalTok{FeynAmpDenominator}\OperatorTok{[}\NormalTok{PropagatorDenominator}\OperatorTok{[}\NormalTok{Momentum}\OperatorTok{[}\FunctionTok{p}\OperatorTok{],} \FunctionTok{m}\OperatorTok{]]}
\end{Highlighting}
\end{Shaded}

\begin{dmath*}\breakingcomma
\frac{1}{\overline{p}^2-m^2}
\end{dmath*}

\begin{Shaded}
\begin{Highlighting}[]
\NormalTok{FeynAmpDenominator}\OperatorTok{[}\NormalTok{PropagatorDenominator}\OperatorTok{[}\NormalTok{Momentum}\OperatorTok{[}\FunctionTok{p}\OperatorTok{,} \FunctionTok{D}\OperatorTok{],} \FunctionTok{m}\OperatorTok{]]}
\end{Highlighting}
\end{Shaded}

\begin{dmath*}\breakingcomma
\frac{1}{p^2-m^2}
\end{dmath*}
\end{document}
