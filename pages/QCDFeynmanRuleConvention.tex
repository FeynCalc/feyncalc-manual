% !TeX program = pdflatex
% !TeX root = QCDFeynmanRuleConvention.tex

\documentclass[../FeynCalcManual.tex]{subfiles}
\begin{document}
\hypertarget{qcdfeynmanruleconvention}{%
\section{QCDFeynmanRuleConvention}\label{qcdfeynmanruleconvention}}

\texttt{QCDFeynmanRuleConvention} fixes the sign convention in the QCD
Feynman rules for the ghost propagator and the ghost-gluon vertex.This
is done by setting the value of
\texttt{QCDFeynmanRuleConvention[\allowbreak{}GhostPropagator]} and
\texttt{QCDFeynmanRuleConvention[\allowbreak{}GluonGhostVertex]}.

The default values are \texttt{1} for both, which corresponds to the
convention used in most books. Setting them to \texttt{-1} enforces the
convention that can be found e.g.~in the book ``Applications of
Perturbative QCD'' by R. Field.

\subsection{See also}

\hyperlink{toc}{Overview},
\hyperlink{gluonghostvertex}{GluonGhostVertex},
\hyperlink{ghostpropagator}{GhostPropagator}.

\subsection{Examples}

Enforce the convention as in ``Applications of Perturbative QCD'' by R.
Field.

\begin{Shaded}
\begin{Highlighting}[]
\NormalTok{QCDFeynmanRuleConvention}\OperatorTok{[}\NormalTok{GhostPropagator}\OperatorTok{]} \ExtensionTok{=} \SpecialCharTok{{-}}\DecValTok{1}\NormalTok{; }
 
\NormalTok{QCDFeynmanRuleConvention}\OperatorTok{[}\NormalTok{GluonGhostVertex}\OperatorTok{]} \ExtensionTok{=} \SpecialCharTok{{-}}\DecValTok{1}\NormalTok{;}
\end{Highlighting}
\end{Shaded}

\begin{Shaded}
\begin{Highlighting}[]
\NormalTok{GHP}\OperatorTok{[}\FunctionTok{p}\OperatorTok{,} \FunctionTok{a}\OperatorTok{,} \FunctionTok{b}\OperatorTok{]} \SpecialCharTok{//}\NormalTok{ Explicit}
\end{Highlighting}
\end{Shaded}

\begin{dmath*}\breakingcomma
-\frac{i \delta ^{ab}}{p^2}
\end{dmath*}

\begin{Shaded}
\begin{Highlighting}[]
\NormalTok{GGV}\OperatorTok{[\{}\FunctionTok{p}\OperatorTok{,} \SpecialCharTok{\textbackslash{}}\OperatorTok{[}\NormalTok{Mu}\OperatorTok{],} \FunctionTok{a}\OperatorTok{\},} \OperatorTok{\{}\FunctionTok{q}\OperatorTok{,} \SpecialCharTok{\textbackslash{}}\OperatorTok{[}\NormalTok{Nu}\OperatorTok{],} \FunctionTok{b}\OperatorTok{\},} \OperatorTok{\{}\FunctionTok{k}\OperatorTok{,} \SpecialCharTok{\textbackslash{}}\OperatorTok{[}\NormalTok{Rho}\OperatorTok{],} \FunctionTok{c}\OperatorTok{\}]} \SpecialCharTok{//}\NormalTok{ Explicit}
\end{Highlighting}
\end{Shaded}

\begin{dmath*}\breakingcomma
g_s k^{\mu } f^{abc}
\end{dmath*}

Back to the standard convention.

\begin{Shaded}
\begin{Highlighting}[]
\NormalTok{QCDFeynmanRuleConvention}\OperatorTok{[}\NormalTok{GhostPropagator}\OperatorTok{]} \ExtensionTok{=} \DecValTok{1} 
 
\NormalTok{QCDFeynmanRuleConvention}\OperatorTok{[}\NormalTok{GluonGhostVertex}\OperatorTok{]} \ExtensionTok{=} \DecValTok{1}
\end{Highlighting}
\end{Shaded}

\begin{dmath*}\breakingcomma
1
\end{dmath*}

\begin{dmath*}\breakingcomma
1
\end{dmath*}

\begin{Shaded}
\begin{Highlighting}[]
\NormalTok{GHP}\OperatorTok{[}\FunctionTok{p}\OperatorTok{,} \FunctionTok{a}\OperatorTok{,} \FunctionTok{b}\OperatorTok{]} \SpecialCharTok{//}\NormalTok{ Explicit}
\end{Highlighting}
\end{Shaded}

\begin{dmath*}\breakingcomma
\frac{i \delta ^{ab}}{p^2}
\end{dmath*}

\begin{Shaded}
\begin{Highlighting}[]
\NormalTok{GGV}\OperatorTok{[\{}\FunctionTok{p}\OperatorTok{,} \SpecialCharTok{\textbackslash{}}\OperatorTok{[}\NormalTok{Mu}\OperatorTok{],} \FunctionTok{a}\OperatorTok{\},} \OperatorTok{\{}\FunctionTok{q}\OperatorTok{,} \SpecialCharTok{\textbackslash{}}\OperatorTok{[}\NormalTok{Nu}\OperatorTok{],} \FunctionTok{b}\OperatorTok{\},} \OperatorTok{\{}\FunctionTok{k}\OperatorTok{,} \SpecialCharTok{\textbackslash{}}\OperatorTok{[}\NormalTok{Rho}\OperatorTok{],} \FunctionTok{c}\OperatorTok{\}]} \SpecialCharTok{//}\NormalTok{ Explicit}
\end{Highlighting}
\end{Shaded}

\begin{dmath*}\breakingcomma
g_s \left(-k^{\mu }\right) f^{abc}
\end{dmath*}
\end{document}
