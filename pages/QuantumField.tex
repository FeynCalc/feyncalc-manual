% !TeX program = pdflatex
% !TeX root = QuantumField.tex

\documentclass[../FeynCalcManual.tex]{subfiles}
\begin{document}
\hypertarget{quantumfield}{
\section{QuantumField}\label{quantumfield}\index{QuantumField}}

\texttt{QuantumField} is the head of quantized fields and their
derivatives.

\texttt{QuantumField[\allowbreak{}par,\ \allowbreak{}ftype,\ \allowbreak{}\{\allowbreak{}lorind\},\ \allowbreak{}\{\allowbreak{}sunind\}]}
denotes a quantum field of type \texttt{ftype} with (possible)
Lorentz-indices \texttt{lorind} and \(SU(N)\) indices \texttt{sunind}.
The optional first argument \texttt{par} denotes a partial derivative
acting on the field.

\subsection{See also}

\hyperlink{toc}{Overview}, \hyperlink{feynrule}{FeynRule},
\hyperlink{fcpartiald}{FCPartialD},
\hyperlink{expandpartiald}{ExpandPartialD}.

\subsection{Examples}

This denotes a scalar field.

\begin{Shaded}
\begin{Highlighting}[]
\NormalTok{QuantumField}\OperatorTok{[}\FunctionTok{S}\OperatorTok{]}
\end{Highlighting}
\end{Shaded}

\begin{dmath*}\breakingcomma
S
\end{dmath*}

Quark fields

\begin{Shaded}
\begin{Highlighting}[]
\NormalTok{QuantumField}\OperatorTok{[}\NormalTok{AntiQuarkField}\OperatorTok{]}
\end{Highlighting}
\end{Shaded}

\begin{dmath*}\breakingcomma
\bar{\psi }
\end{dmath*}

\begin{Shaded}
\begin{Highlighting}[]
\NormalTok{QuantumField}\OperatorTok{[}\NormalTok{QuarkField}\OperatorTok{]}
\end{Highlighting}
\end{Shaded}

\begin{dmath*}\breakingcomma
\psi
\end{dmath*}

This is a field with a Lorentz index.

\begin{Shaded}
\begin{Highlighting}[]
\NormalTok{QuantumField}\OperatorTok{[}\FunctionTok{B}\OperatorTok{,} \OperatorTok{\{}\SpecialCharTok{\textbackslash{}}\OperatorTok{[}\NormalTok{Mu}\OperatorTok{]\}]}
\end{Highlighting}
\end{Shaded}

\begin{dmath*}\breakingcomma
B_{\mu }
\end{dmath*}

Color indices should be put after the Lorentz ones.

\begin{Shaded}
\begin{Highlighting}[]
\NormalTok{QuantumField}\OperatorTok{[}\NormalTok{GaugeField}\OperatorTok{,} \OperatorTok{\{}\SpecialCharTok{\textbackslash{}}\OperatorTok{[}\NormalTok{Mu}\OperatorTok{]\},} \OperatorTok{\{}\FunctionTok{a}\OperatorTok{\}]} \SpecialCharTok{//} \FunctionTok{StandardForm}

\CommentTok{(*QuantumField[GaugeField, LorentzIndex[\textbackslash{}[Mu]], SUNIndex[a]]*)}
\end{Highlighting}
\end{Shaded}

\(A_{\Delta}^a\) is a short form for \(\Delta ^{mu } A_{mu }^a\)

\begin{Shaded}
\begin{Highlighting}[]
\NormalTok{QuantumField}\OperatorTok{[}\FunctionTok{A}\OperatorTok{,} \OperatorTok{\{}\NormalTok{OPEDelta}\OperatorTok{\},} \OperatorTok{\{}\FunctionTok{a}\OperatorTok{\}]}
\end{Highlighting}
\end{Shaded}

\begin{dmath*}\breakingcomma
A_{\Delta }^a
\end{dmath*}

The first list of indices is usually interpreted as type
\texttt{LorentzIndex}, except for \texttt{OPEDelta}, which gets
converted to type \texttt{Momentum}.

\begin{Shaded}
\begin{Highlighting}[]
\NormalTok{QuantumField}\OperatorTok{[}\FunctionTok{A}\OperatorTok{,} \OperatorTok{\{}\NormalTok{OPEDelta}\OperatorTok{\},} \OperatorTok{\{}\FunctionTok{a}\OperatorTok{\}]} \SpecialCharTok{//} \FunctionTok{StandardForm}

\CommentTok{(*QuantumField[A, Momentum[OPEDelta], SUNIndex[a]]*)}
\end{Highlighting}
\end{Shaded}

Derivatives of fields are denoted as follows.

\begin{Shaded}
\begin{Highlighting}[]
\NormalTok{QuantumField}\OperatorTok{[}\NormalTok{FCPartialD}\OperatorTok{[}\NormalTok{LorentzIndex}\OperatorTok{[}\SpecialCharTok{\textbackslash{}}\OperatorTok{[}\NormalTok{Mu}\OperatorTok{]]],} \FunctionTok{A}\OperatorTok{,} \OperatorTok{\{}\SpecialCharTok{\textbackslash{}}\OperatorTok{[}\NormalTok{Mu}\OperatorTok{]\}]}
\end{Highlighting}
\end{Shaded}

\begin{dmath*}\breakingcomma
\left.(\partial _{\mu }A_{\mu }\right)
\end{dmath*}

\begin{Shaded}
\begin{Highlighting}[]
\NormalTok{QuantumField}\OperatorTok{[}\NormalTok{FCPartialD}\OperatorTok{[}\NormalTok{OPEDelta}\OperatorTok{],} \FunctionTok{S}\OperatorTok{]}
\end{Highlighting}
\end{Shaded}

\begin{dmath*}\breakingcomma
\left.(\partial _{\Delta }S\right)
\end{dmath*}

\begin{Shaded}
\begin{Highlighting}[]
\NormalTok{QuantumField}\OperatorTok{[}\NormalTok{FCPartialD}\OperatorTok{[}\NormalTok{OPEDelta}\OperatorTok{],} \FunctionTok{A}\OperatorTok{,} \OperatorTok{\{}\NormalTok{OPEDelta}\OperatorTok{\},} \OperatorTok{\{}\FunctionTok{a}\OperatorTok{\}]}
\end{Highlighting}
\end{Shaded}

\begin{dmath*}\breakingcomma
\left.(\partial _{\Delta }A_{\Delta }^a\right)
\end{dmath*}

\begin{Shaded}
\begin{Highlighting}[]
\NormalTok{QuantumField}\OperatorTok{[}\NormalTok{FCPartialD}\OperatorTok{[}\NormalTok{OPEDelta}\OperatorTok{]}\SpecialCharTok{\^{}}\NormalTok{OPEm}\OperatorTok{,} \FunctionTok{A}\OperatorTok{,} \OperatorTok{\{}\NormalTok{OPEDelta}\OperatorTok{\},} \OperatorTok{\{}\FunctionTok{a}\OperatorTok{\}]}
\end{Highlighting}
\end{Shaded}

\begin{dmath*}\breakingcomma
\partial _{\Delta }^m{}^{A\Delta a}
\end{dmath*}

\begin{Shaded}
\begin{Highlighting}[]
\NormalTok{QuantumField}\OperatorTok{[}\NormalTok{QuantumField}\OperatorTok{[}\FunctionTok{A}\OperatorTok{]]} \ExtensionTok{===}\NormalTok{ QuantumField}\OperatorTok{[}\FunctionTok{A}\OperatorTok{]}
\end{Highlighting}
\end{Shaded}

\begin{dmath*}\breakingcomma
\text{True}
\end{dmath*}
\end{document}
