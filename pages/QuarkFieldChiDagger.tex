% !TeX program = pdflatex
% !TeX root = QuarkFieldChiDagger.tex

\documentclass[../FeynCalcManual.tex]{subfiles}
\begin{document}
\hypertarget{quarkfieldchidagger}{%
\section{QuarkFieldChiDagger}\label{quarkfieldchidagger}}

\texttt{QuarkFieldChiDagger} is the name of a fermionic field. This is
just a name with no functional properties. Only typesetting rules are
attached.

\subsection{See also}

\hyperlink{toc}{Overview}

\subsection{Examples}

\begin{Shaded}
\begin{Highlighting}[]
\NormalTok{QuarkFieldChiDagger}
\end{Highlighting}
\end{Shaded}

\begin{dmath*}\breakingcomma
\chi ^{\dagger }
\end{dmath*}

\begin{Shaded}
\begin{Highlighting}[]
\NormalTok{QuantumField}\OperatorTok{[}\NormalTok{QuarkFieldChiDagger}\OperatorTok{]}\NormalTok{ . GA}\OperatorTok{[}\SpecialCharTok{\textbackslash{}}\OperatorTok{[}\NormalTok{Mu}\OperatorTok{]]}\NormalTok{ . CovariantD}\OperatorTok{[}\SpecialCharTok{\textbackslash{}}\OperatorTok{[}\NormalTok{Mu}\OperatorTok{]]}\NormalTok{ . QuantumField}\OperatorTok{[}\NormalTok{QuarkFieldChi}\OperatorTok{]} 
 
\NormalTok{ExpandPartialD}\OperatorTok{[}\SpecialCharTok{\%}\OperatorTok{]}
\end{Highlighting}
\end{Shaded}

\begin{dmath*}\breakingcomma
\chi ^{\dagger }.\bar{\gamma }^{\mu }.D_{\mu }.\chi
\end{dmath*}

\begin{dmath*}\breakingcomma
\bar{\gamma }^{\mu } \chi ^{\dagger }.\left(\left.(\partial _{\mu }\chi \right)\right)-i T^{\text{c19}} g_s \bar{\gamma }^{\mu } \chi ^{\dagger }.A_{\mu }^{\text{c19}}.\chi
\end{dmath*}
\end{document}
