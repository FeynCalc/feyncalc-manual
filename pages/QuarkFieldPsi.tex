% !TeX program = pdflatex
% !TeX root = QuarkFieldPsi.tex

\documentclass[../FeynCalcManual.tex]{subfiles}
\begin{document}
\hypertarget{quarkfieldpsi}{%
\section{QuarkFieldPsi}\label{quarkfieldpsi}}

\texttt{QuarkFieldPsi} is the name of a fermionic field.This is just a
name with no functional properties. Only typesetting rules are attached.

\subsection{See also}

\hyperlink{toc}{Overview}

\subsection{Examples}

\begin{Shaded}
\begin{Highlighting}[]
\NormalTok{QuarkFieldPsi}
\end{Highlighting}
\end{Shaded}

\begin{dmath*}\breakingcomma
\psi
\end{dmath*}

\begin{Shaded}
\begin{Highlighting}[]
\NormalTok{QuantumField}\OperatorTok{[}\NormalTok{QuarkFieldPsiDagger}\OperatorTok{]}\NormalTok{ . GA}\OperatorTok{[}\SpecialCharTok{\textbackslash{}}\OperatorTok{[}\NormalTok{Mu}\OperatorTok{]]}\NormalTok{ . CovariantD}\OperatorTok{[}\SpecialCharTok{\textbackslash{}}\OperatorTok{[}\NormalTok{Mu}\OperatorTok{]]}\NormalTok{ . QuantumField}\OperatorTok{[}\NormalTok{QuarkFieldPsi}\OperatorTok{]} 
 
\NormalTok{ExpandPartialD}\OperatorTok{[}\SpecialCharTok{\%}\OperatorTok{]}
\end{Highlighting}
\end{Shaded}

\begin{dmath*}\breakingcomma
\psi ^{\dagger }.\bar{\gamma }^{\mu }.D_{\mu }.\psi
\end{dmath*}

\begin{dmath*}\breakingcomma
\bar{\gamma }^{\mu } \psi ^{\dagger }.\left(\left.(\partial _{\mu }\psi \right)\right)-i T^{\text{c19}} g_s \bar{\gamma }^{\mu } \psi ^{\dagger }.A_{\mu }^{\text{c19}}.\psi
\end{dmath*}
\end{document}
