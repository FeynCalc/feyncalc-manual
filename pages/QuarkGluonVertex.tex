% !TeX program = pdflatex
% !TeX root = QuarkGluonVertex.tex

\documentclass[../FeynCalcManual.tex]{subfiles}
\begin{document}
\hypertarget{quarkgluonvertex}{
\section{QuarkGluonVertex}\label{quarkgluonvertex}\index{QuarkGluonVertex}}

\texttt{QuarkGluonVertex[\allowbreak{}mu,\ \allowbreak{}a]} gives the
Feynman rule for the quark-gluon vertex.

\texttt{QGV} can be used as an abbreviation of
\texttt{QuarkGluonVertex}.

The dimension and the name of the coupling constant are determined by
the options \texttt{Dimension} and \texttt{CouplingConstant}.

\subsection{See also}

\hyperlink{toc}{Overview}, \hyperlink{gluonvertex}{GluonVertex}.

\subsection{Examples}

\begin{Shaded}
\begin{Highlighting}[]
\NormalTok{QuarkGluonVertex}\OperatorTok{[}\SpecialCharTok{\textbackslash{}}\OperatorTok{[}\NormalTok{Mu}\OperatorTok{],} \FunctionTok{a}\OperatorTok{,}\NormalTok{ Explicit }\OtherTok{{-}\textgreater{}} \ConstantTok{True}\OperatorTok{]}
\end{Highlighting}
\end{Shaded}

\begin{dmath*}\breakingcomma
i g_s T^a.\gamma ^{\mu }
\end{dmath*}

\begin{Shaded}
\begin{Highlighting}[]
\NormalTok{QGV}\OperatorTok{[}\SpecialCharTok{\textbackslash{}}\OperatorTok{[}\NormalTok{Mu}\OperatorTok{],} \FunctionTok{a}\OperatorTok{]}
\end{Highlighting}
\end{Shaded}

\begin{dmath*}\breakingcomma
Q_a^{\mu }
\end{dmath*}

\begin{Shaded}
\begin{Highlighting}[]
\NormalTok{Explicit}\OperatorTok{[}\SpecialCharTok{\%}\OperatorTok{]}
\end{Highlighting}
\end{Shaded}

\begin{dmath*}\breakingcomma
i g_s T^a.\gamma ^{\mu }
\end{dmath*}

\begin{Shaded}
\begin{Highlighting}[]
\NormalTok{QuarkGluonVertex}\OperatorTok{[}\SpecialCharTok{\textbackslash{}}\OperatorTok{[}\NormalTok{Mu}\OperatorTok{],} \FunctionTok{a}\OperatorTok{,}\NormalTok{ CounterTerm }\OtherTok{{-}\textgreater{}} \DecValTok{1}\OperatorTok{,}\NormalTok{ Explicit }\OtherTok{{-}\textgreater{}} \ConstantTok{True}\OperatorTok{]}
\end{Highlighting}
\end{Shaded}

\begin{dmath*}\breakingcomma
\frac{2 i g_s^3 S_n \left(C_F-\frac{C_A}{2}\right) T^a.\gamma ^{\mu }}{\varepsilon }
\end{dmath*}

\begin{Shaded}
\begin{Highlighting}[]
\NormalTok{QuarkGluonVertex}\OperatorTok{[}\SpecialCharTok{\textbackslash{}}\OperatorTok{[}\NormalTok{Mu}\OperatorTok{],} \FunctionTok{a}\OperatorTok{,}\NormalTok{ CounterTerm }\OtherTok{{-}\textgreater{}} \DecValTok{2}\OperatorTok{,}\NormalTok{ Explicit }\OtherTok{{-}\textgreater{}} \ConstantTok{True}\OperatorTok{]}
\end{Highlighting}
\end{Shaded}

\begin{dmath*}\breakingcomma
\frac{3 i C_A g_s^3 S_n T^a.\gamma ^{\mu }}{\varepsilon }
\end{dmath*}

\begin{Shaded}
\begin{Highlighting}[]
\NormalTok{QuarkGluonVertex}\OperatorTok{[}\SpecialCharTok{\textbackslash{}}\OperatorTok{[}\NormalTok{Mu}\OperatorTok{],} \FunctionTok{a}\OperatorTok{,}\NormalTok{ CounterTerm }\OtherTok{{-}\textgreater{}} \DecValTok{3}\OperatorTok{,}\NormalTok{ Explicit }\OtherTok{{-}\textgreater{}} \ConstantTok{True}\OperatorTok{]}
\end{Highlighting}
\end{Shaded}

\begin{dmath*}\breakingcomma
\frac{2 i g_s^3 S_n \left(C_A+C_F\right) T^a.\gamma ^{\mu }}{\varepsilon }
\end{dmath*}

\begin{Shaded}
\begin{Highlighting}[]
\NormalTok{QuarkGluonVertex}\OperatorTok{[\{}\FunctionTok{p}\OperatorTok{,} \SpecialCharTok{\textbackslash{}}\OperatorTok{[}\NormalTok{Mu}\OperatorTok{],} \FunctionTok{a}\OperatorTok{\},} \OperatorTok{\{}\FunctionTok{q}\OperatorTok{\},} \OperatorTok{\{}\FunctionTok{k}\OperatorTok{\},}\NormalTok{ OPE }\OtherTok{{-}\textgreater{}} \ConstantTok{True}\OperatorTok{,}\NormalTok{ Explicit }\OtherTok{{-}\textgreater{}} \ConstantTok{True}\OperatorTok{]}
\end{Highlighting}
\end{Shaded}

\begin{dmath*}\breakingcomma
\Omega  \Delta ^{\mu } g_s (\gamma \cdot \Delta ).T^a \left(\sum _{i=0}^{-2+m} (-1)^i (k\cdot \Delta )^i (\Delta \cdot q)^{-2-i+m}\right)+i g_s T^a.\gamma ^{\mu }
\end{dmath*}

\begin{Shaded}
\begin{Highlighting}[]
\NormalTok{QuarkGluonVertex}\OperatorTok{[\{}\FunctionTok{p}\OperatorTok{,} \SpecialCharTok{\textbackslash{}}\OperatorTok{[}\NormalTok{Mu}\OperatorTok{],} \FunctionTok{a}\OperatorTok{\},} \OperatorTok{\{}\FunctionTok{q}\OperatorTok{\},} \OperatorTok{\{}\FunctionTok{k}\OperatorTok{\},}\NormalTok{ OPE }\OtherTok{{-}\textgreater{}} \ConstantTok{False}\OperatorTok{,}\NormalTok{ Explicit }\OtherTok{{-}\textgreater{}} \ConstantTok{True}\OperatorTok{]}
\end{Highlighting}
\end{Shaded}

\begin{dmath*}\breakingcomma
i g_s T^a.\gamma ^{\mu }
\end{dmath*}
\end{document}
