% !TeX program = pdflatex
% !TeX root = QuarkPropagator.tex

\documentclass[../FeynCalcManual.tex]{subfiles}
\begin{document}
\hypertarget{quarkpropagator}{%
\section{QuarkPropagator}\label{quarkpropagator}}

\texttt{QuarkPropagator[\allowbreak{}p]} is the massless quark
propagator.

\texttt{QuarkPropagator[\allowbreak{}\{\allowbreak{}p,\ \allowbreak{}m\}]}
gives the quark propagator with mass \(m\).

\texttt{QP} can be used as an abbreviation of \texttt{QuarkPropagator}.

\subsection{See also}

\hyperlink{toc}{Overview}, \hyperlink{gluonpropagator}{GluonPropagator},
\hyperlink{quarkgluonvertex}{QuarkGluonVertex}.

\subsection{Examples}

\begin{Shaded}
\begin{Highlighting}[]
\NormalTok{QuarkPropagator}\OperatorTok{[}\FunctionTok{p}\OperatorTok{,}\NormalTok{ Explicit }\OtherTok{{-}\textgreater{}} \ConstantTok{True}\OperatorTok{]}
\end{Highlighting}
\end{Shaded}

\begin{dmath*}\breakingcomma
\frac{i \gamma \cdot p}{p^2}
\end{dmath*}

\begin{Shaded}
\begin{Highlighting}[]
\NormalTok{QuarkPropagator}\OperatorTok{[\{}\FunctionTok{p}\OperatorTok{,} \FunctionTok{m}\OperatorTok{\},}\NormalTok{ Explicit }\OtherTok{{-}\textgreater{}} \ConstantTok{True}\OperatorTok{]}
\end{Highlighting}
\end{Shaded}

\begin{dmath*}\breakingcomma
\frac{i (m+\gamma \cdot p)}{p^2-m^2}
\end{dmath*}

\begin{Shaded}
\begin{Highlighting}[]
\NormalTok{QP}\OperatorTok{[\{}\FunctionTok{p}\OperatorTok{,} \FunctionTok{m}\OperatorTok{\}]} 
 
\NormalTok{Explicit}\OperatorTok{[}\SpecialCharTok{\%}\OperatorTok{]}
\end{Highlighting}
\end{Shaded}

\begin{dmath*}\breakingcomma
\Pi _q(p)
\end{dmath*}

\begin{dmath*}\breakingcomma
\frac{i (m+\gamma \cdot p)}{p^2-m^2}
\end{dmath*}
\end{document}
