% !TeX program = pdflatex
% !TeX root = RightNablaD.tex

\documentclass[../FeynCalcManual.tex]{subfiles}
\begin{document}
\hypertarget{rightnablad}{
\section{RightNablaD}\label{rightnablad}\index{RightNablaD}}

\texttt{RightNablaD[\allowbreak{}i]} denotes \(\nabla _{i}\), acting to
the right.

\subsection{See also}

\hyperlink{toc}{Overview}, \hyperlink{expandpartiald}{ExpandPartialD},
\hyperlink{fcpartiald}{FCPartialD}, \hyperlink{leftnablad}{LeftNablaD}.

\subsection{Examples}

\begin{Shaded}
\begin{Highlighting}[]
\NormalTok{RightNablaD}\OperatorTok{[}\FunctionTok{i}\OperatorTok{]}
\end{Highlighting}
\end{Shaded}

\begin{dmath*}\breakingcomma
\vec{\nabla }^i
\end{dmath*}

\begin{Shaded}
\begin{Highlighting}[]
\NormalTok{RightNablaD}\OperatorTok{[}\FunctionTok{i}\OperatorTok{]}\NormalTok{ . QuantumField}\OperatorTok{[}\FunctionTok{A}\OperatorTok{,}\NormalTok{ LorentzIndex}\OperatorTok{[}\SpecialCharTok{\textbackslash{}}\OperatorTok{[}\NormalTok{Mu}\OperatorTok{]]]} 
 
\NormalTok{ex }\ExtensionTok{=}\NormalTok{ ExpandPartialD}\OperatorTok{[}\SpecialCharTok{\%}\OperatorTok{]}
\end{Highlighting}
\end{Shaded}

\begin{dmath*}\breakingcomma
\vec{\nabla }^i.A_{\mu }
\end{dmath*}

\begin{dmath*}\breakingcomma
-\left(\partial _iA_{\mu }\right)
\end{dmath*}

\begin{Shaded}
\begin{Highlighting}[]
\NormalTok{ex }\SpecialCharTok{//} \FunctionTok{StandardForm}

\CommentTok{(*{-}QuantumField[FCPartialD[CartesianIndex[i]], A, LorentzIndex[\textbackslash{}[Mu]]]*)}
\end{Highlighting}
\end{Shaded}

\begin{Shaded}
\begin{Highlighting}[]
\NormalTok{RightNablaD}\OperatorTok{[}\FunctionTok{i}\OperatorTok{]} \SpecialCharTok{//} \FunctionTok{StandardForm}

\CommentTok{(*RightNablaD[CartesianIndex[i]]*)}
\end{Highlighting}
\end{Shaded}

\end{document}
