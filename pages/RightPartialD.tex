% !TeX program = pdflatex
% !TeX root = RightPartialD.tex

\documentclass[../FeynCalcManual.tex]{subfiles}
\begin{document}
\hypertarget{rightpartiald}{
\section{RightPartialD}\label{rightpartiald}\index{RightPartialD}}

\texttt{RightPartialD[\allowbreak{}mu]} denotes \(\partial _{\mu }\),
acting to the right.

\subsection{See also}

\hyperlink{toc}{Overview}, \hyperlink{expandpartiald}{ExpandPartialD},
\hyperlink{fcpartiald}{FCPartialD},
\hyperlink{leftpartiald}{LeftPartialD}.

\subsection{Examples}

\begin{Shaded}
\begin{Highlighting}[]
\NormalTok{RightPartialD}\OperatorTok{[}\SpecialCharTok{\textbackslash{}}\OperatorTok{[}\NormalTok{Mu}\OperatorTok{]]}
\end{Highlighting}
\end{Shaded}

\begin{dmath*}\breakingcomma
\vec{\partial }_{\mu }
\end{dmath*}

\begin{Shaded}
\begin{Highlighting}[]
\NormalTok{RightPartialD}\OperatorTok{[}\SpecialCharTok{\textbackslash{}}\OperatorTok{[}\NormalTok{Mu}\OperatorTok{]]}\NormalTok{ . QuantumField}\OperatorTok{[}\FunctionTok{A}\OperatorTok{,}\NormalTok{ LorentzIndex}\OperatorTok{[}\SpecialCharTok{\textbackslash{}}\OperatorTok{[}\NormalTok{Mu}\OperatorTok{]]]} 
 
\NormalTok{ex }\ExtensionTok{=}\NormalTok{ ExpandPartialD}\OperatorTok{[}\SpecialCharTok{\%}\OperatorTok{]}
\end{Highlighting}
\end{Shaded}

\begin{dmath*}\breakingcomma
\vec{\partial }_{\mu }.A_{\mu }
\end{dmath*}

\begin{dmath*}\breakingcomma
\left.(\partial _{\mu }A_{\mu }\right)
\end{dmath*}

\begin{Shaded}
\begin{Highlighting}[]
\NormalTok{ex }\SpecialCharTok{//} \FunctionTok{StandardForm}

\CommentTok{(*QuantumField[FCPartialD[LorentzIndex[\textbackslash{}[Mu]]], A, LorentzIndex[\textbackslash{}[Mu]]]*)}
\end{Highlighting}
\end{Shaded}

\begin{Shaded}
\begin{Highlighting}[]
\NormalTok{RightPartialD}\OperatorTok{[}\SpecialCharTok{\textbackslash{}}\OperatorTok{[}\NormalTok{Mu}\OperatorTok{]]} \SpecialCharTok{//} \FunctionTok{StandardForm}

\CommentTok{(*RightPartialD[LorentzIndex[\textbackslash{}[Mu]]]*)}
\end{Highlighting}
\end{Shaded}

\end{document}
