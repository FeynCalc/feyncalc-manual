% !TeX program = pdflatex
% !TeX root = SFAD.tex

\documentclass[../FeynCalcManual.tex]{subfiles}
\begin{document}
\hypertarget{sfad}{
\section{SFAD}\label{sfad}\index{SFAD}}

\texttt{SFAD[\allowbreak{}\{\allowbreak{}\{\allowbreak{}q1 +...,\ \allowbreak{}p1 . q2 +...,\ \allowbreak{}\} \{\allowbreak{}m^2,\ \allowbreak{}s\},\ \allowbreak{}n\},\ \allowbreak{}...]}
denotes a standard Lorentzian propagator given by
\(\frac{1}{[(q_1+\ldots)^2 + p_1 \cdot q_2 ... + m^2 + s i \eta]^n}\),
where \(q_1^2\) and \(p_1 \cdot q_2\) are Lorentzian scalar products in
\(D\) dimensions.

For brevity one can also use shorter forms such as
\texttt{SFAD[\allowbreak{}\{\allowbreak{}q1+ ...,\ \allowbreak{} m^2\},\ \allowbreak{}...]},
\texttt{SFAD[\allowbreak{}\{\allowbreak{}q1+ ...,\ \allowbreak{} m^2 ,\ \allowbreak{}n\},\ \allowbreak{}...]},
\texttt{SFAD[\allowbreak{}\{\allowbreak{}q1+ ...,\ \allowbreak{} \{\allowbreak{}m^2,\ \allowbreak{}-1\}\},\ \allowbreak{}...]},
\texttt{SFAD[\allowbreak{}q1,\ \allowbreak{}...]} etc.

If \texttt{s} is not explicitly specified, its value is determined by
the option \texttt{EtaSign}, which has the default value \texttt{+1} and
corresponds to \(+ i \eta\)

If \texttt{n} is not explicitly specified, then the default value
\texttt{1} is assumed. Translation into the FeynCalc internal form is
performed by \texttt{FeynCalcInternal}, where an \texttt{SFAD} is
encoded using the special head \texttt{StandardPropagatorDenominator}.

\texttt{SFAD} can represent more versatile propagators as compared to
the old \texttt{FAD}. In particular, \texttt{FAD} does not allow one to
enter eikonal propagators, track the sign of the \(i \eta\) or change
the sign and the form of the mass term.

\subsection{See also}

\hyperlink{toc}{Overview}, \hyperlink{fad}{FAD}, \hyperlink{gfad}{GFAD},
\hyperlink{cfad}{CFAD}.

\subsection{Examples}

\begin{Shaded}
\begin{Highlighting}[]
\NormalTok{SFAD}\OperatorTok{[\{\{}\FunctionTok{p}\OperatorTok{,} \DecValTok{0}\OperatorTok{\},} \FunctionTok{m}\SpecialCharTok{\^{}}\DecValTok{2}\OperatorTok{\}]}
\end{Highlighting}
\end{Shaded}

\begin{dmath*}\breakingcomma
\frac{1}{(p^2-m^2+i \eta )}
\end{dmath*}

\begin{Shaded}
\begin{Highlighting}[]
\NormalTok{SFAD}\OperatorTok{[\{\{}\FunctionTok{p}\OperatorTok{,} \DecValTok{0}\OperatorTok{\},} \OperatorTok{\{}\FunctionTok{m}\SpecialCharTok{\^{}}\DecValTok{2}\OperatorTok{,} \SpecialCharTok{{-}}\DecValTok{1}\OperatorTok{\}\}]}
\end{Highlighting}
\end{Shaded}

\begin{dmath*}\breakingcomma
\frac{1}{(p^2-m^2-i \eta )}
\end{dmath*}

\begin{Shaded}
\begin{Highlighting}[]
\NormalTok{SFAD}\OperatorTok{[\{\{}\FunctionTok{p}\OperatorTok{,} \DecValTok{0}\OperatorTok{\},} \OperatorTok{\{}\SpecialCharTok{{-}}\FunctionTok{m}\SpecialCharTok{\^{}}\DecValTok{2}\OperatorTok{,} \SpecialCharTok{{-}}\DecValTok{1}\OperatorTok{\}\}]}
\end{Highlighting}
\end{Shaded}

\begin{dmath*}\breakingcomma
\frac{1}{(p^2+m^2-i \eta )}
\end{dmath*}

\begin{Shaded}
\begin{Highlighting}[]
\NormalTok{SFAD}\OperatorTok{[\{\{}\DecValTok{0}\OperatorTok{,} \FunctionTok{p}\NormalTok{ . }\FunctionTok{q}\OperatorTok{\},} \FunctionTok{m}\SpecialCharTok{\^{}}\DecValTok{2}\OperatorTok{\}]}
\end{Highlighting}
\end{Shaded}

\begin{dmath*}\breakingcomma
\frac{1}{(p\cdot q-m^2+i \eta )}
\end{dmath*}

\begin{Shaded}
\begin{Highlighting}[]
\NormalTok{SFAD}\OperatorTok{[\{\{}\DecValTok{0}\OperatorTok{,} \FunctionTok{n}\NormalTok{ . }\FunctionTok{q}\OperatorTok{\}\}]}
\end{Highlighting}
\end{Shaded}

\begin{dmath*}\breakingcomma
\frac{1}{(n\cdot q+i \eta )}
\end{dmath*}

\begin{Shaded}
\begin{Highlighting}[]
\NormalTok{SFAD}\OperatorTok{[\{\{}\FunctionTok{p}\OperatorTok{,} \FunctionTok{p}\NormalTok{ . }\FunctionTok{q}\OperatorTok{\},} \FunctionTok{m}\SpecialCharTok{\^{}}\DecValTok{2}\OperatorTok{\}]}
\end{Highlighting}
\end{Shaded}

\begin{dmath*}\breakingcomma
\frac{1}{(p^2+p\cdot q-m^2+i \eta )}
\end{dmath*}

The so called Smirnov-notation for propagators can be achieved by
multiplying the quadratic part by \texttt{I} and switching the sign of
the mass term.

\begin{Shaded}
\begin{Highlighting}[]
\NormalTok{SFAD}\OperatorTok{[\{\{}\FunctionTok{I}\SpecialCharTok{*}\FunctionTok{p}\OperatorTok{,} \DecValTok{0}\OperatorTok{\},} \SpecialCharTok{{-}}\FunctionTok{m}\SpecialCharTok{\^{}}\DecValTok{2}\OperatorTok{\}]}
\end{Highlighting}
\end{Shaded}

\begin{dmath*}\breakingcomma
\frac{1}{(-p^2+m^2+i \eta )}
\end{dmath*}

If one wants to have additional variables multiplying loop or external
momenta, those need to be declared to be of the \texttt{FCVariable} type

\begin{Shaded}
\begin{Highlighting}[]
\NormalTok{DataType}\OperatorTok{[}\NormalTok{la}\OperatorTok{,}\NormalTok{ FCVariable}\OperatorTok{]} \ExtensionTok{=} \ConstantTok{True}
\end{Highlighting}
\end{Shaded}

\begin{dmath*}\breakingcomma
\text{True}
\end{dmath*}

\begin{Shaded}
\begin{Highlighting}[]
\NormalTok{SFAD}\OperatorTok{[\{\{}\DecValTok{0}\OperatorTok{,}\NormalTok{ la }\FunctionTok{p}\NormalTok{ . }\FunctionTok{q}\OperatorTok{\},} \FunctionTok{m}\SpecialCharTok{\^{}}\DecValTok{2}\OperatorTok{\}]}
\end{Highlighting}
\end{Shaded}

\begin{dmath*}\breakingcomma
\frac{1}{(\text{la} (p\cdot q)-m^2+i \eta )}
\end{dmath*}

\begin{Shaded}
\begin{Highlighting}[]
\SpecialCharTok{\%} \SpecialCharTok{//}\NormalTok{ FCI }\SpecialCharTok{//} \FunctionTok{StandardForm}

\CommentTok{(*FeynAmpDenominator[StandardPropagatorDenominator[0, la Pair[Momentum[p, D], Momentum[q, D]], {-}m\^{}2, \{1, 1\}]]*)}
\end{Highlighting}
\end{Shaded}

\end{document}
