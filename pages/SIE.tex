% !TeX program = pdflatex
% !TeX root = SIE.tex

\documentclass[../FeynCalcManual.tex]{subfiles}
\begin{document}
\hypertarget{sie}{
\section{SIE}\label{sie}\index{SIE}}

\texttt{SIE[\allowbreak{}mu]} can be used as input for
\(D-1\)-dimensional \(\sigma^{\mu }\) with \(D-4\)-dimensional Lorentz
index \(\mu\) and is transformed into
\texttt{PauliSigma[\allowbreak{}LorentzIndex[\allowbreak{}mu,\ \allowbreak{}D-4],\ \allowbreak{}D-4]}
by FeynCalcInternal.

\subsection{See also}

\hyperlink{toc}{Overview}, \hyperlink{paulisigma}{PauliSigma},
\hyperlink{si}{SI}.

\subsection{Examples}

\begin{Shaded}
\begin{Highlighting}[]
\NormalTok{SIE}\OperatorTok{[}\SpecialCharTok{\textbackslash{}}\OperatorTok{[}\NormalTok{Mu}\OperatorTok{]]}
\end{Highlighting}
\end{Shaded}

\begin{dmath*}\breakingcomma
\hat{\sigma }^{\mu }
\end{dmath*}

\begin{Shaded}
\begin{Highlighting}[]
\NormalTok{SIE}\OperatorTok{[}\SpecialCharTok{\textbackslash{}}\OperatorTok{[}\NormalTok{Mu}\OperatorTok{],} \SpecialCharTok{\textbackslash{}}\OperatorTok{[}\NormalTok{Nu}\OperatorTok{]]} \SpecialCharTok{{-}}\NormalTok{ SIE}\OperatorTok{[}\SpecialCharTok{\textbackslash{}}\OperatorTok{[}\NormalTok{Nu}\OperatorTok{],} \SpecialCharTok{\textbackslash{}}\OperatorTok{[}\NormalTok{Mu}\OperatorTok{]]}
\end{Highlighting}
\end{Shaded}

\begin{dmath*}\breakingcomma
\hat{\sigma }^{\mu }.\hat{\sigma }^{\nu }-\hat{\sigma }^{\nu }.\hat{\sigma }^{\mu }
\end{dmath*}

\begin{Shaded}
\begin{Highlighting}[]
\FunctionTok{StandardForm}\OperatorTok{[}\NormalTok{FCI}\OperatorTok{[}\NormalTok{SIE}\OperatorTok{[}\SpecialCharTok{\textbackslash{}}\OperatorTok{[}\NormalTok{Mu}\OperatorTok{]]]]}

\CommentTok{(*PauliSigma[LorentzIndex[\textbackslash{}[Mu], {-}4 + D], {-}4 + D]*)}
\end{Highlighting}
\end{Shaded}

\begin{Shaded}
\begin{Highlighting}[]
\NormalTok{SIE}\OperatorTok{[}\SpecialCharTok{\textbackslash{}}\OperatorTok{[}\NormalTok{Mu}\OperatorTok{],} \SpecialCharTok{\textbackslash{}}\OperatorTok{[}\NormalTok{Nu}\OperatorTok{],} \SpecialCharTok{\textbackslash{}}\OperatorTok{[}\NormalTok{Rho}\OperatorTok{],} \SpecialCharTok{\textbackslash{}}\OperatorTok{[}\NormalTok{Sigma}\OperatorTok{]]}
\end{Highlighting}
\end{Shaded}

\begin{dmath*}\breakingcomma
\hat{\sigma }^{\mu }.\hat{\sigma }^{\nu }.\hat{\sigma }^{\rho }.\hat{\sigma }^{\sigma }
\end{dmath*}

\begin{Shaded}
\begin{Highlighting}[]
\NormalTok{SIE}\OperatorTok{[}\SpecialCharTok{\textbackslash{}}\OperatorTok{[}\NormalTok{Mu}\OperatorTok{],} \SpecialCharTok{\textbackslash{}}\OperatorTok{[}\NormalTok{Nu}\OperatorTok{],} \SpecialCharTok{\textbackslash{}}\OperatorTok{[}\NormalTok{Rho}\OperatorTok{],} \SpecialCharTok{\textbackslash{}}\OperatorTok{[}\NormalTok{Sigma}\OperatorTok{]]} \SpecialCharTok{//} \FunctionTok{StandardForm}

\CommentTok{(*SIE[\textbackslash{}[Mu]] . SIE[\textbackslash{}[Nu]] . SIE[\textbackslash{}[Rho]] . SIE[\textbackslash{}[Sigma]]*)}
\end{Highlighting}
\end{Shaded}

\begin{Shaded}
\begin{Highlighting}[]
\NormalTok{SIE}\OperatorTok{[}\SpecialCharTok{\textbackslash{}}\OperatorTok{[}\NormalTok{Alpha}\OperatorTok{]]}\NormalTok{ . (SISE}\OperatorTok{[}\FunctionTok{p}\OperatorTok{]} \SpecialCharTok{+} \FunctionTok{m}\NormalTok{) . SIE}\OperatorTok{[}\SpecialCharTok{\textbackslash{}}\OperatorTok{[}\FunctionTok{Beta}\OperatorTok{]]}
\end{Highlighting}
\end{Shaded}

\begin{dmath*}\breakingcomma
\hat{\sigma }^{\alpha }.\left(m+\hat{\sigma }\cdot \hat{p}\right).\hat{\sigma }^{\beta }
\end{dmath*}
\end{document}
