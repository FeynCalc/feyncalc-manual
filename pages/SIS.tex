% !TeX program = pdflatex
% !TeX root = SIS.tex

\documentclass[../FeynCalcManual.tex]{subfiles}
\begin{document}
\hypertarget{sis}{%
\section{SIS}\label{sis}}

\texttt{SIS[\allowbreak{}p]} can be used as input for \(3\)-dimensional
\(\sigma^{\mu } p_{\mu }\) with 4-dimensional Lorentz vector \(p\) and
is transformed into
\texttt{PauliSigma[\allowbreak{}Momentum[\allowbreak{}p]]} by
FeynCalcInternal.

\subsection{See also}

\hyperlink{toc}{Overview}, \hyperlink{paulisigma}{PauliSigma},
\hyperlink{sisd}{SISD}.

\subsection{Examples}

\begin{Shaded}
\begin{Highlighting}[]
\NormalTok{SIS}\OperatorTok{[}\FunctionTok{p}\OperatorTok{]}
\end{Highlighting}
\end{Shaded}

\begin{dmath*}\breakingcomma
\bar{\sigma }\cdot \overline{p}
\end{dmath*}

\begin{Shaded}
\begin{Highlighting}[]
\NormalTok{SIS}\OperatorTok{[}\FunctionTok{p}\OperatorTok{]} \SpecialCharTok{//}\NormalTok{ FCI }\SpecialCharTok{//} \FunctionTok{StandardForm}

\CommentTok{(*PauliSigma[Momentum[p]]*)}
\end{Highlighting}
\end{Shaded}

\begin{Shaded}
\begin{Highlighting}[]
\NormalTok{SIS}\OperatorTok{[}\FunctionTok{p}\OperatorTok{,} \FunctionTok{q}\OperatorTok{,} \FunctionTok{r}\OperatorTok{,} \FunctionTok{s}\OperatorTok{]}
\end{Highlighting}
\end{Shaded}

\begin{dmath*}\breakingcomma
\left(\bar{\sigma }\cdot \overline{p}\right).\left(\bar{\sigma }\cdot \overline{q}\right).\left(\bar{\sigma }\cdot \overline{r}\right).\left(\bar{\sigma }\cdot \overline{s}\right)
\end{dmath*}

\begin{Shaded}
\begin{Highlighting}[]
\NormalTok{SIS}\OperatorTok{[}\FunctionTok{p}\OperatorTok{,} \FunctionTok{q}\OperatorTok{,} \FunctionTok{r}\OperatorTok{,} \FunctionTok{s}\OperatorTok{]} \SpecialCharTok{//} \FunctionTok{StandardForm}

\CommentTok{(*SIS[p] . SIS[q] . SIS[r] . SIS[s]*)}
\end{Highlighting}
\end{Shaded}

\begin{Shaded}
\begin{Highlighting}[]
\NormalTok{SIS}\OperatorTok{[}\FunctionTok{q}\OperatorTok{]}\NormalTok{ . (SIS}\OperatorTok{[}\FunctionTok{p}\OperatorTok{]} \SpecialCharTok{+} \FunctionTok{m}\NormalTok{) . SIS}\OperatorTok{[}\FunctionTok{q}\OperatorTok{]}
\end{Highlighting}
\end{Shaded}

\begin{dmath*}\breakingcomma
\left(\bar{\sigma }\cdot \overline{q}\right).\left(\bar{\sigma }\cdot \overline{p}+m\right).\left(\bar{\sigma }\cdot \overline{q}\right)
\end{dmath*}
\end{document}
