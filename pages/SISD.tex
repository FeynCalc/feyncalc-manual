% !TeX program = pdflatex
% !TeX root = SISD.tex

\documentclass[../FeynCalcManual.tex]{subfiles}
\begin{document}
\hypertarget{sisd}{%
\section{SISD}\label{sisd}}

\texttt{SISD[\allowbreak{}p]} can be used as input for
\(D-1\)-dimensional \(\sigma^{\mu } p_{\mu }\) with \(D\)-dimensional
Lorentz vector \(p\) and is transformed into
\texttt{PauliSigma[\allowbreak{}Momentum[\allowbreak{}p,\ \allowbreak{}D],\ \allowbreak{}D-1]}
by \texttt{FeynCalcInternal}.

\subsection{See also}

\hyperlink{toc}{Overview}, \hyperlink{paulisigma}{PauliSigma},
\hyperlink{sis}{SIS}.

\subsection{Examples}

\begin{Shaded}
\begin{Highlighting}[]
\NormalTok{SISD}\OperatorTok{[}\FunctionTok{p}\OperatorTok{]}
\end{Highlighting}
\end{Shaded}

\begin{dmath*}\breakingcomma
\sigma \cdot p
\end{dmath*}

\begin{Shaded}
\begin{Highlighting}[]
\NormalTok{SISD}\OperatorTok{[}\FunctionTok{p}\OperatorTok{]} \SpecialCharTok{//}\NormalTok{ FCI }\SpecialCharTok{//} \FunctionTok{StandardForm}

\CommentTok{(*PauliSigma[Momentum[p, D], {-}1 + D]*)}
\end{Highlighting}
\end{Shaded}

\begin{Shaded}
\begin{Highlighting}[]
\NormalTok{SISD}\OperatorTok{[}\FunctionTok{p}\OperatorTok{,} \FunctionTok{q}\OperatorTok{,} \FunctionTok{r}\OperatorTok{,} \FunctionTok{s}\OperatorTok{]}
\end{Highlighting}
\end{Shaded}

\begin{dmath*}\breakingcomma
(\sigma \cdot p).(\sigma \cdot q).(\sigma \cdot r).(\sigma \cdot s)
\end{dmath*}

\begin{Shaded}
\begin{Highlighting}[]
\NormalTok{SISD}\OperatorTok{[}\FunctionTok{p}\OperatorTok{,} \FunctionTok{q}\OperatorTok{,} \FunctionTok{r}\OperatorTok{,} \FunctionTok{s}\OperatorTok{]} \SpecialCharTok{//} \FunctionTok{StandardForm}

\CommentTok{(*SISD[p] . SISD[q] . SISD[r] . SISD[s]*)}
\end{Highlighting}
\end{Shaded}

\begin{Shaded}
\begin{Highlighting}[]
\NormalTok{SISD}\OperatorTok{[}\FunctionTok{q}\OperatorTok{]}\NormalTok{ . (SISD}\OperatorTok{[}\FunctionTok{p}\OperatorTok{]} \SpecialCharTok{+} \FunctionTok{m}\NormalTok{) . SISD}\OperatorTok{[}\FunctionTok{q}\OperatorTok{]}
\end{Highlighting}
\end{Shaded}

\begin{dmath*}\breakingcomma
(\sigma \cdot q).(m+\sigma \cdot p).(\sigma \cdot q)
\end{dmath*}
\end{document}
