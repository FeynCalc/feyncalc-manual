% !TeX program = pdflatex
% !TeX root = SISE.tex

\documentclass[../FeynCalcManual.tex]{subfiles}
\begin{document}
\hypertarget{sise}{
\section{SISE}\label{sise}\index{SISE}}

\texttt{SISE[\allowbreak{}p]} can be used as input for
\(D-4\)-dimensional \(\sigma ^{\mu } p_{\mu }\) with \(D-4\)-dimensional
Lorentz vector \(p\) and is transformed into
\texttt{PauliSigma[\allowbreak{}Momentum[\allowbreak{}p,\ \allowbreak{}D-4],\ \allowbreak{}D-4]}
by \texttt{FeynCalcInternal}.

\subsection{See also}

\hyperlink{toc}{Overview}, \hyperlink{sis}{SIS},
\hyperlink{paulisigma}{PauliSigma}.

\subsection{Examples}

\begin{Shaded}
\begin{Highlighting}[]
\NormalTok{SISE}\OperatorTok{[}\FunctionTok{p}\OperatorTok{]}
\end{Highlighting}
\end{Shaded}

\begin{dmath*}\breakingcomma
\hat{\sigma }\cdot \hat{p}
\end{dmath*}

\begin{Shaded}
\begin{Highlighting}[]
\NormalTok{SISE}\OperatorTok{[}\FunctionTok{p}\OperatorTok{]} \SpecialCharTok{//}\NormalTok{ FCI }\SpecialCharTok{//} \FunctionTok{StandardForm}

\CommentTok{(*PauliSigma[Momentum[p, {-}4 + D], {-}4 + D]*)}
\end{Highlighting}
\end{Shaded}

\begin{Shaded}
\begin{Highlighting}[]
\NormalTok{SISE}\OperatorTok{[}\FunctionTok{p}\OperatorTok{,} \FunctionTok{q}\OperatorTok{,} \FunctionTok{r}\OperatorTok{,} \FunctionTok{s}\OperatorTok{]}
\end{Highlighting}
\end{Shaded}

\begin{dmath*}\breakingcomma
\left(\hat{\sigma }\cdot \hat{p}\right).\left(\hat{\sigma }\cdot \hat{q}\right).\left(\hat{\sigma }\cdot \hat{r}\right).\left(\hat{\sigma }\cdot \hat{s}\right)
\end{dmath*}

\begin{Shaded}
\begin{Highlighting}[]
\NormalTok{SISE}\OperatorTok{[}\FunctionTok{p}\OperatorTok{,} \FunctionTok{q}\OperatorTok{,} \FunctionTok{r}\OperatorTok{,} \FunctionTok{s}\OperatorTok{]} \SpecialCharTok{//} \FunctionTok{StandardForm}

\CommentTok{(*SISE[p] . SISE[q] . SISE[r] . SISE[s]*)}
\end{Highlighting}
\end{Shaded}

\end{document}
