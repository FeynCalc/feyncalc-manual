% !TeX program = pdflatex
% !TeX root = SMPToSymbol.tex

\documentclass[../FeynCalcManual.tex]{subfiles}
\begin{document}
\hypertarget{smptosymbol}{%
\section{SMPToSymbol}\label{smptosymbol}}

\texttt{SMPToSymbol[\allowbreak{}exp]} converts objects of type
\texttt{SMP[\allowbreak{}"sth"]} in \texttt{exp} to symbols using
\texttt{ToExpression[\allowbreak{}"sth"]}.

The option \texttt{StringReplace} can be used to specify string
replacement rules that will take care of special characters
(e.g.~\texttt{^} or \texttt{_}) that cannot appear in valid Mathematica
expressions. \texttt{SMPToSymbol} is useful when exporting FeynCalc
expressions to other tools, e.g.~FORM.

\subsection{See also}

\hyperlink{toc}{Overview}, \hyperlink{smp}{SMP},
\hyperlink{fcgvtosymbol}{FCGVToSymbol}.

\subsection{Examples}

\begin{Shaded}
\begin{Highlighting}[]
\NormalTok{SP}\OperatorTok{[}\FunctionTok{p}\OperatorTok{]} \SpecialCharTok{{-}}\NormalTok{ SMP}\OperatorTok{[}\StringTok{"m\_e"}\OperatorTok{]}\SpecialCharTok{\^{}}\DecValTok{2} 
 
\NormalTok{SMPToSymbol}\OperatorTok{[}\SpecialCharTok{\%}\OperatorTok{]}
\end{Highlighting}
\end{Shaded}

\begin{dmath*}\breakingcomma
\overline{p}^2-m_e^2
\end{dmath*}

\begin{dmath*}\breakingcomma
\overline{p}^2-\text{me}^2
\end{dmath*}
\end{document}
