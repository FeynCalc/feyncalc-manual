% !TeX program = pdflatex
% !TeX root = SMVertex.tex

\documentclass[../FeynCalcManual.tex]{subfiles}
\begin{document}
\hypertarget{smvertex}{%
\section{SMVertex}\label{smvertex}}

\texttt{SMVertex} is a library of SM vertices. Currently it implements
only few vertices and is not really useful.

\subsection{See also}

\hyperlink{toc}{Overview}, \hyperlink{smp}{SMP}.

\subsection{Examples}

This is the \(\gamma W W\) vertex (all momenta ingoing)

\begin{Shaded}
\begin{Highlighting}[]
\NormalTok{SMVertex}\OperatorTok{[}\StringTok{"AWW"}\OperatorTok{,} \FunctionTok{p}\OperatorTok{,} \SpecialCharTok{\textbackslash{}}\OperatorTok{[}\NormalTok{Mu}\OperatorTok{],} \FunctionTok{q}\OperatorTok{,} \SpecialCharTok{\textbackslash{}}\OperatorTok{[}\NormalTok{Nu}\OperatorTok{],} \FunctionTok{k}\OperatorTok{,} \SpecialCharTok{\textbackslash{}}\OperatorTok{[}\NormalTok{Rho}\OperatorTok{]]}
\end{Highlighting}
\end{Shaded}

\begin{dmath*}\breakingcomma
-i \;\text{e} \left(\bar{g}^{\mu \rho } \left(\overline{p}-\overline{k}\right)^{\nu }+\bar{g}^{\nu \rho } \left(\overline{k}-\overline{q}\right)^{\mu }+\bar{g}^{\mu \nu } \left(\overline{q}-\overline{p}\right)^{\rho }\right)
\end{dmath*}

This is the \(HHH\)-coupling

\begin{Shaded}
\begin{Highlighting}[]
\NormalTok{SMVertex}\OperatorTok{[}\StringTok{"HHH"}\OperatorTok{]}
\end{Highlighting}
\end{Shaded}

\begin{dmath*}\breakingcomma
-\frac{3 i \;\text{e} m_H^2}{2 m_W \left(\left.\sin (\theta _W\right)\right)}
\end{dmath*}

This is the \(H e\)-coupling

\begin{Shaded}
\begin{Highlighting}[]
\NormalTok{SMVertex}\OperatorTok{[}\StringTok{"eeH"}\OperatorTok{]}
\end{Highlighting}
\end{Shaded}

\begin{dmath*}\breakingcomma
-\frac{i \;\text{e} m_e}{2 m_W \left(\left.\sin (\theta _W\right)\right)}
\end{dmath*}
\end{document}
