% !TeX program = pdflatex
% !TeX root = SO.tex

\documentclass[../FeynCalcManual.tex]{subfiles}
\begin{document}
\hypertarget{so}{
\section{SO}\label{so}\index{SO}}

\texttt{SO[\allowbreak{}q]} is a four-dimensional scalar product of
\texttt{OPEDelta} with \texttt{q}. It is transformed into
\texttt{Pair[\allowbreak{}Momentum[\allowbreak{}q],\ \allowbreak{}Momentum[\allowbreak{}OPEDelta]}
by \texttt{FCI}.

\subsection{See also}

\hyperlink{toc}{Overview}, \hyperlink{fci}{FCI},
\hyperlink{opedelta}{OPEDelta}, \hyperlink{pair}{Pair},
\hyperlink{scalarproduct}{ScalarProduct}, \hyperlink{sod}{SOD}.

\subsection{Examples}

\begin{Shaded}
\begin{Highlighting}[]
\NormalTok{SO}\OperatorTok{[}\FunctionTok{p}\OperatorTok{]}
\end{Highlighting}
\end{Shaded}

\begin{dmath*}\breakingcomma
\Delta \cdot p
\end{dmath*}

\begin{Shaded}
\begin{Highlighting}[]
\NormalTok{SO}\OperatorTok{[}\FunctionTok{p} \SpecialCharTok{{-}} \FunctionTok{q}\OperatorTok{]}
\end{Highlighting}
\end{Shaded}

\begin{dmath*}\breakingcomma
\Delta \cdot (p-q)
\end{dmath*}

\begin{Shaded}
\begin{Highlighting}[]
\NormalTok{SO}\OperatorTok{[}\FunctionTok{p}\OperatorTok{]} \SpecialCharTok{//}\NormalTok{ FCI }\SpecialCharTok{//} \FunctionTok{StandardForm}

\CommentTok{(*Pair[Momentum[OPEDelta], Momentum[p]]*)}
\end{Highlighting}
\end{Shaded}

\end{document}
