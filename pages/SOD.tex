% !TeX program = pdflatex
% !TeX root = SOD.tex

\documentclass[../FeynCalcManual.tex]{subfiles}
\begin{document}
\hypertarget{sod}{
\section{SOD}\label{sod}\index{SOD}}

\texttt{SOD[\allowbreak{}q]} is a \(D\)-dimensional scalar product of
\texttt{OPEDelta} with \texttt{q}. It is transformed into
\texttt{Pair[\allowbreak{}Momentum[\allowbreak{}q,\ \allowbreak{}D],\ \allowbreak{}Momentum[\allowbreak{}OPEDelta,\ \allowbreak{}D]}
by \texttt{FeynCalcInternal}.

\subsection{See also}

\hyperlink{toc}{Overview}, \hyperlink{opedelta}{OPEDelta},
\hyperlink{pair}{Pair}, \hyperlink{scalarproduct}{ScalarProduct},
\hyperlink{sod}{SOD}.

\subsection{Examples}

\begin{Shaded}
\begin{Highlighting}[]
\NormalTok{SOD}\OperatorTok{[}\FunctionTok{p}\OperatorTok{]}
\end{Highlighting}
\end{Shaded}

\begin{dmath*}\breakingcomma
\Delta \cdot p
\end{dmath*}

\begin{Shaded}
\begin{Highlighting}[]
\NormalTok{SOD}\OperatorTok{[}\FunctionTok{p} \SpecialCharTok{{-}} \FunctionTok{q}\OperatorTok{]}
\end{Highlighting}
\end{Shaded}

\begin{dmath*}\breakingcomma
\Delta \cdot (p-q)
\end{dmath*}

\begin{Shaded}
\begin{Highlighting}[]
\NormalTok{SOD}\OperatorTok{[}\FunctionTok{p}\OperatorTok{]} \SpecialCharTok{//}\NormalTok{ FCI }\SpecialCharTok{//} \FunctionTok{StandardForm}

\CommentTok{(*Pair[Momentum[OPEDelta, D], Momentum[p, D]]*)}
\end{Highlighting}
\end{Shaded}

\end{document}
