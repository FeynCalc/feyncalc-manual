% !TeX program = pdflatex
% !TeX root = SPD.tex

\documentclass[../FeynCalcManual.tex]{subfiles}
\begin{document}
\hypertarget{spd}{%
\section{SPD}\label{spd}}

\texttt{SPD[\allowbreak{}a,\ \allowbreak{}b]} denotes a
\(D\)-dimensional scalar product.

\texttt{SPD[\allowbreak{}a,\ \allowbreak{}b]} is transformed into
\texttt{ScalarProduct[\allowbreak{}a,\ \allowbreak{}b,\ \allowbreak{}Dimension->D]}
by \texttt{FeynCalcInternal}.

\texttt{SPD[\allowbreak{}p]} is the same as
\texttt{SPD[\allowbreak{}p,\ \allowbreak{}p]} \((=p^2)\).

\subsection{See also}

\hyperlink{toc}{Overview}, \hyperlink{pd}{PD}, \hyperlink{calc}{Calc},
\hyperlink{expandscalarproduct}{ExpandScalarProduct},
\hyperlink{scalarproduct}{ScalarProduct}.

\subsection{Examples}

\begin{Shaded}
\begin{Highlighting}[]
\NormalTok{SPD}\OperatorTok{[}\FunctionTok{p}\OperatorTok{,} \FunctionTok{q}\OperatorTok{]} \SpecialCharTok{+}\NormalTok{ SPD}\OperatorTok{[}\FunctionTok{q}\OperatorTok{]}
\end{Highlighting}
\end{Shaded}

\begin{dmath*}\breakingcomma
p\cdot q+q^2
\end{dmath*}

\begin{Shaded}
\begin{Highlighting}[]
\NormalTok{SPD}\OperatorTok{[}\FunctionTok{p} \SpecialCharTok{{-}} \FunctionTok{q}\OperatorTok{,} \FunctionTok{q} \SpecialCharTok{+} \DecValTok{2} \FunctionTok{p}\OperatorTok{]}
\end{Highlighting}
\end{Shaded}

\begin{dmath*}\breakingcomma
(p-q)\cdot (2 p+q)
\end{dmath*}

\begin{Shaded}
\begin{Highlighting}[]
\NormalTok{Calc}\OperatorTok{[}\NormalTok{ SPD}\OperatorTok{[}\FunctionTok{p} \SpecialCharTok{{-}} \FunctionTok{q}\OperatorTok{,} \FunctionTok{q} \SpecialCharTok{+} \DecValTok{2} \FunctionTok{p}\OperatorTok{]} \OperatorTok{]}
\end{Highlighting}
\end{Shaded}

\begin{dmath*}\breakingcomma
-p\cdot q+2 p^2-q^2
\end{dmath*}

\begin{Shaded}
\begin{Highlighting}[]
\NormalTok{ExpandScalarProduct}\OperatorTok{[}\NormalTok{SPD}\OperatorTok{[}\FunctionTok{p} \SpecialCharTok{{-}} \FunctionTok{q}\OperatorTok{]]}
\end{Highlighting}
\end{Shaded}

\begin{dmath*}\breakingcomma
-2 (p\cdot q)+p^2+q^2
\end{dmath*}

\begin{Shaded}
\begin{Highlighting}[]
\NormalTok{SPD}\OperatorTok{[}\FunctionTok{a}\OperatorTok{,} \FunctionTok{b}\OperatorTok{]} \SpecialCharTok{//} \FunctionTok{StandardForm}

\CommentTok{(*SPD[a, b]*)}
\end{Highlighting}
\end{Shaded}

\begin{Shaded}
\begin{Highlighting}[]
\NormalTok{SPD}\OperatorTok{[}\FunctionTok{a}\OperatorTok{,} \FunctionTok{b}\OperatorTok{]} \SpecialCharTok{//}\NormalTok{ FCI }\SpecialCharTok{//} \FunctionTok{StandardForm}

\CommentTok{(*Pair[Momentum[a, D], Momentum[b, D]]*)}
\end{Highlighting}
\end{Shaded}

\begin{Shaded}
\begin{Highlighting}[]
\NormalTok{SPD}\OperatorTok{[}\FunctionTok{a}\OperatorTok{,} \FunctionTok{b}\OperatorTok{]} \SpecialCharTok{//}\NormalTok{ FCI }\SpecialCharTok{//}\NormalTok{ FCE }\SpecialCharTok{//} \FunctionTok{StandardForm}

\CommentTok{(*SPD[a, b]*)}
\end{Highlighting}
\end{Shaded}

\begin{Shaded}
\begin{Highlighting}[]
\NormalTok{FCE}\OperatorTok{[}\NormalTok{ChangeDimension}\OperatorTok{[}\NormalTok{SP}\OperatorTok{[}\FunctionTok{p}\OperatorTok{,} \FunctionTok{q}\OperatorTok{],} \FunctionTok{D}\OperatorTok{]]} \SpecialCharTok{//} \FunctionTok{StandardForm}

\CommentTok{(*SPD[p, q]*)}
\end{Highlighting}
\end{Shaded}

\end{document}
