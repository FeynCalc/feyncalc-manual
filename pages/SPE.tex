% !TeX program = pdflatex
% !TeX root = SPE.tex

\documentclass[../FeynCalcManual.tex]{subfiles}
\begin{document}
\hypertarget{spe}{%
\section{SPE}\label{spe}}

\texttt{SPE[\allowbreak{}a,\ \allowbreak{}b]} denotes a
\(D-4\)-dimensional scalar product.
\texttt{SPE[\allowbreak{}a,\ \allowbreak{}b]} is transformed into
\texttt{Pair[\allowbreak{}Momentum[\allowbreak{}a,\ \allowbreak{}-4 + D],\ \allowbreak{}Momentum[\allowbreak{}b,\ \allowbreak{}-4 + D]]}
by \texttt{FeynCalcInternal}.

\texttt{SPE[\allowbreak{}p]} is the same as
\texttt{SPE[\allowbreak{}p,\ \allowbreak{}p]} \((=p^2)\).

\subsection{See also}

\hyperlink{toc}{Overview}, \hyperlink{pd}{PD}, \hyperlink{calc}{Calc},
\hyperlink{expandscalarproduct}{ExpandScalarProduct},
\hyperlink{scalarproduct}{ScalarProduct}, \hyperlink{spd}{SPD}.

\subsection{Examples}

\begin{Shaded}
\begin{Highlighting}[]
\NormalTok{SPE}\OperatorTok{[}\FunctionTok{p}\OperatorTok{,} \FunctionTok{q}\OperatorTok{]} \SpecialCharTok{+}\NormalTok{ SPE}\OperatorTok{[}\FunctionTok{q}\OperatorTok{]}
\end{Highlighting}
\end{Shaded}

\begin{dmath*}\breakingcomma
\hat{p}\cdot \hat{q}+\hat{q}^2
\end{dmath*}

\begin{Shaded}
\begin{Highlighting}[]
\NormalTok{SPE}\OperatorTok{[}\FunctionTok{p} \SpecialCharTok{{-}} \FunctionTok{q}\OperatorTok{,} \FunctionTok{q} \SpecialCharTok{+} \DecValTok{2} \FunctionTok{p}\OperatorTok{]}
\end{Highlighting}
\end{Shaded}

\begin{dmath*}\breakingcomma
(\hat{p}-\hat{q})\cdot (2 \hat{p}+\hat{q})
\end{dmath*}

\begin{Shaded}
\begin{Highlighting}[]
\NormalTok{Calc}\OperatorTok{[}\NormalTok{ SPE}\OperatorTok{[}\FunctionTok{p} \SpecialCharTok{{-}} \FunctionTok{q}\OperatorTok{,} \FunctionTok{q} \SpecialCharTok{+} \DecValTok{2} \FunctionTok{p}\OperatorTok{]} \OperatorTok{]}
\end{Highlighting}
\end{Shaded}

\begin{dmath*}\breakingcomma
-\hat{p}\cdot \hat{q}+2 \hat{p}^2-\hat{q}^2
\end{dmath*}

\begin{Shaded}
\begin{Highlighting}[]
\NormalTok{ExpandScalarProduct}\OperatorTok{[}\NormalTok{SPE}\OperatorTok{[}\FunctionTok{p} \SpecialCharTok{{-}} \FunctionTok{q}\OperatorTok{]]}
\end{Highlighting}
\end{Shaded}

\begin{dmath*}\breakingcomma
-2 \left(\hat{p}\cdot \hat{q}\right)+\hat{p}^2+\hat{q}^2
\end{dmath*}

\begin{Shaded}
\begin{Highlighting}[]
\NormalTok{SPE}\OperatorTok{[}\FunctionTok{a}\OperatorTok{,} \FunctionTok{b}\OperatorTok{]} \SpecialCharTok{//} \FunctionTok{StandardForm}

\CommentTok{(*SPE[a, b]*)}
\end{Highlighting}
\end{Shaded}

\begin{Shaded}
\begin{Highlighting}[]
\NormalTok{SPE}\OperatorTok{[}\FunctionTok{a}\OperatorTok{,} \FunctionTok{b}\OperatorTok{]} \SpecialCharTok{//}\NormalTok{ FCI }\SpecialCharTok{//} \FunctionTok{StandardForm}

\CommentTok{(*Pair[Momentum[a, {-}4 + D], Momentum[b, {-}4 + D]]*)}
\end{Highlighting}
\end{Shaded}

\begin{Shaded}
\begin{Highlighting}[]
\NormalTok{SPE}\OperatorTok{[}\FunctionTok{a}\OperatorTok{,} \FunctionTok{b}\OperatorTok{]} \SpecialCharTok{//}\NormalTok{ FCI }\SpecialCharTok{//}\NormalTok{ FCE }\SpecialCharTok{//} \FunctionTok{StandardForm}

\CommentTok{(*SPE[a, b]*)}
\end{Highlighting}
\end{Shaded}

\begin{Shaded}
\begin{Highlighting}[]
\NormalTok{FCE}\OperatorTok{[}\NormalTok{ChangeDimension}\OperatorTok{[}\NormalTok{SP}\OperatorTok{[}\FunctionTok{p}\OperatorTok{,} \FunctionTok{q}\OperatorTok{],} \FunctionTok{D} \SpecialCharTok{{-}} \DecValTok{4}\OperatorTok{]]} \SpecialCharTok{//} \FunctionTok{StandardForm}

\CommentTok{(*SPE[p, q]*)}
\end{Highlighting}
\end{Shaded}

\end{document}
