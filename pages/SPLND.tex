% !TeX program = pdflatex
% !TeX root = SPLND.tex

\documentclass[../FeynCalcManual.tex]{subfiles}
\begin{document}
\hypertarget{splnd}{
\section{SPLND}\label{splnd}\index{SPLND}}

\texttt{SPLND[\allowbreak{}p,\ \allowbreak{}q,\ \allowbreak{}n,\ \allowbreak{}nb]}
denotes the negative component in the lightcone decomposition of the
scalar product \(p \cdot q\) along the vectors \texttt{n} and
\texttt{nb} in \(D\)-dimensions. It corresponds to
\(\frac{1}{2} (p \cdot \bar{n}) (q \cdot n)\).

If one omits \texttt{n} and \texttt{nb}, the program will use default
vectors specified via \texttt{\$FCDefaultLightconeVectorN} and
\texttt{\$FCDefaultLightconeVectorNB}.

\subsection{See also}

\hyperlink{toc}{Overview}, \hyperlink{pair}{Pair},
\hyperlink{fvlnd}{FVLND}, \hyperlink{fvlpd}{FVLPD},
\hyperlink{fvlrd}{FVLRD}, \hyperlink{splpd}{SPLPD},
\hyperlink{splrd}{SPLRD}, \hyperlink{mtlpd}{MTLPD},
\hyperlink{mtlnd}{MTLND}, \hyperlink{mtlrd}{MTLRD}.

\subsection{Examples}

\begin{Shaded}
\begin{Highlighting}[]
\NormalTok{SPLND}\OperatorTok{[}\FunctionTok{p}\OperatorTok{,} \FunctionTok{q}\OperatorTok{,} \FunctionTok{n}\OperatorTok{,}\NormalTok{ nb}\OperatorTok{]}
\end{Highlighting}
\end{Shaded}

\begin{dmath*}\breakingcomma
\frac{1}{2} (n\cdot q) (\text{nb}\cdot p)
\end{dmath*}

\begin{Shaded}
\begin{Highlighting}[]
\FunctionTok{StandardForm}\OperatorTok{[}\NormalTok{SPLND}\OperatorTok{[}\FunctionTok{p}\OperatorTok{,} \FunctionTok{q}\OperatorTok{,} \FunctionTok{n}\OperatorTok{,}\NormalTok{ nb}\OperatorTok{]} \SpecialCharTok{//}\NormalTok{ FCI}\OperatorTok{]}
\end{Highlighting}
\end{Shaded}

\begin{dmath*}\breakingcomma
\frac{1}{2} \;\text{Pair}[\text{Momentum}[n,D],\text{Momentum}[q,D]] \;\text{Pair}[\text{Momentum}[\text{nb},D],\text{Momentum}[p,D]]
\end{dmath*}

Notice that the properties of \texttt{n} and \texttt{nb} vectors have to
be set by hand before doing the actual computation

\begin{Shaded}
\begin{Highlighting}[]
\NormalTok{SPLND}\OperatorTok{[}\NormalTok{p1 }\SpecialCharTok{+}\NormalTok{ p2 }\SpecialCharTok{+} \FunctionTok{n}\OperatorTok{,} \FunctionTok{q}\OperatorTok{,} \FunctionTok{n}\OperatorTok{,}\NormalTok{ nb}\OperatorTok{]} \SpecialCharTok{//}\NormalTok{ ExpandScalarProduct}
\end{Highlighting}
\end{Shaded}

\begin{dmath*}\breakingcomma
\frac{1}{2} (n\cdot q) (n\cdot \;\text{nb}+\text{nb}\cdot \;\text{p1}+\text{nb}\cdot \;\text{p2})
\end{dmath*}

\begin{Shaded}
\begin{Highlighting}[]
\NormalTok{FCClearScalarProducts}\OperatorTok{[]}
\NormalTok{SPD}\OperatorTok{[}\FunctionTok{n}\OperatorTok{]} \ExtensionTok{=} \DecValTok{0}\NormalTok{;}
\NormalTok{SPD}\OperatorTok{[}\NormalTok{nb}\OperatorTok{]} \ExtensionTok{=} \DecValTok{0}\NormalTok{;}
\NormalTok{SPD}\OperatorTok{[}\FunctionTok{n}\OperatorTok{,}\NormalTok{ nb}\OperatorTok{]} \ExtensionTok{=} \DecValTok{2}\NormalTok{;}
\end{Highlighting}
\end{Shaded}

\begin{Shaded}
\begin{Highlighting}[]
\NormalTok{SPLND}\OperatorTok{[}\NormalTok{p1 }\SpecialCharTok{+}\NormalTok{ p2 }\SpecialCharTok{+} \FunctionTok{n}\OperatorTok{,} \FunctionTok{q}\OperatorTok{,} \FunctionTok{n}\OperatorTok{,}\NormalTok{ nb}\OperatorTok{]} \SpecialCharTok{//}\NormalTok{ ExpandScalarProduct}
\end{Highlighting}
\end{Shaded}

\begin{dmath*}\breakingcomma
\frac{1}{2} (n\cdot q) (\text{nb}\cdot \;\text{p1}+\text{nb}\cdot \;\text{p2}+2)
\end{dmath*}

\begin{Shaded}
\begin{Highlighting}[]
\NormalTok{FCClearScalarProducts}\OperatorTok{[]}
\end{Highlighting}
\end{Shaded}

\end{document}
