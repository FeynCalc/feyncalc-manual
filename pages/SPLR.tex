% !TeX program = pdflatex
% !TeX root = SPLR.tex

\documentclass[../FeynCalcManual.tex]{subfiles}
\begin{document}
\hypertarget{splr}{
\section{SPLR}\label{splr}\index{SPLR}}

\texttt{SPLR[\allowbreak{}p,\ \allowbreak{}q,\ \allowbreak{}n,\ \allowbreak{}nb]}
denotes the perpendicular component in the lightcone decomposition of
the scalar product \(p \cdot q\) along the vectors \texttt{n} and
\texttt{nb}. It corresponds to \((p \cdot q)_{\perp}\).

If one omits \texttt{n} and \texttt{nb}, the program will use default
vectors specified via \texttt{\$FCDefaultLightconeVectorN} and
\texttt{\$FCDefaultLightconeVectorNB}.

\subsection{See also}

\hyperlink{toc}{Overview}, \hyperlink{pair}{Pair},
\hyperlink{fvln}{FVLN}, \hyperlink{fvlp}{FVLP}, \hyperlink{fvlr}{FVLR},
\hyperlink{splp}{SPLP}, \hyperlink{spln}{SPLN}, \hyperlink{mtlp}{MTLP},
\hyperlink{mtln}{MTLN}, \hyperlink{mtlr}{MTLR}.

\subsection{Examples}

\begin{Shaded}
\begin{Highlighting}[]
\NormalTok{SPLR}\OperatorTok{[}\FunctionTok{p}\OperatorTok{,} \FunctionTok{q}\OperatorTok{,} \FunctionTok{n}\OperatorTok{,}\NormalTok{ nb}\OperatorTok{]}
\end{Highlighting}
\end{Shaded}

\begin{dmath*}\breakingcomma
\overline{p}\cdot \overline{q}_{\perp }
\end{dmath*}

\begin{Shaded}
\begin{Highlighting}[]
\FunctionTok{StandardForm}\OperatorTok{[}\NormalTok{SPLR}\OperatorTok{[}\FunctionTok{p}\OperatorTok{,} \FunctionTok{q}\OperatorTok{,} \FunctionTok{n}\OperatorTok{,}\NormalTok{ nb}\OperatorTok{]} \SpecialCharTok{//}\NormalTok{ FCI}\OperatorTok{]}

\CommentTok{(*Pair[LightConePerpendicularComponent[Momentum[p], Momentum[n], Momentum[nb]], LightConePerpendicularComponent[Momentum[q], Momentum[n], Momentum[nb]]]*)}
\end{Highlighting}
\end{Shaded}

Notice that the properties of \texttt{n} and \texttt{nb} vectors have to
be set by hand before doing the actual computation

\begin{Shaded}
\begin{Highlighting}[]
\NormalTok{SPLR}\OperatorTok{[}\NormalTok{p1 }\SpecialCharTok{+}\NormalTok{ p2}\OperatorTok{,}\NormalTok{ q1 }\SpecialCharTok{+}\NormalTok{ q2}\OperatorTok{,} \FunctionTok{n}\OperatorTok{,}\NormalTok{ nb}\OperatorTok{]} \SpecialCharTok{//}\NormalTok{ FCI }\SpecialCharTok{//}\NormalTok{ ExpandScalarProduct}
\end{Highlighting}
\end{Shaded}

\begin{dmath*}\breakingcomma
\overline{\text{p1}}\cdot \overline{\text{q1}}_{\perp }+\overline{\text{p1}}\cdot \overline{\text{q2}}_{\perp }+\overline{\text{p2}}\cdot \overline{\text{q1}}_{\perp }+\overline{\text{p2}}\cdot \overline{\text{q2}}_{\perp }
\end{dmath*}

\begin{Shaded}
\begin{Highlighting}[]
\NormalTok{SPLR}\OperatorTok{[}\NormalTok{p1 }\SpecialCharTok{+}\NormalTok{ p2 }\SpecialCharTok{+} \FunctionTok{n}\OperatorTok{,} \FunctionTok{q}\OperatorTok{,} \FunctionTok{n}\OperatorTok{,}\NormalTok{ nb}\OperatorTok{]} \SpecialCharTok{//}\NormalTok{ FCI }\SpecialCharTok{//}\NormalTok{ ExpandScalarProduct}
\end{Highlighting}
\end{Shaded}

\begin{dmath*}\breakingcomma
\overline{\text{p1}}\cdot \overline{q}_{\perp }+\overline{\text{p2}}\cdot \overline{q}_{\perp }
\end{dmath*}
\end{document}
