% !TeX program = pdflatex
% !TeX root = SPLRD.tex

\documentclass[../FeynCalcManual.tex]{subfiles}
\begin{document}
\hypertarget{splrd}{
\section{SPLRD}\label{splrd}\index{SPLRD}}

\texttt{SPLRD[\allowbreak{}p,\ \allowbreak{}q,\ \allowbreak{}n,\ \allowbreak{}nb]}
denotes the perpendicular component in the lightcone decomposition of
the scalar product \(p \cdot q\) along the vectors \texttt{n} and
\texttt{nb}. It corresponds to \((p \cdot q)_{\perp}\).

If one omits \texttt{n} and \texttt{nb}, the program will use default
vectors specified via \texttt{\$FCDefaultLightconeVectorN} and
\texttt{\$FCDefaultLightconeVectorNB}.

\subsection{See also}

\hyperlink{toc}{Overview}, \hyperlink{pair}{Pair},
\hyperlink{fvlnd}{FVLND}, \hyperlink{fvlpd}{FVLPD},
\hyperlink{fvlrd}{FVLRD}, \hyperlink{splpd}{SPLPD},
\hyperlink{splnd}{SPLND}, \hyperlink{mtlpd}{MTLPD},
\hyperlink{mtlnd}{MTLND}, \hyperlink{mtlrd}{MTLRD}.

\subsection{Examples}

\begin{Shaded}
\begin{Highlighting}[]
\NormalTok{SPLRD}\OperatorTok{[}\FunctionTok{p}\OperatorTok{,} \FunctionTok{q}\OperatorTok{,} \FunctionTok{n}\OperatorTok{,}\NormalTok{ nb}\OperatorTok{]}
\end{Highlighting}
\end{Shaded}

\begin{dmath*}\breakingcomma
p\cdot q_{\perp }
\end{dmath*}

\begin{Shaded}
\begin{Highlighting}[]
\FunctionTok{StandardForm}\OperatorTok{[}\NormalTok{SPLRD}\OperatorTok{[}\FunctionTok{p}\OperatorTok{,} \FunctionTok{q}\OperatorTok{,} \FunctionTok{n}\OperatorTok{,}\NormalTok{ nb}\OperatorTok{]} \SpecialCharTok{//}\NormalTok{ FCI}\OperatorTok{]}

\CommentTok{(*Pair[LightConePerpendicularComponent[Momentum[p, D], Momentum[n, D], Momentum[nb, D]], LightConePerpendicularComponent[Momentum[q, D], Momentum[n, D], Momentum[nb, D]]]*)}
\end{Highlighting}
\end{Shaded}

Notice that the properties of \texttt{n} and \texttt{nb} vectors have to
be set by hand before doing the actual computation

\begin{Shaded}
\begin{Highlighting}[]
\NormalTok{SPLRD}\OperatorTok{[}\NormalTok{p1 }\SpecialCharTok{+}\NormalTok{ p2}\OperatorTok{,}\NormalTok{ q1 }\SpecialCharTok{+}\NormalTok{ q2}\OperatorTok{,} \FunctionTok{n}\OperatorTok{,}\NormalTok{ nb}\OperatorTok{]} \SpecialCharTok{//}\NormalTok{ FCI }\SpecialCharTok{//}\NormalTok{ ExpandScalarProduct}
\end{Highlighting}
\end{Shaded}

\begin{dmath*}\breakingcomma
\text{p1}\cdot \;\text{q1}_{\perp }+\text{p1}\cdot \;\text{q2}_{\perp }+\text{p2}\cdot \;\text{q1}_{\perp }+\text{p2}\cdot \;\text{q2}_{\perp }
\end{dmath*}

\begin{Shaded}
\begin{Highlighting}[]
\NormalTok{SPLRD}\OperatorTok{[}\NormalTok{p1 }\SpecialCharTok{+}\NormalTok{ p2 }\SpecialCharTok{+} \FunctionTok{n}\OperatorTok{,} \FunctionTok{q}\OperatorTok{,} \FunctionTok{n}\OperatorTok{,}\NormalTok{ nb}\OperatorTok{]} \SpecialCharTok{//}\NormalTok{ FCI }\SpecialCharTok{//}\NormalTok{ ExpandScalarProduct}
\end{Highlighting}
\end{Shaded}

\begin{dmath*}\breakingcomma
\text{p1}\cdot q_{\perp }+\text{p2}\cdot q_{\perp }
\end{dmath*}
\end{document}
