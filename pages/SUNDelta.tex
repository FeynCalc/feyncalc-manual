% !TeX program = pdflatex
% !TeX root = SUNDelta.tex

\documentclass[../FeynCalcManual.tex]{subfiles}
\begin{document}
\hypertarget{sundelta}{%
\section{SUNDelta}\label{sundelta}}

\texttt{SUNDelta[\allowbreak{}a,\ \allowbreak{}b]} is the
Kronecker-delta for \(SU(N)\) with color indices \texttt{a} and
\texttt{b} in the adjoint representation.

\subsection{See also}

\hyperlink{toc}{Overview},
\hyperlink{explicitsunindex}{ExplicitSUNIndex}, \hyperlink{sd}{SD},
\hyperlink{sunf}{SUNF}, \hyperlink{sunindex}{SUNIndex},
\hyperlink{sunsimplify}{SUNSimplify}, \hyperlink{trick}{Trick}.

\subsection{Examples}

\begin{Shaded}
\begin{Highlighting}[]
\NormalTok{SUNDelta}\OperatorTok{[}\NormalTok{SUNIndex}\OperatorTok{[}\FunctionTok{a}\OperatorTok{],}\NormalTok{ SUNIndex}\OperatorTok{[}\FunctionTok{b}\OperatorTok{]]}
\end{Highlighting}
\end{Shaded}

\begin{dmath*}\breakingcomma
\delta ^{ab}
\end{dmath*}

\begin{Shaded}
\begin{Highlighting}[]
\NormalTok{SUNDelta}\OperatorTok{[}\NormalTok{SUNIndex}\OperatorTok{[}\FunctionTok{a}\OperatorTok{],}\NormalTok{ SUNIndex}\OperatorTok{[}\FunctionTok{b}\OperatorTok{]]}\NormalTok{ SUNDelta}\OperatorTok{[}\NormalTok{SUNIndex}\OperatorTok{[}\FunctionTok{b}\OperatorTok{],}\NormalTok{ SUNIndex}\OperatorTok{[}\FunctionTok{c}\OperatorTok{]]} 
 
\NormalTok{SUNSimplify}\OperatorTok{[}\SpecialCharTok{\%}\OperatorTok{]}
\end{Highlighting}
\end{Shaded}

\begin{dmath*}\breakingcomma
\delta ^{ab} \delta ^{bc}
\end{dmath*}

\begin{dmath*}\breakingcomma
\delta ^{ac}
\end{dmath*}

\begin{Shaded}
\begin{Highlighting}[]
\NormalTok{SUNDelta}\OperatorTok{[}\NormalTok{SUNIndex}\OperatorTok{[}\FunctionTok{a}\OperatorTok{],}\NormalTok{ SUNIndex}\OperatorTok{[}\FunctionTok{b}\OperatorTok{]]} \SpecialCharTok{//} \FunctionTok{StandardForm}

\CommentTok{(*SUNDelta[SUNIndex[a], SUNIndex[b]]*)}
\end{Highlighting}
\end{Shaded}

\begin{Shaded}
\begin{Highlighting}[]
\NormalTok{SUNDelta}\OperatorTok{[}\NormalTok{SUNIndex}\OperatorTok{[}\FunctionTok{a}\OperatorTok{],}\NormalTok{ SUNIndex}\OperatorTok{[}\FunctionTok{b}\OperatorTok{]]} \SpecialCharTok{//}\NormalTok{ FCI }\SpecialCharTok{//}\NormalTok{ FCE }\SpecialCharTok{//} \FunctionTok{StandardForm}

\CommentTok{(*SD[a, b]*)}
\end{Highlighting}
\end{Shaded}

\begin{Shaded}
\begin{Highlighting}[]
\NormalTok{SD}\OperatorTok{[}\FunctionTok{a}\OperatorTok{,} \FunctionTok{b}\OperatorTok{]} \SpecialCharTok{//}\NormalTok{ FCI }\SpecialCharTok{//} \FunctionTok{StandardForm}

\CommentTok{(*SUNDelta[SUNIndex[a], SUNIndex[b]]*)}
\end{Highlighting}
\end{Shaded}

The arguments of \texttt{SUNDelta} may also represent explicit integer
indices via the head \texttt{ExplictiSUNIndex}. The difference is that
\texttt{SUNSimplify} will only sum over symbolic indices.

\begin{Shaded}
\begin{Highlighting}[]
\NormalTok{ex }\ExtensionTok{=}\NormalTok{ SUNDelta}\OperatorTok{[}\NormalTok{SUNIndex}\OperatorTok{[}\FunctionTok{a}\OperatorTok{],}\NormalTok{ ExplicitSUNIndex}\OperatorTok{[}\DecValTok{2}\OperatorTok{]]}\NormalTok{ SUNDelta}\OperatorTok{[}\NormalTok{SUNIndex}\OperatorTok{[}\FunctionTok{a}\OperatorTok{],}\NormalTok{ SUNIndex}\OperatorTok{[}\FunctionTok{b}\OperatorTok{]]}\NormalTok{ SUNDelta}\OperatorTok{[}\NormalTok{SUNIndex}\OperatorTok{[}\FunctionTok{c}\OperatorTok{],}\NormalTok{ ExplicitSUNIndex}\OperatorTok{[}\DecValTok{2}\OperatorTok{]]} \SpecialCharTok{//}\NormalTok{ SUNSimplify}
\end{Highlighting}
\end{Shaded}

\begin{dmath*}\breakingcomma
\delta ^{2b} \delta ^{2c}
\end{dmath*}

\begin{Shaded}
\begin{Highlighting}[]
\NormalTok{ex }\SpecialCharTok{//} \FunctionTok{StandardForm}

\CommentTok{(*SUNDelta[ExplicitSUNIndex[2], SUNIndex[b]] SUNDelta[ExplicitSUNIndex[2], SUNIndex[c]]*)}
\end{Highlighting}
\end{Shaded}

\begin{Shaded}
\begin{Highlighting}[]
\NormalTok{SD}\OperatorTok{[}\DecValTok{1}\OperatorTok{,} \DecValTok{2}\OperatorTok{]} \SpecialCharTok{//}\NormalTok{ FCI }\SpecialCharTok{//} \FunctionTok{StandardForm}

\CommentTok{(*SUNDelta[ExplicitSUNIndex[1], ExplicitSUNIndex[2]]*)}
\end{Highlighting}
\end{Shaded}

\end{document}
