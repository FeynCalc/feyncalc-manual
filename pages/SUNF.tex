% !TeX program = pdflatex
% !TeX root = SUNF.tex

\documentclass[../FeynCalcManual.tex]{subfiles}
\begin{document}
\hypertarget{sunf}{
\section{SUNF}\label{sunf}\index{SUNF}}

\texttt{SUNF[\allowbreak{}a,\ \allowbreak{}b,\ \allowbreak{}c]} are the
structure constants of \(SU(N)\). The arguments
\texttt{a,\ \allowbreak{}b,\ \allowbreak{}c} should be of symbolic type.

\subsection{See also}

\hyperlink{toc}{Overview}, \hyperlink{sund}{SUND},
\hyperlink{sundelta}{SUNDelta}, \hyperlink{sunindex}{SUNIndex},
\hyperlink{sunsimplify}{SUNSimplify}, \hyperlink{sunt}{SUNT},
\hyperlink{trick}{Trick}.

\subsection{Examples}

\begin{Shaded}
\begin{Highlighting}[]
\NormalTok{SUNF}\OperatorTok{[}\FunctionTok{a}\OperatorTok{,} \FunctionTok{b}\OperatorTok{,} \FunctionTok{c}\OperatorTok{]} \FunctionTok{x} \SpecialCharTok{+}\NormalTok{ SUNF}\OperatorTok{[}\FunctionTok{b}\OperatorTok{,} \FunctionTok{a}\OperatorTok{,} \FunctionTok{c}\OperatorTok{]} 
 
\NormalTok{Calc}\OperatorTok{[}\SpecialCharTok{\%}\OperatorTok{]} 
 
\NormalTok{SUNSimplify}\OperatorTok{[}\SpecialCharTok{\%\%}\OperatorTok{]}
\end{Highlighting}
\end{Shaded}

\begin{dmath*}\breakingcomma
x f^{abc}+f^{bac}
\end{dmath*}

\begin{dmath*}\breakingcomma
x f^{abc}-f^{abc}
\end{dmath*}

\begin{dmath*}\breakingcomma
(x-1) f^{abc}
\end{dmath*}

\begin{Shaded}
\begin{Highlighting}[]
\NormalTok{SUNF}\OperatorTok{[}\FunctionTok{a}\OperatorTok{,} \FunctionTok{a}\OperatorTok{,} \FunctionTok{b}\OperatorTok{]} 
 
\SpecialCharTok{\%} \SpecialCharTok{//}\NormalTok{ Calc}
\end{Highlighting}
\end{Shaded}

\begin{dmath*}\breakingcomma
f^{aab}
\end{dmath*}

\begin{dmath*}\breakingcomma
0
\end{dmath*}

This is a consequence of the usual choice for the normalization of the
\(T_a\) generators.

\begin{Shaded}
\begin{Highlighting}[]
\NormalTok{SUNF}\OperatorTok{[}\FunctionTok{a}\OperatorTok{,} \FunctionTok{b}\OperatorTok{,} \FunctionTok{c}\OperatorTok{,}\NormalTok{ Explicit }\OtherTok{{-}\textgreater{}} \ConstantTok{True}\OperatorTok{]}
\end{Highlighting}
\end{Shaded}

\begin{dmath*}\breakingcomma
2 i \left(\text{tr}(T^a.T^c.T^b)-\text{tr}(T^a.T^b.T^c)\right)
\end{dmath*}

\begin{Shaded}
\begin{Highlighting}[]
\NormalTok{SUNSimplify}\OperatorTok{[}\NormalTok{SUNF}\OperatorTok{[}\FunctionTok{a}\OperatorTok{,} \FunctionTok{b}\OperatorTok{,} \FunctionTok{c}\OperatorTok{]}\NormalTok{ SUNF}\OperatorTok{[}\FunctionTok{a}\OperatorTok{,} \FunctionTok{b}\OperatorTok{,} \FunctionTok{d}\OperatorTok{]]}
\end{Highlighting}
\end{Shaded}

\begin{dmath*}\breakingcomma
C_A \delta ^{cd}
\end{dmath*}

\begin{Shaded}
\begin{Highlighting}[]
\NormalTok{SUNSimplify}\OperatorTok{[}\NormalTok{SUNF}\OperatorTok{[}\FunctionTok{a}\OperatorTok{,} \FunctionTok{b}\OperatorTok{,} \FunctionTok{c}\OperatorTok{],}\NormalTok{ Explicit }\OtherTok{{-}\textgreater{}} \ConstantTok{True}\OperatorTok{]}
\end{Highlighting}
\end{Shaded}

\begin{dmath*}\breakingcomma
-2 i \left(\text{tr}(T^a.T^b.T^c)-\text{tr}(T^b.T^a.T^c)\right)
\end{dmath*}

\begin{Shaded}
\begin{Highlighting}[]
\NormalTok{SUNF}\OperatorTok{[}\FunctionTok{a}\OperatorTok{,} \FunctionTok{b}\OperatorTok{,} \FunctionTok{c}\OperatorTok{]} \SpecialCharTok{//} \FunctionTok{StandardForm}

\CommentTok{(*SUNF[a, b, c]*)}
\end{Highlighting}
\end{Shaded}

\begin{Shaded}
\begin{Highlighting}[]
\NormalTok{SUNF}\OperatorTok{[}\FunctionTok{a}\OperatorTok{,} \FunctionTok{b}\OperatorTok{,} \FunctionTok{c}\OperatorTok{]} \SpecialCharTok{//}\NormalTok{ FCI }\SpecialCharTok{//} \FunctionTok{StandardForm}

\CommentTok{(*SUNF[SUNIndex[a], SUNIndex[b], SUNIndex[c]]*)}
\end{Highlighting}
\end{Shaded}

\begin{Shaded}
\begin{Highlighting}[]
\NormalTok{SUNF}\OperatorTok{[}\FunctionTok{a}\OperatorTok{,} \FunctionTok{b}\OperatorTok{,} \FunctionTok{c}\OperatorTok{]} \SpecialCharTok{//}\NormalTok{ FCI }\SpecialCharTok{//}\NormalTok{ FCE }\SpecialCharTok{//} \FunctionTok{StandardForm}

\CommentTok{(*SUNF[a, b, c]*)}
\end{Highlighting}
\end{Shaded}

\begin{Shaded}
\begin{Highlighting}[]
\NormalTok{SUNF}\OperatorTok{[}\FunctionTok{b}\OperatorTok{,} \FunctionTok{a}\OperatorTok{,} \FunctionTok{c}\OperatorTok{]} 
 
\SpecialCharTok{\%} \SpecialCharTok{//}\NormalTok{ FCI}
\end{Highlighting}
\end{Shaded}

\begin{dmath*}\breakingcomma
f^{bac}
\end{dmath*}

\begin{dmath*}\breakingcomma
-f^{abc}
\end{dmath*}
\end{document}
