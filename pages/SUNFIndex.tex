% !TeX program = pdflatex
% !TeX root = SUNFIndex.tex

\documentclass[../FeynCalcManual.tex]{subfiles}
\begin{document}
\hypertarget{sunfindex}{
\section{SUNFIndex}\label{sunfindex}\index{SUNFIndex}}

\texttt{SUNFIndex[\allowbreak{}a]} is an \(SU(N)\) index in the
fundamental representation. If the argument is an integer,
\texttt{SUNFIndex[\allowbreak{}a]} turns into
\texttt{ExplicitSUNFIndex[\allowbreak{}a]}.

\subsection{See also}

\hyperlink{toc}{Overview}, \hyperlink{sunindex}{SUNIndex}.

\subsection{Examples}

\begin{Shaded}
\begin{Highlighting}[]
\NormalTok{SUNFIndex}\OperatorTok{[}\FunctionTok{i}\OperatorTok{]}
\end{Highlighting}
\end{Shaded}

\begin{dmath*}\breakingcomma
i
\end{dmath*}

\begin{Shaded}
\begin{Highlighting}[]
\NormalTok{SUNFIndex}\OperatorTok{[}\FunctionTok{i}\OperatorTok{]} \SpecialCharTok{//} \FunctionTok{StandardForm}

\CommentTok{(*SUNFIndex[i]*)}
\end{Highlighting}
\end{Shaded}

\begin{Shaded}
\begin{Highlighting}[]
\NormalTok{SUNFIndex}\OperatorTok{[}\DecValTok{2}\OperatorTok{]}
\end{Highlighting}
\end{Shaded}

\begin{dmath*}\breakingcomma
2
\end{dmath*}

\begin{Shaded}
\begin{Highlighting}[]
\NormalTok{SUNFIndex}\OperatorTok{[}\DecValTok{2}\OperatorTok{]} \SpecialCharTok{//} \FunctionTok{StandardForm}

\CommentTok{(*ExplicitSUNFIndex[2]*)}
\end{Highlighting}
\end{Shaded}

\begin{Shaded}
\begin{Highlighting}[]
\NormalTok{SUNFDelta}\OperatorTok{[}\FunctionTok{i}\OperatorTok{,} \FunctionTok{j}\OperatorTok{]} \SpecialCharTok{//}\NormalTok{ FCI }\SpecialCharTok{//} \FunctionTok{StandardForm}

\CommentTok{(*SUNFDelta[SUNFIndex[i], SUNFIndex[j]]*)}
\end{Highlighting}
\end{Shaded}

\end{document}
