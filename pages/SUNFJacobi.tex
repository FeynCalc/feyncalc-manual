% !TeX program = pdflatex
% !TeX root = SUNFJacobi.tex

\documentclass[../FeynCalcManual.tex]{subfiles}
\begin{document}
\hypertarget{sunfjacobi}{%
\section{SUNFJacobi}\label{sunfjacobi}}

\texttt{SUNFJacobi} is an option for \texttt{SUNSimplify}, indicating
whether the Jacobi identity should be used.

\subsection{See also}

\hyperlink{toc}{Overview}, \hyperlink{sunf}{SUNF},
\hyperlink{sunsimplify}{SUNSimplify}.

\subsection{Examples}

\begin{Shaded}
\begin{Highlighting}[]
\NormalTok{SUNF}\OperatorTok{[}\FunctionTok{a}\OperatorTok{,} \FunctionTok{b}\OperatorTok{,} \FunctionTok{c}\OperatorTok{]}\NormalTok{ SUNF}\OperatorTok{[}\FunctionTok{e}\OperatorTok{,} \FunctionTok{f}\OperatorTok{,} \FunctionTok{c}\OperatorTok{]} \SpecialCharTok{//}\NormalTok{ SUNSimplify}\OperatorTok{[}\NormalTok{\#}\OperatorTok{,}\NormalTok{ SUNFJacobi }\OtherTok{{-}\textgreater{}} \ConstantTok{False}\OperatorTok{]}\NormalTok{ \&}
\end{Highlighting}
\end{Shaded}

\begin{dmath*}\breakingcomma
f^{cef} f^{abc}
\end{dmath*}

\begin{Shaded}
\begin{Highlighting}[]
\NormalTok{SUNF}\OperatorTok{[}\FunctionTok{a}\OperatorTok{,} \FunctionTok{b}\OperatorTok{,} \FunctionTok{c}\OperatorTok{]}\NormalTok{ SUNF}\OperatorTok{[}\FunctionTok{e}\OperatorTok{,} \FunctionTok{f}\OperatorTok{,} \FunctionTok{c}\OperatorTok{]} \SpecialCharTok{//}\NormalTok{ SUNSimplify}\OperatorTok{[}\NormalTok{\#}\OperatorTok{,}\NormalTok{ SUNFJacobi }\OtherTok{{-}\textgreater{}} \ConstantTok{True}\OperatorTok{]}\NormalTok{ \&}
\end{Highlighting}
\end{Shaded}

\begin{dmath*}\breakingcomma
f^{bcf} f^{ace}-f^{acf} f^{bce}
\end{dmath*}
\end{document}
