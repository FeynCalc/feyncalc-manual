% !TeX program = pdflatex
% !TeX root = SUNIndex.tex

\documentclass[../FeynCalcManual.tex]{subfiles}
\begin{document}
\hypertarget{sunindex}{%
\section{SUNIndex}\label{sunindex}}

\texttt{SUNIndex[\allowbreak{}a]} is an \(SU(N)\) index in the adjoint
representation. If the argument is an integer,
\texttt{SUNIndex[\allowbreak{}a]} turns into
\texttt{ExplicitSUNIndex[\allowbreak{}a]}.

\subsection{See also}

\hyperlink{toc}{Overview},
\hyperlink{explicitsunindex}{ExplicitSUNIndex},
\hyperlink{sundelta}{SUNDelta}, \hyperlink{sunf}{SUNF}.

\subsection{Examples}

\begin{Shaded}
\begin{Highlighting}[]
\NormalTok{SUNIndex}\OperatorTok{[}\FunctionTok{i}\OperatorTok{]}
\end{Highlighting}
\end{Shaded}

\begin{dmath*}\breakingcomma
i
\end{dmath*}

\begin{Shaded}
\begin{Highlighting}[]
\NormalTok{SUNIndex}\OperatorTok{[}\FunctionTok{i}\OperatorTok{]} \SpecialCharTok{//} \FunctionTok{StandardForm}

\CommentTok{(*SUNIndex[i]*)}
\end{Highlighting}
\end{Shaded}

\begin{Shaded}
\begin{Highlighting}[]
\NormalTok{SUNIndex}\OperatorTok{[}\DecValTok{2}\OperatorTok{]}
\end{Highlighting}
\end{Shaded}

\begin{dmath*}\breakingcomma
2
\end{dmath*}

\begin{Shaded}
\begin{Highlighting}[]
\NormalTok{SUNIndex}\OperatorTok{[}\DecValTok{2}\OperatorTok{]} \SpecialCharTok{//} \FunctionTok{StandardForm}

\CommentTok{(*ExplicitSUNIndex[2]*)}
\end{Highlighting}
\end{Shaded}

\begin{Shaded}
\begin{Highlighting}[]
\NormalTok{SUNDelta}\OperatorTok{[}\FunctionTok{i}\OperatorTok{,} \FunctionTok{j}\OperatorTok{]} \SpecialCharTok{//}\NormalTok{ FCI }\SpecialCharTok{//} \FunctionTok{StandardForm}

\CommentTok{(*SUNDelta[SUNIndex[i], SUNIndex[j]]*)}
\end{Highlighting}
\end{Shaded}

\end{document}
