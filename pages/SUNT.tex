% !TeX program = pdflatex
% !TeX root = SUNT.tex

\documentclass[../FeynCalcManual.tex]{subfiles}
\begin{document}
\hypertarget{sunt}{
\section{SUNT}\label{sunt}\index{SUNT}}

\texttt{SUNT[\allowbreak{}a]} is the \(SU(N)\) \(T^a\) generator in the
fundamental representation. The fundamental indices are implicit.

\subsection{See also}

\hyperlink{toc}{Overview}, \hyperlink{ca}{CA}, \hyperlink{cf}{CF},
\hyperlink{sund}{SUND}, \hyperlink{sundelta}{SUNDelta},
\hyperlink{sunf}{SUNF}, \hyperlink{sunsimplify}{SUNSimplify}.

\subsection{Examples}

\begin{Shaded}
\begin{Highlighting}[]
\NormalTok{SUNT}\OperatorTok{[}\FunctionTok{a}\OperatorTok{]}
\end{Highlighting}
\end{Shaded}

\begin{dmath*}\breakingcomma
T^a
\end{dmath*}

Since \(T^a\) is a noncommutative object, products have to separated by
a \texttt{Dot} (\texttt{.}).

\begin{Shaded}
\begin{Highlighting}[]
\NormalTok{SUNT}\OperatorTok{[}\FunctionTok{a}\OperatorTok{]}\NormalTok{ . SUNT}\OperatorTok{[}\FunctionTok{b}\OperatorTok{]}\NormalTok{ . SUNT}\OperatorTok{[}\FunctionTok{c}\OperatorTok{]}
\end{Highlighting}
\end{Shaded}

\begin{dmath*}\breakingcomma
T^a.T^b.T^c
\end{dmath*}

\begin{Shaded}
\begin{Highlighting}[]
\NormalTok{SUNT}\OperatorTok{[}\FunctionTok{a}\OperatorTok{,} \FunctionTok{b}\OperatorTok{,} \FunctionTok{c}\OperatorTok{,} \FunctionTok{d}\OperatorTok{]}
\end{Highlighting}
\end{Shaded}

\begin{dmath*}\breakingcomma
T^a.T^b.T^c.T^d
\end{dmath*}

\begin{Shaded}
\begin{Highlighting}[]
\NormalTok{SUNSimplify}\OperatorTok{[}\NormalTok{SUNT}\OperatorTok{[}\FunctionTok{a}\OperatorTok{,} \FunctionTok{b}\OperatorTok{,} \FunctionTok{a}\OperatorTok{],}\NormalTok{ SUNNToCACF }\OtherTok{{-}\textgreater{}} \ConstantTok{False}\OperatorTok{]}
\end{Highlighting}
\end{Shaded}

\begin{dmath*}\breakingcomma
-\frac{T^b}{2 N}
\end{dmath*}

\begin{Shaded}
\begin{Highlighting}[]
\NormalTok{SUNSimplify}\OperatorTok{[}\NormalTok{SUNT}\OperatorTok{[}\FunctionTok{a}\OperatorTok{,} \FunctionTok{b}\OperatorTok{,} \FunctionTok{b}\OperatorTok{,} \FunctionTok{a}\OperatorTok{]]}
\end{Highlighting}
\end{Shaded}

\begin{dmath*}\breakingcomma
C_F^2
\end{dmath*}

\begin{Shaded}
\begin{Highlighting}[]
\NormalTok{SUNSimplify}\OperatorTok{[}\NormalTok{SUNT}\OperatorTok{[}\FunctionTok{a}\OperatorTok{,} \FunctionTok{b}\OperatorTok{,} \FunctionTok{a}\OperatorTok{]]}
\end{Highlighting}
\end{Shaded}

\begin{dmath*}\breakingcomma
-\frac{1}{2} T^b \left(C_A-2 C_F\right)
\end{dmath*}

\begin{Shaded}
\begin{Highlighting}[]
\NormalTok{SUNSimplify}\OperatorTok{[}\NormalTok{SUNT}\OperatorTok{[}\FunctionTok{a}\OperatorTok{,} \FunctionTok{b}\OperatorTok{,} \FunctionTok{a}\OperatorTok{],}\NormalTok{ SUNNToCACF }\OtherTok{{-}\textgreater{}} \ConstantTok{False}\OperatorTok{]}
\end{Highlighting}
\end{Shaded}

\begin{dmath*}\breakingcomma
-\frac{T^b}{2 N}
\end{dmath*}

The normalization of the generators is chosen in the standard way,
therefore \(\textrm{Tr}(T^aT^b) = \frac{1}{2} \delta _{ab}\)

\begin{Shaded}
\begin{Highlighting}[]
\NormalTok{SUNTrace}\OperatorTok{[}\NormalTok{SUNT}\OperatorTok{[}\FunctionTok{a}\OperatorTok{,} \FunctionTok{b}\OperatorTok{]]}
\end{Highlighting}
\end{Shaded}

\begin{dmath*}\breakingcomma
\frac{\delta ^{ab}}{2}
\end{dmath*}

In case you want \(T_f\), you need to include a factor
\texttt{2*Tf}inside the trace.

\begin{Shaded}
\begin{Highlighting}[]
\NormalTok{SUNTrace}\OperatorTok{[}\DecValTok{2}\NormalTok{ Tf SUNT}\OperatorTok{[}\FunctionTok{a}\OperatorTok{,} \FunctionTok{b}\OperatorTok{]]}
\end{Highlighting}
\end{Shaded}

\begin{dmath*}\breakingcomma
T_f \delta ^{ab}
\end{dmath*}

\begin{Shaded}
\begin{Highlighting}[]
\NormalTok{SUNTrace}\OperatorTok{[}\NormalTok{SUNT}\OperatorTok{[}\FunctionTok{a}\OperatorTok{,} \FunctionTok{b}\OperatorTok{]]} \SpecialCharTok{//} \FunctionTok{StandardForm}
\end{Highlighting}
\end{Shaded}

\begin{dmath*}\breakingcomma
\frac{1}{2} \;\text{SUNDelta}[\text{SUNIndex}[a],\text{SUNIndex}[b]]
\end{dmath*}

\begin{Shaded}
\begin{Highlighting}[]
\NormalTok{SUNT}\OperatorTok{[}\FunctionTok{a}\OperatorTok{]} \SpecialCharTok{//}\NormalTok{ FCI }\SpecialCharTok{//} \FunctionTok{StandardForm}

\CommentTok{(*SUNT[SUNIndex[a]]*)}
\end{Highlighting}
\end{Shaded}

\begin{Shaded}
\begin{Highlighting}[]
\NormalTok{SUNT}\OperatorTok{[}\FunctionTok{a}\OperatorTok{]} \SpecialCharTok{//}\NormalTok{ FCI }\SpecialCharTok{//}\NormalTok{ FCE }\SpecialCharTok{//} \FunctionTok{StandardForm}

\CommentTok{(*SUNT[a]*)}
\end{Highlighting}
\end{Shaded}

\end{document}
