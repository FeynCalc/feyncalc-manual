% !TeX program = pdflatex
% !TeX root = SUNTF.tex

\documentclass[../FeynCalcManual.tex]{subfiles}
\begin{document}
\hypertarget{suntf}{%
\section{SUNTF}\label{suntf}}

\texttt{SUNTF[\allowbreak{}\{\allowbreak{}a\},\ \allowbreak{}i,\ \allowbreak{}j]}
is the \(SU(N)\) \(T^a\) generator in the fundamental representation.
The fundamental indices are explicit.

\subsection{See also}

\hyperlink{toc}{Overview}, \hyperlink{sunfindex}{SUNFindex},
\hyperlink{sunt}{SUNT}, \hyperlink{sunsimplify}{SUNSimplify}.

\subsection{Examples}

\begin{Shaded}
\begin{Highlighting}[]
\NormalTok{SUNTF}\OperatorTok{[}\FunctionTok{a}\OperatorTok{,} \FunctionTok{i}\OperatorTok{,} \FunctionTok{j}\OperatorTok{]}
\end{Highlighting}
\end{Shaded}

\begin{dmath*}\breakingcomma
T_{ij}^a
\end{dmath*}

\begin{Shaded}
\begin{Highlighting}[]
\NormalTok{SUNTF}\OperatorTok{[\{}\FunctionTok{a}\OperatorTok{,} \FunctionTok{b}\OperatorTok{\},} \FunctionTok{i}\OperatorTok{,} \FunctionTok{j}\OperatorTok{]}
\end{Highlighting}
\end{Shaded}

\begin{dmath*}\breakingcomma
\left(T^aT^b\right){}_{ij}
\end{dmath*}

\texttt{SUNTF} are \(c\)-numbers, hence they are commutative objects and
do not require a dot

\begin{Shaded}
\begin{Highlighting}[]
\NormalTok{SUNTF}\OperatorTok{[\{}\FunctionTok{a}\OperatorTok{,} \FunctionTok{b}\OperatorTok{\},} \FunctionTok{i}\OperatorTok{,} \FunctionTok{j}\OperatorTok{]}\NormalTok{ SUNTF}\OperatorTok{[\{}\FunctionTok{c}\OperatorTok{,} \FunctionTok{d}\OperatorTok{\},} \FunctionTok{j}\OperatorTok{,} \FunctionTok{k}\OperatorTok{]}
\end{Highlighting}
\end{Shaded}

\begin{dmath*}\breakingcomma
\left(T^aT^b\right){}_{ij} \left(T^cT^d\right){}_{jk}
\end{dmath*}

\begin{Shaded}
\begin{Highlighting}[]
\NormalTok{SUNTF}\OperatorTok{[\{}\FunctionTok{a}\OperatorTok{,} \FunctionTok{b}\OperatorTok{\},} \FunctionTok{i}\OperatorTok{,} \FunctionTok{j}\OperatorTok{]}\NormalTok{ SUNTF}\OperatorTok{[\{}\FunctionTok{c}\OperatorTok{,} \FunctionTok{d}\OperatorTok{\},} \FunctionTok{j}\OperatorTok{,} \FunctionTok{k}\OperatorTok{]} \SpecialCharTok{//}\NormalTok{ SUNFSimplify}
\end{Highlighting}
\end{Shaded}

\begin{dmath*}\breakingcomma
\left(T^aT^bT^cT^d\right){}_{ik}
\end{dmath*}

A chain with closed indices is automatically converted into a trace

\begin{Shaded}
\begin{Highlighting}[]
\NormalTok{SUNTF}\OperatorTok{[\{}\FunctionTok{a}\OperatorTok{,} \FunctionTok{b}\OperatorTok{\},} \FunctionTok{i}\OperatorTok{,} \FunctionTok{j}\OperatorTok{]}\NormalTok{ SUNTF}\OperatorTok{[\{}\FunctionTok{c}\OperatorTok{,} \FunctionTok{d}\OperatorTok{\},} \FunctionTok{j}\OperatorTok{,} \FunctionTok{i}\OperatorTok{]} \SpecialCharTok{//}\NormalTok{ SUNFSimplify}
\end{Highlighting}
\end{Shaded}

\begin{dmath*}\breakingcomma
\text{tr}(T^a.T^b.T^c.T^d)
\end{dmath*}

\texttt{SUNFSimplify} is a dedicated function to deal with
\texttt{SUNTF}s. However, \texttt{SUNSimplify} will also call
\texttt{SUNFSimplify} when it detects \texttt{SUNTF}objects in the input

\begin{Shaded}
\begin{Highlighting}[]
\NormalTok{SUNDelta}\OperatorTok{[}\FunctionTok{a}\OperatorTok{,} \FunctionTok{b}\OperatorTok{]}\NormalTok{ SUNTF}\OperatorTok{[\{}\FunctionTok{a}\OperatorTok{,} \FunctionTok{b}\OperatorTok{\},} \FunctionTok{i}\OperatorTok{,} \FunctionTok{j}\OperatorTok{]}\NormalTok{ SUNTF}\OperatorTok{[\{}\FunctionTok{c}\OperatorTok{,} \FunctionTok{d}\OperatorTok{\},} \FunctionTok{j}\OperatorTok{,} \FunctionTok{i}\OperatorTok{]} \SpecialCharTok{//}\NormalTok{ SUNSimplify}
\end{Highlighting}
\end{Shaded}

\begin{dmath*}\breakingcomma
\frac{1}{2} C_F \delta ^{cd}
\end{dmath*}

\begin{Shaded}
\begin{Highlighting}[]
\NormalTok{SUNTF}\OperatorTok{[\{}\FunctionTok{a}\OperatorTok{,} \FunctionTok{b}\OperatorTok{\},} \FunctionTok{i}\OperatorTok{,} \FunctionTok{j}\OperatorTok{]} \SpecialCharTok{//}\NormalTok{ FCI }\SpecialCharTok{//} \FunctionTok{StandardForm}

\CommentTok{(*SUNTF[\{SUNIndex[a], SUNIndex[b]\}, SUNFIndex[i], SUNFIndex[j]]*)}
\end{Highlighting}
\end{Shaded}

\end{document}
