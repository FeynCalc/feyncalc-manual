% !TeX program = pdflatex
% !TeX root = ScalarGluonVertex.tex

\documentclass[../FeynCalcManual.tex]{subfiles}
\begin{document}
\hypertarget{scalargluonvertex}{%
\section{ScalarGluonVertex}\label{scalargluonvertex}}

\texttt{ScalarGluonVertex[\allowbreak{}\{\allowbreak{}p\},\ \allowbreak{}\{\allowbreak{}q\},\ \allowbreak{}\{\allowbreak{}mu,\ \allowbreak{}a\}]}
or
\texttt{ScalarGluonVertex[\allowbreak{}p,\ \allowbreak{} q,\ \allowbreak{} mu,\ \allowbreak{}a]}
yields the scalar-scalar-gluon vertex, where \texttt{p} and \texttt{q}
are incoming momenta.

\texttt{ScalarGluonVertex[\allowbreak{}\{\allowbreak{}mu,\ \allowbreak{}a\},\ \allowbreak{}\{\allowbreak{}nu,\ \allowbreak{}b\}]}
yields the scalar-scalar-gluon-gluon vertex, where \texttt{p} and
\texttt{q} are incoming momenta.

The dimension and the name of the coupling constant are determined by
the options \texttt{Dimension} and \texttt{CouplingConstant}.

\subsection{See also}

\hyperlink{toc}{Overview}

\subsection{Examples}

\begin{Shaded}
\begin{Highlighting}[]
\NormalTok{ScalarGluonVertex}\OperatorTok{[\{}\FunctionTok{p}\OperatorTok{\},} \OperatorTok{\{}\FunctionTok{q}\OperatorTok{\},} \OperatorTok{\{}\SpecialCharTok{\textbackslash{}}\OperatorTok{[}\NormalTok{Mu}\OperatorTok{],} \FunctionTok{a}\OperatorTok{\}]}
\end{Highlighting}
\end{Shaded}

\begin{dmath*}\breakingcomma
i T^a g_s (p-q)^{\mu }
\end{dmath*}
\end{document}
