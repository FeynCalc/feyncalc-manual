% !TeX program = pdflatex
% !TeX root = ScalarProduct.tex

\documentclass[../FeynCalcManual.tex]{subfiles}
\begin{document}
\hypertarget{scalarproduct}{
\section{ScalarProduct}\label{scalarproduct}\index{ScalarProduct}}

\texttt{ScalarProduct[\allowbreak{}p,\ \allowbreak{}q]} is the input for
the scalar product of two Lorentz vectors p and q.

ScalarProduct{[}p{]} is equivalent to ScalarProduct{[}p, p{]}.

Expansion of sums of momenta in \texttt{ScalarProduct} is done with
\texttt{ExpandScalarProduct}.

Scalar products may be set, e.g.~via
\texttt{ScalarProduct[\allowbreak{}a,\ \allowbreak{}b] = m^2}; but
\texttt{a} and \texttt{b} may not contain sums.

\texttt{ScalarProduct[\allowbreak{}a]} corresponds to
\texttt{ScalarProduct[\allowbreak{}a,\ \allowbreak{}a]}

Note that \texttt{ScalarProduct[\allowbreak{}a,\ \allowbreak{}b] = m^2}
actually sets Lorentzian scalar products in different dimensions
specified by the value of the \texttt{SetDimensions} option.

It is highly recommended to set \texttt{ScalarProduct}s before any
calculation. This improves the performance of FeynCalc.

\subsection{See also}

\hyperlink{toc}{Overview}, \hyperlink{calc}{Calc},
\hyperlink{fcclearscalarproducts}{FCClearScalarProducts},
\hyperlink{expandscalarproduct}{ExpandScalarProduct},
\hyperlink{scalarproductcancel}{ScalarProductCancel},
\hyperlink{pair}{Pair}, \hyperlink{sp}{SP}, \hyperlink{spd}{SPD}.

\subsection{Examples}

\begin{Shaded}
\begin{Highlighting}[]
\NormalTok{ScalarProduct}\OperatorTok{[}\FunctionTok{p}\OperatorTok{,} \FunctionTok{q}\OperatorTok{]}
\end{Highlighting}
\end{Shaded}

\begin{dmath*}\breakingcomma
\overline{p}\cdot \overline{q}
\end{dmath*}

\begin{Shaded}
\begin{Highlighting}[]
\NormalTok{ScalarProduct}\OperatorTok{[}\FunctionTok{p} \SpecialCharTok{+} \FunctionTok{q}\OperatorTok{,} \SpecialCharTok{{-}}\FunctionTok{q}\OperatorTok{]}
\end{Highlighting}
\end{Shaded}

\begin{dmath*}\breakingcomma
-\left(\overline{q}\cdot (\overline{p}+\overline{q})\right)
\end{dmath*}

\begin{Shaded}
\begin{Highlighting}[]
\NormalTok{ScalarProduct}\OperatorTok{[}\FunctionTok{p}\OperatorTok{,} \FunctionTok{p}\OperatorTok{]}
\end{Highlighting}
\end{Shaded}

\begin{dmath*}\breakingcomma
\overline{p}^2
\end{dmath*}

\begin{Shaded}
\begin{Highlighting}[]
\NormalTok{ScalarProduct}\OperatorTok{[}\FunctionTok{q}\OperatorTok{]}
\end{Highlighting}
\end{Shaded}

\begin{dmath*}\breakingcomma
\overline{q}^2
\end{dmath*}

\begin{Shaded}
\begin{Highlighting}[]
\NormalTok{ScalarProduct}\OperatorTok{[}\FunctionTok{p}\OperatorTok{,} \FunctionTok{q}\OperatorTok{]} \SpecialCharTok{//} \FunctionTok{StandardForm}

\CommentTok{(*Pair[Momentum[p], Momentum[q]]*)}
\end{Highlighting}
\end{Shaded}

\begin{Shaded}
\begin{Highlighting}[]
\NormalTok{ScalarProduct}\OperatorTok{[}\FunctionTok{p}\OperatorTok{,} \FunctionTok{q}\OperatorTok{,}\NormalTok{ Dimension }\OtherTok{{-}\textgreater{}} \FunctionTok{D}\OperatorTok{]} \SpecialCharTok{//} \FunctionTok{StandardForm}

\CommentTok{(*Pair[Momentum[p, D], Momentum[q, D]]*)}
\end{Highlighting}
\end{Shaded}

\begin{Shaded}
\begin{Highlighting}[]
\NormalTok{ScalarProduct}\OperatorTok{[}\FunctionTok{Subscript}\OperatorTok{[}\FunctionTok{p}\OperatorTok{,} \DecValTok{1}\OperatorTok{],} \FunctionTok{Subscript}\OperatorTok{[}\FunctionTok{p}\OperatorTok{,} \DecValTok{2}\OperatorTok{]]} \ExtensionTok{=} \FunctionTok{s}\SpecialCharTok{/}\DecValTok{2}
\end{Highlighting}
\end{Shaded}

\begin{dmath*}\breakingcomma
\frac{s}{2}
\end{dmath*}

\begin{Shaded}
\begin{Highlighting}[]
\NormalTok{ExpandScalarProduct}\OperatorTok{[}\NormalTok{ ScalarProduct}\OperatorTok{[}\FunctionTok{Subscript}\OperatorTok{[}\FunctionTok{p}\OperatorTok{,} \DecValTok{1}\OperatorTok{]} \SpecialCharTok{{-}} \FunctionTok{q}\OperatorTok{,} \FunctionTok{Subscript}\OperatorTok{[}\FunctionTok{p}\OperatorTok{,} \DecValTok{2}\OperatorTok{]} \SpecialCharTok{{-}} \FunctionTok{k}\OperatorTok{]]}
\end{Highlighting}
\end{Shaded}

\begin{dmath*}\breakingcomma
-\overline{k}\cdot \overline{p}_1+\overline{k}\cdot \overline{q}-\overline{q}\cdot \overline{p}_2+\frac{s}{2}
\end{dmath*}

\begin{Shaded}
\begin{Highlighting}[]
\NormalTok{Calc}\OperatorTok{[}\NormalTok{ ScalarProduct}\OperatorTok{[}\FunctionTok{Subscript}\OperatorTok{[}\FunctionTok{p}\OperatorTok{,} \DecValTok{1}\OperatorTok{]} \SpecialCharTok{{-}} \FunctionTok{q}\OperatorTok{,} \FunctionTok{Subscript}\OperatorTok{[}\FunctionTok{p}\OperatorTok{,} \DecValTok{2}\OperatorTok{]} \SpecialCharTok{{-}} \FunctionTok{k}\OperatorTok{]]}
\end{Highlighting}
\end{Shaded}

\begin{dmath*}\breakingcomma
-\overline{k}\cdot \overline{p}_1+\overline{k}\cdot \overline{q}-\overline{q}\cdot \overline{p}_2+\frac{s}{2}
\end{dmath*}

\begin{Shaded}
\begin{Highlighting}[]
\NormalTok{ScalarProduct}\OperatorTok{[}\NormalTok{q1}\OperatorTok{]} \ExtensionTok{=}\NormalTok{ qq;}
\end{Highlighting}
\end{Shaded}

\begin{Shaded}
\begin{Highlighting}[]
\NormalTok{SP}\OperatorTok{[}\NormalTok{q1}\OperatorTok{]}
\end{Highlighting}
\end{Shaded}

\begin{dmath*}\breakingcomma
\text{qq}
\end{dmath*}

\begin{Shaded}
\begin{Highlighting}[]
\NormalTok{FCClearScalarProducts}\OperatorTok{[]}
\end{Highlighting}
\end{Shaded}

\end{document}
