% !TeX program = pdflatex
% !TeX root = Schouten.tex

\documentclass[../FeynCalcManual.tex]{subfiles}
\begin{document}
\hypertarget{schouten}{%
\section{Schouten}\label{schouten}}

\texttt{Schouten[\allowbreak{}exp]} attempts to automatically remove
spurious terms in \texttt{exp} by applying the Schouten's identity.

\texttt{Schouten} applies the identity for \(4\)-vectors on at most
\(42\) terms in a sum. If it should operate on a larger expression you
can give a second argument,
e.g.~\texttt{Schouten[\allowbreak{}expr,\ \allowbreak{}4711]} which will
work on sums with less than \(4711\) terms.

\texttt{Schouten} is also an option of \texttt{Contract} and
\texttt{DiracTrace}. It may be set to an integer indicating the maximum
number of terms onto which the function \texttt{Schouten} will be
applied.

\subsection{See also}

\hyperlink{toc}{Overview}, \hyperlink{contract}{Contract},
\hyperlink{diractrace}{DiracTrace},
\hyperlink{fcschoutenbruteforce}{FCSchoutenBruteForce}.

\subsection{Examples}

\begin{Shaded}
\begin{Highlighting}[]
\NormalTok{((LC}\OperatorTok{[}\SpecialCharTok{\textbackslash{}}\OperatorTok{[}\NormalTok{Mu}\OperatorTok{],} \SpecialCharTok{\textbackslash{}}\OperatorTok{[}\NormalTok{Nu}\OperatorTok{],} \SpecialCharTok{\textbackslash{}}\OperatorTok{[}\NormalTok{Rho}\OperatorTok{],} \SpecialCharTok{\textbackslash{}}\OperatorTok{[}\NormalTok{Sigma}\OperatorTok{]]}\NormalTok{ FV}\OperatorTok{[}\FunctionTok{p}\OperatorTok{,} \SpecialCharTok{\textbackslash{}}\OperatorTok{[}\NormalTok{Tau}\OperatorTok{]]} \SpecialCharTok{+}\NormalTok{ LC}\OperatorTok{[}\SpecialCharTok{\textbackslash{}}\OperatorTok{[}\NormalTok{Nu}\OperatorTok{],} \SpecialCharTok{\textbackslash{}}\OperatorTok{[}\NormalTok{Rho}\OperatorTok{],} \SpecialCharTok{\textbackslash{}}\OperatorTok{[}\NormalTok{Sigma}\OperatorTok{],} \SpecialCharTok{\textbackslash{}}\OperatorTok{[}\NormalTok{Tau}\OperatorTok{]]}\NormalTok{ FV}\OperatorTok{[}\FunctionTok{p}\OperatorTok{,} \SpecialCharTok{\textbackslash{}}\OperatorTok{[}\NormalTok{Mu}\OperatorTok{]]} \SpecialCharTok{+}\NormalTok{ LC}\OperatorTok{[}\SpecialCharTok{\textbackslash{}}\OperatorTok{[}\NormalTok{Rho}\OperatorTok{],} \SpecialCharTok{\textbackslash{}}\OperatorTok{[}\NormalTok{Sigma}\OperatorTok{],} \SpecialCharTok{\textbackslash{}}\OperatorTok{[}\NormalTok{Tau}\OperatorTok{],} \SpecialCharTok{\textbackslash{}}\OperatorTok{[}\NormalTok{Mu}\OperatorTok{]]}\NormalTok{ FV}\OperatorTok{[}\FunctionTok{p}\OperatorTok{,} \SpecialCharTok{\textbackslash{}}\OperatorTok{[}\NormalTok{Nu}\OperatorTok{]]} \SpecialCharTok{+} 
\NormalTok{     LC}\OperatorTok{[}\SpecialCharTok{\textbackslash{}}\OperatorTok{[}\NormalTok{Sigma}\OperatorTok{],} \SpecialCharTok{\textbackslash{}}\OperatorTok{[}\NormalTok{Tau}\OperatorTok{],} \SpecialCharTok{\textbackslash{}}\OperatorTok{[}\NormalTok{Mu}\OperatorTok{],} \SpecialCharTok{\textbackslash{}}\OperatorTok{[}\NormalTok{Nu}\OperatorTok{]]}\NormalTok{ FV}\OperatorTok{[}\FunctionTok{p}\OperatorTok{,} \SpecialCharTok{\textbackslash{}}\OperatorTok{[}\NormalTok{Rho}\OperatorTok{]]} \SpecialCharTok{+}\NormalTok{ LC}\OperatorTok{[}\SpecialCharTok{\textbackslash{}}\OperatorTok{[}\NormalTok{Tau}\OperatorTok{],} \SpecialCharTok{\textbackslash{}}\OperatorTok{[}\NormalTok{Mu}\OperatorTok{],} \SpecialCharTok{\textbackslash{}}\OperatorTok{[}\NormalTok{Nu}\OperatorTok{],} \SpecialCharTok{\textbackslash{}}\OperatorTok{[}\NormalTok{Rho}\OperatorTok{]]}\NormalTok{ FV}\OperatorTok{[}\FunctionTok{p}\OperatorTok{,} \SpecialCharTok{\textbackslash{}}\OperatorTok{[}\NormalTok{Sigma}\OperatorTok{]]}\NormalTok{)) }
 
\NormalTok{Schouten}\OperatorTok{[}\SpecialCharTok{\%}\OperatorTok{]}
\end{Highlighting}
\end{Shaded}

\begin{dmath*}\breakingcomma
\overline{p}^{\tau } \bar{\epsilon }^{\mu \nu \rho \sigma }+\overline{p}^{\mu } \bar{\epsilon }^{\nu \rho \sigma \tau }+\overline{p}^{\nu } \bar{\epsilon }^{\rho \sigma \tau \mu }+\overline{p}^{\rho } \bar{\epsilon }^{\sigma \tau \mu \nu }+\overline{p}^{\sigma } \bar{\epsilon }^{\tau \mu \nu \rho }
\end{dmath*}

\begin{dmath*}\breakingcomma
0
\end{dmath*}
\end{document}
