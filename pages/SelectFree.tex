% !TeX program = pdflatex
% !TeX root = SelectFree.tex

\documentclass[../FeynCalcManual.tex]{subfiles}
\begin{document}
\hypertarget{selectfree}{%
\section{SelectFree}\label{selectfree}}

\texttt{SelectFree[\allowbreak{}expr,\ \allowbreak{}a,\ \allowbreak{}b,\ \allowbreak{}...]}
is equivalent to
\texttt{Select[\allowbreak{}expr,\ \allowbreak{}FreeQ2[\allowbreak{}\#{}\allowbreak{},\ \allowbreak{}\{\allowbreak{}a,\ \allowbreak{}b,\ \allowbreak{}...\}]\&{}\allowbreak{}]},
except the special cases:
\texttt{SelectFree[\allowbreak{}a,\ \allowbreak{}b]} returns \texttt{a}
and \texttt{SelectFree[\allowbreak{}a,\ \allowbreak{}a]} returns 1
(where \texttt{a} is not a product or a sum).

\subsection{See also}

\hyperlink{toc}{Overview}, \hyperlink{freeq2}{FreeQ2},
\hyperlink{selectnotfree}{SelectNotFree}.

\subsection{Examples}

\begin{Shaded}
\begin{Highlighting}[]
\NormalTok{SelectFree}\OperatorTok{[}\FunctionTok{a} \SpecialCharTok{+} \FunctionTok{b} \SpecialCharTok{+} \FunctionTok{f}\OperatorTok{[}\FunctionTok{a}\OperatorTok{]} \SpecialCharTok{+} \FunctionTok{d}\OperatorTok{,} \FunctionTok{a}\OperatorTok{]}
\end{Highlighting}
\end{Shaded}

\begin{dmath*}\breakingcomma
b+d
\end{dmath*}

\begin{Shaded}
\begin{Highlighting}[]
\NormalTok{SelectFree}\OperatorTok{[}\FunctionTok{x} \FunctionTok{y}\OperatorTok{,} \FunctionTok{x}\OperatorTok{]}
\end{Highlighting}
\end{Shaded}

\begin{dmath*}\breakingcomma
y
\end{dmath*}

\begin{Shaded}
\begin{Highlighting}[]
\NormalTok{SelectFree}\OperatorTok{[}\DecValTok{2} \FunctionTok{x} \FunctionTok{y} \FunctionTok{z} \FunctionTok{f}\OperatorTok{[}\FunctionTok{x}\OperatorTok{],} \OperatorTok{\{}\FunctionTok{x}\OperatorTok{,} \FunctionTok{y}\OperatorTok{\}]}
\end{Highlighting}
\end{Shaded}

\begin{dmath*}\breakingcomma
2 z
\end{dmath*}

\begin{Shaded}
\begin{Highlighting}[]
\NormalTok{SelectFree}\OperatorTok{[}\FunctionTok{a}\OperatorTok{,} \FunctionTok{b}\OperatorTok{]}
\end{Highlighting}
\end{Shaded}

\begin{dmath*}\breakingcomma
a
\end{dmath*}

\begin{Shaded}
\begin{Highlighting}[]
\NormalTok{SelectFree}\OperatorTok{[}\FunctionTok{a}\OperatorTok{,} \FunctionTok{a}\OperatorTok{]}
\end{Highlighting}
\end{Shaded}

\begin{dmath*}\breakingcomma
1
\end{dmath*}

\begin{Shaded}
\begin{Highlighting}[]
\NormalTok{SelectFree}\OperatorTok{[}\DecValTok{1}\OperatorTok{,} \FunctionTok{c}\OperatorTok{]}
\end{Highlighting}
\end{Shaded}

\begin{dmath*}\breakingcomma
1
\end{dmath*}

\begin{Shaded}
\begin{Highlighting}[]
\NormalTok{SelectFree}\OperatorTok{[}\FunctionTok{f}\OperatorTok{[}\FunctionTok{x}\OperatorTok{],} \FunctionTok{x}\OperatorTok{]}
\end{Highlighting}
\end{Shaded}

\begin{dmath*}\breakingcomma
1
\end{dmath*}
\end{document}
