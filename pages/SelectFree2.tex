% !TeX program = pdflatex
% !TeX root = SelectFree2.tex

\documentclass[../FeynCalcManual.tex]{subfiles}
\begin{document}
\hypertarget{selectfree2}{%
\section{SelectFree2}\label{selectfree2}}

\texttt{SelectFree2[\allowbreak{}expr,\ \allowbreak{}a,\ \allowbreak{}b,\ \allowbreak{}...]}
is similar to \texttt{SelectFree} but it also differs from the latter in
several respects.

If \texttt{expr} is a list, \texttt{SelectFree2} behaves exactly the
same way as \texttt{SelectFree}.

If \texttt{expr} is not a list, \texttt{SelectFree2} first expands the
expression w.r.t. the arguments via \texttt{Expand2}.

Furthermore, \texttt{SelectFree2[\allowbreak{}a,\ \allowbreak{}b]}
returns \texttt{a} and
\texttt{SelectFree2[\allowbreak{}a,\ \allowbreak{}a]} returns
\texttt{0}. This differs from the behavior of \texttt{SelectFree} but is
consistent with the naive expectations when applying the function to a
sum of terms.

\subsection{See also}

\hyperlink{toc}{Overview}, \hyperlink{freeq2}{FreeQ2},
\hyperlink{selectfree}{SelectFree},
\hyperlink{selectnotfree}{SelectNotFree},
\hyperlink{selectnotfree2}{SelectNotFree2}.

\subsection{Examples}

Note the difference between SelectFree and SelectFree2

\begin{Shaded}
\begin{Highlighting}[]
\NormalTok{SelectFree}\OperatorTok{[}\NormalTok{(}\FunctionTok{a} \SpecialCharTok{+} \FunctionTok{b}\NormalTok{) }\FunctionTok{c}\OperatorTok{,} \FunctionTok{b}\OperatorTok{]}
\end{Highlighting}
\end{Shaded}

\begin{dmath*}\breakingcomma
c
\end{dmath*}

\begin{Shaded}
\begin{Highlighting}[]
\NormalTok{SelectFree2}\OperatorTok{[}\NormalTok{(}\FunctionTok{a} \SpecialCharTok{+} \FunctionTok{b}\NormalTok{) }\FunctionTok{c}\OperatorTok{,} \FunctionTok{b}\OperatorTok{]}
\end{Highlighting}
\end{Shaded}

\begin{dmath*}\breakingcomma
a c
\end{dmath*}

\begin{Shaded}
\begin{Highlighting}[]
\NormalTok{SelectFree}\OperatorTok{[}\FunctionTok{a}\OperatorTok{,} \FunctionTok{b}\OperatorTok{]}
\end{Highlighting}
\end{Shaded}

\begin{dmath*}\breakingcomma
a
\end{dmath*}

\begin{Shaded}
\begin{Highlighting}[]
\NormalTok{SelectFree2}\OperatorTok{[}\FunctionTok{a}\OperatorTok{,} \FunctionTok{b}\OperatorTok{]}
\end{Highlighting}
\end{Shaded}

\begin{dmath*}\breakingcomma
a
\end{dmath*}

\begin{Shaded}
\begin{Highlighting}[]
\NormalTok{SelectFree}\OperatorTok{[}\FunctionTok{a}\OperatorTok{,} \FunctionTok{a}\OperatorTok{]}
\end{Highlighting}
\end{Shaded}

\begin{dmath*}\breakingcomma
1
\end{dmath*}

\begin{Shaded}
\begin{Highlighting}[]
\NormalTok{SelectFree2}\OperatorTok{[}\FunctionTok{a}\OperatorTok{,} \FunctionTok{a}\OperatorTok{]}
\end{Highlighting}
\end{Shaded}

\begin{dmath*}\breakingcomma
0
\end{dmath*}

When there are hidden zeros, \texttt{SelectFree2} obviously works better

\begin{Shaded}
\begin{Highlighting}[]
\NormalTok{SelectFree}\OperatorTok{[}\NormalTok{(}\FunctionTok{a} \SpecialCharTok{{-}} \FunctionTok{b} \SpecialCharTok{+} \FunctionTok{c}\NormalTok{)}\SpecialCharTok{\^{}}\DecValTok{2} \SpecialCharTok{{-}}\NormalTok{ (}\FunctionTok{a}\SpecialCharTok{\^{}}\DecValTok{2} \SpecialCharTok{{-}} \DecValTok{2} \FunctionTok{a} \FunctionTok{b} \SpecialCharTok{+} \DecValTok{2} \FunctionTok{a} \FunctionTok{c} \SpecialCharTok{+} \FunctionTok{b}\SpecialCharTok{\^{}}\DecValTok{2} \SpecialCharTok{{-}} \DecValTok{2} \FunctionTok{b} \FunctionTok{c} \SpecialCharTok{+} \FunctionTok{c}\SpecialCharTok{\^{}}\DecValTok{2}\NormalTok{)}\OperatorTok{,}\FunctionTok{a}\OperatorTok{]}
\end{Highlighting}
\end{Shaded}

\begin{dmath*}\breakingcomma
-b^2+2 b c-c^2
\end{dmath*}

\begin{Shaded}
\begin{Highlighting}[]
\NormalTok{SelectFree2}\OperatorTok{[}\NormalTok{(}\FunctionTok{a} \SpecialCharTok{{-}} \FunctionTok{b} \SpecialCharTok{+} \FunctionTok{c}\NormalTok{)}\SpecialCharTok{\^{}}\DecValTok{2} \SpecialCharTok{{-}}\NormalTok{ (}\FunctionTok{a}\SpecialCharTok{\^{}}\DecValTok{2} \SpecialCharTok{{-}} \DecValTok{2} \FunctionTok{a} \FunctionTok{b} \SpecialCharTok{+} \DecValTok{2} \FunctionTok{a} \FunctionTok{c} \SpecialCharTok{+} \FunctionTok{b}\SpecialCharTok{\^{}}\DecValTok{2} \SpecialCharTok{{-}} \DecValTok{2} \FunctionTok{b} \FunctionTok{c} \SpecialCharTok{+} \FunctionTok{c}\SpecialCharTok{\^{}}\DecValTok{2}\NormalTok{)}\OperatorTok{,}\FunctionTok{a}\OperatorTok{]}
\end{Highlighting}
\end{Shaded}

\begin{dmath*}\breakingcomma
0
\end{dmath*}
\end{document}
