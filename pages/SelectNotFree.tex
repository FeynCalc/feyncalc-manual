% !TeX program = pdflatex
% !TeX root = SelectNotFree.tex

\documentclass[../FeynCalcManual.tex]{subfiles}
\begin{document}
\hypertarget{selectnotfree}{%
\section{SelectNotFree}\label{selectnotfree}}

\texttt{SelectNotFree[\allowbreak{}expr,\ \allowbreak{}x]} returns that
part of \texttt{expr} which is not free of any occurrence of \texttt{x}.

\texttt{SelectNotFree[\allowbreak{}expr,\ \allowbreak{}a,\ \allowbreak{}b,\ \allowbreak{}...]}
is equivalent to
\texttt{Select[\allowbreak{}expr,\ \allowbreak{}!FreeQ2[\allowbreak{}\#{}\allowbreak{},\ \allowbreak{}\{\allowbreak{}a,\ \allowbreak{}b,\ \allowbreak{}...\}]\&{}\allowbreak{}]},
except the special cases:
\texttt{SelectNotFree[\allowbreak{}a,\ \allowbreak{}b]} returns
\texttt{1} and \texttt{SelectNotFree[\allowbreak{}a,\ \allowbreak{}a]}
returns \texttt{a} (where \texttt{a} is not a product or a sum).

\subsection{See also}

\hyperlink{toc}{Overview}, \hyperlink{freeq2}{FreeQ2},
\hyperlink{selectfree}{SelectFree}.

\subsection{Examples}

\begin{Shaded}
\begin{Highlighting}[]
\NormalTok{SelectNotFree}\OperatorTok{[}\FunctionTok{a} \SpecialCharTok{+} \FunctionTok{b} \SpecialCharTok{+} \FunctionTok{f}\OperatorTok{[}\FunctionTok{a}\OperatorTok{],} \FunctionTok{a}\OperatorTok{]}
\end{Highlighting}
\end{Shaded}

\begin{dmath*}\breakingcomma
f(a)+a
\end{dmath*}

\begin{Shaded}
\begin{Highlighting}[]
\NormalTok{SelectNotFree}\OperatorTok{[}\DecValTok{2} \FunctionTok{x} \FunctionTok{y} \FunctionTok{f}\OperatorTok{[}\FunctionTok{x}\OperatorTok{]} \FunctionTok{z}\OperatorTok{,} \OperatorTok{\{}\FunctionTok{x}\OperatorTok{,} \FunctionTok{y}\OperatorTok{\}]}
\end{Highlighting}
\end{Shaded}

\begin{dmath*}\breakingcomma
x y f(x)
\end{dmath*}

\begin{Shaded}
\begin{Highlighting}[]
\NormalTok{SelectNotFree}\OperatorTok{[}\FunctionTok{a}\OperatorTok{,} \FunctionTok{b}\OperatorTok{]}
\end{Highlighting}
\end{Shaded}

\begin{dmath*}\breakingcomma
1
\end{dmath*}

\begin{Shaded}
\begin{Highlighting}[]
\NormalTok{SelectNotFree}\OperatorTok{[}\FunctionTok{a} \SpecialCharTok{+} \FunctionTok{x}\OperatorTok{,} \FunctionTok{b}\OperatorTok{]}
\end{Highlighting}
\end{Shaded}

\begin{dmath*}\breakingcomma
0
\end{dmath*}

\begin{Shaded}
\begin{Highlighting}[]
\NormalTok{SelectNotFree}\OperatorTok{[}\FunctionTok{a}\OperatorTok{,} \FunctionTok{a}\OperatorTok{]}
\end{Highlighting}
\end{Shaded}

\begin{dmath*}\breakingcomma
a
\end{dmath*}

\begin{Shaded}
\begin{Highlighting}[]
\NormalTok{SelectNotFree}\OperatorTok{[}\DecValTok{1}\OperatorTok{,} \FunctionTok{c}\OperatorTok{]}
\end{Highlighting}
\end{Shaded}

\begin{dmath*}\breakingcomma
1
\end{dmath*}

\begin{Shaded}
\begin{Highlighting}[]
\NormalTok{SelectNotFree}\OperatorTok{[}\FunctionTok{f}\OperatorTok{[}\FunctionTok{x}\OperatorTok{],} \FunctionTok{x}\OperatorTok{]}
\end{Highlighting}
\end{Shaded}

\begin{dmath*}\breakingcomma
f(x)
\end{dmath*}
\end{document}
