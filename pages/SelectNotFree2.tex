% !TeX program = pdflatex
% !TeX root = SelectNotFree2.tex

\documentclass[../FeynCalcManual.tex]{subfiles}
\begin{document}
\hypertarget{selectnotfree2}{
\section{SelectNotFree2}\label{selectnotfree2}\index{SelectNotFree2}}

\texttt{SelectNotFree2[\allowbreak{}expr,\ \allowbreak{}a,\ \allowbreak{}b,\ \allowbreak{}...]}
is similar to \texttt{SelectNotFree} but it also differs from the latter
in several respects.

If \texttt{expr} is a list, \texttt{SelectNotFree2} behaves exactly the
same way as \texttt{SelectNotFree}.

If \texttt{expr} is not a list, \texttt{SelectNotFree2} first expands
the expression w.r.t. the arguments via \texttt{Expand2}.

Furthermore, \texttt{SelectNotFree2[\allowbreak{}a,\ \allowbreak{}b]}
returns \texttt{0}. This differs from the behavior of
\texttt{SelectFree} but is consistent with the naive expectations when
applying the function to a sum of terms.

\subsection{See also}

\hyperlink{toc}{Overview}, \hyperlink{freeq2}{FreeQ2},
\hyperlink{selectfree}{SelectFree},
\hyperlink{selectnotfree}{SelectNotFree},
\hyperlink{selectfree2}{SelectFree2}.

\subsection{Examples}

Note the difference between SelectNotFree and SelectNotFree2

\begin{Shaded}
\begin{Highlighting}[]
\NormalTok{SelectNotFree}\OperatorTok{[}\NormalTok{(}\FunctionTok{a} \SpecialCharTok{+} \FunctionTok{b}\NormalTok{) }\FunctionTok{c}\OperatorTok{,} \FunctionTok{b}\OperatorTok{]}
\end{Highlighting}
\end{Shaded}

\begin{dmath*}\breakingcomma
a+b
\end{dmath*}

\begin{Shaded}
\begin{Highlighting}[]
\NormalTok{SelectNotFree2}\OperatorTok{[}\NormalTok{(}\FunctionTok{a} \SpecialCharTok{+} \FunctionTok{b}\NormalTok{) }\FunctionTok{c}\OperatorTok{,} \FunctionTok{b}\OperatorTok{]}
\end{Highlighting}
\end{Shaded}

\begin{dmath*}\breakingcomma
b c
\end{dmath*}

\begin{Shaded}
\begin{Highlighting}[]
\NormalTok{SelectNotFree}\OperatorTok{[}\FunctionTok{a}\OperatorTok{,} \FunctionTok{b}\OperatorTok{]}
\end{Highlighting}
\end{Shaded}

\begin{dmath*}\breakingcomma
1
\end{dmath*}

\begin{Shaded}
\begin{Highlighting}[]
\NormalTok{SelectNotFree2}\OperatorTok{[}\FunctionTok{a}\OperatorTok{,} \FunctionTok{b}\OperatorTok{]}
\end{Highlighting}
\end{Shaded}

\begin{dmath*}\breakingcomma
0
\end{dmath*}

Here the behavior is identical

\begin{Shaded}
\begin{Highlighting}[]
\NormalTok{SelectNotFree}\OperatorTok{[}\FunctionTok{a}\OperatorTok{,} \FunctionTok{a}\OperatorTok{]}
\end{Highlighting}
\end{Shaded}

\begin{dmath*}\breakingcomma
a
\end{dmath*}

\begin{Shaded}
\begin{Highlighting}[]
\NormalTok{SelectNotFree2}\OperatorTok{[}\FunctionTok{a}\OperatorTok{,} \FunctionTok{a}\OperatorTok{]}
\end{Highlighting}
\end{Shaded}

\begin{dmath*}\breakingcomma
a
\end{dmath*}

When there are hidden zeros, \texttt{SelectNotFree2} obviously works
better

\begin{Shaded}
\begin{Highlighting}[]
\NormalTok{SelectNotFree}\OperatorTok{[}\NormalTok{(}\FunctionTok{a} \SpecialCharTok{{-}} \FunctionTok{b} \SpecialCharTok{+} \FunctionTok{c}\NormalTok{)}\SpecialCharTok{\^{}}\DecValTok{2} \SpecialCharTok{{-}}\NormalTok{ (}\FunctionTok{a}\SpecialCharTok{\^{}}\DecValTok{2} \SpecialCharTok{{-}} \DecValTok{2} \FunctionTok{a} \FunctionTok{b} \SpecialCharTok{+} \DecValTok{2} \FunctionTok{a} \FunctionTok{c} \SpecialCharTok{+} \FunctionTok{b}\SpecialCharTok{\^{}}\DecValTok{2} \SpecialCharTok{{-}} \DecValTok{2} \FunctionTok{b} \FunctionTok{c} \SpecialCharTok{+} \FunctionTok{c}\SpecialCharTok{\^{}}\DecValTok{2}\NormalTok{)}\OperatorTok{,} \FunctionTok{a}\OperatorTok{]}
\end{Highlighting}
\end{Shaded}

\begin{dmath*}\breakingcomma
-a^2+(a-b+c)^2+2 a b-2 a c
\end{dmath*}

\begin{Shaded}
\begin{Highlighting}[]
\NormalTok{SelectNotFree2}\OperatorTok{[}\NormalTok{(}\FunctionTok{a} \SpecialCharTok{{-}} \FunctionTok{b} \SpecialCharTok{+} \FunctionTok{c}\NormalTok{)}\SpecialCharTok{\^{}}\DecValTok{2} \SpecialCharTok{{-}}\NormalTok{ (}\FunctionTok{a}\SpecialCharTok{\^{}}\DecValTok{2} \SpecialCharTok{{-}} \DecValTok{2} \FunctionTok{a} \FunctionTok{b} \SpecialCharTok{+} \DecValTok{2} \FunctionTok{a} \FunctionTok{c} \SpecialCharTok{+} \FunctionTok{b}\SpecialCharTok{\^{}}\DecValTok{2} \SpecialCharTok{{-}} \DecValTok{2} \FunctionTok{b} \FunctionTok{c} \SpecialCharTok{+} \FunctionTok{c}\SpecialCharTok{\^{}}\DecValTok{2}\NormalTok{)}\OperatorTok{,} \FunctionTok{a}\OperatorTok{]}
\end{Highlighting}
\end{Shaded}

\begin{dmath*}\breakingcomma
0
\end{dmath*}
\end{document}
