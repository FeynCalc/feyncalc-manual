% !TeX program = pdflatex
% !TeX root = Series2.tex

\documentclass[../FeynCalcManual.tex]{subfiles}
\begin{document}
\hypertarget{series2}{%
\section{Series2}\label{series2}}

\texttt{Series2} performs a series expansion around \texttt{0}.
\texttt{Series2} is (up to the \texttt{Gamma}-bug in Mathematica
versions smaller than 5.0) equivalent to \texttt{Series}, except that it
applies \texttt{Normal} on the result and has an option
\texttt{FinalSubstitutions}.

\texttt{Series2[\allowbreak{}f,\ \allowbreak{}e,\ \allowbreak{}n]} is
equivalent to
\texttt{Series2[\allowbreak{}f,\ \allowbreak{}\{\allowbreak{}e,\ \allowbreak{}0,\ \allowbreak{}n\}]}.

\subsection{See also}

\hyperlink{toc}{Overview}, \hyperlink{series3}{Series3}.

\subsection{Examples}

\begin{Shaded}
\begin{Highlighting}[]
\NormalTok{Series2}\OperatorTok{[}\NormalTok{(}\FunctionTok{x}\NormalTok{ (}\DecValTok{1} \SpecialCharTok{{-}} \FunctionTok{x}\NormalTok{))}\SpecialCharTok{\^{}}\NormalTok{(}\SpecialCharTok{\textbackslash{}}\OperatorTok{[}\NormalTok{Delta}\OperatorTok{]}\SpecialCharTok{/}\DecValTok{2}\NormalTok{)}\OperatorTok{,} \SpecialCharTok{\textbackslash{}}\OperatorTok{[}\NormalTok{Delta}\OperatorTok{],} \DecValTok{1}\OperatorTok{]}
\end{Highlighting}
\end{Shaded}

\begin{dmath*}\breakingcomma
\frac{1}{2} \delta  \log (1-x)+\frac{1}{2} \delta  \log (x)+1
\end{dmath*}

\begin{Shaded}
\begin{Highlighting}[]
\NormalTok{Series2}\OperatorTok{[}\FunctionTok{Gamma}\OperatorTok{[}\FunctionTok{x}\OperatorTok{],} \FunctionTok{x}\OperatorTok{,} \DecValTok{1}\OperatorTok{]}
\end{Highlighting}
\end{Shaded}

\begin{dmath*}\breakingcomma
\frac{1}{2} \zeta (2) x+\frac{1}{x}
\end{dmath*}

\begin{Shaded}
\begin{Highlighting}[]
\FunctionTok{Series}\OperatorTok{[}\FunctionTok{Gamma}\OperatorTok{[}\FunctionTok{x}\OperatorTok{],} \OperatorTok{\{}\FunctionTok{x}\OperatorTok{,} \DecValTok{0}\OperatorTok{,} \DecValTok{1}\OperatorTok{\}]}
\end{Highlighting}
\end{Shaded}

\begin{dmath*}\breakingcomma
\frac{1}{x}-\gamma +\frac{1}{12} \left(6 \gamma ^2+\pi ^2\right) x+O\left(x^2\right)
\end{dmath*}

\begin{Shaded}
\begin{Highlighting}[]
\NormalTok{Series2}\OperatorTok{[}\FunctionTok{Gamma}\OperatorTok{[}\FunctionTok{x}\OperatorTok{],} \FunctionTok{x}\OperatorTok{,} \DecValTok{2}\OperatorTok{]}
\end{Highlighting}
\end{Shaded}

\begin{dmath*}\breakingcomma
-\frac{x^2 \zeta (3)}{3}+\frac{1}{2} \zeta (2) x+\frac{1}{x}
\end{dmath*}

\begin{Shaded}
\begin{Highlighting}[]
\NormalTok{Series2}\OperatorTok{[}\FunctionTok{Gamma}\OperatorTok{[}\FunctionTok{x}\OperatorTok{],} \FunctionTok{x}\OperatorTok{,} \DecValTok{2}\OperatorTok{,}\NormalTok{ FinalSubstitutions }\OtherTok{{-}\textgreater{}} \OperatorTok{\{\}]} \SpecialCharTok{//} \FunctionTok{FullSimplify}
\end{Highlighting}
\end{Shaded}

\begin{dmath*}\breakingcomma
\frac{1}{6} \left(-3 \gamma  \left(\zeta (2) x^2+2\right)-2 x^2 \zeta (3)-\gamma ^3 x^2+3 \zeta (2) x+3 \gamma ^2 x+\frac{6}{x}\right)
\end{dmath*}

\begin{Shaded}
\begin{Highlighting}[]
\FunctionTok{Series}\OperatorTok{[}\FunctionTok{Gamma}\OperatorTok{[}\FunctionTok{x}\OperatorTok{],} \OperatorTok{\{}\FunctionTok{x}\OperatorTok{,} \DecValTok{0}\OperatorTok{,} \FunctionTok{If}\OperatorTok{[}\VariableTok{$VersionNumber}\NormalTok{ \textless{} }\DecValTok{5}\OperatorTok{,} \DecValTok{4}\OperatorTok{,} \DecValTok{2}\OperatorTok{]\}]} \SpecialCharTok{//} \FunctionTok{Normal} \SpecialCharTok{//} \FunctionTok{Expand} \SpecialCharTok{//} \FunctionTok{FullSimplify}
\end{Highlighting}
\end{Shaded}

\begin{dmath*}\breakingcomma
\frac{1}{12} \left(-2 \gamma ^3 x^2-\gamma  \left(\pi ^2 x^2+12\right)+x \left(\pi ^2-4 x \zeta (3)\right)+6 \gamma ^2 x+\frac{12}{x}\right)
\end{dmath*}

There is a table of expansions of special hypergeometric functions.

\begin{Shaded}
\begin{Highlighting}[]
\NormalTok{Series2}\OperatorTok{[}\FunctionTok{HypergeometricPFQ}\OperatorTok{[\{}\DecValTok{1}\OperatorTok{,}\NormalTok{ OPEm }\SpecialCharTok{{-}} \DecValTok{1}\OperatorTok{,}\NormalTok{ Epsilon}\SpecialCharTok{/}\DecValTok{2} \SpecialCharTok{+}\NormalTok{ OPEm}\OperatorTok{\},} \OperatorTok{\{}\NormalTok{OPEm}\OperatorTok{,}\NormalTok{ OPEm }\SpecialCharTok{+}\NormalTok{ Epsilon}\OperatorTok{\},} \DecValTok{1}\OperatorTok{],}\NormalTok{ Epsilon}\OperatorTok{,} \DecValTok{1}\OperatorTok{]}
\end{Highlighting}
\end{Shaded}

\begin{dmath*}\breakingcomma
-\frac{2}{\varepsilon }+\frac{2 m}{\varepsilon }+\frac{1}{2} \varepsilon  m \psi ^{(1)}(m)-\frac{\varepsilon  \psi ^{(1)}(m)}{2}+1
\end{dmath*}

\begin{Shaded}
\begin{Highlighting}[]
\NormalTok{Series2}\OperatorTok{[}\FunctionTok{HypergeometricPFQ}\OperatorTok{[\{}\DecValTok{1}\OperatorTok{,}\NormalTok{ OPEm}\OperatorTok{,}\NormalTok{ Epsilon}\SpecialCharTok{/}\DecValTok{2} \SpecialCharTok{+}\NormalTok{ OPEm}\OperatorTok{\},} \OperatorTok{\{}\DecValTok{1} \SpecialCharTok{+}\NormalTok{ OPEm}\OperatorTok{,}\NormalTok{ Epsilon }\SpecialCharTok{+}\NormalTok{ OPEm}\OperatorTok{\},}  \DecValTok{1}\OperatorTok{],}\NormalTok{ Epsilon}\OperatorTok{,} \DecValTok{1}\OperatorTok{]}
\end{Highlighting}
\end{Shaded}

\begin{dmath*}\breakingcomma
\frac{1}{4} \varepsilon  \zeta (2) m+\frac{2 m}{\varepsilon }+\frac{1}{4} \varepsilon  m \psi ^{(0)}(m)^2+\frac{3}{4} \varepsilon  m \psi ^{(1)}(m)-\frac{1}{2} \varepsilon  m S_{11}(m-1)
\end{dmath*}

\begin{Shaded}
\begin{Highlighting}[]
\FunctionTok{Hypergeometric2F1}\OperatorTok{[}\DecValTok{1}\OperatorTok{,}\NormalTok{ Epsilon}\OperatorTok{,} \DecValTok{1} \SpecialCharTok{+} \DecValTok{2}\NormalTok{ Epsilon}\OperatorTok{,} \FunctionTok{x}\OperatorTok{]} 
 
\NormalTok{Series2}\OperatorTok{[}\SpecialCharTok{\%}\OperatorTok{,}\NormalTok{ Epsilon}\OperatorTok{,} \DecValTok{3}\OperatorTok{]}
\end{Highlighting}
\end{Shaded}

\begin{dmath*}\breakingcomma
\, _2F_1(1,\varepsilon ;2 \varepsilon +1;x)
\end{dmath*}

\begin{dmath*}\breakingcomma
-2 \varepsilon ^2 \zeta (2)+2 \varepsilon ^3 \;\text{Li}_3(1-x)+2 \varepsilon ^2 \;\text{Li}_2(1-x)-4 \varepsilon ^3 \;\text{Li}_2(1-x) \log (x)-4 \varepsilon ^3 S_{12}(1-x)-2 \varepsilon ^3 \zeta (2) \log (1-x)+4 \varepsilon ^3 \zeta (2) \log (x)-\frac{1}{6} \varepsilon ^3 \log ^3(1-x)-2 \varepsilon ^3 \log (1-x) \log ^2(x)+\varepsilon ^3 \log ^2(1-x) \log (x)-\frac{1}{2} \varepsilon ^2 \log ^2(1-x)+2 \varepsilon ^2 \log (1-x) \log (x)-\varepsilon  \log (1-x)+2 \varepsilon ^3 \zeta (3)+1
\end{dmath*}

There are over 100 more special expansions of \({}_2 F_1\) tabulated in
\texttt{Series2.m}. The interested user can consult the source code
(search for HYPERLIST).
\end{document}
