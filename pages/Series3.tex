% !TeX program = pdflatex
% !TeX root = Series3.tex

\documentclass[../FeynCalcManual.tex]{subfiles}
\begin{document}
\hypertarget{series3}{
\section{Series3}\label{series3}\index{Series3}}

\texttt{Series3} performs a series expansion around \texttt{0}.
\texttt{Series3} is equivalent to \texttt{Series}, except that it
applies \texttt{Normal} on the result and that some \texttt{Series} bugs
are fixed.

\texttt{Series3[\allowbreak{}f,\ \allowbreak{}e,\ \allowbreak{}n]} is
equivalent to
\texttt{Series3[\allowbreak{}f,\ \allowbreak{}\{\allowbreak{}e,\ \allowbreak{}0,\ \allowbreak{}n\}]}.

\subsection{See also}

\hyperlink{toc}{Overview}, \hyperlink{series2}{Series2}.

\subsection{Examples}

\begin{Shaded}
\begin{Highlighting}[]
\NormalTok{Series3}\OperatorTok{[}\NormalTok{(}\FunctionTok{x}\NormalTok{ (}\DecValTok{1} \SpecialCharTok{{-}} \FunctionTok{x}\NormalTok{))}\SpecialCharTok{\^{}}\NormalTok{(}\SpecialCharTok{\textbackslash{}}\OperatorTok{[}\NormalTok{Delta}\OperatorTok{]}\SpecialCharTok{/}\DecValTok{2}\NormalTok{)}\OperatorTok{,} \SpecialCharTok{\textbackslash{}}\OperatorTok{[}\NormalTok{Delta}\OperatorTok{],} \DecValTok{1}\OperatorTok{]}
\end{Highlighting}
\end{Shaded}

\begin{dmath*}\breakingcomma
\frac{1}{2} \delta  \log ((1-x) x)+1
\end{dmath*}

\begin{Shaded}
\begin{Highlighting}[]
\NormalTok{Series3}\OperatorTok{[}\FunctionTok{Gamma}\OperatorTok{[}\FunctionTok{x}\OperatorTok{],} \FunctionTok{x}\OperatorTok{,} \DecValTok{1}\OperatorTok{]} \SpecialCharTok{//} \FunctionTok{FullSimplify}
\end{Highlighting}
\end{Shaded}

\begin{dmath*}\breakingcomma
\frac{1}{x}-\gamma +1
\end{dmath*}
\end{document}
