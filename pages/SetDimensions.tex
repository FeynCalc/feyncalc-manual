% !TeX program = pdflatex
% !TeX root = SetDimensions.tex

\documentclass[../FeynCalcManual.tex]{subfiles}
\begin{document}
\hypertarget{setdimensions}{
\section{SetDimensions}\label{setdimensions}\index{SetDimensions}}

\texttt{SetDimensions} is an option for \texttt{ScalarProduct},
\texttt{CartesianScalarProduct} and various \texttt{FCLoopBasis*}
functions.

For scalar products it specifies the dimensions for which the scalar
products will be set when \texttt{ScalarProduct} or
\texttt{CartesianScalarProduct} are used with the equality sign, e.g.~in
\texttt{ScalarProduct[\allowbreak{}a,\ \allowbreak{}b] = m^2}. By
default, the scalar products are set for 4 and D dimensions. By changing
this option the user can add other dimensions or remove the existing
ones.

In case of the \texttt{FCLoopBasis*} functions this option specifies the
dimensions of the loop and external momenta to be taken into account
when extracting the propagator basis.

\subsection{See also}

\hyperlink{toc}{Overview}, \hyperlink{scalarproduct}{ScalarProduct},
\hyperlink{cartesianscalarproduct}{CartesianScalarProduct}.

\subsection{Examples}
\end{document}
