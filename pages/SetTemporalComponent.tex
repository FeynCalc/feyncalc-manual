% !TeX program = pdflatex
% !TeX root = SetTemporalComponent.tex

\documentclass[../FeynCalcManual.tex]{subfiles}
\begin{document}
\hypertarget{settemporalcomponent}{%
\section{SetTemporalComponent}\label{settemporalcomponent}}

\texttt{SetTemporalComponent[\allowbreak{}p,\ \allowbreak{}val]} sets
the value of the temporal component of a \(4\)-vector \(p\),
\texttt{TemporalPair[\allowbreak{}ExplicitLorentzIndex[\allowbreak{}0],\ \allowbreak{}TemporalMomentum[\allowbreak{}p]]}
to \texttt{val}.

\subsection{See also}

\hyperlink{toc}{Overview}, \hyperlink{tc}{TC},
\hyperlink{temporalpair}{TemporalPair},
\hyperlink{temporalmomentum}{TemporalMomentum}.

\subsection{Examples}

\begin{Shaded}
\begin{Highlighting}[]
\NormalTok{FCClearScalarProducts}\OperatorTok{[]} 
 
\FunctionTok{ClearAll}\OperatorTok{[}\FunctionTok{t}\OperatorTok{]} 
 
\NormalTok{SetTemporalComponent}\OperatorTok{[}\FunctionTok{p}\OperatorTok{,} \FunctionTok{t}\OperatorTok{]} 
 
\NormalTok{TC}\OperatorTok{[}\FunctionTok{p}\OperatorTok{]}
\end{Highlighting}
\end{Shaded}

\begin{dmath*}\breakingcomma
t
\end{dmath*}

\begin{Shaded}
\begin{Highlighting}[]
\NormalTok{TC}\OperatorTok{[}\FunctionTok{p} \SpecialCharTok{+} \FunctionTok{q}\OperatorTok{]} \SpecialCharTok{//}\NormalTok{ ExpandScalarProduct}
\end{Highlighting}
\end{Shaded}

\begin{dmath*}\breakingcomma
q^0+t
\end{dmath*}

\begin{Shaded}
\begin{Highlighting}[]
\NormalTok{SP}\OperatorTok{[}\FunctionTok{p}\OperatorTok{,} \FunctionTok{q}\OperatorTok{]} \SpecialCharTok{//}\NormalTok{ LorentzToCartesian}
\end{Highlighting}
\end{Shaded}

\begin{dmath*}\breakingcomma
q^0 t-\overline{p}\cdot \overline{q}
\end{dmath*}
\end{document}
