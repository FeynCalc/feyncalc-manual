% !TeX program = pdflatex
% !TeX root = ShiftPartialD.tex

\documentclass[../FeynCalcManual.tex]{subfiles}
\begin{document}
\hypertarget{shiftpartiald}{
\section{ShiftPartialD}\label{shiftpartiald}\index{ShiftPartialD}}

\texttt{ShiftPartialD[\allowbreak{}exp,\ \allowbreak{}\{\allowbreak{}FCPartialD[\allowbreak{}i1],\ \allowbreak{}FCPartialD[\allowbreak{}i2],\ \allowbreak{}...\},\ \allowbreak{}field]}
uses integration-by-parts identities to shift the derivatives of
\texttt{QuantumField}s such, that a term containing derivatives with
indices \texttt{i1,\ \allowbreak{}i2,\ \allowbreak{}...} acting on
\texttt{field} is eliminated from the final expression.

Notice that one must explicitly specify the type of the indices, e.g.~by
writing \texttt{FCPartialD[\allowbreak{}LorentzIndex[\allowbreak{}mu]]}
or \texttt{FCPartialD[\allowbreak{}CartesianIndex[\allowbreak{}i]]}.
Furthermore, the function always assumes that the surface term vanishes.

Often, when dealing with large expressions one would to integrate by
parts only certain terms but not every term containing given fields and
derivatives. In such situation one can specify a filter function via the
option \texttt{Select}.

\subsection{See also}

\hyperlink{toc}{Overview},
\hyperlink{explicitpartiald}{ExplicitPartialD},
\hyperlink{expandpartiald}{ExpandPartialD},
\hyperlink{quantumfield}{QuantumField}

\subsection{Examples}

\begin{Shaded}
\begin{Highlighting}[]
\NormalTok{exp1 }\ExtensionTok{=}\NormalTok{ QuantumField}\OperatorTok{[}\NormalTok{QuarkFieldPsiDagger}\OperatorTok{,}\NormalTok{ PauliIndex}\OperatorTok{[}\NormalTok{di1}\OperatorTok{]]}\NormalTok{ . RightPartialD}\OperatorTok{[}\NormalTok{CartesianIndex}\OperatorTok{[}\FunctionTok{i} 
    \OperatorTok{]]}\NormalTok{ . QuantumField}\OperatorTok{[}\SpecialCharTok{\textbackslash{}}\OperatorTok{[}\NormalTok{Phi}\OperatorTok{]]}\NormalTok{ . RightPartialD}\OperatorTok{[}\NormalTok{CartesianIndex}\OperatorTok{[}\FunctionTok{j}\OperatorTok{]]}\NormalTok{ . QuantumField}\OperatorTok{[}\NormalTok{QuarkFieldPsi}\OperatorTok{,}\NormalTok{ PauliIndex}\OperatorTok{[}\NormalTok{di2}\OperatorTok{]]}
\end{Highlighting}
\end{Shaded}

\begin{dmath*}\breakingcomma
\psi ^{\dagger \;\text{di1}}.\vec{\partial }_i.\phi .\vec{\partial }_j.\psi ^{\text{di2}}
\end{dmath*}

\begin{Shaded}
\begin{Highlighting}[]
\NormalTok{exp1 }\SpecialCharTok{//}\NormalTok{ ExpandPartialD}
\end{Highlighting}
\end{Shaded}

\begin{dmath*}\breakingcomma
\psi ^{\dagger \;\text{di1}}.\phi .\left(\partial _i\partial _j\psi ^{\text{di2}}\right)+\psi ^{\dagger \;\text{di1}}.\left(\partial _i\phi \right).\left(\partial _j\psi ^{\text{di2}}\right)
\end{dmath*}

\begin{Shaded}
\begin{Highlighting}[]
\NormalTok{ShiftPartialD}\OperatorTok{[}\NormalTok{exp1}\OperatorTok{,} \OperatorTok{\{}\NormalTok{FCPartialD}\OperatorTok{[}\NormalTok{CartesianIndex}\OperatorTok{[}\FunctionTok{i}\OperatorTok{]]\},}\NormalTok{ QuarkFieldPsi}\OperatorTok{,}\NormalTok{ FCVerbose }\OtherTok{{-}\textgreater{}} \SpecialCharTok{{-}}\DecValTok{1}\OperatorTok{]}
\end{Highlighting}
\end{Shaded}

\begin{dmath*}\breakingcomma
-\left(\partial _i\psi ^{\dagger \;\text{di1}}\right).\phi .\left(\partial _j\psi ^{\text{di2}}\right)
\end{dmath*}

This expression vanishes if one integrates by parts the term containing
\(\partial_\mu A_\nu\)

\begin{Shaded}
\begin{Highlighting}[]
\NormalTok{exp2 }\ExtensionTok{=}\NormalTok{ QuantumField}\OperatorTok{[}\NormalTok{GaugeField}\OperatorTok{,}\NormalTok{ LorentzIndex}\OperatorTok{[}\NormalTok{nu}\OperatorTok{]]}\NormalTok{ . QuantumField}\OperatorTok{[}\NormalTok{FCPartialD}\OperatorTok{[}\NormalTok{LorentzIndex}\OperatorTok{[}\NormalTok{mu}\OperatorTok{]],}\NormalTok{ FCPartialD}\OperatorTok{[}\NormalTok{LorentzIndex}\OperatorTok{[}\NormalTok{mu}\OperatorTok{]],} 
\NormalTok{     FCPartialD}\OperatorTok{[}\NormalTok{LorentzIndex}\OperatorTok{[}\NormalTok{rho}\OperatorTok{]],}\NormalTok{ FCPartialD}\OperatorTok{[}\NormalTok{LorentzIndex}\OperatorTok{[}\NormalTok{rho}\OperatorTok{]],}\NormalTok{ FCPartialD}\OperatorTok{[}\NormalTok{LorentzIndex}\OperatorTok{[}\NormalTok{nu}\OperatorTok{]],}\NormalTok{ FCPartialD}\OperatorTok{[}\NormalTok{LorentzIndex}\OperatorTok{[}\NormalTok{tau}\OperatorTok{]],} 
\NormalTok{     GaugeField}\OperatorTok{,}\NormalTok{ LorentzIndex}\OperatorTok{[}\NormalTok{tau}\OperatorTok{]]} \SpecialCharTok{+}\NormalTok{ QuantumField}\OperatorTok{[}\NormalTok{FCPartialD}\OperatorTok{[}\NormalTok{LorentzIndex}\OperatorTok{[}\NormalTok{mu}\OperatorTok{]],}\NormalTok{ GaugeField}\OperatorTok{,}\NormalTok{ LorentzIndex}\OperatorTok{[}\NormalTok{nu}\OperatorTok{]]}\NormalTok{ . QuantumField}\OperatorTok{[}\NormalTok{FCPartialD}\OperatorTok{[}\NormalTok{LorentzIndex}\OperatorTok{[}\NormalTok{rho}\OperatorTok{]],} 
\NormalTok{     FCPartialD}\OperatorTok{[}\NormalTok{LorentzIndex}\OperatorTok{[}\NormalTok{rho}\OperatorTok{]],}\NormalTok{ FCPartialD}\OperatorTok{[}\NormalTok{LorentzIndex}\OperatorTok{[}\NormalTok{mu}\OperatorTok{]],}\NormalTok{ FCPartialD}\OperatorTok{[}\NormalTok{LorentzIndex}\OperatorTok{[}\NormalTok{nu}\OperatorTok{]],}\NormalTok{ FCPartialD}\OperatorTok{[}\NormalTok{LorentzIndex}\OperatorTok{[}\NormalTok{tau}\OperatorTok{]],}\NormalTok{ GaugeField}\OperatorTok{,}\NormalTok{ LorentzIndex}\OperatorTok{[}\NormalTok{tau}\OperatorTok{]]}
\end{Highlighting}
\end{Shaded}

\begin{dmath*}\breakingcomma
A_{\text{nu}}.\left(\partial _{\text{mu}}\partial _{\text{mu}}\partial _{\text{rho}}\partial _{\text{rho}}\partial _{\text{nu}}\partial _{\text{tau}}A_{\text{tau}}\right)+\left(\partial _{\text{mu}}A_{\text{nu}}\right).\left(\partial _{\text{rho}}\partial _{\text{rho}}\partial _{\text{mu}}\partial _{\text{nu}}\partial _{\text{tau}}A_{\text{tau}}\right)
\end{dmath*}

By default \texttt{ShiftPartialD} will also apply IBPs to the other
term, which is not useful here

\begin{Shaded}
\begin{Highlighting}[]
\NormalTok{ShiftPartialD}\OperatorTok{[}\NormalTok{exp2}\OperatorTok{,} \OperatorTok{\{}\NormalTok{FCPartialD}\OperatorTok{[}\NormalTok{LorentzIndex}\OperatorTok{[}\NormalTok{mu}\OperatorTok{]]\},}\NormalTok{ GaugeField}\OperatorTok{]}
\end{Highlighting}
\end{Shaded}

\begin{dmath*}\breakingcomma
\text{Applying the following IBP relation: }\left\{\left(\partial _{\text{mu}}A_{\text{nu}}\right).\left(\partial _{\text{rho}}\partial _{\text{rho}}\partial _{\text{mu}}\partial _{\text{nu}}\partial _{\text{tau}}A_{\text{tau}}\right)\to -A_{\text{nu}}.\left(\partial _{\text{mu}}\partial _{\text{mu}}\partial _{\text{rho}}\partial _{\text{rho}}\partial _{\text{nu}}\partial _{\text{tau}}A_{\text{tau}}\right)\right\}
\end{dmath*}

\begin{dmath*}\breakingcomma
\text{Applying the following IBP relation: }\left\{A_{\text{nu}}.\left(\partial _{\text{mu}}\partial _{\text{mu}}\partial _{\text{rho}}\partial _{\text{rho}}\partial _{\text{nu}}\partial _{\text{tau}}A_{\text{tau}}\right)\to -2 \left(\partial _{\text{mu}}A_{\text{nu}}\right).\left(\partial _{\text{rho}}\partial _{\text{rho}}\partial _{\text{mu}}\partial _{\text{nu}}\partial _{\text{tau}}A_{\text{tau}}\right)-\left(\partial _{\text{mu}}\partial _{\text{mu}}A_{\text{nu}}\right).\left(\partial _{\text{rho}}\partial _{\text{rho}}\partial _{\text{nu}}\partial _{\text{tau}}A_{\text{tau}}\right)\right\}
\end{dmath*}

\begin{dmath*}\breakingcomma
-A_{\text{nu}}.\left(\partial _{\text{mu}}\partial _{\text{mu}}\partial _{\text{rho}}\partial _{\text{rho}}\partial _{\text{nu}}\partial _{\text{tau}}A_{\text{tau}}\right)-2 \left(\partial _{\text{mu}}A_{\text{nu}}\right).\left(\partial _{\text{rho}}\partial _{\text{rho}}\partial _{\text{mu}}\partial _{\text{nu}}\partial _{\text{tau}}A_{\text{tau}}\right)-\left(\partial _{\text{mu}}\partial _{\text{mu}}A_{\text{nu}}\right).\left(\partial _{\text{rho}}\partial _{\text{rho}}\partial _{\text{nu}}\partial _{\text{tau}}A_{\text{tau}}\right)
\end{dmath*}

Using a suitable filter function we can readily achieve the desired
result

\begin{Shaded}
\begin{Highlighting}[]
\NormalTok{ShiftPartialD}\OperatorTok{[}\NormalTok{exp2}\OperatorTok{,} \OperatorTok{\{}\NormalTok{FCPartialD}\OperatorTok{[}\NormalTok{LorentzIndex}\OperatorTok{[}\NormalTok{mu}\OperatorTok{]]\},}\NormalTok{ GaugeField}\OperatorTok{,} \FunctionTok{Select} \OtherTok{{-}\textgreater{}} 
     \FunctionTok{Function}\OperatorTok{[}\FunctionTok{x}\OperatorTok{,} \FunctionTok{FreeQ}\OperatorTok{[}\FunctionTok{x}\OperatorTok{,}\NormalTok{ QuantumField}\OperatorTok{[}\NormalTok{GaugeField}\OperatorTok{,}\NormalTok{ LorentzIndex}\OperatorTok{[}\NormalTok{nu}\OperatorTok{]]]]]} 
  
 
\end{Highlighting}
\end{Shaded}

\begin{dmath*}\breakingcomma
\text{Applying the following IBP relation: }\left\{\left(\partial _{\text{mu}}A_{\text{nu}}\right).\left(\partial _{\text{rho}}\partial _{\text{rho}}\partial _{\text{mu}}\partial _{\text{nu}}\partial _{\text{tau}}A_{\text{tau}}\right)\to -A_{\text{nu}}.\left(\partial _{\text{mu}}\partial _{\text{mu}}\partial _{\text{rho}}\partial _{\text{rho}}\partial _{\text{nu}}\partial _{\text{tau}}A_{\text{tau}}\right)\right\}
\end{dmath*}

\begin{dmath*}\breakingcomma
0
\end{dmath*}
\end{document}
