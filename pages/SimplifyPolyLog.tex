% !TeX program = pdflatex
% !TeX root = SimplifyPolyLog.tex

\documentclass[../FeynCalcManual.tex]{subfiles}
\begin{document}
\hypertarget{simplifypolylog}{%
\section{SimplifyPolyLog}\label{simplifypolylog}}

\texttt{SimplifyPolyLog[\allowbreak{}y]} performs several
simplifications assuming that the variables occuring in the \texttt{Log}
and \texttt{PolyLog} functions are between \texttt{0} and \texttt{1}.

The simplifications will in general not be valid if the arguments are
complex or outside the range between 0 and 1.

\subsection{See also}

\hyperlink{toc}{Overview}, \hyperlink{nielsen}{Nielsen}.

\subsection{Examples}

\begin{Shaded}
\begin{Highlighting}[]
\NormalTok{SimplifyPolyLog}\OperatorTok{[}\FunctionTok{PolyLog}\OperatorTok{[}\DecValTok{2}\OperatorTok{,} \DecValTok{1}\SpecialCharTok{/}\FunctionTok{x}\OperatorTok{]]}
\end{Highlighting}
\end{Shaded}

\begin{dmath*}\breakingcomma
\zeta (2)+\text{Li}_2(1-x)-\frac{1}{2} \log ^2(x)+\log (1-x) \log (x)+i \pi  \log (x)
\end{dmath*}

\begin{Shaded}
\begin{Highlighting}[]
\NormalTok{SimplifyPolyLog}\OperatorTok{[}\FunctionTok{PolyLog}\OperatorTok{[}\DecValTok{2}\OperatorTok{,} \FunctionTok{x}\OperatorTok{]]}
\end{Highlighting}
\end{Shaded}

\begin{dmath*}\breakingcomma
\zeta (2)-\text{Li}_2(1-x)-\log (1-x) \log (x)
\end{dmath*}

\begin{Shaded}
\begin{Highlighting}[]
\NormalTok{SimplifyPolyLog}\OperatorTok{[}\FunctionTok{PolyLog}\OperatorTok{[}\DecValTok{2}\OperatorTok{,} \DecValTok{1} \SpecialCharTok{{-}} \FunctionTok{x}\SpecialCharTok{\^{}}\DecValTok{2}\OperatorTok{]]}
\end{Highlighting}
\end{Shaded}

\begin{dmath*}\breakingcomma
-\zeta (2)+2 \;\text{Li}_2(1-x)-2 \;\text{Li}_2(-x)-2 \log (x) \log (x+1)
\end{dmath*}

\begin{Shaded}
\begin{Highlighting}[]
\NormalTok{SimplifyPolyLog}\OperatorTok{[}\FunctionTok{PolyLog}\OperatorTok{[}\DecValTok{2}\OperatorTok{,} \FunctionTok{x}\SpecialCharTok{\^{}}\DecValTok{2}\OperatorTok{]]}
\end{Highlighting}
\end{Shaded}

\begin{dmath*}\breakingcomma
2 \zeta (2)-2 \;\text{Li}_2(1-x)+2 \;\text{Li}_2(-x)-2 \log (1-x) \log (x)
\end{dmath*}

\begin{Shaded}
\begin{Highlighting}[]
\NormalTok{SimplifyPolyLog}\OperatorTok{[}\FunctionTok{PolyLog}\OperatorTok{[}\DecValTok{2}\OperatorTok{,} \SpecialCharTok{{-}}\FunctionTok{x}\SpecialCharTok{/}\NormalTok{(}\DecValTok{1} \SpecialCharTok{{-}} \FunctionTok{x}\NormalTok{)}\OperatorTok{]]}
\end{Highlighting}
\end{Shaded}

\begin{dmath*}\breakingcomma
-\zeta (2)+\text{Li}_2(1-x)-\frac{1}{2} \log ^2(1-x)+\log (x) \log (1-x)
\end{dmath*}

\begin{Shaded}
\begin{Highlighting}[]
\NormalTok{SimplifyPolyLog}\OperatorTok{[}\FunctionTok{PolyLog}\OperatorTok{[}\DecValTok{2}\OperatorTok{,} \FunctionTok{x}\SpecialCharTok{/}\NormalTok{(}\FunctionTok{x} \SpecialCharTok{{-}} \DecValTok{1}\NormalTok{)}\OperatorTok{]]}
\end{Highlighting}
\end{Shaded}

\begin{dmath*}\breakingcomma
-\zeta (2)+\text{Li}_2(1-x)-\frac{1}{2} \log ^2(1-x)+\log (x) \log (1-x)
\end{dmath*}

\begin{Shaded}
\begin{Highlighting}[]
\NormalTok{SimplifyPolyLog}\OperatorTok{[}\NormalTok{Nielsen}\OperatorTok{[}\DecValTok{1}\OperatorTok{,} \DecValTok{2}\OperatorTok{,} \SpecialCharTok{{-}}\FunctionTok{x}\SpecialCharTok{/}\NormalTok{(}\DecValTok{1} \SpecialCharTok{{-}} \FunctionTok{x}\NormalTok{)}\OperatorTok{]]}
\end{Highlighting}
\end{Shaded}

\begin{dmath*}\breakingcomma
S_{12}(x)-\frac{1}{6} \log ^3(1-x)
\end{dmath*}

\begin{Shaded}
\begin{Highlighting}[]
\NormalTok{SimplifyPolyLog}\OperatorTok{[}\FunctionTok{PolyLog}\OperatorTok{[}\DecValTok{3}\OperatorTok{,} \SpecialCharTok{{-}}\DecValTok{1}\SpecialCharTok{/}\FunctionTok{x}\OperatorTok{]]}
\end{Highlighting}
\end{Shaded}

\begin{dmath*}\breakingcomma
\text{Li}_3(-x)+\zeta (2) \log (x)+\frac{\log ^3(x)}{6}
\end{dmath*}

\begin{Shaded}
\begin{Highlighting}[]
\NormalTok{SimplifyPolyLog}\OperatorTok{[}\FunctionTok{PolyLog}\OperatorTok{[}\DecValTok{3}\OperatorTok{,} \DecValTok{1} \SpecialCharTok{{-}} \FunctionTok{x}\OperatorTok{]]}
\end{Highlighting}
\end{Shaded}

\begin{dmath*}\breakingcomma
\text{Li}_3(1-x)
\end{dmath*}

\begin{Shaded}
\begin{Highlighting}[]
\NormalTok{SimplifyPolyLog}\OperatorTok{[}\FunctionTok{PolyLog}\OperatorTok{[}\DecValTok{3}\OperatorTok{,} \FunctionTok{x}\SpecialCharTok{\^{}}\DecValTok{2}\OperatorTok{]]}
\end{Highlighting}
\end{Shaded}

\begin{dmath*}\breakingcomma
4 \;\text{Li}_3(-x)-4 \;\text{Li}_2(1-x) \log (x)-4 S_{12}(1-x)+4 \zeta (2) \log (x)-2 \log (1-x) \log ^2(x)+4 \zeta (3)
\end{dmath*}

\begin{Shaded}
\begin{Highlighting}[]
\NormalTok{SimplifyPolyLog}\OperatorTok{[}\FunctionTok{PolyLog}\OperatorTok{[}\DecValTok{3}\OperatorTok{,} \SpecialCharTok{{-}}\FunctionTok{x}\SpecialCharTok{/}\NormalTok{(}\DecValTok{1} \SpecialCharTok{{-}} \FunctionTok{x}\NormalTok{)}\OperatorTok{]]}
\end{Highlighting}
\end{Shaded}

\begin{dmath*}\breakingcomma
-\text{Li}_3(1-x)+\text{Li}_2(1-x) \log (x)+S_{12}(1-x)+\zeta (2) \log (1-x)-\zeta (2) \log (x)+\frac{1}{6} \log ^3(1-x)-\frac{1}{2} \log (x) \log ^2(1-x)+\frac{1}{2} \log ^2(x) \log (1-x)
\end{dmath*}

\begin{Shaded}
\begin{Highlighting}[]
\NormalTok{SimplifyPolyLog}\OperatorTok{[}\FunctionTok{PolyLog}\OperatorTok{[}\DecValTok{3}\OperatorTok{,} \DecValTok{1} \SpecialCharTok{{-}} \DecValTok{1}\SpecialCharTok{/}\FunctionTok{x}\OperatorTok{]]}
\end{Highlighting}
\end{Shaded}

\begin{dmath*}\breakingcomma
\text{Li}_2(1-x) \log (x)-\text{Li}_2(1-x) \log (1-x)+S_{12}(1-x)+S_{12}(x)+\frac{\log ^3(x)}{6}-\frac{1}{2} \log ^2(1-x) \log (x)-\zeta (3)
\end{dmath*}

\begin{Shaded}
\begin{Highlighting}[]
\NormalTok{SimplifyPolyLog}\OperatorTok{[}\FunctionTok{PolyLog}\OperatorTok{[}\DecValTok{4}\OperatorTok{,} \SpecialCharTok{{-}}\FunctionTok{x}\SpecialCharTok{/}\NormalTok{(}\DecValTok{1} \SpecialCharTok{{-}} \FunctionTok{x}\NormalTok{)}\OperatorTok{]]}
\end{Highlighting}
\end{Shaded}

\begin{dmath*}\breakingcomma
-\text{Li}_4(x)+\frac{1}{2} \;\text{Li}_2(1-x) \log ^2(1-x)-\text{Li}_2(1-x) \log (x) \log (1-x)-S_{13}(x)+S_{22}(x)-S_{12}(1-x) \log (1-x)-S_{12}(x) \log (1-x)-\frac{1}{2} \zeta (2) \log ^2(1-x)+\zeta (2) \log (x) \log (1-x)+\zeta (3) \log (1-x)-\frac{1}{24} \log ^4(1-x)+\frac{1}{2} \log (x) \log ^3(1-x)-\frac{1}{2} \log ^2(x) \log ^2(1-x)
\end{dmath*}

\begin{Shaded}
\begin{Highlighting}[]
\NormalTok{SimplifyPolyLog}\OperatorTok{[}\FunctionTok{Log}\OperatorTok{[}\FunctionTok{a} \SpecialCharTok{+} \FunctionTok{b}\SpecialCharTok{/}\FunctionTok{c}\OperatorTok{]]}
\end{Highlighting}
\end{Shaded}

\begin{dmath*}\breakingcomma
\log \left(\frac{a c+b}{c}\right)
\end{dmath*}

\begin{Shaded}
\begin{Highlighting}[]
\NormalTok{SimplifyPolyLog}\OperatorTok{[}\FunctionTok{Log}\OperatorTok{[}\DecValTok{1}\SpecialCharTok{/}\FunctionTok{x}\OperatorTok{]]}
\end{Highlighting}
\end{Shaded}

\begin{dmath*}\breakingcomma
-\log (x)
\end{dmath*}

\begin{Shaded}
\begin{Highlighting}[]
\NormalTok{SimplifyPolyLog}\OperatorTok{[}\FunctionTok{ArcTanh}\OperatorTok{[}\FunctionTok{x}\OperatorTok{]]}
\end{Highlighting}
\end{Shaded}

\begin{dmath*}\breakingcomma
\frac{1}{2} \log \left(-\frac{x+1}{1-x}\right)
\end{dmath*}

\begin{Shaded}
\begin{Highlighting}[]
\NormalTok{SimplifyPolyLog}\OperatorTok{[}\FunctionTok{ArcSinh}\OperatorTok{[}\FunctionTok{x}\OperatorTok{]]}
\end{Highlighting}
\end{Shaded}

\begin{dmath*}\breakingcomma
\log \left(\sqrt{x^2+1}+x\right)
\end{dmath*}

\begin{Shaded}
\begin{Highlighting}[]
\NormalTok{SimplifyPolyLog}\OperatorTok{[}\FunctionTok{ArcCosh}\OperatorTok{[}\FunctionTok{x}\OperatorTok{]]}
\end{Highlighting}
\end{Shaded}

\begin{dmath*}\breakingcomma
\log \left(\sqrt{x^2-1}+x\right)
\end{dmath*}
\end{document}
