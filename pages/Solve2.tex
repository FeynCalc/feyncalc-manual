% !TeX program = pdflatex
% !TeX root = Solve2.tex

\documentclass[../FeynCalcManual.tex]{subfiles}
\begin{document}
\hypertarget{solve2}{%
\section{Solve2}\label{solve2}}

\texttt{Solve2} is equivalent to \texttt{Solve}, except that it works
only for linear equations (and returns just a list) and accepts the
options \texttt{Factoring} and \texttt{FinalSubstitutions}.

\texttt{Solve2} uses the ``high school algorithm'' and factors
intermediate results. Therefore it can be drastically more useful than
\texttt{Solve}.

\subsection{See also}

\hyperlink{toc}{Overview}, \hyperlink{solve3}{Solve3}.

\subsection{Examples}

\begin{Shaded}
\begin{Highlighting}[]
\NormalTok{Solve2}\OperatorTok{[\{}\DecValTok{2} \FunctionTok{x} \ExtensionTok{==} \FunctionTok{b} \SpecialCharTok{{-}} \FunctionTok{w}\SpecialCharTok{/}\DecValTok{2}\OperatorTok{,} \FunctionTok{y} \SpecialCharTok{{-}} \FunctionTok{d} \ExtensionTok{==} \FunctionTok{p}\OperatorTok{\},} \OperatorTok{\{}\FunctionTok{x}\OperatorTok{,} \FunctionTok{y}\OperatorTok{\}]}
\end{Highlighting}
\end{Shaded}

\begin{dmath*}\breakingcomma
\left\{x\to \frac{1}{4} (2 b-w),y\to d+p\right\}
\end{dmath*}

If no equation sign is given the polynomials are supposed to be \(0\).

\begin{Shaded}
\begin{Highlighting}[]
\NormalTok{Solve2}\OperatorTok{[}\FunctionTok{x} \SpecialCharTok{+} \FunctionTok{y}\OperatorTok{,} \FunctionTok{x}\OperatorTok{]}
\end{Highlighting}
\end{Shaded}

\begin{dmath*}\breakingcomma
\{x\to -y\}
\end{dmath*}

\begin{Shaded}
\begin{Highlighting}[]
\NormalTok{Solve2}\OperatorTok{[}\FunctionTok{x} \SpecialCharTok{+} \FunctionTok{y}\OperatorTok{,} \FunctionTok{x}\OperatorTok{,}\NormalTok{ FinalSubstitutions }\OtherTok{{-}\textgreater{}} \OperatorTok{\{}\FunctionTok{y} \OtherTok{{-}\textgreater{}} \FunctionTok{h}\OperatorTok{\}]}
\end{Highlighting}
\end{Shaded}

\begin{dmath*}\breakingcomma
\{x\to -h\}
\end{dmath*}

\begin{Shaded}
\begin{Highlighting}[]
\NormalTok{Solve2}\OperatorTok{[\{}\DecValTok{2} \FunctionTok{x} \ExtensionTok{==} \FunctionTok{b} \SpecialCharTok{{-}} \FunctionTok{w}\SpecialCharTok{/}\DecValTok{2}\OperatorTok{,} \FunctionTok{y} \SpecialCharTok{{-}} \FunctionTok{d} \ExtensionTok{==} \FunctionTok{p}\OperatorTok{\},} \OperatorTok{\{}\FunctionTok{x}\OperatorTok{,} \FunctionTok{y}\OperatorTok{\},}\NormalTok{ Factoring }\OtherTok{{-}\textgreater{}} \FunctionTok{Expand}\OperatorTok{]}
\end{Highlighting}
\end{Shaded}

\begin{dmath*}\breakingcomma
\left\{x\to \frac{b}{2}-\frac{w}{4},y\to d+p\right\}
\end{dmath*}

\begin{Shaded}
\begin{Highlighting}[]
\FunctionTok{Solve}\OperatorTok{[\{}\DecValTok{2} \FunctionTok{x} \ExtensionTok{==} \FunctionTok{b} \SpecialCharTok{{-}} \FunctionTok{w}\SpecialCharTok{/}\DecValTok{2}\OperatorTok{,} \FunctionTok{y} \SpecialCharTok{{-}} \FunctionTok{d} \ExtensionTok{==} \FunctionTok{p}\OperatorTok{\},} \OperatorTok{\{}\FunctionTok{x}\OperatorTok{,} \FunctionTok{y}\OperatorTok{\}]}
\end{Highlighting}
\end{Shaded}

\begin{dmath*}\breakingcomma
\left\{\left\{x\to \frac{1}{4} (2 b-w),y\to d+p\right\}\right\}
\end{dmath*}
\end{document}
