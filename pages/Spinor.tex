% !TeX program = pdflatex
% !TeX root = Spinor.tex

\documentclass[../FeynCalcManual.tex]{subfiles}
\begin{document}
\hypertarget{spinor}{
\section{Spinor}\label{spinor}\index{Spinor}}

\texttt{Spinor[\allowbreak{}p,\ \allowbreak{}m,\ \allowbreak{}o]} is the
head of Dirac spinors. Which of the spinors \(u\), \(v\), \(\bar{u}\) or
\(\bar{v}\) is understood, depends on the sign of the momentum argument
\texttt{p} and the relative position of \texttt{Spinor} in the chain.

\begin{itemize}
\item
  \texttt{Spinor[\allowbreak{}Momentum[\allowbreak{}p],\ \allowbreak{}m]}
  means \(\bar{u}\) if it stands at the beginning of the chain.
\item
  \texttt{Spinor[\allowbreak{}Momentum[\allowbreak{}p],\ \allowbreak{}m]}
  means \(u\) if it stands at the end of the chain.
\item
  \texttt{Spinor[\allowbreak{}-Momentum[\allowbreak{}p],\ \allowbreak{}m]}
  means \(\bar{v}\) if it stands at the beginning of the chain.
\item
  \texttt{Spinor[\allowbreak{}-Momentum[\allowbreak{}p],\ \allowbreak{}m]}
  means \(v\) if it stands at the end of the chain.
\end{itemize}

Spinors of fermions of mass \(m\) are normalized to have
\(\bar{u} u=2 m\) and \(\bar{v} v=-2 m\).

The optional argument \texttt{o} can be used for additional degrees of
freedom. If no optional argument \texttt{o} is supplied, a \texttt{1} is
substituted in.

\subsection{See also}

\hyperlink{toc}{Overview}, \hyperlink{fermionspinsum}{FermionSpinSum},
\hyperlink{diracsimplify}{DiracSimplify}, \hyperlink{spinoru}{SpinorU},
\hyperlink{spinorv}{SpinorV}, \hyperlink{spinorubar}{SpinorUBar},
\hyperlink{spinorvbar}{SpinorVBar},
\hyperlink{spinorubard}{SpinorUBarD}, \hyperlink{spinorud}{SpinorUD},
\hyperlink{spinorvd}{SpinorVD}, \hyperlink{spinorvbard}{SpinorVBarD}.

\subsection{Examples}

\begin{Shaded}
\begin{Highlighting}[]
\NormalTok{Spinor}\OperatorTok{[}\NormalTok{Momentum}\OperatorTok{[}\FunctionTok{p}\OperatorTok{]]}
\end{Highlighting}
\end{Shaded}

\begin{dmath*}\breakingcomma
\varphi (\overline{p})
\end{dmath*}

\begin{Shaded}
\begin{Highlighting}[]
\NormalTok{Spinor}\OperatorTok{[}\NormalTok{Momentum}\OperatorTok{[}\FunctionTok{p}\OperatorTok{],} \FunctionTok{m}\OperatorTok{]}
\end{Highlighting}
\end{Shaded}

\begin{dmath*}\breakingcomma
\varphi (\overline{p},m)
\end{dmath*}

FeynCalc uses covariant normalization (as opposed to e.g.~the
normalization used in Bjorken \& Drell).

\begin{Shaded}
\begin{Highlighting}[]
\NormalTok{Spinor}\OperatorTok{[}\NormalTok{Momentum}\OperatorTok{[}\FunctionTok{p}\OperatorTok{],} \FunctionTok{m}\OperatorTok{]}\NormalTok{ . Spinor}\OperatorTok{[}\NormalTok{Momentum}\OperatorTok{[}\FunctionTok{p}\OperatorTok{],} \FunctionTok{m}\OperatorTok{]} \SpecialCharTok{//}\NormalTok{ DiracSimplify}
\end{Highlighting}
\end{Shaded}

\begin{dmath*}\breakingcomma
2 m
\end{dmath*}

\begin{Shaded}
\begin{Highlighting}[]
\NormalTok{DiracSimplify}\OperatorTok{[}\NormalTok{Spinor}\OperatorTok{[}\SpecialCharTok{{-}}\NormalTok{Momentum}\OperatorTok{[}\FunctionTok{p}\OperatorTok{],} \FunctionTok{m}\OperatorTok{]}\NormalTok{ . GS}\OperatorTok{[}\FunctionTok{p}\OperatorTok{]]}
\end{Highlighting}
\end{Shaded}

\begin{dmath*}\breakingcomma
-m \left(\varphi (-\overline{p},m)\right)
\end{dmath*}

\begin{Shaded}
\begin{Highlighting}[]
\NormalTok{Spinor}\OperatorTok{[}\NormalTok{Momentum}\OperatorTok{[}\FunctionTok{p}\OperatorTok{]]} \SpecialCharTok{//} \FunctionTok{StandardForm}

\CommentTok{(*Spinor[Momentum[p], 0, 1]*)}
\end{Highlighting}
\end{Shaded}

\begin{Shaded}
\begin{Highlighting}[]
\NormalTok{ChangeDimension}\OperatorTok{[}\NormalTok{Spinor}\OperatorTok{[}\NormalTok{Momentum}\OperatorTok{[}\FunctionTok{p}\OperatorTok{]],} \FunctionTok{D}\OperatorTok{]} \SpecialCharTok{//} \FunctionTok{StandardForm}

\CommentTok{(*Spinor[Momentum[p, D], 0, 1]*)}
\end{Highlighting}
\end{Shaded}

\begin{Shaded}
\begin{Highlighting}[]
\NormalTok{Spinor}\OperatorTok{[}\NormalTok{Momentum}\OperatorTok{[}\FunctionTok{p}\OperatorTok{],} \FunctionTok{m}\OperatorTok{]} \SpecialCharTok{//} \FunctionTok{StandardForm}

\CommentTok{(*Spinor[Momentum[p], m, 1]*)}
\end{Highlighting}
\end{Shaded}

\texttt{SmallVariable}s are discarded by \texttt{Spinor}.

\begin{Shaded}
\begin{Highlighting}[]
\NormalTok{Spinor}\OperatorTok{[}\NormalTok{Momentum}\OperatorTok{[}\FunctionTok{p}\OperatorTok{],}\NormalTok{ SmallVariable}\OperatorTok{[}\FunctionTok{m}\OperatorTok{]]} \SpecialCharTok{//} \FunctionTok{StandardForm}

\CommentTok{(*Spinor[Momentum[p], 0, 1]*)}
\end{Highlighting}
\end{Shaded}

\end{document}
