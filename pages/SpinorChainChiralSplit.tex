% !TeX program = pdflatex
% !TeX root = SpinorChainChiralSplit.tex

\documentclass[../FeynCalcManual.tex]{subfiles}
\begin{document}
\hypertarget{spinorchainchiralsplit}{
\section{SpinorChainChiralSplit}\label{spinorchainchiralsplit}\index{SpinorChainChiralSplit}}

\texttt{SpinorChainChiralSplit[\allowbreak{}exp]} introduces chiral
projectors in spinor chains that contain no \(\gamma^5\).

\subsection{See also}

\hyperlink{toc}{Overview},
\hyperlink{diracsubstitute67}{DiracSubstitute67},
\hyperlink{diracgamma}{DiracGamma},
\hyperlink{todiracgamma67}{ToDiracGamma67}.

\subsection{Examples}

\begin{Shaded}
\begin{Highlighting}[]
\NormalTok{SpinorUBar}\OperatorTok{[}\NormalTok{p1}\OperatorTok{,}\NormalTok{ m1}\OperatorTok{]}\NormalTok{ . GSD}\OperatorTok{[}\FunctionTok{p}\OperatorTok{]}\NormalTok{ . SpinorV}\OperatorTok{[}\NormalTok{p2}\OperatorTok{,}\NormalTok{ m2}\OperatorTok{]} 
 
\NormalTok{SpinorChainChiralSplit}\OperatorTok{[}\SpecialCharTok{\%}\OperatorTok{]}
\end{Highlighting}
\end{Shaded}

\begin{dmath*}\breakingcomma
\bar{u}(\text{p1},\text{m1}).(\gamma \cdot p).v(\text{p2},\text{m2})
\end{dmath*}

\begin{dmath*}\breakingcomma
\left(\varphi (\overline{\text{p1}},\text{m1})\right).(\gamma \cdot p).\bar{\gamma }^6.\left(\varphi (-\overline{\text{p2}},\text{m2})\right)+\left(\varphi (\overline{\text{p1}},\text{m1})\right).(\gamma \cdot p).\bar{\gamma }^7.\left(\varphi (-\overline{\text{p2}},\text{m2})\right)
\end{dmath*}
\end{document}
