% !TeX program = pdflatex
% !TeX root = SpinorChainTranspose.tex

\documentclass[../FeynCalcManual.tex]{subfiles}
\begin{document}
\hypertarget{spinorchaintranspose}{
\section{SpinorChainTranspose}\label{spinorchaintranspose}\index{SpinorChainTranspose}}

\texttt{SpinorChainTranspose[\allowbreak{}exp]} transposes particular
spinor chains in exp, which effectively switches the \(u\) and \(v\)
spinors and reverses the order of the Dirac matrices using charge
conjugation operator. This operation is often required in calculations
that involve Majorana particles. By default, the function will tranpose
all chains of the form \(\bar{v}.x.u\) and \(\bar{v}.x.v\). A different
or more fine grained choice can be obtained via the option
\texttt{Select}.

\subsection{See also}

\hyperlink{toc}{Overview},
\hyperlink{fcchargeconjugatetransposed}{FCChargeConjugateTransposed},
\hyperlink{diracgamma}{DiracGamma}, \hyperlink{spinor}{Spinor}.

\subsection{Examples}

\begin{Shaded}
\begin{Highlighting}[]
\NormalTok{SpinorVBarD}\OperatorTok{[}\NormalTok{p1}\OperatorTok{,}\NormalTok{ m1}\OperatorTok{]}\NormalTok{ . GAD}\OperatorTok{[}\SpecialCharTok{\textbackslash{}}\OperatorTok{[}\NormalTok{Mu}\OperatorTok{]]}\NormalTok{ . (GSD}\OperatorTok{[}\FunctionTok{p}\OperatorTok{]} \SpecialCharTok{+} \FunctionTok{m}\NormalTok{) . GAD}\OperatorTok{[}\SpecialCharTok{\textbackslash{}}\OperatorTok{[}\NormalTok{Mu}\OperatorTok{]]}\NormalTok{ . SpinorUD}\OperatorTok{[}\NormalTok{p2}\OperatorTok{,}\NormalTok{ m2}\OperatorTok{]} 
 
\NormalTok{SpinorChainTranspose}\OperatorTok{[}\SpecialCharTok{\%}\OperatorTok{]}
\end{Highlighting}
\end{Shaded}

\begin{dmath*}\breakingcomma
\bar{v}(\text{p1},\text{m1}).\gamma ^{\mu }.(m+\gamma \cdot p).\gamma ^{\mu }.u(\text{p2},\text{m2})
\end{dmath*}

\begin{dmath*}\breakingcomma
-(\varphi (-\text{p2},\text{m2})).\gamma ^{\mu }.(m-\gamma \cdot p).\gamma ^{\mu }.(\varphi (\text{p1},\text{m1}))
\end{dmath*}

\begin{Shaded}
\begin{Highlighting}[]
\NormalTok{SpinorUBarD}\OperatorTok{[}\NormalTok{p1}\OperatorTok{,}\NormalTok{ m1}\OperatorTok{]}\NormalTok{ . GAD}\OperatorTok{[}\SpecialCharTok{\textbackslash{}}\OperatorTok{[}\NormalTok{Mu}\OperatorTok{]]}\NormalTok{ . (GSD}\OperatorTok{[}\FunctionTok{p}\OperatorTok{]} \SpecialCharTok{+} \FunctionTok{m}\NormalTok{) . GAD}\OperatorTok{[}\SpecialCharTok{\textbackslash{}}\OperatorTok{[}\NormalTok{Mu}\OperatorTok{]]}\NormalTok{ . SpinorVD}\OperatorTok{[}\NormalTok{p2}\OperatorTok{,}\NormalTok{ m2}\OperatorTok{]} 
 
\NormalTok{SpinorChainTranspose}\OperatorTok{[}\SpecialCharTok{\%}\OperatorTok{]}
\end{Highlighting}
\end{Shaded}

\begin{dmath*}\breakingcomma
\bar{u}(\text{p1},\text{m1}).\gamma ^{\mu }.(m+\gamma \cdot p).\gamma ^{\mu }.v(\text{p2},\text{m2})
\end{dmath*}

\begin{dmath*}\breakingcomma
(\varphi (\text{p1},\text{m1})).\gamma ^{\mu }.(m+\gamma \cdot p).\gamma ^{\mu }.(\varphi (-\text{p2},\text{m2}))
\end{dmath*}

\begin{Shaded}
\begin{Highlighting}[]
\NormalTok{SpinorUBarD}\OperatorTok{[}\NormalTok{p1}\OperatorTok{,}\NormalTok{ m1}\OperatorTok{]}\NormalTok{ . GAD}\OperatorTok{[}\SpecialCharTok{\textbackslash{}}\OperatorTok{[}\NormalTok{Mu}\OperatorTok{]]}\NormalTok{ . (GSD}\OperatorTok{[}\FunctionTok{p}\OperatorTok{]} \SpecialCharTok{+} \FunctionTok{m}\NormalTok{) . GAD}\OperatorTok{[}\SpecialCharTok{\textbackslash{}}\OperatorTok{[}\NormalTok{Mu}\OperatorTok{]]}\NormalTok{ . SpinorVD}\OperatorTok{[}\NormalTok{p2}\OperatorTok{,}\NormalTok{ m2}\OperatorTok{]} 
 
\NormalTok{SpinorChainTranspose}\OperatorTok{[}\SpecialCharTok{\%}\OperatorTok{,} \FunctionTok{Select} \OtherTok{{-}\textgreater{}} \OperatorTok{\{\{}\NormalTok{SpinorUBarD}\OperatorTok{[}\NormalTok{\_}\OperatorTok{,}\NormalTok{ \_}\OperatorTok{],}\NormalTok{ SpinorVD}\OperatorTok{[}\NormalTok{\_}\OperatorTok{,}\NormalTok{ \_}\OperatorTok{]\}\}]}
\end{Highlighting}
\end{Shaded}

\begin{dmath*}\breakingcomma
\bar{u}(\text{p1},\text{m1}).\gamma ^{\mu }.(m+\gamma \cdot p).\gamma ^{\mu }.v(\text{p2},\text{m2})
\end{dmath*}

\begin{dmath*}\breakingcomma
-(\varphi (\text{p2},\text{m2})).\gamma ^{\mu }.(m-\gamma \cdot p).\gamma ^{\mu }.(\varphi (-\text{p1},\text{m1}))
\end{dmath*}
\end{document}
