% !TeX program = pdflatex
% !TeX root = SpinorUBar.tex

\documentclass[../FeynCalcManual.tex]{subfiles}
\begin{document}
\hypertarget{spinorubar}{
\section{SpinorUBar}\label{spinorubar}\index{SpinorUBar}}

\texttt{SpinorUBar[\allowbreak{}p,\ \allowbreak{}m]} denotes a
\(\bar{u}(p,m)\)-spinor that depends on the \(4\)-dimensional momentum
\(p\).

\subsection{See also}

\hyperlink{toc}{Overview}, \hyperlink{spinor}{Spinor},
\hyperlink{spinoru}{SpinorU}, \hyperlink{spinorv}{SpinorV},
\hyperlink{spinorvbar}{SpinorVBar},
\hyperlink{spinorubard}{SpinorUBarD}, \hyperlink{spinorud}{SpinorUD},
\hyperlink{spinorvd}{SpinorVD}, \hyperlink{spinorvbard}{SpinorVBarD}.

\subsection{Examples}

\begin{Shaded}
\begin{Highlighting}[]
\NormalTok{SpinorUBar}\OperatorTok{[}\FunctionTok{p}\OperatorTok{,} \FunctionTok{m}\OperatorTok{]}
\end{Highlighting}
\end{Shaded}

\begin{dmath*}\breakingcomma
\bar{u}(p,m)
\end{dmath*}

\begin{Shaded}
\begin{Highlighting}[]
\NormalTok{SpinorUBar}\OperatorTok{[}\FunctionTok{p}\OperatorTok{,} \FunctionTok{m}\OperatorTok{]} \SpecialCharTok{//}\NormalTok{ FCI }\SpecialCharTok{//} \FunctionTok{StandardForm}

\CommentTok{(*Spinor[Momentum[p], m, 1]*)}
\end{Highlighting}
\end{Shaded}

\begin{Shaded}
\begin{Highlighting}[]
\NormalTok{SpinorUBar}\OperatorTok{[}\FunctionTok{p}\OperatorTok{]}
\end{Highlighting}
\end{Shaded}

\begin{dmath*}\breakingcomma
\bar{u}(p)
\end{dmath*}

\begin{Shaded}
\begin{Highlighting}[]
\NormalTok{SpinorUBar}\OperatorTok{[}\FunctionTok{p}\OperatorTok{]} \SpecialCharTok{//}\NormalTok{ FCI }\SpecialCharTok{//} \FunctionTok{StandardForm}

\CommentTok{(*Spinor[Momentum[p], 0, 1]*)}
\end{Highlighting}
\end{Shaded}

\begin{Shaded}
\begin{Highlighting}[]
\NormalTok{SpinorUBar}\OperatorTok{[}\FunctionTok{p}\OperatorTok{]}\NormalTok{ . GS}\OperatorTok{[}\FunctionTok{p}\OperatorTok{]} 
 
\NormalTok{DiracEquation}\OperatorTok{[}\SpecialCharTok{\%}\OperatorTok{]}
\end{Highlighting}
\end{Shaded}

\begin{dmath*}\breakingcomma
\bar{u}(p).\left(\bar{\gamma }\cdot \overline{p}\right)
\end{dmath*}

\begin{dmath*}\breakingcomma
0
\end{dmath*}
\end{document}
