% !TeX program = pdflatex
% !TeX root = SpinorUD.tex

\documentclass[../FeynCalcManual.tex]{subfiles}
\begin{document}
\hypertarget{spinorud}{%
\section{SpinorUD}\label{spinorud}}

\texttt{SpinorUD[\allowbreak{}p,\ \allowbreak{}m]} denotes a
\(u(p,m)\)-spinor that depends on the \(D\)-dimensional momentum \(p\).

\subsection{See also}

\hyperlink{toc}{Overview}, \hyperlink{spinor}{Spinor},
\hyperlink{spinorubar}{SpinorUBar}, \hyperlink{spinoru}{SpinorU},
\hyperlink{spinorv}{SpinorV}, \hyperlink{spinorvbar}{SpinorVBar},
\hyperlink{spinorubard}{SpinorUBarD}, \hyperlink{spinorvd}{SpinorVD},
\hyperlink{spinorvbard}{SpinorVBarD}.

\subsection{Examples}

\begin{Shaded}
\begin{Highlighting}[]
\NormalTok{SpinorUD}\OperatorTok{[}\FunctionTok{p}\OperatorTok{,} \FunctionTok{m}\OperatorTok{]}
\end{Highlighting}
\end{Shaded}

\begin{dmath*}\breakingcomma
u(p,m)
\end{dmath*}

\begin{Shaded}
\begin{Highlighting}[]
\NormalTok{SpinorUD}\OperatorTok{[}\FunctionTok{p}\OperatorTok{,} \FunctionTok{m}\OperatorTok{]} \SpecialCharTok{//}\NormalTok{ FCI }\SpecialCharTok{//} \FunctionTok{StandardForm}

\CommentTok{(*Spinor[Momentum[p, D], m, 1]*)}
\end{Highlighting}
\end{Shaded}

\begin{Shaded}
\begin{Highlighting}[]
\NormalTok{SpinorUD}\OperatorTok{[}\FunctionTok{p}\OperatorTok{]}
\end{Highlighting}
\end{Shaded}

\begin{dmath*}\breakingcomma
u(p)
\end{dmath*}

\begin{Shaded}
\begin{Highlighting}[]
\NormalTok{SpinorUD}\OperatorTok{[}\FunctionTok{p}\OperatorTok{]} \SpecialCharTok{//}\NormalTok{ FCI }\SpecialCharTok{//} \FunctionTok{StandardForm}

\CommentTok{(*Spinor[Momentum[p, D], 0, 1]*)}
\end{Highlighting}
\end{Shaded}

\begin{Shaded}
\begin{Highlighting}[]
\NormalTok{GSD}\OperatorTok{[}\FunctionTok{p}\OperatorTok{]}\NormalTok{ . SpinorUD}\OperatorTok{[}\FunctionTok{p}\OperatorTok{]} 
 
\NormalTok{DiracEquation}\OperatorTok{[}\SpecialCharTok{\%}\OperatorTok{]}
\end{Highlighting}
\end{Shaded}

\begin{dmath*}\breakingcomma
(\gamma \cdot p).u(p)
\end{dmath*}

\begin{dmath*}\breakingcomma
0
\end{dmath*}
\end{document}
