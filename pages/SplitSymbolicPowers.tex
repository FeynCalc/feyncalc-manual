% !TeX program = pdflatex
% !TeX root = SplitSymbolicPowers.tex

\documentclass[../FeynCalcManual.tex]{subfiles}
\begin{document}
\hypertarget{splitsymbolicpowers}{%
\section{SplitSymbolicPowers}\label{splitsymbolicpowers}}

\texttt{SplitSymbolicPowers} is an option for
\texttt{FCFeynmanParametrize} and other functions. When set to
\texttt{True}, propagator powers containing symbols will be split into a
nonnegative integer piece and the remaining piece. This leads to a
somewhat different form of the resulting parametric integral, although
the final result remains the same. The default value is \texttt{False}.

\subsection{See also}

\hyperlink{toc}{Overview},
\hyperlink{fcfeynmanparametrize}{FCFeynmanParametrize}.

\subsection{Examples}

\begin{Shaded}
\begin{Highlighting}[]
\NormalTok{SFAD}\OperatorTok{[\{}\FunctionTok{p}\OperatorTok{,} \FunctionTok{m}\SpecialCharTok{\^{}}\DecValTok{2}\OperatorTok{,} \FunctionTok{r} \SpecialCharTok{+} \DecValTok{2}\OperatorTok{\}]}
\end{Highlighting}
\end{Shaded}

\begin{dmath*}\breakingcomma
(p^2-m^2+i \eta )^{-r-2}
\end{dmath*}

\begin{Shaded}
\begin{Highlighting}[]
\NormalTok{v1 }\ExtensionTok{=}\NormalTok{ FCFeynmanParametrize}\OperatorTok{[}\NormalTok{SFAD}\OperatorTok{[\{}\FunctionTok{p}\OperatorTok{,} \FunctionTok{m}\SpecialCharTok{\^{}}\DecValTok{2}\OperatorTok{,} \FunctionTok{r} \SpecialCharTok{{-}} \DecValTok{1}\OperatorTok{\}],} \OperatorTok{\{}\FunctionTok{p}\OperatorTok{\},} \FunctionTok{Names} \OtherTok{{-}\textgreater{}} \FunctionTok{x}\OperatorTok{,}\NormalTok{ FCReplaceD }\OtherTok{{-}\textgreater{}} \OperatorTok{\{}\FunctionTok{D} \OtherTok{{-}\textgreater{}} \DecValTok{4} \SpecialCharTok{{-}} \DecValTok{2}\NormalTok{ Epsilon}\OperatorTok{\}]}
\end{Highlighting}
\end{Shaded}

\begin{dmath*}\breakingcomma
\left\{1,\frac{m^6 (-1)^{r-1} \left(m^2\right)^{-\varepsilon -r} \Gamma (\varepsilon +r-3)}{\Gamma (r-1)},\{\}\right\}
\end{dmath*}

\begin{Shaded}
\begin{Highlighting}[]
\NormalTok{v2 }\ExtensionTok{=}\NormalTok{ FCFeynmanParametrize}\OperatorTok{[}\NormalTok{SFAD}\OperatorTok{[\{}\FunctionTok{p}\OperatorTok{,} \FunctionTok{m}\SpecialCharTok{\^{}}\DecValTok{2}\OperatorTok{,} \FunctionTok{r} \SpecialCharTok{{-}} \DecValTok{1}\OperatorTok{\}],} \OperatorTok{\{}\FunctionTok{p}\OperatorTok{\},} \FunctionTok{Names} \OtherTok{{-}\textgreater{}} \FunctionTok{x}\OperatorTok{,}\NormalTok{ FCReplaceD }\OtherTok{{-}\textgreater{}} \OperatorTok{\{}\FunctionTok{D} \OtherTok{{-}\textgreater{}} \DecValTok{4} \SpecialCharTok{{-}} \DecValTok{2}\NormalTok{ Epsilon}\OperatorTok{\},} 
\NormalTok{   SplitSymbolicPowers }\OtherTok{{-}\textgreater{}} \ConstantTok{True}\OperatorTok{]}
\end{Highlighting}
\end{Shaded}

\begin{dmath*}\breakingcomma
\left\{1,\frac{m^6 (-1)^{r-1} (r-1) \left(m^2\right)^{-\varepsilon -r} \Gamma (\varepsilon +r-3)}{\Gamma (r)},\{\}\right\}
\end{dmath*}

Both parametrizations lead to the same results (as expected)

\begin{Shaded}
\begin{Highlighting}[]
\FunctionTok{Series}\OperatorTok{[}\NormalTok{v1}\OperatorTok{[[}\DecValTok{2}\OperatorTok{]],} \OperatorTok{\{}\NormalTok{Epsilon}\OperatorTok{,} \DecValTok{0}\OperatorTok{,} \DecValTok{1}\OperatorTok{\}]} \SpecialCharTok{//} \FunctionTok{Normal} 
 
\FunctionTok{Series}\OperatorTok{[}\NormalTok{v2}\OperatorTok{[[}\DecValTok{2}\OperatorTok{]],} \OperatorTok{\{}\NormalTok{Epsilon}\OperatorTok{,} \DecValTok{0}\OperatorTok{,} \DecValTok{1}\OperatorTok{\}]} \SpecialCharTok{//} \FunctionTok{Normal} 
 
\SpecialCharTok{\%} \SpecialCharTok{{-}} \SpecialCharTok{\%\%} \SpecialCharTok{//} \FunctionTok{Simplify} \SpecialCharTok{//} \FunctionTok{FunctionExpand}
\end{Highlighting}
\end{Shaded}

\begin{dmath*}\breakingcomma
\frac{\varepsilon  m^6 (-1)^r \left(m^2\right)^{-r} \Gamma (r-3) \left(\log \left(m^2\right)-\psi ^{(0)}(r-3)\right)}{\Gamma (r-1)}-\frac{m^6 (-1)^r \left(m^2\right)^{-r} \Gamma (r-3)}{\Gamma (r-1)}
\end{dmath*}

\begin{dmath*}\breakingcomma
\frac{\varepsilon  m^6 (-1)^r (r-1) \left(m^2\right)^{-r} \Gamma (r-3) \left(\log \left(m^2\right)-\psi ^{(0)}(r-3)\right)}{\Gamma (r)}-\frac{m^6 (-1)^r (r-1) \left(m^2\right)^{-r} \Gamma (r-3)}{\Gamma (r)}
\end{dmath*}

\begin{dmath*}\breakingcomma
0
\end{dmath*}
\end{document}
