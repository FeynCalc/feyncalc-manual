% !TeX program = pdflatex
% !TeX root = SquareAmplitude.tex

\documentclass[../FeynCalcManual.tex]{subfiles}
\begin{document}
\hypertarget{squareamplitude}{
\section{SquareAmplitude}\label{squareamplitude}\index{SquareAmplitude}}

\texttt{SquareAmplitude[\allowbreak{}m1,\ \allowbreak{}m2]} multiplies
the amplitudes from the list \texttt{m1} with their complex conjugate
from the list \texttt{m2} to obtain the list of products \(m1_i m2_j\).
This function can be useful when exporting amplitudes obtained with
FeynCalc to FORM.

\subsection{See also}

\hyperlink{toc}{Overview}

\subsection{Examples}

\begin{Shaded}
\begin{Highlighting}[]
\FunctionTok{Clear}\OperatorTok{[}\NormalTok{a1}\OperatorTok{,}\NormalTok{ a2}\OperatorTok{,}\NormalTok{ a3}\OperatorTok{,}\NormalTok{ b1}\OperatorTok{,}\NormalTok{ b2}\OperatorTok{,}\NormalTok{ b3}\OperatorTok{]}
\end{Highlighting}
\end{Shaded}

\begin{Shaded}
\begin{Highlighting}[]
\NormalTok{SquareAmplitude}\OperatorTok{[\{}\NormalTok{a1}\OperatorTok{,}\NormalTok{ a2}\OperatorTok{,}\NormalTok{ a3}\OperatorTok{\},} \OperatorTok{\{}\NormalTok{b1}\OperatorTok{,}\NormalTok{ b2}\OperatorTok{,}\NormalTok{ b3}\OperatorTok{\}]}
\end{Highlighting}
\end{Shaded}

\begin{dmath*}\breakingcomma
\{\text{a1} \;\text{b1},\text{a1} \;\text{b2},\text{a1} \;\text{b3},\text{a2} \;\text{b1},\text{a2} \;\text{b2},\text{a2} \;\text{b3},\text{a3} \;\text{b1},\text{a3} \;\text{b2},\text{a3} \;\text{b3}\}
\end{dmath*}

\begin{Shaded}
\begin{Highlighting}[]
\NormalTok{SquareAmplitude}\OperatorTok{[\{}\NormalTok{a1}\OperatorTok{,}\NormalTok{ a2}\OperatorTok{,}\NormalTok{ a3}\OperatorTok{\},} \OperatorTok{\{}\NormalTok{b1}\OperatorTok{,}\NormalTok{ b2}\OperatorTok{,}\NormalTok{ b3}\OperatorTok{\},} \FunctionTok{List} \OtherTok{{-}\textgreater{}} \ConstantTok{False}\OperatorTok{]}
\end{Highlighting}
\end{Shaded}

\begin{dmath*}\breakingcomma
\text{a1} \;\text{b1}+\text{a1} \;\text{b2}+\text{a1} \;\text{b3}+\text{a2} \;\text{b1}+\text{a2} \;\text{b2}+\text{a2} \;\text{b3}+\text{a3} \;\text{b1}+\text{a3} \;\text{b2}+\text{a3} \;\text{b3}
\end{dmath*}

When the option \texttt{Real} is set to \texttt{True}, the amplitudes
are assumed to have no imaginary part

\begin{Shaded}
\begin{Highlighting}[]
\NormalTok{SquareAmplitude}\OperatorTok{[\{}\NormalTok{a1}\OperatorTok{,}\NormalTok{ a2}\OperatorTok{,}\NormalTok{ a3}\OperatorTok{\},} \OperatorTok{\{}\NormalTok{b1}\OperatorTok{,}\NormalTok{ b2}\OperatorTok{,}\NormalTok{ b3}\OperatorTok{\},} \FunctionTok{Real} \OtherTok{{-}\textgreater{}} \ConstantTok{True}\OperatorTok{,} \FunctionTok{List} \OtherTok{{-}\textgreater{}} \ConstantTok{False}\OperatorTok{]}
\end{Highlighting}
\end{Shaded}

\begin{dmath*}\breakingcomma
\text{a1} \;\text{b1}+2 \;\text{a2} \;\text{b1}+\text{a2} \;\text{b2}+2 \;\text{a3} \;\text{b1}+2 \;\text{a3} \;\text{b2}+\text{a3} \;\text{b3}
\end{dmath*}

The option \texttt{Indexed} allows us to attach a marker to each
contribution

\begin{Shaded}
\begin{Highlighting}[]
\NormalTok{SquareAmplitude}\OperatorTok{[\{}\NormalTok{a1}\OperatorTok{,}\NormalTok{ a2}\OperatorTok{,}\NormalTok{ a3}\OperatorTok{\},} \OperatorTok{\{}\NormalTok{b1}\OperatorTok{,}\NormalTok{ b2}\OperatorTok{,}\NormalTok{ b3}\OperatorTok{\},} \FunctionTok{Real} \OtherTok{{-}\textgreater{}} \ConstantTok{True}\OperatorTok{,} \FunctionTok{List} \OtherTok{{-}\textgreater{}} \ConstantTok{False}\OperatorTok{,}\NormalTok{ Indexed }\OtherTok{{-}\textgreater{}}\NormalTok{ mark}\OperatorTok{]}
\end{Highlighting}
\end{Shaded}

\begin{dmath*}\breakingcomma
\text{a1} \;\text{b1} \;\text{mark}(1,1)+2 \;\text{a2} \;\text{b1} \;\text{mark}(2,1)+\text{a2} \;\text{b2} \;\text{mark}(2,2)+2 \;\text{a3} \;\text{b1} \;\text{mark}(3,1)+2 \;\text{a3} \;\text{b2} \;\text{mark}(3,2)+\text{a3} \;\text{b3} \;\text{mark}(3,3)
\end{dmath*}
\end{document}
