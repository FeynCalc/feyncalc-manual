% !TeX program = pdflatex
% !TeX root = StandardPropagatorDenominator.tex

\documentclass[../FeynCalcManual.tex]{subfiles}
\begin{document}
\hypertarget{standardpropagatordenominator}{
\section{StandardPropagatorDenominator}\label{standardpropagatordenominator}\index{StandardPropagatorDenominator}}

\texttt{StandardPropagatorDenominator[\allowbreak{}propSq + ...,\ \allowbreak{}propEik +...,\ \allowbreak{}m^2,\ \allowbreak{}\{\allowbreak{}n,\ \allowbreak{}s\}]}
encodes a generic Lorentzian propagator denominator
\(\frac{1}{[(q_1+ \ldots)^2 + q_1 \cdot p_1 + \ldots + m^2 + s i \eta]^n}\).

\texttt{propSq} should be of the form
\texttt{Momentum[\allowbreak{}q1,\ \allowbreak{}D]}, while
\texttt{propEik} should look like
\texttt{Pair[\allowbreak{}Momentum[\allowbreak{}q1,\ \allowbreak{}D],\ \allowbreak{}Momentum[\allowbreak{}p1,\ \allowbreak{}D]}.

This allows to accommodate for standard propagators of the type
\(1/(p^2-m^2)\) but also for propagators encountered in manifestly
Lorentz covariant effective field theories such as HQET or SCET.

\texttt{StandardPropagatorDenominator} is an internal object. To enter
such propagators in FeynCalc you should use \texttt{SFAD}.

\subsection{See also}

\hyperlink{toc}{Overview},
\hyperlink{propagatordenominator}{PropagatorDenominator},
\hyperlink{cartesianpropagatordenominator}{CartesianPropagatorDenominator},
\hyperlink{genericpropagatordenominator}{GenericPropagatorDenominator},
\hyperlink{feynampdenominator}{FeynAmpDenominator}.

\subsection{Examples}

\begin{Shaded}
\begin{Highlighting}[]
\NormalTok{FeynAmpDenominator}\OperatorTok{[}\NormalTok{StandardPropagatorDenominator}\OperatorTok{[}\NormalTok{Momentum}\OperatorTok{[}\FunctionTok{p}\OperatorTok{,} \FunctionTok{D}\OperatorTok{],} \DecValTok{0}\OperatorTok{,} \SpecialCharTok{{-}}\FunctionTok{m}\SpecialCharTok{\^{}}\DecValTok{2}\OperatorTok{,} \OperatorTok{\{}\DecValTok{1}\OperatorTok{,} \DecValTok{1}\OperatorTok{\}]]}
\end{Highlighting}
\end{Shaded}

\begin{dmath*}\breakingcomma
\frac{1}{(p^2-m^2+i \eta )}
\end{dmath*}

\begin{Shaded}
\begin{Highlighting}[]
\NormalTok{FeynAmpDenominator}\OperatorTok{[}\NormalTok{StandardPropagatorDenominator}\OperatorTok{[}\DecValTok{0}\OperatorTok{,}\NormalTok{ Pair}\OperatorTok{[}\NormalTok{Momentum}\OperatorTok{[}\FunctionTok{p}\OperatorTok{,} \FunctionTok{D}\OperatorTok{],}\NormalTok{ Momentum}\OperatorTok{[}\FunctionTok{q}\OperatorTok{,} \FunctionTok{D}\OperatorTok{]],} \SpecialCharTok{{-}}\FunctionTok{m}\SpecialCharTok{\^{}}\DecValTok{2}\OperatorTok{,} \OperatorTok{\{}\DecValTok{1}\OperatorTok{,} \DecValTok{1}\OperatorTok{\}]]}
\end{Highlighting}
\end{Shaded}

\begin{dmath*}\breakingcomma
\frac{1}{(p\cdot q-m^2+i \eta )}
\end{dmath*}
\end{document}
