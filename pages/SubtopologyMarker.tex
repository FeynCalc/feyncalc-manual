% !TeX program = pdflatex
% !TeX root = SubtopologyMarker.tex

\documentclass[../FeynCalcManual.tex]{subfiles}
\begin{document}
\hypertarget{subtopologymarker}{
\section{SubtopologyMarker}\label{subtopologymarker}\index{SubtopologyMarker}}

\texttt{SubtopologyMarker} is an option for
\texttt{FCLoopFindTopologies}, \texttt{FCLoopFindTopologyMappings} and
other topology related functions. It denotes the symbol that is used to
specify that the given topology is a subtopology of another topology and
has been obtained by removing some of the original propagators
(i.e.~there are no momenta shifts involved)

This information must be put into the very last list of the FCTopology
object describing the corresponding subtopology. The syntax is
\texttt{marker->topoID} where \texttt{topoID} is the ID of the larger
topology.

Setting \texttt{SubtopologyMarker} to \texttt{False} means that the
information about subtopologies will not be added when generating
subtopologies and will be ignored by routines related to topology
mappings.

\subsection{See also}

\hyperlink{toc}{Overview},
\hyperlink{fcloopfindtopologies}{FCLoopFindTopologies},
\hyperlink{fcloopfindtopologymappings}{FCLoopFindTopologyMappings},
\hyperlink{fcloopfindsubtopologies}{FCLoopFindSubtopologies}.

\subsection{Examples}
\end{document}
