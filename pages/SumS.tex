% !TeX program = pdflatex
% !TeX root = SumS.tex

\documentclass[../FeynCalcManual.tex]{subfiles}
\begin{document}
\hypertarget{sums}{
\section{SumS}\label{sums}\index{SumS}}

\texttt{SumS[\allowbreak{}1,\ \allowbreak{}m]} is the harmonic number
\(S_ 1(m) = \sum_ {i=1}^m i^{-1}\).

\texttt{SumS[\allowbreak{}1,\ \allowbreak{}1,\ \allowbreak{}m]} is
\(\sum_{i=1}^m S_ 1 (i)/i\).

\texttt{SumS[\allowbreak{}k,\ \allowbreak{}l,\ \allowbreak{}m]} is
\(\sum_{i=1}^m S_l (i)/i^k\).

\texttt{SumS[\allowbreak{}r,\ \allowbreak{}n]} represents
\texttt{Sum[\allowbreak{}Sign[\allowbreak{}r]^i/i^Abs[\allowbreak{}r],\ \allowbreak{}\{\allowbreak{}i,\ \allowbreak{}1,\ \allowbreak{}n\}]}.

\texttt{SumS[\allowbreak{}r,\ \allowbreak{}s,\ \allowbreak{}n]} is
\texttt{Sum[\allowbreak{}Sign[\allowbreak{}r]^k/k^Abs[\allowbreak{}r] Sign[\allowbreak{}s]^j/j^Abs[\allowbreak{}s],\ \allowbreak{}\{\allowbreak{}k,\ \allowbreak{}1,\ \allowbreak{}n\},\ \allowbreak{}\{\allowbreak{}j,\ \allowbreak{}1,\ \allowbreak{}k\}]}
etc.

\subsection{See also}

\hyperlink{toc}{Overview}, \hyperlink{sump}{SumP},
\hyperlink{sumt}{SumT}.

\subsection{Examples}

\begin{Shaded}
\begin{Highlighting}[]
\NormalTok{SumS}\OperatorTok{[}\DecValTok{1}\OperatorTok{,} \FunctionTok{m} \SpecialCharTok{{-}} \DecValTok{1}\OperatorTok{]}
\end{Highlighting}
\end{Shaded}

\begin{dmath*}\breakingcomma
S_1(m-1)
\end{dmath*}

\begin{Shaded}
\begin{Highlighting}[]
\NormalTok{SumS}\OperatorTok{[}\DecValTok{2}\OperatorTok{,} \FunctionTok{m} \SpecialCharTok{{-}} \DecValTok{1}\OperatorTok{]}
\end{Highlighting}
\end{Shaded}

\begin{dmath*}\breakingcomma
S_2(m-1)
\end{dmath*}

\begin{Shaded}
\begin{Highlighting}[]
\NormalTok{SumS}\OperatorTok{[}\SpecialCharTok{{-}}\DecValTok{1}\OperatorTok{,} \FunctionTok{m}\OperatorTok{]}
\end{Highlighting}
\end{Shaded}

\begin{dmath*}\breakingcomma
S_{-1}(m)
\end{dmath*}

\begin{Shaded}
\begin{Highlighting}[]
\NormalTok{SumS}\OperatorTok{[}\DecValTok{1}\OperatorTok{,} \FunctionTok{m}\OperatorTok{,} \FunctionTok{Reduce} \OtherTok{{-}\textgreater{}} \ConstantTok{True}\OperatorTok{]}
\end{Highlighting}
\end{Shaded}

\begin{dmath*}\breakingcomma
S_1(m-1)+\frac{1}{m}
\end{dmath*}

\begin{Shaded}
\begin{Highlighting}[]
\NormalTok{SumS}\OperatorTok{[}\DecValTok{3}\OperatorTok{,} \FunctionTok{m} \SpecialCharTok{+} \DecValTok{2}\OperatorTok{,} \FunctionTok{Reduce} \OtherTok{{-}\textgreater{}} \ConstantTok{True}\OperatorTok{]}
\end{Highlighting}
\end{Shaded}

\begin{dmath*}\breakingcomma
S_3(m+1)+\frac{1}{(m+2)^3}
\end{dmath*}

\begin{Shaded}
\begin{Highlighting}[]
\FunctionTok{SetOptions}\OperatorTok{[}\NormalTok{SumS}\OperatorTok{,} \FunctionTok{Reduce} \OtherTok{{-}\textgreater{}} \ConstantTok{True}\OperatorTok{]}\NormalTok{; }
 
\NormalTok{SumS}\OperatorTok{[}\DecValTok{3}\OperatorTok{,} \FunctionTok{m} \SpecialCharTok{+} \DecValTok{2}\OperatorTok{]}
\end{Highlighting}
\end{Shaded}

\begin{dmath*}\breakingcomma
\frac{1}{m^3}+S_3(m-1)+\frac{1}{(m+1)^3}+\frac{1}{(m+2)^3}
\end{dmath*}

\begin{Shaded}
\begin{Highlighting}[]
\FunctionTok{SetOptions}\OperatorTok{[}\NormalTok{SumS}\OperatorTok{,} \FunctionTok{Reduce} \OtherTok{{-}\textgreater{}} \ConstantTok{False}\OperatorTok{]}\NormalTok{; }
 
\NormalTok{SumS}\OperatorTok{[}\DecValTok{1}\OperatorTok{,} \DecValTok{4}\OperatorTok{]}
\end{Highlighting}
\end{Shaded}

\begin{dmath*}\breakingcomma
\frac{25}{12}
\end{dmath*}

\begin{Shaded}
\begin{Highlighting}[]
\NormalTok{SumS}\OperatorTok{[}\DecValTok{1}\OperatorTok{,} \DecValTok{2}\OperatorTok{,} \FunctionTok{m} \SpecialCharTok{{-}} \DecValTok{1}\OperatorTok{]}
\end{Highlighting}
\end{Shaded}

\begin{dmath*}\breakingcomma
S_{12}(m-1)
\end{dmath*}

\begin{Shaded}
\begin{Highlighting}[]
\NormalTok{SumS}\OperatorTok{[}\DecValTok{1}\OperatorTok{,} \DecValTok{1}\OperatorTok{,} \DecValTok{1}\OperatorTok{,} \DecValTok{11}\OperatorTok{]}
\end{Highlighting}
\end{Shaded}

\begin{dmath*}\breakingcomma
\frac{31276937512951}{4260000729600}
\end{dmath*}

\begin{Shaded}
\begin{Highlighting}[]
\NormalTok{SumS}\OperatorTok{[}\SpecialCharTok{{-}}\DecValTok{1}\OperatorTok{,} \DecValTok{4}\OperatorTok{]}
\end{Highlighting}
\end{Shaded}

\begin{dmath*}\breakingcomma
-\frac{7}{12}
\end{dmath*}

\begin{Shaded}
\begin{Highlighting}[]
\NormalTok{SumT}\OperatorTok{[}\DecValTok{1}\OperatorTok{,} \DecValTok{4}\OperatorTok{]}
\end{Highlighting}
\end{Shaded}

\begin{dmath*}\breakingcomma
-\frac{7}{12}
\end{dmath*}
\end{document}
