% !TeX program = pdflatex
% !TeX root = TBox.tex

\documentclass[../FeynCalcManual.tex]{subfiles}
\begin{document}
\hypertarget{tbox}{
\section{TBox}\label{tbox}\index{TBox}}

\texttt{TBox[\allowbreak{}a,\ \allowbreak{}b,\ \allowbreak{}...]}
produces a
\texttt{RowBox[\allowbreak{}\{\allowbreak{}a,\ \allowbreak{}b,\ \allowbreak{}...\}]}
where \texttt{a,\ \allowbreak{}b,\ \allowbreak{}...} are boxed in
\texttt{TraditionalForm}.

\texttt{TBox} is used internally by FeynCalc to produce the typeset
output in \texttt{TraditionalForm}.

\subsection{See also}

\hyperlink{toc}{Overview}

\subsection{Examples}

\begin{Shaded}
\begin{Highlighting}[]
\NormalTok{TBox}\OperatorTok{[}\FunctionTok{a} \SpecialCharTok{+} \FunctionTok{b}\OperatorTok{]}
\SpecialCharTok{\%} \SpecialCharTok{//} \FunctionTok{DisplayForm}
\end{Highlighting}
\end{Shaded}

\begin{dmath*}\breakingcomma
\text{FormBox}[\text{RowBox}[\{\text{a},+,\text{b}\}],\text{TraditionalForm}]
\end{dmath*}

\begin{dmath*}\breakingcomma
a+b
\end{dmath*}
\end{document}
