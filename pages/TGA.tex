% !TeX program = pdflatex
% !TeX root = TGA.tex

\documentclass[../FeynCalcManual.tex]{subfiles}
\begin{document}
\hypertarget{tga}{%
\section{TGA}\label{tga}}

\texttt{TGA[\allowbreak{}]} can be used as input for \(\gamma^0\) in
\(4\) dimensions and is transformed into
\texttt{DiracGamma[\allowbreak{}ExplicitLorentzIndex[\allowbreak{}0]]}
by \texttt{FeynCalcInternal}.

\subsection{See also}

\hyperlink{toc}{Overview}, \hyperlink{ga}{GA},
\hyperlink{diracgamma}{DiracGamma}.

\subsection{Examples}

\begin{Shaded}
\begin{Highlighting}[]
\NormalTok{TGA}\OperatorTok{[]}
\end{Highlighting}
\end{Shaded}

\begin{dmath*}\breakingcomma
\bar{\gamma }^0
\end{dmath*}

\begin{Shaded}
\begin{Highlighting}[]
\NormalTok{TGA}\OperatorTok{[]} \SpecialCharTok{//}\NormalTok{ FCI }\SpecialCharTok{//} \FunctionTok{StandardForm}

\CommentTok{(*DiracGamma[ExplicitLorentzIndex[0]]*)}
\end{Highlighting}
\end{Shaded}

\begin{Shaded}
\begin{Highlighting}[]
\NormalTok{TGA}\OperatorTok{[]}\NormalTok{ . TGA}\OperatorTok{[]} \SpecialCharTok{//}\NormalTok{ DiracSimplify}
\end{Highlighting}
\end{Shaded}

\begin{dmath*}\breakingcomma
1
\end{dmath*}
\end{document}
