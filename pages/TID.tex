% !TeX program = pdflatex
% !TeX root = TID.tex

\documentclass[../FeynCalcManual.tex]{subfiles}
\begin{document}
\hypertarget{tid}{%
\section{TID}\label{tid}}

\texttt{TID[\allowbreak{}amp,\ \allowbreak{}q]} performs tensor
decomposition of 1-loop integrals with loop momentum \texttt{q}.

\subsection{See also}

\hyperlink{toc}{Overview}, \hyperlink{oneloopsimplify}{OneLoopSimplify},
\hyperlink{tidl}{TIDL}, \hyperlink{pavelimitto4}{PaVeLimitTo4}.

\subsection{Examples}

\begin{Shaded}
\begin{Highlighting}[]
\NormalTok{FCClearScalarProducts}\OperatorTok{[]}\NormalTok{;}
\end{Highlighting}
\end{Shaded}

\begin{Shaded}
\begin{Highlighting}[]
\NormalTok{int }\ExtensionTok{=}\NormalTok{ FAD}\OperatorTok{[\{}\FunctionTok{k}\OperatorTok{,} \FunctionTok{m}\OperatorTok{\},} \FunctionTok{k} \SpecialCharTok{{-}} \FunctionTok{Subscript}\OperatorTok{[}\FunctionTok{p}\OperatorTok{,} \DecValTok{1}\OperatorTok{],} \FunctionTok{k} \SpecialCharTok{{-}} \FunctionTok{Subscript}\OperatorTok{[}\FunctionTok{p}\OperatorTok{,} \DecValTok{2}\OperatorTok{]]}\NormalTok{ FVD}\OperatorTok{[}\FunctionTok{k}\OperatorTok{,} \SpecialCharTok{\textbackslash{}}\OperatorTok{[}\NormalTok{Mu}\OperatorTok{]]} \SpecialCharTok{//}\NormalTok{ FCI}
\end{Highlighting}
\end{Shaded}

\begin{dmath*}\breakingcomma
\frac{k^{\mu }}{\left(k^2-m^2\right).(k-p_1){}^2.(k-p_2){}^2}
\end{dmath*}

By default, all tensor integrals are reduced to the Passarino-Veltman
scalar integrals \(A_0\), \(B_0\), \(C_0\), \(D_0\) etc.

\begin{Shaded}
\begin{Highlighting}[]
\NormalTok{TID}\OperatorTok{[}\NormalTok{int}\OperatorTok{,} \FunctionTok{k}\OperatorTok{]}
\end{Highlighting}
\end{Shaded}

\begin{dmath*}\breakingcomma
\frac{p_1{}^2 p_2{}^{\mu }-p_1{}^{\mu } \left(p_1\cdot p_2\right)}{2 \left((p_1\cdot p_2){}^2-p_1{}^2 p_2{}^2\right) k^2.\left((k+p_1){}^2-m^2\right)}-\frac{p_2{}^2 \left(m^2+p_1{}^2\right) p_1{}^{\mu }+p_1{}^2 \left(m^2+p_2{}^2\right) p_2{}^{\mu }+\left(m^2+p_1{}^2\right) \left(-p_2{}^{\mu }\right) \left(p_1\cdot p_2\right)-\left(m^2+p_2{}^2\right) p_1{}^{\mu } \left(p_1\cdot p_2\right)}{2 \left((p_1\cdot p_2){}^2-p_1{}^2 p_2{}^2\right) \left(k^2-m^2\right).(k-p_2){}^2.(k-p_1){}^2}-\frac{p_2{}^{\mu } \left(p_1\cdot p_2\right)-p_2{}^2 p_1{}^{\mu }}{2 \left((p_1\cdot p_2){}^2-p_1{}^2 p_2{}^2\right) k^2.\left((k+p_2){}^2-m^2\right)}-\frac{p_1{}^2 p_2{}^{\mu }+p_2{}^2 p_1{}^{\mu }-p_1{}^{\mu } \left(p_1\cdot p_2\right)-p_2{}^{\mu } \left(p_1\cdot p_2\right)}{2 k^2.(k-p_1+p_2){}^2 \left((p_1\cdot p_2){}^2-p_1{}^2 p_2{}^2\right)}
\end{dmath*}

Scalar integrals can be converted to the Passarino-Veltman notation via
the option \texttt{ToPaVe}

\begin{Shaded}
\begin{Highlighting}[]
\NormalTok{TID}\OperatorTok{[}\NormalTok{int}\OperatorTok{,} \FunctionTok{k}\OperatorTok{,}\NormalTok{ ToPaVe }\OtherTok{{-}\textgreater{}} \ConstantTok{True}\OperatorTok{]}
\end{Highlighting}
\end{Shaded}

\begin{dmath*}\breakingcomma
\frac{i \pi ^2 \left(p_1{}^2 p_2{}^{\mu }-p_1{}^{\mu } \left(p_1\cdot p_2\right)\right) \;\text{B}_0\left(p_1{}^2,0,m^2\right)}{2 \left((p_1\cdot p_2){}^2-p_1{}^2 p_2{}^2\right)}-\frac{i \pi ^2 \left(p_2{}^{\mu } \left(p_1\cdot p_2\right)-p_2{}^2 p_1{}^{\mu }\right) \;\text{B}_0\left(p_2{}^2,0,m^2\right)}{2 \left((p_1\cdot p_2){}^2-p_1{}^2 p_2{}^2\right)}-\frac{i \pi ^2 \left(p_1{}^2 p_2{}^{\mu }+p_2{}^2 p_1{}^{\mu }-p_1{}^{\mu } \left(p_1\cdot p_2\right)-p_2{}^{\mu } \left(p_1\cdot p_2\right)\right) \;\text{B}_0\left(p_1{}^2-2 \left(p_1\cdot p_2\right)+p_2{}^2,0,0\right)}{2 \left((p_1\cdot p_2){}^2-p_1{}^2 p_2{}^2\right)}-\frac{i \pi ^2 \left(p_2{}^2 \left(m^2+p_1{}^2\right) p_1{}^{\mu }+p_1{}^2 \left(m^2+p_2{}^2\right) p_2{}^{\mu }+\left(m^2+p_1{}^2\right) \left(-p_2{}^{\mu }\right) \left(p_1\cdot p_2\right)-\left(m^2+p_2{}^2\right) p_1{}^{\mu } \left(p_1\cdot p_2\right)\right) \;\text{C}_0\left(p_1{}^2,p_2{}^2,p_1{}^2-2 \left(p_1\cdot p_2\right)+p_2{}^2,0,m^2,0\right)}{2 \left((p_1\cdot p_2){}^2-p_1{}^2 p_2{}^2\right)}
\end{dmath*}

We can force the reduction algorithm to use Passarino-Veltman
coefficient functions via the option \texttt{UsePaVeBasis}

\begin{Shaded}
\begin{Highlighting}[]
\NormalTok{TID}\OperatorTok{[}\NormalTok{int}\OperatorTok{,} \FunctionTok{k}\OperatorTok{,}\NormalTok{ UsePaVeBasis }\OtherTok{{-}\textgreater{}} \ConstantTok{True}\OperatorTok{]}
\end{Highlighting}
\end{Shaded}

\begin{dmath*}\breakingcomma
-i \pi ^2 p_1{}^{\mu } \;\text{C}_1\left(p_1{}^2,p_1{}^2+p_2{}^2-2 \left(p_1\cdot p_2\right),p_2{}^2,m^2,0,0\right)-i \pi ^2 p_2{}^{\mu } \;\text{C}_1\left(p_2{}^2,p_1{}^2+p_2{}^2-2 \left(p_1\cdot p_2\right),p_1{}^2,m^2,0,0\right)
\end{dmath*}

Very often the integral can be simplified via partial fractioning even
before performing the loop reduction. In this case the output will
contain a mixture of \texttt{FAD} symbols and Passarino-Veltman
functions

\begin{Shaded}
\begin{Highlighting}[]
\NormalTok{TID}\OperatorTok{[}\NormalTok{SPD}\OperatorTok{[}\FunctionTok{p}\OperatorTok{,} \FunctionTok{q}\OperatorTok{]}\NormalTok{ FAD}\OperatorTok{[}\FunctionTok{q}\OperatorTok{,} \OperatorTok{\{}\FunctionTok{q} \SpecialCharTok{{-}} \FunctionTok{p}\OperatorTok{,} \FunctionTok{m}\OperatorTok{\}]}\NormalTok{ FVD}\OperatorTok{[}\FunctionTok{q}\OperatorTok{,}\NormalTok{ mu}\OperatorTok{],} \FunctionTok{q}\OperatorTok{,}\NormalTok{ UsePaVeBasis }\OtherTok{{-}\textgreater{}} \ConstantTok{True}\OperatorTok{]}
\end{Highlighting}
\end{Shaded}

\begin{dmath*}\breakingcomma
\frac{p^{\text{mu}}}{2 \left(q^2-m^2\right)}+\frac{1}{2} i \pi ^2 \left(m^2-p^2\right) p^{\text{mu}} \left(\frac{\left(m^2-p^2\right) \;\text{B}_0\left(p^2,0,m^2\right)}{2 p^2}-\frac{\text{A}_0\left(m^2\right)}{2 p^2}\right)
\end{dmath*}

This can be avoided by setting both \texttt{UsePaVeBasis} and
\texttt{ToPaVe} to \texttt{True}

\begin{Shaded}
\begin{Highlighting}[]
\NormalTok{TID}\OperatorTok{[}\NormalTok{SPD}\OperatorTok{[}\FunctionTok{p}\OperatorTok{,} \FunctionTok{q}\OperatorTok{]}\NormalTok{ FAD}\OperatorTok{[}\FunctionTok{q}\OperatorTok{,} \OperatorTok{\{}\FunctionTok{q} \SpecialCharTok{{-}} \FunctionTok{p}\OperatorTok{,} \FunctionTok{m}\OperatorTok{\}]}\NormalTok{ FVD}\OperatorTok{[}\FunctionTok{q}\OperatorTok{,}\NormalTok{ mu}\OperatorTok{],} \FunctionTok{q}\OperatorTok{,}\NormalTok{ UsePaVeBasis }\OtherTok{{-}\textgreater{}} \ConstantTok{True}\OperatorTok{,}\NormalTok{ ToPaVe }\OtherTok{{-}\textgreater{}} \ConstantTok{True}\OperatorTok{]}
\end{Highlighting}
\end{Shaded}

\begin{dmath*}\breakingcomma
\frac{1}{2} i \pi ^2 \left(m^2-p^2\right) p^{\text{mu}} \left(\frac{\left(m^2-p^2\right) \;\text{B}_0\left(p^2,0,m^2\right)}{2 p^2}-\frac{\text{A}_0\left(m^2\right)}{2 p^2}\right)+\frac{1}{2} i \pi ^2 p^{\text{mu}} \;\text{A}_0\left(m^2\right)
\end{dmath*}

Alternatively, we may set \texttt{ToPaVe} to \texttt{Automatic} which
will automatically invoke the \texttt{ToPaVe} function if the final
result contains even a single Passarino-Veltman function

\begin{Shaded}
\begin{Highlighting}[]
\NormalTok{TID}\OperatorTok{[}\NormalTok{SPD}\OperatorTok{[}\FunctionTok{p}\OperatorTok{,} \FunctionTok{q}\OperatorTok{]}\NormalTok{ FAD}\OperatorTok{[}\FunctionTok{q}\OperatorTok{,} \OperatorTok{\{}\FunctionTok{q} \SpecialCharTok{{-}} \FunctionTok{p}\OperatorTok{,} \FunctionTok{m}\OperatorTok{\}]}\NormalTok{ FVD}\OperatorTok{[}\FunctionTok{q}\OperatorTok{,}\NormalTok{ mu}\OperatorTok{],} \FunctionTok{q}\OperatorTok{,}\NormalTok{ ToPaVe }\OtherTok{{-}\textgreater{}} \ConstantTok{Automatic}\OperatorTok{]}
\end{Highlighting}
\end{Shaded}

\begin{dmath*}\breakingcomma
\frac{\left(m^2-p^2\right)^2 p^{\text{mu}}}{4 p^2 q^2.\left((q-p)^2-m^2\right)}-\frac{\left(m^2-3 p^2\right) p^{\text{mu}}}{4 p^2 \left(q^2-m^2\right)}
\end{dmath*}

\begin{Shaded}
\begin{Highlighting}[]
\NormalTok{TID}\OperatorTok{[}\NormalTok{SPD}\OperatorTok{[}\FunctionTok{p}\OperatorTok{,} \FunctionTok{q}\OperatorTok{]}\NormalTok{ FAD}\OperatorTok{[}\FunctionTok{q}\OperatorTok{,} \OperatorTok{\{}\FunctionTok{q} \SpecialCharTok{{-}} \FunctionTok{p}\OperatorTok{,} \FunctionTok{m}\OperatorTok{\}]}\NormalTok{ FVD}\OperatorTok{[}\FunctionTok{q}\OperatorTok{,}\NormalTok{ mu}\OperatorTok{],} \FunctionTok{q}\OperatorTok{,}\NormalTok{ UsePaVeBasis }\OtherTok{{-}\textgreater{}} \ConstantTok{True}\OperatorTok{,}\NormalTok{ ToPaVe }\OtherTok{{-}\textgreater{}} \ConstantTok{Automatic}\OperatorTok{]}
\end{Highlighting}
\end{Shaded}

\begin{dmath*}\breakingcomma
\frac{1}{2} i \pi ^2 \left(m^2-p^2\right) p^{\text{mu}} \left(\frac{\left(m^2-p^2\right) \;\text{B}_0\left(p^2,0,m^2\right)}{2 p^2}-\frac{\text{A}_0\left(m^2\right)}{2 p^2}\right)+\frac{1}{2} i \pi ^2 p^{\text{mu}} \;\text{A}_0\left(m^2\right)
\end{dmath*}

The basis of Passarino-Veltman coefficient functions is used
automatically if there are zero Gram determinants

\begin{Shaded}
\begin{Highlighting}[]
\NormalTok{FCClearScalarProducts}\OperatorTok{[]}\NormalTok{; }
 
\NormalTok{SPD}\OperatorTok{[}\FunctionTok{Subscript}\OperatorTok{[}\FunctionTok{p}\OperatorTok{,} \DecValTok{1}\OperatorTok{],} \FunctionTok{Subscript}\OperatorTok{[}\FunctionTok{p}\OperatorTok{,} \DecValTok{1}\OperatorTok{]]} \ExtensionTok{=} \DecValTok{0}\NormalTok{; }
 
\NormalTok{SPD}\OperatorTok{[}\FunctionTok{Subscript}\OperatorTok{[}\FunctionTok{p}\OperatorTok{,} \DecValTok{2}\OperatorTok{],} \FunctionTok{Subscript}\OperatorTok{[}\FunctionTok{p}\OperatorTok{,} \DecValTok{2}\OperatorTok{]]} \ExtensionTok{=} \DecValTok{0}\NormalTok{; }
 
\NormalTok{SPD}\OperatorTok{[}\FunctionTok{Subscript}\OperatorTok{[}\FunctionTok{p}\OperatorTok{,} \DecValTok{1}\OperatorTok{],} \FunctionTok{Subscript}\OperatorTok{[}\FunctionTok{p}\OperatorTok{,} \DecValTok{2}\OperatorTok{]]} \ExtensionTok{=} \DecValTok{0}\NormalTok{; }
 
\NormalTok{TID}\OperatorTok{[}\NormalTok{FAD}\OperatorTok{[\{}\FunctionTok{k}\OperatorTok{,} \FunctionTok{m}\OperatorTok{\},} \FunctionTok{k} \SpecialCharTok{{-}} \FunctionTok{Subscript}\OperatorTok{[}\FunctionTok{p}\OperatorTok{,} \DecValTok{1}\OperatorTok{],} \FunctionTok{k} \SpecialCharTok{{-}} \FunctionTok{Subscript}\OperatorTok{[}\FunctionTok{p}\OperatorTok{,} \DecValTok{2}\OperatorTok{]]}\NormalTok{ FVD}\OperatorTok{[}\FunctionTok{k}\OperatorTok{,} \SpecialCharTok{\textbackslash{}}\OperatorTok{[}\NormalTok{Mu}\OperatorTok{]]} \SpecialCharTok{//}\NormalTok{ FCI}\OperatorTok{,} \FunctionTok{k}\OperatorTok{]} 
 
\NormalTok{FCClearScalarProducts}\OperatorTok{[]}\NormalTok{;}
\end{Highlighting}
\end{Shaded}

\begin{dmath*}\breakingcomma
-i \pi ^2 \left(p_1{}^{\mu }+p_2{}^{\mu }\right) \;\text{C}_1\left(0,0,0,0,0,m^2\right)
\end{dmath*}

In FeynCalc, Passarino-Veltman coefficient functions are defined in the
same way as in LoopTools. If one wants to use a different definition, it
is useful to activate the option GenPaVe

\begin{Shaded}
\begin{Highlighting}[]
\NormalTok{FCClearScalarProducts}\OperatorTok{[]}\NormalTok{; }
 
\NormalTok{SPD}\OperatorTok{[}\FunctionTok{Subscript}\OperatorTok{[}\FunctionTok{p}\OperatorTok{,} \DecValTok{1}\OperatorTok{],} \FunctionTok{Subscript}\OperatorTok{[}\FunctionTok{p}\OperatorTok{,} \DecValTok{1}\OperatorTok{]]} \ExtensionTok{=} \DecValTok{0}\NormalTok{; }
 
\NormalTok{SPD}\OperatorTok{[}\FunctionTok{Subscript}\OperatorTok{[}\FunctionTok{p}\OperatorTok{,} \DecValTok{2}\OperatorTok{],} \FunctionTok{Subscript}\OperatorTok{[}\FunctionTok{p}\OperatorTok{,} \DecValTok{2}\OperatorTok{]]} \ExtensionTok{=} \DecValTok{0}\NormalTok{; }
 
\NormalTok{SPD}\OperatorTok{[}\FunctionTok{Subscript}\OperatorTok{[}\FunctionTok{p}\OperatorTok{,} \DecValTok{1}\OperatorTok{],} \FunctionTok{Subscript}\OperatorTok{[}\FunctionTok{p}\OperatorTok{,} \DecValTok{2}\OperatorTok{]]} \ExtensionTok{=} \DecValTok{0}\NormalTok{; }
 
\NormalTok{TID}\OperatorTok{[}\NormalTok{FAD}\OperatorTok{[\{}\FunctionTok{k}\OperatorTok{,} \FunctionTok{m}\OperatorTok{\},} \FunctionTok{k} \SpecialCharTok{{-}} \FunctionTok{Subscript}\OperatorTok{[}\FunctionTok{p}\OperatorTok{,} \DecValTok{1}\OperatorTok{],} \FunctionTok{k} \SpecialCharTok{{-}} \FunctionTok{Subscript}\OperatorTok{[}\FunctionTok{p}\OperatorTok{,} \DecValTok{2}\OperatorTok{]]}\NormalTok{ FVD}\OperatorTok{[}\FunctionTok{k}\OperatorTok{,} \SpecialCharTok{\textbackslash{}}\OperatorTok{[}\NormalTok{Mu}\OperatorTok{]]} \SpecialCharTok{//}\NormalTok{ FCI}\OperatorTok{,} \FunctionTok{k}\OperatorTok{,}\NormalTok{ GenPaVe }\OtherTok{{-}\textgreater{}} \ConstantTok{True}\OperatorTok{]} 
 
\NormalTok{FCClearScalarProducts}\OperatorTok{[]}\NormalTok{;}
\end{Highlighting}
\end{Shaded}

\begin{dmath*}\breakingcomma
-i \pi ^2 p_1{}^{\mu } \;\text{GenPaVe}\left(\{1\},\left(
\begin{array}{cc}
 0 & m \\
 p_1 & 0 \\
 p_2 & 0 \\
\end{array}
\right)\right)-i \pi ^2 p_2{}^{\mu } \;\text{GenPaVe}\left(\{2\},\left(
\begin{array}{cc}
 0 & m \\
 p_1 & 0 \\
 p_2 & 0 \\
\end{array}
\right)\right)
\end{dmath*}

To simplify manifestly IR-finite 1-loop results written in terms of
Passarino-Veltman functions, we may employ the option
\texttt{PaVeLimitTo4} (must be used together with \texttt{ToPaVe}). The
result is valid up to 0th order in \texttt{Epsilon}, i.e.~sufficient for
1-loop calculations.

\begin{Shaded}
\begin{Highlighting}[]
\NormalTok{FCClearScalarProducts}\OperatorTok{[]}\NormalTok{; }
 
\NormalTok{int }\ExtensionTok{=}\NormalTok{ (}\FunctionTok{D} \SpecialCharTok{{-}} \DecValTok{1}\NormalTok{) (}\FunctionTok{D} \SpecialCharTok{{-}} \DecValTok{2}\NormalTok{)}\SpecialCharTok{/}\NormalTok{(}\FunctionTok{D} \SpecialCharTok{{-}} \DecValTok{3}\NormalTok{) FVD}\OperatorTok{[}\FunctionTok{p}\OperatorTok{,}\NormalTok{ mu}\OperatorTok{]}\NormalTok{ FVD}\OperatorTok{[}\FunctionTok{p}\OperatorTok{,}\NormalTok{ nu}\OperatorTok{]}\NormalTok{ FAD}\OperatorTok{[}\FunctionTok{p}\OperatorTok{,} \FunctionTok{p} \SpecialCharTok{{-}} \FunctionTok{q}\OperatorTok{]}
\end{Highlighting}
\end{Shaded}

\begin{dmath*}\breakingcomma
\frac{(D-2) (D-1) p^{\text{mu}} p^{\text{nu}}}{(D-3) p^2.(p-q)^2}
\end{dmath*}

\begin{Shaded}
\begin{Highlighting}[]
\NormalTok{TID}\OperatorTok{[}\NormalTok{int}\OperatorTok{,} \FunctionTok{p}\OperatorTok{,}\NormalTok{ ToPaVe }\OtherTok{{-}\textgreater{}} \ConstantTok{True}\OperatorTok{]}
\end{Highlighting}
\end{Shaded}

\begin{dmath*}\breakingcomma
\frac{i \pi ^2 (2-D) \;\text{B}_0\left(q^2,0,0\right) \left(D q^{\text{mu}} q^{\text{nu}}-q^2 g^{\text{mu}\;\text{nu}}\right)}{4 (3-D)}
\end{dmath*}

\begin{Shaded}
\begin{Highlighting}[]
\NormalTok{TID}\OperatorTok{[}\NormalTok{int}\OperatorTok{,} \FunctionTok{p}\OperatorTok{,}\NormalTok{ ToPaVe }\OtherTok{{-}\textgreater{}} \ConstantTok{True}\OperatorTok{,}\NormalTok{ PaVeLimitTo4 }\OtherTok{{-}\textgreater{}} \ConstantTok{True}\OperatorTok{]}
\end{Highlighting}
\end{Shaded}

\begin{dmath*}\breakingcomma
\frac{1}{2} i \pi ^2 \;\text{B}_0\left(\overline{q}^2,0,0\right) \left(4 \overline{q}^{\text{mu}} \overline{q}^{\text{nu}}-\overline{q}^2 \bar{g}^{\text{mu}\;\text{nu}}\right)+\frac{1}{2} i \pi ^2 \left(2 \overline{q}^{\text{mu}} \overline{q}^{\text{nu}}-\overline{q}^2 \bar{g}^{\text{mu}\;\text{nu}}\right)
\end{dmath*}
\end{document}
