% !TeX program = pdflatex
% !TeX root = TIDL.tex

\documentclass[../FeynCalcManual.tex]{subfiles}
\begin{document}
\hypertarget{tidl}{%
\section{TIDL}\label{tidl}}

TIDL is a database of tensorial reduction formulas.

\subsection{See also}

\hyperlink{toc}{Overview}, \hyperlink{tid}{TID}.

\subsection{Examples}

\begin{Shaded}
\begin{Highlighting}[]
\NormalTok{TIDL}\OperatorTok{[\{}\FunctionTok{q}\OperatorTok{,}\NormalTok{ mu}\OperatorTok{\},} \OperatorTok{\{}\FunctionTok{p}\OperatorTok{\}]}
\end{Highlighting}
\end{Shaded}

\begin{dmath*}\breakingcomma
\frac{p^{\text{mu}} (p\cdot q)}{p^2}
\end{dmath*}

\begin{Shaded}
\begin{Highlighting}[]
\NormalTok{TIDL}\OperatorTok{[\{}\FunctionTok{q}\OperatorTok{,}\NormalTok{ mu}\OperatorTok{\},} \OperatorTok{\{}\NormalTok{p1}\OperatorTok{,}\NormalTok{ p2}\OperatorTok{\}]}
\end{Highlighting}
\end{Shaded}

\begin{dmath*}\breakingcomma
\frac{\text{p1}^{\text{mu}} \left((\text{p1}\cdot \;\text{p2}) (\text{p2}\cdot q)-\text{p2}^2 (\text{p1}\cdot q)\right)}{(\text{p1}\cdot \;\text{p2})^2-\text{p1}^2 \;\text{p2}^2}-\frac{\text{p2}^{\text{mu}} \left(\text{p1}^2 (\text{p2}\cdot q)-(\text{p1}\cdot \;\text{p2}) (\text{p1}\cdot q)\right)}{(\text{p1}\cdot \;\text{p2})^2-\text{p1}^2 \;\text{p2}^2}
\end{dmath*}

\begin{Shaded}
\begin{Highlighting}[]
\NormalTok{TIDL}\OperatorTok{[\{\{}\NormalTok{q1}\OperatorTok{,}\NormalTok{ mu}\OperatorTok{\},} \OperatorTok{\{}\NormalTok{q2}\OperatorTok{,}\NormalTok{ nu}\OperatorTok{\}\},} \OperatorTok{\{}\FunctionTok{p}\OperatorTok{\}]}
\end{Highlighting}
\end{Shaded}

\begin{dmath*}\breakingcomma
\frac{g^{\text{mu}\;\text{nu}} \left((p\cdot \;\text{q1}) (p\cdot \;\text{q2})-p^2 (\text{q1}\cdot \;\text{q2})\right)}{(1-D) p^2}-\frac{p^{\text{mu}} p^{\text{nu}} \left(D (p\cdot \;\text{q1}) (p\cdot \;\text{q2})-p^2 (\text{q1}\cdot \;\text{q2})\right)}{(1-D) p^4}
\end{dmath*}

\begin{Shaded}
\begin{Highlighting}[]
\NormalTok{TIDL}\OperatorTok{[\{\{}\NormalTok{q1}\OperatorTok{,}\NormalTok{ mu}\OperatorTok{\},} \OperatorTok{\{}\NormalTok{q2}\OperatorTok{,}\NormalTok{ nu}\OperatorTok{\}\},} \OperatorTok{\{\}]}
\end{Highlighting}
\end{Shaded}

\begin{dmath*}\breakingcomma
\frac{g^{\text{mu}\;\text{nu}} (\text{q1}\cdot \;\text{q2})}{D}
\end{dmath*}
\end{document}
