% !TeX program = pdflatex
% !TeX root = Tdec.tex

\documentclass[../FeynCalcManual.tex]{subfiles}
\begin{document}
\hypertarget{tdec}{
\section{Tdec}\label{tdec}\index{Tdec}}

\texttt{Tdec[\allowbreak{}\{\allowbreak{}\{\allowbreak{}qi,\ \allowbreak{}mu\},\ \allowbreak{}\{\allowbreak{}qj,\ \allowbreak{}nu\},\ \allowbreak{}...\},\ \allowbreak{}\{\allowbreak{}p1,\ \allowbreak{}p2,\ \allowbreak{}...\}]}
calculates the tensorial decomposition formulas for Lorentzian
integrals. The more common ones are saved in \texttt{TIDL}.

The automatic symmetrization of the tensor basis is done using Alexey
Pak's algorithm described in
\href{https://arxiv.org/abs/1111.0868}{arXiv:1111.0868}.

\subsection{See also}

\hyperlink{toc}{Overview}, \hyperlink{tid}{TID}, \hyperlink{tidl}{TIDL},
\hyperlink{oneloopsimplify}{OneLoopSimplify}.

\subsection{Examples}

Check that
\(\int d^D f(p,q) q^{\mu}= \frac{p^{\mu}}{p^2} \int d^D f(p,q) p \cdot q\)

\begin{Shaded}
\begin{Highlighting}[]
\NormalTok{Tdec}\OperatorTok{[\{}\FunctionTok{q}\OperatorTok{,} \SpecialCharTok{\textbackslash{}}\OperatorTok{[}\NormalTok{Mu}\OperatorTok{]\},} \OperatorTok{\{}\FunctionTok{p}\OperatorTok{\}]} 
 
\SpecialCharTok{\%}\OperatorTok{[[}\DecValTok{2}\OperatorTok{]]} \OtherTok{/.} \SpecialCharTok{\%}\OperatorTok{[[}\DecValTok{1}\OperatorTok{]]}
\end{Highlighting}
\end{Shaded}

\begin{dmath*}\breakingcomma
\left\{\left\{\text{X1}\to p\cdot q,\text{X2}\to p^2\right\},\frac{\text{X1} p^{\mu }}{\text{X2}}\right\}
\end{dmath*}

\begin{dmath*}\breakingcomma
\frac{p^{\mu } (p\cdot q)}{p^2}
\end{dmath*}

This calculates integral transformation for any
\(\int d^D q_1 d^D q_2 d^D q_3\)
\(f(p,q_1,q_2,q_3) q_1^{\mu} q_2^{\nu}q_3^{\rho}\).

\begin{Shaded}
\begin{Highlighting}[]
\NormalTok{Tdec}\OperatorTok{[\{\{}\FunctionTok{Subscript}\OperatorTok{[}\FunctionTok{q}\OperatorTok{,} \DecValTok{1}\OperatorTok{],} \SpecialCharTok{\textbackslash{}}\OperatorTok{[}\NormalTok{Mu}\OperatorTok{]\},} \OperatorTok{\{}\FunctionTok{Subscript}\OperatorTok{[}\FunctionTok{q}\OperatorTok{,} \DecValTok{2}\OperatorTok{],} \SpecialCharTok{\textbackslash{}}\OperatorTok{[}\NormalTok{Nu}\OperatorTok{]\},} \OperatorTok{\{}\FunctionTok{Subscript}\OperatorTok{[}\FunctionTok{q}\OperatorTok{,} \DecValTok{3}\OperatorTok{],} \SpecialCharTok{\textbackslash{}}\OperatorTok{[}\NormalTok{Rho}\OperatorTok{]\}\},} \OperatorTok{\{}\FunctionTok{p}\OperatorTok{\},} \FunctionTok{List} \OtherTok{{-}\textgreater{}} \ConstantTok{False}\OperatorTok{]} 
 
\NormalTok{Contract}\OperatorTok{[}\SpecialCharTok{\%}\NormalTok{ FVD}\OperatorTok{[}\FunctionTok{p}\OperatorTok{,} \SpecialCharTok{\textbackslash{}}\OperatorTok{[}\NormalTok{Mu}\OperatorTok{]]}\NormalTok{ FVD}\OperatorTok{[}\FunctionTok{p}\OperatorTok{,} \SpecialCharTok{\textbackslash{}}\OperatorTok{[}\NormalTok{Nu}\OperatorTok{]]}\NormalTok{ FVD}\OperatorTok{[}\FunctionTok{p}\OperatorTok{,} \SpecialCharTok{\textbackslash{}}\OperatorTok{[}\NormalTok{Rho}\OperatorTok{]]]} \SpecialCharTok{//} \FunctionTok{Factor}
\end{Highlighting}
\end{Shaded}

\begin{dmath*}\breakingcomma
\frac{p^{\rho } g^{\mu \nu } \left(p\cdot q_3\right) \left(\left(p\cdot q_1\right) \left(p\cdot q_2\right)-p^2 \left(q_1\cdot q_2\right)\right)}{(1-D) p^4}+\frac{p^{\nu } g^{\mu \rho } \left(p\cdot q_2\right) \left(\left(p\cdot q_1\right) \left(p\cdot q_3\right)-p^2 \left(q_1\cdot q_3\right)\right)}{(1-D) p^4}+\frac{p^{\mu } g^{\nu \rho } \left(p\cdot q_1\right) \left(\left(p\cdot q_2\right) \left(p\cdot q_3\right)-p^2 \left(q_2\cdot q_3\right)\right)}{(1-D) p^4}-\frac{p^{\mu } p^{\nu } p^{\rho } \left(D \left(p\cdot q_1\right) \left(p\cdot q_2\right) \left(p\cdot q_3\right)+2 \left(p\cdot q_1\right) \left(p\cdot q_2\right) \left(p\cdot q_3\right)-p^2 \left(q_1\cdot q_2\right) \left(p\cdot q_3\right)-p^2 \left(q_1\cdot q_3\right) \left(p\cdot q_2\right)-p^2 \left(q_2\cdot q_3\right) \left(p\cdot q_1\right)\right)}{(1-D) p^6}
\end{dmath*}

\begin{dmath*}\breakingcomma
\left(p\cdot q_1\right) \left(p\cdot q_2\right) \left(p\cdot q_3\right)
\end{dmath*}
\end{document}
