% !TeX program = pdflatex
% !TeX root = TemporalMomentum.tex

\documentclass[../FeynCalcManual.tex]{subfiles}
\begin{document}
\hypertarget{temporalmomentum}{%
\section{TemporalMomentum}\label{temporalmomentum}}

\texttt{TemporalMomentum[\allowbreak{}p]} is the head of the temporal
component of a \(4\)-momentum \(p^0\). The internal representation of
the temporal component \(p^0\) is
\texttt{TemporalMomentum[\allowbreak{}p]}.

\texttt{TemporalMomentum} may appear only inside \texttt{TemporalPair}s.

\subsection{See also}

\hyperlink{toc}{Overview}, \hyperlink{temporalpair}{TemporalPair},
\hyperlink{explicitlorentzindex}{ExplicitLorentzIndex}.

\subsection{Examples}

\begin{Shaded}
\begin{Highlighting}[]
\NormalTok{TemporalMomentum}\OperatorTok{[}\FunctionTok{p}\OperatorTok{]}
\end{Highlighting}
\end{Shaded}

\begin{dmath*}\breakingcomma
p
\end{dmath*}

\begin{Shaded}
\begin{Highlighting}[]
\NormalTok{TemporalMomentum}\OperatorTok{[}\SpecialCharTok{{-}}\FunctionTok{q}\OperatorTok{]} \SpecialCharTok{//} \FunctionTok{StandardForm}

\CommentTok{(*{-}TemporalMomentum[q]*)}
\end{Highlighting}
\end{Shaded}

\begin{Shaded}
\begin{Highlighting}[]
\NormalTok{TemporalMomentum}\OperatorTok{[}\FunctionTok{p} \SpecialCharTok{+} \FunctionTok{q}\OperatorTok{]}
\end{Highlighting}
\end{Shaded}

\begin{dmath*}\breakingcomma
p+q
\end{dmath*}

\begin{Shaded}
\begin{Highlighting}[]
\NormalTok{TemporalMomentum}\OperatorTok{[}\FunctionTok{p} \SpecialCharTok{+} \FunctionTok{q}\OperatorTok{]} \SpecialCharTok{//}\NormalTok{ MomentumExpand }\SpecialCharTok{//} \FunctionTok{StandardForm}

\CommentTok{(*TemporalMomentum[p] + TemporalMomentum[q]*)}
\end{Highlighting}
\end{Shaded}

\begin{Shaded}
\begin{Highlighting}[]
\NormalTok{TemporalMomentum}\OperatorTok{[}\FunctionTok{p} \SpecialCharTok{+} \FunctionTok{q}\OperatorTok{]} \SpecialCharTok{//}\NormalTok{ MomentumExpand }\SpecialCharTok{//}\NormalTok{ MomentumCombine }\SpecialCharTok{//} \FunctionTok{StandardForm}

\CommentTok{(*TemporalMomentum[p + q]*)}
\end{Highlighting}
\end{Shaded}

\end{document}
