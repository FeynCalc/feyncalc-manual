% !TeX program = pdflatex
% !TeX root = TemporalPair.tex

\documentclass[../FeynCalcManual.tex]{subfiles}
\begin{document}
\hypertarget{temporalpair}{
\section{TemporalPair}\label{temporalpair}\index{TemporalPair}}

\texttt{TemporalPair[\allowbreak{}ExplicitLorentzIndex[\allowbreak{}0],\ \allowbreak{}TemporalMomentum[\allowbreak{}p]]}
is a special pairing used in the internal representation to denote
\(p^0\), the temporal component of a 4-momentum \(p\).

\subsection{See also}

\hyperlink{toc}{Overview}, \hyperlink{temporalpair}{TemporalPair},
\hyperlink{tc}{TC},
\hyperlink{explicitlorentzindex}{ExplicitLorentzIndex}.

\subsection{Examples}

\begin{Shaded}
\begin{Highlighting}[]
\NormalTok{TemporalPair}\OperatorTok{[}\NormalTok{ExplicitLorentzIndex}\OperatorTok{[}\DecValTok{0}\OperatorTok{],}\NormalTok{ TemporalMomentum}\OperatorTok{[}\FunctionTok{p}\OperatorTok{]]}
\end{Highlighting}
\end{Shaded}

\begin{dmath*}\breakingcomma
p^0
\end{dmath*}

\begin{Shaded}
\begin{Highlighting}[]
\NormalTok{TemporalPair}\OperatorTok{[}\NormalTok{ExplicitLorentzIndex}\OperatorTok{[}\DecValTok{0}\OperatorTok{],}\NormalTok{ TemporalMomentum}\OperatorTok{[}\FunctionTok{p} \SpecialCharTok{+} \FunctionTok{q}\OperatorTok{]]} 
 
\SpecialCharTok{\%} \SpecialCharTok{//}\NormalTok{ ExpandScalarProduct}
\end{Highlighting}
\end{Shaded}

\begin{dmath*}\breakingcomma
(p+q)^0
\end{dmath*}

\begin{dmath*}\breakingcomma
p^0+q^0
\end{dmath*}
\end{document}
