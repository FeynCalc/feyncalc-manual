% !TeX program = pdflatex
% !TeX root = TensorFunction.tex

\documentclass[../FeynCalcManual.tex]{subfiles}
\begin{document}
\hypertarget{tensorfunction}{
\section{TensorFunction}\label{tensorfunction}\index{TensorFunction}}

\texttt{TensorFunction[\allowbreak{}t,\ \allowbreak{}mu,\ \allowbreak{}nu,\ \allowbreak{}...]}
transform into
\texttt{t[\allowbreak{}LorentzIndex[\allowbreak{}mu],\ \allowbreak{}LorentzIndex[\allowbreak{}nu],\ \allowbreak{}...]},
i.e., it can be used as an unspecified tensorial function \texttt{t}.

A symmetric tensor can be obtained by
\texttt{TensorFunction[\allowbreak{}\{\allowbreak{}t,\ \allowbreak{}"S"\},\ \allowbreak{}mu,\ \allowbreak{}nu,\ \allowbreak{}...]},
and an antisymmetric one by
\texttt{TensorFunction[\allowbreak{}\{\allowbreak{}t,\ \allowbreak{}"A"\},\ \allowbreak{}mu,\ \allowbreak{}nu,\ \allowbreak{}...]}.

\subsection{See also}

\hyperlink{toc}{Overview}, \hyperlink{fcsymmetrize}{FCSymmetrize},
\hyperlink{fcantisymmetrize}{FCAntiSymmetrize}.

\subsection{Examples}

\begin{Shaded}
\begin{Highlighting}[]
\NormalTok{TensorFunction}\OperatorTok{[}\FunctionTok{t}\OperatorTok{,} \SpecialCharTok{\textbackslash{}}\OperatorTok{[}\NormalTok{Mu}\OperatorTok{],} \SpecialCharTok{\textbackslash{}}\OperatorTok{[}\NormalTok{Nu}\OperatorTok{],} \SpecialCharTok{\textbackslash{}}\OperatorTok{[}\NormalTok{Tau}\OperatorTok{]]}
\end{Highlighting}
\end{Shaded}

\begin{dmath*}\breakingcomma
t(\mu ,\nu ,\tau )
\end{dmath*}

\begin{Shaded}
\begin{Highlighting}[]
\NormalTok{TensorFunction}\OperatorTok{[}\FunctionTok{t}\OperatorTok{,} \SpecialCharTok{\textbackslash{}}\OperatorTok{[}\NormalTok{Mu}\OperatorTok{],} \SpecialCharTok{\textbackslash{}}\OperatorTok{[}\NormalTok{Nu}\OperatorTok{],} \SpecialCharTok{\textbackslash{}}\OperatorTok{[}\NormalTok{Tau}\OperatorTok{]]} \SpecialCharTok{//} \FunctionTok{StandardForm}

\CommentTok{(*t[LorentzIndex[\textbackslash{}[Mu]], LorentzIndex[\textbackslash{}[Nu]], LorentzIndex[\textbackslash{}[Tau]]]*)}
\end{Highlighting}
\end{Shaded}

\begin{Shaded}
\begin{Highlighting}[]
\NormalTok{Contract}\OperatorTok{[}\NormalTok{FV}\OperatorTok{[}\FunctionTok{p}\OperatorTok{,} \SpecialCharTok{\textbackslash{}}\OperatorTok{[}\NormalTok{Mu}\OperatorTok{]]}\NormalTok{ TensorFunction}\OperatorTok{[}\FunctionTok{t}\OperatorTok{,} \SpecialCharTok{\textbackslash{}}\OperatorTok{[}\NormalTok{Mu}\OperatorTok{],} \SpecialCharTok{\textbackslash{}}\OperatorTok{[}\NormalTok{Nu}\OperatorTok{],} \SpecialCharTok{\textbackslash{}}\OperatorTok{[}\NormalTok{Tau}\OperatorTok{]]]}
\end{Highlighting}
\end{Shaded}

\begin{dmath*}\breakingcomma
t\left(\overline{p},\nu ,\tau \right)
\end{dmath*}

\begin{Shaded}
\begin{Highlighting}[]
\NormalTok{Contract}\OperatorTok{[}\NormalTok{FV}\OperatorTok{[}\FunctionTok{p}\OperatorTok{,} \SpecialCharTok{\textbackslash{}}\OperatorTok{[}\NormalTok{Mu}\OperatorTok{]]}\NormalTok{ TensorFunction}\OperatorTok{[}\FunctionTok{t}\OperatorTok{,} \SpecialCharTok{\textbackslash{}}\OperatorTok{[}\NormalTok{Mu}\OperatorTok{],} \SpecialCharTok{\textbackslash{}}\OperatorTok{[}\NormalTok{Nu}\OperatorTok{],} \SpecialCharTok{\textbackslash{}}\OperatorTok{[}\NormalTok{Tau}\OperatorTok{]]]} \SpecialCharTok{//} \FunctionTok{StandardForm}

\CommentTok{(*t[Momentum[p], LorentzIndex[\textbackslash{}[Nu]], LorentzIndex[\textbackslash{}[Tau]]]*)}
\end{Highlighting}
\end{Shaded}

\begin{Shaded}
\begin{Highlighting}[]
\NormalTok{TensorFunction}\OperatorTok{[\{}\FunctionTok{f}\OperatorTok{,} \StringTok{"S"}\OperatorTok{\},} \SpecialCharTok{\textbackslash{}}\OperatorTok{[}\NormalTok{Alpha}\OperatorTok{],} \SpecialCharTok{\textbackslash{}}\OperatorTok{[}\FunctionTok{Beta}\OperatorTok{]]}
\end{Highlighting}
\end{Shaded}

\begin{dmath*}\breakingcomma
f(\alpha ,\beta )
\end{dmath*}

\begin{Shaded}
\begin{Highlighting}[]
\NormalTok{TensorFunction}\OperatorTok{[\{}\FunctionTok{f}\OperatorTok{,} \StringTok{"S"}\OperatorTok{\},} \SpecialCharTok{\textbackslash{}}\OperatorTok{[}\FunctionTok{Beta}\OperatorTok{],} \SpecialCharTok{\textbackslash{}}\OperatorTok{[}\NormalTok{Alpha}\OperatorTok{]]} \SpecialCharTok{//} \FunctionTok{StandardForm}

\CommentTok{(*f[LorentzIndex[\textbackslash{}[Alpha]], LorentzIndex[\textbackslash{}[Beta]]]*)}
\end{Highlighting}
\end{Shaded}

\begin{Shaded}
\begin{Highlighting}[]
\FunctionTok{Attributes}\OperatorTok{[}\FunctionTok{f}\OperatorTok{]}
\FunctionTok{ClearAttributes}\OperatorTok{[}\FunctionTok{f}\OperatorTok{,} \FunctionTok{Orderless}\OperatorTok{]}
\end{Highlighting}
\end{Shaded}

\begin{dmath*}\breakingcomma
\{\text{Orderless}\}
\end{dmath*}
\end{document}
