% !TeX program = pdflatex
% !TeX root = TensorReduction.tex

\documentclass[../FeynCalcManual.tex]{subfiles}
\begin{document}
\hypertarget{tensor reduction}{
\section{Tensor reduction}\label{tensor reduction}\index{Tensor reduction}}

\subsection{See also}

\hyperlink{toc}{Overview}.

\subsection{Brief description}\label{brief-description}

Loop tensor integrals that depend on loop momenta with open indices can
be converted to scalar integrals by means of tensor reduction. The same
also applies to cases where the loop momenta are contracted with Dirac
matrices, Levi-Civita tensors, polarization vectors or any other
\(4\)-vectors that are not contained in the propagators of the
respective loop integrals.

This technique is usually called
\href{https://doi.org/10.1016/0550-3213(79)90234-7\%7D}{Passarino-Veltman
reduction} or simply \emph{tensor reduction}. The main idea is to start
from the most generic ansatz for the tensor structures
(e.g.~\(g^{\mu \nu}\), \(p_1^\mu p_1^\nu\), \(p_2^\mu p_2^\nu\),
\(p_2^\mu p_1^\nu\) and \(p_1^\mu p_2^\nu\) for a 3-point rank 2
integral) that can appear in the final result and use it to write down a
tensor equation with unknown scalar coefficients \(S_i\) multiplying
those structures. The tensor structures may only contain metric tensors
and the external momenta the given loop integral depends on.

Then, contracting the resulting tensor equation with the present tensor
structures one can obtain a linear system of equations and solve it for
\(S_i\). The size of the linear system can be considerably simplified by
making use of the symmetries between different coefficients, e.g.~by
employing \href{https://arxiv.org/pdf/1111.0868}{Pak's algorithm}.

\subsection{Implementation}\label{implementation}

FeynCalc features several routines that implement this procedure. At 1
loop it is convenient to use \hyperlink{../tid}{../TID} that can reduce
tensor integrals with quadratic propagators to scalar integrals with
unit numerators. The result can be also presented in terms of the
so-called Passarino-Veltman coefficient (e.g \(B_{11}\), \(C_{00}\),
\(D_{1}\) etc.) or scalar functions (\(A_0\), \(B_0\), \(C_0\),
\(D_0\)).

Loop integrals with more generic propagators as well as integrals with
more loops can be treated using
\hyperlink{../fcmultilooptid}{../FCMultiLoopTID}. For such integrals the
reduction to unit numerators is not always possible. Therefore, the
function merely tries to get rid of loop momenta with open indices by
converting the original integral into a linear combination of scalar
integrals multiplying tensor structures made of metric tensors and
external momenta. The scalar integrals still may have numerators with
loop momenta contracted to other loop momenta or external momenta.

To reduce such expressions to a set of simpler integrals it is necessary
to use IBP-reduction tools that can be accessed e.g.~via the FeynHelpers
add-on.

Calculations that make use of the multiloop functionality introduced in
FeynCalc 10 require a somewhat different approach. To deal with
amplitudes written in terms of \hyperlink{../gli}{../GLI} as they are
usually generated by
\hyperlink{../fcloopfindtopologies}{../FCLoopFindTopologies}, one should
use the special function
\hyperlink{../fclooptensorreduce}{../FCLoopTensorReduce}.

Last but not least, sometimes one wants to calculate a tensor reduction
formula without referring to a particular integral, but rather just
specifying the tensor structure and the external momenta. This is useful
e.g.~when using such tools as
\href{https://github.com/vermaseren/form}{FORM}. To this aim one can use
the auxiliary routine \hyperlink{../tdec}{../Tdec}.

\subsection{Reduction for zero Gram
determinants}\label{reduction-for-zero-gram-determinants}

The naive Passarino-Veltman reduction breaks down when the Gram
determinant of the given set of external momenta vanishes. A zero
Gramian means that the external momenta are linearly dependent,
i.e.~there is a redundancy in the reduction. In this case the linear
system constructed out of these momenta is not solvable and it is not
possible to determine the coefficients \(S_i\).

One way to circumvent this issue consists of constructing a new set of
external momenta, where all vectors are linearly independent from each
other. Reducing the loop integral with respect to these new vectors
produces a solvable linear system which effectively resolves the
problem. Here one should distinguish between two possible cases.

\subsubsection{New basis using only available
momenta}\label{new-basis-using-only-available-momenta}

If the new set of linearly independent momenta is just a subset of the
old external momenta, then the reduction can be done straightforwardly.
The same goes for the subsequent IBP reduction. In particular, it is not
necessary to tell the IBP-reduction tool which external momenta are
linearly dependent.

\subsubsection{New basis containing auxiliary
momentum}\label{new-basis-containing-auxiliary-momentum}

Unfortunately, for some kinematic configurations the available external
momenta are not sufficient to produce a valid tensor basis. This usually
happens when almost all external momenta are light-like (i.e.~their
squares vanish). The simplest example for this case is the 2-point
function with \(p^2=0\), e.g.~

\begin{equation}
\int_k k^{\mu} f(k,p,m_i) = p^\mu A(p,m_i).
\end{equation}

Contracting this equation with \(p^{\mu}\) we get zero on the right hand
side, which does not allow us to extract the value of \(A(p,m_i)\).
Solving it for \(p^2 \neq 0\) also would not help, as the limit
\(p \to 0\) cannot be taken naively.

The workaround here (and for all similar cases) is to extend the tensor
basis by adding an auxiliary momentum (say \(n\)). For simplicity, it is
convenient to choose this auxiliary momentum to be light-like
(i.e.~\(n^2=0\)), but this condition is not strictly necessary. On the
other hand, it is very important to stress that the scalar products of
\(n\) with other external momenta (\(n \cdot p_i\)) must be
nonvanishing. The new equation

\begin{equation}
\int_k k^{\mu} f(k,p,m_i) = p^\mu \tilde{A}(p,m_i) + n^\mu \tilde{B}(p,m_i)
\end{equation}

leads to

\begin{equation}
\tilde{A}(p,m_i) = \frac{1}{n \cdot p} \int_k (n \cdot k) \, f(k,p,m_i), \\
\tilde{B}(p,m_i) = \frac{1}{n \cdot p} \int_k (p \cdot k) \, f(k,p,m_i),
\end{equation}

which formally solves the task of removing loop momenta with open
indices in the numerator.

Nevertheless, the results for the so-obtained reduction formulas are not
immediately usable in the existing calculation.

First of all, the scalar products of \(n\) with the loop momenta in the
numerator cannot be removed using partial fraction decomposition even at
1-loop order. This is because the propagators do not depend on \(n\), so
there is nothing we can cancel those scalar products against. Hence, to
get rid of \(n \cdot k_i\) we need to use IBP reduction. What is more,
to set up the reduction we need to augment the loop integral topologies
with propagators containing \(n\). Otherwise our propagator basis will
not be complete. In practice it is sufficient to use propagators of the
type \(1/(k_i \cdot n + i \eta)\) for each loop momentum \(k_i\). The
reduction also becomes more involved due to the presence of the
additional kinematic invariants \(n \cdot p_i\).

Second, even upon performing the IBP reduction the final result may
still seem to depend on \(n\). We know that this cannot be the case, as
\(n\) is an arbitrary auxiliary vector not present in the original
integral. Usually, upon exploiting the existing symmetries between
master integrals as well as other external vectors and scalar products
one can show that the \(n\)-dependence cancels as it should. However,
for complicated integrals with many kinematic invariants enforcing this
cancellation can be tricky.

\subsubsection{FeynCalc routines}\label{feyncalc-routines}

In FeynCalc, the tensor reduction routines (\hyperlink{../tid}{../TID},
\hyperlink{../fcmultilooptid}{../FCMultiLoopTID},
\hyperlink{../fclooptensorreduce}{../FCLoopTensorReduce}) will
automatically alert the user if the external momenta present in the
given loop integrals lead to a zero Gram determinant. The warning
message will also contain the exact sets of the problematic momenta.

Each of these sets should be passed to the special function
\hyperlink{../fcloopfindtensorbasis}{../FCLoopFindTensorBasis} as its
first argument. The second argument is meant for kinematic constraints,
in the case that they have not been already defined via
\texttt{SPD[\allowbreak{}...] = ...;}. Finally, the third argument
denotes an auxiliary vector that might become necessary in some
particular cases (see below).

The output
\hyperlink{../fcloopfindtensorbasis}{../FCLoopFindTensorBasis} will
contain a set of linearly independent external momenta and the linear
dependencies between the original momenta. The set of linearly
independent momenta should be passed to the tensor reduction routines
via the option \texttt{TensorReductionBasisChange} as follows

\begin{Shaded}
\begin{Highlighting}[]
\NormalTok{TensorReductionBasisChange }\OtherTok{{-}\textgreater{}} \OperatorTok{\{}\NormalTok{oldBasis1}\OtherTok{{-}\textgreater{}}\NormalTok{newBasis1}\OperatorTok{,}\NormalTok{oldBasis2}\OtherTok{{-}\textgreater{}}\NormalTok{newBasis2}\OperatorTok{,}\NormalTok{ ...}\OperatorTok{\}}
\end{Highlighting}
\end{Shaded}

If some of the new bases contain an auxiliary vector, it should be
specified via the option \texttt{AuxiliaryMomenta}, e.g.

\begin{Shaded}
\begin{Highlighting}[]
\NormalTok{AuxiliaryMomenta }\OtherTok{{-}\textgreater{}} \OperatorTok{\{}\FunctionTok{n}\OperatorTok{\}}
\end{Highlighting}
\end{Shaded}

Finally, when using \hyperlink{../tid}{../TID}, to ensure the reduction
to unit numerators (only for bases without the auxiliary vector) it can
be necessary to specify the relations between scalar products involving
loop momenta. Such relations follow from the output of
\hyperlink{../fcloopfindtensorbasis}{../FCLoopFindTensorBasis} and can
be passed via the option \texttt{FinalSubstitutions}. For example, for
two linearly dependent external momenta \(p_1\), \(p_2\)

\begin{Shaded}
\begin{Highlighting}[]
\NormalTok{FCLoopFindTensorBasis}\OperatorTok{[\{}\NormalTok{p1}\OperatorTok{,}\NormalTok{ p2}\OperatorTok{\},} \OperatorTok{\{}\NormalTok{SPD}\OperatorTok{[}\NormalTok{p1}\OperatorTok{]} \OtherTok{{-}\textgreater{}} \FunctionTok{s}\OperatorTok{,}\NormalTok{ SPD}\OperatorTok{[}\NormalTok{p2}\OperatorTok{]} \OtherTok{{-}\textgreater{}} \FunctionTok{s}\OperatorTok{,}\NormalTok{ SPD}\OperatorTok{[}\NormalTok{p1}\OperatorTok{,}\NormalTok{ p2}\OperatorTok{]} \OtherTok{{-}\textgreater{}} \FunctionTok{s}\OperatorTok{\},} \FunctionTok{n}\OperatorTok{]}
\end{Highlighting}
\end{Shaded}

the corresponding substitution reads

\begin{Shaded}
\begin{Highlighting}[]
\NormalTok{FinalSubstitutions }\OtherTok{{-}\textgreater{}} \OperatorTok{\{}\NormalTok{SPD}\OperatorTok{[}\FunctionTok{k}\OperatorTok{,}\NormalTok{ p2}\OperatorTok{]} \OtherTok{{-}\textgreater{}}\NormalTok{ SPD}\OperatorTok{[}\FunctionTok{k}\OperatorTok{,}\NormalTok{ p1}\OperatorTok{]\}}
\end{Highlighting}
\end{Shaded}

Tensor reduction results containing the auxiliary vector \(n\) need to
be processed further using IBPs. For few simple integrals obtained from
\hyperlink{../tid}{../TID} and
\hyperlink{../fcmultilooptid}{../FCMultiLoopTID} it might be easier to
do this via the FeynHelpers interface to the Mathematica version of
FIRE.

Large results as well as results originating from
\hyperlink{../fclooptensorreduce}{../FCLoopTensorReduce} will require a
more explicit treatment. To that aim we need to augment the topologies
of the affected integrals to contain propagators that depend on \(n\).
This can be done using
\hyperlink{../fcloopaugmenttopology}{../FCLoopAugmentTopology} where we
specify the topology in the first argument and the list of propagators
in the second one. Usually, it is sufficient to use an eikonal
propagator per each loop momentum, e.g.~something like

\begin{Shaded}
\begin{Highlighting}[]
\OperatorTok{\{}\NormalTok{newtopo}\OperatorTok{,}\NormalTok{ gliRule}\OperatorTok{\}} \ExtensionTok{=}\NormalTok{ FCLoopAugmentTopology}\OperatorTok{[}\NormalTok{oldtopo}\OperatorTok{,} \OperatorTok{\{}\NormalTok{SFAD}\OperatorTok{[\{\{}\DecValTok{0}\OperatorTok{,}\NormalTok{ k1.}\FunctionTok{n}\OperatorTok{\}\}],}\NormalTok{ SFAD}\OperatorTok{[\{\{}\DecValTok{0}\OperatorTok{,}\NormalTok{ k2.}\FunctionTok{n}\OperatorTok{\}\}],}\NormalTok{ ...}\OperatorTok{\}]}
\end{Highlighting}
\end{Shaded}

Then we can apply the GLI conversion rule \texttt{gliRule} to the output
of \texttt{FCLoopTensorReduce}

\begin{Shaded}
\begin{Highlighting}[]
\NormalTok{ampRed }\OtherTok{/.}\NormalTok{ gliRule}
\end{Highlighting}
\end{Shaded}

and use \texttt{newtopo} when setting up the IBP reduction.

\subsubsection{Examples}\label{examples}

In the following we provide several examples for tensor reduction of
loop integrals with zero Gram determinants. More examples and
explanations can be found on the reference pages for
\hyperlink{../tid}{../TID},
\hyperlink{../fcmultilooptid}{../FCMultiLoopTID} and
\hyperlink{../fclooptensorreduce}{../FCLoopTensorReduce}. Notice that
the so-obtained results can be possibly simplified further using IBPs.

\textbf{1-loop 1-point rank 2 integral with \(p^2=0\)}

\begin{Shaded}
\begin{Highlighting}[]
\NormalTok{FCClearScalarProducts}\OperatorTok{[]}
\NormalTok{SPD}\OperatorTok{[}\FunctionTok{p}\OperatorTok{]} \ExtensionTok{=} \DecValTok{0}\NormalTok{;}
\NormalTok{int }\ExtensionTok{=}\NormalTok{ FVD}\OperatorTok{[}\FunctionTok{k}\OperatorTok{,}\NormalTok{ mu}\OperatorTok{]}\NormalTok{ FVD}\OperatorTok{[}\FunctionTok{k}\OperatorTok{,}\NormalTok{ nu}\OperatorTok{]}\NormalTok{ FAD}\OperatorTok{[\{}\FunctionTok{k}\OperatorTok{,} \FunctionTok{m}\OperatorTok{\},} \OperatorTok{\{}\FunctionTok{k} \SpecialCharTok{{-}} \FunctionTok{p}\OperatorTok{\}]}
\CommentTok{(*TID[int,k];*)}
\CommentTok{(*FCLoopFindTensorBasis[\{{-}p\},\{\},n]*)}
\NormalTok{TID}\OperatorTok{[}\NormalTok{int}\OperatorTok{,} \FunctionTok{k}\OperatorTok{,}\NormalTok{ TensorReductionBasisChange }\OtherTok{{-}\textgreater{}} \OperatorTok{\{\{}\SpecialCharTok{{-}}\FunctionTok{p}\OperatorTok{\}} \OtherTok{{-}\textgreater{}} \OperatorTok{\{}\SpecialCharTok{{-}}\FunctionTok{p}\OperatorTok{,} \FunctionTok{n}\OperatorTok{\}\},}\NormalTok{ AuxiliaryMomenta }\OtherTok{{-}\textgreater{}} \OperatorTok{\{}\FunctionTok{n}\OperatorTok{\}]}
\end{Highlighting}
\end{Shaded}

\textbf{1-loop 2-point rank 2 integral with
\(p_1^2=p_2^2 = p_1 \cdot p_2 = s\)}

\begin{Shaded}
\begin{Highlighting}[]
\NormalTok{FCClearScalarProducts}\OperatorTok{[]}\NormalTok{;}
\NormalTok{SPD}\OperatorTok{[}\NormalTok{p1}\OperatorTok{]} \ExtensionTok{=} \FunctionTok{s}\NormalTok{;}
\NormalTok{SPD}\OperatorTok{[}\NormalTok{p2}\OperatorTok{]} \ExtensionTok{=} \FunctionTok{s}\NormalTok{;}
\NormalTok{SPD}\OperatorTok{[}\NormalTok{p1}\OperatorTok{,}\NormalTok{ p2}\OperatorTok{]} \ExtensionTok{=} \FunctionTok{s}\NormalTok{;}
\NormalTok{int }\ExtensionTok{=}\NormalTok{ FVD}\OperatorTok{[}\FunctionTok{k}\OperatorTok{,}\NormalTok{ mu}\OperatorTok{]}\NormalTok{ FVD}\OperatorTok{[}\FunctionTok{k}\OperatorTok{,}\NormalTok{ nu}\OperatorTok{]}\NormalTok{ FAD}\OperatorTok{[\{}\FunctionTok{k}\OperatorTok{,} \DecValTok{0}\OperatorTok{\},} \OperatorTok{\{}\FunctionTok{k} \SpecialCharTok{+}\NormalTok{ p1}\OperatorTok{,}\NormalTok{ m1}\OperatorTok{\},} \OperatorTok{\{}\FunctionTok{k} \SpecialCharTok{+}\NormalTok{ p2}\OperatorTok{,}\NormalTok{ m2}\OperatorTok{\}]}
\CommentTok{(*TID[int,k];*)}
\CommentTok{(*FCLoopFindTensorBasis[\{{-}p1, {-}p2\}, \{\}, n]*)}
\NormalTok{TID}\OperatorTok{[}\NormalTok{int}\OperatorTok{,} \FunctionTok{k}\OperatorTok{,}\NormalTok{TensorReductionBasisChange }\OtherTok{{-}\textgreater{}} \OperatorTok{\{\{}\SpecialCharTok{{-}}\NormalTok{p1}\OperatorTok{,} \SpecialCharTok{{-}}\NormalTok{p2}\OperatorTok{\}} \OtherTok{{-}\textgreater{}} \OperatorTok{\{}\SpecialCharTok{{-}}\NormalTok{p1}\OperatorTok{\}\},}\NormalTok{ FinalSubstitutions }\OtherTok{{-}\textgreater{}} \OperatorTok{\{}\NormalTok{SPD}\OperatorTok{[}\FunctionTok{k}\OperatorTok{,}\NormalTok{ p2}\OperatorTok{]} \OtherTok{{-}\textgreater{}}\NormalTok{ SPD}\OperatorTok{[}\FunctionTok{k}\OperatorTok{,}\NormalTok{ p1}\OperatorTok{]\}]}
\end{Highlighting}
\end{Shaded}

\textbf{1-loop 2-point rank 1 integral with
\(p_1^2=p_2^2 = p_1 \cdot p_2 = 0\)}

\begin{Shaded}
\begin{Highlighting}[]
\NormalTok{FCClearScalarProducts}\OperatorTok{[]}\NormalTok{;}
\NormalTok{SPD}\OperatorTok{[}\NormalTok{p1}\OperatorTok{]} \ExtensionTok{=} \DecValTok{0}\NormalTok{;}
\NormalTok{SPD}\OperatorTok{[}\NormalTok{p2}\OperatorTok{]} \ExtensionTok{=} \DecValTok{0}\NormalTok{;}
\NormalTok{SPD}\OperatorTok{[}\NormalTok{p1}\OperatorTok{,}\NormalTok{ p2}\OperatorTok{]} \ExtensionTok{=} \DecValTok{0}\NormalTok{;}
\NormalTok{SPD}\OperatorTok{[}\FunctionTok{n}\OperatorTok{]} \ExtensionTok{=} \DecValTok{0}\NormalTok{;}
\NormalTok{int }\ExtensionTok{=}\NormalTok{ FVD}\OperatorTok{[}\FunctionTok{k}\OperatorTok{,}\NormalTok{ mu}\OperatorTok{]}\NormalTok{ FVD}\OperatorTok{[}\FunctionTok{k}\OperatorTok{,}\NormalTok{ nu}\OperatorTok{]}\NormalTok{ FAD}\OperatorTok{[\{}\FunctionTok{k}\OperatorTok{,} \DecValTok{0}\OperatorTok{\},} \OperatorTok{\{}\FunctionTok{k} \SpecialCharTok{+}\NormalTok{ p1}\OperatorTok{,}\NormalTok{ m1}\OperatorTok{\},} \OperatorTok{\{}\FunctionTok{k} \SpecialCharTok{+}\NormalTok{ p2}\OperatorTok{,}\NormalTok{ m2}\OperatorTok{\}]}
\NormalTok{TID}\OperatorTok{[}\NormalTok{int}\OperatorTok{,} \FunctionTok{k}\OperatorTok{,} 
\NormalTok{ TensorReductionBasisChange }\OtherTok{{-}\textgreater{}} \OperatorTok{\{\{}\SpecialCharTok{{-}}\NormalTok{p1}\OperatorTok{,} \SpecialCharTok{{-}}\NormalTok{p2}\OperatorTok{\}} \OtherTok{{-}\textgreater{}} \OperatorTok{\{}\FunctionTok{n}\OperatorTok{,} \SpecialCharTok{{-}}\NormalTok{p1}\OperatorTok{\},} \OperatorTok{\{}\SpecialCharTok{{-}}\NormalTok{p2}\OperatorTok{\}} \OtherTok{{-}\textgreater{}} \OperatorTok{\{}\FunctionTok{n}\OperatorTok{,} \SpecialCharTok{{-}}\NormalTok{p2}\OperatorTok{\},} \OperatorTok{\{}\SpecialCharTok{{-}}\NormalTok{p1}\OperatorTok{\}} \OtherTok{{-}\textgreater{}} \OperatorTok{\{}\FunctionTok{n}\OperatorTok{,} \SpecialCharTok{{-}}\NormalTok{p1}\OperatorTok{\},} 
\OperatorTok{\{}\NormalTok{p2 }\SpecialCharTok{{-}}\NormalTok{ p1}\OperatorTok{\}} \OtherTok{{-}\textgreater{}} \OperatorTok{\{}\FunctionTok{n}\OperatorTok{,}\NormalTok{ p2 }\SpecialCharTok{{-}}\NormalTok{ p1}\OperatorTok{\}\},}\NormalTok{ AuxiliaryMomenta }\OtherTok{{-}\textgreater{}} \OperatorTok{\{}\FunctionTok{n}\OperatorTok{\}]}
\end{Highlighting}
\end{Shaded}

\end{document}
