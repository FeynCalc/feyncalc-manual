% !TeX program = pdflatex
% !TeX root = ToFCPartialFractionForm.tex

\documentclass[../FeynCalcManual.tex]{subfiles}
\begin{document}
\begin{Shaded}
\begin{Highlighting}[]
 
\end{Highlighting}
\end{Shaded}

\hypertarget{tofcpartialfractionform}{
\section{ToFCPartialFractionForm}\label{tofcpartialfractionform}\index{ToFCPartialFractionForm}}

\texttt{ToFCPartialFractionForm[\allowbreak{}exp,\ \allowbreak{}x]}
converts sums of rational functions of the form
\(n + \frac{f_1}{[x-r_1]^p_1} + \frac{f_2}{[x-r_2]^p_2} + \ldots\) to
\texttt{FCPartialFractionForm[\allowbreak{}n,\ \allowbreak{}\{\allowbreak{}\{\allowbreak{}f1,\ \allowbreak{}x-r1,\ \allowbreak{}p1\},\ \allowbreak{}\{\allowbreak{}f2,\ \allowbreak{}x-r2,\ \allowbreak{}p2\},\ \allowbreak{}...\},\ \allowbreak{}x]}.

This facilitates the handling of iterated integrals.

\subsection{See also}

\hyperlink{toc}{Overview},
\hyperlink{fcpartialfractionform}{FCPartialFractionForm},
\hyperlink{fromfcpartialfractionform}{FromFCPartialFractionForm}.

\subsection{Examples}

\begin{Shaded}
\begin{Highlighting}[]
\FunctionTok{x}\SpecialCharTok{/}\NormalTok{(}\FunctionTok{x} \SpecialCharTok{+} \DecValTok{1}\NormalTok{) }
 
\NormalTok{ToFCPartialFractionForm}\OperatorTok{[}\SpecialCharTok{\%}\OperatorTok{,} \FunctionTok{x}\OperatorTok{]}
\end{Highlighting}
\end{Shaded}

\begin{dmath*}\breakingcomma
\frac{x}{x+1}
\end{dmath*}

\begin{dmath*}\breakingcomma
\text{FCPartialFractionForm}\left(1,\left(
\begin{array}{cc}
 \{x+1,-1\} & -1 \\
\end{array}
\right),x\right)
\end{dmath*}

\begin{Shaded}
\begin{Highlighting}[]
\DecValTok{1}\SpecialCharTok{/}\NormalTok{(}\FunctionTok{x}\SpecialCharTok{\^{}}\DecValTok{2} \SpecialCharTok{+} \DecValTok{3}\NormalTok{) }
 
\NormalTok{ToFCPartialFractionForm}\OperatorTok{[}\SpecialCharTok{\%}\OperatorTok{,} \FunctionTok{x}\OperatorTok{]}
\end{Highlighting}
\end{Shaded}

\begin{dmath*}\breakingcomma
\frac{1}{x^2+3}
\end{dmath*}

\begin{dmath*}\breakingcomma
\text{FCPartialFractionForm}\left(0,\left(
\begin{array}{cc}
 \left\{x-i \sqrt{3},-1\right\} & -\frac{i}{2 \sqrt{3}} \\
 \left\{x+i \sqrt{3},-1\right\} & \frac{i}{2 \sqrt{3}} \\
\end{array}
\right),x\right)
\end{dmath*}

\begin{Shaded}
\begin{Highlighting}[]
\NormalTok{(}\SpecialCharTok{{-}}\DecValTok{64}\SpecialCharTok{*}\NormalTok{(}\SpecialCharTok{{-}}\DecValTok{1} \SpecialCharTok{+} \FunctionTok{z}\SpecialCharTok{\^{}}\DecValTok{2}\NormalTok{))}\SpecialCharTok{/}\NormalTok{(}\DecValTok{15}\SpecialCharTok{*}\NormalTok{(}\DecValTok{1} \SpecialCharTok{+} \FunctionTok{z}\SpecialCharTok{\^{}}\DecValTok{2} \SpecialCharTok{+} \FunctionTok{z}\SpecialCharTok{\^{}}\DecValTok{4}\NormalTok{)) }
 
\NormalTok{ToFCPartialFractionForm}\OperatorTok{[}\SpecialCharTok{\%}\OperatorTok{,} \FunctionTok{z}\OperatorTok{]}
\end{Highlighting}
\end{Shaded}

\begin{dmath*}\breakingcomma
-\frac{64 \left(z^2-1\right)}{15 \left(z^4+z^2+1\right)}
\end{dmath*}

\begin{dmath*}\breakingcomma
\text{FCPartialFractionForm}\left(0,\left(
\begin{array}{cc}
 \left\{z-\sqrt[3]{-1},-1\right\} & -\frac{32}{15} \\
 \left\{z+\sqrt[3]{-1},-1\right\} & \frac{32}{15} \\
 \left\{z-(-1)^{2/3},-1\right\} & \frac{32}{15} \\
 \left\{z+(-1)^{2/3},-1\right\} & -\frac{32}{15} \\
\end{array}
\right),z\right)
\end{dmath*}
\end{document}
