% !TeX program = pdflatex
% !TeX root = ToGFAD.tex

\documentclass[../FeynCalcManual.tex]{subfiles}
\begin{document}
\hypertarget{togfad}{
\section{ToGFAD}\label{togfad}\index{ToGFAD}}

\texttt{ToGFAD[\allowbreak{}exp]} converts all occurring propagator
types (\texttt{FAD}, \texttt{SFAD}, \texttt{CFAD}) to \texttt{GFAD}s.
This is mainly useful when doing expansions in kinematic invariants,
where e.g.~scalar products may not be appear explicitly when using
\texttt{FAD}- or \texttt{SFAD}-notation.

\texttt{ToGFAD} is the inverse operation to \texttt{FromGFAD}.

Using the option ``OnlyMixedQuadraticEikonalPropagators'' one can limit
the conversion to a particular type of standard and Cartesian propagator
denominators that contain both quadratic and eikonal pieces. Those are
the ones that usually cause issues when doing topology minimization

\subsection{See also}

\hyperlink{toc}{Overview}, \hyperlink{gfad}{GFAD},
\hyperlink{sfad}{SFAD}, \hyperlink{cfad}{CFAD},
\hyperlink{feynampdenominatorexplicit}{FeynAmpDenominatorExplicit},
\hyperlink{fromgfad}{FromGFAD}

\subsection{Examples}

\begin{Shaded}
\begin{Highlighting}[]
\NormalTok{ToGFAD}\OperatorTok{[}\NormalTok{FAD}\OperatorTok{[}\FunctionTok{p}\OperatorTok{]]}
\end{Highlighting}
\end{Shaded}

\begin{dmath*}\breakingcomma
\frac{1}{(p^2+i \eta )}
\end{dmath*}

\begin{Shaded}
\begin{Highlighting}[]
\NormalTok{ToGFAD}\OperatorTok{[}\NormalTok{FAD}\OperatorTok{[}\FunctionTok{p}\OperatorTok{]]} \SpecialCharTok{//} \FunctionTok{StandardForm}

\CommentTok{(*FeynAmpDenominator[GenericPropagatorDenominator[Pair[Momentum[p, D], Momentum[p, D]], \{1, 1\}]]*)}
\end{Highlighting}
\end{Shaded}

\begin{Shaded}
\begin{Highlighting}[]
\NormalTok{ToGFAD}\OperatorTok{[}\NormalTok{SFAD}\OperatorTok{[\{}\FunctionTok{p} \SpecialCharTok{+} \FunctionTok{q}\OperatorTok{,} \FunctionTok{m}\SpecialCharTok{\^{}}\DecValTok{2}\OperatorTok{\}]]}
\end{Highlighting}
\end{Shaded}

\begin{dmath*}\breakingcomma
\frac{1}{(-m^2+p^2+2 (p\cdot q)+q^2+i \eta )}
\end{dmath*}

\begin{Shaded}
\begin{Highlighting}[]
\NormalTok{ToGFAD}\OperatorTok{[}\NormalTok{SFAD}\OperatorTok{[\{}\FunctionTok{p} \SpecialCharTok{+} \FunctionTok{q}\OperatorTok{,} \FunctionTok{m}\SpecialCharTok{\^{}}\DecValTok{2}\OperatorTok{\}]]} \SpecialCharTok{//} \FunctionTok{StandardForm}

\CommentTok{(*FeynAmpDenominator[GenericPropagatorDenominator[{-}m\^{}2 + Pair[Momentum[p, D], Momentum[p, D]] + 2 Pair[Momentum[p, D], Momentum[q, D]] + Pair[Momentum[q, D], Momentum[q, D]], \{1, 1\}]]*)}
\end{Highlighting}
\end{Shaded}

\begin{Shaded}
\begin{Highlighting}[]
\NormalTok{ToGFAD}\OperatorTok{[}\NormalTok{SFAD}\OperatorTok{[\{}\FunctionTok{p} \SpecialCharTok{+} \FunctionTok{q}\OperatorTok{,} \FunctionTok{m}\SpecialCharTok{\^{}}\DecValTok{2}\OperatorTok{\}],}\NormalTok{ FinalSubstitutions }\OtherTok{{-}\textgreater{}} \OperatorTok{\{}\NormalTok{SPD}\OperatorTok{[}\FunctionTok{q}\OperatorTok{]} \OtherTok{{-}\textgreater{}} \DecValTok{0}\OperatorTok{\}]}
\end{Highlighting}
\end{Shaded}

\begin{dmath*}\breakingcomma
\frac{1}{(-m^2+p^2+2 (p\cdot q)+i \eta )}
\end{dmath*}

\begin{Shaded}
\begin{Highlighting}[]
\NormalTok{ToGFAD}\OperatorTok{[}\NormalTok{SFAD}\OperatorTok{[\{}\FunctionTok{p} \SpecialCharTok{+} \FunctionTok{q}\OperatorTok{,} \FunctionTok{m}\SpecialCharTok{\^{}}\DecValTok{2}\OperatorTok{\}],}\NormalTok{ FinalSubstitutions }\OtherTok{{-}\textgreater{}} \OperatorTok{\{}\NormalTok{SPD}\OperatorTok{[}\FunctionTok{q}\OperatorTok{]} \OtherTok{{-}\textgreater{}} \DecValTok{0}\OperatorTok{\}]} \SpecialCharTok{//} \FunctionTok{StandardForm}

\CommentTok{(*FeynAmpDenominator[GenericPropagatorDenominator[{-}m\^{}2 + Pair[Momentum[p, D], Momentum[p, D]] + 2 Pair[Momentum[p, D], Momentum[q, D]], \{1, 1\}]]*)}
\end{Highlighting}
\end{Shaded}

This is not a mixed quadratic-eikonal propagator so it remains unchanged

\begin{Shaded}
\begin{Highlighting}[]
\NormalTok{ToGFAD}\OperatorTok{[}\NormalTok{SFAD}\OperatorTok{[\{\{}\NormalTok{k2}\OperatorTok{,} \DecValTok{0}\OperatorTok{\},} \OperatorTok{\{}\DecValTok{0}\OperatorTok{,} \DecValTok{1}\OperatorTok{\},} \DecValTok{1}\OperatorTok{\}],} \StringTok{"OnlyMixedQuadraticEikonalPropagators"} \OtherTok{{-}\textgreater{}} \ConstantTok{True}\OperatorTok{,} 
\NormalTok{   FCE }\OtherTok{{-}\textgreater{}} \ConstantTok{True}\OperatorTok{]} \SpecialCharTok{//} \FunctionTok{StandardForm}

\CommentTok{(*SFAD[\{\{k2, 0\}, \{0, 1\}, 1\}]*)}
\end{Highlighting}
\end{Shaded}

This is a mixed propagator that will be converted to a \texttt{GFAD}

\begin{Shaded}
\begin{Highlighting}[]
\NormalTok{ToGFAD}\OperatorTok{[}\NormalTok{SFAD}\OperatorTok{[\{\{}\NormalTok{k1}\OperatorTok{,} \DecValTok{2}\NormalTok{ gkin meta k1 . }\FunctionTok{n} \SpecialCharTok{{-}} \DecValTok{2}\NormalTok{ gkin meta u0b k1 . }\FunctionTok{n} \SpecialCharTok{{-}}\NormalTok{ meta u0b k1 . nb}\OperatorTok{\},} 
     \OperatorTok{\{}\DecValTok{2}\NormalTok{ gkin meta}\SpecialCharTok{\^{}}\DecValTok{2}\NormalTok{ u0b }\SpecialCharTok{{-}} \DecValTok{2}\NormalTok{ gkin meta}\SpecialCharTok{\^{}}\DecValTok{2}\NormalTok{ u0b}\SpecialCharTok{\^{}}\DecValTok{2}\OperatorTok{,} \DecValTok{1}\OperatorTok{\},} \DecValTok{1}\OperatorTok{\}],} 
   \StringTok{"OnlyMixedQuadraticEikonalPropagators"} \OtherTok{{-}\textgreater{}} \ConstantTok{True}\OperatorTok{,}\NormalTok{ FCE }\OtherTok{{-}\textgreater{}} \ConstantTok{True}\OperatorTok{]} \SpecialCharTok{//} \FunctionTok{StandardForm}

\CommentTok{(*GFAD[\{\{{-}2 gkin meta\^{}2 u0b + 2 gkin meta\^{}2 u0b\^{}2 + SPD[k1, k1] + 2 gkin meta SPD[k1, n] {-} 2 gkin meta u0b SPD[k1, n] {-} meta u0b SPD[k1, nb], 1\}, 1\}]*)}
\end{Highlighting}
\end{Shaded}

\end{document}
