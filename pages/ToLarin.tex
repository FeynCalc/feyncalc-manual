% !TeX program = pdflatex
% !TeX root = ToLarin.tex

\documentclass[../FeynCalcManual.tex]{subfiles}
\begin{document}
\hypertarget{tolarin}{%
\section{ToLarin}\label{tolarin}}

ToLarin{[}exp{]} substitutes \(\gamma^{\mu} \gamma^5\) with
\(-\frac{I}{6}\varepsilon^{\mu \nu \lambda \sigma } \gamma^{\nu } \gamma^{\lambda} \gamma^{\sigma }\).

\subsection{See also}

\hyperlink{toc}{Overview}, \hyperlink{eps}{Eps},
\hyperlink{diracgamma}{DiracGamma}.

\subsection{Examples}

\begin{Shaded}
\begin{Highlighting}[]
\NormalTok{GAD}\OperatorTok{[}\SpecialCharTok{\textbackslash{}}\OperatorTok{[}\NormalTok{Mu}\OperatorTok{],} \SpecialCharTok{\textbackslash{}}\OperatorTok{[}\NormalTok{Nu}\OperatorTok{]]}\NormalTok{ . GA}\OperatorTok{[}\DecValTok{5}\OperatorTok{]} 
 
\NormalTok{ToLarin}\OperatorTok{[}\SpecialCharTok{\%}\OperatorTok{]}
\end{Highlighting}
\end{Shaded}

\begin{dmath*}\breakingcomma
\gamma ^{\mu }.\gamma ^{\nu }.\bar{\gamma }^5
\end{dmath*}

\begin{dmath*}\breakingcomma
-\frac{1}{6} i \gamma ^{\mu }.\gamma ^{\text{du19}}.\gamma ^{\text{du20}}.\gamma ^{\text{du21}} \overset{\text{}}{\epsilon }^{\nu \;\text{du19}\;\text{du20}\;\text{du21}}
\end{dmath*}
\end{document}
