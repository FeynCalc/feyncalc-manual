% !TeX program = pdflatex
% !TeX root = ToLightConeComponents.tex

\documentclass[../FeynCalcManual.tex]{subfiles}
\begin{document}
\hypertarget{tolightconecomponents}{
\section{ToLightConeComponents}\label{tolightconecomponents}\index{ToLightConeComponents}}

\texttt{ToLightConeComponents[\allowbreak{}expr,\ \allowbreak{}n,\ \allowbreak{}nb]}
rewrites all Dirac matrices, scalar products, 4-vectors and metric
tensors in terms of their component along the lightcone directions
\texttt{n} and \texttt{nb}

Using the option \texttt{NotMomentum} one can specify that quantities
containing the listed 4-momenta should be left untouched.

\subsection{See also}

\hyperlink{toc}{Overview},
\hyperlink{expandscalarproduct}{ExpandScalarProduct},
\hyperlink{lightconeperpendicularcomponent}{LightConePerpendicularComponent}.

\subsection{Examples}

\begin{Shaded}
\begin{Highlighting}[]
\NormalTok{ToLightConeComponents}\OperatorTok{[}\NormalTok{SP}\OperatorTok{[}\FunctionTok{a}\OperatorTok{,} \FunctionTok{b}\OperatorTok{],} \FunctionTok{n}\OperatorTok{,}\NormalTok{ nb}\OperatorTok{]}
\end{Highlighting}
\end{Shaded}

\begin{dmath*}\breakingcomma
\frac{1}{2} \left(\overline{a}\cdot \overline{\text{nb}}\right) \left(\overline{b}\cdot \overline{n}\right)+\frac{1}{2} \left(\overline{a}\cdot \overline{n}\right) \left(\overline{b}\cdot \overline{\text{nb}}\right)+\overline{a}\cdot \overline{b}_{\perp }
\end{dmath*}

\begin{Shaded}
\begin{Highlighting}[]
\NormalTok{ToLightConeComponents}\OperatorTok{[}\NormalTok{FV}\OperatorTok{[}\FunctionTok{p}\OperatorTok{,} \SpecialCharTok{\textbackslash{}}\OperatorTok{[}\NormalTok{Mu}\OperatorTok{]],} \FunctionTok{n}\OperatorTok{,}\NormalTok{ nb}\OperatorTok{]}
\end{Highlighting}
\end{Shaded}

\begin{dmath*}\breakingcomma
\frac{1}{2} \overline{\text{nb}}^{\mu } \left(\overline{n}\cdot \overline{p}\right)+\frac{1}{2} \overline{n}^{\mu } \left(\overline{\text{nb}}\cdot \overline{p}\right)+\overline{p}^{\mu }{}_{\perp }
\end{dmath*}

\begin{Shaded}
\begin{Highlighting}[]
\NormalTok{ToLightConeComponents}\OperatorTok{[}\NormalTok{GA}\OperatorTok{[}\SpecialCharTok{\textbackslash{}}\OperatorTok{[}\NormalTok{Mu}\OperatorTok{]],} \FunctionTok{n}\OperatorTok{,}\NormalTok{ nb}\OperatorTok{]}
\end{Highlighting}
\end{Shaded}

\begin{dmath*}\breakingcomma
\bar{\gamma }^{\mu }{}_{\perp }+\frac{1}{2} \overline{n}^{\mu } \bar{\gamma }\cdot \overline{\text{nb}}+\frac{1}{2} \overline{\text{nb}}^{\mu } \bar{\gamma }\cdot \overline{n}
\end{dmath*}

\begin{Shaded}
\begin{Highlighting}[]
\NormalTok{ToLightConeComponents}\OperatorTok{[}\NormalTok{GS}\OperatorTok{[}\FunctionTok{p}\OperatorTok{],} \FunctionTok{n}\OperatorTok{,}\NormalTok{ nb}\OperatorTok{]}
\end{Highlighting}
\end{Shaded}

\begin{dmath*}\breakingcomma
\frac{1}{2} \left(\overline{n}\cdot \overline{p}\right) \bar{\gamma }\cdot \overline{\text{nb}}+\frac{1}{2} \bar{\gamma }\cdot \overline{n} \left(\overline{\text{nb}}\cdot \overline{p}\right)+\bar{\gamma }\cdot \overline{p}_{\perp }
\end{dmath*}
\end{document}
