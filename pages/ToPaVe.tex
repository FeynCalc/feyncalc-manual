% !TeX program = pdflatex
% !TeX root = ToPaVe.tex

\documentclass[../FeynCalcManual.tex]{subfiles}
\begin{document}
\hypertarget{topave}{%
\section{ToPaVe}\label{topave}}

\texttt{ToPaVe[\allowbreak{}exp,\ \allowbreak{}q]} converts all scalar
1-loop integrals in \texttt{exp} that depend on the momentum \texttt{q}
to scalar Passarino Veltman functions \texttt{A0}, \texttt{B0},
\texttt{C0}, \texttt{D0} etc.

\subsection{See also}

\hyperlink{toc}{Overview}, \hyperlink{pavetoabcd}{PaVeToABCD},
\hyperlink{topave2}{ToPaVe2}, \hyperlink{a0}{A0}, \hyperlink{a00}{A00},
\hyperlink{b0}{B0}, \hyperlink{b1}{B1}, \hyperlink{b00}{B00},
\hyperlink{b11}{B11}, \hyperlink{c0}{C0}, \hyperlink{d0}{D0}.

\subsection{Examples}

\begin{Shaded}
\begin{Highlighting}[]
\NormalTok{FAD}\OperatorTok{[\{}\FunctionTok{q}\OperatorTok{,}\NormalTok{ m1}\OperatorTok{\}]} 
 
\NormalTok{ToPaVe}\OperatorTok{[}\SpecialCharTok{\%}\OperatorTok{,} \FunctionTok{q}\OperatorTok{]}
\end{Highlighting}
\end{Shaded}

\begin{dmath*}\breakingcomma
\frac{1}{q^2-\text{m1}^2}
\end{dmath*}

\begin{dmath*}\breakingcomma
i \pi ^2 \;\text{A}_0\left(\text{m1}^2\right)
\end{dmath*}

\begin{Shaded}
\begin{Highlighting}[]
\NormalTok{FAD}\OperatorTok{[\{}\FunctionTok{q}\OperatorTok{,}\NormalTok{ m1}\OperatorTok{\},} \OperatorTok{\{}\FunctionTok{q} \SpecialCharTok{+}\NormalTok{ p1}\OperatorTok{,}\NormalTok{ m2}\OperatorTok{\}]} 
 
\NormalTok{ToPaVe}\OperatorTok{[}\SpecialCharTok{\%}\OperatorTok{,} \FunctionTok{q}\OperatorTok{]}
\end{Highlighting}
\end{Shaded}

\begin{dmath*}\breakingcomma
\frac{1}{\left(q^2-\text{m1}^2\right).\left((\text{p1}+q)^2-\text{m2}^2\right)}
\end{dmath*}

\begin{dmath*}\breakingcomma
i \pi ^2 \;\text{B}_0\left(\text{p1}^2,\text{m1}^2,\text{m2}^2\right)
\end{dmath*}

\begin{Shaded}
\begin{Highlighting}[]
\SpecialCharTok{\%} \SpecialCharTok{//} \FunctionTok{StandardForm}

\CommentTok{(*I \textbackslash{}[Pi]\^{}2 B0[Pair[Momentum[p1, D], Momentum[p1, D]], m1\^{}2, m2\^{}2]*)}
\end{Highlighting}
\end{Shaded}

\begin{Shaded}
\begin{Highlighting}[]
\NormalTok{FAD}\OperatorTok{[\{}\FunctionTok{q}\OperatorTok{,}\NormalTok{ m1}\OperatorTok{\},} \OperatorTok{\{}\FunctionTok{q} \SpecialCharTok{+}\NormalTok{ p1}\OperatorTok{,}\NormalTok{ m2}\OperatorTok{\},} \OperatorTok{\{}\FunctionTok{q} \SpecialCharTok{+}\NormalTok{ p2}\OperatorTok{,}\NormalTok{ m3}\OperatorTok{\},} \OperatorTok{\{}\FunctionTok{q} \SpecialCharTok{+}\NormalTok{ p3}\OperatorTok{,}\NormalTok{ m4}\OperatorTok{\},} \OperatorTok{\{}\FunctionTok{q} \SpecialCharTok{+}\NormalTok{ p4}\OperatorTok{,}\NormalTok{ m5}\OperatorTok{\}]} 
 
\NormalTok{ToPaVe}\OperatorTok{[}\SpecialCharTok{\%}\OperatorTok{,} \FunctionTok{q}\OperatorTok{]}
\end{Highlighting}
\end{Shaded}

\begin{dmath*}\breakingcomma
\frac{1}{\left(q^2-\text{m1}^2\right).\left((\text{p1}+q)^2-\text{m2}^2\right).\left((\text{p2}+q)^2-\text{m3}^2\right).\left((\text{p3}+q)^2-\text{m4}^2\right).\left((\text{p4}+q)^2-\text{m5}^2\right)}
\end{dmath*}

\begin{dmath*}\breakingcomma
i \pi ^2 \;\text{E}_0\left(\text{p1}^2,\text{p2}^2,-2 (\text{p2}\cdot \;\text{p3})+\text{p2}^2+\text{p3}^2,-2 (\text{p3}\cdot \;\text{p4})+\text{p3}^2+\text{p4}^2,-2 (\text{p1}\cdot \;\text{p4})+\text{p1}^2+\text{p4}^2,-2 (\text{p1}\cdot \;\text{p2})+\text{p1}^2+\text{p2}^2,\text{p3}^2,-2 (\text{p2}\cdot \;\text{p4})+\text{p2}^2+\text{p4}^2,-2 (\text{p1}\cdot \;\text{p3})+\text{p1}^2+\text{p3}^2,\text{p4}^2,\text{m2}^2,\text{m1}^2,\text{m3}^2,\text{m4}^2,\text{m5}^2\right)
\end{dmath*}

By default, \texttt{ToPaVe} has the option \texttt{PaVeToABCD} set to
\texttt{True}. This means that some of the \texttt{PaVe} functions are
automatically converted to direct Passarino-Veltman functions
(\texttt{A0}, \texttt{A00}, \texttt{B0}, \texttt{B1}, \texttt{B00},
\texttt{B11}, \texttt{C0}, \texttt{D0}). This also has consequences for
\texttt{TID}

\begin{Shaded}
\begin{Highlighting}[]
\NormalTok{TID}\OperatorTok{[}\NormalTok{FVD}\OperatorTok{[}\FunctionTok{q}\OperatorTok{,}\NormalTok{ mu}\OperatorTok{]}\NormalTok{ FAD}\OperatorTok{[\{}\FunctionTok{q}\OperatorTok{,}\NormalTok{ m1}\OperatorTok{\},} \OperatorTok{\{}\FunctionTok{q} \SpecialCharTok{+} \FunctionTok{p}\OperatorTok{\}],} \FunctionTok{q}\OperatorTok{,}\NormalTok{ ToPaVe }\OtherTok{{-}\textgreater{}} \ConstantTok{True}\OperatorTok{]}
\end{Highlighting}
\end{Shaded}

\begin{dmath*}\breakingcomma
\frac{i \pi ^2 p^{\text{mu}} \;\text{A}_0\left(\text{m1}^2\right)}{2 p^2}-\frac{i \pi ^2 \left(\text{m1}^2+p^2\right) p^{\text{mu}} \;\text{B}_0\left(p^2,0,\text{m1}^2\right)}{2 p^2}
\end{dmath*}

\begin{Shaded}
\begin{Highlighting}[]
\SpecialCharTok{\%} \SpecialCharTok{//} \FunctionTok{StandardForm}
\end{Highlighting}
\end{Shaded}

\begin{dmath*}\breakingcomma
\frac{i \pi ^2 \;\text{A0}\left[\text{m1}^2\right] \;\text{Pair}[\text{LorentzIndex}[\text{mu},D],\text{Momentum}[p,D]]}{2 \;\text{Pair}[\text{Momentum}[p,D],\text{Momentum}[p,D]]}-\left.\left(i \pi ^2 \;\text{B0}\left[\text{Pair}[\text{Momentum}[p,D],\text{Momentum}[p,D]],0,\text{m1}^2\right] \;\text{Pair}[\text{LorentzIndex}[\text{mu},D],\text{Momentum}[p,D]] \left(\text{m1}^2+\text{Pair}[\text{Momentum}[p,D],\text{Momentum}[p,D]]\right)\right)\right/(2 \;\text{Pair}[\text{Momentum}[p,D],\text{Momentum}[p,D]])
\end{dmath*}

If you want to avoid direct functions in the output of \texttt{TID} and
other functions that employ \texttt{ToPaVe}, you need to set the option
\texttt{PaVeToABCD} to \texttt{False} globally.

\begin{Shaded}
\begin{Highlighting}[]
\FunctionTok{SetOptions}\OperatorTok{[}\NormalTok{ToPaVe}\OperatorTok{,}\NormalTok{ PaVeToABCD }\OtherTok{{-}\textgreater{}} \ConstantTok{False}\OperatorTok{]}\NormalTok{;}
\end{Highlighting}
\end{Shaded}

\begin{Shaded}
\begin{Highlighting}[]
\NormalTok{TID}\OperatorTok{[}\NormalTok{FVD}\OperatorTok{[}\FunctionTok{q}\OperatorTok{,}\NormalTok{ mu}\OperatorTok{]}\NormalTok{ FAD}\OperatorTok{[\{}\FunctionTok{q}\OperatorTok{,}\NormalTok{ m1}\OperatorTok{\},} \OperatorTok{\{}\FunctionTok{q} \SpecialCharTok{+} \FunctionTok{p}\OperatorTok{\}],} \FunctionTok{q}\OperatorTok{,}\NormalTok{ ToPaVe }\OtherTok{{-}\textgreater{}} \ConstantTok{True}\OperatorTok{]}
\end{Highlighting}
\end{Shaded}

\begin{dmath*}\breakingcomma
\frac{i \pi ^2 p^{\text{mu}} \;\text{A}_0\left(\text{m1}^2\right)}{2 p^2}-\frac{i \pi ^2 \left(\text{m1}^2+p^2\right) p^{\text{mu}} \;\text{B}_0\left(p^2,0,\text{m1}^2\right)}{2 p^2}
\end{dmath*}

\begin{Shaded}
\begin{Highlighting}[]
\SpecialCharTok{\%} \SpecialCharTok{//} \FunctionTok{StandardForm}
\end{Highlighting}
\end{Shaded}

\begin{dmath*}\breakingcomma
\frac{i \pi ^2 \;\text{Pair}[\text{LorentzIndex}[\text{mu},D],\text{Momentum}[p,D]] \;\text{PaVe}\left[0,\{\},\left\{\text{m1}^2\right\}\right]}{2 \;\text{Pair}[\text{Momentum}[p,D],\text{Momentum}[p,D]]}-\left.\left(i \pi ^2 \;\text{Pair}[\text{LorentzIndex}[\text{mu},D],\text{Momentum}[p,D]] \left(\text{m1}^2+\text{Pair}[\text{Momentum}[p,D],\text{Momentum}[p,D]]\right) \;\text{PaVe}\left[0,\{\text{Pair}[\text{Momentum}[p,D],\text{Momentum}[p,D]]\},\left\{0,\text{m1}^2\right\}\right]\right)\right/(2 \;\text{Pair}[\text{Momentum}[p,D],\text{Momentum}[p,D]])
\end{dmath*}
\end{document}
