% !TeX program = pdflatex
% !TeX root = ToPaVe2.tex

\documentclass[../FeynCalcManual.tex]{subfiles}
\begin{document}
\hypertarget{topave2}{%
\section{ToPaVe2}\label{topave2}}

\texttt{ToPaVe2[\allowbreak{}expr]} converts all the direct
Passarino-Veltman functions (\texttt{A0}, \texttt{A00}, \texttt{B0},
\texttt{B1}, \texttt{B00}, \texttt{B11}, \texttt{C0}, \texttt{D0}) to
\texttt{PaVe}-functions.

\subsection{See also}

\hyperlink{toc}{Overview}, \hyperlink{topave}{ToPaVe}.

\subsection{Examples}

\begin{Shaded}
\begin{Highlighting}[]
\NormalTok{ToPaVe2}\OperatorTok{[}\NormalTok{A0}\OperatorTok{[}\FunctionTok{m}\SpecialCharTok{\^{}}\DecValTok{2}\OperatorTok{]]}
\end{Highlighting}
\end{Shaded}

\begin{dmath*}\breakingcomma
\text{A}_0\left(m^2\right)
\end{dmath*}

\begin{Shaded}
\begin{Highlighting}[]
\NormalTok{ToPaVe2}\OperatorTok{[}\NormalTok{A0}\OperatorTok{[}\FunctionTok{m}\SpecialCharTok{\^{}}\DecValTok{2}\OperatorTok{]]} \SpecialCharTok{//}\NormalTok{ FCI }\SpecialCharTok{//} \FunctionTok{StandardForm}

\CommentTok{(*PaVe[0, \{\}, \{m\^{}2\}]*)}
\end{Highlighting}
\end{Shaded}

\begin{Shaded}
\begin{Highlighting}[]
\NormalTok{ToPaVe2}\OperatorTok{[}\NormalTok{B11}\OperatorTok{[}\NormalTok{pp}\OperatorTok{,} \FunctionTok{m}\SpecialCharTok{\^{}}\DecValTok{2}\OperatorTok{,} \FunctionTok{M}\SpecialCharTok{\^{}}\DecValTok{2}\OperatorTok{,}\NormalTok{ BReduce }\OtherTok{{-}\textgreater{}} \ConstantTok{False}\OperatorTok{]]}
\end{Highlighting}
\end{Shaded}

\begin{dmath*}\breakingcomma
\text{B}_{11}\left(\text{pp},m^2,M^2\right)
\end{dmath*}

\begin{Shaded}
\begin{Highlighting}[]
\NormalTok{ToPaVe2}\OperatorTok{[}\NormalTok{B11}\OperatorTok{[}\NormalTok{pp}\OperatorTok{,} \FunctionTok{m}\SpecialCharTok{\^{}}\DecValTok{2}\OperatorTok{,} \FunctionTok{M}\SpecialCharTok{\^{}}\DecValTok{2}\OperatorTok{,}\NormalTok{ BReduce }\OtherTok{{-}\textgreater{}} \ConstantTok{False}\OperatorTok{]]} \SpecialCharTok{//}\NormalTok{ FCI }\SpecialCharTok{//} \FunctionTok{StandardForm}

\CommentTok{(*PaVe[1, 1, \{pp\}, \{m\^{}2, M\^{}2\}]*)}
\end{Highlighting}
\end{Shaded}

\end{document}
