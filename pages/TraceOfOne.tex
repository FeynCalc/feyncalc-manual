% !TeX program = pdflatex
% !TeX root = TraceOfOne.tex

\documentclass[../FeynCalcManual.tex]{subfiles}
\begin{document}
\hypertarget{traceofone}{
\section{TraceOfOne}\label{traceofone}\index{TraceOfOne}}

\texttt{TraceOfOne} is an option for \texttt{Tr} and
\texttt{DiracTrace}. Its setting determines the value of the unit trace.

\subsection{See also}

\hyperlink{toc}{Overview}, \hyperlink{diracsimplify}{DiracSimplify},
\hyperlink{diractrace}{DiracTrace}.

\subsection{Examples}

\begin{Shaded}
\begin{Highlighting}[]
\NormalTok{DiracTrace}\OperatorTok{[}\DecValTok{1}\OperatorTok{]} 
 
\NormalTok{DiracSimplify}\OperatorTok{[}\SpecialCharTok{\%}\OperatorTok{]}
\end{Highlighting}
\end{Shaded}

\begin{dmath*}\breakingcomma
\text{tr}(1)
\end{dmath*}

\begin{dmath*}\breakingcomma
4
\end{dmath*}

\begin{Shaded}
\begin{Highlighting}[]
\NormalTok{DiracTrace}\OperatorTok{[}\DecValTok{1}\OperatorTok{,}\NormalTok{ TraceOfOne }\OtherTok{{-}\textgreater{}}\NormalTok{ tr1}\OperatorTok{]} 
 
\NormalTok{DiracSimplify}\OperatorTok{[}\SpecialCharTok{\%}\OperatorTok{]}
\end{Highlighting}
\end{Shaded}

\begin{dmath*}\breakingcomma
\text{tr}(1)
\end{dmath*}

\begin{dmath*}\breakingcomma
\text{tr1}
\end{dmath*}
\end{document}
