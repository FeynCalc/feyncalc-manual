% !TeX program = pdflatex
% !TeX root = Trick.tex

\documentclass[../FeynCalcManual.tex]{subfiles}
\begin{document}
\hypertarget{trick}{
\section{Trick}\label{trick}\index{Trick}}

\texttt{Trick[\allowbreak{}exp]} performs several basic simplifications
without expansion. \texttt{Trick[\allowbreak{}exp]} uses
\texttt{Contract}, \texttt{DotSimplify} and \texttt{SUNDeltaContract}.

\subsection{See also}

\hyperlink{toc}{Overview}, \hyperlink{calc}{Calc},
\hyperlink{contract}{Contract}, \hyperlink{diractrick}{DiracTrick},
\hyperlink{dotsimplify}{DotSimplify},
\hyperlink{diractrick}{DiracTrick}.

\subsection{Examples}

This calculates \(g^{\mu \nu} \gamma _{\mu }\) and \(g_{\nu }^{\nu}\) in
\(D\) dimensions.

\begin{Shaded}
\begin{Highlighting}[]
\NormalTok{Trick}\OperatorTok{[\{}\NormalTok{GA}\OperatorTok{[}\SpecialCharTok{\textbackslash{}}\OperatorTok{[}\NormalTok{Mu}\OperatorTok{]]}\NormalTok{ MT}\OperatorTok{[}\SpecialCharTok{\textbackslash{}}\OperatorTok{[}\NormalTok{Mu}\OperatorTok{],} \SpecialCharTok{\textbackslash{}}\OperatorTok{[}\NormalTok{Nu}\OperatorTok{]],}\NormalTok{ MTD}\OperatorTok{[}\SpecialCharTok{\textbackslash{}}\OperatorTok{[}\NormalTok{Nu}\OperatorTok{],} \SpecialCharTok{\textbackslash{}}\OperatorTok{[}\NormalTok{Nu}\OperatorTok{]]\}]}
\end{Highlighting}
\end{Shaded}

\begin{dmath*}\breakingcomma
\left\{\bar{\gamma }^{\nu },D\right\}
\end{dmath*}

\begin{Shaded}
\begin{Highlighting}[]
\NormalTok{FV}\OperatorTok{[}\FunctionTok{p} \SpecialCharTok{+} \FunctionTok{r}\OperatorTok{,} \SpecialCharTok{\textbackslash{}}\OperatorTok{[}\NormalTok{Mu}\OperatorTok{]]}\NormalTok{ MT}\OperatorTok{[}\SpecialCharTok{\textbackslash{}}\OperatorTok{[}\NormalTok{Mu}\OperatorTok{],} \SpecialCharTok{\textbackslash{}}\OperatorTok{[}\NormalTok{Nu}\OperatorTok{]]}\NormalTok{ FV}\OperatorTok{[}\FunctionTok{q} \SpecialCharTok{{-}} \FunctionTok{p}\OperatorTok{,} \SpecialCharTok{\textbackslash{}}\OperatorTok{[}\NormalTok{Nu}\OperatorTok{]]} 
 
\NormalTok{Trick}\OperatorTok{[}\SpecialCharTok{\%}\OperatorTok{]}
\end{Highlighting}
\end{Shaded}

\begin{dmath*}\breakingcomma
\bar{g}^{\mu \nu } \left(\overline{q}-\overline{p}\right)^{\nu } \left(\overline{p}+\overline{r}\right)^{\mu }
\end{dmath*}

\begin{dmath*}\breakingcomma
\overline{p}\cdot \overline{q}-\overline{p}\cdot \overline{r}-\overline{p}^2+\overline{q}\cdot \overline{r}
\end{dmath*}

\begin{Shaded}
\begin{Highlighting}[]
\NormalTok{Trick}\OperatorTok{[}\FunctionTok{c}\NormalTok{ . }\FunctionTok{b}\NormalTok{ . }\FunctionTok{a}\NormalTok{ . GA}\OperatorTok{[}\FunctionTok{d}\OperatorTok{]}\NormalTok{ . GA}\OperatorTok{[}\FunctionTok{e}\OperatorTok{]]}
\end{Highlighting}
\end{Shaded}

\begin{dmath*}\breakingcomma
a b c \bar{\gamma }^d.\bar{\gamma }^e
\end{dmath*}

\begin{Shaded}
\begin{Highlighting}[]
\NormalTok{Trick}\OperatorTok{[}\FunctionTok{c}\NormalTok{ . }\FunctionTok{b}\NormalTok{ . }\FunctionTok{a}\NormalTok{ . GA}\OperatorTok{[}\FunctionTok{d}\OperatorTok{]}\NormalTok{ . GA}\OperatorTok{[}\FunctionTok{e}\OperatorTok{]]} \SpecialCharTok{//}\NormalTok{ FCE }\SpecialCharTok{//} \FunctionTok{StandardForm}

\CommentTok{(*a b c GA[d] . GA[e]*)}
\end{Highlighting}
\end{Shaded}

\end{document}
