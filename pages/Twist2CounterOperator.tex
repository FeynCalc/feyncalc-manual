% !TeX program = pdflatex
% !TeX root = Twist2CounterOperator.tex

\documentclass[../FeynCalcManual.tex]{subfiles}
\begin{document}
\hypertarget{twist2counteroperator}{%
\section{Twist2CounterOperator}\label{twist2counteroperator}}

\texttt{Twist2CounterOperator[\allowbreak{}p,\ \allowbreak{}mu,\ \allowbreak{}nu,\ \allowbreak{}a,\ \allowbreak{}b,\ \allowbreak{}5]}
is a special routine for particular QCD calculations.

Also available:
\texttt{Twist2CounterOperator[\allowbreak{}p,\ \allowbreak{}7]},
\texttt{Twist2CounterOperator[\allowbreak{}p1,\ \allowbreak{}p2,\ \allowbreak{}\{\allowbreak{}p3,\ \allowbreak{}mu,\ \allowbreak{}a\},\ \allowbreak{}1]}
(\texttt{p1}: incoming quark momentum, \texttt{p3}: incoming gluon
(count1)).

\subsection{See also}

\hyperlink{toc}{Overview},
\hyperlink{twist2gluonoperator}{Twist2GluonOperator}.

\subsection{Examples}

\begin{Shaded}
\begin{Highlighting}[]
\NormalTok{Twist2CounterOperator}\OperatorTok{[}\FunctionTok{p}\OperatorTok{,}\NormalTok{ mu}\OperatorTok{,}\NormalTok{ nu}\OperatorTok{,} \FunctionTok{a}\OperatorTok{,} \FunctionTok{b}\OperatorTok{,} \DecValTok{5}\OperatorTok{]}
\end{Highlighting}
\end{Shaded}

\begin{dmath*}\breakingcomma
-\frac{1}{2 \varepsilon }\left((-1)^m+1\right) C_A g_s^2 S_n \delta ^{ab} \left(\left(\frac{8}{m}-\frac{12}{m+1}+\frac{8}{m+2}-\frac{8}{m-1}\right) g^{\text{mu}\;\text{nu}} (\Delta \cdot p)^m+\left(\frac{8}{m}-\frac{24}{m+1}+\frac{24}{m+2}-\frac{4}{m-1}\right) p^2 \Delta ^{\text{mu}} \Delta ^{\text{nu}} (\Delta \cdot p)^{m-2}+\left(-\frac{6}{m}+\frac{16}{m+1}-\frac{16}{m+2}+\frac{6}{m-1}\right) (\Delta \cdot p)^{m-1} \left(p^{\text{mu}} \Delta ^{\text{nu}}+\Delta ^{\text{mu}} p^{\text{nu}}\right)\right)
\end{dmath*}

\begin{Shaded}
\begin{Highlighting}[]
\NormalTok{Twist2CounterOperator}\OperatorTok{[}\FunctionTok{p}\OperatorTok{,} \DecValTok{7}\OperatorTok{]}
\end{Highlighting}
\end{Shaded}

\begin{dmath*}\breakingcomma
\frac{\left((-1)^m+1\right) \left(\frac{2}{m}-\frac{1}{m+1}-\frac{2}{m-1}\right) C_F g_s^2 S_n \gamma \cdot \Delta  (\Delta \cdot p)^{m-1}}{\varepsilon }
\end{dmath*}

\begin{Shaded}
\begin{Highlighting}[]
\NormalTok{Twist2CounterOperator}\OperatorTok{[}\NormalTok{p1}\OperatorTok{,}\NormalTok{ p2}\OperatorTok{,} \OperatorTok{\{}\NormalTok{p3}\OperatorTok{,}\NormalTok{ mu}\OperatorTok{,} \FunctionTok{a}\OperatorTok{\},} \DecValTok{1}\OperatorTok{]}
\end{Highlighting}
\end{Shaded}

\begin{dmath*}\breakingcomma
-\frac{\left((-1)^m+1\right) \left(\frac{2}{m}-\frac{1}{m+1}-\frac{2}{m-1}\right) T^a g_s^3 \Delta ^{\text{mu}} S_n \left(C_A-2 C_F\right) \gamma \cdot \Delta  \left(\frac{(\Delta \cdot \;\text{p1})^{m-1}}{\Delta \cdot \;\text{p1}+\Delta \cdot \;\text{p2}}-\frac{(-(\Delta \cdot \;\text{p2}))^{m-1}}{\Delta \cdot \;\text{p1}+\Delta \cdot \;\text{p2}}\right)}{2 \varepsilon }
\end{dmath*}
\end{document}
