% !TeX program = pdflatex
% !TeX root = Twist2GluonOperator.tex

\documentclass[../FeynCalcManual.tex]{subfiles}
\begin{document}
\hypertarget{twist2gluonoperator}{%
\section{Twist2GluonOperator}\label{twist2gluonoperator}}

\texttt{Twist2GluonOperator[\allowbreak{}\{\allowbreak{}p,\ \allowbreak{}mu,\ \allowbreak{}a\},\ \allowbreak{}\{\allowbreak{}nu,\ \allowbreak{}b\}]}
or
\texttt{Twist2GluonOperator[\allowbreak{}p,\ \allowbreak{}\{\allowbreak{}mu,\ \allowbreak{}a\},\ \allowbreak{}\{\allowbreak{}nu,\ \allowbreak{}b\}]}
or
\texttt{Twist2GluonOperator[\allowbreak{}p,\ \allowbreak{}mu,\ \allowbreak{}a,\ \allowbreak{}nu,\ \allowbreak{}b]}
yields the 2-gluon operator (\texttt{p} is ingoing momentum
corresponding to Lorentz index \texttt{mu}).

\texttt{Twist2GluonOperator[\allowbreak{}\{\allowbreak{}p,\ \allowbreak{}mu,\ \allowbreak{}a\},\ \allowbreak{}\{\allowbreak{}q,\ \allowbreak{}nu,\ \allowbreak{}b\},\ \allowbreak{}\{\allowbreak{}k,\ \allowbreak{}la,\ \allowbreak{}c\}]}
or
\texttt{Twist2GluonOperator[\allowbreak{} p,\ \allowbreak{}mu,\ \allowbreak{}a ,\ \allowbreak{}q,\ \allowbreak{}nu,\ \allowbreak{}b ,\ \allowbreak{}k,\ \allowbreak{}la,\ \allowbreak{}c]}
gives the 3-gluon operator.

\texttt{Twist2GluonOperator[\allowbreak{}\{\allowbreak{}p,\ \allowbreak{}mu,\ \allowbreak{}a\},\ \allowbreak{}\{\allowbreak{}q,\ \allowbreak{}nu,\ \allowbreak{}b\},\ \allowbreak{}\{\allowbreak{}k,\ \allowbreak{}la,\ \allowbreak{}c\},\ \allowbreak{}\{\allowbreak{}s,\ \allowbreak{}si,\ \allowbreak{}d\}]}
or
\texttt{Twist2GluonOperator[\allowbreak{}p,\ \allowbreak{}mu,\ \allowbreak{}a ,\ \allowbreak{}q,\ \allowbreak{}nu,\ \allowbreak{}b ,\ \allowbreak{}k,\ \allowbreak{}la,\ \allowbreak{}c ,\ \allowbreak{}s,\ \allowbreak{}si,\ \allowbreak{}d]}
yields the 4-Gluon operator.

The dimension is determined by the option \texttt{Dimension}. The
setting of the option \texttt{Polarization} (unpolarized: \texttt{0};
polarized: \texttt{1}) determines whether the unpolarized or polarized
Feynman rule is returned.

With the setting \texttt{Explicit} set to \texttt{False} the
color-structure and the (\texttt{1+(-1)^OPEm}) (for polarized:
\texttt{(1-(-1)^OPEm)}) is extracted and the color indices are omitted
in the arguments of \texttt{Twist2GluonOperator}.

\subsection{See also}

\hyperlink{toc}{Overview},
\hyperlink{twist2quarkoperator}{Twist2QuarkOperator}.

\subsection{Examples}

The setting All for Explicit performs the sums.

\begin{Shaded}
\begin{Highlighting}[]
\NormalTok{Twist2GluonOperator}\OperatorTok{[\{}\FunctionTok{p}\OperatorTok{,} \SpecialCharTok{\textbackslash{}}\OperatorTok{[}\NormalTok{Mu}\OperatorTok{],} \FunctionTok{a}\OperatorTok{\},} \OperatorTok{\{}\FunctionTok{q}\OperatorTok{,} \SpecialCharTok{\textbackslash{}}\OperatorTok{[}\NormalTok{Nu}\OperatorTok{],} \FunctionTok{b}\OperatorTok{\},} \OperatorTok{\{}\FunctionTok{r}\OperatorTok{,} \SpecialCharTok{\textbackslash{}}\OperatorTok{[}\NormalTok{Rho}\OperatorTok{],} \FunctionTok{c}\OperatorTok{\},}\NormalTok{ Polarization }\OtherTok{{-}\textgreater{}} \DecValTok{1}\OperatorTok{,}\NormalTok{ Explicit }\OtherTok{{-}\textgreater{}} \ConstantTok{All}\OperatorTok{]}
\end{Highlighting}
\end{Shaded}

\begin{dmath*}\breakingcomma
\left(1-(-1)^m\right) g_s f^{abc} \left(O_{\nu \, \rho \, \mu }^{\text{G3}}(q,r,p)\right)
\end{dmath*}
\end{document}
