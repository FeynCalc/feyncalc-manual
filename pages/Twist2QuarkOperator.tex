% !TeX program = pdflatex
% !TeX root = Twist2QuarkOperator.tex

\documentclass[../FeynCalcManual.tex]{subfiles}
\begin{document}
\hypertarget{twist2quarkoperator}{%
\section{Twist2QuarkOperator}\label{twist2quarkoperator}}

Twist2QuarkOperator{[}p{]} or Twist2QuarkOperator{[}p,\emph{,}{]} yields
the quark-antiquark operator (p is momentum in the direction of the
incoming quark). Twist2QuarkOperator{[}\{p,q\}{]} yields the 2-quark
operator for non-zero momentum insertion (p is momentum in the direction
of the incoming quark). Twist2QuarkOperator{[}\{p1,\textbf{\emph{\},
\{p2,}}\}, \{p3, mu, a\}{]} or Twist2QuarkOperator{[}p1,\emph{,},
p2,\emph{,}, p3,mu,a{]} is the quark-antiquark-gluon operator, where p1
is the incoming quark, p2 the incoming antiquark and p3 denotes the
incoming gluon momentum. Twist2QuarkOperator{[}\{p1\}, \{p2\}, \{p3, mu,
a\}, \{p4, nu, b\}{]} or Twist2QuarkOperator{[}\{p1,\textbf{\emph{\},
\{p2,}}\}, \{p3, mu, a\}, \{p4, nu, b\}{]} or
Twist2QuarkOperator{[}p1,\emph{,}, p2,\emph{,}, p3,mu,a, p4, nu, b{]}
gives the Quark-Quark-Gluon-Gluon-operator. The setting of the option
Polarization (unpolarized: 0; polarized: 1) determines whether the
unpolarized or polarized operator is returned.

\subsection{See also}

\hyperlink{toc}{Overview},
\hyperlink{twist2gluonoperator}{Twist2GluonOperator}.

\subsection{Examples}
\end{document}
