% !TeX program = pdflatex
% !TeX root = Twist3QuarkOperator.tex

\documentclass[../FeynCalcManual.tex]{subfiles}
\begin{document}
\hypertarget{twist3quarkoperator}{%
\section{Twist3QuarkOperator}\label{twist3quarkoperator}}

\texttt{Twist3QuarkOperator[\allowbreak{}p]} or
\texttt{Twist3QuarkOperator[\allowbreak{}p,\ \allowbreak{}_,\ \allowbreak{}_]}
yields the 2-quark operator (\texttt{p} is momentum in the direction of
the fermion number flow).

\texttt{Twist3QuarkOperator[\allowbreak{}\{\allowbreak{}p1,\ \allowbreak{}___\},\ \allowbreak{}\{\allowbreak{}p2,\ \allowbreak{}___\},\ \allowbreak{}\{\allowbreak{}p3,\ \allowbreak{}mu,\ \allowbreak{}a\}]}
or
\texttt{Twist3QuarkOperator[\allowbreak{}p1,\ \allowbreak{}_,\ \allowbreak{}_,\ \allowbreak{} p2,\ \allowbreak{}_,\ \allowbreak{}_,\ \allowbreak{} p3,\ \allowbreak{}mu,\ \allowbreak{}a]}
yields the Quark-Quark-Gluon-operator, where \texttt{p1} is the incoming
quark, \texttt{p2} the incoming antiquark and \texttt{p3} denotes the
(incoming) gluon momentum.

\texttt{Twist3QuarkOperator[\allowbreak{}\{\allowbreak{}p1,\ \allowbreak{}___\},\ \allowbreak{}\{\allowbreak{}p2,\ \allowbreak{}___\},\ \allowbreak{}\{\allowbreak{}p3,\ \allowbreak{}mu,\ \allowbreak{}a\},\ \allowbreak{}\{\allowbreak{}p4,\ \allowbreak{}nu,\ \allowbreak{}b\}]}
or
\texttt{Twist3QuarkOperator[\allowbreak{}p1,\ \allowbreak{}_,\ \allowbreak{}_,\ \allowbreak{} p2,\ \allowbreak{}_,\ \allowbreak{}_,\ \allowbreak{} p3,\ \allowbreak{}mu,\ \allowbreak{}a,\ \allowbreak{}p4,\ \allowbreak{}nu,\ \allowbreak{}b]}
gives the Quark-Quark-Gluon-Gluon-operator. The setting of the option
\texttt{Polarization} (unpolarized: \texttt{0}; polarized: \texttt{1})
determines whether the unpolarized or polarized operator is returned.

\subsection{See also}

\hyperlink{toc}{Overview},
\hyperlink{twist2quarkoperator}{Twist2QuarkOperator},
\hyperlink{twist2gluonoperator}{Twist2GluonOperator}.

\subsection{Examples}

\begin{Shaded}
\begin{Highlighting}[]
\NormalTok{Twist3QuarkOperator}\OperatorTok{[}\FunctionTok{p}\OperatorTok{]}
\end{Highlighting}
\end{Shaded}

\begin{dmath*}\breakingcomma
(-1)^m (\gamma \cdot \Delta ).\bar{\gamma }^5 (\Delta \cdot p)^{m-1}
\end{dmath*}
\end{document}
