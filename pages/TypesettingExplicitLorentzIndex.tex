% !TeX program = pdflatex
% !TeX root = TypesettingExplicitLorentzIndex.tex

\documentclass[../FeynCalcManual.tex]{subfiles}
\begin{document}
\hypertarget{typesettingexplicitlorentzindex} \SpecialCharTok{//} \FunctionTok{InputForm}
\end{Highlighting}
\end{Shaded}

\begin{dmath*}\breakingcomma
\text{FeynCalc$\grave{ }$SharedObjects$\grave{ }$Private$\grave{ }$x}\to \;\text{FeynCalc$\grave{ }$SharedObjects$\grave{ }$Private$\grave{ }$x}
\end{dmath*}

\begin{Shaded}
\begin{Highlighting}[]
\FunctionTok{Function}\OperatorTok{[}\NormalTok{FeynCalc\textasciigrave{}SharedObjects\textasciigrave{}Private\textasciigrave{}}\FunctionTok{x}\OperatorTok{,} 
\NormalTok{ FeynCalc\textasciigrave{}SharedObjects\textasciigrave{}Private\textasciigrave{}}\FunctionTok{x}\OperatorTok{]}
\end{Highlighting}
\end{Shaded}

Make explicit Lorentz indices look red

\begin{Shaded}
\begin{Highlighting}[]
\NormalTok{TypesettingExplicitLorentzIndex }\ExtensionTok{=} \FunctionTok{Function}\OperatorTok{[}\FunctionTok{x}\OperatorTok{,} \FunctionTok{Style}\OperatorTok{[}\FunctionTok{x}\OperatorTok{,} \FunctionTok{Red}\OperatorTok{]]}\NormalTok{; }
 
\DecValTok{4} \FunctionTok{M}\SpecialCharTok{\^{}}\DecValTok{2} \FunctionTok{u}\NormalTok{ FV}\OperatorTok{[}\FunctionTok{k}\OperatorTok{,} \DecValTok{0}\OperatorTok{]}\SpecialCharTok{\^{}}\DecValTok{2} \SpecialCharTok{{-}} \DecValTok{4} \FunctionTok{M}\SpecialCharTok{\^{}}\DecValTok{2} \FunctionTok{u}\NormalTok{ FV}\OperatorTok{[}\FunctionTok{k}\OperatorTok{,} \DecValTok{3}\OperatorTok{]}\SpecialCharTok{\^{}}\DecValTok{2} \SpecialCharTok{{-}} \DecValTok{4} \FunctionTok{M}\NormalTok{ SP}\OperatorTok{[}\FunctionTok{k}\OperatorTok{,} \FunctionTok{k}\OperatorTok{]} \SpecialCharTok{{-}} \DecValTok{2} \FunctionTok{M} \FunctionTok{u}\NormalTok{ FV}\OperatorTok{[}\FunctionTok{k}\OperatorTok{,} \DecValTok{0}\OperatorTok{]}\NormalTok{ FV}\OperatorTok{[}\FunctionTok{k}\OperatorTok{,} \DecValTok{3}\OperatorTok{]}\SpecialCharTok{\^{}}\DecValTok{2} \SpecialCharTok{+} \DecValTok{4} \FunctionTok{M} \FunctionTok{u}\NormalTok{ FV}\OperatorTok{[}\FunctionTok{k}\OperatorTok{,} \DecValTok{0}\OperatorTok{]}\NormalTok{ FV}\OperatorTok{[}\FunctionTok{k}\OperatorTok{,} \DecValTok{2}\OperatorTok{]} \SpecialCharTok{{-}} \FunctionTok{u}\SpecialCharTok{\^{}}\DecValTok{2}\NormalTok{ FV}\OperatorTok{[}\FunctionTok{k}\OperatorTok{,} \DecValTok{2}\OperatorTok{]}\SpecialCharTok{\^{}}\DecValTok{2}
\end{Highlighting}
\end{Shaded}

\begin{dmath*}\breakingcomma
-4 M^2 u \left(\overline{k}^3\right)^2-2 k^0 M u \left(\overline{k}^3\right)^2+4 k^0 M u \overline{k}^2-4 M \overline{k}^2-u^2 \left(\overline{k}^2\right)^2+4 \left(k^0\right)^2 M^2 u
\end{dmath*}

Back to the standard settings

\begin{Shaded}
\begin{Highlighting}[]
\NormalTok{TypesettingExplicitLorentzIndex }\ExtensionTok{=} \FunctionTok{Function}\OperatorTok{[}\FunctionTok{x}\OperatorTok{,} \FunctionTok{x}\OperatorTok{]}
\end{Highlighting}
\end{Shaded}

\begin{dmath*}\breakingcomma
x\to x
\end{dmath*}
\end{document}
