% !TeX program = pdflatex
% !TeX root = UnDeclareAllAntiCommutators.tex

\documentclass[../FeynCalcManual.tex]{subfiles}
\begin{document}
\hypertarget{undeclareallanticommutators}{%
\section{UnDeclareAllAntiCommutators}\label{undeclareallanticommutators}}

\texttt{UnDeclareAllAntiCommutators[\allowbreak{}]} undeclares all
user-defined anticommutators.

\subsection{See also}

\hyperlink{toc}{Overview}, \hyperlink{anticommutator}{AntiCommutator},
\hyperlink{commutatorexplicit}{CommutatorExplicit},
\hyperlink{declarenoncommutative}{DeclareNonCommutative},
\hyperlink{dotsimplify}{DotSimplify}.

\subsection{Examples}

\begin{Shaded}
\begin{Highlighting}[]
\NormalTok{DeclareNonCommutative}\OperatorTok{[}\FunctionTok{a}\OperatorTok{,} \FunctionTok{b}\OperatorTok{,} \FunctionTok{c}\OperatorTok{,} \FunctionTok{d}\OperatorTok{]} 
 
\NormalTok{AntiCommutator}\OperatorTok{[}\FunctionTok{a}\OperatorTok{,} \FunctionTok{b}\OperatorTok{]} \ExtensionTok{=}\NormalTok{ x1; }
 
\NormalTok{AntiCommutator}\OperatorTok{[}\FunctionTok{c}\OperatorTok{,} \FunctionTok{d}\OperatorTok{]} \ExtensionTok{=}\NormalTok{ x2; }
 
\NormalTok{DotSimplify}\OperatorTok{[}\FunctionTok{a}\NormalTok{ . }\FunctionTok{b}\NormalTok{ . }\FunctionTok{c}\NormalTok{ . }\FunctionTok{d}\OperatorTok{]}
\end{Highlighting}
\end{Shaded}

\begin{dmath*}\breakingcomma
b.a.d.c-\text{x2} b.a-\text{x1} d.c+\text{x1} \;\text{x2}
\end{dmath*}

\begin{Shaded}
\begin{Highlighting}[]
\NormalTok{UnDeclareAllAntiCommutators}\OperatorTok{[]} 
 
\NormalTok{DotSimplify}\OperatorTok{[}\FunctionTok{a}\NormalTok{ . }\FunctionTok{b}\NormalTok{ . }\FunctionTok{c}\NormalTok{ . }\FunctionTok{d}\OperatorTok{]}
\end{Highlighting}
\end{Shaded}

\begin{dmath*}\breakingcomma
a.b.c.d
\end{dmath*}
\end{document}
