% !TeX program = pdflatex
% !TeX root = UnDeclareCommutator.tex

\documentclass[../FeynCalcManual.tex]{subfiles}
\begin{document}
\hypertarget{undeclarecommutator}{%
\section{UnDeclareCommutator}\label{undeclarecommutator}}

\texttt{UnDeclareCommutator[\allowbreak{}a,\ \allowbreak{}b]} undeclares
the value assigned to the commutator of \texttt{a} and \texttt{b}.

\subsection{See also}

\hyperlink{toc}{Overview}, \hyperlink{commutator}{Commutator},
\hyperlink{commutatorexplicit}{CommutatorExplicit},
\hyperlink{declarenoncommutative}{DeclareNonCommutative},
\hyperlink{dotsimplify}{DotSimplify}.

\subsection{Examples}

\begin{Shaded}
\begin{Highlighting}[]
\NormalTok{Commutator}\OperatorTok{[}\NormalTok{QuantumField}\OperatorTok{[}\NormalTok{FCPartialD}\OperatorTok{[}\NormalTok{LorentzIndex}\OperatorTok{[}\AttributeTok{xxx\_}\OperatorTok{]],} \FunctionTok{A}\OperatorTok{],}\NormalTok{ QuantumField}\OperatorTok{[}\FunctionTok{A}\OperatorTok{]]} \ExtensionTok{=} \DecValTok{0}\NormalTok{;}
\end{Highlighting}
\end{Shaded}

\begin{Shaded}
\begin{Highlighting}[]
\NormalTok{QuantumField}\OperatorTok{[}\FunctionTok{A}\OperatorTok{]}\NormalTok{ . QuantumField}\OperatorTok{[}\FunctionTok{A}\OperatorTok{]}\NormalTok{ . LeftPartialD}\OperatorTok{[}\SpecialCharTok{\textbackslash{}}\OperatorTok{[}\NormalTok{Nu}\OperatorTok{]]}\NormalTok{ . QuantumField}\OperatorTok{[}\FunctionTok{A}\OperatorTok{]}\NormalTok{ . QuantumField}\OperatorTok{[}\FunctionTok{A}\OperatorTok{]}\NormalTok{ . LeftPartialD}\OperatorTok{[}\SpecialCharTok{\textbackslash{}}\OperatorTok{[}\NormalTok{Nu}\OperatorTok{]]} 
 
\NormalTok{ExpandPartialD}\OperatorTok{[}\SpecialCharTok{\%}\OperatorTok{]}
\end{Highlighting}
\end{Shaded}

\begin{dmath*}\breakingcomma
A.A.\overleftarrow{\partial }_{\nu }.A.A.\overleftarrow{\partial }_{\nu }
\end{dmath*}

\begin{dmath*}\breakingcomma
6 A.A.\left(\left.(\partial _{\nu }A\right)\right).\left(\left.(\partial _{\nu }A\right)\right)+A.\left(\partial _{\nu }\partial _{\nu }A\right).A.A+\left(\partial _{\nu }\partial _{\nu }A\right).A.A.A
\end{dmath*}

\begin{Shaded}
\begin{Highlighting}[]
\NormalTok{UnDeclareCommutator}\OperatorTok{[}\NormalTok{QuantumField}\OperatorTok{[}\NormalTok{FCPartialD}\OperatorTok{[}\NormalTok{LorentzIndex}\OperatorTok{[}\AttributeTok{xxx\_}\OperatorTok{]],} \FunctionTok{A}\OperatorTok{],}\NormalTok{ QuantumField}\OperatorTok{[}\FunctionTok{A}\OperatorTok{]]}\NormalTok{;}
\end{Highlighting}
\end{Shaded}

\begin{Shaded}
\begin{Highlighting}[]
\NormalTok{QuantumField}\OperatorTok{[}\FunctionTok{A}\OperatorTok{]}\NormalTok{ . QuantumField}\OperatorTok{[}\FunctionTok{A}\OperatorTok{]}\NormalTok{ . LeftPartialD}\OperatorTok{[}\SpecialCharTok{\textbackslash{}}\OperatorTok{[}\NormalTok{Nu}\OperatorTok{]]}\NormalTok{ . QuantumField}\OperatorTok{[}\FunctionTok{A}\OperatorTok{]}\NormalTok{ . QuantumField}\OperatorTok{[}\FunctionTok{A}\OperatorTok{]}\NormalTok{ . LeftPartialD}\OperatorTok{[}\SpecialCharTok{\textbackslash{}}\OperatorTok{[}\NormalTok{Nu}\OperatorTok{]]} 
 
\NormalTok{ExpandPartialD}\OperatorTok{[}\SpecialCharTok{\%}\OperatorTok{]}
\end{Highlighting}
\end{Shaded}

\begin{dmath*}\breakingcomma
A.A.\overleftarrow{\partial }_{\nu }.A.A.\overleftarrow{\partial }_{\nu }
\end{dmath*}

\begin{dmath*}\breakingcomma
A.\left(\left.(\partial _{\nu }A\right)\right).A.\left(\left.(\partial _{\nu }A\right)\right)+A.\left(\left.(\partial _{\nu }A\right)\right).\left(\left.(\partial _{\nu }A\right)\right).A+\left(\left.(\partial _{\nu }A\right)\right).A.A.\left(\left.(\partial _{\nu }A\right)\right)+\left(\left.(\partial _{\nu }A\right)\right).A.\left(\left.(\partial _{\nu }A\right)\right).A+2 \left(\left.(\partial _{\nu }A\right)\right).\left(\left.(\partial _{\nu }A\right)\right).A.A+A.\left(\partial _{\nu }\partial _{\nu }A\right).A.A+\left(\partial _{\nu }\partial _{\nu }A\right).A.A.A
\end{dmath*}
\end{document}
