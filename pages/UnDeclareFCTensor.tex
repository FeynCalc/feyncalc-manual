% !TeX program = pdflatex
% !TeX root = UnDeclareFCTensor.tex

\documentclass[../FeynCalcManual.tex]{subfiles}
\begin{document}
\hypertarget{undeclarefctensor}{%
\section{UnDeclareFCTensor}\label{undeclarefctensor}}

\texttt{UnDeclareFCTensor[\allowbreak{}a,\ \allowbreak{}b,\ \allowbreak{}...]}
undeclares \texttt{a,\ \allowbreak{}b,\ \allowbreak{}...} to be tensor
heads, i.e.,
\texttt{DataType[\allowbreak{}a,\ \allowbreak{}b,\ \allowbreak{}...,\ \allowbreak{} FCTensor]}
is set to \texttt{False}.

\subsection{See also}

\hyperlink{toc}{Overview}, \hyperlink{declarefctensor}{DeclareFCTensor},
\hyperlink{fctensor}{FCTensor}.

\subsection{Examples}

\begin{Shaded}
\begin{Highlighting}[]
\FunctionTok{ClearAll}\OperatorTok{[}\NormalTok{myTens}\OperatorTok{]} 
 
\NormalTok{DeclareFCTensor}\OperatorTok{[}\NormalTok{myTens}\OperatorTok{]} 
 
\NormalTok{ExpandScalarProduct}\OperatorTok{[}\NormalTok{myTens}\OperatorTok{[}\FunctionTok{z}\OperatorTok{,}\NormalTok{ Momentum}\OperatorTok{[}\FunctionTok{a} \SpecialCharTok{+} \FunctionTok{b}\OperatorTok{],}\NormalTok{ Momentum}\OperatorTok{[}\FunctionTok{c} \SpecialCharTok{+} \FunctionTok{d}\OperatorTok{]]]}
\end{Highlighting}
\end{Shaded}

\begin{dmath*}\breakingcomma
\text{myTens}\left(z,\overline{a},\overline{c}\right)+\text{myTens}\left(z,\overline{a},\overline{d}\right)+\text{myTens}\left(z,\overline{b},\overline{c}\right)+\text{myTens}\left(z,\overline{b},\overline{d}\right)
\end{dmath*}

\begin{Shaded}
\begin{Highlighting}[]
\NormalTok{UnDeclareFCTensor}\OperatorTok{[}\NormalTok{myTens}\OperatorTok{]}
\end{Highlighting}
\end{Shaded}

\end{document}
