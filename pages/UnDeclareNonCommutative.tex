% !TeX program = pdflatex
% !TeX root = UnDeclareNonCommutative.tex

\documentclass[../FeynCalcManual.tex]{subfiles}
\begin{document}
\hypertarget{undeclarenoncommutative}{
\section{UnDeclareNonCommutative}\label{undeclarenoncommutative}\index{UnDeclareNonCommutative}}

\texttt{UnDeclareNonCommutative[\allowbreak{}a,\ \allowbreak{}b,\ \allowbreak{}...]}
undeclares \texttt{a,\ \allowbreak{}b,\ \allowbreak{}...} to be
noncommutative, i.e.,
\texttt{DataType[\allowbreak{}a,\ \allowbreak{}b,\ \allowbreak{}...,\ \allowbreak{}NonCommutative]}
is set to \texttt{False}.

\subsection{See also}

\hyperlink{toc}{Overview}, \hyperlink{datatype}{DataType},
\hyperlink{declarenoncommutative}{DeclareNonCommutative}.

\subsection{Examples}

\begin{Shaded}
\begin{Highlighting}[]
\NormalTok{DeclareNonCommutative}\OperatorTok{[}\FunctionTok{x}\OperatorTok{]}
\end{Highlighting}
\end{Shaded}

As a side-effect of DeclareNonCommutative x is declared to be of
DataType NonCommutative.

\begin{Shaded}
\begin{Highlighting}[]
\NormalTok{DataType}\OperatorTok{[}\FunctionTok{x}\OperatorTok{,}\NormalTok{ NonCommutative}\OperatorTok{]}
\end{Highlighting}
\end{Shaded}

\begin{dmath*}\breakingcomma
\text{True}
\end{dmath*}

The inverse operation is UnDeclareNonCommutative.

\begin{Shaded}
\begin{Highlighting}[]
\NormalTok{UnDeclareNonCommutative}\OperatorTok{[}\FunctionTok{x}\OperatorTok{]} 
 
\NormalTok{DataType}\OperatorTok{[}\FunctionTok{x}\OperatorTok{,}\NormalTok{ NonCommutative}\OperatorTok{]}
\end{Highlighting}
\end{Shaded}

\begin{dmath*}\breakingcomma
\text{False}
\end{dmath*}

\begin{Shaded}
\begin{Highlighting}[]
\NormalTok{DeclareNonCommutative}\OperatorTok{[}\FunctionTok{y}\OperatorTok{,} \FunctionTok{z}\OperatorTok{]} 
 
\NormalTok{DataType}\OperatorTok{[}\FunctionTok{y}\OperatorTok{,} \FunctionTok{z}\OperatorTok{,}\NormalTok{ NonCommutative}\OperatorTok{]}
\end{Highlighting}
\end{Shaded}

\begin{dmath*}\breakingcomma
\{\text{True},\text{True}\}
\end{dmath*}

\begin{Shaded}
\begin{Highlighting}[]
\NormalTok{UnDeclareNonCommutative}\OperatorTok{[}\FunctionTok{y}\OperatorTok{,} \FunctionTok{z}\OperatorTok{]} 
 
\NormalTok{DataType}\OperatorTok{[}\FunctionTok{y}\OperatorTok{,} \FunctionTok{z}\OperatorTok{,}\NormalTok{ NonCommutative}\OperatorTok{]}
\end{Highlighting}
\end{Shaded}

\begin{dmath*}\breakingcomma
\{\text{False},\text{False}\}
\end{dmath*}
\end{document}
