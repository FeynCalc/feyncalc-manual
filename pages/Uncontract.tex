% !TeX program = pdflatex
% !TeX root = Uncontract.tex

\documentclass[../FeynCalcManual.tex]{subfiles}
\begin{document}
\hypertarget{uncontract}{
\section{Uncontract}\label{uncontract}\index{Uncontract}}

\texttt{Uncontract[\allowbreak{}exp,\ \allowbreak{}q1,\ \allowbreak{}q2,\ \allowbreak{}...]}
uncontracts \texttt{Eps} and \texttt{DiracGamma}.

\texttt{Uncontract[\allowbreak{}exp,\ \allowbreak{}q1,\ \allowbreak{}q2,\ \allowbreak{}Pair -> \{\allowbreak{}p\}]}
uncontracts also \(p \cdot q_1\) and \(p \cdot q_2\);

The option \texttt{Pair -> All} uncontracts all momenta except
\texttt{OPEDelta}.

\subsection{See also}

\hyperlink{toc}{Overview}, \hyperlink{contract}{Contract}.

\subsection{Examples}

\begin{Shaded}
\begin{Highlighting}[]
\NormalTok{LC}\OperatorTok{[}\SpecialCharTok{\textbackslash{}}\OperatorTok{[}\NormalTok{Mu}\OperatorTok{],} \SpecialCharTok{\textbackslash{}}\OperatorTok{[}\NormalTok{Nu}\OperatorTok{]][}\FunctionTok{p}\OperatorTok{,} \FunctionTok{q}\OperatorTok{]} 
 
\NormalTok{Uncontract}\OperatorTok{[}\SpecialCharTok{\%}\OperatorTok{,} \FunctionTok{p}\OperatorTok{]}
\end{Highlighting}
\end{Shaded}

\begin{dmath*}\breakingcomma
\bar{\epsilon }^{\mu \nu \overline{p}\overline{q}}
\end{dmath*}

\begin{dmath*}\breakingcomma
\overline{p}^{\text{\$AL}(\text{\$19})} \bar{\epsilon }^{\mu \nu \;\text{\$AL}(\text{\$19})\overline{q}}
\end{dmath*}

\begin{Shaded}
\begin{Highlighting}[]
\NormalTok{GS}\OperatorTok{[}\FunctionTok{p}\OperatorTok{]} 
 
\NormalTok{Uncontract}\OperatorTok{[}\SpecialCharTok{\%}\OperatorTok{,} \FunctionTok{p}\OperatorTok{]}
\end{Highlighting}
\end{Shaded}

\begin{dmath*}\breakingcomma
\bar{\gamma }\cdot \overline{p}
\end{dmath*}

\begin{dmath*}\breakingcomma
\bar{\gamma }^{\text{\$AL}(\text{\$20})} \overline{p}^{\text{\$AL}(\text{\$20})}
\end{dmath*}

\begin{Shaded}
\begin{Highlighting}[]
\NormalTok{Uncontract}\OperatorTok{[}\NormalTok{LC}\OperatorTok{[}\SpecialCharTok{\textbackslash{}}\OperatorTok{[}\NormalTok{Mu}\OperatorTok{],} \SpecialCharTok{\textbackslash{}}\OperatorTok{[}\NormalTok{Nu}\OperatorTok{]][}\FunctionTok{p}\OperatorTok{,} \FunctionTok{q}\OperatorTok{],} \FunctionTok{p}\OperatorTok{,} \FunctionTok{q}\OperatorTok{]}
\end{Highlighting}
\end{Shaded}

\begin{dmath*}\breakingcomma
\overline{p}^{\text{\$AL}(\text{\$22})} \overline{q}^{\text{\$AL}(\text{\$21})} \left(-\bar{\epsilon }^{\mu \nu \;\text{\$AL}(\text{\$21})\text{\$AL}(\text{\$22})}\right)
\end{dmath*}

By default scalar products are not uncontracted.

\begin{Shaded}
\begin{Highlighting}[]
\NormalTok{Uncontract}\OperatorTok{[}\NormalTok{SP}\OperatorTok{[}\FunctionTok{p}\OperatorTok{,} \FunctionTok{q}\OperatorTok{],} \FunctionTok{q}\OperatorTok{]}
\end{Highlighting}
\end{Shaded}

\begin{dmath*}\breakingcomma
\overline{p}\cdot \overline{q}
\end{dmath*}

Use the option \texttt{Pair->All} to make the function take care of the
scalar products as well

\begin{Shaded}
\begin{Highlighting}[]
\NormalTok{Uncontract}\OperatorTok{[}\NormalTok{SP}\OperatorTok{[}\FunctionTok{p}\OperatorTok{,} \FunctionTok{q}\OperatorTok{],} \FunctionTok{q}\OperatorTok{,}\NormalTok{ Pair }\OtherTok{{-}\textgreater{}} \ConstantTok{All}\OperatorTok{]}
\end{Highlighting}
\end{Shaded}

\begin{dmath*}\breakingcomma
\overline{p}^{\text{\$AL}(\text{\$23})} \overline{q}^{\text{\$AL}(\text{\$23})}
\end{dmath*}

\begin{Shaded}
\begin{Highlighting}[]
\NormalTok{Uncontract}\OperatorTok{[}\NormalTok{SP}\OperatorTok{[}\FunctionTok{p}\OperatorTok{,} \FunctionTok{q}\OperatorTok{]}\SpecialCharTok{\^{}}\DecValTok{2}\OperatorTok{,} \FunctionTok{q}\OperatorTok{,}\NormalTok{ Pair }\OtherTok{{-}\textgreater{}} \ConstantTok{All}\OperatorTok{]}
\end{Highlighting}
\end{Shaded}

\begin{dmath*}\breakingcomma
\overline{p}^{\text{\$AL}(\text{\$24})} \overline{p}^{\text{\$AL}(\text{\$25})} \overline{q}^{\text{\$AL}(\text{\$24})} \overline{q}^{\text{\$AL}(\text{\$25})}
\end{dmath*}

For Cartesian scalar products you need to use the option
\texttt{CartesianPair->All}

\begin{Shaded}
\begin{Highlighting}[]
\NormalTok{Uncontract}\OperatorTok{[}\NormalTok{CSP}\OperatorTok{[}\FunctionTok{p}\OperatorTok{,} \FunctionTok{q}\OperatorTok{],} \FunctionTok{q}\OperatorTok{,}\NormalTok{ Pair }\OtherTok{{-}\textgreater{}} \ConstantTok{All}\OperatorTok{]}
\end{Highlighting}
\end{Shaded}

\begin{dmath*}\breakingcomma
\overline{p}\cdot \overline{q}
\end{dmath*}

\begin{Shaded}
\begin{Highlighting}[]
\NormalTok{Uncontract}\OperatorTok{[}\NormalTok{CSP}\OperatorTok{[}\FunctionTok{p}\OperatorTok{,} \FunctionTok{q}\OperatorTok{],} \FunctionTok{q}\OperatorTok{,}\NormalTok{ CartesianPair }\OtherTok{{-}\textgreater{}} \ConstantTok{All}\OperatorTok{]}
\end{Highlighting}
\end{Shaded}

\begin{dmath*}\breakingcomma
\overline{p}^{\text{\$AL}(\text{\$26})} \overline{q}^{\text{\$AL}(\text{\$26})}
\end{dmath*}
\end{document}
