% !TeX program = pdflatex
% !TeX root = Variables2.tex

\documentclass[../FeynCalcManual.tex]{subfiles}
\begin{document}
\hypertarget{variables2}{
\section{Variables2}\label{variables2}\index{Variables2}}

\texttt{Variables2[\allowbreak{}expr]} is like \texttt{Variables}, but
it also works on rules and equalities as well as lists thereof.

\texttt{Variables2} always applies \texttt{Union} to the output.

\subsection{See also}

\hyperlink{toc}{Overview}, \hyperlink{cases2}{Cases2}.

\subsection{Examples}

Some cases where \texttt{Variables2} is much more useful than
\texttt{Variables}

\begin{Shaded}
\begin{Highlighting}[]
\FunctionTok{Variables}\OperatorTok{[\{}\FunctionTok{a} \OtherTok{{-}\textgreater{}}\NormalTok{ x1 }\SpecialCharTok{+}\NormalTok{ y1}\OperatorTok{,} \FunctionTok{b} \OtherTok{{-}\textgreater{}}\NormalTok{ x2 }\SpecialCharTok{+}\NormalTok{ y2}\OperatorTok{\}]}
\end{Highlighting}
\end{Shaded}

\begin{dmath*}\breakingcomma
\{\}
\end{dmath*}

\begin{Shaded}
\begin{Highlighting}[]
\NormalTok{Variables2}\OperatorTok{[\{}\FunctionTok{a} \OtherTok{{-}\textgreater{}}\NormalTok{ x1 }\SpecialCharTok{+}\NormalTok{ y1}\OperatorTok{,} \FunctionTok{b} \OtherTok{{-}\textgreater{}}\NormalTok{ x2 }\SpecialCharTok{+}\NormalTok{ y2}\OperatorTok{\}]}
\end{Highlighting}
\end{Shaded}

\begin{dmath*}\breakingcomma
\{a,b,\text{x1},\text{x2},\text{y1},\text{y2}\}
\end{dmath*}

\begin{Shaded}
\begin{Highlighting}[]
\FunctionTok{Variables}\OperatorTok{[}\FunctionTok{a} \SpecialCharTok{+} \FunctionTok{b} \ExtensionTok{==} \FunctionTok{c} \SpecialCharTok{+} \FunctionTok{d}\OperatorTok{]}
\end{Highlighting}
\end{Shaded}

\begin{dmath*}\breakingcomma
\{\}
\end{dmath*}

\begin{Shaded}
\begin{Highlighting}[]
\NormalTok{Variables2}\OperatorTok{[}\FunctionTok{a} \SpecialCharTok{+} \FunctionTok{b} \ExtensionTok{==} \FunctionTok{c} \SpecialCharTok{+} \FunctionTok{d}\OperatorTok{]}
\end{Highlighting}
\end{Shaded}

\begin{dmath*}\breakingcomma
\{a,b,c,d\}
\end{dmath*}
\end{document}
