% !TeX program = pdflatex
% !TeX root = Zeta10.tex

\documentclass[../FeynCalcManual.tex]{subfiles}
\begin{document}
\hypertarget{zeta10}{
\section{Zeta10}\label{zeta10}\index{Zeta10}}

\texttt{Zeta10} denotes \texttt{Zeta[\allowbreak{}10]}.

\subsection{See also}

\hyperlink{toc}{Overview}, \hyperlink{zeta8}{Zeta8},
\hyperlink{zeta6}{Zeta6}, \hyperlink{zeta4}{Zeta4},
\hyperlink{zeta2}{Zeta2}.

\subsection{Examples}

\begin{Shaded}
\begin{Highlighting}[]
\NormalTok{Zeta10}
\end{Highlighting}
\end{Shaded}

\begin{dmath*}\breakingcomma
\zeta (10)
\end{dmath*}

\begin{Shaded}
\begin{Highlighting}[]
\FunctionTok{N}\OperatorTok{[}\FunctionTok{Zeta}\OperatorTok{[}\DecValTok{10}\OperatorTok{]]}
\end{Highlighting}
\end{Shaded}

\begin{dmath*}\breakingcomma
1.00099
\end{dmath*}

\begin{Shaded}
\begin{Highlighting}[]
\NormalTok{SimplifyPolyLog}\OperatorTok{[}\FunctionTok{Pi}\SpecialCharTok{\^{}}\DecValTok{10}\OperatorTok{]}
\end{Highlighting}
\end{Shaded}

\begin{dmath*}\breakingcomma
93555 \zeta (10)
\end{dmath*}

\begin{Shaded}
\begin{Highlighting}[]
\FunctionTok{Conjugate}\OperatorTok{[}\NormalTok{Zeta10}\OperatorTok{]}
\end{Highlighting}
\end{Shaded}

\begin{dmath*}\breakingcomma
\zeta (10)
\end{dmath*}
\end{document}
