% !TeX program = pdflatex
% !TeX root = Zeta4.tex

\documentclass[../FeynCalcManual.tex]{subfiles}
\begin{document}
\hypertarget{zeta4}{
\section{Zeta4}\label{zeta4}\index{Zeta4}}

\texttt{Zeta4} denotes \texttt{Zeta[\allowbreak{}4]}.

\subsection{See also}

\hyperlink{toc}{Overview}, \hyperlink{zeta2}{Zeta2},
\hyperlink{zeta6}{Zeta6}, \hyperlink{zeta8}{Zeta8},
\hyperlink{zeta10}{Zeta10}.

\subsection{Examples}

\begin{Shaded}
\begin{Highlighting}[]
\NormalTok{Zeta4}
\end{Highlighting}
\end{Shaded}

\begin{dmath*}\breakingcomma
\zeta (4)
\end{dmath*}

\begin{Shaded}
\begin{Highlighting}[]
\FunctionTok{N}\OperatorTok{[}\NormalTok{Zeta4}\OperatorTok{]}
\end{Highlighting}
\end{Shaded}

\begin{dmath*}\breakingcomma
1.08232
\end{dmath*}

\begin{Shaded}
\begin{Highlighting}[]
\NormalTok{SimplifyPolyLog}\OperatorTok{[}\FunctionTok{Pi}\SpecialCharTok{\^{}}\DecValTok{4}\OperatorTok{]}
\end{Highlighting}
\end{Shaded}

\begin{dmath*}\breakingcomma
90 \zeta (4)
\end{dmath*}

\begin{Shaded}
\begin{Highlighting}[]
\FunctionTok{Conjugate}\OperatorTok{[}\NormalTok{Zeta4}\OperatorTok{]}
\end{Highlighting}
\end{Shaded}

\begin{dmath*}\breakingcomma
\zeta (4)
\end{dmath*}
\end{document}
