% !TeX program = pdflatex
% !TeX root = Zeta6.tex

\documentclass[../FeynCalcManual.tex]{subfiles}
\begin{document}
\hypertarget{zeta6}{
\section{Zeta6}\label{zeta6}\index{Zeta6}}

\texttt{Zeta6} denotes \texttt{Zeta[\allowbreak{}6]}.

\subsection{See also}

\hyperlink{toc}{Overview}, \hyperlink{zeta2}{Zeta2},
\hyperlink{zeta4}{Zeta4}, \hyperlink{zeta8}{Zeta8},
\hyperlink{zeta10}{Zeta10}.

\subsection{Examples}

\begin{Shaded}
\begin{Highlighting}[]
\NormalTok{Zeta6}
\end{Highlighting}
\end{Shaded}

\begin{dmath*}\breakingcomma
\zeta (6)
\end{dmath*}

\begin{Shaded}
\begin{Highlighting}[]
\FunctionTok{N}\OperatorTok{[}\NormalTok{Zeta6}\OperatorTok{]}
\end{Highlighting}
\end{Shaded}

\begin{dmath*}\breakingcomma
1.01734
\end{dmath*}

\begin{Shaded}
\begin{Highlighting}[]
\NormalTok{SimplifyPolyLog}\OperatorTok{[}\FunctionTok{Pi}\SpecialCharTok{\^{}}\DecValTok{6}\OperatorTok{]}
\end{Highlighting}
\end{Shaded}

\begin{dmath*}\breakingcomma
945 \zeta (6)
\end{dmath*}

\begin{Shaded}
\begin{Highlighting}[]
\FunctionTok{Conjugate}\OperatorTok{[}\NormalTok{Zeta6}\OperatorTok{]}
\end{Highlighting}
\end{Shaded}

\begin{dmath*}\breakingcomma
\zeta (6)
\end{dmath*}
\end{document}
